\documentclass[twoside,12Q,b5paper]{escoffierltjsbook}
\usepackage{amsmath}%数式
\usepackage{amssymb}
\usepackage[no-math]{fontspec}
%\usepackage{xunicode}


\usepackage[unicode=true]{hyperref}
\hypersetup{breaklinks=true,
             bookmarks=true,
             pdfauthor={},
             pdftitle={},
             colorlinks=true,
             citecolor=blue,
             urlcolor=blue,
             linkcolor=magenta,
             pdfborder={0 0 0}}
\urlstyle{same}

%%欧文フォント設定
\setmainfont[Ligatures=TeX,Scale=1.0]{Linux Libertine O}

%%Garamond
%\usepackage{ebgaramond-maths}
%\setmainfont[Ligatures=TeX,Scale=1.0]{EB Garamond}%fontspecによるフォント設定


%\setmainfont[Ligatures=TeX]{TeX Gyre Pagella}%ギリシャ語を用いる場合はこちら
%\setsansfont[Scale=MatchLowercase]{TeX Gyre Heros}  % \sffamily のフォント
\setsansfont[Ligatures=TeX, Scale=1]{Linux Biolinum O}     % Libertine/Biolinum
\setmonofont[Scale=MatchLowercase]{Inconsolata}       % \ttfamily のフォント

\usepackage[cmintegrals,cmbraces]{newtxmath}%数式フォント

\usepackage{luatexja}
\usepackage{luatexja-fontspec}
%\ltjdefcharrange{8}{"2000-"2013, "2015-"2025, "2027-"203A, "203C-"206F}
%\ltjsetparameter{jacharrange={-2, +8}}
\usepackage{luatexja-ruby}

%%%%和文仮名プロポーショナル
%\usepackage[hiragino-pron,expert,deluxe]{luatexja-preset}
\usepackage[ipaex,expert,deluxe]{luatexja-preset}
%\newopentypefeature{PKana}{On}{pkna} % "PKana" and "On" can be arbitrary string
%\setmainjfont[
%    JFM=prop,PKana=On,Kerning=On,
%    BoldFont={YuMincho-DemiBold},
%    ItalicFont={YuMincho-Medium},
%    BoldItalicFont={YuMincho-DemiBold}
%]{YuMincho-Medium}
%\setsansjfont[
%    JFM=prop,PKana=On,Kerning=On,
%    BoldFont={YuGothic-Bold},
%    ItalicFont={YuGothic-Medium},
%    BoldItalicFont={YuGothic-Bold}
%]{YuGothic-Medium}
%%%%和文仮名プロプーショナルここまで

\renewcommand{\bfdefault}{bx}%和文ボールドを有効にする
\renewcommand{\headfont}{\gtfamily\sffamily\bfseries}%和文ボールドを有効にする

\defaultfontfeatures[\rmfamily]{Scale=1.2}%効いていない様子
\defaultjfontfeatures{Scale=0.92487}%和文フォントのサイズ調整。デフォルトは 0.962212 倍%ltjsclassesでは不要?
%\defaultjfontfeatures{Scale=0.962212}
%\usepackage{libertineotf}%linux libertine font %ギリシア語含む
%\usepackage[T1]{fontenc}
%\usepackage[full]{textcomp}
%\usepackage[osfI,scaled=1.0]{garamondx}
%\usepackage{tgheros,tgcursor}
%\usepackage[garamondx]{newtxmath}
\usepackage{xfrac}

%レイアウト調整
\usepackage{layout}
%\setlength{\hoffset}{-1truein}
\setlength{\hoffset}{5mm}
\setlength{\oddsidemargin}{0pt}
\setlength{\evensidemargin}{-1cm}
%\setlength{\textwidth}{\fullwidth}%%ltjsclassesのみ有効
\setlength{\fullwidth}{13cm}
\setlength{\textwidth}{13cm}
\setlength{\marginparsep}{0pt}
\setlength{\marginparwidth}{0pt}

\def\tightlist{\itemsep1pt\parskip0pt\parsep0pt}

  
%\usepackage{fancyhdr}


%レシピ本文
\usepackage{multicol}
\usepackage{setspace}

%レシピ連番
\usepackage{remreset}
%\newenvironment{recette}{\begin{small}\begin{spacing}{1}\begin{multicols}{2}}{\end{multicols}\end{spacing}\end{small}}
\newenvironment{recette}{\begin{multicols}{2}}{\end{multicols}}

\renewcommand{\thechapter}{}
\renewcommand{\thesection}{}
\renewcommand{\thesubsubsection}{}
\makeatletter
%subsectionに連番をつける
\@removefromreset{subsection}{section}
\def\thesubsection{\arabic{subsection}.}
\newcounter{rnumber}
\renewcommand{\thernumber}{\refstepcounter{rnumber} }


\renewcommand{\prepartname}{\if@english Part~\else {}\fi}
\renewcommand{\postpartname}{\if@english\else {}\fi}
\renewcommand{\prechaptername}{\if@english Chapter~\else {}\fi}
\renewcommand{\postchaptername}{\if@english\else {}\fi}
\renewcommand{\presectionname}{}%  第
\renewcommand{\postsectionname}{}% 節

\makeatother

% PDF/X-1a
% \usepackage[x-1a]{pdfx}
% \Keywords{pdfTeX\sep PDF/X-1a\sep PDF/A-b}
% \Title{Sample LaTeX input file}
% \Author{LaTeX project team}
% \Org{TeX Users Group}
% \pdfcompresslevel=0
%\usepackage[cmyk]{xcolor}

%biblatex
%\usepackage[notes,strict,backend=biber,autolang=other,%
%                   bibencoding=inputenc,autocite=footnote]{biblatex-chicago}
%\addbibresource{hist-agri.bib}
\let\cite=\autocite

% % % % 
\date{}

%%%脚注番号のページ毎のリセット
%\makeatletter
%  \@addtoreset{footnote}{page}
%\makeatother
\usepackage[perpage,marginal,stable]{footmisc}
\makeatletter
\renewcommand\@makefntext[1]{%
  \advance\leftskip 1.5\zw
  \parindent 1\zw
  \noindent
  \llap{\@thefnmark\hskip0.5\zw}#1}
\makeatother  

\begin{document}
%%%%%\layout

%fancyhdr
%\pagestyle{fancy}
%\lhead[\thepage]{\thesection}
%\chead{}
%\rhead[\thechapter]{\thepage}
%\fancyhead{\gdef\headrulewidth{0pt}}
%\lfoot{}
%\cfoot{}
%\rfoot{}



%\begin{recette}%2段組はじまり

\section{基本ソース}\label{ux57faux672cux30bdux30fcux30b9}

\subsection[ソース・エスパニョル SAUCE
ESPAGNOLE]{\texorpdfstring{ソース・エスパニョル\footnote{「スペイン(風)の」意だが、スペイン料理起源というわけでは
  ない。スペインを想起させるトマトを使うから、あるいは、ソースが茶褐
  色であることからムーア系スペイン人を想起させるから、など諸説ある。\\
  カレーム『19世紀フランス料理』第3巻に収められたソース・エスパニョルの作
  り方は、フォンをとるところから始まり4ページにわたって詳細なものとなっている(pp.8-11)。\\
  その中で、肉を入れた鍋に少量のブイヨンを注いで煮詰めることを繰り返
  す。ここまでは18世紀の料理書で一般的な手法であるが、その後に大量の
  ブイヨンを注いだ後、いきなり強火にかけるのではなく、弱火で加熱して
  いくやり方を「スペイン式の方法」と述べている。カレームにおいては、
  これがソースの名称の根拠のひとつになっていると考えていいだろう。も
  ちろん、ソース・エスパニョルという名称のソースはカレーム以前からあ
  り、1806年刊のヴィアール『帝国料理の本』にもカレームのレシピより簡
  単ではあるがほぼ同様のものが基本ソースとして採り上げられている。\\
  また、それ以前にもソース・エスパニョルに類する名称のソースはあった
  が、たとえば1739年刊ムノン『新料理研究』第2巻にある「スペイン風ソー
  ス」はかなり趣きが異なる(コリアンダーひと把みを加えるのが特徴的)。
  同じ料理名でも時代や料理書の著者によってまったく違う料理になってい
  ることは、食文化史において珍しいことではない。エスコフィエにおける
  ソース・エスパニョルの源流は19世紀初頭のヴィアールあたりからと捉え
  ていいだろう。} SAUCE
ESPAGNOLE}{ソース・エスパニョル SAUCE ESPAGNOLE}}\label{ux30bdux30fcux30b9ux30a8ux30b9ux30d1ux30cbux30e7ux30eb102008-sauce-espagnole}

(仕上がり5L分)

\textbf{とろみ付けのためのルー}\ldots{}\ldots{}625g。

\textbf{茶色いフォン(ソースを仕上げるのに必要な全量)}\ldots{}\ldots{}12L。

\textbf{ミルポワ\footnote{mirepoix
  ミルポワ。ソースやフォンにコクを与えるための、細
  かいさいの目に切った香味野菜や塩漬け豚ばら肉を合わせたもの。18世紀
  にミルポワ公爵の料理人が考案したという説が有力。同様のものにマティ
  ニョンmatignonがあるが、ミルポワより大きめのさいの目に切るのが一般
  的とされるが、調理現場によってはあまり区別せずミルポワとのみ呼称す
  るケースも多いようだ。}(香味素材)}\ldots{}\ldots{}小さなさいの目に切った塩漬け豚ばら肉
150g、2mm程度のさいの目\footnote{brunoise
  ブリュノワーズ。1〜2mmのさいの目に切ること。}に切ったにんじん250gと玉ねぎ150g、タ
イム2枝、ローリエの葉2枚。

\textbf{作業手順}

\begin{enumerate}
\def\labelenumi{\arabic{enumi}.}
\item
  フォン8Lを鍋で沸かす。あらかじめ柔らかくしておいたルーを加え、木杓
  子か泡立て器で混ぜながら沸騰させる。\\
  弱火にして\footnote{原文から直訳すると「鍋を火の脇に置く」だが、現代の調理環境
    では単純に「弱火にする」と解釈していい。}微沸騰の状態を保つ。
\item
  以下のようにしてあらかじめ用意しておいたミルポワを投入する。ソテー
  鍋に塩漬け豚ばら肉を入れて火にかけて脂を溶かす。そこに、細かく刻ん
  だにんじんと玉ねぎ、タイム、ローリエの葉を加える。野菜が軽く色づく
  まで強火で炒める。丁寧に、余分な脂を捨てる。これをソースに加える。
  野菜を炒めたソテー鍋に白ワイン約100mlを加えてデグラセし、それを半量
  まで煮詰める。これも同様にソースの鍋に加える。こまめに浮いてくる夾
  雑物を徹底的に取り除き\footnote{原文は dépouiller
    デプイエ。もともとは動物などの皮を剥ぐ、
    剥くことの意で、野うさぎの皮を剥ぐ、うなぎの皮を剥く、という意味で
    用いる。ソースの場合は表面に凝固した蛋白質や油脂の膜が出来、それを
    「剥ぐように」取り除くことから、あるいは表面に浮いてくる不純物を徹
    底的に取り除いてきれいなソースに仕上げることを、動物の皮を剥いてき
    れいな身だけにすることになぞらえて、この用語が用いられるようになっ
    たようだ。現代の調理現場では écumer エキュメ、すなわち浮いてくる泡、
    アクを取る、という用語だけで済ませていることも多いらしい。なお、本
    書においてécumerが単に浮いてくる泡やアクを取る、という作業であるの
    に対して、dépouillerは「徹底的に不純物を取り除いて美しく仕上げる」
    という意味合いが込められている。}ながら弱火で約1時間煮込む。
\item
  ソースをシノワ\footnote{小さな穴が多く空けられた円錐形で、取っ手の付いた漉し器の一
    種。金属製のものが主流。}で、ミルポワ野菜を軽く押しながら漉し、別の
  片手鍋に移す。フォン2Lを注ぎ足す。さらに二時間、微沸騰の状態を保ち
  ならが煮込む。その後、陶製の鍋に移し、ゆっくり混ぜながら冷ます。
\item
  翌日、再び厚手の片手鍋に移してから、フォン2Lとトマトピュレ1Lまた
  は同等の生のトマトつまり2kgを加える。\\
  トマトピュレを用いる場合は、あらかじめオーブンでほとんど茶色になる
  まで焼いておくといい。そうするとトマトピュレの酸味を抜くことが出来
  る。\\
  そうすればソースを澄ませる作業が楽になるし、ソースの色合いも温かそ
  うで美しいものになる。\\
  ソースをヘラか泡立て器で混ぜながら強火で沸騰させる。弱火にして1
  時間微沸騰の状態を保つ。最後に、表面に浮いている夾雑物を、細心の注
  意を払いながら徹底的に取り除く。布で漉し、完全に冷めるまで、ゆっく
  り混ぜ続けること。
\end{enumerate}

【原注】ソース・エスパニョルで仕上げに夾雑物を取り除くのにかかる時間は
いちがいには言えない。これは、ソースに用いるフォンの質次第で変わるから
だ。

ソースにするフォンが上質なものであればある程、仕上げに夾雑物を取り除く
作業は早く済む。そういう場合には、ソース・エスパニョルを5時間で作るこ
とも無理ではない。

\subsection[魚料理用ソース・エスパニョル]{\texorpdfstring{魚料理用\footnote{フランス語のソース名にあるmaigreはこの場合、一般的には「魚
  用、魚料理用」と訳すが、厳密には「小斉の際の料理用」となろう。小斉
  とは、カトリックで古くから特定の期間、曜日に肉類を断つ食事をする宗
  教的食習慣。日本の「お精進」とニュアンスは近いが、小斉においては忌
  避されるのは鳥獣肉のみであり、魚介や乳製品はいいとされた。こじつけ
  のように、水鳥は水のものだから魚介扱いであり、またイルカも魚類とし
  て扱われていた。小斉が行なわれるのは復活祭の前46日間(四旬節、逆に
  言えばカーニバルの最終日マルディグラの翌日から46日)と、週に一度
  (多くの場合は金曜)であった。合計すると小斉が行なわれるのは年間
  100日近くもあり、中世から18世紀の料理人たちは小斉の宴席に供する料
  理に工夫を凝らしていた。この習慣は19世紀になるとだんだん廃れていき、
  エスコフィエの時代には、料理人に対して小斉のための料理を要求するこ
  とは少なくなっていった。}ソース・エスパニョル}{魚料理用ソース・エスパニョル}}\label{ux9b5aux6599ux7406ux75280102006ux30bdux30fcux30b9ux30a8ux30b9ux30d1ux30cbux30e7ux30eb}

(仕上がり5L分)

\textbf{バターを用いて\footnote{初版〜第三版にかけては、茶色いルーを作るのに「バターまたは、
  きれいなグレスドマルミット(コンソメを作る際に表面に浮いてくる脂を
  すくい取って、不純物を漉し取ったものであり、基本的に獣脂)」を用い
  る、とある。上述のように、カトリックにおける「小斉」の場合、獣脂は
  忌避されたがバターなどの乳製品は許容された。そのため特に「バターを
  用いて作ったルー」という指定がなされ、第四版では茶色いルーに澄まし
  バターのみを使う旨が強調されたが、ここでは初版以来の記述がそのまま
  残っているために、やや冗長に思われる表現となっている。}作ったルー}\ldots{}\ldots{}500g。

\textbf{魚のフュメ(フュメドポワソン)(ソースを仕上げるために必要な全量)}\ldots{}\ldots{}10L。

\textbf{ミルポワ}\ldots{}\ldots{}標準的なソース・エスパニョルと同じミルポワ野菜を同量と、
塩漬け豚ばら肉の代わりにバターを用い、マッシュルームまたはマッシュルー
ムの切りくず250gを加える。

\textbf{作業手順}\ldots{}\ldots{}標準的なソース・エスパニョルとまったく同様に作る。

\textbf{加熱時間と夾雑物を取り除くのに必要な時間}\ldots{}\ldots{}5時間。

仕上げに漉してから、標準的なソース・エスパニョルとまったく同様に、完全
に冷めるまでゆっくり混ぜ続けること。

\subsubsection{魚料理用ソース・エスパニョルについての注意}\label{ux9b5aux6599ux7406ux7528ux30bdux30fcux30b9ux30a8ux30b9ux30d1ux30cbux30e7ux30ebux306bux3064ux3044ux3066ux306eux6ce8ux610f}

このソースを日常的な料理のベースとなる仕込みに含めるかどうかについては
意見が分れるところだ。

普通のソース・エスパニョルは、つまるところ風味の点ではほとんどニュート
ラルなものだから、それに魚のフュメを加えれば、魚料理用ソース・エスパニョ
ルとして充分に通用するだろう。どうしても上で挙げた魚料理用ソース・エス
パニョルが必要になるのは、宗教的に厳格に小斉の決まりを守って料理を作る
場合のみで、さすがにその場合は代用品などない。

\subsection[ソース・ドゥミグラス SAUCE
DEMI-GLACE]{\texorpdfstring{ソース・ドゥミグラス\footnote{日本の洋食などで一般的な「デミグラス」とはかなり異なった仕
  上りのソースであることに注意。ソース・エスパニョルの仕上げにあたっ
  て、徹底的に夾雑物を取り除くことを何度も強調しているのは、透き通っ
  た茶色がかった色合いの、なめらかなソースを目指すからであり、それを
  さらに徹底させるということは、透明度、なめらかさの面でさらに徹底さ
  せることを意味するからだ。} SAUCE
DEMI-GLACE}{ソース・ドゥミグラス SAUCE DEMI-GLACE}}\label{ux30bdux30fcux30b9ux30c9ux30a5ux30dfux30b0ux30e9ux30b9102009-sauce-demi-glace}

一般に「ドゥミグラス」と呼ばれているものは、いったん仕上がったソース・
エスパニョルをさらに、もうこれ以上は無理という位に徹底的に夾雑物を取り
除いたもののことだ。

最後の仕上げにグラスドヴィアンドなどを加える。風味付けに何らかのワイン
を加えれば、当然ながらソースの性格も変わるので、最終的な使い途に応じて
決めること。

【原注】ソースの色合いを決めるワインを仕上げに加える際には、「火から外
して」行なうこと。沸騰しているとワインの香りがとんでしまうからだ。

\subsection{とろみを付けた仔牛のジュ JUS DE VEAU
LIE}\label{ux3068ux308dux307fux3092ux4ed8ux3051ux305fux4ed4ux725bux306eux30b8ux30e5-jus-de-veau-lie}

(仕上り1L分)

\textbf{仔牛のフォン}\ldots{}\ldots{}仔牛の茶色いフォン4L。

\textbf{とろみ付け材料}\ldots{}\ldots{}アロールート\footnote{allow-root
  南米産のクズウコンを原料とした良質のでんぷん。日
  本では入手が難しいこともあり、コーンスターチが用いられることが多い}30g。

\textbf{作業手順}\ldots{}\ldots{}よく澄んだ仔牛のフォン4Lを強火にかけ、1/4量つまり1L
になるまで煮詰める。

大さじ数杯分の冷たいフォンでアロールートを溶く。これを沸騰している鍋に
加える。一分程度だけ火にかけ続けたら、布で漉す。

【原注】この、とろみを付けた仔牛のジュは、本書では頻繁に使う指示をして
いるが、必ず、しっかりした味で透き通った、きれいな薄茶色に仕上げること。

\subsection[ヴルテ(標準的な白いソース) VELOUTE OU SAUCE BLANCHE
GRASSE]{\texorpdfstring{ヴルテ\footnote{velouté
  原義は「ビロードのように柔らかな、なめらかな」。日
  本ではベシャメルソースと混同されやすいが、内容がまったく異なるソー
  スなので注意。}(標準的な白いソース) VELOUTE OU SAUCE BLANCHE
GRASSE}{ヴルテ(標準的な白いソース) VELOUTE OU SAUCE BLANCHE GRASSE}}\label{ux30f4ux30ebux30c6102013ux6a19ux6e96ux7684ux306aux767dux3044ux30bdux30fcux30b9-veloute-ou-sauce-blanche-grasse}

(仕上がり5L分)

\textbf{とろみ付けの材料}\ldots{}\ldots{}バターを用いて作った\footnote{魚料理用ソース・エスパニョル、訳注XX参照。}きつね色のルー625g。

\textbf{よく澄んだ仔牛の白いフォン}\ldots{}\ldots{}5L。

\textbf{作業手順}\ldots{}\ldots{}ルーをフォンに溶かし込む。フォンは冷たくても熱くてもい
いが、フォンが熱い場合にはソースが充分なめらかになるよう注意して溶かす
こと。混ぜながら沸騰させる。微沸騰の状態を保ちながら、浮いてくる夾雑物
を完全に取り除いていく\footnote{デプイエのこと。ソース・エスパニョル、訳注2参照。}。この作業はとりわけ細心の注意を払って
行なうこと。

\textbf{加熱時間と夾雑物を取り除く作業に必要な時間}\ldots{}\ldots{}1時間半。

その後、ヴルテを布で漉す\footnote{ある程度濃度のある液体やピュレを布で漉す場合、昔は「二人が
  かりで行なう必要があり、それぞれが巻いた布の端を左手に持ち、右手に
  持った木杓子を使って圧し搾る」(『ラルース・ガストロノミーク』初版、
  1938年)という方法が一般的だった。}。陶製の鍋に移してゆっくり混ぜながら完全に冷
ます。

\subsection{鶏のヴルテ VELOUTE DE
VOLAILLE}\label{ux9d8fux306eux30f4ux30ebux30c6-veloute-de-volaille}

このヴルテの作り方は、上で書いた標準的なヴルテと材料比率と作業はまっ
たく同じ。使用する液体として鶏の白いフォン(フォンドヴォライユ)を使う。

\subsection{魚料理用ヴルテ VELOUTE DE
POISSON}\label{ux9b5aux6599ux7406ux7528ux30f4ux30ebux30c6-veloute-de-poisson}

ルーと液体の分量は標準的なヴルテとまったく同じだが、仔牛のフォンでは
なく魚のフュメを用いて作る。

ただし、魚を素材として用いるストックはどれもそうだが、手早く作業するこ
と。夾雑物を取り除く作業も20分程度にとどめること。その後、布で漉し、陶
製の鍋に移してゆっくり混ぜながら完全に冷ます。

\subsection{パリ風ソース(旧名ソース・アルマンド)SAUCE PARISIENNE
(ex-Allemande)}\label{ux30d1ux30eaux98a8ux30bdux30fcux30b9ux65e7ux540dux30bdux30fcux30b9ux30a2ux30ebux30deux30f3ux30c9sauce-parisienne-ex-allemande}

(仕上がり1L分)

標準的なヴルテに卵黄でとろみを付けたソース。

\textbf{標準的なヴルテ}\ldots{}\ldots{}1L。

\textbf{追加素材}\ldots{}\ldots{}卵黄5個、白いフォン(冷たいもの)1/2L、粗く砕いたこしょ
う1ひとつまみ、すりおろしたナツメグ少々、マッシュルームの煮汁2dl、レ
モン汁少々。

\textbf{作業手順}\ldots{}\ldots{}厚手のソテー鍋にマッシュルームの煮汁と白いフォン、卵黄、
粗く砕いたこしょう、ナツメグ、レモン汁を入れる。泡立て器でよく混ぜ、そ
こにヴルテを加える。火にかけて沸騰させ、強火で2/3量になるまで、ヘラで
混ぜながら煮詰める。

ヘラの表面がソースでコーティングされる状態になるまで煮詰めたら、布で漉す。

膜が張らないよう、表面にバターのかけらをいくつか載せてやり、湯煎にかけ
ておく。

\textbf{仕上げ}\ldots{}\ldots{}提供直前に、バター100gを加えて仕上げる。

【原注】ソース・アルマンド(ドイツ風)とも呼ばれるが、本書では「パリ風」
の名称を採用した。そもそも「アルマンド」というの名称に正当性がないから
だ。習慣としてそう呼ばれてきただけであって、明らかに理屈に合わない名称
だ\footnote{エスコフィエは普仏戦争に従軍した経歴があり、ドイツ嫌いとし
  て知られていた。}。1883年に雑誌「料理技術」にタヴェルネとかいう人が寄せた記
事には、当時ある優秀な料理人がアルマンドなどという理屈に合わない名称を
使うのはやめたという話が出ている。

こんにち既に「パリ風ソース」の名称を採用している料理長もいる。そう呼ん
だほうが好ましいわけだが、残念なことにまだ一般的にはなっていない\footnote{エスコフィエの願いもむなしく、現代においてもソース・アルマ
  ンドの名称で定着している。なお、「ドイツ風」というソース名の由来に
  ついては、ソースの淡い黄色がドイツ人に多い金髪を想起させるからだと
  カレームは述べている。}。

\subsection{ソース・シュプレーム SAUCE
SUPREME}\label{ux30bdux30fcux30b9ux30b7ux30e5ux30d7ux30ecux30fcux30e0-sauce-supreme}

鶏のヴルテに生クリーム\footnote{フランスの生クリームのうち、料理でよく使われるのは、日本の
  生クリームにやや近い「クレーム・フレッシュ・パストゥリゼ」(低温殺
  菌した生クリームで乳脂肪分30〜38%)のほか、「クレーム・フレッシュ・
  エペス」(低温殺菌後に乳酸醗酵させたもので日本で一般的な生クリーム
  より濃度がある)、「クレーム・ドゥーブル」(殺菌後に乳酸醗酵させた
  もので乳脂肪分40%程度でかなり濃度がある)などがある。}を加えてなめらかに仕上げ\footnote{monter
  モンテ。原義は「上げる、ホイップする」だが、ソースの
  仕上げの際などに、バターや生クリームを加えて、なめらかに仕上げるこ
  とも「モンテ」の語を使用する場合が多い。}たもの。
ソース・シュプレームは、正しく作った場合「際だった白さでとても繊細な」
仕上がりのものでなくてはいけない。

(仕上がり1L分)

\textbf{鶏のヴルテ}\ldots{}\ldots{}1L。

\textbf{追加素材}\ldots{}\ldots{}鶏の白いフォン1L、マッシュルームの煮汁1dl、良質な生
クリーム21/2dl。

\textbf{作業手順}\ldots{}\ldots{}鍋に鶏のフォンとマッシュルームの煮汁、鶏のヴルテを入れ
て強火にかけ、ヘラで混ぜながら、生クリームを少しずつ加え、煮詰めていく。
このヴルテと生クリームを煮詰めたものの分量は、上で示した仕上がり1Lの
ソース・シュプレームを作るには、1/3量まで煮詰まっていなくてはならない。

布で漉し、仕上げに1dlの生クリームとバター80gを加えてゆっくり混ぜなが
ら冷ますと、丁度最初のヴルテと同量になる。

\subsection[ベシャメルソース SAUCE
BECHAMEL]{\texorpdfstring{ベシャメルソース SAUCE BECHAMEL\footnote{17世紀にルイ14世のメートルドテルを務めたこともあるルイ・ベ
  シャメイユ Louis Béchameil(1630〜1703)の名が冠されているこのソー
  スは、彼自身の創案あるいは彼に仕えていた料理人によるものという説も
  あったが真偽は疑わしい。17世紀頃の成立であることは確かだが、おそら
  くは古くからあったソースを改良したものに過ぎず、また、19世紀前半の
  カレームのレシピはヴルテを煮詰め、卵黄と煮詰めた生クリームでとろみ
  を付けるというものだった。同様に1867年刊グーフェ『料理の本』のレシ
  ピも、炒めた仔牛肉と野菜に小麦粉を振りかけてからブイヨン注ぎ、これ
  を煮詰め、漉してから生クリームを加えるというものだった。}}{ベシャメルソース SAUCE BECHAMEL}}\label{ux30d9ux30b7ux30e3ux30e1ux30ebux30bdux30fcux30b9-sauce-bechamel102020}

(仕上がり5L分)

\textbf{白いルー}\ldots{}\ldots{}650g。

\textbf{使用する液体}\ldots{}\ldots{}沸かした牛乳5L。

\textbf{追加素材}\ldots{}\ldots{}白身で脂肪のない仔牛肉300gをさいの目に切り、みじん切り
にした玉ねぎ(小)2個分とタイム1枝、粗く砕いたこしょう1つまみ、塩
25gとバターを鍋に入れて蓋をし、色付かないように弱火で蒸し煮したもの。

\textbf{作業手順}\ldots{}\ldots{}沸かした牛乳でルーを溶く。混ぜながら沸騰させる。ここに、
先に蒸し煮しておいた野菜と調味料、仔牛肉を加える。弱火で1時間煮込む。
布で漉し\footnote{XX脚注参照。}、表面にバターのかけらをいくつか載せて膜が張らないよ
うにする。肉類を絶対に使わない\footnote{小斉のこと。XX脚注参照。}で調理する必要がある場合は、仔
牛肉を省き、香味野菜などは上記のとおりに作ること。

このソースは次のようなやり方をすると手早く作ることも出来る。沸かした牛
乳に塩、薄切りにした玉ねぎ、タイム、粗く砕いたこしょう、ナツメグを加え
る。蓋をして弱火で10分煮る。これを漉してルーを入れた鍋の中に入れ、強火
にかけて沸騰させる。その後15〜20分だけ煮込めばいい。

\subsection{トマトソース SAUCE
TOMATE}\label{ux30c8ux30deux30c8ux30bdux30fcux30b9-sauce-tomate}

(仕上がり5L分)

\textbf{主素材}\ldots{}\ldots{}トマトピュレ4L、または生のトマト6kg。

\textbf{ミルポワ}\ldots{}\ldots{}さいの目に切って下茹でしておいた塩漬け豚ばら肉140g、1〜
2mm角のさいの目に刻んだにんじん200gと玉ねぎ150g、ローリエの葉1枚、タ
イム1枝、バター100g。

\textbf{追加素材}\ldots{}\ldots{}小麦粉150g、白いフォン2L、にんにく2片。

\textbf{調味料}\ldots{}\ldots{}塩20g、砂糖30g、こしょう1つまみ。

\textbf{作業手順}\ldots{}\ldots{}厚手の片手鍋で、塩漬け豚ばら肉を軽く色付くまで炒める。
ミルポワの野菜を加え、野菜も色よく炒める。小麦粉を振りかける。きつね色
になるまで炒めてから、トマトピュレまたは潰した生トマトと白いフォン、砕
いたにんにく、塩、砂糖、こしょうを加える。

火にかけて混ぜながら沸騰させる。鍋に蓋をして弱火のオーブンに入れ1時間
半〜2時間加熱する。

目の細かい漉し器または布で漉す。再度、火にかけて数分間沸騰させる。保存
用の器に移し、ソースが空気に触れて表面に膜が張らないよう、バターのかけら
を載せてやる。

【原注】トマトピュレを使い、小麦粉は使わず、その他は上記のとおりに作っ
てもいい。漉し器か布で漉してから、充分な濃度になるまでしっかり煮詰めて
やること。




%\end{recette}%%2段組おわり

\end{document}
