\hypertarget{ux30dbux30efux30a4ux30c8ux7cfbux306eux6d3eux751fux30bdux30fcux30b9}{%
\section{ホワイト系の派生ソース}\label{ux30dbux30efux30a4ux30c8ux7cfbux306eux6d3eux751fux30bdux30fcux30b9}}

\hypertarget{petites-sauces-blanches-composuxe9es-et-de-ruxe9ductions}{%
\subsection{Petites Sauces Blanches, Composées et de
Réductions}\label{petites-sauces-blanches-composuxe9es-et-de-ruxe9ductions}}
\begin{recette}
\hypertarget{ux30bdux30fcux30b9ux30a2ux30ebux30d3ux30e5ux30d5ux30a7ux30e91}{%
\subsubsection[ソース・アルビュフェラ]{\texorpdfstring{ソース・アルビュフェラ\footnote{ナポレオン軍の元帥、ルイ・ガブリエル・スーシェ
  Louis-Gabriel Suchet, duc d'Albufera
  (1770〜1826)のこと。スペイン戦役の際にそれ
  までの軍功を称えられ、ナポレオンが1812年にアルビュフェラ公爵位を新
  設して授けた。帝政期の英雄のひとりであり、アルビュフェラおよびスー
  シェの名を冠した料理がいくつかある。1814年に帝政が崩壊した後も軍務、
  政務に携わり、最終的にフランス貴族院議員の地位を得た。アルビュフェ
  ラ公爵位については、1815年7月24日の勅令においてに正式に抹消されて
  いる。このソースの特徴は赤ピーマン(パプリカ)を加熱してなめらかに
  すり潰し、バターに練り込んだものを使う点にあるが、どのような経緯で
  このソースに赤ピーマンを用いるようになったのかは不明。ただし、この
  ソースを合わせる「肥鶏 アルビュフェラ」は詰め物(ファルス)に米を
  用いるが、アルビュフェラは湖の周辺の湿地帯で米の生産がおこなわれて
  いるという点では一応の関連性が認められよう。なお、アルビュフェラは
  バレンシアの湖とそこに形成された潟であり、現在はバレンシア州のアル
  ブフェーラ自然公園となっている。}}{ソース・アルビュフェラ}}\label{ux30bdux30fcux30b9ux30a2ux30ebux30d3ux30e5ux30d5ux30a7ux30e91}}

\hypertarget{sauce-albufera}{%
\paragraph{Sauce Albuféra}\label{sauce-albufera}}

\index{そーす@ソース!あるひゆふえら@---・アルビュフェラ}
\index{あるひゆふえら@アルビュフェラ!そーす@ソース・---}
\index{sauce@sauce!albufera@--- Albuféra}
\index{albufera@Albuféra!sauce@Sauce ---}

\protect\hyperlink{sauce-supreme}{ソース・シュプレーム}1
Lあたりに、溶かしたブロンド色
の\protect\hyperlink{glace-de-viande}{グラスドヴィアンド}2
dlと、標準的な分量比率で作っ た\href{}{赤ピーマンバター}50 gを加える。

\maeaki

\hypertarget{ux30bdux30fcux30b9ux30a2ux30e1ux30eaux30b1ux30fcux30cc3}{%
\subsubsection[ソース・アメリケーヌ]{\texorpdfstring{ソース・アメリケーヌ\footnote{アメリケーヌという名称の由来は諸説あるが、19世紀フランスの料理人
  ピエール・フレス Pierre Fraysse がアメリカで働いた後にパリで1853年
  に開いたレストラン「シェ・ピーターズ」でこの料理名で提供したという
  のが定説。ただし、1853年以前にレストラン「ボヌフォワ」に「ラングドッ
  ク産オマール ソース・アメリケーヌ添え」というメニューあり、フレス
  はその料理に改変を加えたか、名前だけをシンプルに「アメリケーヌ」と
  した程度という説もある。かつては、オマールの主産地のひとつブルター
  ニュ地方を意味する古い形容詞 armoricain(e) アルモリカン、アルモリ
  ケーヌの音が変化した料理名だと主張されることもあったが、19世紀には
  南仏産が中心であったトマトを用いる点で矛盾が生じてしまう。いずれに
  しても、この料理名がフレスの店シェ・ピーターズを基点として広く知ら
  れるようになったことは事実と考えていい。}}{ソース・アメリケーヌ}}\label{ux30bdux30fcux30b9ux30a2ux30e1ux30eaux30b1ux30fcux30cc3}}

\hypertarget{sauce-americaine}{%
\paragraph{Sauce Américaine}\label{sauce-americaine}}

\index{そーす@ソース!あめりけーぬ@---・アメリケーヌ}
\index{あめりふう@アメリカ風!そーす@ソース・アメリケーヌ}
\index{sauce@sauce!americaine@--- Américaine}
\index{americain@américain!sauce americaine@Sauce Américaine}

このソースは\protect\hyperlink{homard-a-l-americaine}{オマール・アメリケーヌ}という料理
そのものと言っていい(「魚料理」の章、甲殻類、\protect\hyperlink{homard-a-l-americaine}{オマール・アメリケー
ヌ}参照)。

このソースは通常、オマールの身をガルニチュールとした魚料理に添えられる。
オマールの身をやや斜めになるよう厚さ1 cm程度の輪切りにし\footnote{escalopper
  エスカロペ。エスカロップに切る。ここで使用するオマー
  ルは900g〜1kg程度のものを想定していることに注意。}、魚料理の
ガルニチュールとして供するわけだ。

\maeaki

\hypertarget{ux30a2ux30f3ux30c1ux30e7ux30d3ux30bdux30fcux30b9}{%
\subsubsection{アンチョビソース}\label{ux30a2ux30f3ux30c1ux30e7ux30d3ux30bdux30fcux30b9}}

\hypertarget{sauce-anchois}{%
\paragraph{Sauce Anchois}\label{sauce-anchois}}

\index{そーす@ソース!あんちょうい@アンチョビ---}
\index{あんちょひ@アンチョビ!そーす@---ソース}
\index{sauce@sauce!anchois@--- Anchois}
\index{anchois@anchois!sauce anchois@Sauce ---}

\href{}{ノルマンディー風ソース}8
dlを、バターを加える前の段階まで作る。\href{}{ア ンチョビバター}125
gを混ぜ込む。アンチョビのフィレ50 gを洗い、よく水
気を絞ってから小さなさいの目に切ったのを加えて仕上げる。

\ldots{}\ldots{}魚料理用。

\maeaki

\hypertarget{ux30bdux30fcux30b9ux30aaux30fcux30edux30fcux30eb4}{%
\subsubsection[ソース・オーロール]{\texorpdfstring{ソース・オーロール\footnote{夜明けの光、曙光のこと。オーロラの意味もあるため、日本では「オー
  ロラソース」と呼ばれることもあるが、マヨネーズとトマトケチャップを
  同量で混ぜ合わせたものもそう呼ばれることが多いので注意。}}{ソース・オーロール}}\label{ux30bdux30fcux30b9ux30aaux30fcux30edux30fcux30eb4}}

\hypertarget{sauce-aurore}{%
\paragraph{Sauce Aurore}\label{sauce-aurore}}

\index{そーす@ソース!おーろーる@---・オーロール}
\index{おーろーる@オーロール!そーす@ソース・---}
\index{sauce@sauce!aurore@--- Aurore}
\index{aurore@aurore!sauce@Sauce ---}

\protect\hyperlink{veloute}{ヴルテ}に真っ赤なトマトピュレを加えたもの。分量は、ヴルテが\troisquarts{}に対し、トマトピュレ\unquart{}とする。仕上げに、ソース1
Lあたり100 gのバターを加える。

\ldots{}\ldots{}卵料理、仔牛、仔羊肉の料理、鶏料理用。

\maeaki

\hypertarget{ux9b5aux6599ux7406ux7528ux30bdux30fcux30b9ux30aaux30fcux30edux30fcux30eb}{%
\subsubsection{魚料理用ソース・オーロール}\label{ux9b5aux6599ux7406ux7528ux30bdux30fcux30b9ux30aaux30fcux30edux30fcux30eb}}

\hypertarget{sauce-aurore-maigre}{%
\paragraph{Sauce Aurore maigre}\label{sauce-aurore-maigre}}

\index{そーす@ソース!おーろーるさかなよう@魚料理用---・オーロール}
\index{おーろーる@オーロール!そーすさかな@魚料理用ソース・---}
\index{sauce@sauce!aurore maigre@--- Aurore maigre}
\index{aurore@aurore!sauce maigre@Sauce --- maigre}

\protect\hyperlink{veloute-de-poisson}{魚料理用ヴルテ}に、上記と同じ割合でトマトピュレ
を加える。ソース1 Lあたりバター125 gを加えて仕上げる。

\ldots{}\ldots{}魚料理用

\maeaki

\hypertarget{ux30d0ux30a4ux30a8ux30ebux30f3ux98a8ux30bdux30fcux30b9}{%
\subsubsection{バイエルン風ソース}\label{ux30d0ux30a4ux30a8ux30ebux30f3ux98a8ux30bdux30fcux30b9}}

\hypertarget{sauce-bavaroise}{%
\paragraph{Sauce Bavaroise}\label{sauce-bavaroise}}

\index{そーす@ソース!はいえるんふう@バイエルン風---}
\index{はいえるんふう@バイエルン風!そーす@---ソース}
\index{sauce@sauce!bavarois@--- Bavaroise}
\index{bavarois@bavarois!sauce bavaroise@Sauce Bavaroise}

ヴィネガー5
dlにタイムとローリエの葉少々とパセリの枝4本、大粒のこしょう7〜8個と、おろした\footnote{原文
  râpé \textless{} râpe
  ラープと呼ばれる器具を用いておろすが、日本のおろし金と目の大きさが違うので注意。多くの場合、マンドリーヌ
  mandrine と呼ばれる野菜用スライサーにこの機能が付属している。}レフォール\footnote{raifort
  西洋わさび、ホースラディッシュ。}大さじ2杯を加え、半量になるまで煮詰める。

この煮詰めた汁に卵黄6個を加え\footnote{卵黄を加える前に一度漉しておいたほうがいいだろう。}、\protect\hyperlink{sauce-hollandaise}{オランデーズソース}を作る要領で、バター400
gと大さじ1\undemi{}杯の水を少しずつ加えながら、ソースがしっかり乳化するまで混ぜていく。布で漉す。

\protect\hyperlink{beurre-d-ecrevisse}{エクルヴィスバター}100
gと泡立てた生クリーム大さじ2杯、さいの目に切ったエクルヴィスの尾の身を加えて仕上げる。

\ldots{}\ldots{}魚料理用のこのソースは、ムースのような仕上りにすること。

\maeaki

\hypertarget{ux30bdux30fcux30b9ux30d9ux30a2ux30ebux30cdux30fcux30ba8}{%
\subsubsection[ソース・ベアルネーズ]{\texorpdfstring{ソース・ベアルネーズ\footnote{ベアルヌというのは旧地方名で、フランス南西部、現在のピレネー・ア
  トランティック県のことを指すが、このソースはその地方とはまったく関
  係がない。19世紀パリ郊外のレストラン「パヴィヨン・アンリIV」の店名
  に掲げられているアンリ四世がベアルヌのポーで生誕したことにちなんで
  命名したソース名というのが定説。}}{ソース・ベアルネーズ}}\label{ux30bdux30fcux30b9ux30d9ux30a2ux30ebux30cdux30fcux30ba8}}

\hypertarget{sauce-bearnaise}{%
\paragraph{Sauce Béarnaise}\label{sauce-bearnaise}}

\index{そーす@ソース!へあるねーす@---・ベアルネーズ}
\index{へあるぬふう@ベアルヌ風!そーす@ソース・---}
\index{へあるねーす@ベアルネーズ!そーす@ソース・---}
\index{sauce@sauce!bearnaise@--- Béarnaise}
\index{bearnais@béarnais!sauce bearnaise@Sauce Béarnaise}

白ワイン2 dlとエストラゴンヴィネガー2 dlに、エシャロットのみじん切り大
さじ4杯、枝のままの粗く刻んだエストラゴン20 g、セルフイユ10 g、粗挽き
こしょう5 g、塩1つまみを加えて、\untiers{}量になるまで煮詰める。

煮詰まったら、数分間放置して温度を下げる。ここに卵黄6個を加え、弱火に
かけて、生のバター(あるいはあらかじめ溶かしておいてもいい)500 gを加
えて軽くホイップしながらなめらかになるよう混ぜる。

卵黄に徐々に火が通っていくことでソースにとろみが付くので、絶対に弱火で
作業をすること\footnote{卵黄をソースのとろみ付けに用いること自体は中世から行なわれていた。
  開放式の炉の上に鍋を鉤で吊っている場合は鍋を火から外す必要があった
  が、その後の閉鎖式かまどや、オーブンの機能も備えた fourneau フルノー
  (日本の調理現場ではストーブあるいはピアノと呼ばれることも多い)の
  場合、熱の弱い部分に鍋を置けばいいことになる。また、このソースのよ
  うにバターが中心となる場合は水よりも高温になりやすいので本文にある
  ように注意が必要だが、ブランケットのような水が中心のものに卵黄を加
  えてとろみを付ける場合、よく溶きほぐした卵黄を、鍋全体をしっかり混
  ぜながら加えれば比較的高温(微沸騰程度)でも問題なくきれいにとろみ
  が付く。}。

バターを混ぜ込んだら、布で漉して味を調える。カイエンヌごく少量を加えて
風味を引き締める。仕上げに、刻んだエストラゴン大さじ杯とセルフイユ大さ
じ\undemi{}杯を加える。

\ldots{}\ldots{}牛、羊肉のグリル用。

\hypertarget{ux539fux6ce8}{%
\subparagraph{【原注】}\label{ux539fux6ce8}}

このソースを熱々で提供しようとは考えないこと。このソースは要するにバター
で作ったマヨネーズなのだ。ほの温い程度で充分であり、もし熱くし過ぎてし
まうと、ソースが分離してしまう。

そうなってしまったら、冷水少々を加えて泡立て器でホイップして元のあるべ
き状態に戻してやること。

\maeaki

\hypertarget{ux30c8ux30deux30c8ux5165ux308aux30bdux30fcux30b9ux30d9ux30a2ux30ebux30cdux30fcux30ba-ux30bdux30fcux30b9ux30b7ux30e7ux30edux30f310}{%
\subsubsection[トマト入りソース・ベアルネーズ /
ソース・ショロン]{\texorpdfstring{トマト入りソース・ベアルネーズ /
ソース・ショロン\footnote{19世紀後半、パリで有名レストラン「ヴォワザン」の料理長を務めた
  アレクサンドル・ショロン Alexandre Choron (1837〜1924)。自ら考案
  し、命名したという。}}{トマト入りソース・ベアルネーズ / ソース・ショロン}}\label{ux30c8ux30deux30c8ux5165ux308aux30bdux30fcux30b9ux30d9ux30a2ux30ebux30cdux30fcux30ba-ux30bdux30fcux30b9ux30b7ux30e7ux30edux30f310}}

\hypertarget{sauce-bearnaise-tomatee}{%
\paragraph{Sauce Béarnaise tomatée, dite Sauce
Choron}\label{sauce-bearnaise-tomatee}}

\index{そーす@ソース!へあるねーすとまといり@トマト入り---・ベアルネーズ}
\index{へあるぬふう@ベアルヌ風!そーすとまといり@トマト入りソース・ベアルネーズ}
\index{そーす@ソース!しょろん@---・ショロン}
\index{しょろん@ショロン!そーす@ソース・---}
\index{sauce@sauce!bearnaise tomatee@--- Béarnaise tomatée}
\index{bearnais@b\'earnais!sauce bearnaise tomatee@Sauce Béarnaise tomatée}
\index{sauce@sauce!choron@--- Choron}
\index{choron@Choron!sauce@Sauce ---}

ソース・ベアルネーズを上記のとおりに作るが、最後にセルフイユとエストラ
ゴンのみじん切りは加えない。充分固めに作っておき、ソースの\unquart{}量
の、充分に煮詰めたトマトピュレを加える。ソースの濃度が丁度いい具合にな
るよう注意すること。

\ldots{}\ldots{}\href{}{トゥルヌド・ショロン}、および他のさまざまな料理に添える。

\maeaki

\hypertarget{ux30b0ux30e9ux30b9ux30c9ux30f4ux30a3ux30a2ux30f3ux30c9ux5165ux308aux30bdux30fcux30b9ux30d9ux30a2ux30ebux30cdux30fcux30ba-ux30bdux30fcux30b9ux30d5ux30a9ux30a4ux30e811-ux30bdux30fcux30b9ux30f4ux30a1ux30edux30ef12}{%
\subsubsection[グラスドヴィアンド入りソース・ベアルネーズ /
ソース・フォイヨ /
ソース・ヴァロワ]{\texorpdfstring{グラスドヴィアンド入りソース・ベアルネーズ
/ ソース・フォイヨ\footnote{19世紀〜20世紀初頭にパリにあったレストランおよびそのオーナーシェ
  フの名。このソースを使った「仔牛の背肉・フォイヨ」がスペシャリテだっ
  たという。} / ソース・ヴァロワ\footnote{ヴァロワ王家およびヴァロワ公爵であったルイ・フィリップ(7月王政
  期のフランス国王。在位1830〜1848)にちなんだ名称。前出のフォイヨは
  レストランを開く以前、ルイ・フィリップに仕えていた。}}{グラスドヴィアンド入りソース・ベアルネーズ / ソース・フォイヨ / ソース・ヴァロワ}}\label{ux30b0ux30e9ux30b9ux30c9ux30f4ux30a3ux30a2ux30f3ux30c9ux5165ux308aux30bdux30fcux30b9ux30d9ux30a2ux30ebux30cdux30fcux30ba-ux30bdux30fcux30b9ux30d5ux30a9ux30a4ux30e811-ux30bdux30fcux30b9ux30f4ux30a1ux30edux30ef12}}

\hypertarget{sauce-bearnaise-a-la-glace-de-viande}{%
\paragraph{Sauce Béarnaise à la glace de viande, dite Foyot, ou
Valois}\label{sauce-bearnaise-a-la-glace-de-viande}}

\index{へあるぬふう@ベアルヌ風!そーすぐらすとういあんといり@グラスドヴィアンド入りソース・ベアルネーズ}
\index{そーす@ソース!へあるねーすくらすどういあんといり@---・ベアルネーズ(グラス・ド・ヴィアンド入り)}
\index{そーす@ソース!ふおいよ@---・フォイヨ}
\index{ふおいよ@フォイヨ!そーす@ソース・---}
\index{そーす@ソース!うあろわ@---・ヴァロワ}
\index{うあろわ@ヴァロワ!そーす@ソース・---}
\index{sauce@sauce!bearnaise a la glace de viande@--- Béarnaise à la glace de viande}
\index{bearnais@b\'earnais!sauce bearnaise a la glace de viande@Sauce Béarnaise à la glace de viande}
\index{sauce@sauce!foyot@--- Foyot} \index{foyot@Foyot!sauce@Sauce ---}
\index{sauce@sauce!valois@--- Valois}
\index{valois@Valois!sauce@Sauce ---}

標準的な\protect\hyperlink{sauce-bearnaise}{ソース・ベアルネーズ}を上記の分量で、固めに作る。溶かした\protect\hyperlink{glace-de-viande}{グラスドヴィアンド}を少しずつ加えて仕上げる。

\ldots{}\ldots{}牛、羊肉のグリル用。

\maeaki

\hypertarget{ux30bdux30fcux30b9ux30d9ux30ebux30b7ux30fc13}{%
\subsubsection[ソース・ベルシー]{\texorpdfstring{ソース・ベルシー\footnote{パリ東部、セーヌ川左岸にある地名。かつては荷揚げ港があり、19世
  紀には小さなレストランが多く店を構えていたという。}}{ソース・ベルシー}}\label{ux30bdux30fcux30b9ux30d9ux30ebux30b7ux30fc13}}

\hypertarget{sauce-bercy}{%
\paragraph{Sauce Bercy}\label{sauce-bercy}}

\index{そーす@ソース!へるしー@---・ベルシー}
\index{へるしー@ベルシー!そーす@ソース・---}
\index{sauce@sauce!bercy@--- Bercy} \index{bercy@Bercy!sauce@Sauce ---}

細かくみじん切りにしたエシャロット大さじ2杯をバターでさっと色付かない
よう炒める。白ワイン2\undemi{}
dlと\protect\hyperlink{fumet-de-poisson}{魚のフュメ}か、
このソースを合わせる魚の煮汁2\undemi{} dlを注ぐ。

\deuxtiers{}量弱まで煮詰めたら、\protect\hyperlink{veloute-de-poisson}{ヴル
テ}\troisquarts{} Lを加える。ひと煮立ちさせてから、
鍋を火から外し、バター100 gとパセリのみじん切り大さじ1杯を加えて仕上げ
る。

\maeaki

\hypertarget{ux30bdux30fcux30b9ux30aaux30d6ux30fcux30eb16-ux30bdux30fcux30b9ux30d0ux30bfux30ebux30c914}{%
\subsubsection[ソース・オ・ブール /
ソース・バタルド]{\texorpdfstring{ソース・オ・ブール\footnote{本書には、日本でもかつて有名だった、エシャロットのみじん切りを
  加えたヴィネガーを煮詰めてバターを溶かし込んだ魚料理用ソース「ソー
  ス・ブールブラン」Sauce (au) Beurre blanc は収録されていない。この
  ソース・ブールブランはナント地方やアンジュー地方で淡水魚アローズや
  ブロシェに合わせる伝統的なソース。1890年頃にナント地方の女性料理人
  クレマンス・ルフーヴルが、ソース・ベアルネーズを作るつもりが誤って
  卵を加えるのを忘れてしまった結果として出来たものだとも言われている。}
/ ソース・バタルド\footnote{バタルドは「雑種の、中間の」の意。卵黄とバターだけでとろみを付
  ける\protect\hypertarget{sauce-hollandaise}{}{ソース・オランデーズ}と似てはいるが小麦粉
  も使うことからこの名が付いたと言われている。なお、パンのバタール
  bâtard も同じ語だが、細いバゲットと太いドゥーリーヴルの「中間」
  の太さとだからというのが通説。}}{ソース・オ・ブール / ソース・バタルド}}\label{ux30bdux30fcux30b9ux30aaux30d6ux30fcux30eb16-ux30bdux30fcux30b9ux30d0ux30bfux30ebux30c914}}

\hypertarget{sauce-au-beurre}{%
\paragraph{Sauce au Beurre, dite Sauce Bâtarde}\label{sauce-au-beurre}}

\index{そーす@ソース!ふーる@---・オ・ブール}
\index{はたー@バター!そーす@ソース・オ・ブール}
\index{そーす@ソース!はたると@---・バタルド}
\index{はたると@バタルド!そーす@ソース・---}
\index{sauce@sauce!beurre@--- au Beurre}
\index{beurre@beurre!sauce@Sauce au Beurre}
\index{sauce@sauce!batarde@--- Bâtarde}
\index{batard@bâtard!sauce@Sauce Bâtarde}

小麦粉45 gと溶かしバター45gをよく混ぜ合わせ粘土状にする。そこに、7 gの
塩を加えた熱湯7\undemi{} dlを一気に注ぎ、泡立て器で勢いよく混ぜ合わせ
る。とろみ付け用の卵黄5個を生クリーム大さじ1\undemi{}杯でゆるめたもの
と、レモン汁少々を加える。

布で漉し、鍋を火から外して、良質なバター300gを加えて仕上げる。

\ldots{}\ldots{}アスパラガスや、さまざまな魚のブイイ\footnote{茹でたもの、の意。料理では、シンプルに茹でた肉、魚のこと。}

\hypertarget{ux539fux6ce8-1}{%
\subparagraph{【原注】}\label{ux539fux6ce8-1}}

このソースはとろみを付けた後、湯煎にかけておき、提供直前にバターを加え
るようにするといい。

\maeaki

\hypertarget{ux30bdux30fcux30b9ux30dcux30ccux30d5ux30a9ux30ef-ux767dux30efux30a4ux30f3ux3067ux4f5cux308bux30dcux30ebux30c9ux30fcux98a8ux30bdux30fcux30b9}{%
\subsubsection{ソース・ボヌフォワ /
白ワインで作るボルドー風ソース}\label{ux30bdux30fcux30b9ux30dcux30ccux30d5ux30a9ux30ef-ux767dux30efux30a4ux30f3ux3067ux4f5cux308bux30dcux30ebux30c9ux30fcux98a8ux30bdux30fcux30b9}}

\hypertarget{sauce-bonnefoy}{%
\paragraph{Sauce Bonnefoy, ou Sauce Bordelaise au vin
blanc}\label{sauce-bonnefoy}}

\index{ほぬふおわ@ボヌフォワ!そーす@ソース・---}
\index{そーす@ソース!おぬふおわ@---・ボヌフォワ}
\index{そーす@ソース!ほるどーふうしろわいん@ボルドー風--- (白)}
\index{ほるどーふう@ボルドー風!そーす@---ソース(白)}
\index{sauce@sauce!bonnefoy@--- Bonnefoy}
\index{bonnefoy@Bonnefoy!sauce@Sauce ---}
\index{sauce@sauce!bordelaise vin blanc@--- Bordelaise au vin blanc}
\index{bordelais@bordelais!sauce vin blanc@Sauce Bordelaise au vin blanc}

ブラウン系の派生ソースの節で採り上げた、赤ワインを用いて作る\protect\hyperlink{sauce-bordelaise}{ボルドー
風ソース}とまったく同じ作り方だが、赤ワインではなく、
グラーヴかソテルヌの白ワインを用いる。また\protect\hyperlink{sauce-espagnole}{ソース・エスパニョ
ル}ではなく\protect\hyperlink{veloute}{標準的なヴルテ}を使うこと。

このソースは仕上げに、みじん切りにしたエストラゴンを加える。

\ldots{}\ldots{}魚のグリル、白身肉のグリル用。

\maeaki

\hypertarget{ux30d6ux30ebux30bfux30fcux30cbux30e5ux98a8ux30bdux30fcux30b9}{%
\subsubsection{ブルターニュ風ソース}\label{ux30d6ux30ebux30bfux30fcux30cbux30e5ux98a8ux30bdux30fcux30b9}}

\hypertarget{sauce-bretonne}{%
\paragraph{Sauce Bretonne}\label{sauce-bretonne}}

\index{そーす@ソース!ぶるたーにゅふうしろ@ブルターニュ風---(ホワイト系)}
\index{ぶるたーにゅふう@ブルターニュ風!そーすしろ@---ソース(ホワイト系)}
\index{sauce@sauce!bretonne blanche@--- Bretonne (blanche)}
\index{breton@breton!sauce blanche@Sauce Bretonne (blanche)}

長さ3〜5 cm位の、ごく細い千切り\footnote{julienne ジュリエンヌ。}にしたポワローの白い部分30
gとセロリの白い部分30 g、玉ねぎ30 g、マッシュルーム30
gをバターで完全に火が通るまで鍋に蓋をして弱火で蒸し煮する\footnote{étuver
  エチュヴェ。本来は油脂とごく少量の水分を加えて弱火で蒸し煮することだが、野菜については、バターだけを使う場合も多い。étouffer
  エトゥフェとほぼ同じ意味で用いられることも多い。}。

\protect\hyperlink{veloute-de-poisson}{魚のヴルテ}\troisquarts{}
Lを加え、しばらく弱火にかけて浮いてくる不純物を丁寧に取り除く\footnote{dépouiller
  デプイエ ≒ écumer エキュメ。}。生クリーム大さじ3杯とバター50gを加えて仕上げる。

\maeaki

\hypertarget{ux30bdux30fcux30b9ux30abux30ceux30c6ux30a3ux30a8ux30fcux30eb20}{%
\subsubsection[ソース・カノティエール]{\texorpdfstring{ソース・カノティエール\footnote{小舟の漕ぎ手、の意。}}{ソース・カノティエール}}\label{ux30bdux30fcux30b9ux30abux30ceux30c6ux30a3ux30a8ux30fcux30eb20}}

\hypertarget{sauce-canotiuxe8re}{%
\paragraph{Sauce Canotière}\label{sauce-canotiuxe8re}}

\index{そーす@ソース!かのてぃえーる@---・カノティエール}
\index{かのてぃえーる@カノティエール!そーす@ソース・---}
\index{sauce@sauce!canotiere@--- Canotière}
\index{canotiere@Canotière!sauce@Sauce ---}

淡水魚を煮るのに用いた、\href{}{白ワイン入りクールブイヨン}を\untiers{}量に
煮詰める。クールブイヨンにはしっかり香り付けしてあり塩はごく少量しか入っ
ていないこと。

1 Lあたり80 gのブールマニエを加えてとろみを付ける。軽く煮立たせたら、
鍋を火から外してバター150 gとカイエンヌごく少量を加えて仕上げる。

\ldots{}\ldots{}淡水魚のクールブイヨン煮用。

\hypertarget{ux539fux6ce8-2}{%
\subparagraph{【原注】}\label{ux539fux6ce8-2}}

バターでグラセした小玉ねぎと小ぶりのマッシュルームを加えると、「\protect\hyperlink{sauce-matelote-blanche}{白いソー
ス・マトロット}」の代用となる。

\maeaki

\hypertarget{ux30b1ux30a4ux30d1ux30fcux5165ux308aux30bdux30fcux30b9}{%
\subsubsection{ケイパー入りソース}\label{ux30b1ux30a4ux30d1ux30fcux5165ux308aux30bdux30fcux30b9}}

\hypertarget{sauce-aux-cuxe2pres}{%
\paragraph{Sauce aux Câpres}\label{sauce-aux-cuxe2pres}}

\index{そーす@ソース!けいぱー@ケイパー---}
\index{けいぱー@ケイパー!そーす@---ソース}
\index{sauce@sauce!capres@--- aux Câpres}
\index{capre@câpre!sauce capres@Sauce aux Câpres}

上記の\protect\hyperlink{sauce-au-beurre}{ソース・オ・ブール}に、ソース1
Lあたり大さじ4 杯のケイパーを提供直前に加える。

\ldots{}\ldots{}いろいろな種類の魚を煮た料理に用いる。

\maeaki

\hypertarget{ux30bdux30fcux30b9ux30abux30ebux30c7ux30a3ux30caux30eb21}{%
\subsubsection[ソース・カルディナル]{\texorpdfstring{ソース・カルディナル\footnote{カトリックの枢機卿(カルディナル)の衣が伝統的に赤いものである
  ことと、オマールが「海の枢機卿」と呼ばれることに由来。}}{ソース・カルディナル}}\label{ux30bdux30fcux30b9ux30abux30ebux30c7ux30a3ux30caux30eb21}}

\hypertarget{sauce-cardinal}{%
\paragraph{Sauce Cardinal}\label{sauce-cardinal}}

\index{そーす@ソース!かるでぃなる@---・カルディナル}
\index{かるでぃなる@カルディナル!そーす@ソース・---}
\index{sauce@sauce!cardinal@--- Cardinal}
\index{cardinal@cardinal!sauce@Sauce ---}

\protect\hyperlink{sauce-bechamel}{ベシャメルソース}\troisquarts{}
Lに、(1)\protect\hyperlink{fumet-de-poisson}{魚のフュ
メ}とトリュフエッセンスを同量ずつ合わせて
\troisquarts{}量まで煮詰めたものを1\undemi{} dl加える。(2)生クリーム
1\undemi{} dlを加える。

鍋を火から外し、真っ赤に作った\protect\hyperlink{beurre-de-homard}{オマールバター}を加え、カ
イエンヌごく少量で風味を引き締める。

\ldots{}\ldots{}魚料理用。

\maeaki

\hypertarget{ux30deux30c3ux30b7ux30e5ux30ebux30fcux30e0ux5165ux308aux30bdux30fcux30b9}{%
\subsubsection{マッシュルーム入りソース}\label{ux30deux30c3ux30b7ux30e5ux30ebux30fcux30e0ux5165ux308aux30bdux30fcux30b9}}

\hypertarget{sauce-aux-champignons}{%
\paragraph{Sauce aux Champignons}\label{sauce-aux-champignons}}

\index{そーす@ソース!まっしゅるーむしろ@マッシュルーム---(ホワイト系)}
\index{まっしゅるーむ@マッシュルーム!そーすしろ@---ソース(ホワイト系)}
\index{sauce@sauce!champignonsblanche@--- aux Champignons (blanches)}
\index{champignon@champignon!sauce blanche@Sauce aux Champignons (blanche)}

マッシュルームを茹でた汁3
dlを\untiers{}量まで煮詰める。\protect\hyperlink{sauce-allemande}{ソース・アルマン
ド}\footnote{エスコフィエはドイツ嫌いであったために、「ドイツ風」の意味であ
  る「ソース・アルマンド」の名称を嫌い、原書においては\ruby{頑}{かた
  くな}に「パリ風ソース」としている。\protect\hyperlink{sauce-allemande}{パリ風ソース(ソース・
  アルマンド)}原注参照。}\troisquarts{} Lを加え、数分間沸騰させる。あ
らかじめ\ruby{螺旋}{らせん}状に刻みを入れて整形\footnote{tourner
  トゥルネ。原義は「回す」。包丁を動かさずに材料の方を回
  すようにして切る、刻み目を入れることがこの用語の由来。マッシュルー
  ムの場合はその際に大量の切りくずが発生するので、それをソースなどの
  風味付けに利用することも多い。}してから茹でておいた真っ
白で小さなマッシュルーム100 gを加えて仕上げる。

\ldots{}\ldots{}鶏料理用。魚料理に添えることもある。魚料理に合わせる場合は、ソース・
アルマンドではなく\protect\hyperlink{veloute-de-poisson}{魚料理用ヴルテ}を用いること。

\maeaki

\hypertarget{ux30bdux30fcux30b9ux30b7ux30e3ux30f3ux30c6ux30a3ux30a422}{%
\subsubsection[ソース・シャンティイ]{\texorpdfstring{ソース・シャンティイ\footnote{料理においては生クリームをホイップしたクレーム・シャンティイが
  有名だが、元来は、パリ北方に位置する町の名。17世紀、コンデ公ルイ2
  世(大コンデとも呼ばれる)の城館があり、ヴァテル Vatel
  (Watel)(1635〜1671)がメートルドテルとして仕えていた。その館でル
  イ14世をはじめとする約千名もの賓客を招いて開かれた数日にわたる宴会
  の際に、食材の魚が少ししか届かないと誤解したヴァテルは責任をとるた
  めに自殺したと言われている。なお、魚はその後すぐに大量に館に届けら
  れたという。ヴァテルという人物についての記録は少ないが、この逸話は
  非常に有名で、2000年にジェラール・ドパルデュー主演で映画化された。}}{ソース・シャンティイ}}\label{ux30bdux30fcux30b9ux30b7ux30e3ux30f3ux30c6ux30a3ux30a422}}

\hypertarget{sauce-chantilly-chaude}{%
\paragraph{Sauce Chantilly}\label{sauce-chantilly-chaude}}

\index{そーす@ソース!しやんていい@---・シャンティイ}
\index{しやんていい@シャンティイ!そーす@ソース・---}
\index{sauce@sauce!chantilly@--- Chantilly}
\index{Chantilly@Chantilly!sauce@Sauce ---}

まれに「ソース・シャンティイ」の名で呼ばれることもあるが、これは後述の
「\protect\hyperlink{sauce-mousseline}{ソース・ムスリーヌ}」に他ならない。

\maeaki

\hypertarget{ux30bdux30fcux30b9ux30b7ux30e3ux30c8ux30fcux30d6ux30eaux30e4ux30f323}{%
\subsubsection[ソース・シャトーブリヤン]{\texorpdfstring{ソース・シャトーブリヤン\footnote{料理において通常、シャトーブリヤンは牛フィレの中心部分を3cm程度
  の厚さに切ったものを指す。この名称の由来には主に2説あり、ひとつは
  フランスロマン主義文学の父と言われる小説家フランソワ・ルネ・シャトー
  ブリヤン François René Chateaubriand (1768〜1848)の名を冠したと
  いうもの。ちなみにフランスロマン主義文学の母と呼ばれているのはス
  タール夫人Anne Louise Germaine de Staël(1766〜1817)。料理における
  シャトーブリヤンという名の由来のもうひとつの説は、ブルターニュ地
  方で畜産物の集積地であったシャトーブリヤン Châteaubriant という地
  名に由来するというもの。なお、本書の初版および第四版では
  Chateaubriandの綴り、第二版はChâteaubriantであり、第三版は
  Châteaubrian\textbf{d}という奇妙な綴りとなっている。}}{ソース・シャトーブリヤン}}\label{ux30bdux30fcux30b9ux30b7ux30e3ux30c8ux30fcux30d6ux30eaux30e4ux30f323}}

\hypertarget{sauce-chateaubriand}{%
\paragraph{Sauce Chateaubriand}\label{sauce-chateaubriand}}

\index{そーす@ソース!しゃとーふりやん@---・シャトーブリヤン}
\index{しゃとーふりやん@シャトーブリヤン! そーす@ソース・---}
\index{sauce@sauce!chateaubriand@--- Chateaubriand}
\index{chateaubriand@Chateaubriand!sauce@Sauce ---}

(仕上り5 dl分)

白ワイン4 dlに、みじん切りにしたエシャロット4個分とタイム少々、ローリ
エの葉少々、マッシュルームの切りくず40 gを加え、\untiers{}量になるまで
煮詰める。

\protect\hyperlink{jus-de-veau-brun}{仔牛のジュ}\footnote{本書では「仔牛の茶色いジュ」のレシピは掲載されているが、仔牛の
  「白い」ジュについての言及はない。ここでは通常の仔牛の茶色いジュを
  用いればいい。また、\protect\hyperlink{sauce-colbert}{ソース・コルベール}の項(第二
  版で加えられた)で、\href{}{ブール・コルベール}とこのソースを比較するに
  あたり、このソースを「軽く仕上げたグラスドヴィアンドにバターとパセ
  リのみじん切りを加えたもの」と述べている(\protect\hyperlink{sauce-colbert}{ソース・コルベー
  ル}本文参照)。このため、なぜこのソース・シャトー
  ブリヤンが「ブラウン系の派生ソース」の節ではなく「ホワイト系の派生
  ソース」に分類されているのか疑問が残るところ。}4
dlを加え、半量になるまで煮詰める。
布で漉し、鍋を火から外して、メートルドテルバター250 gと細かく刻んだエ
ストラゴン小さじ\undemi{}杯を加えて仕上げる。

\ldots{}\ldots{}牛、羊の赤身肉のグリル用。

\maeaki

\hypertarget{ux767dux3044ux30bdux30fcux30b9ux30b7ux30e7ux30d5ux30edux30efux6a19ux6e96}{%
\subsubsection{白いソース・ショフロワ(標準)}\label{ux767dux3044ux30bdux30fcux30b9ux30b7ux30e7ux30d5ux30edux30efux6a19ux6e96}}

\hypertarget{sauce-chaud-froid-blanche-ordinaire}{%
\paragraph{Sauce Chaud-froid blanche
ordinaire}\label{sauce-chaud-froid-blanche-ordinaire}}

\index{そーす@ソース!しよふろわしろ@白い---・ショフロワ(標準)}
\index{しよふろわ@ショフロワ!そーすしろ@白いソース---(標準)}
\index{sauce@sauce!chaud-froid blanche ordinaire@--- Chaud-froid blanche ordinaire}
\index{chaud-froid@chaud-froid!sauce blanche ordinaire@Sauce --- blanche ordinaire}

(仕上り1
L分)\ldots{}\ldots{}\protect\hyperlink{veloute}{標準的なヴルテ}\troisquarts{}
L、\protect\hyperlink{gelee-de-volaille}{鶏でとっ た白いジュレ}6〜7
dl、生クリーム\footnote{フランスの生クリームについては\protect\hyperlink{sauce-supreme}{ソース・シュプレー
  ム}訳注参照。}3 dl。

厚手のソテー鍋にヴルテを入れる。強火にかけ、ヘラで混ぜながらジュレと用
意した生クリーム\untiers{}量を少しずつ加えていく。

所定の分量にするには、\deuxtiers{}量くらいまで煮詰めることになる。

味見をして、固さを確認する。これを布で漉す\footnote{粘度の高いソースなどを布で漉す方法については、\protect\hyperlink{veloute}{ヴルテ}訳
  注参照。}。生クリームの残りを少
しずつ加え、ゆっくり混ぜながら、ショフロワに仕立てる食材を覆うのにいい
固さになるまで冷ましてやる。

\maeaki

\hypertarget{ux30d6ux30edux30f3ux30c9ux306eux30bdux30fcux30b9ux30b7ux30e7ux30d5ux30edux30ef}{%
\subsubsection{ブロンドのソース・ショフロワ}\label{ux30d6ux30edux30f3ux30c9ux306eux30bdux30fcux30b9ux30b7ux30e7ux30d5ux30edux30ef}}

\hypertarget{sauce-chaud-froid-blonde}{%
\paragraph{Sauce Chaud-froid blonde}\label{sauce-chaud-froid-blonde}}

\index{そーす@ソース!しよふろわふろんと@ブロンドの---・ショーフロワ}
\index{しよふろわ@ショーフロワ!そーす(きつねいろ)@ブロンドのソース---}
\index{sauce@sauce!chaud-froid blonde@--- Chaud-froid blonde}
\index{chaud-froid@chaud-froid!sauce blonde@Sauce --- blonde}

上記と同様に作るが、ヴルテではなく\protect\hyperlink{sauce-allemande}{ソース・アルマン
ド}を用いる。また、生クリームの量は半分に減らすこと。

\maeaki

\hypertarget{ux30bdux30fcux30b9ux30b7ux30e7ux30d5ux30edux30efux30aaux30fcux30edux30fcux30eb28}{%
\subsubsection[ソース・ショフロワ・オーロール]{\texorpdfstring{ソース・ショフロワ・オーロール\footnote{夜明け、曙光の意。}}{ソース・ショフロワ・オーロール}}\label{ux30bdux30fcux30b9ux30b7ux30e7ux30d5ux30edux30efux30aaux30fcux30edux30fcux30eb28}}

\hypertarget{sauce-chaud-froid-aurore}{%
\paragraph{Sauce Chaud-froid Aurore}\label{sauce-chaud-froid-aurore}}

\index{そーす@ソース!しよふろわおーろーる@---・ショーフロワ・オーロール}
\index{しよふろわ@ショーフロワ!そーすおーろーる@ソース・---・オーロール}
\index{おーろーる@オーロール!そーすしよふろわおーろーる@ソース・ショーフロワ・---}
\index{sauce@sauce!chaud-froid aurore@--- Chaud-froid Aurore}
\index{chaud-froid@chaud-froid!sauce aurore@Sauce --- Aurore}
\index{aurore@aurore!sauce chaud-froid aurore@Sauce Chaud-froid ---}

標準的な\protect\hyperlink{sauce-chaud-froid-blanche-ordinaire}{白いソース・ショフロワ}
を上記のとおり作る。そこに、真っ赤なトマトピュレを布で漉したもの
1\undemi{} dlとパプリカ粉末0.25 gを少量のコンソメで煎じた\footnote{infuser
  アンフュゼ。煮出す、煎じる、の意。}ものを加 える。

\ldots{}\ldots{}鶏のショフロワ用。

\hypertarget{ux539fux6ce8-3}{%
\subparagraph{【原注】}\label{ux539fux6ce8-3}}

あまり鮮かな色にしたくない場合は、パプリカを煎じた汁は数滴だけ加えるに
とどめるといい。

\maeaki

\hypertarget{ux30bdux30fcux30b9ux30b7ux30e7ux30d5ux30edux30efux30f4ux30a7ux30fcux30ebux30d7ux30ec}{%
\subsubsection{ソース・ショフロワ・ヴェールプレ}\label{ux30bdux30fcux30b9ux30b7ux30e7ux30d5ux30edux30efux30f4ux30a7ux30fcux30ebux30d7ux30ec}}

\hypertarget{sauce-choud-froid-vert-pre}{%
\paragraph[Sauce Chaud-froid au Vert-pré]{\texorpdfstring{Sauce
Chaud-froid au Vert-pré\footnote{緑の野原、草原、の意。}}{Sauce Chaud-froid au Vert-pré}}\label{sauce-choud-froid-vert-pre}}

\index{そーす@ソース!しよふろわうえーるふれ@---・ショーフロワ・ヴェールプレ}
\index{しよふろわ@ショーフロワ!そーすうえーるふれ@ソース・---・ヴェールプレ}
\index{うえーるふれ@ヴェールプレ!そーすしよふろわうえーるふれ@ソース・ショーフロワ・---}
\index{sauce@sauce!chaud-froid vert-pre@--- Chaud-froid au Vert-pré}
\index{chaud-froid@chaud-froid!sauce vert-pre@Sauce --- au Vert-pré}
\index{vert-pre@vert-pré!sauce chaud-froid vert-pre@Sauce Chaud-froid au ---}

鍋に白ワイン2 dlを沸かし、セルフイユとエストラゴン、刻んだシブレット、
刻んだパセリの葉を各1つまみずつ投入する。蓋をして火から外し、10分間煎
じてから布で漉す。

最初に示したとおりの分量で\protect\hyperlink{sauce-chaud-froid-blanche-ordinaire}{標準的なソース・ショフロ
ワ}を作り、煮詰めながら、上記の
香草を煎じた液体を少しずつ混ぜ込む。この段階で1 Lになるまで煮詰めてお
くこと。

\protect\hyperlink{}{ほうれんそうから採った緑の色素}をソースに加え、\textbf{ほんのり薄い緑色}にする。

この色素を加える際にはよく注意して、上で示したとおりの色合いになるよう少しずつ投入すること。

このソースは各種の鶏\footnote{日本語では鶏と一言で済ませるが、フランス語では
  poussin プサン (ひよこ、ひな鶏)、poulette
  プレット(若い雌鶏)、poulet プレ(若 鶏)、poule
  プール(雌鶏)、poulet de grain プレドグラン(50〜70日
  の若鶏)、poulet reine プレレーヌ(若鶏と肥鶏の中間のサイズでソテー
  やローストにする)、poulet quatre quarts プレカトルカール(45日程
  で食用にする)、poularde プラルド(肥鶏、1.8kg以上のものが多く、
  AOCを取得している産地もある)、chapon シャポン(去勢鶏、最大で6kg
  程になるというが、肉質は雌鶏に近く、高級品とされている)、coq コッ
  ク(雄鶏)などに細かく分類されている。}のショフロワ、とりわけ「\protect\hyperlink{}{ショフロワ・プランタニエ}」に用いる。

\maeaki

\hypertarget{ux9b5aux6599ux7406ux7528ux30bdux30fcux30b9ux30b7ux30e7ux30d5ux30edux30ef}{%
\subsubsection{魚料理用ソース・ショフロワ}\label{ux9b5aux6599ux7406ux7528ux30bdux30fcux30b9ux30b7ux30e7ux30d5ux30edux30ef}}

\hypertarget{sauce-chaud-froid-maigre}{%
\paragraph{Sauce Chaud-froid maigre}\label{sauce-chaud-froid-maigre}}

作り方の手順と分量は\protect\hyperlink{sauce-chaud-froid-blanche-ordinaire}{標準的なソース・ショフロ
ワ}とまったく同じだが、以下
の点を変更する。(1)通常の\protect\hyperlink{veloute}{ヴルテ}ではなく\protect\hyperlink{veloute-de-poisson}{魚料理用ヴル
テ}を用いる。(2)\protect\hyperlink{}{鶏のジュレ}ではなく\protect\hyperlink{}{白
い魚のジュレ}を用いること。

\hypertarget{ux539fux6ce8-4}{%
\subparagraph{【原注】}\label{ux539fux6ce8-4}}

一般的に、このソースは魚のフィレやエスカロップ、甲殻類に\protect\hyperlink{}{マヨネーズコ
レ}の代わりとして用いることをお勧めする。マヨネーズコレはいろいろ不
都合な点があり、そのうちの最大のものは、ゼラチンが溶けるにつれて油が浸
み出してきてしまうことだ。こういう不都合はこの魚料理用ソース・ショフロ
ワを使う場合には出てこない。このソースは風味も明確ですっきりしているか
らマヨネーズコレよりも好ましいだろう。

\maeaki

\hypertarget{ux30bdux30fcux30b9ux30b7ux30f4ux30ea33}{%
\subsubsection[ソース・シヴリ]{\texorpdfstring{ソース・シヴリ\footnote{19世紀フランスの作家フレデリック・スリエ
  Frédéric Soulié (1800〜
  1847)の劇『ディアーヌ・ド・シヴリ』\emph{Diane de Chivry}
  (1838年)ある
  いは1897年に新聞「フィガロ」に掲載されたエルネスト・カペンデュの小
  説あ『ビビタパン』の登場人物名Chivryにちなんだか、あるいはまったく別の人物の
  名を冠したものかは不明。}}{ソース・シヴリ}}\label{ux30bdux30fcux30b9ux30b7ux30f4ux30ea33}}

\hypertarget{sacue-chivry}{%
\paragraph{Sauce Chivry}\label{sacue-chivry}}

白ワイン1\undemi{} dlに以下を各1つまみずつ投入する\footnote{明記されていないが、この時点で白ワインは沸かしておく。}\ldots{}\ldots{}セルフイユ、
パセリ、エストラゴン、シブレット、時季が合えばサラダバーネット\footnote{pumprenelle
  パンプルネル、和名ワレモコウ。}の
若い葉。蓋をして鍋を火から外し、10分間煎じる\footnote{infuser
  アンフュゼ。}。布で絞るようにして 漉す。

こうしてハーブ類を煎じた液体を、あらかじめ沸かしておいた\protect\hyperlink{veloute}{ヴル
テ}\troisquarts{}
Lに加える。火から外し、\protect\hyperlink{beurre-a-la-chivry}{ブール・シヴ
リ}100
を加えて仕上げる(\protect\hyperlink{beurres-composes}{合わせバターの
節}参照)。

\ldots{}\ldots{}ポシェ\footnote{pocher
  原則的には、沸騰しない程度の温度で加熱調理すること。この
  場合は、下処理した鶏一羽まるごとをぎりぎり入るくらいの大きさの鍋に
  入れて水あるいはクールブイヨンを用いてゆっくり火を通す調理を意味し
  ている(温度管理が難しい場合はオーブンを用いることもある)。}あるいは茹でた鶏の料理用。

\hypertarget{ux539fux6ce8-5}{%
\subparagraph{【原注】}\label{ux539fux6ce8-5}}

サラダバーネットは生育するにつれて苦味が強くなるの、必ず若いものを使うこと。

\maeaki

\hypertarget{ux30bdux30fcux30b9ux30b7ux30e7ux30edux30f3}{%
\subsubsection{ソース・ショロン}\label{ux30bdux30fcux30b9ux30b7ux30e7ux30edux30f3}}

\hypertarget{sauce-choron}{%
\paragraph{Sauce Choron}\label{sauce-choron}}

\protect\hyperlink{sauce-bearnaise-tomatee}{トマト入りソース・ベアルネーズ}参照。

\maeaki

\hypertarget{ux30bdux30fcux30b9ux30afux30ecux30fcux30e0}{%
\subsubsection{ソース・クレーム}\label{ux30bdux30fcux30b9ux30afux30ecux30fcux30e0}}

\hypertarget{sauce-creme}{%
\paragraph{Sauce à la Crème}\label{sauce-creme}}

\index{そーす@ソース!くれーむ@---・クレーム}
\index{くりーむ@クリーム!そーす@ソース・クレーム}
\index{sauce@sauce!creme@--- à la Crème}
\index{creme@crème!sauce@Sauce à la ---}

\protect\hyperlink{sauce-bechamel}{ベシャメルソース}1 Lに生クリーム2
dlを加えて、ヘラで
混ぜながら強火で、全体量の\troisquarts{}になるまで煮詰める。

布で漉す\footnote{粘度や濃度の高いソースを漉す方法については\protect\hyperlink{veloute}{ヴルテ}訳注参照。}。フレッシュなクレーム・ドゥーブル\footnote{乳酸醗酵させた濃度の高い生クリーム。詳しくは\protect\hyperlink{sauce-supreme}{ソース・シュプレーム}訳注参照。}2\undemi{}
dlとレモン果汁半個分を少しずつ加えて仕上げる。

\ldots{}\ldots{}茹でた魚、野菜料理、鶏、卵料理用。

\maeaki

\hypertarget{ux30bdux30fcux30b9ux30afux30ebux30f4ux30a7ux30c3ux30c840}{%
\subsubsection[ソース・クルヴェット]{\texorpdfstring{ソース・クルヴェット\footnote{小海老のこと。フランスでよく料理に用いられるのは生の状態で甲殻
  が灰色がかった小さめのcrevettes grisesクルヴェット・グリーズと、や
  や大きめでピンク色のcrevettes rosesクルヴェット・ローズ。美味しい。
  ちなみに日本でよく食べられているブラックタイガーはフランス語にする
  とcrevette géante tigréeと言う。}}{ソース・クルヴェット}}\label{ux30bdux30fcux30b9ux30afux30ebux30f4ux30a7ux30c3ux30c840}}

\hypertarget{sauce-aux-crevettes}{%
\paragraph{Sauce aux Crevettes}\label{sauce-aux-crevettes}}

\index{そーす@ソース!くるうえつと@---・クルヴェット}
\index{くるうえつと@クルヴェット!そーす@ソース・---}
\index{sauce@sauce!crevette@--- aux Crevettes}
\index{crevette@crevette!sauce@Sauce aux Crevettes}

\protect\hyperlink{veloute-de-poisson}{魚料理用ヴルテ}または\protect\hyperlink{sauce-bechamel}{ベシャメルソー
ス}1 Lに、生クリーム1\undemi{}
dlと\protect\hyperlink{fumet-de-poisson}{魚のフュ メ}1\undemi{}
dlを加える。

火にかけて9
dlになるまで煮詰める。鍋を火から外し、\protect\hyperlink{}{ブール・ルー
ジュ}25 g(ソース全体に淡いピンクの色合いを付けるのが目的)を足した
\protect\hyperlink{}{クルヴェットバター}100gを加える。殻を剥いたクルヴェットの尾の身大
さじ3杯を加え、カイエンヌ1つまみで風味を引き締めて仕上げる。

\ldots{}\ldots{}魚料理およびある種の卵料理用。

\maeaki

\hypertarget{ux30abux30ecux30fcux30bdux30fcux30b9}{%
\subsubsection{カレーソース}\label{ux30abux30ecux30fcux30bdux30fcux30b9}}

\hypertarget{sauce-currie}{%
\paragraph{Sauce Currie}\label{sauce-currie}}

\index{そーす@ソース!かれー@カレー---}
\index{かれー@カレー!そーす@---ソース}
\index{sauce@sauce!currie@---  Currie}
\index{currie@currie!sauce@Sauce ---}

以下の材料をバターで軽く色付くまで炒める\ldots{}\ldots{}玉ねぎ250
g、セロリ100 g、 パセリの根\footnote{パセリには根パセリpersil
  tubéreuxといって根が肥大する品種系統も
  ある。平葉で、葉の香りはフランスで一般的なモスカールドタイプ(葉の
  縮れるタイプ)とやや異なる。イタリアンパセリのように用いることが可 能。}30
g、これらはすべてやや厚めにスライスする。タイム1枝と
ローリエの葉少々、メース少々を加える。小麦粉50gとカレー粉\footnote{カレーは植民地インドの料理としてイギリスに伝わり、18世紀にはC\&B
  社によって混合スパイスであるカレー粉が開発された。フランスはあまり
  インドやその他のカレーの食文化と接することもなかったために、こんに
  ちでも「珍しい料理」の範疇にとどまっている。とはいえ、19世紀にイン
  ドからアンティル諸島のうちの英領地域に連れて来られたインド人たちが
  カレーを伝え、それが広まってフランス領アンティーユにおいてコロンボ
  colomboというカレーのバリエーションが成立した。コロンボはこんにち
  のフランスでも(インドのカレーとは別のものとして)比較的よく知られ
  たものとなっている(少なくともcurry, currieという語よりは一般的認
  知度が高いと言えるだろう)。}小さじ1
杯弱を振り入れる。小麦粉が色付かない程度に炒めて火を通したら、\protect\hyperlink{}{白いコ
ンソメ} \troisquarts{} Lを注ぐ。沸騰したら、弱火にして約45分煮る。
軽く押し絞るように布で漉す。ソースを温めて、浮いてきた油脂は取り除き
\footnote{dégraisser デグレセ。}、湯煎にかけておく。

\ldots{}\ldots{}魚料理、甲殻類、鶏、さまざまな卵料理に合わせる。

\hypertarget{ux539fux6ce8-6}{%
\subparagraph{【原注】}\label{ux539fux6ce8-6}}

ココナツミルクをソースに加えることもある。その場合、白いコンソメの
\unquart{}量をココナツミルクに代えること。

\maeaki

\hypertarget{ux30a4ux30f3ux30c9ux98a8ux30abux30ecux30fcux30bdux30fcux30b9}{%
\subsubsection{インド風カレーソース}\label{ux30a4ux30f3ux30c9ux98a8ux30abux30ecux30fcux30bdux30fcux30b9}}

\hypertarget{sauce-currie-indienne}{%
\paragraph{Sauce Currie à l'Indienne}\label{sauce-currie-indienne}}

\index{そーす@ソース!いんどかれー@インドカレー---}
\index{かれー@カレー!そーすいんど@インド---ソース}
\index{sauce@sauce!currie indienne@---  Currie à l'Indienne}
\index{currie@currie!sauce indienne@Sauce --- à l'Indienne}

みじん切り\footnote{原文ciseler
  シズレ。鋭利な刃物でみじん切りにすること、スライス
  すること。原義は「ハサミで切る」。なお、日本語でみじん切りに相当す
  る用語にはhacherアシェもある(hache斧から派生した語)。後者は野菜
  の他、肉類を細かく刻む際にも用いられる。ミートチョッパーをフランス
  語ではhachoirアショワールと呼ぶ。}にした玉ねぎ1個と、パセリ、タイム、ローリエ、メース、シ
ナモン各少々のブーケガルニを、バターとともに弱火にかけて色付かないよう蒸
し煮する。

カレー粉3 gを振り入れ、ココナツミルク\undemi{} Lを注ぐ。ヴルテ \undemi{}
Lを加える(ソースを肉料理に合わせるか、魚料理に合わせるかで、
ヴルテも標準的なものを使うか、魚料理用を使うか決めること)。弱火で15分
程煮る。布で漉し、生クリーム1 dlとレモン果汁少々を加えて仕上げる。

\hypertarget{ux539fux6ce8-7}{%
\subparagraph{【原注】}\label{ux539fux6ce8-7}}

ここで示した量のココナツミルクは、生のココヤシの実700 gをおろして、
4\undemi{} dlの温めた牛乳で溶いて作る。それを布で強く絞って漉してから
使うこと。

ココナツミルクがない場合には、同量のアーモンドミルクを用いてもいい。

インドの料理人によるこのソースの作り方はさまざまで、基本だけが同じというものだ。

だが、本来のレシピがあったところで、使い物にはならないだろう。インドの
カレーは我が国の大多数にとっては我慢ならぬものだろうから。ここで記した
作り方は、ヨーロッパ人の味覚を勘案したものなので、本来のものよりいい筈
だ。

\maeaki

\hypertarget{ux30bdux30fcux30b9ux30c7ux30a3ux30d7ux30edux30deux30c3ux30c844}{%
\subsubsection[ソース・ディプロマット]{\texorpdfstring{ソース・ディプロマット\footnote{外交官風、の意。繊細で豪華な仕立ての料理に付けられる名称。}}{ソース・ディプロマット}}\label{ux30bdux30fcux30b9ux30c7ux30a3ux30d7ux30edux30deux30c3ux30c844}}

\hypertarget{sauce-diplomate}{%
\paragraph{Sauce Diplomate}\label{sauce-diplomate}}

\ruby{既}{すで}に仕上げでおいた\protect\hyperlink{sauce-normande}{ノルマンディ風ソース}1
Lに、\protect\hyperlink{}{オマールバター}75 gを加える。

さいの目に切ったオマールの尾の身大さじ2杯と同様にさいの目に切ったトリュ
フ大さじ1杯を加えて仕上げる。

\ldots{}\ldots{}大きな魚一尾まるごとの\footnote{relevé ルルヴェ。}料理用。

\maeaki

\hypertarget{ux30b9ux30b3ux30c3ux30c8ux30e9ux30f3ux30c9ux98a8ux30bdux30fcux30b9}{%
\subsubsection{スコットランド風ソース}\label{ux30b9ux30b3ux30c3ux30c8ux30e9ux30f3ux30c9ux98a8ux30bdux30fcux30b9}}

\hypertarget{sauce-ecossaise}{%
\paragraph{Sauce Ecossaise}\label{sauce-ecossaise}}

\index{そーす@ソース!すこつとらんとふう@スコットランド風---}
\index{すこつとらんとふう@スコットランド風!そーす@---ソース}
\index{sauce@sauce!ecossaise@--- Ecossaise}
\index{ecossais@écossais!sauce@Sauce Ecossaise}

上記の分量どおりに作った\protect\hyperlink{sauce-creme}{ソース・クレーム}9
dlに以下を加
えて作る。1〜2mmの細さに千切りにしたにんじん、セロリ、さやいんげんをバ
ターを加えて鍋に蓋をして弱火で蒸し煮し\footnote{étuver エチュヴェ。}、\protect\hyperlink{}{白いコンソメ}に完全に
浸したものを1dl。

\noindent\ldots{}\ldots{}卵料理、鶏料理に添える。

\maeaki

\hypertarget{ux30bdux30fcux30b9ux30a8ux30b9ux30c8ux30e9ux30b4ux30f350}{%
\subsubsection[ソース・エストラゴン]{\texorpdfstring{ソース・エストラゴン\footnote{ヨモギ科のハーブ。詳しくは茶色い派生ソースの\protect\hyperlink{sauce-chasseur}{ソース・シャスー
  ル}訳注参照。}}{ソース・エストラゴン}}\label{ux30bdux30fcux30b9ux30a8ux30b9ux30c8ux30e9ux30b4ux30f350}}

\hypertarget{sauce-estragon}{%
\paragraph{Sauce Estragon}\label{sauce-estragon}}

\index{そーす@ソース!えすとらこんしろ@---・エストラゴン(ホワイト系)}
\index{えすとらこん@エストラゴン!そーす@ソース・--- (ホワイト系)}
\index{sauce@sauce!estragon blanche@--- Estragon (blanche)}
\index{estragon@estragon!sauce blanche@Sauce --- (blanche)}

エストラゴンの枝30 gを粗く刻み\footnote{concasser コンカセ。}、強火で下茹でする\footnote{blanchir
  ブランシール。}。水気をしっ
かりときり、エストラゴンをスプーンですり潰し、あらかじめ用意しておいた
\protect\hyperlink{veloute}{ヴルテ}を大さじ4杯加える。これを布で漉す。こうして作ったエ
ストラゴンのピュレを\protect\hyperlink{veloute-de-volaille}{鶏のヴルテ}または\protect\hyperlink{veloute-de-poisson}{魚料理用
ヴルテ}1 Lに混ぜ込む。どちらのヴルテを使うから、
合わせる料理によって決めること。味を調え、みじん切りにしたエストラゴン
大さじ\undemi{}杯を加えて仕上げる。

\ldots{}\ldots{}卵料理、鶏肉料理、魚料理に合わせる。

\maeaki

\hypertarget{ux9999ux8349ux30bdux30fcux30b9}{%
\subsubsection{香草ソース}\label{ux9999ux8349ux30bdux30fcux30b9}}

\hypertarget{sauce-aux-fines-herbes-blanche}{%
\paragraph{Sauce aux Fines
Herbes}\label{sauce-aux-fines-herbes-blanche}}

\index{そーす@ソース!こうそうしろ@香草---(ホワイト系)}
\index{こうそう@香草!そーすしろ@---ソース(ホワイト系)}
\index{はーぶ@ハーブ!こうそうそーすしろ@香草ソース(ホワイト系)}
\index{sauce@sauce!fines herbes blanche@--- aux Fines Herbes (blanche)}
\index{fines herbes@fines herbes!sauce blanche@Sauce aux --- (blanche)}

(仕上り5 dl分)

あらかじめ2種のうちどちらかの方法(\protect\hyperlink{sauce-vin-blanc}{白ワインソース}
参照)で作っておいた\protect\hyperlink{sauce-vin-blanc}{白ワインソース}\undemi{}
Lに、 \protect\hyperlink{}{エシャロットバター}40
gと、パセリ、セルフイユ、エストラゴンのみじ
ん切りを大さじ1\undemi{}杯加える。

\ldots{}\ldots{}魚料理用。

\maeaki

\hypertarget{ux30bdux30fcux30b9ux30d5ux30a9ux30a4ux30e8}{%
\subsubsection{ソース・フォイヨ}\label{ux30bdux30fcux30b9ux30d5ux30a9ux30a4ux30e8}}

\hypertarget{sauce-foyot}{%
\paragraph{Sauce Foyot}\label{sauce-foyot}}

\protect\hyperlink{sauce-bearnaise-a-la-glace-de-viande}{グラスドヴィアンド入りソース・ベアルネーズ}参照。

\maeaki

\hypertarget{ux30bdux30fcux30b9ux30b0ux30edux30bcux30a4ux30e651}{%
\subsubsection[ソース・グロゼイユ]{\texorpdfstring{ソース・グロゼイユ\footnote{日本語で「すぐりの実」のことだが、こんにちでは「黒すぐり」の方
  が一般的かも知れない。黒すぐりはフランス語では cassis カシスと呼ば
  れる。一般的なグロゼイユにはフサスグリと呼ばれる groseille rouge
  グロゼイユ・ルージュ(赤すぐり)とgroseille blancheグロゼイユ・ブ
  ランシュ(白すぐり)の2種があり、どちらもブドウのように房なりする。
  上記とは別に、このソースで用いられるgroseille à maquereauグロゼイ
  ヤマクロー(maquereauは鯖の意。日本では英語経由のグーズベリーまた
  はグースベリーの名称でも呼ばれることが多い。単に西洋すぐりとも呼ぶ)
  という比較的大粒で薄く縞模様の入る種類もある。これは通常は緑色だが、ま
  れに紫色になる変種もあるという。いずれもフランスでは料理や菓子作り
  によく用いられる。}}{ソース・グロゼイユ}}\label{ux30bdux30fcux30b9ux30b0ux30edux30bcux30a4ux30e651}}

\hypertarget{sauce-groseilles}{%
\paragraph{Sauce Groseilles}\label{sauce-groseilles}}

\index{そーす@ソース!くろせいゆ@---・グロゼイユ}
\index{くろせいゆ@グロゼイユ!そーす@ソース・---}
\index{sauce@sauce!groseilles@--- Groseilles}
\index{groseille@groseille!sauce@Sauce Groseilles}

緑色の濃いグーズベリー500 gを銅の片手鍋で下茹でする。

5分間煮立てたら、水気をきって、粉砂糖大さじ3杯と白ワイン大さじ2〜3杯を
加えて、完全に火をとおす。布で漉す。

こうして出来たピュレに、\protect\hyperlink{sauce-au-beurre}{ソース・オ・ブール}5
dlを加 え、よく混ぜる。

\ldots{}\ldots{}このソースはグリルあるいはイギリス風\footnote{à
  l'anglaise
  アラングレーズ。通常は塩適量を加えた湯でボイルすることを指す。}に茹でた鯖によく合う。と
はいえ、他の魚料理にも合わせてもいい。

\hypertarget{ux539fux6ce8-8}{%
\subparagraph{【原注】}\label{ux539fux6ce8-8}}

このソースは緑色の房なりのグロゼイユ\footnote{一般的なフサスグリであれば白系統の「未熟果」を用いるということと解釈される。}でも作ることが可能。
\end{recette}