\hypertarget{ux30dbux30efux30a4ux30c8ux7cfbux306eux6d3eux751fux30bdux30fcux30b9}{%
\section{ホワイト系の派生ソース}\label{ux30dbux30efux30a4ux30c8ux7cfbux306eux6d3eux751fux30bdux30fcux30b9}}

\hypertarget{petites-sauces-blanches-composuxe9es-et-de-ruxe9ductions}{%
\subsection{Petites Sauces Blanches, Composées et de
Réductions}\label{petites-sauces-blanches-composuxe9es-et-de-ruxe9ductions}}
\begin{recette}
\hypertarget{ux30bdux30fcux30b9ux30a2ux30ebux30d3ux30e5ux30d5ux30a7ux30e91}{%
\subsubsection[ソース・アルビュフェラ]{\texorpdfstring{ソース・アルビュフェラ\footnote{ナポレオン軍の元帥、ルイ・ガブリエル・スーシェ
  Louis-Gabriel Suchet, duc d'Albufera
  (1770〜1826)のこと。スペイン戦役の際にそれ
  までの軍功を称えられ、ナポレオンが1812年にアルビュフェラ公爵位を新
  設して授けた。帝政期の英雄のひとりであり、アルビュフェラおよびスー
  シェの名を冠した料理がいくつかある。1814年に帝政が崩壊した後も軍務、
  政務に携わり、最終的にフランス貴族院議員の地位を得た。アルビュフェ
  ラ公爵位については、1815年7月24日の勅令においてに正式に抹消されて
  いる。このソースの特徴は赤ピーマン(パプリカ)を加熱してなめらかに
  すり潰し、バターに練り込んだものを使う点にあるが、どのような経緯で
  このソースに赤ピーマンを用いるようになったのかは不明。ただし、この
  ソースを合わせる「肥鶏 アルビュフェラ」は詰め物(ファルス)に米を
  用いるが、アルビュフェラは湖の周辺の湿地帯で米の生産がおこなわれて
  いるという点では一応の関連性が認められよう。なお、アルビュフェラは
  バレンシアの湖とそこに形成された潟であり、現在はバレンシア州のアル
  ブフェーラ自然公園となっている。}}{ソース・アルビュフェラ}}\label{ux30bdux30fcux30b9ux30a2ux30ebux30d3ux30e5ux30d5ux30a7ux30e91}}

\hypertarget{sauce-albufera}{%
\paragraph{Sauce Albuféra}\label{sauce-albufera}}

\index{そーす@ソース!あるひゆふえら@---・アルビュフェラ}
\index{あるひゆふえら@アルビュフェラ!そーす@ソース・---}
\index{sauce@sauce!albufera@--- Albuféra}
\index{albufera@Albuféra!sauce@Sauce ---}

\protect\hyperlink{sauce-supreme}{ソース・シュプレーム}1
Lあたりに、溶かしたブロンド色
の\protect\hyperlink{glace-de-viande}{グラスドヴィアンド}2
dlと、標準的な分量比率で作っ た\href{}{赤ピーマンバター}50 gを加える。

\maeaki

\hypertarget{ux30bdux30fcux30b9ux30a2ux30e1ux30eaux30b1ux30fcux30cc3}{%
\subsubsection[ソース・アメリケーヌ]{\texorpdfstring{ソース・アメリケーヌ\footnote{アメリケーヌという名称の由来は諸説あるが、19世紀フランスの料理人
  ピエール・フレス Pierre Fraysse がアメリカで働いた後にパリで1853年
  に開いたレストラン「シェ・ピーターズ」でこの料理名で提供したという
  のが定説。ただし、1853年以前にレストラン「ボヌフォワ」に「ラングドッ
  ク産オマール ソース・アメリケーヌ添え」というメニューあり、フレス
  はその料理に改変を加えたか、名前だけをシンプルに「アメリケーヌ」と
  した程度という説もある。かつては、オマールの主産地のひとつブルター
  ニュ地方を意味する古い形容詞 armoricain(e) アルモリカン、アルモリ
  ケーヌの音が変化した料理名だと主張されることもあったが、19世紀には
  南仏産が中心であったトマトを用いる点で矛盾が生じてしまう。いずれに
  しても、この料理名がフレスの店シェ・ピーターズを基点として広く知ら
  れるようになったことは事実と考えていい。}}{ソース・アメリケーヌ}}\label{ux30bdux30fcux30b9ux30a2ux30e1ux30eaux30b1ux30fcux30cc3}}

\hypertarget{sauce-americaine}{%
\paragraph{Sauce Américaine}\label{sauce-americaine}}

\index{そーす@ソース!あめりけーぬ@---・アメリケーヌ}
\index{あめりふう@アメリカ風!そーす@ソース・アメリケーヌ}
\index{sauce@sauce!americaine@--- Américaine}
\index{americain@américain!sauce americaine@Sauce Américaine}

このソースは\protect\hyperlink{homard-a-l-americaine}{オマール・アメリケーヌ}という料理
そのものと言っていい(「魚料理」の章、甲殻類、\protect\hyperlink{homard-a-l-americaine}{オマール・アメリケー
ヌ}参照)。

このソースは通常、オマール\footnote{アカザエビ科の甲殻類。加熱すると殻が真紅になることから、「海の
  枢機卿」(カトリックの枢機卿は赤い衣服を着るのが通常だった)とも呼
  ばれる。日本語では英語由来のロブスターと言うことも多い。ヨーロッパ
  オマールは一般的には300〜500 g程度のものが多いが、高級料理では800 g〜1
  kgのものが好んで用いられる。また、アメリカのオマールと異なり、
  活けの状態では甲殻が青みがかった黒褐色のものがしばしば存在し、 homard
  bleuオマールブルーといって珍重される。ちなみに日本の伊勢エ
  ビはフランス語のLangousteラングーストに近いもので、大きさ、色など
  にあまり違いは認められない。}の身をガルニチュールとした魚料理に添えられる。
オマールの身をやや斜めになるよう厚さ1 cm程度の輪切りにし\footnote{escalopper
  エスカロペ。エスカロップに切る。ここで使用するオマー
  ルは900g〜1kg程度のものを想定していることに注意。}、魚料理の
ガルニチュールとして供するわけだ。

\maeaki

\hypertarget{ux30a2ux30f3ux30c1ux30e7ux30d3ux30bdux30fcux30b9}{%
\subsubsection{アンチョビソース}\label{ux30a2ux30f3ux30c1ux30e7ux30d3ux30bdux30fcux30b9}}

\hypertarget{sauce-anchois}{%
\paragraph{Sauce Anchois}\label{sauce-anchois}}

\index{そーす@ソース!あんちょうい@アンチョビ---}
\index{あんちょひ@アンチョビ!そーす@---ソース}
\index{sauce@sauce!anchois@--- Anchois}
\index{anchois@anchois!sauce anchois@Sauce ---}

\href{}{ノルマンディー風ソース}8
dlを、バターを加える前の段階まで作る。\href{}{ア ンチョビバター}125
gを混ぜ込む。アンチョビのフィレ50 gを洗い、よく水
気を絞ってから小さなさいの目に切ったのを加えて仕上げる。

\ldots{}\ldots{}魚料理用。

\maeaki

\hypertarget{ux30bdux30fcux30b9ux30aaux30fcux30edux30fcux30eb4}{%
\subsubsection[ソース・オーロール]{\texorpdfstring{ソース・オーロール\footnote{夜明けの光、曙光のこと。オーロラの意味もあるため、日本では「オー
  ロラソース」と呼ばれることもあるが、マヨネーズとトマトケチャップを
  同量で混ぜ合わせたものもそう呼ばれることが多いので注意。}}{ソース・オーロール}}\label{ux30bdux30fcux30b9ux30aaux30fcux30edux30fcux30eb4}}

\hypertarget{sauce-aurore}{%
\paragraph{Sauce Aurore}\label{sauce-aurore}}

\index{そーす@ソース!おーろーる@---・オーロール}
\index{おーろーる@オーロール!そーす@ソース・---}
\index{sauce@sauce!aurore@--- Aurore}
\index{aurore@aurore!sauce@Sauce ---}

\protect\hyperlink{veloute}{ヴルテ}に真っ赤なトマトピュレを加えたもの。分量は、ヴルテが\troisquarts{}に対し、トマトピュレ\unquart{}とする。仕上げに、ソース1
Lあたり100 gのバターを加える。

\ldots{}\ldots{}卵料理、仔牛、仔羊肉の料理、鶏料理用。

\maeaki

\hypertarget{ux9b5aux6599ux7406ux7528ux30bdux30fcux30b9ux30aaux30fcux30edux30fcux30eb}{%
\subsubsection{魚料理用ソース・オーロール}\label{ux9b5aux6599ux7406ux7528ux30bdux30fcux30b9ux30aaux30fcux30edux30fcux30eb}}

\hypertarget{sauce-aurore-maigre}{%
\paragraph{Sauce Aurore maigre}\label{sauce-aurore-maigre}}

\index{そーす@ソース!おーろーるさかなよう@魚料理用---・オーロール}
\index{おーろーる@オーロール!そーすさかな@魚料理用ソース・---}
\index{sauce@sauce!aurore maigre@--- Aurore maigre}
\index{aurore@aurore!sauce maigre@Sauce --- maigre}

\protect\hyperlink{veloute-de-poisson}{魚料理用ヴルテ}に、上記と同じ割合でトマトピュレ
を加える。ソース1 Lあたりバター125 gを加えて仕上げる。

\ldots{}\ldots{}魚料理用

\maeaki

\hypertarget{ux30d0ux30a4ux30a8ux30ebux30f3ux98a8ux30bdux30fcux30b9}{%
\subsubsection{バイエルン風ソース}\label{ux30d0ux30a4ux30a8ux30ebux30f3ux98a8ux30bdux30fcux30b9}}

\hypertarget{sauce-bavaroise}{%
\paragraph{Sauce Bavaroise}\label{sauce-bavaroise}}

\index{そーす@ソース!はいえるんふう@バイエルン風---}
\index{はいえるんふう@バイエルン風!そーす@---ソース}
\index{sauce@sauce!bavarois@--- Bavaroise}
\index{bavarois@bavarois!sauce bavaroise@Sauce Bavaroise}

ヴィネガー5 dlにタイムとローリエの葉少々とパセリの枝4本、大粒のこしょ
う7〜8個と、おろした\footnote{原文 râpé \textless{} râpe
  ラープと呼ばれる器具を用いておろすが、日本のお
  ろし金と目の大きさが違うので注意。多くの場合、マンドリーヌ mandrine
  と呼ばれる野菜用スライサーにこの機能が付属している。}レフォール\footnote{raifort
  西洋わさび、ホースラディッシュ。}大さじ2杯を加え、半量になるまで
煮詰める。

この煮詰めた汁に卵黄6個を加え\footnote{卵黄を加える前に一度漉しておいたほうがいいだろう。}、\protect\hyperlink{sauce-hollandaise}{オランデーズソー
ス}を作る要領で、バター400 gと大さじ1\undemi{}杯の
水を少しずつ加えながら、ソースがしっかり乳化するまで混ぜていく。布で漉
す。

\protect\hyperlink{beurre-d-ecrevisse}{エクルヴィスバター}100
gと泡立てた生クリーム大さ じ2杯、さいの目に切ったエクルヴィス\footnote{ざりがにのこと。通常はヨーロッパザリガニécrevisse
  à pattes
  rougesエクルヴィスアパットルージュを指す。高級食材としてとても好ま
  れている。現在は代用としてécrevisse de Californieエクルヴィスドカ
  リフォルニ(ウチダザリガニ)が用いられることもある。日本在来のニホ
  ンザリガニや、外来種だが多く生息しているアメリカザリガニは通常、フ
  ランス料理には用いられない。いずれもジストマ(寄生虫)のリスクがあ
  るため、生食は厳禁。}の尾の身を加えて仕上げる。

\ldots{}\ldots{}魚料理用のこのソースは、ムースのような仕上りにすること。

\maeaki

\hypertarget{ux30bdux30fcux30b9ux30d9ux30a2ux30ebux30cdux30fcux30ba8}{%
\subsubsection[ソース・ベアルネーズ]{\texorpdfstring{ソース・ベアルネーズ\footnote{ベアルヌは旧地方名で、フランス南西部、現在のピレネー・アトラン
  ティック県のことを指すが、このソースはその地方とまったく関係がない。
  19世紀パリ郊外のレストラン「パヴィヨン・アンリIV」が店名に掲げてい
  るアンリ四世がベアルヌのポー生まれであることにちなんで命名したソー
  ス名というのが定説。}}{ソース・ベアルネーズ}}\label{ux30bdux30fcux30b9ux30d9ux30a2ux30ebux30cdux30fcux30ba8}}

\hypertarget{sauce-bearnaise}{%
\paragraph{Sauce Béarnaise}\label{sauce-bearnaise}}

\index{そーす@ソース!へあるねーす@---・ベアルネーズ}
\index{へあるぬふう@ベアルヌ風!そーす@ソース・---}
\index{へあるねーす@ベアルネーズ!そーす@ソース・---}
\index{sauce@sauce!bearnaise@--- Béarnaise}
\index{bearnais@béarnais!sauce bearnaise@Sauce Béarnaise}

白ワイン2 dlとエストラゴンヴィネガー2 dlに、エシャロットのみじん切り大
さじ4杯、枝のままの粗く刻んだエストラゴン20 g、セルフイユ10 g、粗挽き
こしょう5 g、塩1つまみを加えて、\untiers{}量になるまで煮詰める。

煮詰まったら、数分間放置して温度を下げる。ここに卵黄6個を加え、弱火に
かけて、生のバター(あるいはあらかじめ溶かしておいてもいい)500 gを加
えて軽くホイップしながらなめらかになるよう混ぜる。

卵黄に徐々に火が通っていくことでソースにとろみが付くので、絶対に弱火で
作業をすること\footnote{卵黄をソースのとろみ付けに用いること自体は中世から行なわれていた。
  開放式の炉の上に鍋を鉤で吊っている場合は鍋を火から外す必要があった
  が、その後の閉鎖式かまどや、オーブンの機能も備えた fourneau フルノー
  (日本の調理現場ではストーブあるいはピアノと呼ばれることも多い)の
  場合、熱の弱い部分に鍋を置けばいいことになる。また、このソースのよ
  うにバターが中心となる場合は水よりも高温になりやすいので本文にある
  ように注意が必要だが、ブランケットのような水が中心のものに卵黄を加
  えてとろみを付ける場合は、生クリームなどでよく溶きほぐした卵黄(こ
  の時点でしっかり乳化させておくのがポイント)を、鍋全体をしっかり混
  ぜながら加える場合は比較的高温でも問題なくきれいにとろみが付く。}。

バターを混ぜ込んだら、布で漉して味を調える。カイエンヌごく少量を加えて
風味を引き締める。仕上げに、刻んだエストラゴン大さじ杯とセルフイユ大さ
じ\undemi{}杯を加える。

\ldots{}\ldots{}牛、羊肉のグリル用。

\hypertarget{ux539fux6ce8}{%
\subparagraph{【原注】}\label{ux539fux6ce8}}

このソースを熱々で提供しようとは考えないこと。このソースは要するにバター
で作ったマヨネーズなのだ。ほの温い程度で充分であり、もし熱くし過ぎてし
まうと、ソースが分離してしまう。

そうなってしまったら、冷水少々を加えて泡立て器でホイップして元のあるべ
き状態に戻してやること。

\maeaki

\hypertarget{ux30c8ux30deux30c8ux5165ux308aux30bdux30fcux30b9ux30d9ux30a2ux30ebux30cdux30fcux30ba-ux30bdux30fcux30b9ux30b7ux30e7ux30edux30f310}{%
\subsubsection[トマト入りソース・ベアルネーズ /
ソース・ショロン]{\texorpdfstring{トマト入りソース・ベアルネーズ /
ソース・ショロン\footnote{19世紀後半、パリで有名レストラン「ヴォワザン」の料理長を務めた
  アレクサンドル・ショロン Alexandre Choron (1837〜1924)。自ら考案
  し、命名したという。}}{トマト入りソース・ベアルネーズ / ソース・ショロン}}\label{ux30c8ux30deux30c8ux5165ux308aux30bdux30fcux30b9ux30d9ux30a2ux30ebux30cdux30fcux30ba-ux30bdux30fcux30b9ux30b7ux30e7ux30edux30f310}}

\hypertarget{sauce-bearnaise-tomatee}{%
\paragraph{Sauce Béarnaise tomatée, dite Sauce
Choron}\label{sauce-bearnaise-tomatee}}

\index{そーす@ソース!へあるねーすとまといり@トマト入り---・ベアルネーズ}
\index{へあるぬふう@ベアルヌ風!そーすとまといり@トマト入りソース・ベアルネーズ}
\index{そーす@ソース!しょろん@---・ショロン}
\index{しょろん@ショロン!そーす@ソース・---}
\index{sauce@sauce!bearnaise tomatee@--- Béarnaise tomatée}
\index{bearnais@b\'earnais!sauce bearnaise tomatee@Sauce Béarnaise tomatée}
\index{sauce@sauce!choron@--- Choron}
\index{choron@Choron!sauce@Sauce ---}

ソース・ベアルネーズを上記のとおりに作るが、最後にセルフイユとエストラ
ゴンのみじん切りは加えない。充分固めに作っておき、ソースの\unquart{}量
の、充分に煮詰めたトマトピュレを加える。ソースの濃度が丁度いい具合にな
るよう注意すること。

\ldots{}\ldots{}\href{}{トゥルヌド・ショロン}、および他のさまざまな料理に添える。

\maeaki

\hypertarget{ux30b0ux30e9ux30b9ux30c9ux30f4ux30a3ux30a2ux30f3ux30c9ux5165ux308aux30bdux30fcux30b9ux30d9ux30a2ux30ebux30cdux30fcux30ba-ux30bdux30fcux30b9ux30d5ux30a9ux30a4ux30e811-ux30bdux30fcux30b9ux30f4ux30a1ux30edux30ef12}{%
\subsubsection[グラスドヴィアンド入りソース・ベアルネーズ /
ソース・フォイヨ /
ソース・ヴァロワ]{\texorpdfstring{グラスドヴィアンド入りソース・ベアルネーズ
/ ソース・フォイヨ\footnote{19世紀〜20世紀初頭にパリにあったレストランおよびそのオーナーシェ
  フの名。このソースを使った「仔牛の背肉・フォイヨ」がスペシャリテだっ
  たという。} / ソース・ヴァロワ\footnote{ヴァロワ王家およびヴァロワ公爵であったルイ・フィリップ(7月王政
  期のフランス国王。在位1830〜1848)にちなんだ名称。前出のフォイヨは
  レストランを開く以前、ルイ・フィリップに仕えていた。}}{グラスドヴィアンド入りソース・ベアルネーズ / ソース・フォイヨ / ソース・ヴァロワ}}\label{ux30b0ux30e9ux30b9ux30c9ux30f4ux30a3ux30a2ux30f3ux30c9ux5165ux308aux30bdux30fcux30b9ux30d9ux30a2ux30ebux30cdux30fcux30ba-ux30bdux30fcux30b9ux30d5ux30a9ux30a4ux30e811-ux30bdux30fcux30b9ux30f4ux30a1ux30edux30ef12}}

\hypertarget{sauce-bearnaise-a-la-glace-de-viande}{%
\paragraph{Sauce Béarnaise à la glace de viande, dite Foyot, ou
Valois}\label{sauce-bearnaise-a-la-glace-de-viande}}

\index{へあるぬふう@ベアルヌ風!そーすぐらすとういあんといり@グラスドヴィアンド入りソース・ベアルネーズ}
\index{そーす@ソース!へあるねーすくらすどういあんといり@---・ベアルネーズ(グラス・ド・ヴィアンド入り)}
\index{そーす@ソース!ふおいよ@---・フォイヨ}
\index{ふおいよ@フォイヨ!そーす@ソース・---}
\index{そーす@ソース!うあろわ@---・ヴァロワ}
\index{うあろわ@ヴァロワ!そーす@ソース・---}
\index{sauce@sauce!bearnaise a la glace de viande@--- Béarnaise à la glace de viande}
\index{bearnais@b\'earnais!sauce bearnaise a la glace de viande@Sauce Béarnaise à la glace de viande}
\index{sauce@sauce!foyot@--- Foyot} \index{foyot@Foyot!sauce@Sauce ---}
\index{sauce@sauce!valois@--- Valois}
\index{valois@Valois!sauce@Sauce ---}

標準的な\protect\hyperlink{sauce-bearnaise}{ソース・ベアルネーズ}を上記の分量で、固めに作る。溶かした\protect\hyperlink{glace-de-viande}{グラスドヴィアンド}を少しずつ加えて仕上げる。

\ldots{}\ldots{}牛、羊肉のグリル用。

\maeaki

\hypertarget{ux30bdux30fcux30b9ux30d9ux30ebux30b7ux30fc13}{%
\subsubsection[ソース・ベルシー]{\texorpdfstring{ソース・ベルシー\footnote{パリ東部、セーヌ川左岸にある地名。かつては荷揚げ港があり、19世
  紀には小さなレストランが多く店を構えていたという。}}{ソース・ベルシー}}\label{ux30bdux30fcux30b9ux30d9ux30ebux30b7ux30fc13}}

\hypertarget{sauce-bercy}{%
\paragraph{Sauce Bercy}\label{sauce-bercy}}

\index{そーす@ソース!へるしー@---・ベルシー}
\index{へるしー@ベルシー!そーす@ソース・---}
\index{sauce@sauce!bercy@--- Bercy} \index{bercy@Bercy!sauce@Sauce ---}

細かくみじん切りにしたエシャロット大さじ2杯をバターでさっと色付かない
よう炒める。白ワイン2\undemi{}
dlと\protect\hyperlink{fumet-de-poisson}{魚のフュメ}か、
このソースを合わせる魚の煮汁2\undemi{} dlを注ぐ。

\deuxtiers{}量弱まで煮詰めたら、\protect\hyperlink{veloute-de-poisson}{ヴル
テ}\troisquarts{} Lを加える。ひと煮立ちさせてから、
鍋を火から外し、バター100 gとパセリのみじん切り大さじ1杯を加えて仕上げ
る。

\maeaki

\hypertarget{ux30bdux30fcux30b9ux30aaux30d6ux30fcux30eb16-ux30bdux30fcux30b9ux30d0ux30bfux30ebux30c914}{%
\subsubsection[ソース・オ・ブール /
ソース・バタルド]{\texorpdfstring{ソース・オ・ブール\footnote{本書には、日本でもかつて有名だった、エシャロットのみじん切りを
  加えたヴィネガーを煮詰めてバターを溶かし込んだ魚料理用ソース「ソー
  ス・ブールブラン」Sauce (au) Beurre blanc は収録されていない。この
  ソース・ブールブランはナント地方やアンジュー地方で淡水魚アローズや
  ブロシェに合わせる伝統的なソース。1890年頃にナント地方の女性料理人
  クレマンス・ルフーヴルが、ソース・ベアルネーズを作るつもりが誤って
  卵を加えるのを忘れてしまった結果として出来たものだとも言われている。}
/ ソース・バタルド\footnote{バタルドは「雑種の、中間の」の意。卵黄とバターだけでとろみを付
  ける\protect\hypertarget{sauce-hollandaise}{}{ソース・オランデーズ}と似てはいるが小麦粉
  も使うことからこの名が付いたと言われている。なお、パンのバタール
  bâtard も同じ語だが、細いバゲットと太いドゥーリーヴルの「中間」
  の太さとだからというのが通説。}}{ソース・オ・ブール / ソース・バタルド}}\label{ux30bdux30fcux30b9ux30aaux30d6ux30fcux30eb16-ux30bdux30fcux30b9ux30d0ux30bfux30ebux30c914}}

\hypertarget{sauce-au-beurre}{%
\paragraph{Sauce au Beurre, dite Sauce Bâtarde}\label{sauce-au-beurre}}

\index{そーす@ソース!ふーる@---・オ・ブール}
\index{はたー@バター!そーす@ソース・オ・ブール}
\index{そーす@ソース!はたると@---・バタルド}
\index{はたると@バタルド!そーす@ソース・---}
\index{sauce@sauce!beurre@--- au Beurre}
\index{beurre@beurre!sauce@Sauce au Beurre}
\index{sauce@sauce!batarde@--- Bâtarde}
\index{batard@bâtard!sauce@Sauce Bâtarde}

小麦粉45 gと溶かしバター45gをよく混ぜ合わせ粘土状にする。そこに、7 gの
塩を加えた熱湯7\undemi{} dlを一気に注ぎ、泡立て器で勢いよく混ぜ合わせ
る。とろみ付け用の卵黄5個を生クリーム大さじ1\undemi{}杯でゆるめたもの
と、レモン汁少々を加える。

布で漉し、鍋を火から外して、良質なバター300gを加えて仕上げる。

\ldots{}\ldots{}アスパラガスや、さまざまな魚のブイイ\footnote{茹でたもの、の意。料理では、シンプルに茹でた肉、魚のこと。}

\hypertarget{ux539fux6ce8-1}{%
\subparagraph{【原注】}\label{ux539fux6ce8-1}}

このソースはとろみを付けた後、湯煎にかけておき、提供直前にバターを加え
るようにするといい。

\maeaki

\hypertarget{ux30bdux30fcux30b9ux30dcux30ccux30d5ux30a9ux30ef-ux767dux30efux30a4ux30f3ux3067ux4f5cux308bux30dcux30ebux30c9ux30fcux98a8ux30bdux30fcux30b9}{%
\subsubsection{ソース・ボヌフォワ /
白ワインで作るボルドー風ソース}\label{ux30bdux30fcux30b9ux30dcux30ccux30d5ux30a9ux30ef-ux767dux30efux30a4ux30f3ux3067ux4f5cux308bux30dcux30ebux30c9ux30fcux98a8ux30bdux30fcux30b9}}

\hypertarget{sauce-bonnefoy}{%
\paragraph{Sauce Bonnefoy, ou Sauce Bordelaise au vin
blanc}\label{sauce-bonnefoy}}

\index{ほぬふおわ@ボヌフォワ!そーす@ソース・---}
\index{そーす@ソース!おぬふおわ@---・ボヌフォワ}
\index{そーす@ソース!ほるどーふうしろわいん@ボルドー風--- (白)}
\index{ほるどーふう@ボルドー風!そーす@---ソース(白)}
\index{sauce@sauce!bonnefoy@--- Bonnefoy}
\index{bonnefoy@Bonnefoy!sauce@Sauce ---}
\index{sauce@sauce!bordelaise vin blanc@--- Bordelaise au vin blanc}
\index{bordelais@bordelais!sauce vin blanc@Sauce Bordelaise au vin blanc}

ブラウン系の派生ソースの節で採り上げた、赤ワインを用いて作る\protect\hyperlink{sauce-bordelaise}{ボルドー
風ソース}とまったく同じ作り方だが、赤ワインではなく、
グラーヴかソテルヌの白ワインを用いる。また\protect\hyperlink{sauce-espagnole}{ソース・エスパニョ
ル}ではなく\protect\hyperlink{veloute}{標準的なヴルテ}を使うこと。

このソースは仕上げに、みじん切りにしたエストラゴンを加える。

\ldots{}\ldots{}魚のグリル、白身肉のグリル用。

\maeaki

\hypertarget{ux30d6ux30ebux30bfux30fcux30cbux30e5ux98a8ux30bdux30fcux30b9}{%
\subsubsection{ブルターニュ風ソース}\label{ux30d6ux30ebux30bfux30fcux30cbux30e5ux98a8ux30bdux30fcux30b9}}

\hypertarget{sauce-bretonne-blanche}{%
\paragraph{Sauce Bretonne}\label{sauce-bretonne-blanche}}

\index{そーす@ソース!ぶるたーにゅふうしろ@ブルターニュ風---(ホワイト系)}
\index{ぶるたーにゅふう@ブルターニュ風!そーすしろ@---ソース(ホワイト系)}
\index{sauce@sauce!bretonne blanche@--- Bretonne (blanche)}
\index{breton@breton!sauce blanche@Sauce Bretonne (blanche)}

長さ3〜5 cm位の、ごく細い千切り\footnote{julienne ジュリエンヌ。}にしたポワローの白い部分30
gとセロリの白い部分30 g、玉ねぎ30 g、マッシュルーム30
gをバターで完全に火が通るまで鍋に蓋をして弱火で蒸し煮する\footnote{étuver
  エチュヴェ。本来は油脂とごく少量の水分を加えて弱火で蒸し煮することだが、野菜については、バターだけを使う場合も多い。étouffer
  エトゥフェとほぼ同じ意味で用いられることも多い。}。

\protect\hyperlink{veloute-de-poisson}{魚のヴルテ}\troisquarts{}
Lを加え、しばらく弱火にかけて浮いてくる不純物を丁寧に取り除く\footnote{dépouiller
  デプイエ ≒ écumer エキュメ。}。生クリーム大さじ3杯とバター50gを加えて仕上げる。

\maeaki

\hypertarget{ux30bdux30fcux30b9ux30abux30ceux30c6ux30a3ux30a8ux30fcux30eb20}{%
\subsubsection[ソース・カノティエール]{\texorpdfstring{ソース・カノティエール\footnote{小舟の漕ぎ手、の意。}}{ソース・カノティエール}}\label{ux30bdux30fcux30b9ux30abux30ceux30c6ux30a3ux30a8ux30fcux30eb20}}

\hypertarget{sauce-canotiere}{%
\paragraph{Sauce Canotière}\label{sauce-canotiere}}

\index{そーす@ソース!かのてぃえーる@---・カノティエール}
\index{かのてぃえーる@カノティエール!そーす@ソース・---}
\index{sauce@sauce!canotiere@--- Canotière}
\index{canotiere@Canotière!sauce@Sauce ---}

淡水魚を煮るのに用いた、\href{}{白ワイン入りクールブイヨン}を\untiers{}量に
煮詰める。クールブイヨンにはしっかり香り付けしてあり塩はごく少量しか入っ
ていないこと。

1 Lあたり80 gのブールマニエを加えてとろみを付ける。軽く煮立たせたら、
鍋を火から外してバター150 gとカイエンヌごく少量を加えて仕上げる。

\ldots{}\ldots{}淡水魚のクールブイヨン煮用。

\hypertarget{ux539fux6ce8-2}{%
\subparagraph{【原注】}\label{ux539fux6ce8-2}}

バターでグラセした小玉ねぎと小ぶりのマッシュルームを加えると、「\protect\hyperlink{sauce-matelote-blanche}{白いソー
ス・マトロット}」の代用となる。

\maeaki

\hypertarget{ux30b1ux30a4ux30d1ux30fcux5165ux308aux30bdux30fcux30b9}{%
\subsubsection{ケイパー入りソース}\label{ux30b1ux30a4ux30d1ux30fcux5165ux308aux30bdux30fcux30b9}}

\hypertarget{sauce-aux-capres}{%
\paragraph{Sauce aux Câpres}\label{sauce-aux-capres}}

\index{そーす@ソース!けいぱー@ケイパー---}
\index{けいぱー@ケイパー!そーす@---ソース}
\index{sauce@sauce!capres@--- aux Câpres}
\index{capre@câpre!sauce capres@Sauce aux Câpres}

上記の\protect\hyperlink{sauce-au-beurre}{ソース・オ・ブール}に、ソース1
Lあたり大さじ4 杯のケイパーを提供直前に加える。

\ldots{}\ldots{}いろいろな種類の魚を煮た料理に用いる。

\maeaki

\hypertarget{ux30bdux30fcux30b9ux30abux30ebux30c7ux30a3ux30caux30eb21}{%
\subsubsection[ソース・カルディナル]{\texorpdfstring{ソース・カルディナル\footnote{カトリックの枢機卿(カルディナル)の衣が伝統的に赤いものである
  ことと、オマールが「海の枢機卿」と呼ばれることに由来。}}{ソース・カルディナル}}\label{ux30bdux30fcux30b9ux30abux30ebux30c7ux30a3ux30caux30eb21}}

\hypertarget{sauce-cardinal}{%
\paragraph{Sauce Cardinal}\label{sauce-cardinal}}

\index{そーす@ソース!かるでぃなる@---・カルディナル}
\index{かるでぃなる@カルディナル!そーす@ソース・---}
\index{sauce@sauce!cardinal@--- Cardinal}
\index{cardinal@cardinal!sauce@Sauce ---}

\protect\hyperlink{sauce-bechamel}{ベシャメルソース}\troisquarts{}
Lに、(1)\protect\hyperlink{fumet-de-poisson}{魚のフュ
メ}とトリュフエッセンスを同量ずつ合わせて
\troisquarts{}量まで煮詰めたものを1\undemi{} dl加える。(2)生クリーム
1\undemi{} dlを加える。

鍋を火から外し、真っ赤に作った\protect\hyperlink{beurre-de-homard}{オマールバター}を加え、カ
イエンヌごく少量で風味を引き締める。

\ldots{}\ldots{}魚料理用。

\maeaki

\hypertarget{ux30deux30c3ux30b7ux30e5ux30ebux30fcux30e0ux5165ux308aux30bdux30fcux30b9}{%
\subsubsection{マッシュルーム入りソース}\label{ux30deux30c3ux30b7ux30e5ux30ebux30fcux30e0ux5165ux308aux30bdux30fcux30b9}}

\hypertarget{sauce-aux-champignons-blanche}{%
\paragraph{Sauce aux Champignons}\label{sauce-aux-champignons-blanche}}

\index{そーす@ソース!まっしゅるーむしろ@マッシュルーム---(ホワイト系)}
\index{まっしゅるーむ@マッシュルーム!そーすしろ@---ソース(ホワイト系)}
\index{sauce@sauce!champignonsblanche@--- aux Champignons (blanches)}
\index{champignon@champignon!sauce blanche@Sauce aux Champignons (blanche)}

マッシュルームを茹でた汁3
dlを\untiers{}量まで煮詰める。\protect\hyperlink{sauce-allemande}{ソース・アルマン
ド}\footnote{エスコフィエはドイツ嫌いであったために、「ドイツ風」の意味であ
  る「ソース・アルマンド」の名称を嫌い、原書においては\ruby{頑}{かた
  くな}に「パリ風ソース」としている。\protect\hyperlink{sauce-allemande}{パリ風ソース(ソース・
  アルマンド)}原注参照。}\troisquarts{} Lを加え、数分間沸騰させる。あ
らかじめ\ruby{螺旋}{らせん}状に刻みを入れて整形\footnote{tourner
  トゥルネ。原義は「回す」。包丁を動かさずに材料の方を回
  すようにして切る、刻み目を入れることがこの用語の由来。マッシュルー
  ムの場合はその際に大量の切りくずが発生するので、それをソースなどの
  風味付けに利用することも多い。}してから茹でておいた真っ
白で小さなマッシュルーム100 gを加えて仕上げる。

\ldots{}\ldots{}鶏料理用。魚料理に添えることもある。魚料理に合わせる場合は、ソース・
アルマンドではなく\protect\hyperlink{veloute-de-poisson}{魚料理用ヴルテ}を用いること。

\maeaki

\hypertarget{ux30bdux30fcux30b9ux30b7ux30e3ux30f3ux30c6ux30a3ux30a422}{%
\subsubsection[ソース・シャンティイ]{\texorpdfstring{ソース・シャンティイ\footnote{料理においては生クリームをホイップしたクレーム・シャンティイが
  有名だが、元来は、パリ北方に位置する町の名。17世紀、コンデ公ルイ2
  世(大コンデとも呼ばれる)の城館があり、ヴァテル Vatel
  (Watel)(1635〜1671)がメートルドテルとして仕えていた。その館でル
  イ14世をはじめとする約千名もの賓客を招いて開かれた数日にわたる宴会
  の際に、食材の魚が少ししか届かないと誤解したヴァテルは責任をとるた
  めに自殺したと言われている。なお、魚はその後すぐに大量に館に届けら
  れたという。ヴァテルという人物についての記録は少ないが、この逸話は
  非常に有名で、2000年にジェラール・ドパルデュー主演で映画化された。}}{ソース・シャンティイ}}\label{ux30bdux30fcux30b9ux30b7ux30e3ux30f3ux30c6ux30a3ux30a422}}

\hypertarget{sauce-chantilly}{%
\paragraph{Sauce Chantilly}\label{sauce-chantilly}}

\index{そーす@ソース!しやんていい@---・シャンティイ}
\index{しやんていい@シャンティイ!そーす@ソース・---}
\index{sauce@sauce!chantilly@--- Chantilly}
\index{Chantilly@Chantilly!sauce@Sauce ---}

まれに「ソース・シャンティイ」の名で呼ばれることもあるが、これは後述の
「\protect\hyperlink{sauce-mousseline}{ソース・ムスリーヌ}」に他ならない。

\maeaki

\hypertarget{ux30bdux30fcux30b9ux30b7ux30e3ux30c8ux30fcux30d6ux30eaux30e4ux30f323}{%
\subsubsection[ソース・シャトーブリヤン]{\texorpdfstring{ソース・シャトーブリヤン\footnote{料理において通常、シャトーブリヤンは牛フィレの中心部分を3cm程度
  の厚さに切ったものを指す。この名称の由来には主に2説あり、ひとつは
  フランスロマン主義文学の父と言われる小説家フランソワ・ルネ・シャトー
  ブリヤン François René Chateaubriand (1768〜1848)の名を冠したと
  いうもの。ちなみにフランスロマン主義文学の母と呼ばれているのはス
  タール夫人Anne Louise Germaine de Staël(1766〜1817)。料理における
  シャトーブリヤンという名の由来のもうひとつの説は、ブルターニュ地
  方で畜産物の集積地であったシャトーブリヤン Châteaubriant という地
  名に由来するというもの。なお、本書の初版および第四版では
  Chateaubriandの綴り、第二版はChâteaubriantであり、第三版は
  Châteaubrian\textbf{d}という奇妙な綴りとなっている。}}{ソース・シャトーブリヤン}}\label{ux30bdux30fcux30b9ux30b7ux30e3ux30c8ux30fcux30d6ux30eaux30e4ux30f323}}

\hypertarget{sauce-chateaubriand}{%
\paragraph{Sauce Chateaubriand}\label{sauce-chateaubriand}}

\index{そーす@ソース!しゃとーふりやん@---・シャトーブリヤン}
\index{しゃとーふりやん@シャトーブリヤン! そーす@ソース・---}
\index{sauce@sauce!chateaubriand@--- Chateaubriand}
\index{chateaubriand@Chateaubriand!sauce@Sauce ---}

(仕上り5 dl分)

白ワイン4 dlに、みじん切りにしたエシャロット4個分とタイム少々、ローリ
エの葉少々、マッシュルームの切りくず40 gを加え、\untiers{}量になるまで
煮詰める。

\protect\hyperlink{jus-de-veau-brun}{仔牛のジュ}\footnote{本書では「仔牛の茶色いジュ」のレシピは掲載されているが、仔牛の
  「白い」ジュについての言及はない。ここでは通常の仔牛の茶色いジュを
  用いればいい。また、\protect\hyperlink{sauce-colbert}{ソース・コルベール}の項(第二
  版で加えられた)で、\href{}{ブール・コルベール}とこのソースを比較するに
  あたり、このソースを「軽く仕上げたグラスドヴィアンドにバターとパセ
  リのみじん切りを加えたもの」と述べている(\protect\hyperlink{sauce-colbert}{ソース・コルベー
  ル}本文参照)。このため、なぜこのソース・シャトー
  ブリヤンが「ブラウン系の派生ソース」の節ではなく「ホワイト系の派生
  ソース」に分類されているのか疑問が残るところ。}4
dlを加え、半量になるまで煮詰める。
布で漉し、鍋を火から外して、メートルドテルバター250 gと細かく刻んだエ
ストラゴン小さじ\undemi{}杯を加えて仕上げる。

\ldots{}\ldots{}牛、羊の赤身肉のグリル用。

\maeaki

\hypertarget{ux767dux3044ux30bdux30fcux30b9ux30b7ux30e7ux30d5ux30edux30efux6a19ux6e96}{%
\subsubsection{白いソース・ショフロワ(標準)}\label{ux767dux3044ux30bdux30fcux30b9ux30b7ux30e7ux30d5ux30edux30efux6a19ux6e96}}

\hypertarget{sauce-chaud-froid-blanche-ordinaire}{%
\paragraph{Sauce Chaud-froid blanche
ordinaire}\label{sauce-chaud-froid-blanche-ordinaire}}

\index{そーす@ソース!しよふろわしろ@白い---・ショフロワ(標準)}
\index{しよふろわ@ショフロワ!そーすしろ@白いソース---(標準)}
\index{sauce@sauce!chaud-froid blanche ordinaire@--- Chaud-froid blanche ordinaire}
\index{chaud-froid@chaud-froid!sauce blanche ordinaire@Sauce --- blanche ordinaire}

(仕上り1
L分)\ldots{}\ldots{}\protect\hyperlink{veloute}{標準的なヴルテ}\troisquarts{}
L、\protect\hyperlink{gelee-de-volaille}{鶏でとっ た白いジュレ}6〜7
dl、生クリーム\footnote{フランスの生クリームについては\protect\hyperlink{sauce-supreme}{ソース・シュプレー
  ム}訳注参照。}3 dl。

厚手のソテー鍋にヴルテを入れる。強火にかけ、ヘラで混ぜながらジュレと用
意した生クリーム\untiers{}量を少しずつ加えていく。

所定の分量にするには、\deuxtiers{}量くらいまで煮詰めることになる。

味見をして、固さを確認する。これを布で漉す\footnote{粘度の高いソースなどを布で漉す方法については、\protect\hyperlink{veloute}{ヴルテ}訳
  注参照。}。生クリームの残りを少
しずつ加え、ゆっくり混ぜながら、ショフロワに仕立てる食材を覆うのにいい
固さになるまで冷ましてやる。

\maeaki

\hypertarget{ux30d6ux30edux30f3ux30c9ux306eux30bdux30fcux30b9ux30b7ux30e7ux30d5ux30edux30ef}{%
\subsubsection{ブロンドのソース・ショフロワ}\label{ux30d6ux30edux30f3ux30c9ux306eux30bdux30fcux30b9ux30b7ux30e7ux30d5ux30edux30ef}}

\hypertarget{sauce-chaud-froid-blonde}{%
\paragraph{Sauce Chaud-froid blonde}\label{sauce-chaud-froid-blonde}}

\index{そーす@ソース!しよふろわふろんと@ブロンドの---・ショーフロワ}
\index{しよふろわ@ショーフロワ!そーす(きつねいろ)@ブロンドのソース---}
\index{sauce@sauce!chaud-froid blonde@--- Chaud-froid blonde}
\index{chaud-froid@chaud-froid!sauce blonde@Sauce --- blonde}

上記と同様に作るが、ヴルテではなく\protect\hyperlink{sauce-allemande}{ソース・アルマン
ド}を用いる。また、生クリームの量は半分に減らすこと。

\maeaki

\hypertarget{ux30bdux30fcux30b9ux30b7ux30e7ux30d5ux30edux30efux30aaux30fcux30edux30fcux30eb28}{%
\subsubsection[ソース・ショフロワ・オーロール]{\texorpdfstring{ソース・ショフロワ・オーロール\footnote{夜明け、曙光の意。}}{ソース・ショフロワ・オーロール}}\label{ux30bdux30fcux30b9ux30b7ux30e7ux30d5ux30edux30efux30aaux30fcux30edux30fcux30eb28}}

\hypertarget{sauce-chaud-froid-aurore}{%
\paragraph{Sauce Chaud-froid Aurore}\label{sauce-chaud-froid-aurore}}

\index{そーす@ソース!しよふろわおーろーる@---・ショーフロワ・オーロール}
\index{しよふろわ@ショーフロワ!そーすおーろーる@ソース・---・オーロール}
\index{おーろーる@オーロール!そーすしよふろわおーろーる@ソース・ショーフロワ・---}
\index{sauce@sauce!chaud-froid aurore@--- Chaud-froid Aurore}
\index{chaud-froid@chaud-froid!sauce aurore@Sauce --- Aurore}
\index{aurore@aurore!sauce chaud-froid aurore@Sauce Chaud-froid ---}

標準的な\protect\hyperlink{sauce-chaud-froid-blanche-ordinaire}{白いソース・ショフロワ}
を上記のとおり作る。そこに、真っ赤なトマトピュレを布で漉したもの
1\undemi{} dlとパプリカ粉末0.25 gを少量のコンソメで煎じた\footnote{infuser
  アンフュゼ。煮出す、煎じる、の意。}ものを加 える。

\ldots{}\ldots{}鶏のショフロワ用。

\hypertarget{ux539fux6ce8-3}{%
\subparagraph{【原注】}\label{ux539fux6ce8-3}}

あまり鮮かな色にしたくない場合は、パプリカを煎じた汁は数滴だけ加えるに
とどめるといい。

\maeaki

\hypertarget{ux30bdux30fcux30b9ux30b7ux30e7ux30d5ux30edux30efux30f4ux30a7ux30fcux30ebux30d7ux30ec}{%
\subsubsection{ソース・ショフロワ・ヴェールプレ}\label{ux30bdux30fcux30b9ux30b7ux30e7ux30d5ux30edux30efux30f4ux30a7ux30fcux30ebux30d7ux30ec}}

\hypertarget{sauce-choud-froid-vert-pre}{%
\paragraph[Sauce Chaud-froid au Vert-pré]{\texorpdfstring{Sauce
Chaud-froid au Vert-pré\footnote{緑の野原、草原、の意。}}{Sauce Chaud-froid au Vert-pré}}\label{sauce-choud-froid-vert-pre}}

\index{そーす@ソース!しよふろわうえーるふれ@---・ショーフロワ・ヴェールプレ}
\index{しよふろわ@ショーフロワ!そーすうえーるふれ@ソース・---・ヴェールプレ}
\index{うえーるふれ@ヴェールプレ!そーすしよふろわうえーるふれ@ソース・ショーフロワ・---}
\index{sauce@sauce!chaud-froid vert-pre@--- Chaud-froid au Vert-pré}
\index{chaud-froid@chaud-froid!sauce vert-pre@Sauce --- au Vert-pré}
\index{vert-pre@vert-pré!sauce chaud-froid vert-pre@Sauce Chaud-froid au ---}

鍋に白ワイン2 dlを沸かし、セルフイユとエストラゴン、刻んだシブレット、
刻んだパセリの葉を各1つまみずつ投入する。蓋をして火から外し、10分間煎
じてから布で漉す。

最初に示したとおりの分量で\protect\hyperlink{sauce-chaud-froid-blanche-ordinaire}{標準的なソース・ショフロ
ワ}を作り、煮詰めながら、上記の
香草を煎じた液体を少しずつ混ぜ込む。この段階で1 Lになるまで煮詰めてお
くこと。

\protect\hyperlink{}{ほうれんそうから採った緑の色素}をソースに加え、\textbf{ほんのり薄い緑色}にする。

この色素を加える際にはよく注意して、上で示したとおりの色合いになるよう少しずつ投入すること。

このソースは各種の鶏\footnote{日本語では鶏と一言で済ませるが、フランス語では
  poussin プサン (ひよこ、ひな鶏)、poulette
  プレット(若い雌鶏)、poulet プレ(若 鶏)、poule
  プール(雌鶏)、poulet de grain プレドグラン(50〜70日
  の若鶏)、poulet reine プレレーヌ(若鶏と肥鶏の中間のサイズでソテー
  やローストにする)、poulet quatre quarts プレカトルカール(45日程
  で食用にする)、poularde プラルド(肥鶏、1.8kg以上のものが多く、
  AOCを取得している産地もある)、chapon シャポン(去勢鶏、最大で6kg
  程になるというが、肉質は雌鶏に近く、高級品とされている)、coq コッ
  ク(雄鶏)などに細かく分類されている。}のショフロワ、とりわけ「\protect\hyperlink{}{ショフロワ・プランタニエ}」に用いる。

\maeaki

\hypertarget{ux9b5aux6599ux7406ux7528ux30bdux30fcux30b9ux30b7ux30e7ux30d5ux30edux30ef}{%
\subsubsection{魚料理用ソース・ショフロワ}\label{ux9b5aux6599ux7406ux7528ux30bdux30fcux30b9ux30b7ux30e7ux30d5ux30edux30ef}}

\hypertarget{sauce-chaud-froid-maigre}{%
\paragraph{Sauce Chaud-froid maigre}\label{sauce-chaud-froid-maigre}}

作り方の手順と分量は\protect\hyperlink{sauce-chaud-froid-blanche-ordinaire}{標準的なソース・ショフロ
ワ}とまったく同じだが、以下
の点を変更する。(1)通常の\protect\hyperlink{veloute}{ヴルテ}ではなく\protect\hyperlink{veloute-de-poisson}{魚料理用ヴル
テ}を用いる。(2)\protect\hyperlink{}{鶏のジュレ}ではなく\protect\hyperlink{}{白
い魚のジュレ}を用いること。

\hypertarget{ux539fux6ce8-4}{%
\subparagraph{【原注】}\label{ux539fux6ce8-4}}

一般的に、このソースは魚のフィレやエスカロップ、甲殻類に\protect\hyperlink{}{マヨネーズコ
レ}の代わりとして用いることをお勧めする。マヨネーズコレはいろいろ不
都合な点があり、そのうちの最大のものは、ゼラチンが溶けるにつれて油が浸
み出してきてしまうことだ。こういう不都合はこの魚料理用ソース・ショフロ
ワを使う場合には出てこない。このソースは風味も明確ですっきりしているか
らマヨネーズコレよりも好ましいだろう。

\maeaki

\hypertarget{ux30bdux30fcux30b9ux30b7ux30f4ux30ea33}{%
\subsubsection[ソース・シヴリ]{\texorpdfstring{ソース・シヴリ\footnote{19世紀フランスの作家フレデリック・スリエ
  Frédéric Soulié (1800〜
  1847)の劇『ディアーヌ・ド・シヴリ』\emph{Diane de Chivry}
  (1838年)ある
  いは1897年に新聞「フィガロ」に掲載されたエルネスト・カペンデュの小
  説あ『ビビタパン』の登場人物名Chivryにちなんだか、あるいはまったく別の人物の
  名を冠したものかは不明。}}{ソース・シヴリ}}\label{ux30bdux30fcux30b9ux30b7ux30f4ux30ea33}}

\hypertarget{sacue-chivry}{%
\paragraph{Sauce Chivry}\label{sacue-chivry}}

白ワイン1\undemi{} dlに以下を各1つまみずつ投入する\footnote{明記されていないが、この時点で白ワインは沸かしておく。}\ldots{}\ldots{}セルフイユ、
パセリ、エストラゴン、シブレット、時季が合えばサラダバーネット\footnote{pumprenelle
  パンプルネル、和名ワレモコウ。}の
若い葉。蓋をして鍋を火から外し、10分間煎じる\footnote{infuser
  アンフュゼ。}。布で絞るようにして 漉す。

こうしてハーブ類を煎じた液体を、あらかじめ沸かしておいた\protect\hyperlink{veloute}{ヴル
テ}\troisquarts{}
Lに加える。火から外し、\protect\hyperlink{beurre-a-la-chivry}{ブール・シヴ
リ}100
を加えて仕上げる(\protect\hyperlink{beurres-composes}{合わせバターの
節}参照)。

\ldots{}\ldots{}ポシェ\footnote{pocher
  原則的には、沸騰しない程度の温度で加熱調理すること。この
  場合は、下処理した鶏一羽まるごとをぎりぎり入るくらいの大きさの鍋に
  入れて水あるいはクールブイヨンを用いてゆっくり火を通す調理を意味し
  ている(温度管理が難しい場合はオーブンを用いることもある)。}あるいは茹でた鶏の料理用。

\hypertarget{ux539fux6ce8-5}{%
\subparagraph{【原注】}\label{ux539fux6ce8-5}}

サラダバーネットは生育するにつれて苦味が強くなるの、必ず若いものを使うこと。

\maeaki

\hypertarget{ux30bdux30fcux30b9ux30b7ux30e7ux30edux30f3}{%
\subsubsection{ソース・ショロン}\label{ux30bdux30fcux30b9ux30b7ux30e7ux30edux30f3}}

\hypertarget{sauce-choron}{%
\paragraph{Sauce Choron}\label{sauce-choron}}

\protect\hyperlink{sauce-bearnaise-tomatee}{トマト入りソース・ベアルネーズ}参照。

\maeaki

\hypertarget{ux30bdux30fcux30b9ux30afux30ecux30fcux30e0}{%
\subsubsection{ソース・クレーム}\label{ux30bdux30fcux30b9ux30afux30ecux30fcux30e0}}

\hypertarget{sauce-creme}{%
\paragraph{Sauce à la Crème}\label{sauce-creme}}

\index{そーす@ソース!くれーむ@---・クレーム}
\index{くりーむ@クリーム!そーす@ソース・クレーム}
\index{sauce@sauce!creme@--- à la Crème}
\index{creme@crème!sauce@Sauce à la ---}

\protect\hyperlink{sauce-bechamel}{ベシャメルソース}1 Lに生クリーム2
dlを加えて、ヘラで
混ぜながら強火で、全体量の\troisquarts{}になるまで煮詰める。

布で漉す\footnote{粘度や濃度の高いソースを漉す方法については\protect\hyperlink{veloute}{ヴルテ}訳注参照。}。フレッシュなクレーム・ドゥーブル\footnote{乳酸醗酵させた濃度の高い生クリーム。詳しくは\protect\hyperlink{sauce-supreme}{ソース・シュプレーム}訳注参照。}2\undemi{}
dlとレモン果汁半個分を少しずつ加えて仕上げる。

\ldots{}\ldots{}茹でた魚、野菜料理、鶏、卵料理用。

\maeaki

\hypertarget{ux30bdux30fcux30b9ux30afux30ebux30f4ux30a7ux30c3ux30c840}{%
\subsubsection[ソース・クルヴェット]{\texorpdfstring{ソース・クルヴェット\footnote{小海老のこと。フランスでよく料理に用いられるのは生の状態で甲殻
  が灰色がかった小さめのcrevettes grisesクルヴェット・グリーズと、や
  や大きめでピンク色のcrevettes rosesクルヴェット・ローズ。美味しい。
  ちなみに日本でよく食べられているブラックタイガーはフランス語にする
  とcrevette géante tigréeと言う。}}{ソース・クルヴェット}}\label{ux30bdux30fcux30b9ux30afux30ebux30f4ux30a7ux30c3ux30c840}}

\hypertarget{sauce-aux-crevettes}{%
\paragraph{Sauce aux Crevettes}\label{sauce-aux-crevettes}}

\index{そーす@ソース!くるうえつと@---・クルヴェット}
\index{くるうえつと@クルヴェット!そーす@ソース・---}
\index{sauce@sauce!crevette@--- aux Crevettes}
\index{crevette@crevette!sauce@Sauce aux Crevettes}

\protect\hyperlink{veloute-de-poisson}{魚料理用ヴルテ}または\protect\hyperlink{sauce-bechamel}{ベシャメルソー
ス}1 Lに、生クリーム1\undemi{}
dlと\protect\hyperlink{fumet-de-poisson}{魚のフュ メ}1\undemi{}
dlを加える。

火にかけて9
dlになるまで煮詰める。鍋を火から外し、\protect\hyperlink{}{ブール・ルー
ジュ}25 g(ソース全体に淡いピンクの色合いを付けるのが目的)を足した
\protect\hyperlink{}{クルヴェットバター}100gを加える。殻を剥いたクルヴェットの尾の身大
さじ3杯を加え、カイエンヌ1つまみで風味を引き締めて仕上げる。

\ldots{}\ldots{}魚料理およびある種の卵料理用。

\maeaki

\hypertarget{ux30abux30ecux30fcux30bdux30fcux30b9}{%
\subsubsection{カレーソース}\label{ux30abux30ecux30fcux30bdux30fcux30b9}}

\hypertarget{sauce-currie}{%
\paragraph{Sauce Currie}\label{sauce-currie}}

\index{そーす@ソース!かれー@カレー---}
\index{かれー@カレー!そーす@---ソース}
\index{sauce@sauce!currie@---  Currie}
\index{currie@currie!sauce@Sauce ---}

以下の材料をバターで軽く色付くまで炒める\ldots{}\ldots{}玉ねぎ250
g、セロリ100 g、 パセリの根\footnote{パセリには根パセリpersil
  tubéreuxといって根が肥大する品種系統も
  ある。平葉で、葉の香りはフランスで一般的なモスカールドタイプ(葉の
  縮れるタイプ)とやや異なる。イタリアンパセリのように用いることが可 能。}30
g、これらはすべてやや厚めにスライスする。タイム1枝と
ローリエの葉少々、メース少々を加える。小麦粉50gとカレー粉\footnote{カレーは植民地インドの料理としてイギリスに伝わり、18世紀にはC\&B
  社によって混合スパイスであるカレー粉が開発された。フランスはあまり
  インドやその他のカレーの食文化と接することもなかったために、こんに
  ちでも「珍しい料理」の範疇にとどまっている。とはいえ、19世紀にイン
  ドからアンティル諸島のうちの英領地域に連れて来られたインド人たちが
  カレーを伝え、それが広まってフランス領アンティーユにおいてコロンボ
  colomboというカレーのバリエーションが成立した。コロンボはこんにち
  のフランスでも(インドのカレーとは別のものとして)比較的よく知られ
  たものとなっている(少なくともcurry, currieという語よりは一般的認
  知度が高いと言えるだろう)。}小さじ1
杯弱を振り入れる。小麦粉が色付かない程度に炒めて火を通したら、\protect\hyperlink{}{白いコ
ンソメ} \troisquarts{} Lを注ぐ。沸騰したら、弱火にして約45分煮る。
軽く押し絞るように布で漉す。ソースを温めて、浮いてきた油脂は取り除き
\footnote{dégraisser デグレセ。}、湯煎にかけておく。

\ldots{}\ldots{}魚料理、甲殻類、鶏、さまざまな卵料理に合わせる。

\hypertarget{ux539fux6ce8-6}{%
\subparagraph{【原注】}\label{ux539fux6ce8-6}}

ココナツミルクをソースに加えることもある。その場合、白いコンソメの
\unquart{}量をココナツミルクに代えること。

\maeaki

\hypertarget{ux30a4ux30f3ux30c9ux98a8ux30abux30ecux30fcux30bdux30fcux30b9}{%
\subsubsection{インド風カレーソース}\label{ux30a4ux30f3ux30c9ux98a8ux30abux30ecux30fcux30bdux30fcux30b9}}

\hypertarget{sauce-currie-indienne}{%
\paragraph{Sauce Currie à l'Indienne}\label{sauce-currie-indienne}}

\index{そーす@ソース!いんどかれー@インドカレー---}
\index{かれー@カレー!そーすいんど@インド---ソース}
\index{sauce@sauce!currie indienne@---  Currie à l'Indienne}
\index{currie@currie!sauce indienne@Sauce --- à l'Indienne}

みじん切り\footnote{原文ciseler
  シズレ。鋭利な刃物でみじん切りにすること、スライス
  すること。原義は「ハサミで切る」。なお、日本語でみじん切りに相当す
  る用語にはhacherアシェもある(hache斧から派生した語)。後者は野菜
  の他、肉類を細かく刻む際にも用いられる。ミートチョッパーをフランス
  語ではhachoirアショワールと呼ぶ。}にした玉ねぎ1個と、パセリ、タイム、ローリエ、メース、シ
ナモン各少々のブーケガルニを、バターとともに弱火にかけて色付かないよう蒸
し煮する。

カレー粉3 gを振り入れ、ココナツミルク\undemi{} Lを注ぐ。ヴルテ \undemi{}
Lを加える(ソースを肉料理に合わせるか、魚料理に合わせるかで、
ヴルテも標準的なものを使うか、魚料理用を使うか決めること)。弱火で15分
程煮る。布で漉し、生クリーム1 dlとレモン果汁少々を加えて仕上げる。

\hypertarget{ux539fux6ce8-7}{%
\subparagraph{【原注】}\label{ux539fux6ce8-7}}

ここで示した量のココナツミルクは、生のココヤシの実700 gをおろして、
4\undemi{} dlの温めた牛乳で溶いて作る。それを布で強く絞って漉してから
使うこと。

ココナツミルクがない場合には、同量のアーモンドミルクを用いてもいい。

インドの料理人によるこのソースの作り方はさまざまで、基本だけが同じというものだ。

だが、本来のレシピがあったところで、使い物にはならないだろう。インドの
カレーは我が国の大多数にとっては我慢ならぬものだろうから。ここで記した
作り方は、ヨーロッパ人の味覚を勘案したものなので、本来のものよりいい筈
だ。

\maeaki

\hypertarget{ux30bdux30fcux30b9ux30c7ux30a3ux30d7ux30edux30deux30c3ux30c844}{%
\subsubsection[ソース・ディプロマット]{\texorpdfstring{ソース・ディプロマット\footnote{外交官風、の意。繊細で豪華な仕立ての料理に付けられる名称。}}{ソース・ディプロマット}}\label{ux30bdux30fcux30b9ux30c7ux30a3ux30d7ux30edux30deux30c3ux30c844}}

\hypertarget{sauce-diplomate}{%
\paragraph{Sauce Diplomate}\label{sauce-diplomate}}

\ruby{既}{すで}に仕上げでおいた\protect\hyperlink{sauce-normande}{ノルマンディ風ソース}1
Lに、\protect\hyperlink{}{オマールバター}75 gを加える。

さいの目に切ったオマールの尾の身大さじ2杯と同様にさいの目に切ったトリュ
フ大さじ1杯を加えて仕上げる。

\ldots{}\ldots{}大きな魚一尾まるごとの\footnote{relevé
  ルルヴェ。17世紀〜19世紀前半ににスタイルとして完成したフ
  ランス式サービスでは、最初に、大きな食卓(しばしば長い楕円形)の両
  側の目立つ場所にポタージュが置かれ、その周囲にアントレ(煮込みやソ
  テーなど今日では「メイン」にもなるもの)およびオルドゥーヴル(「作
  品でないもの」の意で、比較的簡単で小さな皿)が所狭しと並べられた。
  客はまずポタージュから食べはじめるのが基本であり、そのポタージュの
  大きな器が空くと、それは下げられて、ポタージュのあった場所に、豪華
  な装飾を施した飾り台(socleソークル)に載せられ、皿の周囲を飾るよ
  うにガルニチュールが配され(bordureボルデュール)、主役である大き
  な塊肉や魚まるごと1尾の料理にはしばしば飾り串(hâteletアトレ)が刺
  してある、きわめて壮麗な大皿料理が置かれた。ポタージュを取り上げた
  後に「より一層高くそびえ立つ(releverルルヴェした)もの、という意
  味でこの語が用いられるようになった。19世紀後半のロシア式サービスに
  おいても、まずポタージュが配られ、その後にオルドゥーヴル、アントレ
  と続き、ルルヴェを供するという習慣はしばらくの間残っていた。このた
  め、初版、第二版に付属している献立表、および第三版以降独立して出版
  された『メニューの本』にはルルヴェの語はしばしば見られる。1970年代
  ごろから宴席での大皿料理を給仕が取り分けるということが減り、厨房で
  銘々の皿に盛り付けをすることが一般化したために、こんにちでは滅多に
  このスタイルの料理は作られる機会がない。}料理用。

\maeaki

\hypertarget{ux30b9ux30b3ux30c3ux30c8ux30e9ux30f3ux30c9ux98a8ux30bdux30fcux30b9}{%
\subsubsection{スコットランド風ソース}\label{ux30b9ux30b3ux30c3ux30c8ux30e9ux30f3ux30c9ux98a8ux30bdux30fcux30b9}}

\hypertarget{sauce-ecossaise}{%
\paragraph{Sauce Ecossaise}\label{sauce-ecossaise}}

\index{そーす@ソース!すこつとらんとふう@スコットランド風---}
\index{すこつとらんとふう@スコットランド風!そーす@---ソース}
\index{sauce@sauce!ecossaise@--- Ecossaise}
\index{ecossais@écossais!sauce@Sauce Ecossaise}

上記の分量どおりに作った\protect\hyperlink{sauce-creme}{ソース・クレーム}9
dlに以下を加
えて作る。1〜2mmの細さに千切りにしたにんじん、セロリ、さやいんげんをバ
ターを加えて鍋に蓋をして弱火で蒸し煮し\footnote{étuver エチュヴェ。}、\protect\hyperlink{}{白いコンソメ}に完全に
浸したものを1dl。

\noindent\ldots{}\ldots{}卵料理、鶏料理に添える。

\maeaki

\hypertarget{ux30bdux30fcux30b9ux30a8ux30b9ux30c8ux30e9ux30b4ux30f350}{%
\subsubsection[ソース・エストラゴン]{\texorpdfstring{ソース・エストラゴン\footnote{ヨモギ科のハーブ。詳しくは茶色い派生ソースの\protect\hyperlink{sauce-chasseur}{ソース・シャスー
  ル}訳注参照。}}{ソース・エストラゴン}}\label{ux30bdux30fcux30b9ux30a8ux30b9ux30c8ux30e9ux30b4ux30f350}}

\hypertarget{sauce-estragon-blanche}{%
\paragraph{Sauce Estragon}\label{sauce-estragon-blanche}}

\index{そーす@ソース!えすとらこんしろ@---・エストラゴン(ホワイト系)}
\index{えすとらこん@エストラゴン!そーす@ソース・--- (ホワイト系)}
\index{sauce@sauce!estragon blanche@--- Estragon (blanche)}
\index{estragon@estragon!sauce blanche@Sauce --- (blanche)}

エストラゴンの枝30 gを粗く刻み\footnote{concasser コンカセ。}、強火で下茹でする\footnote{blanchir
  ブランシール。}。水気をしっ
かりときり、エストラゴンをスプーンですり潰し、あらかじめ用意しておいた
\protect\hyperlink{veloute}{ヴルテ}を大さじ4杯加える。これを布で漉す。こうして作ったエ
ストラゴンのピュレを\protect\hyperlink{veloute-de-volaille}{鶏のヴルテ}または\protect\hyperlink{veloute-de-poisson}{魚料理用
ヴルテ}1 Lに混ぜ込む。どちらのヴルテを使うから、
合わせる料理によって決めること。味を調え、みじん切りにしたエストラゴン
大さじ\undemi{}杯を加えて仕上げる。

\ldots{}\ldots{}卵料理、鶏肉料理、魚料理に合わせる。

\maeaki

\hypertarget{ux9999ux8349ux30bdux30fcux30b9}{%
\subsubsection{香草ソース}\label{ux9999ux8349ux30bdux30fcux30b9}}

\hypertarget{sauce-aux-fines-herbes-blanche}{%
\paragraph{Sauce aux Fines
Herbes}\label{sauce-aux-fines-herbes-blanche}}

\index{そーす@ソース!こうそうしろ@香草---(ホワイト系)}
\index{こうそう@香草!そーすしろ@---ソース(ホワイト系)}
\index{はーぶ@ハーブ!こうそうそーすしろ@香草ソース(ホワイト系)}
\index{sauce@sauce!fines herbes blanche@--- aux Fines Herbes (blanche)}
\index{fines herbes@fines herbes!sauce blanche@Sauce aux --- (blanche)}

(仕上り5 dl分)

あらかじめ2種のうちどちらかの方法(\protect\hyperlink{sauce-vin-blanc}{白ワインソース}
参照)で作っておいた\protect\hyperlink{sauce-vin-blanc}{白ワインソース}\undemi{}
Lに、 \protect\hyperlink{}{エシャロットバター}40
gと、パセリ、セルフイユ、エストラゴンのみじ
ん切りを大さじ1\undemi{}杯加える。

\ldots{}\ldots{}魚料理用。

\maeaki

\hypertarget{ux30bdux30fcux30b9ux30d5ux30a9ux30a4ux30e8}{%
\subsubsection{ソース・フォイヨ}\label{ux30bdux30fcux30b9ux30d5ux30a9ux30a4ux30e8}}

\hypertarget{sauce-foyot}{%
\paragraph{Sauce Foyot}\label{sauce-foyot}}

\protect\hyperlink{sauce-bearnaise-a-la-glace-de-viande}{グラスドヴィアンド入りソース・ベアルネーズ}参照。

\maeaki

\hypertarget{ux30bdux30fcux30b9ux30b0ux30edux30bcux30a4ux30e651}{%
\subsubsection[ソース・グロゼイユ]{\texorpdfstring{ソース・グロゼイユ\footnote{日本語で「すぐりの実」のことだが、こんにちでは「黒すぐり」の方
  が一般的かも知れない。黒すぐりはフランス語では cassis カシスと呼ば
  れる。一般的なグロゼイユにはフサスグリと呼ばれる groseille rouge
  グロゼイユ・ルージュ(赤すぐり)とgroseille blancheグロゼイユ・ブ
  ランシュ(白すぐり)の2種があり、どちらもブドウのように房なりする。
  上記とは別に、このソースで用いられるgroseille à maquereauグロゼイ
  ヤマクロー(maquereauは鯖の意。日本では英語経由のグーズベリーまた
  はグースベリーの名称でも呼ばれることが多い。単に西洋すぐりとも呼ぶ)
  という比較的大粒で薄く縞模様の入る種類もある。これは通常は緑色だが、ま
  れに紫色になる変種もあるという。いずれもフランスでは料理や菓子作り
  によく用いられる。}}{ソース・グロゼイユ}}\label{ux30bdux30fcux30b9ux30b0ux30edux30bcux30a4ux30e651}}

\hypertarget{sauce-groseilles}{%
\paragraph{Sauce Groseilles}\label{sauce-groseilles}}

\index{そーす@ソース!くろせいゆ@---・グロゼイユ}
\index{くろせいゆ@グロゼイユ!そーす@ソース・---}
\index{sauce@sauce!groseilles@--- Groseilles}
\index{groseille@groseille!sauce@Sauce Groseilles}

緑色の濃いグーズベリー500 gを銅の片手鍋で下茹でする。

5分間煮立てたら、水気をきって、粉砂糖大さじ3杯と白ワイン大さじ2〜3杯を
加えて、完全に火をとおす。布で漉す。

こうして出来たピュレに、\protect\hyperlink{sauce-au-beurre}{ソース・オ・ブール}5
dlを加 え、よく混ぜる。

\ldots{}\ldots{}このソースはグリルあるいはイギリス風\footnote{à
  l'anglaise
  アラングレーズ。通常は塩適量を加えた湯でボイルすることを指す。}に茹でた鯖によく合う。と
はいえ、他の魚料理にも合わせてもいい。

\hypertarget{ux539fux6ce8-8}{%
\subparagraph{【原注】}\label{ux539fux6ce8-8}}

このソースは緑色の房なりのグロゼイユ\footnote{一般的なフサスグリであれば白系統の「未熟果」を用いるということと解釈される。}でも作ることが可能。

\maeaki

\hypertarget{ux30aaux30e9ux30f3ux30fcux30baux30bdux30fcux30b954}{%
\subsubsection[オランーズソース]{\texorpdfstring{オランーズソース\footnote{ニューヨーク発祥の朝食メニューとして知られるエッグ・ベネディク
  ト\emph{Egg
  Benedict}に必ず用いられることで有名なうえ、一般的には「バター
  で作るマヨネーズ」のイメージが強いかも知れない。実際のところは、ラ・
  ヴァレーヌ『フランス料理の本』(1651年)において「アスパラガスの白
  いソース添え」Asperges à la sauce blancheというレシピにおいて、こ
  のオランデーズソースの原型ともいうべきものが示されている。アスパラ
  ガスは固めに塩茹でする。「新鮮なバター、卵黄、塩、ナツメグ、ヴィネ
  ガー少々をよくかき混ぜる。ソースが滑らかになったら、アスパラガスに
  添えて供する(p.238)」。簡潔な記述だが、これがオランデーズソースの
  原型であることは間違いないだろう。おそらくはラ・ヴァレーヌ以前から
  存在していた可能性も否定できない。なお植物油を用いたマヨネーズが文
  献上で確認されるのが18世紀以降で、19世紀初頭から爆発的に流行し、広
  まったもの。また、マヨネーズについては、現代ヨーロッパにおいても卵
  黄ではなく全卵を用いて作るほうが多数を占めている点が異なることに注
  意。なお、オランデーズとは「オランダ風」の意だが、なぜこの名称となっ
  たのかについては不明な点が多い。また、2007年版の『ラルース・ガスト
  ロノミック』では、オランデーズソースを作る際には温度に注意すること
  と、よくメッキされた銅鍋かステンレス製の鍋を用いる必要があり、アル
  ミ製の鍋だと緑色に変色する可能性があることに注意を促している (p.455)。}}{オランーズソース}}\label{ux30aaux30e9ux30f3ux30fcux30baux30bdux30fcux30b954}}

\hypertarget{sauce-hollandaise}{%
\paragraph{Sauce Hollandaise}\label{sauce-hollandaise}}

\index{そーす@ソース!おらんてーす@オランデーズ---}
\index{おらんてーす@オランデーズ!そーす@---ソース}
\index{おらんたふう@オランダ風!そーす@オランデーズソース}
\index{sauce@sauce!hollandaise@--- Hollandaise}
\index{hollandais@hollandais!sauce@Sauce Hollandaise}

大さじ4杯の水とヴィネガー大さじ2杯に、粗挽きこしょう1つまみと肌理の細
かい塩1つまみを加えて、\untiers{}量まで煮詰める。この鍋を熱源のそばか、
湯煎にかける。

大さじ5杯の水と卵黄5個を加える。生のまま、あるいは溶かしたバター500 g
を加えながらしっかりホイップする。ホイップしている途中で、水を大さじ3〜
4杯、少量ずつ足してやる。水を足すのは、軽やかな仕上りにするため。

レモンの搾り汁少々と必要なら塩を足して味を調え、布で漉す。

湯煎にかけておくが、ソースが分離しないように、温度は微温くしておく。

\ldots{}\ldots{}魚料理、野菜料理用。

\hypertarget{ux539fux6ce8-9}{%
\subparagraph{【原注】}\label{ux539fux6ce8-9}}

ヴィネガーを煮詰めて使うのは、いつも最高品質のものが使えるとはかぎらな
いからで、水は\untiers{}量まで減らしたほうがいい。ただし、煮詰める作業
を完全に省いてしまわないこと。

\maeaki

\hypertarget{ux30bdux30fcux30b9ux30aaux30deux30fcux30eb}{%
\subsubsection{ソース・オマール}\label{ux30bdux30fcux30b9ux30aaux30deux30fcux30eb}}

\hypertarget{sauce-homard}{%
\paragraph{Sauce Homard}\label{sauce-homard}}

\index{そーす@ソース!おまーる@---・オマール}
\index{おまーる@オマール!そーす@ソース・---}
\index{sauce@sauce!homard@--- Homard}
\index{homard@homard!sauce@Sauce ---}

\protect\hyperlink{veloute-de-poisson}{魚料理用ヴルテ}\troisquarts{}
Lに、生クリーム1 \undemi{} dlと\protect\hyperlink{}{オマールバター}80
g、\protect\hyperlink{}{赤いバター}40 gを加えて仕上 げる。

\ldots{}\ldots{}魚料理用。

\hypertarget{ux539fux6ce8-10}{%
\subparagraph{【原注】}\label{ux539fux6ce8-10}}

このソースを魚1尾まるごとの料理に添える場合には、さいの目に切ったオマー
ルの尾の身を大さじ3杯加える。

\maeaki

\hypertarget{ux30cfux30f3ux30acux30eaux30fcux98a856ux30bdux30fcux30b9}{%
\subsubsection[ハンガリー風ソース]{\texorpdfstring{ハンガリー風\footnote{原書でも用いられている語paprikaパプリカはハンガリー語。唐辛子、
  ピーマンの仲間であり、16世紀以降17世紀にヨーロッパ全土に広まり、そ
  の土地ごとの風土に合わせて品種が多様化した。パプリカはとりわけ辛味
  成分をほとんど含んでいないのが特徴。ただし、ハンガリーの食文化にお
  いて大きな役割を果すようになったのは19世紀以降になってからと言われ
  ている。}ソース}{ハンガリー風ソース}}\label{ux30cfux30f3ux30acux30eaux30fcux98a856ux30bdux30fcux30b9}}

\hypertarget{sauce-hongroise}{%
\paragraph{Sauce Hongroise}\label{sauce-hongroise}}

\index{そーす@ソース!はんかりーふう@ハンガリー風---}
\index{はんかりーふう@ハンガリー風!そーす@---ソース}
\index{sauce@sauce!hongroise@---  Hongroise}
\index{hongrois@hongrois!sauce@Sauce Hongroise}

大きめの玉ねぎ1個のみじん切りをバターで色付かないよう強火で炒める。塩1
つまみとパプリカ粉末1 gで味付けする。

このソースを添える料理に合わせて\protect\hyperlink{veloute}{標準的なヴルテ}あるいは\protect\hyperlink{veloute-de-poisson}{魚
料理用ヴルテ} 1 Lを加え、数分間軽く煮立てる。

布で漉し、バター100 gを加えて仕上げる。

このソースは淡いピンク色に仕上げるべきであり、その色を出しているのがパ
プリカ粉末だけによるものだということに注意。

\ldots{}\ldots{}仔羊や仔牛のノワゼット\footnote{noisette
  ロースの中心部分を円筒形に切り出して調理したもの。}にとりわけよく合う。卵料理、鶏料理、魚
料理にも。

\maeaki

\hypertarget{ux7261ux8823ux5165ux308aux30bdux30fcux30b9}{%
\subsubsection{牡蠣入りソース}\label{ux7261ux8823ux5165ux308aux30bdux30fcux30b9}}

\hypertarget{sauce-aux-huitres}{%
\paragraph{Sauce aux Huîtres}\label{sauce-aux-huitres}}

\index{そーす@ソース!かきいり@牡蠣入り---}
\index{かき@牡蠣!そーす@牡蠣入りソース}
\index{sauce@sauce!huitres@---  aux Huîtres}
\index{huitre@huître!sauce@Sauce aux Huîtres}

後述の\protect\hyperlink{sauce-normande}{ノルマンディ風ソース}に、ポシェ\footnote{pocher
  \textless{} poche ポシュ(ポケット)、からの派生語。ポーチドエッグ
  を作る際に、ポケット状になるところからこの用語が定着した。沸騰しな
  い程度の温度で加熱調理すること。}して周囲をきれ
いにした牡蠣の身を加えたもの。

\maeaki

\hypertarget{ux30a4ux30f3ux30c9ux98a8ux30bdux30fcux30b9}{%
\subsubsection{インド風ソース}\label{ux30a4ux30f3ux30c9ux98a8ux30bdux30fcux30b9}}

\hypertarget{sauce-indienne}{%
\paragraph{Sauce Indienne}\label{sauce-indienne}}

\index{そーす@ソース!いんとふう@インド風---}
\index{いんとふう@インド風!そーす@---ソース}
\index{sauce@sauce!indienne@---  Indienne}
\index{indien@indien!sauce@Sauce Indienne}

\protect\hyperlink{sauce-currie-indienne}{インド風カレーソース}参照。

\maeaki

\hypertarget{ux30bdux30fcux30b9ux30a4ux30f4ux30a9ux30efux30fcux30eb}{%
\subsubsection{ソース・イヴォワール}\label{ux30bdux30fcux30b9ux30a4ux30f4ux30a9ux30efux30fcux30eb}}

\hypertarget{sauce-ivoire}{%
\paragraph[Sauce Ivoire]{\texorpdfstring{Sauce Ivoire\footnote{象牙、の意。}}{Sauce Ivoire}}\label{sauce-ivoire}}

\index{そーす@ソース!いうおわーる@---・イヴォワール}
\index{いうおわーる@イヴォワール!そーす@ソース・---}
\index{sauce@sauce!ivoire@--- Ivoire}
\index{ivoire@ivoire!sauce@Sauce ---}

\protect\hyperlink{sauce-supreme}{ソース・シュプレーム}1
Lに、ブロンド色の\protect\hyperlink{glace-de-viande}{グラスドヴィ
アンド}大さじ3杯を加え、象牙のようなくすんだ色合いに する。

\ldots{}\ldots{}ポシェした鶏に添える。

\maeaki

\hypertarget{ux30bdux30fcux30b9ux30b8ux30e7ux30efux30f3ux30f4ux30a3ux30eb61}{%
\subsubsection[ソース・ジョワンヴィル]{\texorpdfstring{ソース・ジョワンヴィル\footnote{19世紀、7月王政期の国王ルイ・フィリップの第3子、フランソワ・ド
  ルレアン・ジョワンヴィル海軍大将(1818〜1900)のこと。エクルヴィス
  とクルヴェットを用いた料理に彼の名が冠されたものがいくつかある。}}{ソース・ジョワンヴィル}}\label{ux30bdux30fcux30b9ux30b8ux30e7ux30efux30f3ux30f4ux30a3ux30eb61}}

\hypertarget{sauce-joinville}{%
\paragraph{Sauce Joinville}\label{sauce-joinville}}

\index{しよわんういる@ジョワンヴィル!そーす@ソース・---}
\index{そーす@ソース!しよわんういる@---・ジョワンヴィル}
\index{sauce@sauce!joinville@--- Joinville}
\index{joinville@Joinville!sauce@Sauce ---}

\protect\hyperlink{sauce-normande}{ノルマンディ風ソース}1
Lを、仕上げる直前の段階まで作 る\footnote{すなわち、布で漉すところまで。}。\protect\hyperlink{}{エクルヴィスバター}60
gと\protect\hyperlink{}{クルヴェットバター}60 gを加 えて仕上げる。

このソースを添える魚料理にガルニチュールが既にある場合は、これ以上は何も加えない。

ガルニチュールを伴なわない大きな魚のブイイ\footnote{魚の場合は、クールブイヨンを用いてやや低めの温度で煮たもの。}に添える場合には、細さ1〜
2mmの千切りにした真黒なトリュフを大さじ2杯加えること。

\hypertarget{ux539fux6ce8-11}{%
\subparagraph{【原注】}\label{ux539fux6ce8-11}}

同様のソースはいろいろあるが、最後の仕上げにエクルヴィスバターとクル
ヴェットバターを組み合わせて加える点がソース・ジョワンビルが他のものと
違うポイント。

\hypertarget{ux30bdux30fcux30b9ux30e9ux30aeux30d4ux30a8ux30fcux30eb62}{%
\subsubsection[ソース・ラギピエール]{\texorpdfstring{ソース・ラギピエール\footnote{18世紀末〜19世紀初頭にかけて活躍したフランスを代表する料理人の
  名(?〜1812)。はじめコンデ公に仕え、革命時にコンデ公の亡命にも随
  行したが、後にフランスに帰国し、ナポレオン\ruby{麾下}{きか}に入っ
  た。ナポレオン自身は食に無頓着であったが、直接的にはミュラ元帥のも
  とで料理長として活躍した。タレーランに仕えていたアントナン・カレー
  ムは2年程の期間であったが、ラギピエールとともに宴席の仕事に携わり、
  生涯を通して師と仰ぐ程に尊敬してやまなかった。当然だが料理において
  カレームはラギピエールから大きく影響を受け、そのことを後年、数冊の
  自著で明記している。ラギピエール自身はミュラ元帥に従ってロシア戦線
  に赴き、その撤退の途中、極寒の地で凍死した。カレームは1828年刊『パ
  リ風の料理』の冒頭2ページを「ラギピエールの想い出に」と題し、とて
  も力強い文体でその死を悼んだ。}}{ソース・ラギピエール}}\label{ux30bdux30fcux30b9ux30e9ux30aeux30d4ux30a8ux30fcux30eb62}}

\hypertarget{sauce-laguipiere}{%
\paragraph{Sauce Laguipière}\label{sauce-laguipiere}}

上述のとおりに作った\protect\hyperlink{sauce-au-beurre}{ソース・オ・ブール}
1 Lに、レモ
ン1個の搾り汁と\protect\hyperlink{glace-de-poisson}{魚のグラス}またはそれと同等に煮詰め
た\protect\hyperlink{fumet-de-poisson}{魚のフュメ}大さじ4杯を加える。

このソースは魚のブイイに添える。

\hypertarget{ux539fux6ce8-12}{%
\subparagraph{【原注】}\label{ux539fux6ce8-12}}

カレームが考案したこのソースのレシピに、本書で加えた変更点はただ1箇所
のみ、\protect\hyperlink{glace-de-volaille}{鶏のグラス}ではなく魚のグラスに代えたことだ
けだ。さらに言うと、このソースはカレームによって「ソース・オ・ブール 
ラギピエール風」と名付けられたものだ\footnote{カレームの未完の大著『19世紀フランス料理』第3巻に、このソースの
  レシピが掲載されている。少し長くなるが引用すると「ラグー用片手鍋に、\textbf{
  魚料理用グランドソース}の章で示したソース・オ・ブールをレードル1杯入
  れる。ここに上等のコンソメ大さじ1杯か鶏のグラス少々を加える。塩1つまみ、
  ナツメグ少々、良質のヴィネガーまたはレモン果汁適量を加える。数秒間煮立
  たせ、上等なバターをたっぷり加えてから供する。(中略)ソースに火を通して
  からバターを加えるというこの方法によって、なめらかな口あたりで、油っぽ
  くならない仕上りになる。だからこそ私はこのソース・オ・ブールをグランド
  ソースに分類しなかったのだし、バターを加える派生ソースにおいてこれは重
  要なことだからだ。それは魚料理用ソースについても同様のことだ
  (pp.117-118)」。このレシピにおいて、カレームの表現には一箇所だけ矛盾が
  ある。「魚用グランドソースの章で示した」とあるのに、「グランドソースに
  分類しなかった」となっていることだ。実際、ソース・オ・ブールそれ自体の
  記述はこの「ラギピエール風」の直前のページにある。とはいえ、カレームの
  著作にはある種の「雑なところ」があり、こうした矛盾を読み解くこともまた
  カレームの魅力のひとつであるとは言えるだろう。さて、このソースが「ラギ
  ピエール風」であることの理由だが、同じ巻の「魚料理用ソース・エスパニョ
  ル」の説明の冒頭において、ラギピエールから聞いた話として、ラギピエール
  が若い頃、四旬節の期間(小斉=肉断ちをする慣習がカトリックに根強くあっ
  た)、魚料理用のソースにコンソメや仔牛のブロンドのジュを混ぜている修道
  士料理人がいたのだ、と述べている。それなら美味しくて当然だろう、とカレー
  ムが問うと、ラギピエールは「しかもそうやって作った料理は、通常の肉を食
  べていい時の料理とは違うものであり、かといって肉断ちの料理でもない、ま
  さに中間のものだ。その判定は天のみぞ知るところだろう。結局のところ、修
  道士たちは元気に暮していたのだから、それは正しかったのだよ」と答えたと
  いう。このエピソードにある、カトリックの習慣としての肉断ちのための魚料
  理用ソースに、肉由来である鶏のグラスもしくはコンソメを加えるというとこ
  ろが、ラギピエール風と名付けた\ruby{所以}{ゆえん}であり、まさにこれこ
  そがソース・ラギピエールの重要なポイントだと推測されよう。『料理の手引
  き』においてこのレシピを担当した執筆者はこののエピソードを読んでいなかっ
  たのだろうか? あるいは何らかの誤解ゆえに改変をしたのか、もしくは信仰
  上の理由からか、ラギピエール風の\ruby{所以}{ゆえん}である鶏のグラス、
  コンソメを用いるべきところを、魚のグラスに代えてしまい、このソース名の
  由来を換骨奪胎してしまう結果となっているのは非常に不思議だ。本書の初版
  において原注がその文体から、ほぼエスコフィエの手になるものか、あるいは
  聞き書きしたコメントであることは明らかなので、なぜエスコフィエがこの点
  を見逃したか、あるいは許容したのかは非常に興味深い。ところで、カレーム
  が、バターを仕上げの際に加えるということ、いわゆるブールモンテmonter
  au beurreによってソースの口あたりをなめらかなものにし、色艶をよくする
  ということをことさらに言及していることもまた、注目に値すべき点だろう。}。

\maeaki

\hypertarget{ux30eaux30f4ux30a9ux30cbux30a264ux98a8ux30bdux30fcux30b9}{%
\subsubsection[リヴォニア風ソース]{\texorpdfstring{リヴォニア\footnote{現在のラトビア東北部からエストニア南部にかけての古い地域名、い
  わゆるバルト三国の一地域と捉えていい。本書執筆時にはロシア帝国の一
  部となっていた。なお、料理名に冠される地名のうちの少からずのものに
  明確な由来のないのと同様に、このソースについても名称の由来は不明。}風ソース}{リヴォニア風ソース}}\label{ux30eaux30f4ux30a9ux30cbux30a264ux98a8ux30bdux30fcux30b9}}

\hypertarget{sauce-livonienne}{%
\paragraph{Sauce Livonienne}\label{sauce-livonienne}}

\index{そーす@ソース!りうおにあふう@リヴォニア風---}
\index{りうおにあふう@リヴォニア風!そーす@---ソース}
\index{sauce@sauce!livonienne@---  Livonienne}
\index{livonien@livonien!sauce@Sauce Livonienne}

バターを加えて仕上げた\footnote{monter au beurre バターでモンテする。}\protect\hyperlink{veloute-de-poisson}{魚のフュメで作ったヴル
テ}1 Lに、1〜2mmの細さで長さ3〜4cmの千切り\footnote{julienne
  ジュリエンヌ}に
したにんじん、セロリ、マッシュリューム、玉ねぎをあらかじめバターを加え
て弱火で蒸し煮\footnote{étuver au beurre バターでエチュヴェする。}したおいたもの100
gを加える。最後に、1〜2mmの細さの
トリュフの千切りと粗く刻んだパセリを加える。\ldots{}\ldots{}味を調えること。

\ldots{}\ldots{}このソースは、トラウト、サーモン、舌びらめ、チュルボタン\footnote{turbotin
  \textless{} turbo チュルボ。鰈の近縁種。}、バルビュ\footnote{barbue
  鰈の近縁種。}のような魚によく合う。

\maeaki

\hypertarget{ux30deux30ebux30bfux98a870ux30bdux30fcux30b9}{%
\subsubsection[マルタ風ソース]{\texorpdfstring{マルタ風\footnote{シチリアの南方に位置するマルタ島を中心とした国、マルタはオレン
  ジをはじめとした柑橘類の産地であり、とりわけ19世紀にはマルタ産のブ
  ラッドオレンジが人気であった。}ソース}{マルタ風ソース}}\label{ux30deux30ebux30bfux98a870ux30bdux30fcux30b9}}

\hypertarget{sauce-maltaise}{%
\paragraph{Sauce Maltaise}\label{sauce-maltaise}}

\index{そーす@ソース!まるたふう@マルタ風---}
\index{まるたふう@マルタ風!そーす@---ソース}
\index{maltais@maltais!sauce@Sauce Maltaise}
\index{sauce@sauce!maltaise@--- Maltaise}

前述のとおりに、\protect\hyperlink{sauce-hollandaise}{ソース・オランデーズ}を作り、提供
直前に、\textbf{ブラッドオレンジ}2個の搾り汁を加える。ブラッドオレンジを用
いないとこのソースは成立しないので注意。オレンジの皮の表面をおろしたもの\footnote{zeste
  ゼスト。} 1つまみを加えて仕上げる。

\ldots{}\ldots{}アスパラガスに添える。

\maeaki

\hypertarget{ux30bdux30fcux30b9ux30deux30eaux30cbux30a8ux30fcux30eb72}{%
\subsubsection[ソース・マリニエール]{\texorpdfstring{ソース・マリニエール\footnote{marinier
  / marinière \textless{} mare ラテン語「海」から派生した語。貝や
  魚を白ワインで煮た料理にも付けられる名称。}}{ソース・マリニエール}}\label{ux30bdux30fcux30b9ux30deux30eaux30cbux30a8ux30fcux30eb72}}

\hypertarget{sauce-mariniere}{%
\paragraph{Sauce Marinière}\label{sauce-mariniere}}

\index{まりにえーる@マリニエール!そーす@ソース・---}
\index{そーす@ソース!まりにえーる@---・マリニエール}
\index{sauce@sauce!mariniere@--- Marini\`ere}
\index{mariniere@marini\`ere!sauce@Sauce ---}

\protect\hyperlink{sauce-bercy}{ソース・ベルシー}を本書で示したとおりの分量で用意する。
これにムール貝の煮汁を煮詰めたもの大さじ3〜4杯を加え、卵黄6個でとろみ
を付ける\footnote{卵黄でとろみ付けをする場合、よく混ぜてさえいれば、必ずしも弱火
  でなくても問題ない。ただし、沸騰状態だと滑かに仕上がらないリスクが
  残るので、ある程度は弱火にした方がいいだろう。}。

\ldots{}\ldots{}ムール貝の料理専用。

\maeaki

\hypertarget{ux767dux3044ux30bdux30fcux30b9ux30deux30c8ux30edux30c3ux30c874}{%
\subsubsection[白いソース・マトロット]{\texorpdfstring{白いソース・マトロット\footnote{水夫風、船員風、の意。}}{白いソース・マトロット}}\label{ux767dux3044ux30bdux30fcux30b9ux30deux30c8ux30edux30c3ux30c874}}

\hypertarget{sauce-matelote-blanche}{%
\paragraph{Sauce Matelote blanche}\label{sauce-matelote-blanche}}

\index{そーす@ソース!まとろつとしろ@---・マトロット(白)}
\index{まとろつと@マトロット!そーす@ソース・---(白)}
\index{sauce@sauce!matelote blonche@--- Matelote blanche}
\index{matelote@matelote!sauce@Sauce --- blanche}

白ワインで作った魚のクールブイヨン3 dlにフレッシュなマッシュルームの切
りくず\footnote{料理、ガルニチュールとして供するマッシュルームは、トゥルネといっ
  て\ruby{螺旋}{らせん}状に切り込みを入れて装飾するのが一般的。その
  下ごしらえの際に大量のマッシュルームの切りくずが出るので、それを利
  用する。}25 gを加えて\untiers{}量まで煮詰める。

\protect\hyperlink{veloute-de-poisson}{魚料理用ヴルテ}8
dlを加える。数分間煮立たせる。 布で漉し、バター150 gを加える。

カイエンヌ\footnote{cayenne
  唐辛子の1品種。日本で一般的なカエンペッパーよりは辛さが
  マイルドで風味も異なる。}ごく少量で風味を引き締める。

ガルニチュールとして、下茹でしてからバターで色艶よく炒めた\footnote{glacer
  au beurre グラセオブール。バターでグラセする、と表現する
  調理現場も多い。glace グラス(鏡)が語源であるため、本来は「光沢を
  出させる、照りをつける」の意だが、食材や料理によってその手法はさま
  ざま。にんじんや小玉ねぎの場合にはあらかじめ下茹でしておく必要があ
  る。}小玉ねぎ20個
と、あらかじめ茹でておいた小さな白いマッシュルーム\footnote{これを用意している段階で、上述のトゥルネを行なう。常識的なこと
  として明記されていないことに注意。この作業の結果、ソースを作る際に
  魚の煮汁(クールブイヨン)に加えるマッシュルームの切りくずが発生す
  ることになる。}20個を加える。

\maeaki

\hypertarget{ux30bdux30fcux30b9ux30e2ux30ebux30cdux30fc79}{%
\subsubsection[ソース・モルネー]{\texorpdfstring{ソース・モルネー\footnote{19世紀中頃にパリのレストラン、デュランの料理長ジョゼフ・ヴォワ
  ロンが創案したと言われている。モルネーは人名だが、具体的に誰を指し
  ているかについては諸説ある。}}{ソース・モルネー}}\label{ux30bdux30fcux30b9ux30e2ux30ebux30cdux30fc79}}

\hypertarget{sauce-mornay}{%
\paragraph{Sauce Mornay}\label{sauce-mornay}}

\index{もるねー@モルネー!そーす@---ソース}
\index{そーす@ソース!もるねー@---・モルネー}
\index{sauce@sauce!mornay@--- Mornay}
\index{mornay@Mornay!sauce@Sauce ---}

\protect\hyperlink{sauce-bechamel}{ベシャメルソース}1
Lに、このソースを合わせる魚の煮汁 2
dlを加え、\deuxtiers{}量程に煮詰める\footnote{初版ではこの煮詰める作業はなく「固めに作ったベシャメルソース1
  L に対し、魚の煮汁2 dlを加える」となっている。}。おろした\footnote{râper
  ラペ \textless{} râpe ラプという器具を用いておろすこと。パルメザン
  (パルミジャーノ)は硬質チーズなので一般的な半筒形のチーズおろし器
  でいいが、グリュイエールは比較的軟質なので、より目の粗い器具(例え
  ばマンドリーヌに付属している機能のうち、にんじんをおろす際に使う部
  分など)を用いるといい。}グリュイエー ルチーズ50 gとパルメザンチーズ50
gを加える。少しの間、火にかけたままに
してよく混ぜ、チーズを完全に溶かし込む。バター100
gを加えて仕上げる\footnote{monter au beurre
  モンテオブール。バターでモンテする、と表現する ことも多い。}。

\hypertarget{ux539fux6ce8-13}{%
\subparagraph{【原注】}\label{ux539fux6ce8-13}}

魚以外の料理に合わせる場合\footnote{例えば茹でた野菜などにかけて、サラマンダー(強力な上火だけのオー
  ブンの一種)に入れて軽く焦げ目を付け、グラタンにするようなケースも
  多い。}も作り方はまったく同じだが、魚の煮汁は加えない。

\maeaki

\hypertarget{ux30bdux30fcux30b9ux30e0ux30b9ux30eaux30fcux30cc84-ux30bdux30fcux30b9ux30b7ux30e3ux30f3ux30c6ux30a3ux30a485}{%
\subsubsection[ソース・ムスリーヌ /
ソース・シャンティイ]{\texorpdfstring{ソース・ムスリーヌ\footnote{mousseline
  \textless{} mousse ムース。-ine は「小さい」を意味する接尾辞。
  その前にLの文字が入るのは、mousseの語源がメソポタミアの都市Mossoul
  (モスリン布の生産地だった)であることによる。} /
ソース・シャンティイ\footnote{シャンティイの由来などについては\protect\hyperlink{sauce-chantilly}{ソース・シャンティイ}参照。}}{ソース・ムスリーヌ / ソース・シャンティイ}}\label{ux30bdux30fcux30b9ux30e0ux30b9ux30eaux30fcux30cc84-ux30bdux30fcux30b9ux30b7ux30e3ux30f3ux30c6ux30a3ux30a485}}

\hypertarget{sauce-mousseline}{%
\paragraph{Sauce Mousseline, dite Sauce
Chantilly}\label{sauce-mousseline}}

前述のとおりの分量と作り方で\protect\hyperlink{sauce-hollandaise}{ソース・オランデーズ}を用意する(\protect\hyperlink{sauce-hollandaise}{ソース・オランデーズ}参照)。

提供直前に、固く泡立てた生クリーム大さじ4杯\footnote{大さじ1杯=15ccという考えにとらわれないよう注意。この計量単位は
  日本で戦後普及したものに過ぎず、本書においては文字通りに「大きなス
  プーンで4杯」という大雑把な単位として考える必要がある。このソース
  の場合は「固く泡立てた生クリームを適量」と読み替えてもいいだろう。
  名称どおりに滑らかでふんわりとした口あたりに仕上げるのがポイント。}をソースに混ぜ込む。

\ldots{}\ldots{}このソースは、魚のブイイ\footnote{bouilli 茹でた、の意。}や、アスパラガス、カルドン\footnote{cardon
  アーティチョークの近縁種で、アーティチョークが開花前の蕾
  を食用とするのに対し、カルドンは軟白させた茎葉を食用とする。フラン
  スではトゥーレーヌ地方産が有名。草丈1.5m位まで成長させた株を紐で束
  ねて軟白する。厳冬期は株元から刈り取って小屋などで保管するのが伝統
  的な手法。イタリア北部ピエモンテでは株を倒してその上に土を被せて軟
  白するというユニークな方法で栽培するcardo gobboカルドゴッボもよく
  知られている。}、セロリ\footnote{セロリには緑の濃い品種系統と、やや緑が薄く、中心部が自然に軟白
  されたようになる系統がある。野菜料理として用いられるのは主として後
  者の芯に近い、自然に軟白された部分。coeur de céleri クールドセルリ
  と呼ぶ。前者については、もっぱら香味野菜としてフォンやポタージュ、
  煮込み料理などに用いられる。このタイプは風味に癖があるため、生食に
  はあまり適していない。}に添える。

\maeaki

\hypertarget{ux30bdux30fcux30b9ux30e0ux30b9ux30fcux30ba90}{%
\subsubsection[ソース・ムスーズ]{\texorpdfstring{ソース・ムスーズ\footnote{細かく泡立った、の意。なお、シャンパーニュのようなvin
  mousseux ヴァン・ムスー(発泡ワイン)のムスーは同じ語の男性形.}}{ソース・ムスーズ}}\label{ux30bdux30fcux30b9ux30e0ux30b9ux30fcux30ba90}}

\hypertarget{sauce-mousseuse}{%
\paragraph{Sauce Mousseuse}\label{sauce-mousseuse}}

沸騰した湯の中に、小さめのソテー鍋を入れて熱し、水気をよく拭き取る。こ
のソテー鍋に、あらかじめ充分に柔らかくしておいたバター500 gを入れる。
塩8 gを加え、泡立て器でしっかり混ぜながら、レモン\unquart{}個分の搾り
汁と冷水4 dlを少しずつ加える。

最後に、固く泡立てた生クリーム大さじ4杯を混ぜ込む。

このレシピは、ソースに分類してはいるが、むしろ合わせバターというべきも
のだ。魚のブイイに合わせる。

茹でた魚から伝わる熱だけでバターは充分に溶けるので、見た目も風味も溶か
しバターをソースにするよりずっといいものだ。

\maeaki

\hypertarget{ux30bdux30fcux30b9ux30e0ux30bfux30ebux30c991}{%
\subsubsection[ソース・ムタルド]{\texorpdfstring{ソース・ムタルド\footnote{マスタードのこと。マスタードソースと呼んでもいいが、アメリカ風
  の印象を与えるかも知れない。}}{ソース・ムタルド}}\label{ux30bdux30fcux30b9ux30e0ux30bfux30ebux30c991}}

\hypertarget{sauce-moutarde}{%
\paragraph{Sauce Moutarde}\label{sauce-moutarde}}

\index{そーす@ソース!ますたーど@---・ムタルド}
\index{ますたーと@マスタード!そーす@ソース・ムタルド}
\index{むたると@ムタルド!そーす@ソース・ムタルド}
\index{sauce@sauce!moutarde@--- Moutarde}
\index{moutarde@moutarde!sauce@Sauce ---}

普通、このソースは提供直前に作ること。

必要の分量の\protect\hyperlink{sauce-au-beurre}{ソース・オ・ブール}を用意する。鍋を火か
ら外し、ソース2\undemi{} dlあたり大さじ1杯のマスタードを加える。

このソースを仕上げて、提供するまで時間を空けなくてはならない場合は、湯
煎にかけておく。沸騰させないよう注意すること。

\maeaki

\hypertarget{ux30bdux30fcux30b9ux30caux30f3ux30c1ux30e5ux30a292}{%
\subsubsection[ソース・ナンチュア]{\texorpdfstring{ソース・ナンチュア\footnote{ローヌ・アルプ地方にあるナンチュア湖でエクルヴィスが穫れること
  に由来したソース名。エクルヴィスについて詳しくは\protect\hyperlink{sauce-bavaroise}{バイエルン風ソー
  ス}訳注参照。}}{ソース・ナンチュア}}\label{ux30bdux30fcux30b9ux30caux30f3ux30c1ux30e5ux30a292}}

\hypertarget{sauce-nantua}{%
\paragraph{Sauce Nantua}\label{sauce-nantua}}

\protect\hyperlink{sauce-bechamel}{ベシャメルソース}1 Lに生クリーム2
dlを加え、 \deuxtiers{}量まで煮詰める。

布で漉し、生クリームをさらに1\undemi{} dl加えて、通常の濃度に戻す。

良質な\protect\hyperlink{beurre-d-ecrevisse}{エクルヴィスバター}125
gと、小さめのエクル ヴィスの尾の身\footnote{しっかり下茹でして殻を剥いたものを用いること。}20を加えて仕上げる。

\maeaki

\hypertarget{ux6d3bux3051ux30aaux30deux30fcux30ebux3067ux4f5cux308bux30bdux30fcux30b9ux30cbux30e5ux30fcux30d0ux30fcux30b095}{%
\subsubsection[活けオマールで作るソース・ニューバーグ]{\texorpdfstring{活けオマールで作るソース・ニューバーグ\footnote{ここでは英語由来のソース名のため英語風にカタカナ書きしたが、フ
  ランスでは「ニュブール」のように発音されることも多い。}}{活けオマールで作るソース・ニューバーグ}}\label{ux6d3bux3051ux30aaux30deux30fcux30ebux3067ux4f5cux308bux30bdux30fcux30b9ux30cbux30e5ux30fcux30d0ux30fcux30b095}}

\hypertarget{sauce-new-burg-avec-le-homard-cru}{%
\paragraph{Sauce New-burg avec le homard
cru}\label{sauce-new-burg-avec-le-homard-cru}}

\index{そーす@ソース!にゆーはーくいけおまーる@活けオマールを使う---・ニューバーグ}
\index{にゆーはーく@ニューバーグ!そーす@活けオマールを使うソース・---}
\index{sauce@sauce!new-burg homard cru@--- New-burg avec le homard cru}
\index{new-burg@New-burg!sauce homard cru@Sauce --- avec le homard cru}

800〜900 gのオマールを切り分ける。

胴の中のクリーム状の部分をスプーンで取り出し、これをよくすり潰して30 g
のバターを合わせ、別に取り置いておく。

バター40 gと植物油大さじ4杯を鍋に入れて熱し、切り分けたオマールの身を
色付くまで焼く。塩とカイエンヌで調味する。殻が真っ赤になったら、鍋の油
を完全に捨て、コニャック大さじ2杯と、マルサラ酒もしくはマデラの古酒2
dlを注いで火を付けてアルコール分を燃やす\footnote{flamber フランベする。}。注いだ酒が\untiers{}量
になるまで煮詰めたら、生クリーム2
dlと\protect\hyperlink{fumet-de-poisson}{魚のフュメ}2
dlを注ぐ。弱火で25分間煮る。

オマールの身をざるにあげて水気をきる。殻から身を取り出して、さいの目に
切る。

取り置いておいたオマールのクリーム状の部分をソースに混ぜ込み、完全に火
が通るように軽く煮立たせてやる。さいの目に切ったオマールの身を加えて混
ぜる。味見をして、必要なら塩を加えて修正する。

\hypertarget{ux539fux6ce8-14}{%
\subparagraph{【原注】}\label{ux539fux6ce8-14}}

さいの目に切ったオマールの身をソースに混ぜ込むのは絶対必要というわけで
はない。薄くやや斜めにスライスして、このソースを合わせる魚料理に添えて
もいい。

\maeaki

\hypertarget{ux8339ux3067ux305fux30aaux30deux30fcux30ebux3067ux4f5cux308bux30bdux30fcux30b9ux30cbux30e5ux30fcux30d0ux30fcux30b0100}{%
\subsubsection[茹でたオマールで作るソース・ニューバーグ]{\texorpdfstring{茹でたオマールで作るソース・ニューバーグ\footnote{このソースの元となった料理「オマール・ニューバーグ」は、19世
  紀後半にニューヨークのレストラン、デルモニコーズで常連客のアイデア
  をもとにフランス出身の料理長シャルル・ラノフェール(チャールズ・レ
  ンフォーファー)が完成させたと言われており、そのレシピがラノフェー
  ルの著書\href{https://archive.org/details/epicureancomplet00ranhrich}{『ジ・エピキュリア
  ン』}(英
  語)に掲載されている(p.411)。現在もデルモニコーズのスペシャリテと
  して知られている。ただし、ラノフェールのレシピは先にオマールを茹で
  るという、本項のレシピに近いものであり、前項の活けオマールを使うレシ
  ピはエスコフィエもしくは他の料理人によって改変させたものと考えられ
  る。なお、このレシピと次項のソース・ニューバーグは第二版で追加され
  たものであり、その後は原注も含めて異同がない。}}{茹でたオマールで作るソース・ニューバーグ}}\label{ux8339ux3067ux305fux30aaux30deux30fcux30ebux3067ux4f5cux308bux30bdux30fcux30b9ux30cbux30e5ux30fcux30d0ux30fcux30b0100}}

\hypertarget{sauce-new-burg-avec-le-homard-cuit}{%
\paragraph{Sauce New-burg avec le homard
cuit}\label{sauce-new-burg-avec-le-homard-cuit}}

オマールを\protect\hyperlink{}{標準的なクールブイヨン}で茹でる。尾の身を殻から外し、や
や斜めに厚さ1cm程度の筒切りにする\footnote{détailler en escalopes =
  escalopper エスカロップ(厚さ1〜2cm程度の薄切り)に切る。}。ソテー鍋の内側にたっぷりとバター
を塗り、そこに切ったオマールを並べるように入れる。塩とカイエンヌでしっ
かりと味を付け、表皮が赤く発色するように両面を焼く。上等なマデラ酒をひ
たひたの高さまで注ぎ、ほぼ完全になくなるまで煮詰める。

提供直前に、オマールのスライスの上に、生クリーム2 dlと卵黄3個を溶いた
ものを注ぎ、火から外して、ゆっくり混ぜながら\footnote{vanner
  ヴァネする。}しっかりととろみを付 ける。

\hypertarget{ux539fux6ce8-15}{%
\subparagraph{【原注】}\label{ux539fux6ce8-15}}

\protect\hyperlink{sauce-americaine}{ソース・アメリケーヌ}と同様に、これら2種のソースも
元来はオマールを供するための料理だった。ソースとオマールが、要するにひ
とつの料理を構成していたわけだ。

ところが、そのような料理は午餐(ランチ)でしか提供することが出来ない。
多くの人々は胃が弱く、夕食では消化しきれないのだ\footnote{レシピにおいて指示されているオマールが大きなものであることに注
  意。}。

そうした問題解決のために、我々はこれを、舌びらめのフィレやムスリーヌに
添えるオマールのソースとして使うことにしたのだ。オマールの身はガルニ
チュールとして添えるにとどめることにした。結果は好評であった。

カレー粉やパプリカ粉末を調味料として用いれば、このソースのとてもいいバ
リエーションが作れる。とりわけ舌びらめや脂身の少ない白身魚によく合
う。\ldots{}\ldots{}その場合、魚に少量の\protect\hyperlink{riz-a-l-indienne}{インド風ライス}を添えるといい。

\maeaki

\hypertarget{ux30bdux30fcux30b9ux30ceux30efux30bcux30c3ux30c8102}{%
\subsubsection[ソース・ノワゼット]{\texorpdfstring{ソース・ノワゼット\footnote{ヘーゼルナッツ、榛の実。}}{ソース・ノワゼット}}\label{ux30bdux30fcux30b9ux30ceux30efux30bcux30c3ux30c8102}}

\hypertarget{sauce-noisette}{%
\paragraph{Sauce Noisette}\label{sauce-noisette}}

\index{そーす@ソース!へーせるなっつ@---・ノワゼット}
\index{へーせるなつつ@ヘーゼルナッツ!そーす@ソース・ノワゼット}
\index{のわせつと@ノワゼット!へーぜるなっつそーす@ヘーゼルナッツソース}
\index{sauce@sauce!noisette@--- Noisette}
\index{noisette@noisette!sauce@Sauce ---}

\protect\hyperlink{sauce-hollandaise}{ソース・オランデーズ}を本書のレシピのとおりに作る。
提供直前に仕上げとして、上等なバターで作った\protect\hyperlink{}{ブール・ド・ノワゼット}75
g を加える。

\ldots{}\ldots{}ポシェ\footnote{pocher
  沸騰しない程度の温度で茹でること。魚の場合は\protect\hyperlink{}{クールブイ
  ヨン}を用いてやや低めの温度で火を通すこと。}したサーモン、トラウトにとてもよく合う。

\maeaki

\hypertarget{ux30ceux30ebux30deux30f3ux30c7ux30a3ux30fcux98a8ux30bdux30fcux30b9}{%
\subsubsection{ノルマンディー風ソース}\label{ux30ceux30ebux30deux30f3ux30c7ux30a3ux30fcux98a8ux30bdux30fcux30b9}}

\hypertarget{sauce-normande}{%
\paragraph{Sauce Normande}\label{sauce-normande}}

\index{そーす@ソース!のるまんてふう@ノルマンディ風---}
\index{のるまんていふう@ノルマンディ風!そーす@---ソース}
\index{sauce@sauce!normande@--- Normande}
\index{normande@normande!sauce@Sauce ---}

\protect\hyperlink{veloute-de-poisson}{魚料理用ヴルテ}\troisquarts{}
Lに\footnote{原書にはリットルの表記がないが、本書における標準的な仕上り量が
  1 Lであることと、文脈から訳者が補った。}、マッシュ ルームの茹で汁1
dlとムール貝の煮汁1 dl、舌びらめのフュメ\footnote{舌びらめの料理に合わせるソースであるために、舌びらめのアラなど
  が必然的に出るのを無駄にせず使うということだが、現代のレストランの
  厨房などではかえって無理が生じることになる。このレシピの通りに作る
  場合には何らかのオペレーション上の工夫が必要だろう。} 2 dlを加える。
レモン果汁少々と、とろみ付け用に卵黄5個を生クリーム2dlで溶いたものを加
える。強火で\deuxtiers{}量つまり約8 dlまで煮詰める。

布で漉し、クレーム・ドゥーブル\footnote{乳酸醗酵した濃い生クリーム。\protect\hyperlink{sauce-supreme}{ソース・シュプレー
  ム}訳注参照。}1 dlとバター125 gを加える。

\ldots{}\ldots{}このソースは\protect\hyperlink{sole-normande}{舌びらめのノルマンディ風}専用。とはい
え、使い方によっては無限の可能性がある。

\hypertarget{ux539fux6ce8-16}{%
\subparagraph{【原注】}\label{ux539fux6ce8-16}}

基本的に本書では、どんなレシピにおいても、牡蠣の煮汁は使わないことにし
ている。牡蠣の煮汁は塩味がするだけで風味がない。だから、可能であればムー
ル貝の煮汁を大さじ何杯か加えるほうがずっといい\footnote{このレシピは初版からの異同が大きい。初版では「魚料理用ヴルテ1
  Lあたり卵黄6個でとろみを付け、牡蠣の煮汁2 dlと魚のエッセンス、生ク
  リーム2 dlを加えながら煮詰める。仕上げにバター100gとクレーム・ドゥー
  ブル1 dlを加える」となっており、用途には触れられていない。第二版、
  第三版ではやや細かなレシピとなり用途も「舌びらめのノルマンディ風」
  と指定されて現行版に近いものになるが、牡蠣の煮汁を使うことは初版と
  同じ。つまり、第四版で牡蠣の煮汁からムール貝の煮汁を使うことに変更
  し、この原注が付けられた。このソースにおける改変は、前出のソース・
  ラギピエールのケースとやや似ているところもある。牡蠣を用いることか
  ら、牡蠣の産地であるノルマンディ風という名称となったソースであるの
  に、そこから牡蠣を排除するという、いわば換骨奪胎がなされているから
  だ。とはいえ、このことが、第四版の改訂にエスコフィエ自身が携わった
  という証拠のひとつともなり得る可能性はある。初版刊行時56才、1921年
  刊の第四版の改訂にあたった頃には70才を過ぎていたことになり、味覚や
  嗅覚における感受性に変化があった可能性も考えられる。第三版までは牡
  蠣の煮汁を指定したいたのに、第一次大戦後、食料事情の変化があったと
  はいえ、きわめて風味の強いムール貝の煮汁を使うことを第四版で唐突に
  推奨しているということからは、まったくの第三者による改竄か、改訂者
  本人の身体的、感覚的もしくは思想的な変化がうかがわれる。その意味で
  も、やはりエスコフィエ自身が改訂作業に真摯に取り組んだ結果として、
  このレシピの変遷を捉えるべきだろう。}。

\maeaki

\hypertarget{ux30aaux30eaux30a8ux30f3ux30c8ux98a8ux30bdux30fcux30b9108}{%
\subsubsection[オリエント風ソース]{\texorpdfstring{オリエント風ソース\footnote{フランス語の
  orient オリヨン(東方)は、具体的にいうと北アフリ
  カの一部、アラビア半島、西アジアくらいまでを指すのが一般的。その意
  味では、カレー粉を加えたことで「オリエント風」と称するのは、当時の
  フランス人にとって、理解できなくもないだろうが実感は伴わなかった可
  能性がある。フランス人にとっての「オリエント」である北アフリカやト
  ルコといった地域の食文化は19世紀に既にかなりフランスに伝わっていた
  からだ。つまりは、ロンドンのカールトンホテルとパリのオテルリッツの
  それぞれで、もし仮にこのソースを添えた料理の名をメニューで見たとき、
  食べ手すなわち客が受ける印象はかなり異なる可能性が高い。もちろん、
  これらのパレスホテルがインターナショナルな社交の場として機能してい
  たということを考慮に入れても、同じ料理名がイメージさせる内容には確
  実にずれが生じると考えるのだ妥当だろう。こういった文化的なイメージ
  のずれは、エスコフィエ本人が料理長としてのキャリアの大半をイギリス
  で過ごしたこととも関係があるとだろう。つまり、フランス人にとっての
  「オリエント」とインドという植民地を持つイギリス人の「オリエント」
  は同じ言葉であっても、想起される具体的な内容が違うということである。
  ちなみに、インドより東、たとえば日本などはextrème orientエクストレー
  モリヨン(極東)と呼ばれる。}}{オリエント風ソース}}\label{ux30aaux30eaux30a8ux30f3ux30c8ux98a8ux30bdux30fcux30b9108}}

\hypertarget{sauce-orientale}{%
\paragraph{Sauce Orientale}\label{sauce-orientale}}

\index{そーす@ソース!おりえんとふう@オリエント風---}
\index{おりえんとふう@オリエント風!そーす@---ソース}
\index{とうほうふう@東方風!おりえんたるそーす@オリエント風ソース}
\index{sauce@sauce!orientale@--- Orientale}
\index{oriental@oriental!sauce@Sauce Orientale}

\protect\hyperlink{sauce-americaine}{ソース・アメリケーヌ}\undemi{}
Lを用意し、カレー粉
で風味付けをして\deuxtiers{}量まで煮詰める。鍋を火から外し、生クリーム
1\undemi{} dlを混ぜ込む。

\ldots{}\ldots{}このソースの用途は\protect\hyperlink{sauce-americaine}{ソース・アメリケーヌ}と同じ。

\maeaki

\hypertarget{ux30ddux30fcux98a8ux30bdux30fcux30b9}{%
\subsubsection{ポー風ソース}\label{ux30ddux30fcux98a8ux30bdux30fcux30b9}}

\hypertarget{sauce-paloise110}{%
\paragraph[Sauce paloise]{\texorpdfstring{Sauce paloise\footnote{ポーは15世紀以来、ベアルヌ地方の中心都市。}}{Sauce paloise}}\label{sauce-paloise110}}

\index{そーす@ソース!ほーふう@ポー風---}
\index{ほーふう@ポー風!そーす@---ソース}
\index{sauce@sauce!paloise@--- Paloise}
\index{palois@palois!sauce@Sauce Paloise}
\index{pau@Pau!sauce paloise@Sauce Paloise}

\protect\hyperlink{sauce-bearnaise}{ソース・ベアルネーズ}を本書に書いてあるとおりの方法
と分量で用意する(\protect\hyperlink{sauce-bearnaise}{ソース・ベアルネーズ}参照)が、以
下の点を変える。

\begin{enumerate}
\def\labelenumi{\arabic{enumi}.}
\item
  香りの中心となるエストラゴンを同量のミント\footnote{フランス料理よりはむしろイギリス料理でよく使われるミントを用い
    たこのソースをポー風と呼ぶのは、かつてこの地がイギリス貴族たちに保
    養地として好まれたことにちなんでいるという説もある。}に変更し、白ワインとヴィネガーを煮詰める際に加える。
\item
  さらに、仕上げの際に、細かく刻んだエストラゴンも使わない。細かく刻んだミントを使う。
\end{enumerate}

\ldots{}\ldots{}このソースの用途はソース・ベアルネーズとまったく同じ。

\maeaki

\hypertarget{ux30bdux30fcux30b9ux30d7ux30ecux30c3ux30c8109}{%
\subsubsection[ソース・プレット]{\texorpdfstring{ソース・プレット\footnote{ひな鶏、の意。かつて鶏のフリカセがこのソースと同様の作り方であっ
  たためこの名称になったという説もある。}}{ソース・プレット}}\label{ux30bdux30fcux30b9ux30d7ux30ecux30c3ux30c8109}}

\hypertarget{sauce-poulette}{%
\paragraph{Sauce Poulette}\label{sauce-poulette}}

\index{そーす@ソース!ふれっと@---・プレット}
\index{ふれっと@プレット!そーす@ソース・---}
\index{sauce@sauce!poulette@--- Poulette}
\index{poulette@poulette!sauce@Sauce ---}

マッシュルームの茹で汁2
dlを\untiers{}量まで煮詰める。ここに\protect\hyperlink{sauce-allemande}{ソース・
アルマンド}1 Lを加え、数分間沸騰させる。鍋を火から外
し、レモン果汁少々とバター60g、パセリのみじん切り大さじ1杯を加えて仕上
げる。

\ldots{}\ldots{}このソースは野菜料理に合わせるが、羊の足の料理にもよく合う。
\end{recette}