\hypertarget{sauces}{%
\chapter{I. ソース Sauces}\label{sauces}}

\hypertarget{les-fonds-de-cuisine}{%
\section{フォン、その他のストック}\label{les-fonds-de-cuisine}}

\frsec{Les Fonds de Cuisine}

\index{fonds@fonds} \index{ふおん@フォン}

\normalsize
\setstretch{1.0}

本書は実際に厨房で働く料理人を対象としたものだが、まず最初に料理のベー
スとして仕込んでストックしておくもの\footnote{本書での fonds の語は fond
  (基礎、土台)、fonds (資産、資
  本)、そして料理用語として一般に用いられているフォン、のトリプルミー
  ニングになっている。そのまま「フォン」と訳したいところだが、日本語
  の場合「出汁」としての意味合いが強いため、本文中では分りやすさを重
  視してやや冗長に「料理のベースとして仕込んでストックしておくもの」
  のように訳している。}について少々述べておきた い\footnote{この部分は経営者に向けて書かれているようにも読めるが、エス
  コフィエの時代以降、料理人がオーナーシェフとして経営に携わるケース
  が激増したことを考えると、その先見の明に驚かざるを得ない。}。我々料理人にとって重要なものだからだ。

ここで述べる料理のベースとして仕込んでストックしておくものは、実際、料
理の土台そのものであり、それなしでは美味しい料理を作ることの出来ない、
まず最初に必要なものだ。だからこそ、料理のベースとして仕込んでおくストッ
クはとても重要であり、いい仕事をしたいと努めている料理人ほどこれらを重
視している。

これらは、料理において常に立ち戻るべき出発点となるものだが、料理人がい
い仕事をしたいと望んでも、才能があっても、それだけでいいものを作ること
は出来ない。料理のベースを作るにも材料が必要なのだ。だから、必要な材料
は良質のものを自由に使えるようにしなければならない。

筆者としては、むやみな贅沢には反対だが、それと同じくらい、食材コストを
抑え過ぎるのも良くないと考えている。そんなことをしていては、伸びる筈の
才能の芽を摘んでしまうばかりか、意識の高い料理人ならモチベーションの維
持すら出来ないだろう。

どんなに優秀な料理人だって、無から何かを作り出すことは不可能だ。期待さ
れる結果に対して、素材の質が劣っていたり量が足りないことがあれば、それ
でも料理人にいい仕事をしろと要求するなど言語道断である。

料理のベースとして仕込んでおくストックに関するの重要ポイントは、必要な
材料は質、量ともに充分に、惜しげもなく使えるようにすることだ。

ある調理現場で可能なことが、別の調理現場では不可能な場合があるのは言う
までもない。料理人の仕事内容は顧客層によっても変わる。到達すべき目標に
よって手段も変わるということだ。

そういう意味で、何事も相対的なものであるとはいえ、こと料理のベースとし
て仕込んでストックすべきものに関しては絶対に外してはならないポイントが
あるわけだ。組織のトップがこの点で出費を惜しんだり、コスト面で過度に目
くじらを立てるようでは、美味しい料理なんて出来るわけがないのだから、現
実に厨房を仕切っている料理長を批判する資格もない。そんなのが根拠のない
言い掛かりなのは明らかだ。素材の質が悪かったり、量が足りないのであれば、
料理長が素晴しい料理を出せないのは言うまでもあるまい。ぶどうの搾りかす
に水を加えて醗酵させた安ワインを立派な瓶に詰めてしまえば高級ワインにな
ると思う程に馬鹿げたことはないのだ。

料理人は、必要なものを何でも使っていいなら、料理のベースとして仕込んで
おくストックにとりわけ力を入れるべきであり、文句のつけようのない出来に
なるよう気を使うべきだ。そこに手間隙かけていればそれだけ厨房全体の仕事
がきちんと進むのだから、注文を受けた料理をきちんと作れるかどうかは、結
局のところ、料理のベースとなる仕込み類にどれだけ手間\ruby{隙}{ひま}をかけるかという
ことなのだ。

\newpage

\hypertarget{principaux-fonds-de-cuisine}{%
\section{主要なフォンとストック}\label{principaux-fonds-de-cuisine}}

\frsec{Principaux Fonds de Cuisine}

料理のベースとして仕込んでおくべきものは主として\ldots{}\ldots{}

\begin{itemize}
\tightlist
\item
  \textbf{コンソメ・サンプルとコンソメ・ドゥーブル}
\item
  \textbf{茶色いフォン、白いフォン、鶏のフォン、ジビエのフォン、魚のフォン}\ldots{}\ldots{}これらはとろみを付けたジュ、基本ソースのベースになる
\item
  \textbf{フュメ、エッセンス}\ldots{}\ldots{}派生ソースに用いる
\item
  \textbf{グラスドヴィアンド、鶏のグラス、ジビエのグラス}
\item
  \textbf{茶色いルー、ブロンドのルー、白いルー}
\item
  \textbf{基本ソース}\ldots{}\ldots{}エスパニョル、ヴルテ、ベシャメル、トマト
\item
  \textbf{肉料理用ジュレ、魚料理用ジュレ}
\end{itemize}

\vspace{1\zw}

以下も日常的に使う料理のベースとして仕込んでおくものとして扱う。

\begin{itemize}
\tightlist
\item
  \textbf{ミルポワ、マティニョン}
\item
  \textbf{クールブイヨン、肉および野菜用のブラン}
\item
  \textbf{マリナード、ソミュール}
\item
  \textbf{肉料理用ファルス、魚料理用ファルス}
\item
  \textbf{ガルニチュールに用いるアパレイユ}、など\ldots{}\ldots{}
\end{itemize}

\vspace{1\zw}

本書は上記を順に説明していく構成にはなっていない。グリル、ロースト、グ
ラタン等の調理技法についても順を追っていくわけではない。料理の種類ごと
に一定の位置、つまりは関連の深い料理の章の冒頭において説明していくこと
になる。

\vspace{1\zw}

そのようなわけで、本書においては以下のようになる\ldots{}\ldots{}

\begin{itemize}
\tightlist
\item
  フォン、フュメ、エッセンス、グラス、マリナード、ジュレの説明\ldots{}\ldots{}
  \textbf{ 第1章 ソース}
\item
  コンソメおよびそのクラリフィエ、ポタージュの浮き実についての説
  明\ldots{}\ldots{}\textbf{第3章 ポタージュ}
\item
  ファルスとガルニチュール用アパレイユの作り方\ldots{}\ldots{}\textbf{第2章
  ガルニチュー ル}
\item
  クールブイヨン、魚料理用ファルス等\ldots{}\ldots{}\textbf{第6章
  魚料理}
\item
  グリル、ブレゼ、 ポワレの調理理論\ldots{}\ldots{}\textbf{第7章 肉料理}
\end{itemize}

\newpage

\hypertarget{grandes-sauces-de-base}{%
\section{基本ソース}\label{grandes-sauces-de-base}}

\frsec{Grandes Sauces de Base}

\index{そーす@ソース!きほん@基本---}
\index{sauce@sauce!00grandes@*Grandes ---s de Base}

\begin{itemize}
\item
  \textbf{およびそれらを組み合せたり煮詰めるなどの方法で作る派生ソース}
\item
  \textbf{イギリス風ソース(温製および冷製)}
\item
  \textbf{いろいろな冷製ソース}
\item
  \textbf{ブール・コンポゼ(ミックスバター)}
\item
  \textbf{マリナード}
\item
  \textbf{ジュレ}
\end{itemize}

\hypertarget{osbservationsurlasauce}{%
\section{概説}\label{osbservationsurlasauce}}

ソースは料理においてもっとも主要な位置にある。フランス料理が世界に冠た
るものであるのもひとえにソースの存在によるのだ。だから、ソースは出来る
かぎり手間をかけ、細心の注意を払って作るようにしなければならない。

ソースを作るうえでその基礎となるのが何らかの「ジュ」である\footnote{ここではジュといわゆるフォンが同じ意味で使われている。}。すなわ
ち、茶色いソースは「茶色いジュ」(エストゥファード)から作る。ヴルテ
には「澄んだジュ(白いフォン\footnote{日本の調理現場で「白いフォン」を意味する「フォン・ブラン」は主と
  して鶏のフォンを指すことが多いが、本書で扱われている白いフォンのう
  ち標準的なものは仔牛肉、家禽類をベースとしており、鶏のフォンは別途
  説明されている。})を使う。ソースを担当する料理人はまず
第一に、完璧なジュを作るところから始めなければならない。キュシー侯爵
\footnote{1767-1841。19世紀の著名な美食家。
  著書に『食卓の古典』(1843)があ
  る。料理名にキュシーの名を冠したものも多い。}が言うように、ソース担当の料理人は「頭脳明晰な化学者\footnote{原文
  chimiste。現代は分子ガストロノミーが盛んだが、料理を
  作る過程で起きる現象や結果を「化学」で説明しようとする試みは少なく
  ともカレームまで遡ることが出来る。\protect\hyperlink{fonds-brun}{茶色いフォン}のレシピにおいて言
  及されるオスマゾームという想像上の物質もその範疇に含まれるだろう。
  また、化学の前身たる「錬金術」的概念は中世以来いくつかの料理書にお
  いて散見される。}でありかつ天才
的なクリエイターで、卓越した料理という建造物のいわば大黒柱たる存在」な
のだ。

昔のフランス料理\footnote{本書において「昔の料理」と表現される場合は概ね17〜18世紀末と考え
  ていい。}では、素材に串を刺してあぶり焼きするローストを別に
すれば、どんな料理も「ブレゼ」か「エチュヴェ」のようなものばかりだった。
だが、その時代には既に、フォンが料理という大建築の丸天井の\ruby{要}{か
なめ}だったし、材料コストが重視されるこんにちの我々と比べたら想像も出
来ないくらい贅沢に材料を使ってフォンをとっていたのだ。実際、アンヌ・ドー
トリッシュ\footnote{17世紀に絶対王政を確立したルイ14世の母。}がスペインからルイ13世に嫁いだ際に随行してきたスペインの
料理人たちによってフランス料理にルーを用いる方法が伝えられたが\footnote{ルーがスペインからもたらされたというのは逸話、伝承の域を出ない。}、当時は
ほとんど看過された。ジュそれ自体で充分だったからだ。ところが時代が下り、
料理におけるコストの問題が重視されるようになった。ジュはその結果、貧相な
ものになってしまった。その美味しさを補うものとして、ルーを用いて作るソー
ス・エスパニョルが欠くべからざる存在となった。

ソース・エスパニョルはその完成度の高さゆえに成功をおさめたわけだ。だが、
すぐに当初の目的を越えた使い方をされるようになった。19世紀末には本当に
このソースが必要な場合以外にも使われたわけだ。ソース・エスパニョルの濫
用によって、どんな料理も固有の香りのない、全部の風味の混ざりあったのっ
ぺりとした調子のものばかりになってしまった。

ようやく近年になって、料理の風味がどれも同じようなものであることに批判
が集まってきて、その結果として激しい揺り戻しが起きたのだった。グランド
キュイジーヌでは、透き通ったような薄い色合いでしかも風味のしっかりした
仔牛のフォンが見直されつつある。そのようなわけで、ソース・エスパニョル
それ自体の重要性はだんだん減っていくだろうと思われる。

ソース・エスパニョルが基本ソースとして扱われるべき理由は何か? ソース・
エスパニョルそれ自体に固有の色合いや風味というものはなく、これらはどん
なフォンを用いて作るかで決まる。まさにこの点にソース・エスパニョルの長
所が存するのだ。補助材料としてルーを加えるが、ルーにはとろみを付けると
いう意味しかなく、風味にはまったく寄与しない。そもそも、ソースを完璧に
仕上げるためには、とろみ以外のルーに含まれる成分はソースからほぼ完全に
取り除いてしまっても差し支えはない。不純物を丁寧に取り除いたソースには
ルーに含まれていたでんぷん質だけが残っているわけだ。だから、ソースの口
あたりを滑らかなものにするために必要なのがでんぷん質だけなら、純粋なで
んぷんだけを用いる方がずっと簡単で、作業時間も大幅に短縮されるし、その
結果として、ソースを火にかけ過ぎてしまうようなミスも防げる。将来的には、
小麦粉ではなく純粋なでんぷんでルーを作るようになるかも知れない。

料理界の現状を\ruby{鑑}{かんが}みるに、\textbf{ソース・エスパニョル}と\textbf{と
ろみを付けたジュ}をそれぞれ使い分けざるを得ない。これにはさまざまな理
由があるが、大きな仕立てのブレゼや、羊や仔羊以外を材料にしたラグーでは、
肉汁が煮汁に染み出してきて美味しくなるわけだから、トマトを加えたソース・
エスパニョルを用いるのがいい。なお、ソース・エスパニョルをさらに丁寧に
仕上げるとソース・ドゥミグラスとなる。これはいろいろなソテーに不可欠な
もので、今後も変わることはないだろう。

一方、牛や羊、家禽を使った繊細で軽い仕立ての料理にはとろみを付けたジュ
の方が好まれる。デグラセの際に少量だけ、料理の主素材と同じものからとっ
たジュを用いる。

こんにちのフランス料理においては、肉とソースの調和がとれているべきとい
う、まことに理に適った厳守すべき決まりがある。

だから、ジビエ料理にはジビエのフォンを用いるか、とりたてて際立った個性
を持たないフォンを用いて作ったソースを添える。牛や羊のフォンは用いない。
ジビエのフォンというのは、さほど濃厚なものを作ることは出来ないが、素材
の個性的な風味を表現するには最適だ。こういった事情は魚料理にも当て
\ruby{嵌}{はま}る。ソースそれ自体が際だった風味を持たないものの場合に
は必ず魚のフュメを加えてやるのだ。このようにしてそれぞれの料理に個性的
な風味を実現させることになる。

もちろん、ここまで述べた原則を実現しようにも、コストの問題がしばしば起
こることは承知している。けれども、熱意のある、他者の評価を意識している
料理人なら問題点を熟考して、完璧とは言わぬまでも満足のいく結果を得るこ
とが出来るだろう。\newpage

\normalsize
\setstretch{1.0}

\hypertarget{traitement-des-elements-de-base}{%
\section{ソースのベース作り}\label{traitement-des-elements-de-base}}

\frsec{Traitement des Éléments de Base dans le Travail des Sauces}

\index{そーす@ソース!そーすつくりのべーす@---のベース作り}
\index{sauce@sauce!Traitement des elements de base dans le travail des sauces@Traitement des Éléments de Base dans le Travail des ---s}
\begin{recette}
\hypertarget{fonds-brun}{%
\subsubsection{茶色いフォン(エストゥファード)}\label{fonds-brun}}

\frsub{Fonds brun ou Estouffade}

\index{ふおん@フォン!ちやいろいふおん@茶色い---}
\index{えすとうふあーと@エストゥファード}
\index{fonds@fonds!brun@--- brun}
\index{fonds@fonds!estouffade@estouffade (fonds brun)}
\index{estouffade@estouffade!fonds brun@ --- (fonds brun)}

(仕上がり10L分)

\begin{itemize}
\item
  \textbf{主素材}\ldots{}\ldots{}牛すね6
  kg、仔牛のすね6kgまたは仔牛の端肉で脂身を含まな いもの6
  kg、骨付きハムのすねの部分1本(前もって下茹でしておくこと)、
  塩漬けしていない豚皮を下茹でしたもの650 g。
\item
  \textbf{香味素材}\ldots{}\ldots{}にんじん650 g、玉ねぎ650
  g、ブーケガルニ(パセリの枝100 g、 タイム10 g、ローリエ5
  g、にんにく1片)。
\item
  \textbf{作業手順}\ldots{}\ldots{}肉を骨から外す。
\end{itemize}

骨は細かく砕き、オーブンに入れて軽く焼き色を付ける。野菜は焼き色が付く
まで炒める。これらを鍋に入れて14 Lの水を注ぎ、ゆっくりと、最低12時間煮
込む。水位が下がらぬように、適宜沸騰した湯を足すこと。

大きめのさいの目に切った牛すね肉を別鍋で焼き色が付くまで炒める。先に煮
込んでいたフォンを少量加えて煮詰める。この作業を2〜3回行ない、フォン
の残りを注ぐ。

鍋を沸騰させて、浮いてくる泡を取り除く。浮き脂も丁寧に取り除く。蓋をし
て弱火で完全に火が通るまで煮込んだら、布で漉してストックしておく。

\hypertarget{nota-fonds-brun}{%
\subparagraph{【原注】}\label{nota-fonds-brun}}

フォンの材料に牛の骨などが含まれている場合には、事前にその骨だけで12〜
15時間かけてとろ火でフォンをとるといい。

フォンの材料を鍋に焦げ付くくらいまで強く焼き色を付ける\footnote{パンセ
  pincer と呼ばれる手法。原義は「抓む」。材料が鍋底に張り付
  いて、トングなどでしっかり「抓ま」ないと取れないくらい強く焼き付け
  ることからそう呼ばれるようになった。古い料理書では推奨するものも多
  かった。}のはよろしく
ない。経験からいって、丁度いい色合いのフォンに仕上げるには、肉に含まれてい
るオスマゾーム\footnote{19世紀頃、赤身肉の美味しさの本質であると考えられていた想像上の物
  質。赤褐色をした窒素化合物の一種で水に溶ける性質があるとされた。な
  お、当時のヨーロッパではグルタミン酸はもとよりイノシン酸が「うま味」
  の要素であるという概念すらなく、「コクがある」corsé とか「肉汁たっ
  ぷり」onctueux (オンクチュー)や succulent
  (スュキロン)などの表現で肉料理やソースの美味しさが表 現された。}の働きだけで充分だ。

\maeaki

\hypertarget{fonds-blanc}{%
\subsubsection{白いフォン}\label{fonds-blanc}}

\frsub{Fonds blanc ordinaire}

\index{ふおん@フォン!しろい@白い---}
\index{fonds@fonds!blanc ordinaire@--- blanc ordinaire}

(仕上がり10 L分)

\begin{itemize}
\item
  \textbf{主素材}\ldots{}\ldots{}仔牛のすね、および端肉10k
  g、鶏の手羽やとさか、足など、ま たは鶏がら4羽分、
\item
  \textbf{香味素材}\ldots{}\ldots{}にんじん800 g、玉ねぎ400
  g、ポワロー300 g、セロリ100 g、ブー ケガルニ(パセリの枝100
  g、タイム1枝、ローリエの葉1枚、クローブ4本)。
\item
  \textbf{使用する液体と味付け}\ldots{}\ldots{}水12 L、塩60 g。
\item
  \textbf{作業手順}\ldots{}\ldots{}肉は骨を外し、紐で縛る。骨は細かく砕く。鍋に肉と骨を入
  れ、水を注ぎ塩を加える。火にかけ、浮いてくるアクを取り除き香味素材を加
  える。
\item
  \textbf{加熱時間}\ldots{}\ldots{}弱火で3時間。
\end{itemize}

\hypertarget{nota-fonds-blanc}{%
\subparagraph{【原注】}\label{nota-fonds-blanc}}

このフォンは火加減を抑えて、出来るだけ澄んだ仕上がりにすること。アクや
浮き脂は丁寧に取り除くこと。

茶色いフォンの場合と同様に、始めに細かく砕いた骨だけを煮てから指定量の
水を注ぎ、弱火で5時間煮る方法もある。

この骨を煮た汁で肉を煮るわけだ。その作業内容は上記茶色いフォンの場合と
同様。この方法は、骨からゼラチン質を完全に抽出出来るという利点がある。
当然のことだが、煮ている間に蒸発して失なわれてしまった分は湯を足してや
り、全体量を12Lにしてから肉を煮ること。

\maeaki

\hypertarget{fonds-de-volaille}{%
\subsubsection{鶏のフォン(フォンドヴォライユ)}\label{fonds-de-volaille}}

\frsub{Fonds de volaille}

\index{ふおん@フォン!とりのふおん@鶏の---}
\index{fonds@fonds!volaille@--- de volaille}
\index{かきん@家禽!とりのふおん@鶏のフォン}
\index{うおらいゆ@ヴォライユ!ふおんとうおらいゆ@フォンドヴォライユ}

白いフォンと同じ主素材、香味素材、水の量で、さらに鶏のとさかや手羽、ガ
ラを適宜増量し、廃鶏3羽を加えて作る。

\maeaki

\hypertarget{jus-de-veau-brun}{%
\subsubsection{仔牛の茶色いフォン(仔牛の茶色いジュ)}\label{jus-de-veau-brun}}

\frsub{Fonds, ou Jus de veau brun}

\index{ふおん@フォン!こうしのちやいろい@仔牛の茶色い---}
\index{しゆ@ジュ!こうしのちやいろいしゆ@仔牛の茶色い---}
\index{fonds@fonds!fonds de veau brun@--- de veau brun}
\index{jus@jus!jus de veau brun@--- de veau brun}
\index{こうし@仔牛!こうしのちやいろいふおん@---の茶色いフォン(ジュ)}
\index{veau@veau!fonds brun@fonds ou jus de --- brun}

(仕上がり10 L分)

\begin{itemize}
\item
  \textbf{主素材}\ldots{}\ldots{}骨を取り除いた仔牛のすね肉と肩肉(紐で縛っておく)6kg、
  細かく砕いた仔牛の骨5 kg。
\item
  \textbf{香味素材}\ldots{}\ldots{}にんじん600 g、玉ねぎ400
  g、パセリの枝100 g、ローリエの葉 2枚、タイム2枝。
\item
  \textbf{使用する液体}\ldots{}\ldots{}白いフォンまたは水12
  L。水を用いる場合は1 Lあたり3 gの塩を加える。
\item
  \textbf{作業手順}\ldots{}\ldots{}厚手の片手鍋または寸胴鍋の底に輪切りにしたにんじんと玉
  ねぎを敷きつめる。その他の香味素材と、あらかじめオーブンで焼き色を付けておい
  た骨と肉を鍋に加える。
\end{itemize}

蓋をして約10分間、蓋をして弱火にかけた野菜から水分が汗をかくように出る
イメージで蒸し焼き状態にし、素材の味を引き出す\footnote{suer
  (スュエ)シュエ。}。フォンま
たは水少量を加え、煮詰める。この作業をさらに1〜2回行なう。残りのフォ
ンまたは水を注ぎ、蓋をし、沸騰させる。アクを丁寧に取る。微沸騰の状
態で6時間煮る。

布で漉し、ストックしておく。使用目的や必要に応じて、さらに煮詰めてから
ストックしてもいい。

\maeaki

\hypertarget{fonds-de-gibier}{%
\subsubsection{ジビエのフォン}\label{fonds-de-gibier}}

\frsub{Fonds de gibier}

\index{ふおん@フォン!しひえ@ジビエの---}
\index{fonds@fonds!fonds de gibier@--- de gibier}
\index{しひえ@ジビエ!ふおん@---のフォン}
\index{gibier@gibier!fonds@fonds de ---}

(仕上がり5 L分)

\begin{itemize}
\item
  \textbf{主素材}\ldots{}\ldots{}ノロ鹿の頸、胸肉および端肉3
  kg(老いたノロ鹿がいいが、新 鮮なものを使うこと)、野うさぎ\footnote{lièvre
    (リエーヴル)。}の端肉1 kg、老うさぎ2羽、山うずら2羽、 老きじ1羽。
\item
  \textbf{香味素材}\ldots{}\ldots{}にんじん250 g、玉ねぎ250
  g、セージ1枝、ジュニパーベリー \footnote{セイヨウネズの樹の実。}15粒、標準的なブーケガルニ。
\end{itemize}

\begin{itemize}
\item
  \textbf{使用する液体}\ldots{}\ldots{}水6 Lおよび白ワイン1瓶。
\item
  \textbf{加熱時間}\ldots{}\ldots{}3時間。
\item
  \textbf{作業手順}\ldots{}\ldots{}ジビエは事前にオーブンで焼き色を付けておき、野菜と香草を
  敷き詰めた鍋に入れる。野菜類も事前に焼き色を付けておくこと。ジビエを焼
  くのに用いた天板を白ワインでデグラセし、これを鍋に注ぐ。同量の水も加え、
  ほぼ水分がなくなるまで煮詰める。
\end{itemize}

この作業の後で、残りの水全量を注ぎ、沸騰させる。丁寧にアクを引きながら
ごく弱火で煮る\footnote{最後に布で漉す必要があるが、当然のこととして明記されていないの
  で注意。}。

\maeaki

\hypertarget{fumet-de-poisson}{%
\subsubsection{魚のフュメ(フュメドポワソン)}\label{fumet-de-poisson}}

\frsub{Fonds, ou Fumet de poisson}\footnote{本質的には前出の「フォン」と同様のものだが、魚(およびジビエ)
  を素材としたフォンは香りがポイントとなるため、フュメ fumet (香気、
  良い香りの意)の名称のほうが一般的に使われている。}

\index{ふおん@フォン!さかな@魚の---}
\index{ふゆめ@フュメ!さかな@魚の---}
\index{ふゆめ@フュメ!ほわそん@フュメドポワソン}
\index{fumet@fumet!fumet de poisson@--- de poisson}
\index{fonds@fonds!fumet de poisson@fumet de poisson}

(仕上がり10L分)

\begin{itemize}
\item
  \textbf{主素材}\ldots{}\ldots{}舌びらめ、メルラン\footnote{タラの近縁種。}やバルビュ\footnote{ヒラメの近縁種。}のあら10
  kg。
\item
  \textbf{香味素材}\ldots{}\ldots{}薄切りにした玉ねぎ500
  g、パセリの根\footnote{パセリには根がにんじん形に肥大する品種もある(persil
    tubéreux 根パセリ。葉は平らでイタリアンパセリのように使う)。}と茎100
  g、マッ シュルームの切りくず250g、レモンの搾り汁1個分、粒こしょう15
  g(これは フュメを漉す10分前に投入する)。
\item
  \textbf{使用する液体と調味料}\ldots{}\ldots{}水10
  L、白ワイン1瓶。液体1Lあたり3〜4 gの 塩。
\item
  \textbf{加熱時間}\ldots{}\ldots{}30分。
\item
  \textbf{作業手順}\ldots{}\ldots{}鍋底に香味野菜を敷き詰め、魚のあらを入れる。水と白ワイ
  ンを注ぎ、強火にかける。丁寧にアクを引き、微沸騰の状態を保つようにする。
  30分煮たら目の細かい網で漉す。
\end{itemize}

\hypertarget{nota-fumet-de-poisson}{%
\subparagraph{【原注】}\label{nota-fumet-de-poisson}}

質の悪い白ワインを使うと灰色がかったフュメになってしまう。品質の疑わし
いワインは使わないほうがいい。

このフュメはソースを作る際に加える液体として用いる。魚料理用ソース・エ
スパニョルを作ることを想定する場合には、魚のあらをバターでエチュベして
から水と白ワインを注いで煮るといい。

\maeaki

\hypertarget{fonds-de-poisson-au-vin-rouge}{%
\subsubsection{赤ワインを用いた魚のフォン}\label{fonds-de-poisson-au-vin-rouge}}

\frsub{Fonds de poisson au vin rouge}

\index{ふおん@フォン!あかわいんをもちいたさかなのふおん@赤ワインを用いた魚の---}
\index{fonds@fonds!fonds de poisson au vin rouge@--- de poisson au vin rouge}

このフォンそれ自体を用意することは滅多にない。というのも、例えばマトロッ
トのような料理の魚の煮汁そのものだからだ。

とはいえ、こんにちでは魚のアラをすっかり取り除いた状態で料理を提供する
必要がますます高まってきているので、ここでそのレシピを記しておくべきだ
ろう。このフォンの必要性と有用さはどんどん高まっていくと思われる。

原則として、このフォンの仕込みには、料理として提供するのと同じ種類の魚
のアラを用いて、その香りの特徴を生かす必要がある。だが、どんな種類の魚
を使う場合でも作り方は同じだ。

(仕上がり5 L分)

\begin{itemize}
\item
  \textbf{主素材}\ldots{}\ldots{}料理に用いるのと同じ魚種の頭とアラ2.5
  kg。
\item
  \textbf{香味素材}\ldots{}\ldots{}薄切りにして下茹でした玉ねぎ300
  g、パセリの枝100 g、タイ
  ムの小枝1本、小さめのローリエの葉2枚、にんにく5片、マッシュルームの切
  りくず100 g。
\item
  \textbf{使用する液体と調味料}\ldots{}\ldots{}水3.5 L、良質の赤ワイン2
  L、塩15 g。
\item
  \textbf{加熱時間}\ldots{}\ldots{}30分。
\item
  \textbf{作業手順}\ldots{}\ldots{}「魚の白いフォン\footnote{前項のフュメドポワソンのこと。}」と同様にする。
\end{itemize}

\hypertarget{nota-fonds-de-poisson-au-vin-rouge}{%
\subparagraph{【原注】}\label{nota-fonds-de-poisson-au-vin-rouge}}

このフォンは魚の白いフォンよりも濃く煮詰めることが可能。とはい
え、保存のために煮詰めないでいいように、その都度、必要な量だけ仕込むこ
とを勧める。

\maeaki

\hypertarget{essence-de-poisson}{%
\subsubsection{魚のエッセンス}\label{essence-de-poisson}}

\frsub{Essence de poisson}

\index{えつせんす@エッセンス!さかな@魚の---}
\index{essence@essence!poisson@--- de poisson}

\begin{itemize}
\item
  \textbf{主素材}\ldots{}\ldots{}メルラン\footnote{タラの近縁種。}および舌びらめの頭、アラ2
  kg。
\item
  \textbf{香味素材}\ldots{}\ldots{}薄切りにした玉ねぎ125
  g、マッシュルームの切りくず300 g、 パセリの枝50
  g、レモンの搾り汁1個分。
\item
  \textbf{使用する液体}\ldots{}\ldots{}煮詰めていないフュメドポワソン1\undemi{}L、良質の白ワイ
  ン3dl。
\item
  \textbf{所要時間}\ldots{}\ldots{}45分。
\item
  \textbf{作業手順}\ldots{}\ldots{}鍋にバター100
  gと玉ねぎ、パセリの枝、マッシュルームの切
  りくずを入れ、強火で色づかないようさっと炒める。アラと端肉を加える。蓋をして約15分弱火で蒸
  し煮する\footnote{素材を入れた鍋に蓋をして弱火にかけ、少量の水分で蒸し煮状態にす
    ることを étuver エチュベという。このフランス語をそのまま用いている
    調理現場も少なくない。}。その間、小まめに混ぜてやること。白ワインを注ぎ、半量になるま
  で煮詰める。最後にフュメドポワソンを注ぎ、レモン汁と塩2 gを加える。
\end{itemize}

再び火にかけて、とろ火で15分程煮込んだら、布で漉す。

\hypertarget{nota-essence-de-poisson}{%
\subparagraph{【原注】}\label{nota-essence-de-poisson}}

魚のエッセンスは、舌びらめやチュルボ、チュルボタン、バルビュ\footnote{いずれも鰈、ひらめの近縁種。チュルボタンはチュルボの小さいもの
  を言う。} などのフィレ\footnote{3枚おろし、または5枚おろしにして、頭とアラを取り除いた状態。}をポシェする際に用いる。

さらに、このエッセンスを煮詰めて、上記でポシェした魚のソースに加えて風
味を強くするのに使う。

\maeaki

\hypertarget{essences-diverses}{%
\subsubsection{エッセンスについて}\label{essences-diverses}}

\frsub{Essences diverses}

\index{えつせんす@エッセンス!01えつせんすについて@---について(フォン)}
\index{essence@essence!01 diverses@---s diverses (fonds)}

その名のとおり、エッセンスとはごく少量になるまで煮詰めて非常に強い風味
を持たせたフォンのこと。

エッセンスは普通のフォンと本質的には同じものだが、素材の風味をしっかり
出すために、使用する液体の量はずっと少ない。したがって、仕上げにエッセ
ンスを加える指示がある料理の場合でも、そもそも充分に風味ゆたかなフォン
を用いていれば、エッセンスは必要ないことが分かるだろう。

まず最初に、美味しく風味ゆたかなフォンを用いるほうが、あまり出来のよく
ないフォンで調理し、後からエッセンスで欠点を補うよりもずっと簡単なのだ。
その方がいい結果が得られるし、時間と材料の節約にもなる。

セロリ、マッシュルーム、モリーユ\footnote{morille
  キノコの一種。和名アミガサタケ。}、トリュフなど、とりわけ明確な風
味の素材のエッセンスを、必要に応じて用いるにとどめるのがいい。

また、十中八九、フォンを仕込む際に素材そのものを加えた方が、エッセンス
を仕込むよりもいい結果が得られることは頭に入れておくこと。

そのようなわけで、エッセンスについてこれ以上長々と述べる必要もないと思
われる。ベースとなるフォンがコクと風味がゆたかなものならであるなら、エッ
センスはまったく無用の長物と言える。

\maeaki

\hypertarget{glaces-diverses}{%
\subsubsection{グラスについて}\label{glaces-diverses}}

\frsub{Glaces diverses}

\index{ぐらす@グラス!01ぐらすについて@---について}
\index{glace@glace!01 diverses@---s diverses}

グラスドヴィアンド、鶏のグラス(グラスドヴォライユ)、ジビエのグラス、
魚のグラスの用途は多岐にわたる。これらは、上記いずれかの素材でとったフォ
ンをシロップ状になるまで煮詰めたもののことだ。

これらの使い途は、料理の仕上げに表面に塗ってしっとりとした艶を出させる
のに用いる場合もあれば、ソースの味や色合いを濃くするために用いたり、あ
るいは、あまりに出来のよくないフォンで作った料理の場合にはコクを与える
ために使うこともある。また、料理によっては適量のバターやクリームを加え
てグラスそのものをソースとして用いることもある。

グラスとエッセンスの違いだが、エッセンスが料理の風味そのものを強くする
ことだけが目的であるのに対して、グラスは素材の持つコクと風味をごく少量
にまで濃縮したものだ。

だからほとんどの場合、エッセンスよりもグラスを使うほうがいい。

とはいえ昔の料理長たちの中には、グラスの使用を絶対に認めない者もいた。
その理由は、料理を作る度に毎回その料理のためのフォンをとるべきであり、
それだけで料理として充分なものにすべき、ということだった。

確かに時間と費用の点で制限がなければその理屈は正しい。だが、こんにちで
は、そのようなことの出来る調理現場はほとんどない。そもそもグラスは、正
しく適量を用いるのであれば、そのグラスが丁寧に作られたものであるならな、
素晴しい結果が得られる。 だから多くの場合、グラスはまことに有用なもの
と言える。

\maeaki

\hypertarget{glace-de-viande}{%
\subsubsection{グラスドヴィアンド}\label{glace-de-viande}}

\frsub{Glace de viande}

\index{くらす@グラス!くらすとういあんと@---ドヴィアンド}
\index{glace@glace!viande@--- de viande}

茶色いフォン(エストゥファード)を煮詰めて作る。

煮詰めて濃くなっていく途中、何度か布で漉して、より小さな鍋に移しかえて
いく。煮詰めている際に、丁寧にアクを引くことが、澄んだグラスを作るポイ
ント。

煮詰めている際には、フォンの濃縮具合に応じて、火加減を弱めていくこと。
最初は強火でいいが、作業の最後の方は弱火にしてゆっくり煮詰めてやること。

スプーンを入れてみて、引き上げた際に、艶のあるグラスの層でスプーンが覆
われ、しっかり張り付いているくらいが丁度いい。要するに、スプーンがグラ
スでコーティングされた状態になればいいということだ。

\hypertarget{nota-glace-de-viande}{%
\subparagraph{【原注】}\label{nota-glace-de-viande}}

色が薄くて軽い仕上がりのグラスが必要な場合には、茶色いフォンではなく、標
準的な仔牛のフォンを用いる。

\maeaki

\hypertarget{glace-de-volaille}{%
\subsubsection{鶏のグラス(グラスドヴォライユ)}\label{glace-de-volaille}}

\frsub{Glace de volaille}

\index{くらす@グラス!とり@鶏の---(---ドヴォライユ)}
\index{くらす@グラス!うおらいゆ@---ドヴォライユ}
\index{glace@glace!volaille@--- de volaille}

鶏のフォン(フォンドヴォライユ)を用いて、グラスドヴィアンドと同様にし
て作る。

\maeaki

\hypertarget{glace-de-gibier}{%
\subsubsection{ジビエのグラス}\label{glace-de-gibier}}

\frsub{Glace de gibier}

\index{くらす@グラス!しひえ@ジビエの---}
\index{glace@glace!glace de gibier@--- de gibier}
\index{しひえ@ジビエ!くらす@---のグラス}
\index{gibier@gibier!gibier@glace de ---}

ジビエのフォンを煮詰めて作る。ある特定のジビエの風味を生かしたグラスを
作るには、そのジビエだけでとったフォンを用いること。

\maeaki

\hypertarget{glace-de-poisson}{%
\subsubsection{魚のグラス}\label{glace-de-poisson}}

\frsub{Glace de poisson}

\index{くらす@グラス!さかな@魚の---}
\index{glace@glace!poisson@--- de poisson}

このグラスを用いることはあまり多くない。日常的な業務においては「魚のエッ
センス」を用いることが好まれる。そのほうが魚の風味も繊細になる。魚のエッ
センスで魚をポシェした後に煮詰めてソースに加える。
\end{recette}
\hypertarget{roux}{%
\section{ルー}\label{roux}}

\frsec{Roux}

\index{るー@ルー} \index{roux@roux}

ルーはいろいろな派生ソースのベースとなる基本ソースにとろみを付ける役目
を持つ。ルーの仕込みは、一見したところさほど重要に思われぬだろうが、実
際には正反対だ。丁寧に注意深く作業すること。

茶色いルーは加熱に時間がかかるので、大規模な調理現場では前もって仕込ん
でおく。ブロンドのルーと白いルーはその都度用意すればいい。

\maeaki
\begin{recette}
\hypertarget{roux-brun}{%
\subsubsection{茶色いルー}\label{roux-brun}}

\frsub{Roux brun}

\index{るー@ルー!ちやいろ@茶色い---} \index{roux@roux!brun@--- brun}

(仕上がり1 kg分)

\begin{enumerate}
\def\labelenumi{\arabic{enumi}.}
\tightlist
\item
  澄ましバター\ldots{}\ldots{}500 g
\item
  ふるった小麦粉\ldots{}\ldots{}600 g
\end{enumerate}

\maeaki

\hypertarget{cuisson-des-roux}{%
\subsubsection{ルーの火入れについて}\label{cuisson-des-roux}}

\index{るー@ルー!ひいれについて@---の火入れについて}
\index{roux@roux!cuisson@cuisson du ---}

加熱時間は使用する熱源の強さで変わってくる。だから数字で何分とは言えな
い。ただし、火力が強過ぎるよりは弱いくらいの方がいい。というのも、温度
が高すぎると小麦粉の細胞が硬化して中身を閉じ込めてしまい、そうなると後
でフォンなどの液体を加えた際に上手く混ざらず、滑らかなとろみの付いたソー
スにならない。乾燥豆をいきなり熱湯で茹でるのと同じようなことが起きるわ
けだ。低い温度から始めてだんだんと熱くしていけば、小麦粉の細胞壁がゆる
んで細胞中のでんぷんが膨張し、熱によって発酵状態の初期のようになる。こ
のようにして、でんぷんをデキストリンに変化させる\footnote{現代の科学的見地からすると必ずしも正確な記述ではないので注意。}。デキストリ
ンは水溶性の物質で、これが「とろみ」の主な要素なのだ。茶色いルーは淡褐
色の美しい色合いで滑らかな仕上がりにする。だまがあってはいけない。

ルーを作る際には必ず、澄ましバターを使うこと\footnote{初版〜第三版では「澄ましバターまたは充分に澄ましたグレスド
  マルミット」となっている。グレスドマルミットとは、コンソメなどを作
  る際に、浮いてくる油脂を取り除く必要があるが、それを捨てずにまとめ
  てから漉して澄ませたもののこと。基本的に獣脂と考えていい。なお、同
  時代の料理書 --- 例えばペラプラ『近代料理技術』(1935年)--- には、
  ルーを作るのにバターを使う必要はなく、グレスドマルミットで充分、と
  しているものもある。}。 生のバターに
は相当量のカゼインが含まれている。カゼインがあると火を均質に通すことが
出来なくなってしまう。とはいえ、以下を覚えておくといい。ソースとして仕
上げた段階で、ルーで使ったバターは風味という点ではほとんど意味が失なわ
れている。そもそもソースの仕上げに不純物を取り除く\footnote{dépouiller
  デプイエ。ソースや煮込み料理を仕上げる際に、浮
  き上がってくる不純物を徹底的に取り除き、目の細かい布などで漉すこと。
  原義は動物などの皮を剥ぐ、剥くことの意で、野うさぎの皮を剥ぐ、うな
  ぎの皮を剥く、という意味で現代の厨房でも用いられているる。ソースの
  場合は表面に凝固した蛋白質や油脂の膜が出来、それを「剥ぐように」取
  り除くことから、あるいは表面に浮いてくる不純物を徹底的に取り除いて
  きれいなソースに仕上げることを、動物の皮を剥いてきれいな身だけにす
  ることになぞらえて、この用語が用いられるようになったようだ。なお、
  本書においてécumer(エキュメ)が単に浮いてくる泡やアクを取る、とい
  う作業であるのに対して、dépouiller(デプイエ)は「徹底的に不純物を
  取り除いて美しく仕上げる」という意味合いが込められている。現代では
  品種改良や農法の変化によって野菜のアクも少なくなり、小麦粉も精製度
  の高いものを利用出来るなど、食材および調味料の多くで純度の高いもの
  を使用する場合がほとんどであり、このデプイエという作業は20世紀後半
  にはほとんど行なわれなくなり、écumer(エキュメ)という用語だけで済ませ
  ることがほとんど(cf.辻静雄監訳『オリヴェ ソースの本』柴田書店、
  1970年、27〜28頁)。}段階でバター
も完全に取り除かれてしまうわけだ。だからルーに用いるバターは小麦粉に熱
を通すためだけのものと考えていい。

ルーはソース作りの出発点だ。だから次の点も記憶に\ruby{留}{とど}めるこ
と。小麦粉にでんぷんが含まれているからこそソースに「とろみ」が付く。だ
から純粋なでんぷん(特性が小麦のでんぷんと同じでも異なったものでも)で
ルーを作っても、小麦粉の場合と同様の結果が得られるだろう。ただしその場
合は小麦粉でルーを作る場合より注意して作業する必要がある。また、小麦粉
と違って余計な物質が含まれていないために、全体の分量比率を考え直すこと
になる。

\hypertarget{nota-roux}{%
\subparagraph{【原注】}\label{nota-roux}}

本文で述べたように、茶色いルーを作る際には澄ましバターを用いる。他の動
物性油脂はよほど経済的事情が逼迫していない限り使わないこと。材料コスト
が問題になる場合でも、ソースの仕上げに不純物を取り除く際に多少の注意を
払えば、ルーに用いたバターを回収するのはさして難しいことではない\footnote{既に述べたように初版〜第三版まではバターまたはグレスドマル
  ミットという指示であったことに留意する必要はあるだろう。実際のとこ
  ろ、良質のバターを用いてルーを作ったほうが、軽やかな仕上りのソース
  になる傾向があることは言うまでもない。}。それ
を後で他の用途で使えばいいだろう。

\maeaki

\hypertarget{roux-blond}{%
\subsubsection{ブロンドのルー}\label{roux-blond}}

\frsub{Roux blond}

\index{るー@ルー!ふろんと@ブロンドの---}
\index{roux@roux!blond@--- blond}

(仕上がり1kg分)

材料の比率は茶色いルーと同じ。すなわちバター500 gと、ふるった小麦粉600
g。

火入れは、ルーがほんのりブロンド色になるまで、ごく弱火で行なう。

\maeaki

\hypertarget{roux-blanc}{%
\subsubsection{白いルー}\label{roux-blanc}}

\frsub{Roux blanc}

\index{るー@ルー!しろい@白い---} \index{roux@roux!blanc@--- blanc}

500 gのバターと、ふるった小麦粉600 g。

このルーの火入れは数分、つまり粉っぽさがなくなるまでの時間でいい。
\end{recette}