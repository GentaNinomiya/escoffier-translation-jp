\hypertarget{ux30a4ux30aeux30eaux30b9ux98a8ux30bdux30fcux30b9ux6e29ux88fd24}{%
\section[イギリス風ソース(温製)]{\texorpdfstring{イギリス風ソース(温製)\footnote{この節では初版で31、第二版は33、第三版と第四版で30のレシピが掲
  載されている。1907年刊の英語版\emph{A Guide to Modern Cookery}でこの節
  に相当する``Hot English Sauces''には10のレシピしか掲載されていない。
  この大きな数の差をどう解釈するかは意見の分かれるところだろうが、対
  象読者がフランス人であるかイギリス人であるかという違いを意識し、ニー
  ズに応えるかたちをとったと考えるのが妥当だろう。ただし、あくまでも
  エスコフィエあるいは共同執筆者の解釈を経た「イギリス風」のソースが
  ほとんどであることは、例えば「\protect\hyperlink{roe-buck-sauce}{ローバックソース}」
  において\protect\hyperlink{sauce-espagnole}{ソース・エスパニョル}を用いていること、
  つまりはエスコフィエが構築したソースの体系に組み込まれ得るものであ
  ることから判断がつく。}}{イギリス風ソース(温製)}}\label{ux30a4ux30aeux30eaux30b9ux98a8ux30bdux30fcux30b9ux6e29ux88fd24}}

\frsec{Sauces Anglaises Chaudes}

\index{そーす@ソース!いきりすふうおんせい@イギリス風---(温製)}
\index{いきりすふう@イギリス風!そーすおんせい@---ソース(温製)}
\index{sauce@sauce!anglaises chaudes@---s anglaises chaudes}
\index{anglais@anglais(e)!sauces chaudes@sauces ---es chaudes}
\begin{recette}
\hypertarget{cranberries-sauce}{%
\subsubsection[クランベリーソース]{\texorpdfstring{クランベリーソース\footnote{英語のcranberryはツルコケモモ(学名Vaccinium
  oxycoccos)であり、 フランス語airelles rougesはコケモモ(学名Vaccinium
  vitis-idaea
  L.)で、非常によく似た近縁種であり、しばしば混同される。本書でもと
  くに区別されていない。}}{クランベリーソース}}\label{cranberries-sauce}}

\frsub{Sauce aux Airelles\hspace{1em}\normalfont(\textit{Cranberries-Sauce})}

\index{いきりすふう@イギリス風!そーすおんせい@---ソース(温製)!くらんへりーそーす@クランベリーソース}
\index{そーす@ソース!いきりすふうおんせい@イギリス風---(温製)!くらんへりーそーす@クランベリー---}
\index{くらんへりー@クランベリー!そーす@---ソース}

\index{sauce@sauce!anglaises chaudes@---s anglaises chaudes!airelles@--- aux Airelles (Cranberries-Sauce)}
\index{airelle@airelle!sauce airelles@Sauce aux Airelles (Cranberries-Sauce)}
\index{anglais@anglais(e)!sauces chaudes@sauces ---es chaudes!airelles@Sauce aux Airelles (Cranberries-Sauce)}
\index{cranberry!Cranberries-Sauce}

クランベリー500 gを1
L湯で、鍋に蓋をして茹でる。果肉に火が通ったら、湯をきって、目の細かい網で裏漉しする。

こうして出来たピュレに茹で汁を適量加えてやや濃度のあるソースの状態にする。好みに応じて砂糖を加える。

このソースは市販品があり\footnote{\protect\hyperlink{sauce-robert-escoffier}{ソース・ロベール・エスコフィエ}などの
  ようなエスコフィエブランドの商品というわけではないと思われる。}、水少々を加えて温めるだけで使える。

\ldots{}\ldots{}七面鳥のロースト用。

\maeaki

\hypertarget{albert-sauce}{%
\subsubsection[アルバートソース]{\texorpdfstring{アルバートソース\footnote{ザクセン=コーブルク=ゴータ公アルバート王配(ヴィクトリア女王の
  夫)(1819〜1861)のこと。女王エリザベス二世の高祖父。本書序文p.ii
  において触れられている料理人エルーイがアルバート王配に仕えていたこ
  とがある。なお、本書に掲載されていないが、Sole Albert 「舌びらめ 
  アルベール」という料理がある。しかしながら、これはパリのレストラン、
  マキシムズMaxim'sでメートルドテルを務めたアルベール・ブラゼール Albert
  Blazerの名を冠したもので1930年代に創案されたもの。このソー
  スとはまったく関係がないことに注意。}}{アルバートソース}}\label{albert-sauce}}

\frsub{Sauce Albert\hspace{1em}\normalfont(\textit{Albert-Sauce})}

\index{いきりすふう@イギリス風!そーすおんせい@---ソース(温製)!あるはーと@アルバートソース}
\index{そーす@ソース!いきりすふうおんせい@イギリス風---(温製)!あるはーと@アルバート---}
\index{あるはーと@アルバート!そーす@---ソース}
\index{sauce@sauce!anglaises chaudes@---s anglaises chaudes!albert@--- Albert (Albert-Sauce)}
\index{albert@Albert!sauce@Sauce --- (Albert-Sauce)}
\index{anglais@anglais(e)!sauces chaudes@sauces ---es chaudes!albert@Sauce Albert (Albert-Sauce)}

すりおろしたレフォール\footnote{raifort ホースラディッシュ、西洋わさび。}150
gに\protect\hyperlink{}{白いコンソメ}2 dLを注ぎ、弱火で20分間煮る。

\protect\hyperlink{butter-sauce}{イギリス式バターソース}3
dLと生クリーム2\undemi{} dL、パンの白い身の部分40
gを加える。強火にかけて煮詰め、木ヘラで圧し絞るようにしながら布で漉す\footnote{二人で作業すると容易。\protect\hyperlink{veloute}{ヴルテ}訳注参照。}。卵黄2個を加えてとろみを付
け\footnote{このソースの特徴として、イギリスのローストビーフに欠かせないもの
  とされるレフォール(ホースラディッシュ)を用いていることの他に、と
  ろみ付けにパンと卵黄を使っている点にも注目すべきだろう。とろみ付け
  の要素としてはきわめて中世料理風と言ってもいい。ただし、中世の料理
  では、パンはこんがりと焼いてからヴィネガーなどでふやかしてよくすり
  潰し、さらに布で漉してとろみ付けに用いるのが一般的だった。パンの白
  い身の部分をそのまま使えるということは、それだけ小麦の精白度合いが
  高いということでもある。}、塩1つまみとこしょう少々で味を調える。

仕上げに、マスタード小さじ1杯をヴィネガー大さじ1杯で溶いてから加える。

\ldots{}\ldots{}牛肉、主としてフィレ肉のブレゼに添える。

\maeaki

\hypertarget{aromatic-sauce}{%
\subsubsection{アロマティックソース}\label{aromatic-sauce}}

\frsub{Sauce aux Aromates\hspace{1em}\normalfont(\textit{Aromatic-Sauce})}

\index{そーす@ソース!いきりすふうおんせい@イギリス風---(温製)!あろまていつく@アロマティック---}
\index{いきりすふう@イギリス風!そーすおんせい@---ソース(温製)!あろまていつく@アロマティックソース}
\index{こうそう@香草!あろまていつく@アロマティックソース}

\index{sauce@sauce!anglaises chaudes@---s anglaises chaudes!aromates@--- aux Aromates (Aromatic-Sauce}
\index{aromate@aromate!sauce@Sauce aux Aromates}
\index{anglais@anglais(e)!sauces chaudes@sauces ---es chaudes!aromates@Sauce aux Aromates (Aromatic-Sauce)}

\protect\hyperlink{consomme-blanc}{コンソメ}\undemi{}
Lに、タイム1枝、バジル4 g、サリエット\footnote{シソ科の香草。サマーセイヴォリー。和名キダチハッカ。}1
g、マジョラム1 g、セージ1 g、シブレット\footnote{ciboulette
  チャイヴ。アサツキと訳されることもあるが、日本のアサツキとは風味が違うので注意。}1を刻んだもの1つまみ、エシャロット\footnote{玉ねぎによく似ているが小さくて水分量の少ない香味野菜。英語由来
  のシャロットと呼ばれることも。日本の青果マーケットに見られる「エシャ
  レット」はらっきょうの若どりであってまったく別のもの。}2個のみじん切り、ナツメグ少々、大粒のこしょう4個を入れて、10分
間煎じる\footnote{infuser アンフュゼ。}。

シノワ\footnote{円錐形で取っ手の付いた漉し器。}で漉し、バターで作った\footnote{本書第四版ではルーは必ずバターを用いる指示がなされているが、初
  版から第三版までは、バターもしくはグレスドマルミット(コンソメなど
  を作る際に浮いてきた油脂をすくい取って漉したもの)を使うという指示
  だっため、「バターで作った」という記述がこのように残っているレシピ
  が散見される。}ブロンドのルー50
gを入れてとろみを付ける。数分間沸かしてから、レモン\undemi{}個分の搾り汁と、みじん切りにして下茹でしておいたセルフイユ\footnote{cerfeuil
  チャービル。}とエストラゴン\footnote{estragon フレンチタラゴン。}計大さじ1杯を加えて仕上げる\footnote{このソースで用いられている香草類の種類の多さは特筆に値するだろ
  う。ブラウン系の派生ソースにある\protect\hyperlink{sauce-aux-fines-herbes}{香草ソー
  ス}およびホワイト系派生ソースの\protect\hyperlink{sauce-aux-fines-herbes-blanche}{香草ソー
  ス}と比較されたい。}。

\ldots{}\ldots{}大きな魚まるごと1尾のポシェあるいは牛、羊肉の大掛かりな仕立て(ルルヴェ\footnote{relevé
  \protect\hyperlink{sauce-diplomate}{ソース・ディプロマット}訳注参照。})に添える。

\maeaki

\hypertarget{butter-sauce}{%
\subsubsection{バターソース}\label{butter-sauce}}

\frsub{Sauce au Beurre à l'anglaise\hspace{1em}\normalfont(\textit{Butter Sauce})}

\index{いきりすふう@イギリス風!そーすおんせい@---ソース(温製)!はたーそーす@バターソース}
\index{そーす@ソース!いきりすふうおんせい@イギリス風---(温製)!はたーそーす@バター--}
\index{はたー@バター!そーすいきりすふう@---ソース(イギリス風)}
\index{sauce@sauce!anglaises chaudes@---s anglaises chaudes!beurre@--- au Beurre à l'anglaise (Butter Sauce)}
\index{beurre@beurre!sauce anglaise@Sauce au Beurre à l'anglaise (Butter Sauce)}
\index{anglais@anglais(e)!sauces chaudes@sauces ---es chaudes!beurre@Sauce au Beurre à l'anglaise (Butter Sauce)}

フランスの\protect\hyperlink{sauce-au-beurre}{ソース・オ・ブール}と同様に作るが、より濃度の高い仕上がりにする点が違う。分量は、バター60
g、小麦粉60 g、1 Lあたり塩7 gを加えて沸かした湯\troisquarts{}
L。レモンの搾り汁5〜6滴、バター 200 g。とろみ付け用の卵黄は用いない。

\maeaki

\hypertarget{capers-sauce}{%
\subsubsection{ケイパーソース}\label{capers-sauce}}

\frsub{Sauce aux Câpres\hspace{1em}\normalfont(\textit{Capers-Sauce})}

\index{いきりすふう@イギリス風!そーすおんせい@---ソース(温製)!けいはー@ケイパーソース}
\index{そーす@ソース!いきりすふうおんせい@イギリス風---(温製)!けいはーそーす@ケイパー---}
\index{けいはー@ケイパー!そーすいきりすふう@---ソース(イギリス風)}
\index{sauce@sauce!anglaise chaude@--- anglaie chaude!capres@--- aux Câpres (Capers-Sauce)}
\index{capre@câpre!sauce capres anglaise@Sauce aux Câpres (Capers-Sauce)}
\index{anglais@anglais(e)!sauces chaudes@sauces ---es chaudes!capres@Sauce aux Câpres (Capers-Sauce)}

上記の\protect\hyperlink{butter-sauce}{バターソース}1
Lあたり大さじ4杯のケイパーを加えたもの。

\ldots{}\ldots{}茹でた魚に添える。また、イギリス風\footnote{à l'anglaise
  アラングレーズ。茹でる(下茹でも含む)場合には、塩を加えた湯で茹でることを指す。なお、パン粉衣
  pané à l'anglaise
  という場合には、現代の日本でもなじみのある、小麦粉、溶きほぐした卵、パン粉の順で衣を付けて揚げることを言う。調理法全体を通しての規則性はなく、あくまでも「イギリス風に由来する」または「イギリス風」を意味するものなので注意。}に茹でた仔羊腿肉には欠かせない。

\maeaki

\hypertarget{celery-sauce}{%
\subsubsection{セロリソース}\label{celery-sauce}}

\frsub{Sauce au Céleri\hspace{1em}\normalfont(\textit{Celery-Sauce})}

\index{いきりすふう@イギリス風!そーすおんせい@---ソース(温製)!せろり@セロリソース}
\index{そーす@ソース!いきりすふうおんせい@イギリス風---(温製)!せろり@セロリ---}
\index{せろり@セロリ!そーすいきりすふう@---ソース(イギリス風)}
\index{sauce@sauce!anglaises chaudes@---s anglaises chaudes!celeri@--- au Céleri (Celery-Sauce)}
\index{celeri@céleri!sauce anglaise@Sauce au Céleri (Celery-Sauce)}
\index{anglais@anglais(e)!sauces chaudes@sauces ---es chaudes!celeri@Sauce au Céleri (Celery-Sauce)}

セロリ6株を掃除して、芯のところだけを使う\footnote{緑色が薄いタイプのセロリは中心部が自然に軟白され、柔らかいので、フランス料理でも非常に好まれる。}。これをソテー鍋に並べ、\protect\hyperlink{}{白いコンソメ}をセロリがかぶるまで注ぐ。ブーケガルニとクローブを刺した玉ねぎ1個を入れ、弱火で加熱する。

セロリの水気をきり、鉢に入れてすり潰す。これを布で漉す。こうして出来たセロリのピュレと同量の\protect\hyperlink{cream-sauce}{クリームソース}を加える。セロリの茹で汁を煮詰めたものを大さじ2〜3杯加える。

沸騰しない程度に温め、すぐに提供しない場合は湯煎にかけておく。

\ldots{}\ldots{}茹でた鶏または鶏のブレゼに添える。

\maeaki

\hypertarget{roe-buck-sauce}{%
\subsubsection[ローバックソース]{\texorpdfstring{ローバックソース\footnote{roebuck
  英語でノロ鹿のこと。}}{ローバックソース}}\label{roe-buck-sauce}}

\frsub{Sauce Chevreuil\hspace{1em}\normalfont(\textit{Roe-buck Sauce})}

\index{いきりすふう@イギリス風!そーすおんせい@---ソース(温製)!ろーはつく@ローバックソース}
\index{そーす@ソース!いきりすふうおんせい@イギリス風---(温製)!ろーはっく@ローバック---}
\index{ろーはつく@ローバック!そーすいきりすふう@---ソース(イギリス風)}
\index{のろしか@ノロ鹿 ⇒ シュヴルイユ!そーす@ソース!ろーはつくそーす@ローバックソース(イギリス風)}
\index{しゆうるいゆ@シュヴルイユ!ろーはつくそーす@ローバックソース(イギリス風)}
\index{sauce@sauce!anglaises chaudes@---s anglaises chaudes!chevreuil@--- Chevreuil (Roe-buck Sauce)}
\index{chevreuil@chevreuil!sauce chevreuil anglaise@Sauce Chevreuil (Roe-buck Sauce)}
\index{anglais@anglais(e)!sauces chaudes@sauces ---es chaudes!chevreuil@Sauce Chevreuil (Roe-buck Sauce)}

中位の大きさの玉ねぎを1 cm角くらいの粗みじん切\footnote{paysanne
  ペイザンヌに切る、と言う。主として野菜について言うが、1 cm角で厚さ1〜2
  mm程度。}りにし、生ハム80 g
も同様に刻む。これをバターで軽く色付くまで炒める。ブーケガルニを入れ、
ヴィネガー1\undemi{} dLを注ぎ、ほとんど完全に煮詰める。

\protect\hyperlink{sauce-espagnole}{ソース・エスパニョル}3
dLを注ぎ、15分程弱火にかけて、 浮いてくる不純物を取り除く\footnote{dépouiller
  デプイエ ≒ écumer エキュメ。}。

15分経ったら、ブーケガルニを取り出し、ポルト酒コップ1杯\footnote{約1
  dL。}と\protect\hyperlink{}{グロゼ
イユのジュレ}大さじ1杯強を加えて仕上げる。

\ldots{}\ldots{}大型ジビエ肉\footnote{この場合は当然、ノロ鹿の料理だが、フランス料理でノロ鹿は時間をかけてマリネしてから調理し、そのマリナード(漬け汁)もソースに用いるのと比べると非常にシンプルなソースになっている点が興味深い。}の料理に添える。

\maeaki

\hypertarget{cream-sauce}{%
\subsubsection{クリームソース}\label{cream-sauce}}

\frsub{Sauce Crème à l'anglaise\hspace{1em}\normalfont(\textit{Cream-Sauce})}

\index{いきりすふう@イギリス風!そーすおんせい@---ソース(温製)!くりーむ@クリームソース}
\index{そーす@ソース!いきりすふうおんせい@イギリス風---(温製)!くりーむ@クリーム---}
\index{くりーむ@クリーム!そーすいきりすふう@---ソース(イギリス風)}
\index{sauce@sauce!anglaises chaudes@---s anglaises chaudes!creme@--- Crème à l'anglaise (Cream-Sauce)}
\index{creme@crème!sauce creme anglaise@Sauce Crème à l'anglaise (Cream-Sauce)}
\index{anglais@anglais(e)!sauces chaudes@sauces ---es chaudes!creme@Sauce Crème à l'anglaise (Cream-Sauce)}

バター100 gと小麦粉60
gで\protect\hyperlink{roux-blanc}{白いルー}を作る。

\protect\hyperlink{consomme-blanc}{白いコンソメ}7
dLでルーをのばし、マッシュルームのエッセンス1 dLと生クリーム2
dLを加える。

火にかけて沸騰させる。小玉ねぎ1個とパセリ1束を加え、弱火で15分程煮込む。提供直前に小玉ねぎとパセリは取り出す。

\ldots{}\ldots{}仔牛の骨付き背肉の塊\footnote{carré
  カレ。もとは「四角形」の意。料理では、肋骨ごとに切り分けていない仔牛および仔羊の骨付き背肉の塊を指す。}のローストに合わせる。

\maeaki

\hypertarget{shrimps-sauce}{%
\subsubsection{シュリンプソース}\label{shrimps-sauce}}

\frsub{Sauce Crevettes à l'anglaise\hspace{1em}\normalfont(\textit{Shrimps-Sauce})}

\index{いきりすふう@イギリス風!そーすおんせい@---ソース(温製)!しゆりんふ@シュリンプソース}
\index{そーす@ソース!いきりすふうおんせい@イギリス風---(温製)!しゆりんふ@シュリンプ---}
\index{くるうえつと@クルヴェット!そーすいきりすふう@シュリンプソース(イギリス風)}
\index{sauce@sauce!anglaises chaudes@---s anglaises chaudes!crevettes@--- Crevettes à l'anglaise (Shrimps-Sauce)}
\index{crevette@crevette!sauce crevette anglaise@Sauce Crevettes à l'anglaise (Shrimps-Sauce)}
\index{anglais@anglais(e)!sauces chaudes@sauces ---es chaudes!crevettes@Sauce Crevettes à l'anglaise (Shrimps-Sauce)}

カイエンヌ少量を加えて風味を引き締めた\protect\hyperlink{butter-sauce}{イギリス風バターソー
ス}1 Lに、アンチョビエッセンス小さじ1杯と殻を剥いた小海 老\footnote{フランス語は
  crevette(s)
  クルヴェット。\protect\hyperlink{sauce-aux-crevettes}{ソース・クルヴェッ
  ト}訳注参照。}の尾の身125 gを加える。

\ldots{}\ldots{}魚料理用。

\maeaki

\hypertarget{devilled-sauce}{%
\subsubsection{デビルソース}\label{devilled-sauce}}

\frsub{Sauce Diable\hspace{1em}\normalfont(\textit{Devilled Sauce})}

\index{いきりすふう@イギリス風!そーすおんせい@---ソース(温製)!てひるそーす@デビルソース}
\index{そーす@ソース!いきりすふうおんせい@イギリス風---(温製)!てひる@デビル---}
\index{あくま@悪魔 ⇒ ディアーブル!そーす@ソース!てひる@デビルソース(イギリス風)}
\index{ていあーふる@ディアーブル!てひるそーす@デビルソース(イギリス風)}
\index{sauce@sauce!anglaises chaudes@---s anglaises chaudes!diable@--- Diable (Devilled Sauce)}
\index{diable@diable!sauce diable anglaise@Sauce Diable (Devilled Sauce)}
\index{anglais@anglais(e)!sauces chaudes@sauces ---es chaudes!Sauce Diable (Devilled Sauce)}

1\undemi{} dLのヴィネガーにエシャロットのみじん切り大さじ1杯強を加えて、
半量になるまで煮詰める。\protect\hyperlink{sauce-espagnole}{ソース・エスパニョ
ル}2\undemi{} dLとトマトピュレ大さじ2杯を加え、5分間 程煮る。

仕上げに、ダービーソース\footnote{原文Derby-sauce、1940年代にアメリカで市販されていたのは確認され
  ているが、ここで言及されているのとまったく同じかは不明。なお、初版
  および第二版でこの部分は「ハーヴェイソースとウスターシャーソース各
  大さじ1杯」、第三版では「ハーヴェイソースとエスコフィエソース各大
  さじ1」となっている。「ダービーソース」が当初「エスコフィエソース」
  として商品化された後に何らかの事情により名称変更がなされたという可
  能性も否定できないが、第二版および英語版において\protect\hyperlink{sauce-diable-escoffier}{ソース・ディアー
  ブル・エスコフィエ}および\protect\hyperlink{sauce-robert-escoffier}{ソース・ロベー
  ル・エスコフィエ}、さらに第二版と同年刊の
  英語版のみに掲載されているSauce aux Cerises Escoffierソース・オ・
  スリーズ・エスコフィエのように既にエスコフィエブランドの既製品ソー
  スがあるために、矛盾が生じてしまう。第三版の記述が\protect\hyperlink{sauce-diable-escoffier}{ソース・ディアー
  ブル・エスコフィエ}を意味していると解釈す
  れば矛盾は生じないだろう。ハーヴェイソースについては\protect\hyperlink{brown-gravy}{ブラウングレ
  イヴィー}訳注参照。}大さじ1杯とカイエンヌ1つまみ強を加え、シ
ノワ\footnote{円錐形で取っ手の付いた漉し器。}か布で漉す。

\maeaki

\hypertarget{scotch-eggs-sauce}{%
\subsubsection{スコッチエッグソース}\label{scotch-eggs-sauce}}

\frsub{Sauce Ecossaise\hspace{1em}\normalfont(\textit{Scotch eggs Sauce})}

\index{いきりすふう@イギリス風!そーすおんせい@---ソース(温製)!すこつちえつくそーす@スコッチエッグソース}
\index{そーす@ソース!いきりすふうおんせい@イギリス風---(温製)!すこつとらんといきりすふう@スコッチエッグ---}
\index{すこつとらんと@スコットランド!すこつちえつくそーす@スコッチエッグソース(イギリス風)}
\index{sauce@sauce!anglaises chaudes@---s anglaises chaudes!ecossaise@--- Ecossaise (Scotch eggs Sauce)}
\index{scotland@Scotland!sauce ecossaise@Sauce Ecossaise (Scotch eggs Sauce)}
\index{anglais@anglais(e)!sauces chaudes@sauces ---es chaudes!Sauce Ecossaise (Scotch eggs Sauce)}

バター60 gと小麦粉30 g、沸かした牛乳4
dLで\protect\hyperlink{sauce-bechamel}{ベシャメルソー
ス}を用意する。味付けは通常どおりにすること。ソースが
沸騰したらすぐに、固茹で卵の白身4個を薄切りにした\footnote{émincer
  エマンセ、薄切りにすること。}ものを加える。

提供直前に、茹で卵の卵黄を目の粗い漉し器で漉したものを混ぜ込む。

\ldots{}\ldots{}\ruby{鱈}{たら}には欠かせないソース。

\maeaki

\hypertarget{fennel-sauce}{%
\subsubsection[フェンネルソース]{\texorpdfstring{フェンネルソース\footnote{日本語でフェンネルと呼ばれるものは、(a)主に香草として葉を利用
  するタイプfenouil sauvage(フヌイユソヴァージュ)と、(b)白く肥大し
  た株元を食用とするフローレンス・フェンネルfenouil de florence(フ
  ヌイユ・ド・フロロンス)またはfenouil bulbeux(フヌイユビュルブー)
  と呼ばれる2種がある。本書ではどちらを用いるのか明記されていないこ
  とが多いが、一般に、葉を利用するタイプは香りが非常に強く、フローレ
  ンスフェンネルの葉も食用可能だが、香りは比較的おとなしい。}}{フェンネルソース}}\label{fennel-sauce}}

\frsub{Sauce au Fenouil\hspace{1em}\normalfont(\textit{Fennel Sauce})}

\index{いきりすふう@イギリス風!そーすおんせい@---ソース(温製)!ふえんねる@フェンネルソース}
\index{そーす@ソース!いきりすふうおんせい@イギリス風---(温製)!ふえんねる@フェンネル---}
\index{ふえんねる@フェンネル!そーすいきりすふう@フェンネルソース(イギリス風)}
\index{sauce@sauce!anglaises chaudes@---s anglaises chaudes!fenouil@--- au Fenouil (Fennel Sauce)}
\index{fenouil@fenouil!sauce anglaise@Sauce au Fenouil (Fennel Sauce)}
\index{anglais@anglais(e)!sauces chaudes@sauces ---es chaudes!fenouil@Sauce au Fenouil (Fennel Sauce)}

普通に作った\protect\hyperlink{butter-sauce}{バターソース}2\undemi{}
dLあたり、細かく刻 んで下茹でしたフェンネル大さじ1杯を加える。

\ldots{}\ldots{}このソースは主として、グリルあるいは茹でた鯖に合わせる。

\maeaki

\hypertarget{gooseberry-sauce}{%
\subsubsection{グーズベリーソース}\label{gooseberry-sauce}}

\frsub{Sauce aux Groseilles\hspace{1em}\normalfont(\textit{Gooseberry Sauce})}

\index{いきりすふう@イギリス風!そーすおんせい@---ソース(温製)!くーすへりーそーす@グーズベリーソース}
\index{そーす@ソース!いきりすふうおんせい@イギリス風---(温製)!くーすへりーいきりすふう@グーズベリー---}
\index{くーすへりー@グーズベリー!そーすいきりすふう@グーズベリーソース(イギリス風)}
\index{すくり@すぐり!そーす@ソース!くーすへりーそーすいきりすふう@グーズベリーソース(イギリス風)}
\index{くろせいゆ@グロゼイユ!そーす@ソース!くーすへりーそーすいきりすふう@グーズベリーソース(イギリス風)}
\index{sauce@sauce!anglaises chaudes@---s anglaises chaudes!groseilles@--- aux Groseilles (Gooseberry Sauce)}
\index{groseille@groseille!sauce anglaise@Sauce aux Groseilles (Gooseberry Sauce)}
\index{anglais@anglais(e)!sauces chaudes@sauces ---es chaudes!groseilles@Sauce aux Groseilles (Gooseberry Sauce)}

グーズベリー1 Lの皮を剥いて洗い、砂糖125 gと水1 dLを加えて火にかける。
目の細かい漉し器で裏漉しする。

\ldots{}\ldots{}このピュレはグリルした鯖に合わせる。

\maeaki

\hypertarget{lobster-sauce}{%
\subsubsection{ロブスターソース}\label{lobster-sauce}}

\frsub{Sauce Homard à l'anglaise\hspace{1em}\normalfont(\textit{Lobster Sauce})}

\index{いきりすふう@イギリス風!そーすおんせい@---ソース(温製)!ろふすたーそーす@ロブスターソース}
\index{そーす@ソース!いきりすふうおんせい@イギリス風---(温製)!ろふすたーいきりすふう@ロブスター---}
\index{ろふすたー@ロブスター!そーすいきりすふう@ロブスターソース(イギリス風)}
\index{おまーる@オマール!そーす@ソース!ろふすたーそーすいきりすふう@ロブスターソース(イギリス風)}
\index{sauce@sauce!anglaises chaudes@---s anglaises chaudes!homard@--- Homard à l'anglaise (Lobster Sauce)}
\index{homard@homard!sauce anglaise@Sauce Homard à l'anglaise (Lobster Sauce)}
\index{anglais@anglais(e)!sauces chaudes@sauces ---es chaudes!homard@Sauce Homard à l'anglaise (Lobster Sauce)}

カイエンヌを加えて風味を引き締めた\protect\hyperlink{sauce-bechamel}{ベシャメルソース}1
Lに、アンチョビエッセンス大さじ1杯と、さいの目に切ったオマールの尾の身100
gを加える\footnote{ホワイト系派生ソースの節にある\protect\hyperlink{sauce-homard}{ソース・オマール}
  を比較すると、このソースのシンプルさが際立って見えるが、ベシャメル
  を基本ソースにしている点で、やはり「ソースの体系」に組込まれたもの
  であり、純粋にイギリス料理由来というわけでもないと思われる。なお、
  このレシピは初版からほぼ異同がなく、1907年の英語版には含まれていな
  い。}。

\ldots{}\ldots{}魚料理用。

\maeaki

\hypertarget{oyster-sauce}{%
\subsubsection{牡蠣入りソース}\label{oyster-sauce}}

\frsub{Sauce aux Huîtres\hspace{1em}\normalfont(\textit{Oyster Sauce})}

\index{いきりすふう@イギリス風!そーすおんせい@---ソース(温製)!かきいりそーす@牡蠣入りソース}
\index{そーす@ソース!いきりすふうおんせい@イギリス風---(温製)!かきいりいきりすふう@牡蠣入り---}
\index{かき@牡蠣!そーすいきりすふう@牡蠣入りソース(イギリス風)}
\index{sauce@sauce!anglaises chaudes@---s anglaises chaudes!huitres@--- aux huitres (Oyster Sauce)}
\index{huitre@huître!sauce anglaise@Sauce aux Huîtres (Oyster Sauce)}
\index{anglais@anglais(e)!sauces chaudes@sauces ---es chaudes!huitre@Sauce aux Huîtres (Oyster Sauce)}

バター20 gと小麦粉15 gでブロンドのルーを作る。

このルーを、牛乳1 dLと生クリーム1 dLで溶く。塩1つまみを加えて調味し、
火にかけて沸騰させたら弱火にして10分間煮る。

布で漉し、カイエンヌを加えて風味を引き締める。沸騰しない程度の温度で火
を通して周囲をきれいに掃除した牡蠣の身12個を1 cm程度の厚さに切って、ソー
スに加える。

\ldots{}\ldots{}もっぱら茹でた魚\footnote{初版および第二版では「もっぱら茹でた生鱈に合わせる」とある。こ
  のレシピも1907年の英語版には掲載されていない。}に添える。

\maeaki

\hypertarget{brown-oyster-sauce}{%
\subsubsection{牡蠣入りブラウンソース}\label{brown-oyster-sauce}}

\frsub{Sauce brune aux Huîtres\hspace{1em}\normalfont(\textit{Brown Oyster Sauce})}

\index{いきりすふう@イギリス風!そーすおんせい@---ソース(温製)!かきいりふらうんそーす@牡蠣入りブラウンソース}
\index{そーす@ソース!いきりすふうおんせい@イギリス風---(温製)!かきいりふらうんいきりすふう@牡蠣入りブラウン---}
\index{かき@牡蠣!そーすふらうんいきりすふう@牡蠣入りブラウンソース(イギリス風)}
\index{sauce@sauce!anglaises chaudes@---s anglaises chaudes!brune huitres@--- brune aux huitres (Brown Oyster Sauce)}
\index{huitre@huître!sauce brune anglaise@Sauce brune aux Huîtres (Brown Oyster Sauce)}
\index{anglais@anglais(e)!sauces chaudes@sauces ---es chaudes!brune huitre@Sauce brune aux Huîtres (Brown Oyster Sauce)}

上記の牡蠣入りソースと作り方はまったく同じだが、牛乳と生クリームではな
く、\protect\hyperlink{fonds-brun}{茶色いフォン}2 dLを使うこと。

\ldots{}\ldots{}このソースは、グリル焼きした肉や、肉のプディング\footnote{本書にはイギリス風の肉料理としてのプディングのレシピも掲載され
  ている。\protect\hyperlink{beefteak-pudding}{ビーフステークのプディング}、\protect\hyperlink{beefsteak-and-kidney-pudding}{ビーフ
  ステークとキドニーのプディング}、
  \protect\hyperlink{beefsteak-and-oysters-pudding}{ビーフステークと牡蠣のプディング}。
  なお、本書でのbeefsteakビーフステークとは肉の切り方のことを意味し
  ており、グリル焼きあるいはソテーしたもののことではない。ここでは厚 さ1
  cm程度にスライスした牛肉のことを指している。}、生鱈のグリル焼きに合わせる。

\maeaki

\hypertarget{brown-gravy}{%
\subsubsection{ブラウングレイヴィー}\label{brown-gravy}}

\frsub{Jus coloré\hspace{1em}\normalfont(\textit{Brown Gravy})}

\index{いきりすふう@イギリス風!そーすおんせい@---ソース(温製)!ふらうんくれいういー@ブラウングレイビヴィー}
\index{そーす@ソース!いきりすふうおんせい@イギリス風---(温製)!ふらうんくれいういー@ブラウングレイヴィー(イギリス風)}
\index{くれいういー@グレイヴィー!そーすふらうんいきりすふう@ブラウングレイヴィー(イギリス風ソース)}
\index{sauce@sauce!anglaises chaudes@---s anglaises chaudes!jus colore@Jus coloré (Brown Gravy)}
\index{gravy@gravy!jus colore anglaise@Jus coloré (Brown Gravy)}
\index{anglais@anglais(e)!sauces chaudes@sauces ---es chaudes!jus colore@Jus coloré (Brown Gravy)}

\protect\hyperlink{butter-sauce}{イギリス風バターソース}4
dLに、ローストの肉汁2 dLとケ チャップ\footnote{ここではマッシュルームケチャップのこと。マッシュルームの薄切り
  を塩、こしょう、香辛料で5〜6日漬け込み、その絞り汁を沸かして香辛料
  とトマトを加えて味を調え、漉してから保存する (『ラルース・ガストロ
  ノミック』初版)。なお、ketchupは語源が、中国福建省アモイの方言で、
  香辛料を加えて醗酵させた魚醤の一種を意味するのkôe-chiapまたは
  kê-chiap(鮭汁)だとされている。これがマレー語に伝播し、kecap(発
  音はケーチャプ)と変化し、17世紀頃、現在のシンガポールおよびマレー
  シアを植民地支配していたイギリス人の知るところとなった。イギリスに
  も古くから魚醤の類はあり、そのバリエーションのひとつとして、マッシュ
  ルームとエシャロットを添加した魚醤をketchupと呼ぶようになった。や
  がて魚醤文化の衰退とともに、ケチャップと呼ばれるものはマッシュルー
  ムが主原料となり、いわゆるマッシュルームケチャップが18世紀頃に成立
  したとされる。これは、塩漬けにして醗酵させたマッシュルームの搾り汁
  にメース、ナツメグ、こしょうなどの香辛料を加えて煮詰め、漉したもの。
  これにトマトを添加するようになった時期は判然としないが、おそらくは
  19世紀初頭だったと思われる。フランスの料理書では1814年刊ボヴィリエ
  『調理技術』第1巻に作り方が詳述されているが(p.72)、トマトは用いな
  いマッシュルームケチャップのバリエーション。トマトを主原料としたケ
  チャップは、アメリカのハインツHeinzが1876年にハインツ・トマトケチャッ
  プを製品化して以降、徐々に広まっていった。このため、英語圏で成立、
  普及したトマトケチャップがフランスにおいて知られるようになるのは、
  少なくとも上記『ラルース・ガストロノミック』初版(1938年)よりも後
  のことであり、おそらくは第二次大戦後だろうと思われる。}大さじ\undemi{}杯、ハーヴェイソース\footnote{Herwey
  Sauce 19世紀〜20世紀前半にかけて既製品が流通していた。現
  在は商品としては存在していないと思われる。原料はアンチョビ、ヴィネ
  ガー、マッシュルームケチャップ、にんにく、大豆由来原料(詳細不明、
  おそらくは大豆レシチンすなわち大豆油かと思われる)、カイエンヌ、コ
  チニール色素などであったという。}大さじ\undemi{}杯 を加える。

\ldots{}\ldots{}もっぱら仔牛のローストに添える。

\hypertarget{eggs-sauce}{%
\subsubsection{エッグソース}\label{eggs-sauce}}

\frsub{Sauce aux OEufs à l'anglaise\hspace{1em}\normalfont(\textit{Eggs Sauce})}

\index{いきりすふう@イギリス風!そーすおんせい@---ソース(温製)!えつくそーす@エッグソース}
\index{そーす@ソース!いきりすふうおんせい@イギリス風---(温製)!えつくそーす@エッグソース}
\index{たまこ@卵!そーすうーいきりすふう@エッグソース(イギリス風)}
\index{sauce@sauce!anglaises chaudes@---s anglaises chaudes!oeufs anglaise@--- aux OEufs à l'anglaise (Eggs Sauce)}
\index{oeuf@oeuf!sauce anglaise@Sauce aux OEufs à l'anglaise (Eggs Sauce)}
\index{anglais@anglais(e)!sauces chaudes@sauces ---es chaudes!oeufs anglaise@Sauce aux OEufs à l'anglaise (Eggs Sauce)}

小麦粉60 gとバター30
gで\protect\hyperlink{roux-blanc}{白いルー}を作る。あらかじめ沸か
しておいた牛乳\undemi{} Lで溶く。塩、白こしょう、ナツメグ少々で味を調
える。火にかけて沸騰したら弱火にして5〜6分間煮る。

固茹で卵2個を白身、黄身ともに、さいの目に刻んでソースに加える。

\ldots{}\ldots{}ハドック\footnote{Haddock
  鱈の一種。フランス語では同じ綴りでアドックまたは églefin,
  aiglefinエーグルファンと呼ばれる。イギリスでは主に塩漬け
  を燻製にしたものを指す。}やモリュ\footnote{morue
  モリュ。干し鱈、塩鱈のこと。生のものはcabillaudカビヨと呼ばれる。}の料理に合わせるのが一般的。

\maeaki

\hypertarget{eggs-and-butter-sauce}{%
\subsubsection{エッグアンドバターソース}\label{eggs-and-butter-sauce}}

\frsub{Sauce aux OEufs au beurre à l'anglaise\hspace{1em}\normalfont(\textit{Eggs and Butter Sauce})}

\index{いきりすふう@イギリス風!そーすおんせい@---ソース(温製)!えつくあんとはたーそーす@エッグアンドバターソース}
\index{そーす@ソース!いきりすふうおんせい@イギリス風---(温製)!えつくあんとはたーそーす@エッグアンドバターソース}
\index{たまこ@卵!そーすうーるいきりすふう@エッグアンドバターソース(イギリス風)}
\index{はたー@バター!えつくあんとはたーそーす@エッグアンドバターソース(イギリス風)}
\index{sauce@sauce!anglaises chaudes@---s anglaises chaudes!oeufs beurre fondu@--- aux OEufs au Beurre fondu (Eggs and butter Sauce)}
\index{oeuf@oeuf!sauce oeufs beurre fondu@Sauce aux OEufs au beurre fondu (Eggs and butter Sauce)}
\index{anglais@anglais(e)!sauces chaudes@sauces ---es chaudes!oeufs beurre fondu@Sauce aux OEufs au beurre fondue (Eggs and butter Sauce)}

バター250 gを溶かし、塩適量、こしょう少々、レモン\undemi{}個分の搾り汁、
固茹で卵3個を熱いうちに大きめのさいの目に刻んだもの、みじん切りにして
下茹でしたパセリ小さじ1杯を加える。

\ldots{}\ldots{}茹でた魚の大きな仕立ての料理\footnote{relevé
  ルルヴェ。\protect\hyperlink{releve}{第二版序文訳注2}、
  および\protect\hyperlink{sauce-diplomate}{ソース・ディプロマット}訳注参照。}に添える。

\maeaki

\hypertarget{onions-sauce}{%
\subsubsection{オニオンソース}\label{onions-sauce}}

\frsub{Sauce aux Oignons\hspace{1em}\normalfont(\textit{Onions Sauce})}

\index{いきりすふう@イギリス風!そーすおんせい@---ソース(温製)!おにおんそーす@オニオンソース}
\index{そーす@ソース!いきりすふうおんせい@イギリス風---(温製)!おにおんそーす@オニオンソース(イギリス風)}
\index{たまねき@玉ねぎ!そーすおにおんいきりすふう@オニオンソース(イギリス風)}
\index{sauce@sauce!anglaises chaudes@---s anglaises chaudes!oignons@--- aux Oignons (Onions Sauce)}
\index{oignon@oignon!sauce anglaise@Sauce aux Oignons (Onions Sauce)}
\index{anglais@anglais(e)!sauces chaudes@sauces ---es chaudes!oignons@Sauce aux Oignons (Onions Sauce)}

玉ねぎ200 gを薄切りにする\footnote{émincer エマンセ。}。牛乳6
dLに塩、こしょう、ナツメグを加え て玉ねぎを茹でる。

火が通ったらすぐに、玉ねぎの水気をしっかりきって、みじん切りにする。

バター40 gと小麦粉40
gで\protect\hyperlink{roux-blanc}{白いルー}を作る。これを玉ねぎを
茹でた牛乳でのばす。火にかけて沸騰させ、みじん切りにした玉ねぎを加える。
ソースはとても濃い状態になっていること。そのまま7〜8分煮る。

\ldots{}\ldots{}このソースは何にでも合わせられる。うさぎ、鶏、牛などの胃や腸の料理
\footnote{tripes
  トリップ。主として反芻動物(すなわち牛)の胃腸の食材とし
  ての総称。日本ではTripes à la mode de Caen(トリップアラモードドコ
  ン)「カン風トリップ煮込み」が有名。}、茹でたマトン、ジビエのブレゼなど\ldots{}\ldots{}このソースは必ず合わせる肉
の上にかけてやること\footnote{本書におけるソースは特に指示がない場合はソース入れ(saucière
  ソ シエール)で料理本体と別添して供すると考えておくといい。}。

\maeaki

\hypertarget{bread-sauce}{%
\subsubsection{ブレッドソース}\label{bread-sauce}}

\frsub{Sauce au Pain\hspace{1em}\normalfont(\textit{Bread Sauce})}

\index{いきりすふう@イギリス風!そーすおんせい@---ソース(温製)!ふれつとそーす@ブレッドソース}
\index{そーす@ソース!いきりすふうおんせい@イギリス風---(温製)!ふれつとそーす@ブレッドソース}
\index{はん@パン!そーすふれつといきりすふう@ブレッドソース(イギリス風)}
\index{sauce@sauce!anglaises chaudes@---s anglaises chaudes!pain@--- au Pain (Bread Sauce)}
\index{pain@pain!sauce anglaise@Sauce au Pain (Bread Sauce)}
\index{anglais@anglais(e)!sauces chaudes@sauces ---es chaudes!pain@Sauce au Pain (Bread Sauce)}

牛乳\undemi{} Lを沸かし、フレッシュなパンの白い身80 gを投入する。塩1つ
まみ強、クローブ1本を刺した小玉ねぎ1個、バター30 gを加える。

弱火で15分程煮る。玉ねぎを取り出し、泡立て器でソースが滑かになるまでよ
く混ぜる。生クリーム約1 dLを加えて仕上げる。

\ldots{}\ldots{}鶏やジビエ(鳥類)のローストに合わせる。

\hypertarget{nota-bread-sauce}{%
\subparagraph{【原注】}\label{nota-bread-sauce}}

このブレッドソースを鶏のローストに添える場合は、ローストの肉汁もソース
入れで添えること。ジビエの場合はさらに、よく乾かしたパンを揚げた「ブレッ
ドクランプス」をソース入れに入れて添えること。また、フライドポテトの皿
も添えること。

\maeaki

\hypertarget{fried-bread-sauce}{%
\subsubsection{フライドブレッドソース}\label{fried-bread-sauce}}

\frsub{Sauce au Pain frit\hspace{1em}\normalfont(\textit{Fried bread Sauce})}

\index{いきりすふう@イギリス風!そーすおんせい@---ソース(温製)!ふらいとふれつとそーす@フライドブレッドソース}
\index{そーす@ソース!いきりすふうおんせい@イギリス風---(温製)!ふらいとふれつとそーす@フライドブレッドソース}
\index{はん@パン!そーすふらいとふれつといきりすふう@フライドブレッドソース(イギリス風)}
\index{sauce@sauce!anglaises chaudes@---s anglaises chaudes!pain frit@--- au Pain frit (Fried bread Sauce)}
\index{pain@pain!sauce anglaise pain frit@Sauce au Pain frit (Fried bread Sauce)}
\index{anglais@anglais(e)!sauces chaudes@sauces ---es chaudes!pain frit@Sauce au Pain frit (Fried bread Sauce)}

\protect\hyperlink{}{コンソメ}2
dLに、小さなさいの目に切った脂身のないハム80 gとエシャ
ロット2個のみじん切りを加える。弱火で10分間煮る\footnote{mijoter
  ミジョテ。弱火で煮込むこと。}。

その間に、バター50 gを熱してパンの身50 gを揚げておく。提供直前に、揚げ
たパンをコンソメに入れる。パセリのみじん切り1つまみとレモンの搾り汁少々
で仕上げる。

\ldots{}\ldots{} このレシピは小鳥\footnote{つぐみ(grive
  グリーヴ)など小さな鳥類のローストは、下処理し
  た後に胸肉の部分を豚背脂のシートで一羽ずつ包み、数羽をまとめて串刺しにしてロー
  ストするのが一般的だった。}のロースト用。

\maeaki

\hypertarget{perseley-sauce}{%
\subsubsection{パセリソース}\label{perseley-sauce}}

\frsub{Sauce Persil\hspace{1em}\normalfont(\textit{Persley Sauce})}

\index{いきりすふう@イギリス風!そーすおんせい@---ソース(温製)!はせりそーす@パセリソース}
\index{そーす@ソース!いきりすふうおんせい@イギリス風---(温製)!はせりそーす@パセリソース}
\index{はせり@パセリ!そーすはせりいきりすふう@パセリソース(イギリス風)}
\index{sauce@sauce!anglaises chaudes@---s anglaises chaudes!persil@--- Persil (Perseley Sauce)}
\index{persil@persil!sauce anglaise@Sauce Persil (Perseley Sauce)}
\index{anglais@anglais(e)!sauces chaudes@sauces ---es chaudes!persil@Sauce Persil (Perseley Sauce)}

\protect\hyperlink{bread-sauce}{イギリス風バターソース}\undemi{}
Lに、パセリの香りを煮 出した湯\footnote{infusion アンフュジオン
  \textless{} infuser アンフュゼ(煎じる、香りなど
  を煮出す)。なお、いわゆるハーブティはthéテよりもむしろ、infusion
  と呼ばれるのが一般的。}1
dLを加える。みじん切りして下茹でした\footnote{blanchir
  ブランシール。下茹ですること。モスカールド(葉の縮れる
  タイプ)のパセリは葉が厚く固くなりやすいためにこの作業の指示が書か
  れているのだろう。新鮮で柔らかいパセリの葉であれば、細かく刻んでそ
  のまま用いた方がいい結果を得られる。}パセリの葉大さ
じ1杯強を加えて仕上げる。

\ldots{}\ldots{}仔牛の頭肉、仔牛の足、脳などに合わせる。

\maeaki

\hypertarget{ux9b5aux6599ux7406ux7528ux30d1ux30bbux30eaux30bdux30fcux30b945bis}{%
\subsubsection[魚料理用パセリソース]{\texorpdfstring{魚料理用パセリソース\footnote{このソースは第二版から。英語名は付されていない。}}{魚料理用パセリソース}}\label{ux9b5aux6599ux7406ux7528ux30d1ux30bbux30eaux30bdux30fcux30b945bis}}

\frsub{Sauce Persil pour Poissons}

\index{いきりすふう@イギリス風!そーすおんせい@---ソース(温製)!はせりそーすさかなりようりよう@パセリソース(魚料理用)}
\index{そーす@ソース!いきりすふうおんせい@イギリス風---(温製)!さかなりようりようはせりそーす@魚料理用パセリソース}
\index{はせり@パセリ!そーすはせりいきりすふうさかなりようりよう@パセリソース(魚料理用、イギリス風)}
\index{sauce@sauce!anglaises chaudes@---s anglaises chaudes!persil poissons@--- Persil pour Poissons}
\index{persil@persil!sauce anglaise poissons@Sauce Persil pour Poissons}
\index{anglais@anglais(e)!sauces chaudes@sauces ---es chaudes!persil poissons@Sauce Persil pour Poissons}

\protect\hyperlink{roux-blanc}{白いルー}60
gを、このソースを合わせる魚に火を通すのに使ったクールブイヨン\undemi{}
Lでのばす。クールブイヨンはパセリの香りをしっかり効かせたものであること。そうでない場合は、パセリの香りを煮出した湯を加えてこのソースの特徴をきちんと出してやること。

5〜6分間煮て、細かく刻んで下茹でしたパセリの葉大さじ1杯とレモン果汁少々で仕上げる。

\maeaki

\hypertarget{apple-sauce}{%
\subsubsection{アップルソース}\label{apple-sauce}}

\frsub{Sauce aux pommes\hspace{1em}\normalfont(\textit{Apple Sauce})}

\index{いきりすふう@イギリス風!そーすおんせい@---ソース(温製)!あつふるそーす@アップルソース}
\index{そーす@ソース!いきりすふうおんせい@イギリス風---(温製)!あつふるそーす@アップルソース}
\index{りんこ@リンゴ!あつふるそーす@アップルソース(イギリス風)}
\index{sauce@sauce!anglaises chaudes@---s anglaises chaudes!pommes@ aux Pommes (Apple Sauce)}
\index{pomme@pomme!sauce anglaise@Sauce aux Pommes (Apple Sauce)}
\index{anglais@anglais(e)!sauces chaudes@sauces ---es chaudes!pommes@Sauce aux Pommes (Apple Sauce)}

普通にリンゴのマーマレードを作る。砂糖ごく少なめにし、シナモンの粉末を
ほんの少量加えること。\ldots{}\ldots{}これを提供直前に泡立て器で滑らかになるまでよ
く混ぜる。

\ldots{}\ldots{}このマーマレードは微温い温度で供する。鴨、がちょう、豚のローストなど、何にでも合う。

\hypertarget{nota-apple-sauce}{%
\subparagraph{【原注】}\label{nota-apple-sauce}}

ある種のローストにこのマーマレードを添えるというのは、とくにイギリスに
限ったものではない。ドイツ、ベルギー、オランダでも同様に行なわれている
ことだ。

これらの国では、ジビエのローストにはリンゴかコケモモのマーマレード、あ
るいは果物のコンポート(冷製、温製どちらも)のいずれかを必ず添えるもの
だ\footnote{果物のコンポートへの言及は第三版から。また、1907年の英語版\emph{A
  Guide to Modern Cookery}には原注そのものがない。英語版のレシピは
  「中位の大きさのリンゴ2ポンド(約900 g)を四つ割りにして皮を剥き、
  芯を取り除いて刻む。これをシチュー鍋に入れ、大さじ1杯の砂糖とシナ
  モン少々、水を大さじ2〜3杯加える。蓋をして弱火にかけて煮る。提供直
  前に泡立て器で滑らかにする。このソースは微温い温度で、鴨、がちょう、
  うさぎのローストなどに添える」(p.45)となっている。}。

\maeaki

\hypertarget{porto-wine-sauce}{%
\subsubsection{ポートワインソース}\label{porto-wine-sauce}}

\frsub{Sauce au Porto\hspace{1em}\normalfont(\textit{Porto Wine Sauce})}

\index{いきりすふう@イギリス風!そーすおんせい@---ソース(温製)!ほーとわいんそーす@ポートワインソース}
\index{そーす@ソース!いきりすふうおんせい@イギリス風---(温製)!ほーとわいんそーす@ポートワインソース}
\index{ほるとしゆ@ポルト酒!ほーとわいんそーす@ポートワインソース(イギリス風)}
\index{sauce@sauce!anglaises chaudes@---s anglaises chaudes!porto@--- au Porto (Port Wine Sauce)}
\index{porto@porto!sauce anglaise@Sauce au Porto (Porto Wine Sauce)}
\index{anglais@anglais(e)!sauces chaudes@sauces ---es chaudes!porto@Sauce au Porto (Porto Wine Sauce)}

ポルト酒1\undemi{} dLにエシャロットのみじん切り大さじ1杯とタイム1枝を
加えて半量になるまで煮詰める。オレンジ2個とレモン\undemi{}個の搾り汁を
加える。オレンジの外皮の硬い部分を器具でおろしたもの\footnote{zeste
  ゼスト。オレンジやレモンの外皮の硬い部分(ごく表面の部分だけ)を薄く剥いて千切りにしたり、この場合のようにrâpeラップという器具でおろして風味付けに用いる。}小さじ1杯と塩
1つまみ、カイエンヌごく少量を加える。

これを布で漉し、美味しい\protect\hyperlink{jus-de-veau-lie}{とろみを付けた仔牛のジュ}5
dLを加える。

\ldots{}\ldots{}野生の鴨、その他のジビエ全般に合わせる。

\hypertarget{nota-porto-wine-sauce}{%
\subparagraph{【原注】}\label{nota-porto-wine-sauce}}

このイギリス料理のソースは、フランスの多くの飲食店で使われている。

\maeaki

\hypertarget{horse-radish-sauce}{%
\subsubsection{ホースラディッシュソース}\label{horse-radish-sauce}}

\frsub{Sauce Raifort chaude\hspace{1em}\normalfont(\textit{Horse radish Sauce})}

\index{いきりすふう@イギリス風!そーすおんせい@---ソース(温製)!ほーすらていつしゆそーす@ホースラディッシュソース}
\index{そーす@ソース!いきりすふうおんせい@イギリス風---(温製)!ほーすらていつしゆそーす@ホースラディッシュソース}
\index{れふおーる@レフォール!ほーすらていつしゆそーす@ホースラディッシュソース(イギリス風)}
\index{sauce@sauce!anglaises chaudes@---s anglaises chaudes!raifort chaude@--- au Raifort Chaude (Horse radish Sauce)}
\index{raifort@raifort!sauce anglaise chaude@Sauce au Raifort chaude (Horse radish Sauce)}
\index{anglais@anglais(e)!sauces chaudes@sauces ---es chaudes!raifort chaude@Sauce au Raifort chaude (Horse radish Sauce)}

\protect\hyperlink{albert-sauce}{アルバートソース}の別名。

\maeaki

\hypertarget{reform-sauce}{%
\subsubsection[リフォームソース]{\texorpdfstring{リフォームソース\footnote{19世紀ロンドンの会員制クラブ、リフォームでフランス人料理長アレクシス・ソワイエが考案したソース。このような場合、Reformを固有名詞扱いとして英語のままとするのが現代のフランス語における考え方だが、20世紀初頭にはまだ、固有名詞さえもフランス語的に言い換えることがごく普通であった。}}{リフォームソース}}\label{reform-sauce}}

\frsub{Sauce Réforme\hspace{1em}\normalfont(\textit{Reform Sauce})}

\index{いきりすふう@イギリス風!そーすおんせい@---ソース(温製)!りふおーむそーす@リフォームソース}
\index{そーす@ソース!いきりすふうおんせい@イギリス風---(温製)!りふおーむそーす@リフォームソース}
\index{りふおーむ@リフォーム!りふおーむそーす@リフォームソース(イギリス風)}
\index{sauce@sauce!anglaises chaudes@---s anglaises chaudes!reforme@--- Réforme (Reform Sauce)}
\index{reform@reform!sauce anglaise@Sauce Réforme (Reform Sauce)}
\index{anglais@anglais(e)!sauces chaudes@sauces ---es chaudes!reform@Sauce Réforme (Reform Sauce)}

\protect\hyperlink{sauce-poivrade}{ソース・ポワヴラード}と\protect\hyperlink{sauce-demi-glace}{ソース・ドゥミグラス}を合わせ、ガルニチュールとして1〜2
mmの細さで短かめの千切り\footnote{julienne courte ジュリエンヌクルト。}にした中位のサイズのコルニション2個、固茹で卵の白身、中位の大きさのマッシュルーム2個、トリュフ20
gおよび赤く漬けた牛舌肉\footnote{langue écarlate ラングエカルラット。}を加える。

\ldots{}\ldots{}このソースは「リフォーム風」羊のコトレット\footnote{côtelette
  コトレット。羊、仔牛、仔羊の肋骨付きでカットした背肉のこと。牛の場合はcôteコットと呼ばれるが、côteという語そのものは元来「肋骨」の意。côtelette
  の -ette
  は「縮小辞」といって、より小さいものという意味を付加している。つまり、牛のcôteよりも小さいからcôteletteとなる。なおこの語が日本語の「カツレツ」の語源だといわれている。}用。

\maeaki

\hypertarget{sage-and-onions-sauce}{%
\subsubsection{セージと玉ねぎのソース}\label{sage-and-onions-sauce}}

\frsub{Sauce Sauge et Oignons\hspace{1em}\normalfont(\textit{Sage and onions Sauce})}

\index{いきりすふう@イギリス風!そーすおんせい@---ソース(温製)!せーしとたまねきのそーす@セージと玉ねぎのソース}
\index{そーす@ソース!いきりすふうおんせい@イギリス風---(温製)!せーしとたまねきのそーす@セージと玉ねぎのソース}
\index{たまねき@玉ねぎ!せーしとたまねきのそーす@セージと玉ねぎのソース(イギリス風)}
\index{せーし@セージ!せーしとたまねきのそーす@セージと玉ねぎのソース(イギリス風)}
\index{sauce@sauce!anglaises chaudes@---s anglaises chaudes!sauge oignons@--- Sauge et Oignons (Sage and onions Sauce)}
\index{oignon@oignon!sauce sauge anglaise@Sauce Sauge et Oignons (Sage and onions Sauce)}
\index{sauge@sauge!sauce oignons anglaise@Sauce Sauge et Oignons (Sage and onions Sauce)}
\index{anglais@anglais(e)!sauces chaudes@sauces ---es chaudes!sauge oignons@Sauce Sauge et Oignons (Sage and onions Sauce)}

大きめの玉ねぎ2個をオーブンで焼く。冷めたら皮を剥き、みじん切りにする
\footnote{hacher アシェ。}。パンの身150
gを牛乳に浸してから圧しつぶして水分を抜く。これを玉 ねぎにを混ぜ込む。

セージのみじん切り大さじ2杯と塩、こしょうで調味する。

\ldots{}\ldots{}これは鴨の詰め物にする。

\hypertarget{nota-sage-and-onions-sauce}{%
\subparagraph{【原注】}\label{nota-sage-and-onions-sauce}}

鴨をローストした際のジュを大さじ5〜6杯この詰め物に加えてソース入れで供する。

パンの身と同量の牛の脂身を茹でてみじん切りにしたものを加えることも多い。

\maeaki

\hypertarget{sauce-yorkshire}{%
\subsubsection[ヨークシャーソース]{\texorpdfstring{ヨークシャーソース\footnote{このレシピは初版からほぼ異同がなく(初版ではソース名が
  Yorkshire Sauceだったことと、「仔鴨とハムに合わせる」だったのが第
  二版で現在とまったく同じになることのみ)、原注もない。1907年版の英
  語版にも掲載されていないが、1903年アメリカ、シカゴで刊行された
  『\href{https://archive.org/details/stewardshandbook00whitiala}{スチュワードハンドブッ
  ク}』のソー
  スの項目のなかに、「ヨークシャーソース\ldots{}\ldots{}ハム用のオレンジソース。
  エスパニョル、カラント(=グロゼイユ)ゼリー、ポートワイン、オレン
  ジジュース、茹でて千切りにしたオレンジの外皮」(p.434)とあり、エス
  コフィエの『料理の手引き』初版当時には既にアメリカで知られているソー
  スであったことがわかる。ただしイギリスのヨークシャー州とどのような
  関係あるいはソース名の由来があるのかは不明。}}{ヨークシャーソース}}\label{sauce-yorkshire}}

\frsub{Sauce Yorkshire}

\index{いきりすふう@イギリス風!そーすおんせい@---ソース(温製)!よーくしやーそーす@ヨークシャーソース}
\index{そーす@ソース!いきりすふうおんせい@イギリス風---(温製)!よーくしやーそーす@ヨークシャーソース}
\index{よーくしやー@ヨークシャー!よーくしやーそーす@ヨークシャーソース(イギリス風)}
\index{sauce@sauce!anglaises chaudes@---s anglaises chaudes!yorkshire@--- Yorkshire}
\index{yorkshire@Yorkshire!sauce anglaise@Sauce Yorkshire}
\index{anglais@anglais(e)!sauces chaudes@sauces ---es chaudes!yorkshire@Sauce Yorkshire}

オレンジの外皮の硬い表面だけを薄く削って細かい千切りにしたもの大さじ1
杯強を、ポルト酒2 dLでしっかり茹でる。

オレンジの皮の千切りを取り出して水気をきる。ポルト酒の入った鍋に、\protect\hyperlink{sauce-espagnole}{ソー
ス・エスパニョル}大さじ1杯強と、\protect\hyperlink{}{グロゼイユのジュ
レ}も大さじ1杯強を加える。粉末のシナモン少々と、カイエンヌ少々を加
える。

わずかの時間、煮詰める。布で漉し、オレンジ1個の搾り汁と千切りにした皮
を加えて仕上げる。

\ldots{}\ldots{}仔鴨のローストやブレゼ、およびハムのブレゼに添える。
\end{recette}