\hypertarget{ux30a4ux30aeux30eaux30b9ux98a8ux30bdux30fcux30b9ux6e29ux88fd24}{%
\section[イギリス風ソース(温製)]{\texorpdfstring{イギリス風ソース(温製)\footnote{この節では初版で31、第二版は33、第三版と第四版で30のレシピが掲
  載されている。1907年刊の英語版\emph{A Guide to Modern Cookery}でこの節
  に相当する``Hot English Sauces''には10のレシピしか掲載されていない。
  この大きな数の差をどう解釈するかは意見の分かれるところだろうが、対
  象読者がフランス人であるかイギリス人であるかという違いを意識し、ニー
  ズに応えるかたちをとったと考えるのが妥当だろう。ただし、あくまでも
  エスコフィエあるいは共同執筆者の解釈を経た「イギリス風」のソースが
  ほとんどであることは、例えば「\protect\hyperlink{roe-buck-sauce}{ローバックソース}」
  において\protect\hyperlink{sauce-espagnole}{ソース・エスパニョル}を用いていること、
  つまりはエスコフィエが構築したソースの体系に組み込まれ得るものであ
  ることから判断がつく。}}{イギリス風ソース(温製)}}\label{ux30a4ux30aeux30eaux30b9ux98a8ux30bdux30fcux30b9ux6e29ux88fd24}}

\hypertarget{sauces-anglaises-chaudes}{%
\subsection{Sauces Anglaises Chaudes}\label{sauces-anglaises-chaudes}}

\index{いきりすふう@イギリス風!そーすおんせい@ソース(温製)}
\index{sauce@sauce!anglais chaud@--- anglaise chaude}
\index{anglais@anglais!sauces chaudes@sauces anglaises chaudes}
\begin{recette}
\hypertarget{ux30afux30e9ux30f3ux30d9ux30eaux30fc1ux30bdux30fcux30b9}{%
\subsubsection[クランベリーソース]{\texorpdfstring{クランベリー\footnote{英語のcranberryはツルコケモモ(学名Vaccinium
  oxycoccos)であり、 フランス語airelles rougesはコケモモ(学名Vaccinium
  vitis-idaea
  L.)で、非常によく似た近縁種であり、しばしば混同される。本書でもと
  くに区別されていない。}ソース}{クランベリーソース}}\label{ux30afux30e9ux30f3ux30d9ux30eaux30fc1ux30bdux30fcux30b9}}

\hypertarget{cranberries-sauce}{%
\paragraph{\texorpdfstring{Sauce aux Airelles
(\emph{Cranberries-Sauce})}{Sauce aux Airelles (Cranberries-Sauce)}}\label{cranberries-sauce}}

\index{いきりすふう@イギリス風!そーすおんせい@ソース(温製)!くらん
へりー@クランベリーソース} \index{そーす@ソース!くらんへりー@クランベ
リー---} \index{くらんへりー@クランベリー!そーす@---ソース}
\index{sauce@sauce!airelles@--- aux Airelles}
\index{airelle@airelle!sauce airelles@Sauce aux Airelles}
\index{sauce@sauce!cranberries@Cranberries-Sauce}
\index{sauce@sauce!anglais chaud@--- anglaise chaude!sauce aux
Airelles (Cranberries-Sauce} \index{anglais@anglais!sauces
chaudes@sauces anglaises chaudes!sauce aux Airelles
(Cranberries-Sauce)} \index{cranberry!Cranberries-Sauce}

クランベリー500 gを1
Lの湯で、鍋に蓋をして茹でる。果肉に火が通ったら、湯をきって、目の細かい網で裏漉しする。

こうして出来たピュレに茹で汁を適量加えてやや濃度のあるソースの状態にする。好みに応じて砂糖を加える。

このソースは市販品があり\footnote{\protect\hyperlink{sauce-robert-escoffier}{ソース・ロベール・エスコフィエ}などの
  ようなエスコフィエブランドの商品というわけではないと思われる。}、水少々を加えて温めるだけで使える。

\ldots{}\ldots{}七面鳥のロースト用。

\maeaki

\hypertarget{ux30a2ux30ebux30d0ux30fcux30c8ux30bdux30fcux30b9}{%
\subsubsection{アルバートソース}\label{ux30a2ux30ebux30d0ux30fcux30c8ux30bdux30fcux30b9}}

\hypertarget{albert-sauce}{%
\paragraph[Sauce Albert (\emph{Albert-Sauce})]{\texorpdfstring{Sauce
Albert\footnote{ザクセン=コーブルク=ゴータ公アルバート王配(ヴィクトリア女王の
  夫)(1819〜1861)のこと。女王エリザベス二世の高祖父。本書序文p.ii
  において触れられている料理人エルーイがアルバート王配に仕えていたこ
  とがある。なお、本書に掲載されていないが、Sole Albert 「舌びらめ 
  アルベール」という料理がある。しかしながら、これはパリのレストラン、
  マキシムズMaxim'sでメートルドテルを務めたアルベール・ブラゼール Albert
  Blazerの名を冠したもので1930年代に創案されたもの。このソー
  スとはまったく関係がないことに注意。}
(\emph{Albert-Sauce})}{Sauce Albert (Albert-Sauce)}}\label{albert-sauce}}

\index{いきりすふう@イギリス風!そーすおんせい@ソース(温製)!あるはー
と@アルバートソース} \index{そーす@ソース!あるべーる@---・アルベール}
\index{あるへーる@アルベール!そーす@ソース・---} \index{あるはーと@ア
ルバート!そーす@---ソース} \index{sauce@sauce!albert@--- Albert}
\index{albert@Albert!sauce@Sauce ---}
\index{albert@Albert!sauce@Albert-Sauce} \index{sauce@sauce!anglais
chaud@--- anglaise chaude!Sauce Albert (Albert-Sauce)}
\index{anglais@anglais!sauces chaudes@sauces anglaises chaudes!Sauce
Albert (Albert-Sauce}

すりおろしたレフォール\footnote{raifort ホースラディッシュ、西洋わさび。}150
gに\protect\hyperlink{}{白いコンソメ}2 dlを注ぎ、弱火で20分間煮る。

\protect\hyperlink{sauce-au-beurre-a-l-anglaise}{イギリス式バターソース}3
dlと生クリーム2\undemi{} dl、パンの白い身の部分40
gを加える。強火にかけて煮詰め、木ヘラで圧し絞るようにしながら布で漉す\footnote{二人で作業すると容易。\protect\hyperlink{veloute}{ヴルテ}訳注参照。}。卵黄2個を加えてとろみを付
け\footnote{このソースの特徴として、イギリスのローストビーフに欠かせないもの
  とされるレフォール(ホースラディッシュ)を用いていることの他に、と
  ろみ付けにパンと卵黄を使っている点にも注目すべきだろう。とろみ付け
  の要素としてはきわめて中世料理風と言ってもいい。ただし、中世の料理
  では、パンはこんがりと焼いてからヴィネガーなどでふやかしてよくすり
  潰し、さらに布で漉してとろみ付けに用いるのが一般的だった。パンの白
  い身の部分をそのまま使えるということは、それだけ小麦の精白度合いが
  高いということでもある。}、塩1つまみとこしょう少々で味を調える。

仕上げに、マスタード小さじ1杯をヴィネガー大さじ1杯で溶いてから加える。

\ldots{}\ldots{}牛肉、主としてフィレ肉のブレゼに添える。

\maeaki

\hypertarget{ux30a2ux30edux30deux30c6ux30a3ux30c3ux30afux30bdux30fcux30b9}{%
\subsubsection{アロマティックソース}\label{ux30a2ux30edux30deux30c6ux30a3ux30c3ux30afux30bdux30fcux30b9}}

\hypertarget{aromatic-sauce}{%
\paragraph{\texorpdfstring{Sauce aux Aromates
(\emph{Aromatic-Sauce})}{Sauce aux Aromates (Aromatic-Sauce)}}\label{aromatic-sauce}}

\index{いきりすふう@イギリス風!そーすおんせい@ソース(温製)!あろま
ていつく@アロマティックソース}
\index{こうそう@香草!あろまていつく@アロマティッ
ク---} \index{そーす@ソース!あろまていつく@アロマティック---}
\index{sauce@sauce!aromates@--- aux Aromates}
\index{aromate@aromate!sauce@Sauce aux Aromates}
\index{sauce@sauce!anglais chaud@--- anglaise chaude!Sauce aux
Aromates (Aromatic-Sauce)} \index{anglais@anglais!sauces
chaudes@sauces anglaises chaudes!Sauce aux Aromates (Aromatic-Sauce)}

\protect\hyperlink{}{コンソメ}\undemi{} Lに、タイム1枝、バジル4
g、サリエット\footnote{シソ科の香草。サマーセイヴォリー。和名キダチハッカ。}1
g、マジョラム1 g、セージ1 g、シブレット\footnote{ciboulette
  チャイヴ。アサツキと訳されることもあるが、日本のアサツキとは風味が違うので注意。}1を刻んだもの1つまみ、エシャロット\footnote{玉ねぎによく似ているが小さくて水分量の少ない香味野菜。英語由来のシャロットと呼ばれることも。日本の青果マーケットに見られる「エシャレット」はらっきょうの若どりであってまったく別のもの。}2個のみじん切り、ナツメグ少々、大粒のこしょう4個を入れて、10分
間煎じる\footnote{infuser アンフュゼ。}。

シノワ\footnote{円錐形で取っ手の付いた漉し器。}で漉し、バターで作った\footnote{本書第四版ではルーは必ずバターを用いる指示がなされているが、初版から第三版までは、バターもしくはグレスドマルミット(コンソメなどを作る際に浮いてきた油脂をすくい取って漉したもの)を使うという指示だっため、「バターで作った」という記述がこのように残っているレシピが散見される。}ブロンドのルー50
gを入れてとろみを付ける。数分間沸かしてから、レモン\undemi{}個分の搾り汁と、みじん切りにして下茹でしておいたセルフイユ\footnote{cerfeuil
  チャービル。}とエストラゴン\footnote{estragon フレンチタラゴン。}計大さじ1杯を加えて仕上げる\footnote{このソースで用いられている香草類の種類の多さは特筆に値するだろう。ブラウン系の派生ソースにある\protect\hyperlink{sauce-aux-fines-herbes}{香草ソース}およびホワイト系派生ソースの\protect\hyperlink{sauce-aux-fines-herbes-blanche}{香草ソース}と比較されたい。}。

\ldots{}\ldots{}大きな魚まるごと1尾のポシェあるいは牛、羊肉の大掛かりな仕立て(ルルヴェ\footnote{relevé
  \protect\hyperlink{sauce-diplomate}{ソース・ディプロマット}訳注参照。})に添える。

\maeaki

\hypertarget{ux30d0ux30bfux30fcux30bdux30fcux30b9}{%
\subsubsection{バターソース}\label{ux30d0ux30bfux30fcux30bdux30fcux30b9}}

\hypertarget{butter-sauce}{%
\paragraph{\texorpdfstring{Sauce au Beurre à l'anglaise (\emph{Butter
Sauce})}{Sauce au Beurre à l'anglaise (Butter Sauce)}}\label{butter-sauce}}

\index{いきりすふう@イギリス風!そーすおんせい@ソース(温製)!はたーそーす@バターソース}
\index{そーす@ソース!はたーいきりすふう@バター---(イギリス風)}
\index{はたー@バター!そーすいきりすふう@---ソース(イギリス風)}
\index{sauce@sauce!beurre anglaise@--- au Beurre à l'anglaise}
\index{butter@butter!sauce beurre anglaise@Sauce au Beurre à l'anglaise}
\index{anglais@anglais!sauces chaudes@sauces anglaises chaudes!Sauce au Beurre à l'anglaise (Butter Sauce)}

フランスの\protect\hyperlink{sauce-au-beurre}{ソース・オ・ブール}と同様に作るが、より濃度の高い仕上りにする点が違う。分量は、バター60
g、小麦粉60 g、1 Lあたり塩7 gを加えて沸かした湯\troisquarts{}
L。レモンの搾り汁5〜6滴、バター 200 g。とろみ付け用の卵黄は用いない。

\maeaki

\hypertarget{ux30b1ux30a4ux30d1ux30fcux30bdux30fcux30b9}{%
\subsubsection{ケイパーソース}\label{ux30b1ux30a4ux30d1ux30fcux30bdux30fcux30b9}}

\hypertarget{capers-sauce}{%
\paragraph{\texorpdfstring{Sauce aux Câpres
(\emph{Capers-Sauce})}{Sauce aux Câpres (Capers-Sauce)}}\label{capers-sauce}}

\index{いきりすふう@イギリス風!そーすおんせい@ソース(温製)!けいはー@ケイパーソース}
\index{そーす@ソース!けいはーいきりすふう@ケイパー---(イギリス風)}
\index{けいはー@ケイパー!そーすいきりすふう@---ソース(イギリス風)}
\index{sauce@sauce!capres anglaise@--- aux Câpres (Capers-Sauce)}
\index{capre@câpre!sauce capres anglaise@Sauce aux Câpres (Capers-Sauce)}
\index{anglais@anglais!sauces chaudes@sauces anglaises chaudes!Sauce aux Câpres (Capers-Sauce)}

上記の\protect\hyperlink{butter-sauce}{バターソース}1
Lあたり大さじ4杯のケイパーを加えたもの。

\ldots{}\ldots{}茹でた魚に添える。また、イギリス風\footnote{à l'anglaise
  アラングレーズ。茹でる(下茹でも含む)場合には、塩を加えた湯で茹でることを指す。なお、パン粉衣
  pané à l'anglaise
  という場合には、現代の日本でもなじみのある、小麦粉、溶きほぐした卵、パン粉の順で衣を付けて揚げることを言う。調理法全体を通しての規則性はなく、あくまでも「イギリス風に由来する」または「イギリス風」を意味するものなので注意。}に茹でた仔羊腿肉には欠かせない。

\maeaki

\hypertarget{ux30bbux30edux30eaux30bdux30fcux30b9}{%
\subsubsection{セロリソース}\label{ux30bbux30edux30eaux30bdux30fcux30b9}}

\hypertarget{celery-sauce}{%
\paragraph{\texorpdfstring{Sauce au Céleri
(\emph{Celery-Sauce})}{Sauce au Céleri (Celery-Sauce)}}\label{celery-sauce}}

\index{いきりすふう@イギリス風!そーすおんせい@ソース(温製)!せろり@セロリソース}
\index{そーす@ソース!けいはーいきりすふう@セロリ---(イギリス風)}
\index{せろり@セロリ!そーすいきりすふう@---ソース(イギリス風)}
\index{sauce@sauce!celeri anglaise@--- au Céleri (Celery-Sauce)}
\index{celeri@céleri!sauce celeri anglaise@Sauce au Céleri (Celery-Sauce)}
\index{anglais@anglais!sauces chaudes@sauces anglaises chaudes!Sauce au Céleri (Celery-Sauce)}

セロリ6株を掃除して、芯のところだけを使う\footnote{緑色が薄いタイプのセロリは中心部が自然に軟白され、柔らかいので、フランス料理でも非常に好まれる。}。これをソテー鍋に並べ、\protect\hyperlink{}{白いコンソメ}をセロリがかぶるまで注ぐ。ブーケガルニとクローブを刺した玉ねぎ1個を入れ、弱火で加熱する。

セロリの水気をきり、鉢に入れてすり潰す。これを布で漉す。こうして出来たセロリのピュレと同量の\protect\hyperlink{cream-sauce}{クリームソース}を加える。セロリの茹で汁を煮詰めたものを大さじ2〜3杯加える。

沸騰しない程度に温め、すぐに提供しない場合は湯煎にかけておく。

\ldots{}\ldots{}茹でた鶏または鶏のブレゼに添える。

\maeaki

\hypertarget{ux30edux30fcux30d0ux30c3ux30afux30bdux30fcux30b9}{%
\subsubsection{ローバックソース}\label{ux30edux30fcux30d0ux30c3ux30afux30bdux30fcux30b9}}

\hypertarget{roe-buck-sauce}{%
\paragraph[Sauce Chevreuil ()]{\texorpdfstring{Sauce Chevreuil
(\emph{Roe-buck\footnote{英語でノロ鹿のこと。}
Sauce})}{Sauce Chevreuil (Roe-buck Sauce)}}\label{roe-buck-sauce}}

\index{いきりすふう@イギリス風!そーすおんせい@ソース(温製)!ろーはつく@ローバックソース}
\index{そーす@ソース!ろーはっくいきりすふう@ローバック---(イギリス風)}
\index{ろーはつく@ローバック!そーすいきりすふう@---ソース(イギリス風)}
\index{のろしか@ノロ鹿!そーすいきりすふう@ローバックソース(イギリス風)}
\index{しゆうるいゆ@シュヴルイユ!ろーはつくそーす@ローバックソース(イギリス風)}
\index{sauce@sauce!chevreuil anglaise@--- Chevreuil (Roe-buck Sauce)}
\index{chevreuil@chevreuil!sauce chevreuil anglaise@Sauce Chevreuil (Roe-buck Sauce)}
\index{anglais@anglais!sauces chaudes@sauces anglaises chaudes!Sauce Chevreuil (Roe-buck Sauce)}

中位の大きさの玉ねぎを1cm角くらいの粗みじん切\footnote{paysanne
  ペイザンヌに切る、と言う。主として野菜について言うが、1 cm角で厚さ1〜2
  mm程度。}りにし、生ハム80gも同様に刻む。これをバターで軽く色付くまで炒める。ブーケガルニを入れ、ヴィネガー1\undemi{}
dlを注ぎ、ほとんど完全に煮詰める。

\protect\hyperlink{sauce-espagnole}{ソース・エスパニョル}3
dlを注ぎ、15分程弱火にかけて、浮いてくる不純物を取り除く\footnote{dépouiller
  デプイエ ≒ écumer エキュメ。}。

15分経ったら、ブーケガルニを取り出し、ポルト酒コップ1杯\footnote{約1
  dl。}と\protect\hyperlink{}{グロゼイユのジュレ}大さじ1杯強を加えて仕上げる。

\ldots{}\ldots{}大型ジビエ肉\footnote{この場合は当然、ノロ鹿の料理だが、フランス料理でノロ鹿は時間をかけてマリネしてから調理し、そのマリナード(漬け汁)もソースに用いるのと比べると非常にシンプルなソースになっている点が興味深い。}の料理に添える。

\maeaki

\hypertarget{ux30afux30eaux30fcux30e0ux30bdux30fcux30b9}{%
\subsubsection{クリームソース}\label{ux30afux30eaux30fcux30e0ux30bdux30fcux30b9}}

\hypertarget{cream-sauce}{%
\paragraph{\texorpdfstring{Sauce Crème à l'anglaise
(\emph{Cream-Sauce})}{Sauce Crème à l'anglaise (Cream-Sauce)}}\label{cream-sauce}}

\index{いきりすふう@イギリス風!そーすおんせい@ソース(温製)!くりーむ@クリームソース}
\index{そーす@ソース!くりーむいきりすふう@クリーム---(イギリス風)}
\index{くりーむ@クリーム!そーすいきりすふう@---ソース(イギリス風)}
\index{sauce@sauce!creme anglaise@--- Crème à l'anglaise (Cream-Sauce)}
\index{cremee@crème!sauce creme anglaise@Sauce Crème à l'anglaise (Cream-Sauce)}
\index{anglais@anglais!sauces chaudes@sauces anglaises chaudes!Sauce Crème à l'anglaise (Cream-Sauce)}

バター100 gと小麦粉60
gで\protect\hyperlink{roux-blanc}{白いルー}を作る。

\protect\hyperlink{}{白いコンソメ}7
dlでルーをのばし、マッシュルームのエッセンス1 dlと生クリーム2
dlを加える。

火にかけて沸騰させる。小玉ねぎ1個とパセリ1束を加え、弱火で15分程煮込む。提供直前に小玉ねぎとパセリは取り出す。

\ldots{}\ldots{}仔牛の骨付き背肉の塊\footnote{carré
  カレ。もとは「四角形」の意。料理では、肋骨ごとに切り分けていない仔牛および仔羊の骨付き背肉の塊を指す。}のローストに合わせる。

\maeaki

\hypertarget{ux30b7ux30e5ux30eaux30f3ux30d7ux30bdux30fcux30b9}{%
\subsubsection{シュリンプソース}\label{ux30b7ux30e5ux30eaux30f3ux30d7ux30bdux30fcux30b9}}

\hypertarget{shrimps-sauce}{%
\paragraph{\texorpdfstring{Sauce Crevettes à l'anglaise
(\emph{Shrimps-Sauce})}{Sauce Crevettes à l'anglaise (Shrimps-Sauce)}}\label{shrimps-sauce}}

\index{いきりすふう@イギリス風!そーすおんせい@ソース(温製)!しゆりんふ@シュリンプソース}
\index{そーす@ソース!しゆりんふいきりすふう@シュリンプ---(イギリス風)}
\index{くるうえつと@クルヴェット!そーすいきりすふう@シュリンプソース(イギリス風)}
\index{sauce@sauce!crevettes anglaise@--- Crevettes à l'anglaise (Shrimps-Sauce)}
\index{crevette@crevette!sauce crevette anglaise@Sauce Crevettes à l'anglaise (Shrimps-Sauce)}
\index{anglais@anglais!sauces chaudes@sauces anglaises chaudes!Sauce Crevettes à l'anglaise (Shrimps-Sauce)}

カイエンヌ少量を加えて風味を引き締めた\protect\hyperlink{butter-sauce}{イギリス風バターソース}1
Lに、アンチョビエッセンス小さじ1杯と殻を剥いた小海老\footnote{フランス語は
  crevette(s)
  クルヴェット。\protect\hyperlink{sauce-aux-crevettes}{ソース・クルヴェット}訳注参照。}の尾の身125
gを加える。

\ldots{}\ldots{}魚料理用。

\maeaki

\hypertarget{ux30c7ux30d3ux30ebux30bdux30fcux30b9}{%
\subsubsection{デビルソース}\label{ux30c7ux30d3ux30ebux30bdux30fcux30b9}}

\hypertarget{devilled-sauce}{%
\paragraph{\texorpdfstring{Sauce Diable (\emph{Devilled
Sauce})}{Sauce Diable (Devilled Sauce)}}\label{devilled-sauce}}

\index{いきりすふう@イギリス風!そーすおんせい@ソース(温製)!てひるそーす@デビルソース}
\index{そーす@ソース!てひるいきりすふう@デビル---(イギリス風)}
\index{あくま@悪魔!そーすいきりすふう@デビルソース(イギリス風)}
\index{ていあーふる@ディアーブル!てひるそーす@デビルソース(イギリス風)}
\index{sauce@sauce!diable anglaise@--- Diable (Devilled Sauce)}
\index{diable@diable!sauce diable anglaise@Sauce Diable (Devilled Sauce)}
\index{anglais@anglais!sauces chaudes@sauces anglaises chaudes!Sauce Diable (Devilled Sauce)}

1\undemi{}
dlのヴィネガーにエシャロットのみじん切り大さじ1杯強を加えて、半量になるまで煮詰める。\protect\hyperlink{sauce-espagnole}{ソース・エスパニョル}2\undemi{}
dlとトマトピュレ大さじ2杯を加え、5分間程煮る。

仕上げに、ダービーソース\footnote{原文Derby-sauce、1940年代にアメリカで市販されていたのは確認され
  ているが、ここで言及されているのとまったく同じかは不明。なお、初版
  および第二版でこの部分は「ハーヴェイソースとウスターシャーソース各
  大さじ1杯」、第三版では「ハーヴェイソースとエスコフィエソース各大
  さじ1」となっている。「ダービーソース」が当初「エスコフィエソース」
  として商品化された後に何らかの事情により名称変更がなされたという可
  能性も否定できないが、第二版および英語版において\protect\hyperlink{sauce-diable-escoffier}{ソース・ディアー
  ブル・エスコフィエ}および\protect\hyperlink{sauce-robert-escoffier}{ソース・ロベー
  ル・エスコフィエ}、さらに第二版と同年刊の
  英語版のみに掲載されているSauce aux Cerises Escoffierソース・オ・
  スリーズ・エスコフィエのように既にエスコフィエブランドの既製品ソー
  スがあるために、矛盾が生じてしまう。第三版の記述が\protect\hyperlink{sauce-diable-escoffier}{ソース・ディアー
  ブル・エスコフィエ}を意味していると解釈す
  れば矛盾は生じないだろう。ハーヴェイソースについては\protect\hyperlink{brown-gravy}{ブラウングレ
  イヴィー}訳注参照.}大さじ1杯とカイエンヌ1つまみ強を加え、シノワ\footnote{円錐形で取っ手の付いた漉し器。}か布で漉す。

\maeaki

\hypertarget{ux30b9ux30b3ux30c3ux30c1ux30a8ux30c3ux30b0ux30bdux30fcux30b9}{%
\subsubsection{スコッチエッグソース}\label{ux30b9ux30b3ux30c3ux30c1ux30a8ux30c3ux30b0ux30bdux30fcux30b9}}

\hypertarget{scotch-eggs-sauce}{%
\paragraph{\texorpdfstring{Sauce Ecossaise (\emph{Scotch eggs
Sauce})}{Sauce Ecossaise (Scotch eggs Sauce)}}\label{scotch-eggs-sauce}}

\index{いきりすふう@イギリス風!そーすおんせい@ソース(温製)!すこつちえつくそーす@スコッチエッグソース}
\index{そーす@ソース!すこつとらんといきりすふう@スコッチエッグ---(イギリス風)}
\index{すこつとらんと@スコットランド!すこつちえつくそーす@スコッチエッグソース(イギリス風)}
\index{sauce@sauce!ecossaise anglaise@--- Ecossaise (Scotch eggs Sauce)}
\index{scotland@Scotland!sauce ecossaise anglaise@Sauce Ecossaise (Scotch eggs Sauce)}
\index{anglais@anglais!sauces chaudes@sauces anglaises chaudes!Sauce Ecossaise (Scotch eggs Sauce)}

バター60 gと小麦粉30 g、沸かした牛乳4
dlで\protect\hyperlink{sauce-bechamel}{ベシャメルソース}を用意する。味付けは通常どおりにすること。ソースが沸騰したらすぐに、固茹で卵の白身4個を薄切りにした\footnote{émincer
  エマンセ、薄切りにすること。}ものを加える。

提供直前に、茹で卵の卵黄を目の粗い漉し器で漉したものを混ぜ込む。

\ldots{}\ldots{}\ruby{鱈}{たら}には欠かせないソース。

\maeaki

\hypertarget{ux30d5ux30a7ux30f3ux30cdux30eb30ux30bdux30fcux30b9}{%
\subsubsection[フェンネルソース]{\texorpdfstring{フェンネル\footnote{日本語でフェンネルと呼ばれるものは、(a)主に香草として葉を利用するタイプfenouil
  sauvage(フヌイユソヴァージュ)と、(b)白く肥大した株元を食用とするフローレンス・フェンネルfenouil
  de florence(フヌイユ・ド・フロロンス)またはfenouil
  bulbeux(フヌイユビュルブー)と呼ばれる2種がある。本書ではどちらを用いるのか明記されていないことが多いが、一般に、葉を利用するタイプは香りが非常に強く、フローレンスフェンネルの葉も食用可能だが、香りは比較的おとなしい。}ソース}{フェンネルソース}}\label{ux30d5ux30a7ux30f3ux30cdux30eb30ux30bdux30fcux30b9}}

\hypertarget{fennel-sauce}{%
\paragraph{\texorpdfstring{Sauce au Fenouil (\emph{Fennel
Sauce})}{Sauce au Fenouil (Fennel Sauce)}}\label{fennel-sauce}}

\index{いきりすふう@イギリス風!そーすおんせい@ソース(温製)!ふえんねるそーす@フェンネルソース}
\index{そーす@ソース!ふえんねるいきりすふう@フェンネル---(イギリス風)}
\index{ふえんねる@フェンネル!そーすいきりすふう@フェンネルソース(イギリス風)}
\index{sauce@sauce!fenouil anglaise@--- au Fenouil (Fennel Sauce)}
\index{fenouil@fenouil!sauce fenouil anglaise@Sauce au Fenouil (Fennel Sauce)}
\index{anglais@anglais!sauces chaudes@sauces anglaises chaudes!Sauce au Fenouil (Fennel Sauce)}

普通に作った\protect\hyperlink{butter-sauce}{バターソース}2\undemi{}
dlあたり、細かく刻んで下茹でしたフェンネル大さじ1杯を加える。

\ldots{}\ldots{}このソースは主として、グリルあるいは茹でた鯖に合わせる。

\maeaki

\hypertarget{ux30b0ux30fcux30baux30d9ux30eaux30fcux30bdux30fcux30b9}{%
\subsubsection{グーズベリーソース}\label{ux30b0ux30fcux30baux30d9ux30eaux30fcux30bdux30fcux30b9}}

\hypertarget{gooseberry-sauce}{%
\paragraph{\texorpdfstring{Sauce aux Groseilles (\emph{Gooseberry
Sauce})}{Sauce aux Groseilles (Gooseberry Sauce)}}\label{gooseberry-sauce}}

\index{いきりすふう@イギリス風!そーすおんせい@ソース(温製)!くーすへりーそーす@グーズベリーソース}
\index{そーす@ソース!くーすへりーいきりすふう@グーズベリー---(イギリス風)}
\index{くーすへりー@グーズベリー!そーすいきりすふう@グーズベリーソース(イギリス風)}
\index{すくり@すぐり!そーす@ソース!くーすへりーそーすいきりすふう@グーズベリーソース(イギリス風)}
\index{くろせいゆ@グロゼイユ!そーす@ソース!くーすへりーそーすいきりすふう@グーズベリーソース(イギリス風)}
\index{sauce@sauce!groseilles anglaise@--- aux Groseilles (Gooseberry Sauce)}
\index{groseille@groseille!sauce groseilles anglaise@Sauce aux Groseilles (Gooseberry Sauce)}
\index{anglais@anglais!sauces chaudes@sauces anglaises chaudes!Sauce aux Groseilles (Gooseberry Sauce)}

グーズベリー1 Lの皮を剥いて洗い、砂糖125 gと水1
dlを加えて火にかける。目の細かい漉し器で裏漉しする。

\ldots{}\ldots{}このピュレはグリルした鯖に合わせる。

\maeaki

\hypertarget{ux30edux30d6ux30b9ux30bfux30fcux30bdux30fcux30b9}{%
\subsubsection{ロブスターソース}\label{ux30edux30d6ux30b9ux30bfux30fcux30bdux30fcux30b9}}

\hypertarget{lobster-sauce}{%
\paragraph{\texorpdfstring{Sauce Homard à l'anglaise (\emph{Lobster
Sauce})}{Sauce Homard à l'anglaise (Lobster Sauce)}}\label{lobster-sauce}}

\index{いきりすふう@イギリス風!そーすおんせい@ソース(温製)!ろふすたーそーす@ロブスターソース}
\index{そーす@ソース!ろふすたーいきりすふう@ロブスター---(イギリス風)}
\index{ろふすたー@ロブスター!そーすいきりすふう@ロブスターソース(イギリス風)}
\index{おまーる@オマール!そーす@ソース!ろふすたーそーすいきりすふう@ロブスターソース(イギリス風)}
\index{sauce@sauce!homard anglaise@--- Homard à l'anglaise (Lobster Sauce)}
\index{homard@homard!sauce anglaise@Sauce Homard à l'anglaise (Lobster Sauce)}
\index{anglais@anglais!sauces chaudes@sauces anglaises chaudes!Sauce Homard à l'anglaise (Lobster Sauce)}

カイエンヌを加えて風味を引き締めた\protect\hyperlink{sauce-bechamel}{ベシャメルソース}1
Lに、アンチョビエッセンス大さじ1杯と、さいの目に切ったオマールの尾の身100
gを加える\footnote{ホワイト系派生ソースの節にある\protect\hyperlink{sauce-homard}{ソース・オマール}
  を比較すると、このソースのシンプルさが際立って見えるが、ベシャメル
  を基本ソースにしている点で、やはり「ソースの体系」に組込まれたもの
  であり、純粋にイギリス料理由来というわけでもないと思われる。なお、
  このレシピは初版からほぼ異同がなく、1907年の英語版には含まれていな
  い。}。

\ldots{}\ldots{}魚料理用。

\maeaki

\hypertarget{ux7261ux8823ux5165ux308aux30bdux30fcux30b9}{%
\subsubsection{牡蠣入りソース}\label{ux7261ux8823ux5165ux308aux30bdux30fcux30b9}}

\hypertarget{oyster-sauce}{%
\paragraph{\texorpdfstring{Sauce aux Huîtres (\emph{Oyster
Sauce})}{Sauce aux Huîtres (Oyster Sauce)}}\label{oyster-sauce}}

\index{いきりすふう@イギリス風!そーすおんせい@ソース(温製)!かきいりそーす@牡蠣入りソース}
\index{そーす@ソース!かきいりいきりすふう@牡蠣入り---(イギリス風)}
\index{かき@牡蠣!そーすいきりすふう@牡蠣入りソース(イギリス風)}
\index{sauce@sauce!huitres anglaise@--- aux huitres (Oyster Sauce)}
\index{huitre@hu[itre!sauce anglaise@Sauce aux Huîtres (Oyster Sauce)}
\index{anglais@anglais!sauces chaudes@sauces anglaises chaudes!Sauce aux Huîtres (Oyster Sauce)}

バター20 gと小麦粉15 gでブロンドのルーを作る。

このルーを、牛乳1 dlと生クリーム1 dlで溶く。塩1つまみを加えて調味し、
火にかけて沸騰させたら弱火にして10分間煮る。

布で漉し、カイエンヌを加えて風味を引き締める。沸騰しない程度の温度で火
を通して周囲をきれいに掃除した牡蠣の身12個を1 cm程度の厚さに切って、ソー
スに加える。

\ldots{}\ldots{}もっぱら茹でた魚\footnote{初版および第二版では「もっぱら茹でた生鱈に合わせる」とある。こ
  のレシピも1907年の英語版には掲載されていない。}に添える。

\maeaki

\hypertarget{ux7261ux8823ux5165ux308aux30d6ux30e9ux30a6ux30f3ux30bdux30fcux30b9}{%
\subsubsection{牡蠣入りブラウンソース}\label{ux7261ux8823ux5165ux308aux30d6ux30e9ux30a6ux30f3ux30bdux30fcux30b9}}

\hypertarget{brown-oyster-sauce}{%
\paragraph{\texorpdfstring{Sauce brune aux Huîtres (\emph{Brown Oyster
Sauce})}{Sauce brune aux Huîtres (Brown Oyster Sauce)}}\label{brown-oyster-sauce}}

上記の牡蠣入りソースと作り方はまったく同じだが、牛乳と生クリームではな
く、\protect\hyperlink{fonds-brun}{茶色いフォン}2 dlを使うこと。

\ldots{}\ldots{}このソースは、グリル焼きした肉や、肉のプディング\footnote{本書にはイギリス風の肉料理としてのプディングのレシピも掲載され
  ている。\protect\hyperlink{beefsteak-pudding}{ビーフステークのプディング}、\protect\hyperlink{beefsteak-and-kidney-pudding}{ビーフ
  ステークとキドニーのプディング}、
  \protect\hyperlink{beefsteak-and-oysters-pudding}{ビーフステークと牡蠣のプディング}。
  なお、本書でのbeefsteakビーフステークとは肉の切り方のことを意味し
  ており、グリル焼きあるいはソテーしたもののことではない。ここでは厚 さ1
  cm程度にスライスした牛肉のことを指している。}、生鱈のグリル焼きに合わせる。
\end{recette}