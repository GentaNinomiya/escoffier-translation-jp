\hypertarget{petites-sauces-brunes-composees}{%
\section{ブラウン系の派生ソース}\label{petites-sauces-brunes-composees}}

\frsec{Petites Sauces Brunes Composées}

\index{そーす@ソース!ふらうんはせい@ブラウン系の派生---}
\index{sauce@sauce!petites brunes composees@Petites ---s Brunes Composées}
\begin{recette}
\hypertarget{sauce-bigarade}{%
\subsubsection{ソース・ビガラード}\label{sauce-bigarade}}

\frsub{Sauce Bigarade}\footnote{ビガラードは本来、南フランスで栽培されるビターオレンジの一種。}

\index{そーす@ソース!ひからーと@---・ビガラード}
\index{そーす@ソース!ふらうんはせい@ブラウン系の派生---!ひからーと@---・ビガラード}
\index{ひからーと@ビガラード!そーす@ソース・---}
\index{sauce@sauce!petites brunes composees@Petites ---s Brunes Composées!bigarade@--- Bigarade}
\index{sauce@sauce!bigarade@--- Bigarade}
\index{bigarade@bigarade!sauce@Sauce ---}

\hypertarget{sauce-bigarade-pour-caneton-braise}{%
\subparagraph{仔鴨のブレゼ用}\label{sauce-bigarade-pour-caneton-braise}}

\ldots{}\ldots{}仔鴨をブレゼ\footnote{料理の仕立てとしてのブレゼはたんに「蒸し煮」することではない。原
  則的な手順をごく簡単に述べておく。厚めに輪切りにしたにんじんと玉ね
  ぎをバターまたはラードで炒め、ブーケガルニとともに鍋に入れる。表面
  を色よく焼き固めた肉を、脂身の少ない肉の場合には豚背脂のシートで包
  んで素材がぴったり入る大きさ鍋に入れ、\protect\hyperlink{fonds-brun}{茶色いフォン}
  を注ぎ、蓋をしてオーブンに入れ、微沸騰の状態を保つようにして煮込む。
  火が通ったら肉を取り出し、鍋に残った煮汁でソースを作る。詳細につい
  ては\protect\hyperlink{}{第7章 肉料理}参照。}した際の煮汁を漉してから浮き脂を取り除き\footnote{dégraisser
  デグレセ。}、煮詰める。 煮詰まったらさらに目の細かい布で漉し、ソース1
Lあたりオレンジ4個とレモ ン1個の搾り汁でのばす。

\hypertarget{sauce-bigarade-pour-caneton-poele}{%
\subparagraph{仔鴨のポワレ用}\label{sauce-bigarade-pour-caneton-poele}}

\ldots{}\ldots{}仔鴨をポワレ\footnote{ポワレについても簡単に述べておく。本書においてポワレは「フライパ
  ンで焼く」という意味で用いられることは決してない(フライパンで魚な
  どを焼くことをポワレと呼ぶようになったのは20世紀後半のこと)。本書
  では「ローストの一種」と定義されており(この点がカレームとはまった
  く異なる)、3〜4mm角に切った香味野菜(マティニョン)を生のまま鍋の
  底に入れ、その上に味付けをした肉を置く。溶かしバターをかけてから、
  蓋をして中火のオーブンに入れて蒸し焼きにする。時折様子を見て溶かし
  バターをかけてやること。肉に火が通ったら鍋から取り出し、\protect\hyperlink{jus-de-veau-brun}{茶色い仔
  牛のフォン}を注いで弱火にかけて10分程煮込み、
  マティニョンとして用いた野菜から風味を引き出してソースにする。これ
  がレシピにある「ポワレのフォン」となる。}のフォンから浮き脂を取り除き、でんぷんで軽くとろみ付け
する。砂糖20gに大さじ\undemi{}杯のヴィネガーを加えて火にかけカラメル状
にしたものを加える。ブレゼ用と同様に、オレンジとレモンの搾り汁でのばす。

仔鴨のブレゼ用、ポワレ用いずれの場合も、細かい千切りにしてよく下茹でし
ておいたオレンジの皮大さじ2とレモンの皮\footnote{柑橘類の表皮を薄く剥いてごく細い千切りにしたり、器具を用いてお
  ろしたものをzeste(ゼスト)と呼ぶ。千切りにしたものは苦味を取り除く
  ために下茹ですることが多い。}大さじ1を加えて仕上げる。

\hypertarget{sauce-bordelaise}{%
\subsubsection{ボルドー風ソース}\label{sauce-bordelaise}}

\frsub{Sauce Bordelaise}

\index{そーす@ソース!ほるとーふう@ボルドー風---}
\index{そーす@ソース!ふらうんはせい@ブラウン系の派生---!ほるとーふう@ボルドー風---}
\index{ほるとーふう@ボルドー風!そーす@---ソース}
\index{sauce@sauce!petites brunes composees@Petites ---s Brunes Composées!bordelaise@--- Bordelaise}
\index{sauce@sauce!bordelaise@--- Bordelaise}
\index{bordelais@bordelais(e)!sauce@Sauce Bordelaise}

赤ワイン3 dlにエシャロットのみじん切り大さじ2、粗く砕いたこしょう、タ
イム、ローリエの葉\undemi{}枚を加えて火にかけ、\unquart{}量になるまで
煮詰める。ソース・エスパニョル1 dlを加えて火にかけ、浮いてくる夾雑物を
丁寧に取り除きながら弱火で15分間煮る。目の細かい布で漉す。

溶かした\protect\hyperlink{glace-de-viande}{グラスドヴィアンド}大さじ1杯とレモン汁\unquart{}個分、細かいさ
いの目か輪切りにしてポシェしておいた牛骨髄を加えて仕上げる。

\ldots{}\ldots{}牛、羊の赤身肉のグリル用

【原注】こんにちではボルドー風ソースをこのように赤ワインを用いて作るが、
本来的には誤りである。元来は白ワインが用いられていた。これは\protect\hyperlink{sauce-bonnefoy}{ボルドー
風ソース・ボヌフォワ}として後述。

\hypertarget{sauce-bourguignonne}{%
\subsubsection{ブルゴーニュ風ソース}\label{sauce-bourguignonne}}

\frsub{Sauce Bourguignonne}

\index{そーす@ソース!ふるこーにゆふう@ブルゴーニュ風---}
\index{そーす@ソース!ふらうんはせい@ブラウン系の派生---!ふるこーにゆふう@ブルゴーニュ風---}
\index{ふるこーにゆふう@ブルゴーニュ風!そーす@---ソース}
\index{sauce@sauce!petites brunes composees@Petites ---s Brunes Composées!bourguignonne@--- bourguignonne}
\index{sauce@sauce!bourguignonne@--- Bourguignonne}
\index{bourguignon@bourguignon(ne)!sauce@Sauce Bourguignonne}

上質の赤ワイン1\undemi{} L に、エシャロット5個の薄切りとパセリの枝、タ
イム、ローリエの葉\undemi{}枚、マッシュルームの切りくず\footnote{料理に使うマッシュルームは通常、トゥルネ(包丁を持った側の手は動
  かさずに材料を回して切ることからついた用語)すなわち螺旋状に切って
  供するが、その際に少なくない量の切りくずが出るのでこれを使う。}25gを加えて、
半量になるまで煮詰める。布で漉し、ブールマニエ80g(バター45gと小麦粉
35g)を加えてとろみを付ける。提供直前にバター150gを溶かし込み、カイエ
ンヌ\footnote{赤唐辛子の粉末だがカイエンヌは本来、品種名。日本のタカノツメと
  比べると辛さもややマイルドで、風味も異なる。}ごく少量で加えて風味よく仕上げる。

\ldots{}\ldots{}いろいろな卵料理や、家庭料理に好適なソース。

\hypertarget{sauce-bretonne}{%
\subsubsection{ブルターニュ風ソース}\label{sauce-bretonne}}

\frsub{Sauce Bretonne}

\index{そーす@ソース!ふるたーにゆふうふらうんけい@ブルターニュ風---(ブラウン系)}
\index{そーす@ソース!ふらうんはせい@ブラウン系の派生---!ふるたーにゆふう@ブルターニュ風---}
\index{ふるたーにゆふう@ブルターニュ風!そーすふらうんけい@---ソース(ブラウン系)}
\index{sauce@sauce!petites brunes composees@Petites ---s Brunes Composées!bretonne@--- Bretonne}
\index{sauce@sauce!bretonne brune@--- Bretonne (brune)}
\index{breton@breton(ne)!sauce brune@Sauce Bretonne (brune)}

中位の玉ねぎ2個をみじん切りにして、バターでブロンド色になるまで炒める。
白ワイン2\undemi{} dlを注ぎ、半量になるまで煮詰める。ここにソース・エス
パニョル3\undemi{}
dlおよびトマトソース同量を加える。7〜8分間煮立ててから、
刻んだパセリを加えて仕上げる。

【原注】このソースは\protect\hyperlink{haricots-blancs-bretonne}{白いんげん豆のブルターニュ風}以外にはほとんど使わ
れない。

\hypertarget{sauce-aux-cerises}{%
\subsubsection{ソース・スリーズ}\label{sauce-aux-cerises}}

\frsub{Sauce aux Cerises}\footnote{さくらんぼのこと。このレシピでグロゼイユ(すぐり)のジュレを用い
  るが、古くはさくらんぼを用いていたことからこの名称となったと言われ
  ている。}

\index{そーす@ソース!すりーす@---・スリーズ}
\index{そーす@ソース!ふらうんはせい@ブラウン系の派生---!すりーす@---・スリーズ}
\index{くろせいゆ@グロゼイユ!そーす@ソース!すりーす@---・スリーズ}
\index{さくらんほ@サクランボ!そーす@ソース!すりーす@---・スリーズ}
\index{sauce@sauce!petites brunes composees@Petites ---s Brunes Composées!cerises@--- aux Cerises}
\index{sauce@sauce!cerise@--- aux Cerises}

ポルト酒2 dlにイギリス風ミックススパイス\footnote{Mixed
  spiceのこと。Pudding spiceとも呼ばれる。シナモン、ナツメ
  グ、オールスパイスの組み合わせが典型的。これにクローブ、生姜、コリ
  アンダーシード、キャラウェイシードなどが加わっていることも多い。}1つまみと、すりおろしたオレ
ンジの皮を大さじ\undemi{}杯加えて\deuxtiers{}量になるまで煮詰める。
\protect\hyperlink{gelee-de-groseilles-a}{グロゼイユのジュレ} 2\undemi{}
dlを加え、仕上げにオレンジ果汁を加える。

\ldots{}\ldots{}大型ジビエの料理用だが、鴨のポワレやブレゼにも用いられる。

\hypertarget{sauce-aux-champignons}{%
\subsubsection{ソース・シャンピニョン}\label{sauce-aux-champignons}}

\frsub{Sauce aux Champignons}\footnote{champignons
  キノコ全般を意味する語だが、単独で用いられる場合はい
  わゆるマッシュルームを指す。}

\index{そーす@ソース!まつしゆるーむ@マッシュルーム ⇒ ---・シャンピニョン}
\index{そーす@ソース!ふらうんはせい@ブラウン系の派生---!しやんひによん@---・シャンピニョン}
\index{そーす@ソース!ふらうんはせい@ブラウン系の派生---!まつしゆるーむ@マッシュルーム ⇒ ---・シャンピニョン}
\index{まっしゅるーむ@マッシュルーム ⇒ シャンピニョン}
\index{しやんひによん@シャンピニョン!そーすふらうんけいはせい@ソース・---(ブラウン系)}
\index{sauce@sauce!petites brunes composees@Petites ---s Brunes Composées!champignons@--- aux Champignons}
\index{sauce@sauce!champignons brune@--- aux Champignons (brune)}
\index{champignon@champignon!sauce brune@Sauce aux Champignons (brune)}

マッシュルームの茹で汁2\undemi{} dl を半量になるまで煮詰める。
\protect\hyperlink{sauce-demi-glace}{ソース・ ドゥミグラス}8
dlを加えて数分間煮立てる。布で漉し、 バター50
gを投入して味を調え、あらかじめ下茹でしておいた小さめのマッシュ
ルームの笠100 gを加えて仕上げる。

\hypertarget{sauce-charcutiere}{%
\subsubsection{ソース・シャルキュティエール}\label{sauce-charcutiere}}

\frsub{Sauce Charcutière}\footnote{シャルキュトリ(豚肉加工業)風、の意。Charcutrieの語源はchar(肉)
  +cuite(調理された)+rie(業)。ハムやソーセージなどと定番の組合せ
  であるマスタードを使う\protect\hyperlink{sauce-robert}{ソース・ロベール}と、おなじ
  く定番のつけ合わせであるコルニション(小さいうちに収穫してヴィネガー
  漬けにしたきゅうり。専用品種がある)を使うことに由来。}

\index{そーす@ソース!しやるききゆとりふう@シャルキュトリ風 ⇒ ---・シャルキュティエール}
\index{そーす@ソース!ふらうんはせい@ブラウン系の派生---!しやるきゆていえーる@---・シャルキュティエール}
\index{しやるきゆとりふう@シャルキュトリ風!そーす@---ソース ⇒ ソース・シャルキュティエール}
\index{sauce@sauce!petites brunes composees@Petites ---s Brunes Composées!charcutiere@--- Charcutière}
\index{sauce@sauce!charcutiere@--- Charcutière}
\index{charcutier@charcutier(ère)!sauce@Sauce Charcutière}

提供直前に、\protect\hyperlink{sauce-robert}{ソース・ロベール}1 L
に細さ2 mm程度で短かめの千切り\footnote{1〜2mm程度の細さの千切りにした野菜などをジュリエンヌjulienneと呼
  ぶ。} にしたもの100
gを加える(\protect\hyperlink{sauce-robert}{ソース・ロベール}参照)。

\hypertarget{sauce-chasseur}{%
\subsubsection{ソース・シャスール}\label{sauce-chasseur}}

\frsub{Sauce Chasseur}\footnote{狩人風、の意。古くは猟獣肉をすり潰したものを使った料理を指した
  という説もある。マッシュルームとエシャロット、白ワインを使うのが特
  徴であり、このソースを使った料理にも「シャスール」の名が付けられる。}

\index{そーす@ソース!しやすーる@---・シャスール}
\index{そーす@ソース!ふらうんはせい@ブラウン系の派生---!しやすーる@---・シャスール}
\index{しやすーる@シャスール!そーす@ソース・---}
\index{かりうとふう@狩人風 ⇒ ソース・シャスール}
\index{sauce@sauce!petites brunes composees@Petites ---s Brunes Composées!chasseur@--- Chasseur}
\index{sauce@sauce!chasseur@--- Chasseur}
\index{chasseur@chasseur!sauce@Sauce ---}

生のマッシュルームを薄切りにしたもの150gをバターで炒める。エシャロット
\footnote{échalote
  玉ねぎによく似ているが、小ぶりで水分が少なく、香味野菜
  としてよく用いられる。伝統的な品種は種子ではなく種球を植えて栽培す
  る。なお、日本でしばしば「エシャレット」の名称で流通しているものは
  ラッキョウの若どりであり、フランス料理で用いるエシャロットとはまっ
  たく異なる。}のみじん切り大さじ2\undemi{}杯を加えてさらに軽く炒め、白ワイン3
dl
を注ぎ、半量になるまで煮詰める。\protect\hyperlink{sauce-tomate}{トマトソース}3
dl と\protect\hyperlink{sauce-demi-glace}{ソース・ドゥミグラス}2
dlを加える。数分間沸騰さ せたら、バター150 gと、セルフイユ\footnote{cerfeuil
  日本ではチャービルとも呼ばれるセリ科のハーブ。}とエストラゴン\footnote{estragon
  日本ではタラゴンとも呼ばれるヨモギ科のハーブ。フランス
  料理ではとても好まれる重要なハーブのひとつ。フレンチタラゴンとロシ
  アンタラゴンの2種がある。料理に用いるのはフレンチタラゴンであり、
  この品種は種子ではなく株分けや挿し芽で殖やして栽培される。寒さには
  比較的強いが、日本の梅雨の湿度や夏の暑さには弱い。}をみじん切り
にしたもの大さじ1\undemi{}杯を加えて仕上げる。

\hypertarget{sauce-chasseur-procede-escoffier}{%
\subsubsection{ソース・シャスール(エスコフィエ流)}\label{sauce-chasseur-procede-escoffier}}

\frsub{Sauce Chasseur (Procédé Escoffier)}

\index{そーす@ソース!しやすーるえすこふいえ@---・シャスール(エスコフィエ流)}
\index{そーす@ソース!ふらうんはせい@ブラウン系の派生---!しやすーるえすこふいえ@---・シャスール(エスコフィエ流)}
\index{しやすーる@シャスール!そーすしゃすーるえすこふぃえ@ソース・---(エスコフィエ流)}
\index{かりうとふう@狩人風 ⇒ ソース・シャスール(エスコフィエ流)}
\index{sauce@sauce!petites brunes composees@Petites ---s Brunes Composées!chasseur escoffier@--- Chasseur (Procédé Escoffier)}
\index{sauce@sauce!chasseur escoffier@--- Chasseur (Procédé Escoffier)}
\index{chasseur@chasseur!sauce escoffier@Sauce --- (Procédé Escoffier)}

生のマッシュルームを薄切りにしたもの150gを、バターと植物油で軽く色付く
まで炒める。みじん切りにしたエシャロット大さじ1杯を加え、なるべくすぐ
に余分な油をきる。白ワイン2 dl とコニャック約50 ml を注ぎ、半量になるま
で煮詰める。\protect\hyperlink{sauce-demi-glace}{ソース・ドゥミグラス}4
dlと\protect\hyperlink{sauce-tomate}{トマトソー ス}2
dl、\protect\hyperlink{glace-de-viande}{グラスドヴィアンド}大さじ\undemi{}杯を加え
る。

5分間沸騰させたら、仕上げにパセリのみじん切り少々を加える。

\hypertarget{sauce-chaud-froid-brune}{%
\subsubsection{茶色いソース・ショフロワ}\label{sauce-chaud-froid-brune}}

\frsub{Sauce Chaud-froid brune}\footnote{chaudショ「熱い、温かい」とfroidフロワ「冷たい」の合成語で、火
  を通した肉や魚を冷まし、表面にこのソース・ショフロワを覆うように塗
  り付け、さらにジュレを覆いかけた料理。料理の発祥については諸説あり、
  なかでもルイ15世に仕えていた料理長ショフロワChaufroixが考案したと
  いう説を支持してなのか、英語ではこの料理をChaufroixと綴ることも多
  い。Chaud-froidの表記は19世紀後半には文献に見られる。なお、複数形
  はchauds-froidsと綴る。トリュフの薄切りやエストラゴンなどのハーブ
  その他で表面に華麗な装飾を施すことが19世紀には盛んに行なわれていた。
  現代でも装飾に凝った仕立てにするケースは多い。}

\index{そーす@ソース!ふらうんはせい@ブラウン系の派生---!しよふろわちやいろ@茶色い---・ショフロワ(エスコフィエ流)}
\index{そーす@ソース!しよふろわちやいろ@茶色い---・ショフロワ}
\index{そーす@ソース!ふらうんはせい@ブラウン系の派生---!しよふろわふらうんけい@茶色い---・ショフロワ}
\index{しよふろわ@ショフロワ!そーすふらうんけい@茶色いソース・---}
\index{sauce@sauce!petites brunes composees@Petites ---s Brunes Composées!chaud-froid brune@--- Chaud-froid brune}
\index{sauce@sauce!chaud-froid brune@--- Chaud-froid brune}
\index{chaud-froid@chaud-froid!sauce brune@Sauce --- brune}

(仕上がり1 L分)

\protect\hyperlink{sauce-demi-glace}{ソース・ドゥミグラス}\troisquarts{}
Lとトリュフエッ センス1 dl、ジュレ6〜7 dlを用意する。

ソース・ドゥミグラスにトリュフエッセンスを加えて、強火で煮詰めるが、こ
の時に鍋から離れないこと。煮詰めながらジュレを少量ずつ加えていく。最終
的に\deuxtiers{}量程度まで煮詰める。

味見をして、ソースがショフロワに使うのに丁度いい濃さになっているか確認
すること。

マデイラ酒またはポルト酒\undemi{}dlを加える。布で漉し、ショフロワの主素
材の表面に塗り付けるのに丁度いい固さになるまで、丁寧にゆっくり混ぜなが
ら冷ます。

\hypertarget{sauce-chaud-froid-brune-pour-canards}{%
\subsubsection{茶色いソース・ショフロワ(鴨用)}\label{sauce-chaud-froid-brune-pour-canards}}

\frsub{Sauce Chaud-froid brune pour Canards}

\index{そーす@ソース!ふらうんはせい@ブラウン系の派生---!しよふろわちやいろかもよう@茶色い---・ショフロワ(鴨用)}
\index{そーす@ソース!しよふろわちやいと@茶色い---・ショフロワ(鴨用)}
\index{しよふろわ@ショフロワ!ちやいろいそーすかもよう@茶色いソース・---(鴨用)}
\index{sauce@sauce!petites brunes composees@Petites ---s Brunes Composées!chaud-frois canard@--- Chaud-froid pouir Canards}
\index{sauce@sauce!chaud-froid brune pour canards@--- Chaud-froid brune pour Canards}
\index{chaud-froid@chaud-froid!sauce brune pour Canards@Sauce --- brune pour Canards}

作り方は上記、\protect\hyperlink{sauce-chaud-froid-brune}{茶色いソース・ショフロワ}と同
様だが、トリュフエッセンスではなく、鴨のガラでとったフュメ1\undemi{}
dlを用いること。また、上記のレシピよりややしっかり煮詰めること。

ソースを布で漉したら、オレンジ3個分の搾り汁、とオレンジの皮をごく薄く
剥いて細かい千切りにしたもの\footnote{zeste
  ゼスト。オレンジやレモンの皮の表面を器具を用いてすりおろ
  すか、ナイフでごく薄く表皮を向き、細かい千切りにしたもの。ここでは
  後者を使う指定になっている。}大さじ2杯を加える。オレンジの皮の千切
りはしっかりと下茹でしてよく水気をきっておくこと。

\hypertarget{sauce-chaud-froid-brune-pour-gibier}{%
\subsubsection{茶色いソース・ショフロワ(ジビエ用)}\label{sauce-chaud-froid-brune-pour-gibier}}

\frsub{Sauce Chaud-froid brune pour Gibier}

\index{そーす@ソース!ふらうんはせい@ブラウン系の派生---!しよふろわちやいろしひえよう@茶色い---・ショフロワ(ジビエ用)}
\index{そーす@ソース!しよふろわちやいろじびえよう@茶色い---・ショフロワ(ジビエ用)}
\index{しよふろわ@ショフロワ!そーすしよふろわちやいろじびえよう@茶色いソース・---(ジビエ用)}
\index{sauce@sauce!petites brunes composees@Petites ---s Brunes Composées!chaud-frois gibier@--- Chaud-froid pour Gibier}
\index{sauce@sauce!chaud-froid brune pour Gibier@--- Chaud-froid brune pour Gibier}
\index{chaud-froid@chaud-froid!sauce brune pour Gibier@Sauce --- brune
pour Gibier}

作り方は上記\protect\hyperlink{sauce-chaud-froid-brune}{標準的なソース・ショフロワ}と同
じだが、トリュフエッセンスではなく、ショフロワとして供するジビエのガラ
でとったフュメ\footnote{\protect\hyperlink{fonds-de-gibier}{ジビエのフォン}参照。}2dlを用いること。

\hypertarget{sauce-chaud-froid-tomatee}{%
\subsubsection{トマト入りソース・ショフロワ}\label{sauce-chaud-froid-tomatee}}

\frsub{Sauce Chaud-froid tomatée}

\index{そーす@ソース!ふらうんはせい@ブラウン系の派生---!とまといりしよふろわ@トマト入り---・ショフロワ}
\index{そーす@ソース!しよふろわとまといり@トマト入り---・ショフロワ}
\index{しよふろわ@ショフロワ!そーすちやいろとまといり@トマト入りソース・---}
\index{sauce@sauce!petites brunes composees@Petites ---s Brunes Composées!chaud-froid tomatee@--- Chaud-froid tomatée}
\index{sauce@sauce!chaud-froid tomatée@--- Chaud-froid tomatée}
\index{chaud-froid@chaud-froid!sauce tomatée@Sauce --- tomatée}

良質で、既によく煮詰めてあるトマトピュレ1 Lを、さらに煮詰めながら7〜8
dlのジュレを少しずつ加えていく。全体量が1L以下になるまで煮詰めること。

布で漉し、使いやすい固さになるまで、ゆっくり混ぜながら冷ます。

\hypertarget{sauce-chevreuil}{%
\subsubsection{ソース・シュヴルイユ}\label{sauce-chevreuil}}

\frsub{Sauce Chevreuil}

\index{そーす@ソース!ふらうんはせい@ブラウン系の派生---!しゆうるいゆ@---・シュヴルイユ}
\index{しゆうるいゆ@シュヴルイユ!そーす@ソース・---}
\index{そーす@ソース!しゅうるいゆ@---・シュヴルイユ}
\index{のろしか@ノロ鹿 ⇒ シュヴルイユ!そーす@ソース!しゆうるいゆ@ソース・シュヴルイユ}
\index{sauce@sauce!petites brunes composees@Petites ---s Brunes Composées!chevreuil@--- Chevreuil}
\index{sauce@sauce!chevreuil@--- Chevreuil}
\index{chevreuil@chevreuil!sauce@Sauce ---}

\protect\hyperlink{sauce-poivrade}{標準的なソース・ポワヴラード}と同様に作るが、

\begin{enumerate}
\def\labelenumi{\arabic{enumi}.}
\item
  マリネした牛・羊肉の料理に添える場合\footnote{chevreuil
    シュヴルイユはノロ鹿のことだが、このように事前にマリ
    ネした牛・羊肉を用いた料理にもこのソースを使い「シュヴルイユ(風)(仕立て)」
    と\ruby{謳}{うた}う。1806年刊ヴィアール『帝国料理の本』においてノ
    ロ鹿のフィレは香辛料を加えたワインヴィネガーで48時間マリネしてから
    調理すると書かれている。オド『女性料理人のための本』では、確認出来
    た1834年の第4版から1900年の第78版に至るまで、ノロ鹿の項において
    「一週間もヴィネガーたっぷりの漬け汁でマリネするのはやりすぎだが、
    強い味が好みなら1〜4日間」香辛料と赤ワインあるいはヴィネガーでマリ
    ネするといい、と説明されている。つまり、ノロ鹿とは必ずマリネしてか
    ら調理するものという一種のコンセンサスがあったために、マリネした牛・
    羊肉の料理にも「シュヴルイユ(風)」の名称が謳われるようになったと考
    えられる。}は、ハム入りの\protect\hyperlink{mirepoix}{ミルポ
  ワ}を加える。
\item
  ジビエ料理に添える場合は、そのジビエの端肉を加える。
\end{enumerate}

素材をヘラなどで強く押し付けるようにして漉す\footnote{シノワ(\protect\hyperlink{sauce-espagnole}{ソース・エスパニョル}訳注参照)などを用いる。}。良質の赤ワイン
1\undemi{}dlをスプーン1杯ずつ加えながら煮て、浮き上がってくる不純物を
丁寧に取り除いていく\footnote{dépouiller デプイエ ≒ écumer エキュメ。}。

最後に、カイエンヌごく少量と砂糖1つまみを加えて味を\ruby{調}{とと
の}え、布で漉す。

\hypertarget{sauce-colbert}{%
\subsubsection{ソース・コルベール}\label{sauce-colbert}}

\frsub{Sauce Colbert}\footnote{17世紀の政治家、ジャン・バティスト・コルベール(1619〜1683)の
  名を冠したもの。}

\index{そーす@ソース!ふらうんはせい@ブラウン系の派生---!こるへーる@---・コルベール}
\index{そーす@ソース!こるへーる@---・コルベール}
\index{こるへーる@コルベール!そーす@ソース・---}
\index{sauce@sauce!petites brunes composees@Petites ---s Brunes Composées!colbert@--- Colbert}
\index{sauce@sauce!colbert@--- Colbert}
\index{colbert@Colbert!sauce@Sauce ---}

\protect\hyperlink{beurre-maitre-d-hotel}{メートルドテルバター}に\protect\hyperlink{glace-de-viande}{グラスドヴィアンド}を加
えたもののことだが、正しくは「\protect\hyperlink{beurre-colbert}{ブール・コルベール}」と
呼ぶべきものだ\footnote{具体的なレシピは\protect\hyperlink{beurre-colbert}{ブール・コルベール}参照のこと。}。

また、ブール・コルベールと\protect\hyperlink{sauce-chateaubriand}{ソース・シャトーブリアン}との違いを明確にさ
せようとして、メートルドテルバターにエストラゴンを加える者もいる。だが、
必ずそうすべきということではない。実際、ブール・コルベールとソース・シャ
トーブリアンは明らかに違うものだからだ。ソース・シャトーブリアンは軽く
仕上げたグラスドヴィアントにバターとパセリのみじん切りを加えたものであ
る。一方、ブール・コルベールあるいはソース・コルベールと呼ばれているもの
はあくまでもバターが主であって、グラスドヴィアンドは補助的なものに過ぎ
ない。

\hypertarget{sauce-diable}{%
\subsubsection{ソース・ディアーブル}\label{sauce-diable}}

\frsub{Sauce Diable}\footnote{悪魔の意。}

\index{そーす@ソース!ふらうんはせい@ブラウン系の派生---!ていあーふる@---・ディアーブル}
\index{そーす@ソース!ていあーふる@---・ディアーブル}
\index{ていあーふる@ディアーブル!そーす@ソース・---}
\index{あくま@悪魔 ⇒ ディアーブル!そーす@ソース!そーすていあーふる@ソース・ディアーブル}
\index{sauce@sauce!petites brunes composees@Petites ---s Brunes Composées!diable@--- Diable}
\index{sauce@sauce!diable@--- Diable}
\index{diable@diable!sauce@Sauce ---}

このソースはごく少量ずつ作るのが一般的だが、ここではそれを守らずに、仕
上り2\undemi{} dlとして説明する

白ワイン3 dlにエシャロット3個分のみじん切りを加え、\untiers{}量以下にな
るまで煮詰める。

\protect\hyperlink{sauce-demi-glace}{ソース・ドゥミグラス}2
dlを加えて数分間煮立たせ、
仕上げにカイエンヌの粉末をたっぷり効かせる\footnote{「たっぷり」という表現に惑わされないよう注意。}。

\ldots{}\ldots{}鶏と鳩のグリルに合わせる。

\hypertarget{nota-sauce-diable}{%
\subparagraph{【原注】}\label{nota-sauce-diable}}

白ワインではなくヴィネガーを煮詰め、仕上げにハーブを加えて作る調理現場
もあるが、著者としては本書で示しているの作り方がいいと思う。

\hypertarget{sauce-diable-escoffier}{%
\subsubsection{ソース・ディアーブル・エスコフィエ}\label{sauce-diable-escoffier}}

\frsub{Sauce Diable Escoffier}

\index{そーす@ソース!ふらうんはせい@ブラウン系の派生---!ていあーふるえすこふいえ@---・ディアーブル・エスコフィエ}
\index{そーす@ソース!ていあーふるえすこふいえ@---・ディアーブル・エスコフィエ}
\index{ていあーふる@ディアーブル!そーす@ソース!えすこふいえ@ソース・---・エスコフィエ}
\index{あくま@悪魔 ⇒ ディアーブル!そーす@ソース!エスコフイエ@ソース・ディアーブル・エスコフィエ}
\index{sauce@sauce!petites brunes composees@Petites ---s Brunes Composées!diable escoffier@--- Diable Escoffier}
\index{sauce@sauce!diable escoffier@--- Diable Escoffier}
\index{diable@diable!sauce escoffier@Sauce --- Escoffier}

このソースは完成品が市販\footnote{現在は市販されていないと思われる。フランスにおいては未確認だが、
  1980年代までアメリカ合衆国ではナビスコがソース・ロベール・エスコフィ
  エとともに瓶詰めを生産、販売していた。初版ではこれら2つの製品への
  言及がなく、第二版で追加されたことから、1903年〜1907年の間に製品化
  された可能性もある。また、第二版(1907年)と同年の英訳版、第三版
  (1912年)にはソース・スリーズ・エスコフィエの記述が見られるが、こ
  れは第四版で削除されており、生産中止になったと思われる。エスコフィ
  エ・ブランドの既製品ソースはさらに他にもあったようだが詳細は不明。なお、
  エスコフィエは1922年頃、ジュリユス・マジがブイヨンキューブ(日本で
  は「マギーブイヨン」の商品名)を開発する際にも協力した。}されている。同量の柔くしたバターを混ぜ合
わせるだけでいい。

\hypertarget{sauce-diane}{%
\subsubsection{ソース・ディアーヌ}\label{sauce-diane}}

\frsub{Sauce Diane}\footnote{ローマ神話の女神ディアーナのこと。ギリシア神話のアルテミスに相
  当し、狩猟、貞潔の女神。また月の女神ルーナ(セレーネー)と同一視さ
  れた。ここでは大型ジビエ料理用のソースであるから、狩猟の女神という
  意味合いが強い。}

\index{そーす@ソース!ふらうんはせい@ブラウン系の派生---!ていあーぬ@---・ディアーヌ}
\index{そーす@ソース!ていあーぬ@---・ディアーヌ}
\index{ていあーぬ@ディアーヌ!そーす@ソース・---}
\index{sauce@sauce!petites brunes composees@Petites ---s Brunes Composées!diane@--- Diane}
\index{sauce@sauce!diane@--- Diane} \index{diane@Diane!sauce@Sauce ---}

不純物を充分に取り除き、コクと風味ゆたかな\protect\hyperlink{sauce-poivrade}{ソース・ポワヴラー
ド}5 dlを用意する。提供直前に、泡立てた生クリーム4 dl
(生クリーム2dlを泡立てて倍量にする)と、小さな三日月の形にしたトリュ
フのスライスと固茹で卵の白身を加える。

\ldots{}\ldots{}大型ジビエの骨付き背肉および、その中心部を円筒形に切り出したもの
\footnote{noisette ノワゼット。}、フィレ料理用。

\hypertarget{sauce-duxelles}{%
\subsubsection{ソース・デュクセル}\label{sauce-duxelles}}

\frsub{Sauce Duxelles}{[}\^{}

\index{そーす@ソース!ふらうんはせい@ブラウン系の派生---!てゆくせる@---・デュクセル}
\index{そーす@ソース!てゆくせる@---・デュクセル}
\index{てゆくせる@デュクセル!そーす@ソース・---}
\index{sauce@sauce!petites brunes composees@Petites ---s Brunes Composées!duxelles@--- Duxelles}
\index{sauce@sauce!duxelles@--- Duxelles}
\index{duxelles@duxelles!sauce@Sauce ---}

白ワイン2dlとマッシュルームの茹で汁2 dlにエシャロットのみじん切り大さじ2
杯を加えて、\untiers{}量まで煮詰める。\protect\hyperlink{sauce-demi-glace}{ソース・ドゥミグラ
ス}\undemi{} Lとトマトピュレ1\undemi{}
dl、\protect\hyperlink{duxelles-seche}{デュク
セル・セッシュ}大さじ4杯を加える。5分間煮立たせ、パセリのみじん切り
大さじ\undemi{}を加える。

\ldots{}\ldots{}グラタンの他、いろいろな料理に用いられる。

\hypertarget{ux539fux6ce8}{%
\subparagraph{【原注】}\label{ux539fux6ce8}}

ソース・デュクセルはイタリア風ソースと混同されることが多いが、ソース・
デュクセルにはハムも、赤く漬けた舌肉も入れないので、まったく別のものだ。

\hypertarget{sauce-estragon}{%
\subsubsection{ソース・エストラゴン}\label{sauce-estragon}}

\frsub{Sauce Estragon}\footnote{ヨモギ科のハーブ。\protect\hyperlink{sauce-chasseur}{ソース・シャスール}訳注参照。}

\index{そーす@ソース!ふらうんはせい@ブラウン系の派生---!えすとらこん@---・エストラゴン}
\index{そーす@ソース!えすとらこんちゃいろ@---・エストラゴン(ブラウン系)}
\index{えすとらこん@エストラゴン!そーすふらうんけい@ソース・---(ブラウン系)}
\index{sauce@sauce!petites brunes composees@Petites ---s Brunes Composées!estragon@--- Estragon}
\index{sauce@sauce!estragon brune@--- Estragon (brune)}
\index{estragon@estragon!sauce brune@Sauce --- (brune)}

(仕上がり2\undemi{} dl分)

白ワイン2dlを沸かし、エストラゴンの枝20 gを投入する。蓋をして10分間、煎
じる\footnote{infuser(アンフュゼ)。}。2\undemi{}
dlの\protect\hyperlink{sauce-demi-glace}{ソース・ドゥミグラス}また
は、\protect\hyperlink{jus-de-veau-lie}{とろみを付けた仔牛のジュ}を加え、約\deuxtiers{}
量になるまで煮詰める。布で漉し、みじん切りにしたエストラゴン小さじ1杯
を加えて仕上げる。

\ldots{}\ldots{}仔牛や仔羊の背肉の中心を円筒形に切り出した料理や家禽料理用。

\hypertarget{sauce-financiere}{%
\subsubsection{ソース・フィナンシエール}\label{sauce-financiere}}

\frsub{Sauce Financière}\footnote{Financier徴税官(財務官)風の意。フランス革命以前の徴税官は、王
  に代わって徴税を行なう大貴族が就く役職であり、膨大な利権によりきわめて
  裕福であったという。このソースと組み合わせる\protect\hyperlink{garniture-financiere}{ガルニチュール・フィナン
  シエール}が、雄鶏のとさかと睾丸、仔羊の胸腺肉、トリュフなどの比較的
  入手困難あるいは高級とされる食材で構成されていることが名称の由来と思わ
  れる。ブリヤ=サヴァランは『美味礼讃』(味覚の生理学)において、徴税官
  たちは旬のはしりの食材を真っ先に食べられる、いわば特権階級だと述べてい
  る。なお、カレーム『19世紀フランス料理』においては、ソースとガルニチュー
  ルを分離せず、「ラグー・アラ・フィナンシエール」として採りあげられてい
  るが、全ての素材を別々に加熱調理してソースと合わせるものであり、いわゆ
  る「煮込み」とは呼びがたいものとなっている。フランス料理の影響が比較的
  強かった北イタリアにこの原型に近いと思われるラグー「ピエモンテ風フィナ
  ンツィエラ」がある。鶏のとさか、肉垂、睾丸、鶏レバーおよび仔牛の胸腺肉
  などを煮込んだものだが、レシピを読む限りにおいては比較的庶民的あるいは
  農民的料理に変化したものと思われる (cf.~Anna Gosetti della Salda,
  \emph{Le Ricette Regionali Italiane}, Milano, Solares, 1967,
  p.57.)。ちなみに焼
  き菓子のフィナンシエfinancierも同語源だが、何故その名称になったかは不
  明。}

\index{そーす@ソース!ふらうんはせい@ブラウン系の派生---!ふいなんしえーる@---・フィナンシエール}
\index{そーす@ソース!ふいなんしえーる@---・フィナンシエール}
\index{ふいなんしえーる@フィナンシエール!そーす@ソース・---}
\index{ちょうせいかんふう@徴税官風 ⇒ フィナンシエール!そーすふぃなんしえーる@ソース・フィナンシエール}
\index{sauce@sauce!petites brunes composees@Petites ---s Brunes Composées!financiere@--- Financière}
\index{sauce@sauce!financiere@--- Financière}
\index{financier@financier(ère)!sauce@Sauce Financière}

\protect\hyperlink{sauce-madere}{ソース・マデール}1\unquart{}Lを\troisquarts{}量以下に
なるまで煮詰め、火から外してトリュフエッセンス1 dlを加える。布で漉して
仕上げる。

\ldots{}\ldots{}\protect\hyperlink{garniture-financiere}{ガルニチュール・フィナンシエール}用だが、その他の肉料理にも用い
られる。

\hypertarget{sauce-aux-fines-herbes}{%
\subsubsection{香草ソース}\label{sauce-aux-fines-herbes}}

\frsub{Sauce aux Fines Herbes}\footnote{料理名では、いわゆる「ハーブ」についてかつてfines
  herbesの表
  現が多く用いられた。だが、こんにちでは特定のハーブ名をソースや料理名
  に添えて言うことが多い。例えばCôtelette de veau au thymコトレット
  ドヴォオタン(仔牛の骨付き背肉、タイム風味)、やFilet de bar poêlé,
  compote de tomate au basilicフィレドバールポワレ コンポットートド
  トマトバジリック(スズキのフィレとトマトのコンポート、バジル風味)
  など。また、栽培レベルで「香草、ハーブ」の総称としては herbes
  aromatiques
  (エルブザロマティック)、あるいはたんにaromatiques(アロマ
  ティック)が一般的。}

\index{そーす@ソース!ふらうんはせい@ブラウン系の派生---!こうそう@香草---}
\index{そーす@ソース!こうそうふらうんけい@香草---(ブラウン系)}
\index{こうそう@香草!そーすふらうんけい@---ソース(ブラウン系)}
\index{はーぶ@ハーブ ⇒ 香草!こうそうそーすふらうんけい@香草ソース(ブラウン系)}
\index{sauce@sauce!petites brunes composees@Petites ---s Brunes Composées!fines herbes@--- aux Fines Herbes}
\index{sauce@sauce!fines herbes@--- aux Fines Herbes} \index{fines
herbes@fines herbes!sauce@Sauce aux ---}

白ワイン3 dlを沸かし、パセリの葉、セルフイユ、エストラゴン、シブレット
を各1つまみ強、投入する。約20分間煎じる。布で漉し、\protect\hyperlink{sauce-demi-glace}{ソース・ドゥミグラ
ス}または\protect\hyperlink{jus-de-veau-lie}{とろみを付けた仔牛の ジュ}6
dlを加える。仕上げに、煎じるのに使ったのと同
じ香草を細かく刻んだもの計、大さじ2\undemi{}杯とレモンの搾り汁少々を加
える。

\hypertarget{nota-sauce-aux-fines-herbes}{%
\subparagraph{【原注】}\label{nota-sauce-aux-fines-herbes}}

古典料理ではこの「香草ソース」と\protect\hyperlink{sauce-duxelles}{ソース・デュクセル}
が混同されることもあったが、こんにちではまったく違うものとして扱われて
いる。

\hypertarget{sauce-genevoise}{%
\subsubsection{ジュネーヴ風ソース}\label{sauce-genevoise}}

\frsub{Sauce Genevoise}

\index{そーす@ソース!ふらうんはせい@ブラウン系の派生---!しゆねーうふう@ジュネーヴ風---}
\index{そーす@ソース!しゆねーうふう@ジュネーヴ風---}
\index{しゆねーうふう@ジュネーヴ風!そーす@---ソース}
\index{sauce@sauce!petites brunes composees@Petites ---s Brunes Composées!genevoise@--- Genevoise}
\index{sauce@sauce!genevoise@--- Genevoise}
\index{genevois@genevois(e)!sauce@Sauce Genevoise}

鍋にバターを熱し、細かく刻んだミルポワを色付かないよう強火でさっと炒め
る。ミルポワの材料は、にんじん100 g、玉ねぎ80 g、タイムとローリエ少々、
パセリの枝20 g。そこにサーモンの頭1kgと粗く砕いたこしょう1つまみを入れ、
蓋をして弱火で15分程蒸し煮する。

鍋に残ったバターを捨て、赤ワイン1Lを注ぐ。半量になるまで煮詰める。そこ
に\protect\hyperlink{sauce-espagnole-maigre}{魚料理用ソース・エスパニョル}\undemi{}
Lを
加える。弱火で1時間煮込む。漉し器を使い、材料を押しつけながら漉す。し
ばらく休ませてから、表面に浮いた油脂を取り除く\footnote{dégraisser
  デグレセ。レードルなどを用いて浮いてきた余計な油脂を取り除く作業。}

さらに赤ワイン\undemi{} Lと、魚のフュメ\undemi{} Lを加える。ソースの表
面に浮いてくる不純物を徹底的に取り除き\footnote{dépouiller デプイエ ≒
  écumer エキュメ。}、丁度いい濃さになるまで煮 詰める。

これを布で漉し、静かに混ぜながら、アンチョヴィのエッセンス大さじ1杯と
バター150 gを加えて仕上げる。

\ldots{}\ldots{}サーモン、鱒料理用。

\hypertarget{nota-sauce-genevoise}{%
\subparagraph{【原注】}\label{nota-sauce-genevoise}}

このソースはもともとカレームが「ジェノヴァ風」\footnote{Sauce à la
  génoise au vin de Bordeaux ボルドー産ワインを用いた
  ジェノヴァ風ソース(『19世紀フランス料理』第3巻、80頁)。本書のこのレシ
  ピと同様に魚料理用ソースだ。ボルドーの赤ワインにみじん切りにして下茹で
  したマッシュルーム、トリュフ、エシャロットを加えてオールスパイスとこしょ
  う少々を入れ、適度に煮詰める。ソース・エスパニョルと赤ワインを加え、湯
  煎にかけておく。提供直前にバター少量を加えて仕上げる、というもの。本書
  においてこのソースを「原型」とするのには疑問が残るところだろう。}と名付けたものだが、その
後ルキュレ、グフェ\footnote{グフェ『料理の本』(1867年)の420ページにあるジュネーヴ風ソー
  スは、薄切りにした玉ねぎ、エシャロット、粗挽きこしょう、にんにく、
  バターを鍋に入れて色付くまで炒め、そこにブルゴーニュ産赤ワインを注
  ぐ。弱火で玉ねぎに火が通るまで煮る。ソース・エスパニョルと仔牛のブ
  ロンドのジュを加えて煮詰め、布で漉す。提供直前にマデイラ酒の風味を加
  えて茹でたトリュフのみじん切りとアンチョビバターを加える、というも
  の。赤ワインと玉ねぎ、仕上げにアンチョビを加える点は共通しているが、
  グフェのが肉料理用であるのに対して、本書のこのソースは明らかに魚料
  理用であり、まったく同じソースと呼べるとは言い難い。}が立て続けに「ジュネーヴ風」の名称を用いた。だが、ジュ
ネーヴは赤ワインの産地ではないから理屈としてはおかしい\footnote{料理名に冠された地名は、由来が明確にあるものがある一方で、まっ
  たく意味不明か、あるいはいい加減な思い付きで付けられたのではないか
  とさえ思われるものも少なくない。(à la) russe「ロシア風」や (à la)
  moscovite「モスクワ風」などはロシア料理起源か、あるいは18世紀末〜
  19世紀前半にかけてロシア帝国の宮廷や貴族がこぞってフランスから料理
  人を招聘し、帰国した彼らが創案した料理などはある程度しっかりとした
  由来がわかるものも多い。一方で、(à l')espagnole「スペイン風」(à
  l')italienne「イタリア風」(à la) romaine「ローマ風」(à la grecque)
  「ギリシア風」(à l')allemande「ドイツ風」(à l')hollandaise「オラン
  ダ風」などは由来の不明なケースが非常に多い。\protect\hyperlink{sauce-espagnole}{ソース・エスパニョ
  ル}などはその典型例とも言うべきものだろう。\\
  この原注では由来に非常にこだわっているが、そもそもカレームのレシピ
  は上述のように「ボルドー産ワインを用いたジェノバ風ソース」であるか
  ら、赤ワインの産地かどうかということは実はさしたる問題にはならない。
  重要なのは後半の、赤ワインを用いることがこのソースのポイントという
  こと。}。

間違っているとはいえ、ジュネーヴ風という名称で定着してしまっているので、
本書でもそのままにしている。だが、ジュネーヴ風であれジェノヴァ風であれ、
カレーム、ルキュレ、デュボワ、グフェはいずれもこのソースに赤ワインを用
いるよう指示している。つまり赤ワインを用いることがこのソースのポイント。

\hypertarget{sauce-godard}{%
\subsubsection{ソース・ゴダール}\label{sauce-godard}}

\frsub{Sauce Godard}\footnote{ガルニチュール・ゴダールの構成要素がガルニチュール・フィナンシ
  エールとよく似ている点などから、おそらくは18世紀の徴税官(つまりフィ
  ナンシエ)であり作家としても活動したクロード・ゴダール・ドクール
  Claude Godard d'Aucour(1716〜1795)の名を冠したものと考えられる。
  なお、底本とした現行版(第四版)では最後がdではなくtとなっているが、
  初版から第三版にいたるまでdとなっており、現行版は明らかな誤植。}

\index{そーす@ソース!ふらうんはせい@ブラウン系の派生---!こたーる@---・ゴダール}
\index{そーす@ソース!こたーる@---・ゴダール}
\index{こたーる@ゴダール!そーす@ソース・---}
\index{sauce@sauce!petites brunes composees@Petites ---s Brunes Composées!godard@--- Godard}
\index{sauce@sauce!godard@--- Godard}
\index{godard@Godard!sauce@Sauce ---}

シャンパーニュまたは辛口の白ワイン4
dlにハム入りの細かく刻んだ\protect\hyperlink{mirepoix}{ミルポ
ワ}、\protect\hyperlink{sauce-demi-glace}{ソース・ドゥミグラス}1
Lとマッシュルームのエッセンス2dlを加える。
弱火に10分かけ、シノワ\footnote{\protect\hyperlink{sauce-espagnole}{ソース・エスパニョル}訳注参照。}で漉す。

\deuxtiers{}量になるまで煮詰め、布で漉す。

\ldots{}\ldots{}\protect\hyperlink{garniture-godard}{ガルニチュール ゴタール}用。

\hypertarget{sauce-grand-veneur}{%
\subsubsection{ソース・グランヴヌール}\label{sauce-grand-veneur}}

\frsub{Sauce Grand-Veneur}\footnote{王家や貴族に仕える狩猟長のことをグランヴヌールと呼ぶ。}

\index{そーす@ソース!ふらうんはせい@ブラウン系の派生---!くらんうぬーる@---・グランヴヌール}
\index{そーす@ソース!くらんうぬーる@---・グランヴヌール}
\index{くらんうぬーる@グランヴヌール!そーす@ソース・---}
\index{sauce@sauce!petites brunes composees@Petites ---s Brunes Composées!grand-veneur@--- Grand-Veneur}
\index{sauce@sauce!grand-veneur@--- Grand-Veneur}
\index{grand-veneur@grand-veneur!sauce@Sauce ---}

\protect\hyperlink{fonds-de-gibier}{大型ジビエのフュメ}で澄んだ色合いに作った\protect\hyperlink{sauce-poivrade}{ソース・
ポワヴラード}に、ソース1Lあたり野うさぎの血1dlをマリ
ネ液1dlで薄めたものを加える。

火をごく弱くして、血が沸騰しないよう気をつけながら数分間煮る。布で漉す。

\hypertarget{sauce-grand-veneur-procede-escoffier}{%
\subsubsection{ソース・グランヴヌール(エスコフィエ流)}\label{sauce-grand-veneur-procede-escoffier}}

\frsub{Sauce Grand-Veneur (Procédé Escoffier)}

\index{そーす@ソース!ふらうんはせい@ブラウン系の派生---!くらんうぬーるえすこふいえ@---・グランヴヌール(エスコフィエ流)}
\index{そーす@ソース!くらんうぬーるえすこふいえ@---・グランヴヌール(エスコフィエ流)}
\index{くらんうぬーる@グランヴヌール!そーすえすこふいえ@ソース・---(エスコフィエ流)}
\index{sauce@sauce!petites brunes composees@Petites ---s Brunes Composées!grand-veneur escoffier@--- Grand-Veneur (Procédé Escoffier)}
\index{sauce@sauce!grand-veneur escoffier@--- Grand-Veneur (Procédé Escoffier)}
\index{grand-veneur@grand-veneur!sauce escoffier@Sauce --- (Procédé Escoffier)}

軽く仕上げた\protect\hyperlink{sauce-poivrade}{ソース・ポワヴラード}1
Lあたり\protect\hyperlink{gelee-de-groseilles-a}{グロゼイ
ユのジュレ}大さじ2杯と生クリーム2\undemi{} dlを加える。

\ldots{}\ldots{}上記2つのソースは鹿、猪などの大きな塊肉の料理に用いる。

\hypertarget{sauce-gratin}{%
\subsubsection{ソース・グラタン}\label{sauce-gratin}}

\frsub{Sauce Gratin}\footnote{魚のグラタン用ソースだが、グラタンの技術的ポイントについては\protect\hyperlink{gratins}{第
  7章「肉料理」のグタランの項目}参照。}

\index{そーす@ソース!ふらうんはせい@ブラウン系の派生---!くらたん@---・グラタン}
\index{そーす@ソース!くらたん@---・グラタン}
\index{くらたん@グラタン!そーす@ソース・---}
\index{sauce@sauce!petites brunes composees@Petites ---s Brunes Composées!gratin@--- Gratin}
\index{sauce@sauce!gratin@--- Gratin}
\index{gratin@gratin!sauce@Sauce ---}

白ワインと、このソースを合わせる魚のアラなどでとった\protect\hyperlink{fumet-de-poisson}{魚のフュ
メ}各3 dlにエシャロットのみじん切り大さじ1\undemi{}
杯を加え、半量以下になるまで煮詰める。

\protect\hyperlink{duxelles-seche}{デュクセル・セッシュ}大さじ3杯と、\protect\hyperlink{sauce-espagnole-maigre}{魚料理用ソース・エスパニョ
ル}または\protect\hyperlink{sauce-demi-glace}{ソース・ドゥミグラ ス}5
dlを加える。5〜6分間煮立たせる。提供直前に、パ
セリのみじん切り大さじ\undemi{}を加えて仕上げる。

\ldots{}\ldots{}舌びらめ、メルラン\footnote{タラの近縁種。}、バルビュ\footnote{鰈の近縁種。この場合のフィレはいわゆる「五枚おろし」にしたもの。}のフィレなどのグラタン用。

\hypertarget{sauce-hachee}{%
\subsubsection{ソース・アシェ}\label{sauce-hachee}}

\frsub{Sauce Hachée}\footnote{細かく刻んだもの、の意。}

\index{そーす@ソース!ふらうんはせい@ブラウン系の派生---!あしえ@---・アシェ}
\index{そーす@ソース!あしえ@---・アシェ}
\index{sauce@sauce!petites brunes composees@Petites ---s Brunes Composées!hachee@--- Hachée}
\index{sauce@sauce!hachee@--- Hachée}
\index{hache@haché(e)!sauce@Sauce Hachée}

玉ねぎの細かいみじん切り100gと、エシャロットの細かいみじん切り大さじ
1\undemi{}杯をバターで色付かないよう炒める。ヴィネガー3 dlを注ぎ、半量
まで煮詰める。\protect\hyperlink{sauce-espagnole}{ソース・エスパニョル}4
dlと\protect\hyperlink{sauce-tomate}{トマトソース}1\undemi{} dl
を加える。5〜6分煮立たせる。

ハムの脂身のない部分を細かく刻んだもの大さじ1\undemi{}杯と小ぶりのケイ
パー大さじ1\undemi{}杯、{[}デュクセル・セッシュ{]}大さじ1\undemi{}杯、パセ
リのみじん切り大さじ\undemi{}杯を加えて仕上げる

\ldots{}\ldots{}このソースは\protect\hyperlink{sauce-piquante}{ソース・ピカント}と等価のものと考えていい。用途も同じ。

\hypertarget{sauce-hachee-maigre}{%
\subsubsection{魚料理用ソース・アシェ}\label{sauce-hachee-maigre}}

\frsub{Sauce Hachée maigre}

\index{そーす@ソース!ふらうんはせい@ブラウン系の派生---!あしえさかな@魚料理用---・アシェ}
\index{そーす@ソース!あしえ@魚料理用---・アシェ}
\index{sauce@sauce!petites brunes composees@Petites ---s Brunes Composées!hachee maigre@--- Hachée maigre}
\index{sauce@sauce!hachee maigre@--- Hachée maigre}
\index{hache@haché(e)!sauce maigre@Sauce Hachée maigre}

上記と同様に、玉ねぎとエシャロットを色付かないようバターで炒め、ヴィネ
ガーを注いで煮詰める。

魚の\protect\hyperlink{courts-bouillons-de-poisson}{クールブイヨン}5
dlを注ぎ、\protect\hyperlink{roux-brun}{茶色いルー}45 gまたはブー
ルマニエ50 gでとろみを付ける。弱火で8〜10分間煮込む。

提供直前に、細かく刻んだハーブミックス大さじ1杯と\protect\hyperlink{duxelles-seche}{デュクセル・セッシュ}大
さじ1\undemi{}杯、小粒のケイパー大さじ1\undemi{}杯、アンチョヴィソース
大さじ\undemi{}杯とバター60 g、または80〜100 gのアンチョヴィバターを加
えて仕上げる。

\ldots{}\ldots{}エイのような、あまり高級ではない茹でた魚\footnote{原文
  poissons bouillis。このフランス語の表現だと加熱する際に沸
  騰させているニュアンスがあるが、本書の「魚料理」の章において、魚を
  塩を加えて茹でる、あるいはクールブイヨンで煮る際に、沸騰しない程度
  の温度で加熱(ポシェ pocher)すべきと強調されている。この表現は初
  版からのものであり、恐らくはこのソースの部分を実際に執筆した者と、
  魚料理の説明部分を執筆した者が異なることによるわかりにくさ、という
  可能性も排除出来ない。いずれにしても、このソースの場合は、合わせる
  魚をクールブイヨンで沸騰しない程度の温度で加熱(ポシェ)し、そのクー
  ルブイヨンの一部をソースに加えていることから、単に「茹でた魚」と言っ
  ても、本書における魚の加熱方法に則った調理をすべきと解されよう。}用。

\hypertarget{sauce-hussarde}{%
\subsubsection{ソース・ユサルド}\label{sauce-hussarde}}

\frsub{Sauce Hussarde}\footnote{もとはハンガリーで農家20戸につき1人の割合で招集された騎兵
  hussard を指す。この語は16世紀まで遡ることが出来るが、のちに「乱暴
  者」といったニュアンスでも使われるようになった。à la hussarde は
  「乱暴に、粗野に」の意味でも用いられるが、料理においてはレフォール
  を使ったものに名付けられることが多い。}

\index{そーす@ソース!ふらうんはせい@ブラウン系の派生---!ゆさると@---・ユサルド}
\index{そーす@ソース!ゆさると@---・ユサルド}
\index{ゆさると@ユサルド!そーす@ソース・---}
\index{sauce@sauce!petites brunes composees@Petites ---s Brunes Composées!hussarde@--- Hussarde}
\index{sauce@sauce!hussarde@--- Hussarde}
\index{hussard@Hussard(e)!sauce@Sauce Hussarde}

玉ねぎ2個とエシャロット2個を細かくみじん切りにして、バターで色よく炒め
る。白ワイン4
dlを注ぎ、半量になるまで煮詰める。\protect\hyperlink{sauce-demi-glace}{ソース・ドゥミグラ
ス}4 dlとトマトピュレ大さじ2杯、\protect\hyperlink{fonds-blanc}{白いフォ
ン}2 dl、生ハムの脂身のないところ80 g、潰した
にんにく1片、ブーケガルニを加える。弱火で25〜30分煮込む。

ハムを取り出して、ソースをスプーンで押すようにして布で漉す。

火にかけて温め、小さなさいの目\footnote{brunoise ブリュノワーズ。}に刻んだハムと、おろしたレフォール
\footnote{raifort (レフォール)いわゆる西洋わさび、ホースラディッシュ。}少々、パセリのみじん切りをたっぷり1つまみ加えて仕上げる。

\ldots{}\ldots{}牛、羊肉のグリルまたは串を刺してローストしてアントレ\footnote{通常、ローストは料理区分としてアントレに含められることはないが、
  牛フィレは牛の部位のなかでも比較的小さいものとして、まるごと1本の
  ローストであっても原則的にはアントレに分類される。このソースを用い
  る「牛フィレ ユサルド」は牛フィレの塊に串を刺してローストし、ポム・
  デュシェスとマッシュルームを合わせる。}として供 する際に用いる。

\hypertarget{sauce-italienne}{%
\subsubsection{イタリア風ソース}\label{sauce-italienne}}

\frsub{Sauce Italienne}\footnote{この「イタリア風」には根拠も由来も見出すことが出来ない。地名、
  国名を料理名に冠した代表例のひとつ。}

\index{そーす@ソース!ふらうんはせい@ブラウン系の派生---!イタリア風@---}
\index{そーす@ソース!いたりあふう@イタリア風---}
\index{いたりあふう@イタリア風!そーす@---ソース}
\index{sauce@sauce!petites brunes composees@Petites ---s Brunes Composées!italienne@--- Italienne}
\index{sauce@sauce!Italienne@--- Italienne}
\index{italien@italien(ne)!sauce@Sauce Italienne}

トマトの風味の効いた\protect\hyperlink{sauce-demi-glace}{ソース・ドゥミグラ
ス}\troisquarts{}
Lに、\protect\hyperlink{duxelles-seche}{デュクセル・セッシュ}大さ
じ4杯と、加熱ハムの脂身のないところを小さなさいの目に切ったもの125 gを
加える。5〜6分間煮る。提供直前に、パセリとセルフイユ、エスゴラゴンのみ
じん切り大さじ1杯を加えて仕上げる。

\ldots{}\ldots{}いろいろな肉料理に合わせる。

\hypertarget{nota-sauce-italienne}{%
\subparagraph{【原注】}\label{nota-sauce-italienne}}

このソースを魚料理に合わせる場合、ハムは使わずに\protect\hyperlink{fumet-de-poisson}{魚のフュ
メ}を煮詰めて加える。

\hypertarget{jus-lie-a-lestragon}{%
\subsubsection{とろみを付けたジュ エストラゴン風味}\label{jus-lie-a-lestragon}}

\frsub{Jus lié à l'Estragon}

\index{そーす@ソース!ふらうんはせい@ブラウン系の派生---!とろみをつけたしゆえすとらこん@とろみを付けたジュ エストラゴン風味}
\index{そーす@ソース!とろみをつけたしゆえすとらこん@とろみを付けたジュエストラゴン風味}
\index{しゆ@ジュ!とろみをつけたえすとらごん@とろみを付けた--- エストラゴン風味}
\index{sauce@sauce!petites brunes composees@Petites ---s Brunes Composées!jus lie estragon@Jus lié à l'estragon}
\index{sauce@sauce!jus lie estragon@Jus lié à l'Estragon}
\index{estragon@estragon!jus lie estragon@Jus lié à l'Estragon}
\index{jus@jus!lie estragon@--- lié à l'Estragon}

\protect\hyperlink{jus-de-veau-brun}{仔牛のフォン}または\protect\hyperlink{fonds-de-volaille}{鶏のフォ
ン}に、エストラゴン50gを加えて香りを煮出し\footnote{imfuser アンフュゼ。}た
もの。

布で漉してから、アロールート\footnote{コーンスターチで代用する。}または、でんぷん30
gでとろみを付ける。

\ldots{}\ldots{}白身肉のノワゼットや家禽のフィレなどに添える。

\hypertarget{jus-lie-tomate}{%
\subsubsection{とろみを付けたジュ トマト風味}\label{jus-lie-tomate}}

\frsub{Jus lié tomaté}

\index{そーす@ソース!ふらうんはせい@ブラウン系の派生---!とろみをつけたしゆとまとふうみ@とろみを付けたジュ トマト風味}
\index{そーす@ソース!とろみをつけたしゆとまと@とろみを付けたジュ トマト風味}
\index{しゆ@ジュ!とろみをつけたとまとふうみ@とろみを付けた--- トマト風味}
\index{sauce@sauce!petites brunes composees@Petites ---s Brunes Composées!jus lie tomatee@Jus lié tomatée}
\index{sauce@sauce!jus lie tomatee@Jus lié tomaté}
\index{tomate@tomate!jus lie tomate@Jus lié tomaté}
\index{jus@jus!lie tomate@--- lié tomaté}

\protect\hyperlink{jus-de-veau-brun}{仔牛のフォン}1
Lあたり\protect\hyperlink{essence-de-tomate}{トマトエッセンス}3
dlを加え、 \quatrecinquiemes{}量まで煮詰める。

\ldots{}\ldots{}牛、羊肉料理用。

\hypertarget{sauce-lyonnaise}{%
\subsubsection{リヨン風ソース}\label{sauce-lyonnaise}}

\frsub{Sauce Lyonnaise}

\index{そーす@ソース!ふらうんはせい@ブラウン系の派生---!りよんふう@リヨン風---}
\index{そーす@ソース!りよんふう@リヨン風---}
\index{りよんふう@リヨン風!そーす@---ソース}
\index{sauce@sauce!petites brunes composees@Petites ---s Brunes Composées!lyonnaise@--- Lyonnaise}
\index{sauce@sauce!lyonnaise@--- Lyonnaise}
\index{liyonnais@lyonnais(e)!sauce lyonnaise@Sauce Lyonnaise}

中位の大きさの玉ねぎ3個をみじん切りにし、バターでじっくり、ごく弱火で
ブロンド色になるまで炒める。白ワイン2 dlとヴィネガー2 dlを注ぐ。
\untiers{}量まで煮詰め、\protect\hyperlink{sauce-demi-glace}{ソース・ドゥミグラ
ス}\troisquarts{} Lを加える。5〜6分かけて表面に浮い
てくる不純物を丁寧に取り除き\footnote{dépouiller
  デプイエ。現代ではエキュメと呼ぶ現場が多い。}、布で漉す。

\hypertarget{ux539fux6ce8-nota-sauce-lyonnaise}{%
\subparagraph{【原注】
\{nota-sauce-lyonnaise\}}\label{ux539fux6ce8-nota-sauce-lyonnaise}}

このソースを合わせる料理によっては、ソースを布で漉さずに玉ねぎを残して
もいい。

\hypertarget{sauce-madere}{%
\subsubsection{ソース・マデール}\label{sauce-madere}}

\frsub{Sauce Madère}

\index{そーす@ソース!ふらうんはせい@ブラウン系の派生---!まてーる@---・マデール}
\index{そーす@ソース!までーる@---・マデール}
\index{までらしゆ@マデイラ酒 ⇒ マデール!そーす@ソース・マデール}
\index{sauce@sauce!petites brunes composees@Petites ---s Brunes Composées!madere@--- Madère}
\index{sauce@sauce!madere@--- Madère}
\index{madere@madère!sauce@Sauce ---}

\protect\hyperlink{sauce-demi-glace}{ソース・ドゥミグラス}を煮詰め\footnote{ソース・ドゥミグラスは既に煮詰めて仕上がった状態のものなので、9
  割程度にまでしか煮詰めないことに注意。}、火から外して、 ソース1
Lあたりマデイラ酒1 dlの割合で加え、普通の濃度にする。

\hypertarget{sauce-matelote}{%
\subsubsection{ソース・マトロット}\label{sauce-matelote}}

\frsub{Sauce Matelote}\footnote{水夫風、船員風、の意。トゥーレーヌ地方の郷土料理Matelote
  d'anguille(マトロットダンギーユ)うなぎの赤ワイン煮込み、が有名。
  とはいえ本書にも数種のレシピが収録されているように、赤ワイン煮込み
  にとどまらず、マトロットの名称を持つ料理は他にも複数存在する。}

\index{そーす@ソース!ふらうんはせい@ブラウン系の派生---!まとろつと@---・マトロット}
\index{そーす@ソース!まとろつと@---・マトロット}
\index{まとろつと@マトロット!そーす@ソース・---}
\index{sauce@sauce!petites brunes composees@Petites ---s Brunes Composées!matelote@--- Matelote}
\index{sauce@sauce!matelote@--- Matelote}
\index{matelote@matelote!sauce@Sauce ---}

魚をポシェするのに使った\protect\hyperlink{court-bouillon-c}{赤ワイン入りの魚用クールブイヨン}3
dlにマッ シュルームの切りくず25
gを加え、\untiers{}量になるまで煮詰める。

煮詰めたら\protect\hyperlink{sauce-espagnole-maigre}{魚料理用ソース・エスパニョル}8
dl を加えてひと煮立ちさせる。布で漉し、バター150 gとごく少量のカイエンヌ
の粉末を加えて仕上げる。

\hypertarget{sauce-moelle}{%
\subsubsection{ソース・モワル}\label{sauce-moelle}}

\frsub{Sauce Moelle}\footnote{moelle 骨髄のこと。}

\index{そーす@ソース!ふらうんはせい@ブラウン系の派生---!もわる@---・モワル}
\index{そーす@ソース!もわる@---・モワル}
\index{こつずい@骨髄 ⇒ モワル!そーすもわる@ソース・モワル}
\index{sauce@sauce!petites brunes composees@Petites ---s Brunes Composées!moelle@--- Moelle}
\index{sauce@sauce!moelle@--- Moelle}
\index{moelle@moelle!sauce@Sauce ---}

ソースの作り方は\protect\hyperlink{sauce-bordelaise}{ボルドー風ソース}とまったく同じだ
が、バターを加えるのは何らかの野菜料理に添える場合のみであり、その場合
のバターの量は通常どおりとするこ。

どんな場合にせよ、仕上げに、小さなさいの目に切ってポシェしておいた骨髄
をソース1 Lあたり150〜180 gおよび刻んで下茹でしたパセリの葉小さじ1杯を
加える。

\hypertarget{sauce-moscovite}{%
\subsubsection{モスクワ風ソース}\label{sauce-moscovite}}

\frsub{Sauce Moscovite}\footnote{moscovite(モスコヴィット)すなわちモスクワ風の名称を持つ料理や
  菓子は多い。 18世紀後半から19世紀前半にかけて、ロシアの宮廷や貴族
  らの間でフランスの食文化が流行し、多くのフランス人料理人が招聘され、
  彼らはロシア料理のレシピをフランスに持ち帰った。クーリビヤックなど
  が代表的な例だろう。また、19世紀後半になると、とりわけフランス料理
  においてもロシア料理からの影響が多く見られるようになる。キャビアと
  ウォトカを食前に愉しむのが流行したのもその時代からである。フランス
  とロシアの食文化は相互に影響関係にあったと言えよう。}

\index{そーす@ソース!ふらうんはせい@ブラウン系の派生---!もすくわふう@モスクワ---}
\index{そーす@ソース!もすくわふう@モスクワ風---}
\index{もすくわふう@モスクワ風!そーす@---ソース}
\index{sauce@sauce!petites brunes composees@Petites ---s Brunes Composées!moscovite@--- Moscovite}
\index{sauce@sauce!moscovite@--- Moscovite}
\index{moscovite@moscovite!sauce@Sauce ---}

\protect\hyperlink{fonds-de-gibier}{大型ジビエのフュメ}で作った\protect\hyperlink{sauce-poivrade}{ソース・ポワヴラー
ド}を\troisquarts{} L用意する。提供直前にマラガ酒1 dl
とジェニパーベリーを煎じた汁7 cl\footnote{1 cl = 10
  ml、つまりこの場合は70 ml。}、焼いた松の実かスライスして焼いたアー
モンド40 g、大きさを揃えてぬるま湯でもどしておいたコリント産干しぶどう
\footnote{小粒で黒いギリシア産干しぶどう。}40 gを加えて仕上げる。

\ldots{}\ldots{}大型ジビエ\footnote{venaison
  ヴネゾン。ジビエのうち大型のものを指す。実際は
  ノロ鹿や猪を指すことがほとんど。}の塊肉の料理用。

\hypertarget{sauce-perigueux}{%
\subsubsection{ソース・ペリグー}\label{sauce-perigueux}}

\frsub{Sauce Périgueux}\footnote{トリュフの産地として有名なペリゴール地方の町の名。}

\index{そーす@ソース!ふらうんはせい@ブラウン系の派生---!へりくー@---・ペリグー}
\index{そーす@ソース!へりくー@---・ペリグー}
\index{へりくー@ペリグー!そーす@ソース・---}
\index{sauce@sauce!petites brunes composees@Petites ---s Brunes Composées!perigueux@--- Périgueux}
\index{sauce@sauce!perigueux@--- Périgueux}
\index{perigueux@Périgueux!sauce@Sauce ---}

やや濃いめに煮詰めた\protect\hyperlink{sauce-demi-glace}{ソース・ドゥミグラ
ス}\troisquarts{} Lに、トリュフエッセンス1 \undemi{}
dlと細かく刻んだトリュフ100 gを加える。

\ldots{}\ldots{}いろいろな肉料理、\protect\hyperlink{}{タンバル}、\protect\hyperlink{}{温製パテ}に合わせる。

\hypertarget{sauce-perigourdine}{%
\subsubsection{ソース・ペリグルディーヌ}\label{sauce-perigourdine}}

\frsub{Sauce Périgourdine}\footnote{ペリゴール地方風の意。}

\index{そーす@ソース!ふらうんはせい@ブラウン系の派生---!へりくるていーぬ@---・ペリグルディーヌ}
\index{そーす@ソース!へりくるていーぬ@---・ペリグゥルディーヌ}
\index{へりこーるふう@ペリゴール風 ⇒ ペリグルダン/ペリグルディーヌ!そーす@ソース・ペリグルディーヌ}
\index{sauce@sauce!petites brunes composees@Petites ---s Brunes Composées!perigourdine@--- Périgourdine}
\index{sauce@sauce!perigourdine@--- Périgourdine}
\index{perigourdin@périgourdin(e)!sauce@Sauce Périgourdine}

ソース・ペリグーのバリエーション。トリュフを細かく刻むのではなく、オリー
ブ形か小さな真珠のような形状にナイフで成形\footnote{tourner
  トゥルネ。包丁を持っている側の手は動かさずに材料を回す
  ようにして形を整えること。}したものを加える。トリュ
フを厚めにスライスして加える場合もある。

\hypertarget{sauce-piquante}{%
\subsubsection{ソース・ピカント}\label{sauce-piquante}}

\frsub{Sauce Piquante}\footnote{piquant(e) (ピカン、ピカント)
  一般的には唐辛子などが「辛い」
  の意だが、このソースでは唐辛子の類は使われておらず、むしろ酸味の効
  いたソースと言えよう。古くからのソース名。}

\index{そーす@ソース!ふらうんはせい@ブラウン系の派生---!ひかんと@---・ピカント}
\index{そーす@ソース!ひかんと@---・ピカント}
\index{sauce@sauce!petites brunes composees@Petites ---s Brunes Composées!piquante@--- piquante}
\index{sauce@sauce!piquante@--- Piquante}
\index{piquant@piquant(e)!sauce@Sauce Piquante}

白ワイン3 dlと良質のヴィネガー3 dlにエシャロットのみじん切り大さじ2
\undemi{}杯を合わせて半量に煮詰める。

\protect\hyperlink{sauce-espagnole}{ソース・エスパニョル}6
dlを加え、浮いてくる不純物を 取り除きながら\footnote{dépouiller
  デプイエ。エキュメécumerと呼ぶ現場も多い。}10分間煮る。

火から外し、コルニション\footnote{専用品種のきゅうりを小さなうちに収穫して酢漬けにしたもの。同様
  のピクルス用きゅうりとしてガーキンスという品種系統があるがもっぱら
  アメリカのハンバーガーに挟まれるようなサイズで収穫して漬けたもので
  あり、フランス料理では用いない。}、パセリ、セルフイユ、エストラゴンを細か
く刻んだもの大さじ2杯を加えて仕上げる。

\ldots{}\ldots{}豚肉のグリル焼き、ブイイ\footnote{bouilli
  茹で肉。もとはブイヨンをとった後の茹で肉のことを指した。
  単純に「茹でた肉」としてもいいのだが、17世紀にはこの食べ方が流行し
  たという歴史もあり、野菜などと共に、あるいは他の素材なしに茹でた肉
  はたんに「ブイイ」bouilli と呼ばれる。}、ローストによく合わせるソース。牛肉
のブイイや牛や羊の\protect\hyperlink{}{エマンセ}にも合わせることが出来る。

\hypertarget{sauce-poivrade}{%
\subsubsection{ソース・ポワヴラード (標準)}\label{sauce-poivrade}}

\frsub{Sauce Poivrade ordinaire}\footnote{このソースは遅くとも16世紀まで遡ることが出来る。1505年に出版さ
  れた\href{http://gallica.bnf.fr/ark:/12148/bpt6k792720}{『フランス語版プラティ
  ナ』}がpoivradeとい
  うフランス語の初出。この本において「ジビエ用こしょうのソース、ポワ
  ヴラード」Saulce de poyvre ou poyvrade pour saulvagieとしてレシピ
  が見られる。パンをよく焼いてヴィネガーに浸してすり潰す。水でもどし
  た干しぶどうと獣の血を加えて混ぜ、玉ねぎと未熟ぶどう果汁、パンを浸
  した残りのヴィネガーを加えて漉し器か布で漉す。これを鍋に入れ、こしょ
  う、生姜、シナモンを入れて炭火の上で30分程煮込む。獣の肉を獣脂を熱
  したフライパンで焼き、皿に盛る。上からポワヴラードをかけて供する、
  という内容(f.LXII)。またこの本には、魚料理用のポワヴラードも掲載
  されている。ただし、これが現代まで続くソース・ポワヴラードの原型と
  捉えるのは早計に過ぎる。ここで注目すべきは、最終的に肉あるいは魚の
  ような主素材とソースが一体化したものは中世〜ルネサンス期にはポター
  ジュと呼ばれていたのに対し、ここではソースを別のものと捉えている点
  である。ポワヴラードという語そのものは「こしょうを効かせたもの」と
  いう意味に過ぎず、1660年刊ピエール・ド・リュヌPierre de Lune『新フ
  ランス料理』におけるPoivrade de pigeonneaux 若鳩のポワヴラードは、
  背開きにした若鳩を平たくのばし、塩、こしょう をして弱火でグリルす
  る。薔薇の香りもしくはにんにく風味のヴィネガーを添えて供する、とい
  うもの(p.190)。ピエール・ド・リュヌのレシピにおいてソースに相当す
  るものはヴィネガーであり、むしろ味付けでこしょうを効かせているとい
  うことが料理名の根拠となっているに過ぎない。ちなみに、生食可能な小
  さなサイズのアーティチョークも古くからポワヴラードと呼ばれている。}

\index{そーす@ソース!ふらうんはせい@ブラウン系の派生---!ほわうらーと@---・ポワヴラード(標準)}
\index{そーす@ソース!ほわうらーと@---・ポワヴラード(標準)}
\index{ほわうらーと@ポワヴラード!そーす@ソース・---(標準)}
\index{sauce@sauce!petites brunes composees@Petites ---s Brunes Composées!poivrade ordinaire@--- Poivrade}
\index{sauce@sauce!poivrade ordinaire@--- Poivrade ordinaire}
\index{poivrade@poivrade!sauce ordinaire@Sauce --- ordinaire}

細かいさいの目に切ったにんじん100 gと玉ねぎ80 g、刻んだパセリの茎、タ
イム少々、ローリエの葉少々からなる\protect\hyperlink{mirepoix}{ミルポワ}を油で色付くま
で炒める。

ヴィネガー1 dlとマリナード2 dlを注ぎ、\untiers{}量になるまで煮詰める。
\protect\hyperlink{sauce-espagnole}{ソース・エスパニョル}1
Lを注ぎ、約45分間煮込む。

ソースを漉す10分前に、大粒のこしょう8個を叩きつぶして加える。ソースに
こしょうを入れてからの時間がこれ以上少しでも長いと、こしょうの風味が支
配的になり過ぎることになるので注意。

漉し器で香味素材を軽く押すようにして漉す。\protect\hyperlink{marinades-et-saumuresux5cux257D}{マリナード}\footnote{ヴィネガーやワイン、香味素材、塩などを合わせて肉を漬け込む液体。
  マリネ液と呼ぶこともある。}2 dlでソー
スをのばす。火にかけて35分間、所定の量\footnote{明記されていないが、ここでは約1
  L。}になるまで煮詰めながら、表
面に浮いてくる不純物を徹底的に取り除く\footnote{dépouiller
  デプイエ。現代ではécumerエキュメの語を使う現場が多い。}。

さらに布で漉し、バター50 gを加えて仕上げる\footnote{現代では、バターでモンテするmonter
  au beurreという表現を用いる 現場も多い。}。

\hypertarget{sauce-poivrade-pour-gibier}{%
\subsubsection{ソース・ポワヴラード(ジビエ用)}\label{sauce-poivrade-pour-gibier}}

\frsub{Sauce Poivrade pour Gibier}

\index{そーす@ソース!ふらうんはせい@ブラウン系の派生---!ほわうらーとしひえ@---・ポワヴラード(ジビエ用)}
\index{そーす@ソース!ほわうらーとしひえ@---・ポワヴラード(ジビエ用)}
\index{ほわうらーと@ポワヴラード!そーすしひえ@ソース・---(ジビエ用)}
\index{sauce@sauce!petites brunes composees@Petites ---s Brunes Composées!poivrade gibier@--- Poivrade pour Gibier}
\index{sauce@sauce!poivrade gibier@--- Poivrade pour Gibier}
\index{poivrade@poivrade!sauce  gibier@Sauce --- pour Gibier}

細かいさいの目に切ったにんじん125 gと玉ねぎ125 g、タイムの枝と鳥類では
ないジビエ\footnote{gibier à poil
  逐語訳すると「毛の生えているジビエ」すなわち」鹿、
  猪、野うさぎなどを指す。}の端肉1
kgからなる\protect\hyperlink{mirepoix}{ミルポワ}を油で色よく炒め る。

ミルポワが色付いてきたら、鍋の油を捨てる。ヴィネガー3 dlと白ワイン2 dl
を注ぎ、完全に煮詰める。

\protect\hyperlink{sauce-espagnole}{ソース・エスパニョル}1
Lと\protect\hyperlink{fonds-de-gibier}{ジビエの茶色いフォン}2 L、
\protect\hyperlink{marinades-et-saumures}{マリナード}1 Lを加える。

鍋に蓋をして弱火にかける。可能ならオーブンがいい。3時間半〜4時間加熱す
る。

ソースを漉す8分前に、大粒のこしょう12個を叩きつぶして加える。

漉し器で材料を押すようにして漉す。

これをジビエのフォン\unquart{} Lとマリナード\unquart{} Lでのばし、再び
火にかけて40分間、表面に浮いてくる不純物を丁寧に取り除きながら、1 Lに
なるまで煮詰める。

これを布で漉し、バター75gを加えて仕上げる。

\hypertarget{sauce-poivrade-pour-gibier}{%
\subparagraph{【原注】}\label{sauce-poivrade-pour-gibier}}

一般的にはジビエ料理のソースにはバターを加えないことになっているが、本
書では軽くバターを加えることを推奨する。そうすると、ソースの色の赤みは
薄まるが、繊細で滑らかな口あたりに仕上がる。

\hypertarget{sauce-au-porto}{%
\subsubsection{ソース・ポルト}\label{sauce-au-porto}}

\frsub{Sauce au Porto}

\index{そーす@ソース!ふらうんはせい@ブラウン系の派生---!ほると@---・ポルト}
\index{そーす@ソース!ほると@---・ポルト}
\index{ほるとしゆ@ポルト酒 ⇒ ポルト!そーす@ソース・---}
\index{sauce@sauce!petites brunes composees@Petites ---s Brunes Composées!porto@--- Porto}
\index{sauce@sauce!porto@--- au Porto}
\index{porto@Porto!sauce@Sauce au ---}

マデイラ酒ではなくポルト酒を用いて、\protect\hyperlink{sauce-madere}{ソース・マデール}と
同様に作る。

\hypertarget{sauce-portugaise}{%
\subsubsection{ポルトガル風ソース}\label{sauce-portugaise}}

\frsub{Sauce Portugaise}\footnote{日本でもフランス語のままソース・ポルチュゲーズと呼ばれることは
  多い。フランス料理においてポルトガル風の名称を付けた料理はトマトをベー
  スとしたものがほとんど。ただし、トマトを使うからといってポルトガル風の
  名が必ず付くというわけではない。}

\index{そーす@ソース!ふらうんはせい@ブラウン系の派生---!ほるとかるふう@ポルトガル風---}
\index{そーす@ソース!ほるとかるふう@ポルトガル風---}
\index{ほるとかるふう@ポルトガル風!そーす@---ソース}
\index{sauce@sauce!petites brunes composees@Petites ---s Brunes Composées!portugaise@--- Portugaise}
\index{sauce@sauce!porugaise@--- Portugaise}
\index{portugais@portugais(e)!sauce@Sauce Portugaise}

(仕上がり1 L分)

大きめの玉ねぎ1個を細かくみじん切りにする。鍋に油を熱し、強火で玉ねぎ
を炒める。玉ねぎがブロンド色になったら、皮を剥いて種子を取り除き、粗み
じん切りにしたトマト750 gと、つぶしたにんにく1片、塩、こしょうを加える。
トマトの酸味が強い場合は砂糖少々も加える。鍋に蓋をして、弱火で煮る。
\protect\hyperlink{essences-de-tomate}{トマトエッセンス}少々と、薄めに作ったトマトソース
を適量\footnote{仕上がりの全体量が1
  Lなので、トマトソースを加える量は、グラスドヴィアンド
  を加える前の段階で0.9 L程度になるよう調整する。}、温めて溶かした\protect\hyperlink{glace-de-viande}{グラスドヴィアンド}1
dl、 新鮮なパセリの葉のみじん切り大さじ1杯を加えて仕上げる。

\hypertarget{sauce-provencal}{%
\subsubsection{プロヴァンス風ソース}\label{sauce-provencal}}

\frsub{Sauce Provençale}

\index{そーす@ソース!ふらうんはせい@ブラウン系の派生---!ふろうあんすふう@プロヴァンス風---}
\index{そーす@ソース!ふろうあんすふう@プロヴァンス風---}
\index{ふろうあんすふう@プロヴァンス風!そーす@---ソース}
\index{sauce@sauce!petites brunes composees@Petites ---s Brunes Composées!provencale@--- Provençale}
\index{sauce@sauce!provencale@--- Provençale}
\index{provencal@provençal(e)!sauce@Sauce Provençale}

大ぶりのトマト12個の皮を剥き、つぶして種子は取り除いて、粗く刻む\footnote{concasser
  コンカセ。}。 ソテー鍋に2\undemi{}
dlの油を熱し、そこにトマトを入れる。塩、こしょう、
粉砂糖1つまみで味を調える。しっかりつぶしたにんにく(小)1片と細かく刻
んだパセリ小さじ1杯を加える。

蓋をして弱火で30分間程、煮溶かす。

\hypertarget{nota-sauce-provencale}{%
\subparagraph{【原注】}\label{nota-sauce-provencale}}

このソースについてはさまざまな解釈があるが、本書ではブルジョワ料理にお
ける本物の「プロヴァンス風ソース」のレシピ、つまりはトマトの「フォン
デュ」\footnote{加熱によって溶かしたもの、の意。このレシピはあくまでも「ソース」
  であり、料理を作る際のアパレイユ≒パーツとしてのいわゆる\protect\hyperlink{portugaise}{トマトフォ
  ンデュ}については第2章ガルニチュール、温製ガルニチュー
  ル用のアパレイユなど、の項を参照。}、を収録した。

\hypertarget{sauce-regence}{%
\subsubsection{ソース・レジャンス}\label{sauce-regence}}

\frsub{Sauce Régence}\footnote{Régence(レジョンス)はこの場合固有名詞としての「摂政時代」を指
  す。すなわちオルレアン公フィリップがルイ15世の幼少期に摂政を務めて
  いたた時代(1715〜1723年)のこと。オルレアン公は美食家として有名で、
  とりわけシャンパーニュを好んだという。この時期はフランス宮廷料理の
  絶頂期でもあった。}

\index{そーす@ソース!ふらうんはせい@ブラウン系の派生---!れしやんす@---・レジャンス}
\index{そーす@ソース!れしやんす@---・レジャンス}
\index{れしやんす@レジャンス!そーす@ソース・---}
\index{sauce@sauce!petites brunes composees@Petites ---s Brunes Composées!regence@--- Régence}
\index{sauce@sauce!regence@--- Régence}
\index{regence@Régence!sauce@Sauce ---}

ライン産ワイン3
dlに、細かく刻んであらかじめ火を通しておいた\protect\hyperlink{mirepoix}{ミルポ
ワ}1 dlと生トリュフの切りくず25gを加え、半量になるまで煮詰
める。トリュフのシーズンでない時季はトリュフエッセンスを使う。\protect\hyperlink{sauce-demi-glace}{ソース・
ドゥミグラス}8 dlを加え、数分間弱火にかけて浮いてく
る不純物を丁寧に取り除き\footnote{dépouiller デプイエ ≒ écumer
  エキュメ。}、布で漉す。

\ldots{}\ldots{}牛、羊の大きな塊肉の料理用。

\hypertarget{sauce-robert}{%
\subsubsection{ソース・ロベール}\label{sauce-robert}}

\frsub{Sauce Robert}\footnote{この名称のソースは古くからある。文献で初めて出てくるのは16世紀
  フランソワ・ラブレーの小説『ガルガンチュアとパンタグリュエル』。そ
  の「第四の書」で料理人の名が大量に列挙される章がある。そのうちの多
  くは架空の人名だが、その中のロベールという料理人がこのソースを考案
  したと書いている。ただし、具体的にどのようなソースかまでは描写され
  ておらず「うさぎのロースト、鴨、加工していない豚肉、卵のポシェ、塩
  漬けのメルラン{[}鱈の近縁種{]}、その他まことに多くの料理に欠かせない
  ソース」と書いてあるのみ(第40章)。どんな料理にも合うと書かれてし
  まうとむしろ特徴を捉え難くなってしまう。いずれにせよ、遅くとも16世
  紀には「ソース」として成立していたと考えられる。また、17世紀のシャ
  ルル・ペロー著『物語集』の「眠れる森の美女」においても、このソース
  名が登場する一節がある。このように16世紀以降多くの文学作品をはじめ
  とする文献にこのソース名は見られる。レシピとしては、1651年刊ラ・ヴァ
  レーヌ『フランス料理の本』における「豚腰肉 ソース・ロベール添え」
  がもっとも古いもののひとつだろう。概略は、豚腰肉を、ヴェルジュ{[}未
  熟ぶどう果汁、中世料理においてよく用いられた{]}とヴィネガー、セージ
  を振り掛けながらローストする。下に置いた脂受け皿に焼いた豚肉から流
  れ落ちた脂がたまるので、これを使って玉ねぎをこんがり炒める。炒めた
  玉ねぎの上に豚後ろ身を載せ、豚腰肉をローストする際にかけたのと同じ
  ソースをかける。このソースはソースロベールと呼ばれている(p.51)。ま
  た、干鱈のソース・ロベール添えの場合は、バターとヴェルジュ少々、マ
  スタードで作るが、ケイパーやシブール{[}葱{]}を加えてもいい(p.202)と
  あり、同じ名称のソースとは見做しがたい。18世紀以降のソース・ロベー
  ルは多かれ少なかれいずれもマスタードを加える点が共通しているので、
  名称が先にあり、内容が時代とともにはっきりしたものになっていたのだ
  ろう。}

\index{そーす@ソース!ふらうんはせい@ブラウン系の派生---!ろへーる@---・ロベール}
\index{そーす@ソース!ろへーる@---・ロベール}
\index{ろへーる@ロベール!そーす@ソース・---}
\index{sauce@sauce!petites brunes composees@Petites ---s Brunes Composées!Robert@--- Robert}
\index{sauce@sauce!robert@--- Robert}
\index{robert@Robert!sauce@Sauce ---}

(仕上がり5 dl分)

大きめの玉ねぎを細かくみじん切りにし、バターで色付かないよう強火でさっ
と炒める。

白ワイン2
dlを注ぎ、\untiers{}量になるまで煮詰める。\protect\hyperlink{sauce-demi-glace}{ソース・ドゥミグ
ラス}3 dlを加え、弱火で10分間煮る。

シノワ\footnote{主として金属製で円錐形に取っ手の付いた漉し器。清朝の高級役人が
  かぶっていた帽子の形状から「中国の」を意味するchinoisの名称となっ
  たと言われている。}で漉し(これは任意。漉さなくてもいい)、火から外して、粉砂
糖1つまみとマスタード大さじ1杯を加えて仕上げる。

\hypertarget{sauce-robert-escoffier}{%
\subsubsection{ソース・ロベール・エスコフィエ}\label{sauce-robert-escoffier}}

\frsub{Sauce Robert Escoffier}

\index{そーす@ソース!ふらうんはせい@ブラウン系の派生---!ろへーるえすこふいえ@---・ロベール・エスコフィエ}
\index{そーす@ソース!ろへーるえすこふいえ@---・ロベール・エスコフィエ}
\index{ろへーる@ロベール!そーすえすこふいえ@ソース・---・エスコフィエ}
\index{sauce@sauce!petites brunes composees@Petites ---s Brunes Composées!robert escoffier@--- Robert Escoffier}
\index{sauce@sauce!robert escoffier@--- Robert Escoffier}
\index{robert@Robert!sauce escoffier@Sauce --- Escoffier}

このソースは完成品が市販されている\footnote{\protect\hyperlink{sauce-diable-escoffier}{ソース・ディアーブル・エスコフィエ}訳注参照。}。

温かい料理にも冷たい料理にもよく合う。温かい料理に合わせる場合は、同量
の\protect\hyperlink{jus-de-veau-brun}{仔牛の茶色いフォン}と混ぜること。

\ldots{}\ldots{}豚、仔牛、鶏、魚のグリル焼きに特によく合う。

\hypertarget{sauce-romaine}{%
\subsubsection{ローマ風ソース}\label{sauce-romaine}}

\frsub{Sauce Romaine}\footnote{フランス料理における「ローマ風」の名称は「イタリア風」と同様に
  とくに根拠や由来が見出せないものが多い。このソースの場合は松の実を
  使うところから、20世紀前半に活躍したイタリアの作曲家レスピーギのロー
  マ三部作のうちの「ローマの松」を想起させるが、残念ながらこの曲が作
  曲されたのは1924年、つまり本書より後なので関係はない。だが、松の実
  を採るイタリアカサマツは、アッピア街道の並木などで有名なように、イ
  タリアとりわけローマ近辺において多く見られる(だからこそレスピーギ
  が曲の題材にしたわけだが)。その意味においては、松の実を使っている
  ということがこのソース名の根拠と見ることも不可能ではないだろう。し
  かしながら、それを証明する文献、史料があるかは不明。}

\index{そーす@ソース!ふらうんはせい@ブラウン系の派生---!ろーまふう@ローマ風---}
\index{そーす@ソース!ろーまふう@ローマ風---}
\index{ろーまふう@ローマ風!そーす@---ソース}
\index{sauce@sauce!petites brunes composees@Petites ---s Brunes Composées!romaine@--- Romaine}
\index{sauce@sauce!romain@--- Romaine}
\index{romain@romain(e)!Sauce Romaine}

砂糖50 gを火にかけてブロンド色にカラメリゼ\footnote{焦がさないように弱火で混ぜながら熱で砂糖を溶かしていく。}する。これをヴィネガー
1\undemi{}
dlでのばす。砂糖を完全に溶かし込めたら、\protect\hyperlink{sauce-espagnole}{ソース・エスパニョ
ル}6 dlと\protect\hyperlink{fonds-de-gibier}{ジビエのフォン}3 dlを加
える。これを\troisquarts{}量弱まで煮詰める。布で漉し、松の実20 gをロー
ストしたものと、大きさが揃るよう選別したスミヌル干しぶどう\footnote{トルコ産の白い干しぶどう。}20
gお よびコリント干しぶとう\footnote{ギリシア産の黒い小粒の干しぶどう(\protect\hyperlink{sauce-moscovite}{モスクワ風ソー
  ス}参照)。}20 gを温湯でもどしたものを加えて仕上げる。

\hypertarget{ux539fux6ce8-1}{%
\subparagraph{【原注】}\label{ux539fux6ce8-1}}

上記のとおり作る場合、このソースは大型ジビエ料理用だが、ジビエのフォン
ではなく通常の\protect\hyperlink{fonds-brun}{茶色いフォン}を使えば、マリネした牛、羊肉
の料理に合わせることも可能。

\hypertarget{sauce-rouennaise}{%
\subsubsection{ルーアン風ソース}\label{sauce-rouennaise}}

\frsub{Sauce Rouennaise}\footnote{ルーアンは野生のcolvertコルヴェール、いわゆる青首鴨を家禽化した
  ルーアン鴨の産地として有名。}

\index{そーす@ソース!ふらうんはせい@ブラウン系の派生---!るーあんふう@ルーアン風---}
\index{そーす@ソース!るーあんふう@ルーアン風---}
\index{るーあんふう@ルーアン風!そーす@---ソース}
\index{sauce@sauce!petites brunes composees@Petites ---s Brunes Composées!rouannaise@--- Rouannaise}
\index{sauce@sauce!rouannaise@--- Rouannaise}
\index{rouannais@rouannais(e)!sauce@Sauce Rouannaise}

(仕上がり5 dl分)

\protect\hyperlink{sauce-bordelaise}{ボルドー風ソース}4 dl
を用意する。ただし、良質な赤
ワインを使って作ること。(\protect\hyperlink{sauce-bordelaise}{ボルドー風ソース}参照)。

中位の大きさの鴨のレバー3個を裏漉しする。こうして出来たレバーのピュレ
をソースに加え、沸騰させない程度の温度で火を通す\footnote{pocher
  ポシェする。}。絶対に沸騰させ
ないこと。沸騰させてしまうと途端にレバーのピュレが粒状になってしまう。

布で漉し、塩こしょうを効かせる。

このソースの特質\ldots{}\ldots{}エシャロットを加えた赤ワインを煮詰めたものに鴨の生
レバーのピュレを加えたもの。

\ldots{}\ldots{}ルーアン産鴨のローストには、いわば必須といってもいいソース。

\hypertarget{sauce-salmis}{%
\subsubsection{ソース・サルミ}\label{sauce-salmis}}

\frsub{Sauce Salmis}\footnote{語源は「ごった煮」を意味する salmigondis
  とするのが定説のようだ
  が、salmigondisがその意味で用いられるようになったのは19世紀以降と
  考えられ、それ以前はragoûtラグーと同義と見なされていた。ラグーはそ
  の語源的意味が「食欲をそそるもの」であり、17世紀に、それまでポター
  ジュと呼ばれていた煮込み料理についてラグーの名称をつけることが流行
  した。また、salmigondisの古い語形のひとつsalmigondinは16世紀の小説
  家フランソワ・ラブレー『ガルガンチュアとパンタグリュエル』の「第四
  の書」において用いられているが、日本語の「ごった煮」のニュアンスと
  はかなり違う意味で、美味な料理のひとつとして挙げられている。いずれ
  にしても、salmigondin, salmigondisというラグーの別称が、ある時期か
  ら鳥類を材料にしたものに限定されるようになったことは確かで、カレー
  ムの『19世紀フランス料理』ではsalmisの語で、野鳥などのラグーを呼ん
  でいる。例えば「ベカスのサルミ」「ペルドローのサルミ」など。カレー
  ムとエスコフィエを比較すると、しばしばカレームにおいてラグーとして
  ひとまとめにされていた料理とソースの組合せが、『料理の手引き』にお
  いては、例えば\protect\hyperlink{garniture-financiere}{ガルニチュール・フィナンシエール}と\protect\hyperlink{sauce-financiere}{ソース・フィ
  ナンシエール}のように、別々の項目に分離されてい るものが多くある。}

\index{そーす@ソース!ふらうんはせい@ブラウン系の派生---!さるみ@---・サルミ}
\index{そーす@ソース!さるみ@---・サルミ}
\index{さるみ@サルミ!そーす@ソース・---}
\index{sauce@sauce!petites brunes composees@Petites ---s Brunes Composées!salmis@--- Salmis}
\index{sauce@sauce!salmis@--- Salmis}
\index{salmis@salmis!sauce@Sauce ---}

ソースというよりはむしろクリ\footnote{coulis \textless{} couler
  クレ「流れる」から派生した語だが、料理用語とし
  ては、やや水分の多いピュレと理解するといい。日本では「クーリ」と呼
  ぶことも多い。ここでは二つの解釈が可能で、ひとつは\protect\hyperlink{}{ポタージュ・
  クリ}に近いという意味。もうひとつは「昔ながらのソース」の意。後
  者の場合、エスコフィエが「古典料理」と呼ぶ17、18世紀においてソース
  のことをクリと呼んでいたのを踏まえていると考えられる。}と呼んだほうがいいこのソースの作り方
はどんな場合も一点を除いて変わることがない。それは、このソースを合わせ
るジビエ(鳥)の種類によって、つまり普通に肉料理として扱えるジビエか、
肉断ち\footnote{小斉のこと。カトリックの習慣として(厳密な教義ではない)四旬節
  (復活祭までの46日間)や毎週金曜などに行なわれる、肉食を断つ行為の
  こと。}の際の食材として扱えるもの\footnote{ある種の水鳥はイルカと同様に魚と同等のものと見做され、小斉の場
  合にも食材として認められていた。具体的にはハシヒロ鴨、オナガ鴨、サ
  ルセル鴨など。もっとも、水鳥を肉断ちの際の食材として扱うというのは
  一種の詭弁ともいえなくないわけで、このソースを作る際に\protect\hyperlink{sauce-espagnole-maigre}{魚料理用ソー
  ス・エスパニョル}をベースとした\protect\hyperlink{sauce-demi-glace}{ソース・
  ドゥミグラス}を使うとは考え難く、本文にあるよう
  にフォンの代用としてマッシュルームの茹で汁を用いるという指示を守るだ
  けで、厳密に小斉の料理として成立するレシピと言えるかは疑問の残ると
  ころだ。}かで、どんな液体を用いるかと いうことだけだ。

細かく刻んだ\protect\hyperlink{mirepoix}{ミルポワ}150
gをバターでじっくり色付くまで炒め
る。そこに、その料理で用いているジビエの手羽と腿の皮、ガラを細かく刻ん
で加える。

白ワイン3
dlを注ぎ、\untiers{}量まで煮詰める。\protect\hyperlink{sauce-demi-glace}{ソース・ドゥミグラ
ス}8 dlを加えて、約45分間弱火で煮込む。漉し器で漉す
が、その際に香味野菜とガラのエキス\footnote{原文quintessence(カンテソンス)。本来の意味は錬金術でいう「第五元
  素」。16世紀の作家フランソワ・ラブレーは存命当時、自著を筆名「カン
  テサンス抽出をなし遂げたアルコフリバス師」で出版していた時期がある。
  もっとも、このカンテサンスという語自体は中世以来、料理において「エ
  キス」「美味しさの本質」程度の意味でよく用いられた。}が得られるよう、強く押し絞って
やること。こうして出来たクリを、このソースを合わせる鳥と同種のものでとっ
たフォン4 dlで薄める。

ジビエが肉断ちの食材と見做されるもので、なおかつそれを厳格に守って作ら
なければならない場合は、このときフォンの代わりにマッシュルームの茹で汁を
用いればいい。

約45分〜1時間、弱火にかけて浮いてくる不純物を丁寧に取り除いてやる\footnote{dépouiller
  デプイエ。現代ではécumerエキュメの語を用いる現場が多 い。}。
さらにソースを\deuxtiers{}以下の量になるまで煮詰める。これにマッシュルー
ムの茹で汁とトリュフエッセンスを適量加えて丁度いい濃度になるよう調製する。

布で漉し、軽くバターを加えて仕上げる\footnote{原文は légèrement
  beurrerでありそのまま訳したが、現代の調理現場 ではmonter au beurre
  バターでモンテする、という表現がよく使われる。}。

\hypertarget{ux539fux6ce8-2}{%
\subparagraph{【原注】}\label{ux539fux6ce8-2}}

仕上げの際に、ソース1 Lあたりバター約50 gを加えるが、これは任意。

\hypertarget{sauce-tortue}{%
\subsubsection{ソース・トルチュ}\label{sauce-tortue}}

\frsub{Sauce Tortue}\footnote{tortue
  (トルチュ)は海亀のこと。古くは海亀料理用のソースだった
  が、19世紀以降は仔牛の頭肉料理に合わせるのが一般的になった。なお、
  tortu(e)という形容詞があり「曲がりくねった、(性格が)ひねくれた」
  という同音異義語があるが、このソースの由来とは無関係。}

\index{そーす@ソース!ふらうんはせい@ブラウン系の派生---!とるちゆ@---・トルチュ}
\index{そーす@ソース!とるちゅ@---・トルチュ}
\index{とるちゅ@トルチュ!そーす@ソース・---}
\index{うみかめ@海亀 ⇒ トルチュ!そーすとるちゆ@ソース・トルチュ}
\index{sauce@sauce!petites brunes composees@Petites ---s Brunes Composées!tortue@--- Tortue}
\index{sauce@sauce!tortue@--- Tortue}
\index{tortue@tortue!sauce@Sauce ---}

2\undemi{}
Lの\protect\hyperlink{jus-de-veau-brun}{仔牛のフォン}を鍋で沸かし、セージ3
g、マジョラム1 g、ローズマリー1 g、バジル2 g、タイム1 g、ローリエの葉1
g、パセリの葉1つまみ、マッシュルームの切りくず25 gを投入する。蓋をして
25分間煎じる。こうして煎じた液体を漉す2分前に大粒のこしょう4個を加える。

布で漉し、\protect\hyperlink{sauce-demi-glace}{ソース・ドゥミグラス}7
dlに\protect\hyperlink{sauce-tomate}{トマトソー ス}3
dlを合わせたものに、上記で煎じた液体を、風味が際立
つ程度に適量加える。\troisquarts{}量まで煮詰め、布で漉す。仕上げにマデ
ラ酒1 dlとトリュフエッセンス少々を加え、さらにカイエンヌで風味を引き締
める。

\hypertarget{nota-sauce-tortue}{%
\subparagraph{【原注】}\label{nota-sauce-tortue}}

このソースはある程度まとまった量で作る必要がある。カイエンヌを使う指示
があるからだ。それでも、カイエンヌはとても気をつけて量を加減する必要が
ある\footnote{フランス料理において(というよりも伝統的かつ一般的なフランス人
  にとって)、唐辛子の辛さは嫌われる傾向が非常に強い。}。

\hypertarget{sauce-venaison}{%
\subsubsection{ソース・ヴネゾン}\label{sauce-venaison}}

\frsub{Sauce Venaison}\footnote{Venaison(ヴネゾン)とはノロ鹿chevreuilや猪sanglierなどの大型ジ
  ビエのこと。なおニホンジカやエゾジカはcerf(セール)に分類され、フ
  ランス料理の食材としてはあまり高く評価されない傾向がある。}

\index{そーす@ソース!ふらうんはせい@ブラウン系の派生---!うねそん@---・ヴネゾン}
\index{そーす@ソース!うねそん@---・ヴネゾン}
\index{うねそん@ヴネゾン!そーす@ソース・---}
\index{おおかたしひえ@大型ジビエ ⇒ ヴネゾン!そーす@ソース・ヴネゾン}
\index{sauce@sauce!petites brunes composees@Petites ---s Brunes Composées!venaison@--- Venaison}
\index{sauce@sauce!venaison@--- Venaison}
\index{venaison@venaison!sauce@Sauce ---}

完全に仕上げた「\protect\hyperlink{sauce-poivrade-pour-gibier}{ジビエ用ソース・ポワヴラー
ド}」\troisquarts{}
Lに、\protect\hyperlink{gelee-de-groseilles-a}{グロゼイユのジュ
レ}大さじ3杯強を生クリーム1dlで溶いてから加える。

グロゼイユのジュレと生クリームを加えるのは、鍋を火から外して、提供直前
にすること。

\ldots{}\ldots{}大型ジビエ料理用。

\hypertarget{sauce-vin-rouge}{%
\subsubsection{赤ワインソース}\label{sauce-vin-rouge}}

\frsub{Sauce au Vin rouge}

\index{そーす@ソース!ふらうんはせい@ブラウン系の派生---!あかわいん@赤ワイン---}
\index{そーす@ソース!あかわいん@赤ワイン---}
\index{あかわいん@赤ワイン!そーす@---ソース}
\index{sauce@sauce!petites brunes composees@Petites ---s Brunes Composées!vin rouge@--- au Vin rouge}
\index{sauce@sauce!vin rouge@--- au Vin rouge}
\index{vin@vin!sauce rouge@Sauce au --- rouge}

「赤ワインソース」という場合、煮詰めてからブールマニエでとろみを付ける
ブルゴーニュ風の仕立てか、魚を煮るのに用いた赤ワインを使うことが特徴で
ある「ソース・マトロット」のいずれかから派生したものなのは言うまでもな
い。もっとも、後者の場合はワインの風味は失われてしまっていてソースの水
気と味付けの意味しか持っていないと言える。

両者どちらもまさしく「赤ワインソース」だが、\protect\hyperlink{sauce-bourguignonne}{ブルゴーニュ風ソー
ス}と\protect\hyperlink{sauce-matelote}{ソース・マトロット}はそれ
ぞれ作り方も用途も違うから別々の名称として、この「茶色い派生ソース」の
節で説明した。

筆者としては、本当の「赤ワインソース」は以下のように作るものと考えてい
る。

ごく細かく刻んだ標準的な\protect\hyperlink{mirepoix}{ミルポワ}125
gをバターで炒める。良 質の赤ワイン\undemi{}
Lを注ぐ。半量になるまで煮詰める。つぶしたにんに
く1片、\protect\hyperlink{sauce-espagnole}{ソース・エスパニョル}7\undemi{}
dlを加え、12〜
15分、火にかけて浮いてくる不純物を丁寧に取り除く\footnote{dépouiller
  デプイエ ≒ écumer エキュメ。}。

布で漉し、バター100 gとアンチョビエッセンス小さじ1杯、カイエンヌ1つま
みを加えて仕上げる。

\ldots{}\ldots{}魚料理用ソース。

\hypertarget{sauce-zingara-a}{%
\subsubsection{ソース・ザンガラ A}\label{sauce-zingara-a}}

\frsub{Sauce Zingara A}\footnote{もとの語形はzingaro
  ザンガロ、またはヂンガロ。ジプシー、ボヘミ
  アンの意。料理ではパプリカ粉末やカイエンヌを用いたものに命名される
  ことが多い。}

\index{そーす@ソース!ふらうんはせい@ブラウン系の派生---!さんからa@---・ザンガラA}
\index{そーす@ソース!さんからa@---・ザンガラ A}
\index{さんから@ザンガラ!そーすa@ソース・--- A}
\index{しふしーふう@ジプシー風!そーすa@ソース・ザンガラ A}
\index{sauce@sauce!petites brunes composees@Petites ---s Brunes Composées!zingara a@--- Zingara A}
\index{sauce@sauce!zingara a@--- Zingara A}
\index{zingara@Zingara!sauce a@Sauce --- A}

このソースは古典料理の\protect\hyperlink{garniture-zingara}{ガルニチュール・ザンガラ}とはまったく関係が
ない。むしろイギリス料理に由来し、本書でもイギリス風ソースの節において
似たようなものをいくつか採り上げている。

ヴィネガー2\undemi{} dlにエシャロットのみじん切り大さじ1杯を加えて半量
になるまで煮詰める。\protect\hyperlink{jus-de-veau-lie}{茶色いジュ}7
dlを注ぎ、バターで
揚げたパンの身160gを加える。弱火で5〜6分間煮る。パセリのみじん切り大さ
じ1杯とレモン\undemi{}個分の搾り汁を加えて仕上げる。

\hypertarget{sauce-zingara-b}{%
\subsubsection{ソース・ザンガラ B}\label{sauce-zingara-b}}

\frsub{Sauce Zingara B}

\index{そーす@ソース!ふらうんはせい@ブラウン系の派生---!さんからb@---・ザンガラB}
\index{そーす@ソース!さんからb@---・ザンガラ B}
\index{さんから@ザンガラ!そーすb@ソース・--- B}
\index{しふしーふう@ジプシー風!そーすb@ソース・ザンガラ B}
\index{sauce@sauce!petites brunes composees@Petites ---s Brunes Composées!zingara b@--- Zingara B}
\index{sauce@sauce!zingara b@--- Zingara B}
\index{zingara@Zingara!sauce b@Sauce --- B}

白ワイン3 dlとマッシュルームの茹で汁3 dlを合わせて\untiers{}量になるまで
煮詰める。

\protect\hyperlink{sauce-demi-glace}{ソース・ドゥミグラス}4
dlと\protect\hyperlink{sauce-tomate}{トマトソー ス}2\undemi{}
dl、\protect\hyperlink{fonds-blanc}{白いフォン}1 dlを注ぐ。
浮いてくる不純物を徹底的に取り除きながら5〜6分火にかける。

仕上げに、カイエンヌ1つまみで風味を引き締め、太さ1〜2 mmの千切りにした
\footnote{julienne
  (ジュリエーヌ)。日本語では「ジュリエンヌ」と言うこと
  が多いが、「ジュリヤン」のように言う調理現場もある。}ハム(脂身のないところ)と赤く漬けた舌肉70
gおよびマッシュルーム 50 g、トリュフ 30 gを加える。

\ldots{}\ldots{}仔牛料理、鶏料理用。
\end{recette}
