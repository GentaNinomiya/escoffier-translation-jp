\hypertarget{grandes-sauces-de-base}{%
\section{基本ソース}\label{grandes-sauces-de-base}}

\frsec{Grandes Sauces de Base}

\index{そーす@ソース!きほん@基本---}
\index{sauce@sauce!00grandes@*Grandes ---s de Base}
\begin{recette}
\hypertarget{sauce-espagnole}{%
\subsubsection[ソース・エスパニョル]{\texorpdfstring{ソース・エスパニョル\footnote{本節冒頭では、ルーがスペインの料理人によってもたらされ、その結果としてソース・エスパニョルが作られるようになったと読める記述があるが、これはむしろ誤りと考えるべき。エスパニョル
  espagnol(e)は「スペイン(風)の」意だが、スペイン料理起源というわけでもない。スペインを想起させるトマトを使うから、あるいは、ソースが茶褐色なのがムーア系スペイン人を想起させるから、など定説はない。カレーム『19世紀フランス料理』第3巻に収められたソース・エスパニョルの作り方は、フォンをとるところから始まり4ページにわたって詳細なものとなっている(pp.8-11)。その中で、肉を入れた鍋に少量のブイヨンを注いで煮詰めることを繰り返す。ここまでは18世紀の料理書で一般的な手法であるが、その後に大量のブイヨンを注いだ後、いきなり強火にかけるのではなく、弱火で加熱していくやり方を「スペイン式の方法」と述べている。カレームにおいては、これがソースの名称の根拠のひとつになっていると考えていいだろう。もちろん、ソース・エスパニョルという名称のソースはカレーム以前からあり、1806年刊のヴィアール『帝国料理の本』にもカレームのレシピより簡単だが、ほぼ同様のものが基本ソースとして収録されている。また、それ以前にもソース・エスパニョルに類する名称のソースはあったが、たとえば1739年刊ムノン『新料理研究』第2巻にある「スペイン風ソース」はかなり趣きが異なる(コリアンダーひと把みを加えるのが特徴的)。同じ料理名でも時代や料理書の著者によってまったく違う料理になっていることは、食文化史において珍しいことではない。また、とりわけ料理名に地名、国名が冠されているものの中には根拠や由来のはっきりしないものも多い。いずれにしても、本書のソース・エスパニョルの源流は19世紀初頭のヴィアールあたりからと考えられる。ソース・エスパニョルは19世紀を通して普及し、茶色いソースの代表的な存在となった。こんにちでもフォンドヴォーをベースとしたソースは、ルーでとろみ付けこそしないが、仔牛の骨などから出るコラーゲンによるとろみを利用したもので、仕上がりの色合いや、ごく標準的ともいえる風味付けの方法などが引継がれ続けている調理現場も少なくない。もっとも、上述のように本書では「茶色いルー」を使うところに「エスパニョル」であることの理由を見い出そうとしていると解釈される。}}{ソース・エスパニョル}}\label{sauce-espagnole}}

\frsub{Sauce espagnole}

\index{そーす@ソース!えすはによる@---・エスパニョル}
\index{そーす@ソース!きほん@基本---!えすはによる@---・エスパニョル}
\index{きほんそーす@基本ソース!えすはによる@---・エスパニョル}
\index{えすはによる@エスパニョル!そーす@ソース・---}
\index{すへいんふう@スペイン風(エスパニョル)!そーすえすはによる@ソース・エスパニョル}
\index{sauce@sauce!00grandes@*Grandes ---s de Base!espagnole@--- Espagnole}
\index{sauce@sauce!espagnole@--- Espagnole}
\index{espagnol@espagnol!sauce@Sauce ---e}

(仕上がり5 L分)

\begin{itemize}
\item
  とろみ付けのためのルー\ldots{}\ldots{}625 g。
\item
  茶色いフォン(ソースを仕上げるのに必要な全量)\ldots{}\ldots{}12 L。
\item
  \protect\hyperlink{mirepoix}{ミルポワ}\footnote{mirepoix
    (ミルポワ)。ソースやフォンにコクを与える目的で、細かいさいの目に切った香味野菜や塩漬け豚ばら肉を合わせたもの。18世紀にミルポワ公爵の料理人が考案したといわれているが真偽は不明。同様のものにmatignon(マティニョン)がある。ミルポワより大きめのさいの目に切るのが一般的とされるが、調理現場によってはあまり区別せずミルポワとのみ呼称するケースも多い。第2章ガルニチュール、\protect\hyperlink{mirepoix}{ミルポワ}訳注参照。}(香味素材)\ldots{}\ldots{}小さなさいの目に切った塩漬け豚ばら肉150
  g、2 mm程度のさいの目\footnote{brunoise (ブリュノワーズ)。1〜2 mm
    のさいの目に切ること。 couper en
    mirepoix(クゥペオンミルポワ)ミルポワに切るとも言う。}に切ったにんじん250
  gと玉ねぎ 150
  g、タイム2枝、ローリエの葉2枚。\index{みるほわ@ミルポワ}\index{mirepoix}
\item
  作業手順
\end{itemize}

\begin{enumerate}
\def\labelenumi{\arabic{enumi}.}
\item
  フォン8
  Lを鍋で沸かす。あらかじめ柔らかくしておいたルーを加え、木杓子か泡立て器で混ぜながら沸騰させる。

  弱火にして\footnote{原文から直訳すると「鍋を火の脇に置く」だが、現代の調理環境では単純に「弱火にする」と解釈していい。}微沸騰の状態を保つ。
\item
  以下のようにしてあらかじめ用意しておいたミルポワを投入する。ソテー鍋に塩漬け豚ばら肉を入れて火にかけて脂を溶かす。そこに、細かく刻んだにんじんと玉ねぎ、タイム、ローリエの葉を加える。野菜が軽く色づくまで強火で炒める。丁寧に、余分な脂を捨てる。これをソースに加える。野菜を炒めたソテー鍋に白ワイン約100
  mL\footnote{原文 un verre de vin blanc
    (アンヴェールドヴァンブロン)。直訳すると「グラス1杯の白ワイン」だが、本書において
    un verre de 〜は「約1 dL=100 mL」と覚えておくといいだろう。}を加えてデグラセ\footnote{dégrasser
    鍋に粘液状になって付着している肉汁を酒類あるいは水で溶かし出してソースなどに利用すること。}し、それを半量まで煮詰める。これも同様にソースの鍋に加える。こまめに浮いてくる夾雑物を徹底的に取り除き\footnote{dépouiller
    デプイエ。前節「ルーの火入れについて」訳注参照。}ながら弱火で約1時間煮込む。
\item
  ソースをシノワ\footnote{小さな穴が多く空けられた円錐形で、取っ手の付いた漉し器の一種。金属製のものが主流。}で、ミルポワ野菜を軽く押しながら漉し、別の片手鍋に移す。フォン2
  Lを注ぎ足す。さらに2時間、微沸騰の状態を保ち ~
  ながら煮込む。その後、陶製の鍋に移し、ゆっくり混ぜながら冷ます。
\item
  翌日、再び厚手の片手鍋に移してから、フォン2 Lとトマトピュレ1
  Lまたは同等の生のトマトつまり2 kgを加える。\\
  トマトピュレを用いる場合は、あらかじめオーブンでほとんど茶色になるまで焼いておくといい。そうするとトマトピュレの酸味を抜くことが出来る。\\
  そうすればソースを澄ませる作業が楽になるし、ソースの色合いも温かそうで美しいものになる。\\
  ソースをヘラか泡立て器で混ぜながら強火で沸騰させる。弱火にして1時間微沸騰の状態を保つ。最後に、表面に浮いている不純物を、細心の注意を払いながら徹底的に取り除く。布で漉し、完全に冷めるまで、ゆっくり混ぜ続けること。
\end{enumerate}

\hypertarget{nota-sauce-espagnole}{%
\subparagraph{【原注】}\label{nota-sauce-espagnole}}

ソース・エスパニョルで仕上げに不純物を取り除くのにかかる時間はいちがいには言えない。これは、ソースに用いるフォンの質次第で変わるからだ。

ソースにするフォンが上質なものであればある程、仕上げに不純物を取り除く作業は早く済む。そういう場合には、ソース・エスパニョルを5時間で作ることも無理ではない。

\hypertarget{sauce-espagnole-maigre}{%
\subsubsection[魚料理用ソース・エスパニョル]{\texorpdfstring{魚料理用ソース・エスパニョル\footnote{フランス語のソース名にあるmaigre(メーグル)はこの場合、一般的に「魚用、魚料理用」と訳すが、厳密には「小斉の際の料理用」となろう。小斉とは、カトリックで古くから特定の期間、曜日に肉類を断つ食事をする宗教的食習慣。日本の「お精進」とニュアンスは近いが、小斉においては忌避されるのは鳥獣肉のみであり、魚介や乳製品はいいとされた。こじつけのように、水鳥は水のものだから魚介扱いであり、またイルカも魚類として扱われていた。小斉が行なわれるのは復活祭の前46日間(四旬節、逆に言えばカーニバルの最終日マルディグラの翌日から46日)と、週に一度(多くの場合は金曜)であった。合計すると小斉が行なわれるのは年間100日近くもあり、中世から18世紀の料理人たちは小斉の宴席に供する料理に工夫を凝らしていた。この習慣は19世紀になるとだんだん廃れていき、エスコフィエの時代には、料理人に対して小斉のための料理を要求することは少なくなっていった。}}{魚料理用ソース・エスパニョル}}\label{sauce-espagnole-maigre}}

\frsub{Sauce espagnole maigre}

\index{そーす@ソース!えすはによるるさかな@---・エスパニョル (魚料理用)}
\index{そーす@ソース!きほん@基本---!えすはによるさかな@魚料理用---・エスパニョル}
\index{きほんそーす@基本ソース!えすはによるさかな@魚料理用---・エスパニョル}
\index{えすはによる@エスパニョル!そーすさかなよう@ソース・--- (魚料理用)}
\index{すへいんふう@スペイン風(エスパニョル)!そーすえすはによるさかな@ソース・エスパニョル(魚料理用)}
\index{sauce@sauce!00grandes@*Grandes ---s de Base!espagnole maigre@--- Espagnole maigre}
\index{sauce@sauce!espagnole maigre@--- Espagnole maigre}
\index{espagnol@espagnol!sauce maigre@Sauce Espagnole maigre}

(仕上がり5 L分)

\begin{itemize}
\item
  バターを用いて\footnote{初版〜第三版にかけては、茶色いルーを作るのに「バターまたは、きれいなグレスドマルミット(コンソメなどを作る際に表面に浮いてくる脂をすくい取って、不純物を漉し取ったものであり、基本的に獣脂)」を用いる、とある。上述のように、カトリックにおける「小斉」の場合、獣脂は忌避されたがバターなどの乳製品は許容された。そのため特に「バターを用いて作ったルー」という指定がなされ、第四版では茶色いルーに澄ましバターのみを使う旨が強調されたが、ここでは初版以来の記述がそのまま残っているために、やや冗長に思われる表現となっている。}作ったルー\ldots{}\ldots{}500
  g。
\item
  魚のフュメ(フュメドポワソン)(ソースを仕上げるために必要な全量)\ldots{}\ldots{}10
  L。
\item
  ミルポワ\ldots{}\ldots{}標準的なソース・エスパニョルと同じ\protect\hyperlink{mirepoix}{ミルポワ}野菜を同量と、塩漬け豚ばら肉の代わりにバターを用い、マッシュルームまたはマッシュルームの切りくず\footnote{champignons
    de Paris
    (シャンピニョン ドパリ)いわゆるマッシュルームは、料理の一部として提供する際にはトゥルネ
    tourner
    といって\{螺旋\}\{らせん\}状の切れ込みを入れて装飾したものを使う。その際に少なくない量の切りくずが発生するのでそれを利用する。なお、tourner
    (トゥルネ)の原義は「回す」であり、包丁を持った側の手は動かさずに、材料のほうを回すようにして切れ目を入れたり、アーティチョークや果物などの皮を剥くことを意味する。}250
  gを加える。
\item
  作業手順\ldots{}\ldots{}標準的なソース・エスパニョルとまったく同様に作る。
\item
  加熱時間と不純物を取り除くのに必要な時間\ldots{}\ldots{}5時間。
\end{itemize}

仕上げに漉してから、標準的なソース・エスパニョルとまったく同様に、完全に冷めるまでゆっくり混ぜ続けること。

\hypertarget{observation-sauce-espagnole-maigre}{%
\subparagraph{魚料理用ソース・エスパニョル補足}\label{observation-sauce-espagnole-maigre}}

このソースを日常的な料理のベースとなる仕込みに含めるかどうかについては意見が分れるところだ。

普通のソース・エスパニョルは、つまるところ風味の点ではほとんどニュートラルなものだから、それに魚のフュメを加えれば、魚料理用ソース・エスパニョルとして充分に通用するだろう。どうしても上で挙げた魚料理用ソース・エスパニョルが必要になるのは、宗教的に厳格に小斉の決まりを守って料理を作る場合のみで、さすがにその場合は代用品などない。

\hypertarget{sauce-demi-glace}{%
\subsubsection[ソース・ドゥミグラス]{\texorpdfstring{ソース・ドゥミグラス\footnote{日本の洋食などで一般的な「デミグラス」あるいは「ドミグラス」」とはかなり異なった仕上りのソースであることに注意。ソース・エスパニョルの仕上げにあたって、徹底的に不純物を取り除くことを何度も強調しているのは、透き通った茶色がかった色合いの、なめらかなソースを目指すからであり、それをさらに徹底させるということは、透明度、なめらかさの面でさらに上を目指すということを意味するからだ。ちなみに、アメリカに本社のあるメーカーの「デミグラスソース」の缶詰はもっぱら日本で販売されている製品であり、ヨーロッパおよびアメリカでは同一ブランドに該当する商品は存在しないようだ。}}{ソース・ドゥミグラス}}\label{sauce-demi-glace}}

\frsub{Sauce demi-glace}

\index{そーす@ソース!とうみくらす@---・ドゥミグラス}
\index{そーす@ソース!きほん@基本---!とうみくらす@---・ドゥミグラス}
\index{きほんそーす@基本ソース!とうみくらす@---・ドゥミグラス}
\index{sauce@sauce!00grandes@*Grandes ---s de Base!demi-glace@--- Demi-glace}
\index{sauce@sauce!demi-glace@--- Demi-glace}

一般に「ドゥミグラス」と呼ばれているものは、いったん仕上がったソース・エスパニョルをさらに、もうこれ以上は無理という位に徹底的に不純物を取り除いたもののことだ。

最後の仕上げに\protect\hyperlink{glace-de-viande}{グラスドヴィアンド}などを加える。風味付けに何らかの酒類\footnote{本書ではマデイラ酒(マデイラワイン、ポルトガルの酒精強化ワイン、すなわちブドウ果汁が酵母により醗酵している途中で蒸留酒を加えて醗酵を止める製法のもので、甘口のものが多い)が用いられることが多い。}
を加えれば、当然ながらソースの性格も変わるので、最終的な使い途に応じて決めること。

\hypertarget{nota-sauce-demi-glace}{%
\subparagraph{【原注】}\label{nota-sauce-demi-glace}}

ソースの色合いを決めるワインを仕上げに加える際には、「火から外して」行なうこと。沸騰しているとワインの香りがとんでしまうからだ。

\hypertarget{jus-de-veau-lie}{%
\subsubsection{とろみを付けた仔牛のジュ}\label{jus-de-veau-lie}}

\frsub{Jus de veau lié}

\index{そーす@ソース!きほん@基本---!とろみをつけたこうしのしゆ@とろみを付けた仔牛のジュ}
\index{きほんそーす@基本ソース!とろみをつけたこうしのしゆ@とろみを付けた仔牛のジュ}
\index{しゆ@ジュ!こうしのしゆ@仔牛の---(とろみを付けた)}
\index{そーす@ソース!とろみをつけたこうしのしゆ@とろみを付けた仔牛のジュ}
\index{こうし@仔牛!とろみをつけたこうしのしゆ@とろみを付けた---のジュ}
\index{sauce@sauce!00grandes@*Grandes ---s de Base!jus veau lie@--- de veau lié}
\index{jus@jus!jus veau lie@--- de veau lié}
\index{veau@veau!jus lie@jus de --- lié}

(仕上がり1 L分)

\begin{itemize}
\item
  仔牛のフォン\ldots{}\ldots{}仔牛の茶色いフォン4 L。
\item
  とろみ付け材料\ldots{}\ldots{}アロールート\footnote{allow-root
    南米産のクズウコンを原料とした良質のでんぷん。日本では入手が難しいこともあり、コーンスターチが用いられることがほとんど。}30
  g。
\item
  作業手順\ldots{}\ldots{}よく澄んだ仔牛のフォン4
  Lを強火にかけ、\(\frac{1}{4}\) 量つまり1 Lになるまで煮詰める。
\end{itemize}

大さじ数杯分の冷たいフォンでアロールートを溶く。これを沸騰している鍋に加える。1分程度だけ火にかけ続けたら、布で漉す。

\hypertarget{nota-jus-de-veau-lie}{%
\subparagraph{【原注】}\label{nota-jus-de-veau-lie}}

この、とろみを付けた仔牛のジュは、本書では頻繁に使う指示をしているが、必ず、しっかりした味で透き通った、きれいな薄茶色に仕上げること。

\hypertarget{veloute}{%
\subsubsection[ヴルテ(標準的な白いソース)]{\texorpdfstring{ヴルテ\footnote{velouté
  (ヴルテ)原義は「ビロードのように柔らかな、なめらかな」。日本ではベシャメルソースと混同されやすいが、内容がまったく異なるソースなので注意。}(標準的な白いソース)}{ヴルテ(標準的な白いソース)}}\label{veloute}}

\frsub{Velouté, ou sauce blanche graisse}

\index{そーす@ソース!きほん@基本---!うるて@ヴルテ(標準的な)}
\index{きほんそーす@基本ソース!うるて@ヴルテ(標準的な)}
\index{うるて@ヴルテ!ひようひゆんてきなそーすうるて@標準的なソース・---}
\index{そーす@ソース!うるてひようひゆん@ヴルテ(標準的な)}
\index{ふるーて@ブルーテ ⇒ ヴルテ} \index{veloute@velouté}
\index{veloute@velouté!sauce blanche grasse@sauce blanche grasse}
\index{sauce@sauce!00grandes@*Grandes ---s de Base!veloute@Velouté}

(仕上がり5 L分)

\begin{itemize}
\item
  とろみ付けの材料\ldots{}\ldots{}バターを用いて作った\footnote{\protect\hyperlink{sauce-espagnole-maigre}{魚料理用ソース・エスパニョル}、訳注参照。}ブロンドのルー
  625 g。
\item
  よく澄んだ仔牛の白いフォン\ldots{}\ldots{}5 L。
\item
  作業手順\ldots{}\ldots{}ルーをフォンに溶かし込む。フォンは冷たくても熱くてもいいが、フォンが熱い場合にはソースが充分なめらかになるよう注意して溶かすこと。混ぜながら沸騰させる。微沸騰の状態を保ちながら、浮いてくる不純物を完全に取り除いていく\footnote{dépouiller
    (デプイエ)。\protect\hyperlink{sauce-espagnole}{ソース・エスパニョル}、訳注参照。}。この作業はとりわけ細心の注意を払って行なうこと。
\item
  加熱時間と不純物を取り除く作業に必要な時間\ldots{}\ldots{}1時間半。
\end{itemize}

その後、ヴルテを布で漉す\footnote{ある程度濃度のある液体やピュレを布で漉す場合、昔は「二人がかりで行なう必要があり、それぞれが巻いた布の端を左手に持ち、右手に持った木杓子を使って圧し搾る」(『ラルース・ガストロノミーク』初版、
  1938年)という方法が一般的だった。}。陶製の鍋に移してゆっくり混ぜながら完全に冷ます。

\hypertarget{veloute-de-volaille}{%
\subsubsection{鶏のヴルテ}\label{veloute-de-volaille}}

\frsub{Velouté de volaille}

\index{きほんそーす@基本ソース!とりのうるて@鶏のヴルテ}
\index{そーす@ソース!きほん@基本---!とりのうるて@鶏のヴルテ}
\index{うるて@ヴルテ!とりのうるて@鶏の---(ヴルテドヴォライユ)}
\index{そーす@ソース!うるてとり@ヴルテ(鶏)}
\index{ふるーて@ブルーテ ⇒ ヴルテ}
\index{うおらいゆ@ヴォライユ!うるてとうおらいゆ@ヴルテドヴォライユ(鶏
のヴルテ)} \index{かきん@家禽!とりのうるて@鶏のヴルテ}
\index{veloute@velouté!volaille@--- de Volaille}
\index{sauce@sauce!veloute volaille@Velout\'e de Volaille}

このヴルテの作り方だが、上述の標準的なヴルテと、材料比率と作業はまったく同じ。使用する液体として鶏の白いフォン(フォンドヴォライユ)を使う。

\hypertarget{veloute-de-poisson}{%
\subsubsection{魚料理用ヴルテ}\label{veloute-de-poisson}}

\frsub{Velouté de poisson}

\index{そーす@ソース!きほん@基本---!さかなりようりよううるて@魚料理用ヴルテ}
\index{きほんそーす@基本ソース!さかなりようりよううるて@魚料理用ヴルテ}
\index{うるて@ヴルテ!さかなうるて@魚料理用---} \index{そーす@ソース!う
るてさかな@ヴルテ(魚料理用)} \index{veloute@velouté!poisson@--- de
Poisson} \index{sauce@sauce!veloute poisson@Velouté de Poisson}

ルーと液体の分量は標準的なヴルテとまったく同じだが、仔牛のフォンではなく魚のフュメを用いて作る。

ただし、魚を素材として用いるストックはどれもそうだが、手早く作業すること。不純物を取り除く作業も20分程度にとどめること。その後、布で漉し、陶製の鍋に移してゆっくり混ぜながら完全に冷ます。

\hypertarget{sauce-allemande}{%
\subsubsection[ソース・アルマンド(パリ風ソース)]{\texorpdfstring{ソース・アルマンド(パリ風ソース\footnote{原書では「パリ風ソース(元ソース・アルマンド)」となっているが、後述のように、こんにちでもソース・アルマンドの名称のほうが一般的であるため、ここではSauce
  Parisienneの「訳語」としてソース・アルマンドをあてることとした。})}{ソース・アルマンド(パリ風ソース)}}\label{sauce-allemande}}

\frsub{Sauce parisienne (ex-Allemande)}

\index{そーす@ソース!きほん@基本---!あるまんと@---・アルマンド}
\index{きほんそーす@基本ソース!あるまんと@---・アルマンド}
\index{そーす@ソース!ぱりふう@パリ風--- ⇒ ---・アルマンド}
\index{はりふう@パリ風!そーす@---ソース ⇒ ---・アルマンド}
\index{といつふう@ドイツ風!そーす@ソース・アルマンド(ドイツ風ソース)}
\index{あるまん@アルマン(ド)!そーす@ソース・アルマンド}
\index{sauce@sauce!00grandes@*Grandes ---s de Base!--- Allemande}
\index{sauce@sauce!parisienne@--- parisienne (ex-Allemande)}
\index{parisien@parisien!sauce@Sauce Parisienne = Sauce Allemande}
\index{allemand@allemand!sauce@Sauce allemande (--- Parisienne)}

(仕上がり1 L分)

標準的なヴルテに卵黄でとろみを付けたソース。

\begin{itemize}
\item
  標準的なヴルテ\ldots{}\ldots{}1 L。
\item
  追加素材\ldots{}\ldots{}卵黄5個、白いフォン(冷たいもの)
  \(\frac{1}{2}\)
  L、粗く砕いたこしょう1ひとつまみ、すりおろしたナツメグ少々、マッシュルームの煮汁2
  dL、レモン汁少々。
\item
  作業手順\ldots{}\ldots{}厚手のソテー鍋にマッシュルームの茹で汁と白いフォン、卵黄、粗く砕いたこしょう、ナツメグ、レモン汁を入れる。泡立て器でよく混ぜ、そこにヴルテを加える。火にかけて沸騰させ、強火で
  \$\frac{2}{3} 量になるまで、ヘラで混ぜながら煮詰める。
\end{itemize}

ヘラの表面がソースでコーティングされる状態になるまで煮詰めたら、布で漉す。

膜が張らないよう、表面にバターのかけらをいくつか載せてやり、湯煎にかけておく。

\begin{itemize}
\tightlist
\item
  仕上げ\ldots{}\ldots{}提供直前に、バター100 gを加えて仕上げる。
\end{itemize}

\hypertarget{nota-sauce-allemande}{%
\subparagraph{【原注】}\label{nota-sauce-allemande}}

ソース・アルマンド(ドイツ風)とも呼ばれるが、本書では「パリ風」の名称を採用した。そもそも「アルマンド」というの名称に正当性がないからだ。習慣としてそう呼ばれてきただけであって、明らかに理屈に合わない名称だ
\footnote{エスコフィエは普仏戦争に従軍した経歴があり、ドイツ嫌いとして知られていた。}。1883年に雑誌「料理技術」に某タヴェルネ氏が寄せた記事には、当時ある優秀な料理人がアルマンドなどという理屈に合わない名称を使うのはやめたという話が出ている。

こんにち既に「パリ風ソース」の名称を採用している料理長もいる。そう呼んだほうが好ましいわけだが、残念なことにまだ一般的にはなっていない
\footnote{エスコフィエの願いもむなしく、現代においてもソース・アルマンドの名称で定着している。この「全注解」においても以後は「ソース・アルマンド」と訳しているので注意されたい。なお、「ドイツ風」というソース名の由来について、ソースの淡い黄色がドイツ人に多い金髪を想起させるからだとカレームは述べている。}。

\hypertarget{sauce-supreme}{%
\subsubsection[ソース・シュプレーム]{\texorpdfstring{ソース・シュプレーム\footnote{suprême
  原義は「至高の」だが、料理においてはしばしば鶏や鴨の胸肉、白身魚のフィレなどを意味する。また、このソースのように、とくに意味もなくこの名を料理につけられているケースも多い。}}{ソース・シュプレーム}}\label{sauce-supreme}}

\frsub{Sauce supême}

\index{きほんそーす@基本ソース!しゆふれーむ@---・シュプレーム}
\index{そーす@ソース!きほん@基本---!しゆふれーむ@---・シュプレーム}
\index{そーす@ソース!そーすしゆふれーむ@---・シュプレーム}
\index{しゆふれーむ@シュプレーム!そーす@ソース・---}
\index{sauce@sauce!00grandes@*Grandes ---s de Base!supreme@--- Suprême}
\index{sauce@sauce!supreme@--- Suprême}
\index{supreme@suprême!sauce@Sauce ---}

\protect\hyperlink{veloute-de-volaille}{鶏のヴルテ}に生クリーム\footnote{フランスの生クリームのうち、料理でよく使われるのは、日本の生クリームにやや近い「クレーム・フレッシュ・パストゥリゼ」(低温殺菌した生クリームで乳脂肪分30〜38%)のほか、「クレーム・フレッシュ・エペス」(低温殺菌後に乳酸醗酵させたもので日本で一般的な生クリームより濃度がある)、「クレーム・ドゥーブル」(殺菌後に乳酸醗酵させたもので乳脂肪分40%程度でかなり濃度がある)などがある。}を加えてなめらかに仕上げ\footnote{monter
  モンテ。原義は「上げる、ホイップする」だが、ソースの仕上げの際などに、バターや生クリームを加えて、なめらかに仕上げることも「モンテ」の語を使用する場合が多い。}たもの。ソース・シュプレームは、正しく作った場合「白さの\ruby{際}{きわ}だったとても繊細な」仕上がりのものでなくてはいけない。

(仕上がり1 L分)

\begin{itemize}
\item
  鶏のヴルテ\ldots{}\ldots{}1 L。
\item
  追加素材\ldots{}\ldots{}鶏の白いフォン1
  L、マッシュルームの茹で汁1dL、良質な生クリーム2 \(\frac{1}{2}\) dL。
\item
  作業手順\ldots{}\ldots{}鍋に鶏のフォンとマッシュルームの茹で汁、鶏のヴルテを入れて強火にかけ、ヘラで混ぜながら、生クリームを少しずつ加え、煮詰めていく。このヴルテと生クリームを煮詰めたものの分量は、上で示した仕上がり1L
  のソース・シュプレームを作るには、 \(\frac{1}{3}\)
  量まで煮詰まっていなくてはならない。
\end{itemize}

布で漉し、仕上げに1 dLの生クリームとバター80
gを加えてゆっくり混ぜながら冷ますと、丁度最初のヴルテと同量になる。

\hypertarget{sauce-bechamel}{%
\subsubsection[ベシャメルソース]{\texorpdfstring{ベシャメルソース\footnote{17世紀にルイ14世のメートルドテルを務めたこともあるルイ・ベシャメイユLouis
  Béchameil(1630〜1703)の名が冠されているこのソースは、彼自身の創案あるいは彼に仕えていた料理人によるものという説もあったが真偽は疑わしい。17世紀頃の成立であることは確かだが、おそらくは古くからあったソースを改良したものに過ぎず、また、19世紀前半のカレームのレシピはヴルテを煮詰め、卵黄と煮詰めた生クリームでとろみを付けるというものだった。同様に1867年刊グフェ『料理の本』のレシピも、炒めた仔牛肉と野菜に小麦粉を振りかけてからブイヨン注ぎ、これを煮詰め、漉してから生クリームを加えるというものだった。}}{ベシャメルソース}}\label{sauce-bechamel}}

\frsub{Sauce Béchamel}

\index{きほんそーす@基本ソース!へしやめる@ベシャメル---}
\index{そーす@ソース!きほん@基本---!へしやめる@ベシャメル---}
\index{そーす@ソース!へしやめる@ベシャメル---}
\index{へしやめる@ベシャメル!そーす@---ソース}
\index{sauce@sauce!00grandes@*Grandes ---s de Base!bechamel@--- Béchamel}
\index{sauce@sauce!bechamel@--- Béchamel}
\index{bechamel@Béchamel (sauce)}

(仕上がり 5 L分)

\begin{itemize}
\item
  白いルー\ldots{}\ldots{}650 g。
\item
  使用する液体\ldots{}\ldots{}沸かした牛乳5 L。
\item
  追加素材\ldots{}\ldots{}白身で脂肪のない仔牛肉300
  gをさいの目に切り、みじん切りにした玉ねぎ(小)2個分とタイム1枝、粗く砕いたこしょう1つまみ、塩25
  g とバターを鍋に入れて蓋をし、色付かないように弱火で蒸し煮したもの。
\item
  作業手順\ldots{}\ldots{}沸かした牛乳でルーを溶く。混ぜながら沸騰させる。ここに、先に蒸し煮しておいた野菜と調味料、仔牛肉を加える。弱火で1時間煮込む。布で漉し\footnote{\protect\hyperlink{veloute}{ヴルテ}訳注参照。}、表面にバターのかけらをいくつか載せて膜が張らないようにする。肉類を絶対に使わない\footnote{小斉のこと。\protect\hyperlink{sauce-espagnole-maigre}{魚料理用ソース・エスパニョル}訳注参照。}で調理する必要がある場合は、仔牛肉を省き、香味野菜などは上記のとおりに作ること。
\end{itemize}

このソースは次のようなやり方をすると手早く作ることも出来る。沸かした牛乳に塩、薄切りにした玉ねぎ、タイム、粗く砕いたこしょう、ナツメグを加える。蓋をして弱火で10分煮る。これを漉してルーを入れた鍋の中に入れ、強火にかけて沸騰させる。その後15〜20分だけ煮込めばいい。

\hypertarget{sauce-tomate}{%
\subsubsection{トマトソース}\label{sauce-tomate}}

\frsub{Sauce tomate}

\index{きほんそーす@基本ソース!とまと@トマト---}
\index{そーす@ソース!きほん@基本---!とまと@トマト---}
\index{そーす@ソース!とまとそーす@トマト---}
\index{とまと@トマト!ソース@---ソース}
\index{sauce@sauce!00grandes@*Grandes ---s de Base!tomate@--- tomate}
\index{sauce@sauce!tomate@--- tomate}
\index{tomate@tomate!sauce@Sauce ---}

(仕上がり5 L分)

\begin{itemize}
\item
  主素材\ldots{}\ldots{}トマトピュレ4 L、または生のトマト6 kg。
\item
  \protect\hyperlink{mirepoix}{ミルポワ}\ldots{}\ldots{}さいの目に切って下茹でしておいた塩漬け豚ばら肉140
  g 、1〜2 mm 角のさいの目に刻んだにんじん200 gと玉ねぎ150
  g、ローリエの葉 1枚、タイム1枝、バター100 g。
\item
  追加素材\ldots{}\ldots{}小麦粉150 g、白いフォン2 L、にんにく2片。
\item
  調味料\ldots{}\ldots{}塩20 g、砂糖30 g、こしょう1つまみ。
\item
  作業手順\ldots{}\ldots{}厚手の片手鍋で、塩漬け豚ばら肉を軽く色付くまで炒める。ミルポワの野菜を加え、野菜も色よく炒める。小麦粉を振りかける。ブロンド色になるまで炒めてから、トマトピュレまたは潰した生トマトと白いフォン、砕いたにんにく、塩、砂糖、こしょうを加える。
\end{itemize}

火にかけて混ぜながら沸騰させる。鍋に蓋をして弱火のオーブンに入れ1時間半〜2時間加熱する。

目の細かい漉し器または布で漉す。再度、火にかけて数分間沸騰させる。保存用の器に移し、ソースが空気に触れて表面に膜が張らないよう、バターのかけらを載せてやる。

\hypertarget{nota-sauce-tomate}{%
\subparagraph{【原注】}\label{nota-sauce-tomate}}

トマトピュレを使い、小麦粉は使わず、その他は上記のとおりに作ってもいい。漉し器か布で漉してから、充分な濃度になるまでしっかり煮詰めてやること。
\end{recette}