\hypertarget{ux30deux30eaux30caux30fcux30c9ux3068ux30bdux30dfux30e5ux30fcux30eb1}{%
\section[マリナードとソミュール]{\texorpdfstring{マリナードとソミュール\footnote{マリナードはマリネ液とも言う。marinade
  \textless{} mariner (マリネ)語源
  はラテン語のmare(海)だが、中世フランス語ではもっぱら「海で泳ぐ、
  海に潜る」の意で使われていたが、16世紀には既に、料理用語として用い
  られていた。ラブレー『ガルガンチュアとパンタグリュエル』第四の書
  (1548年)において、lancerons marinez (マリネしたブロシェの幼魚)
  という表現が見られる。なおブロシェ brochet はノーザンパイク、和名
  キタカワカマス。川カマス属の淡水、汽水魚。この場面はパンタグリュエ
  ルに「小斉」の御馳走として捧げられた料理のリストの一部であり、「塩
  漬けのメルルーサ、卵料理各種、モリュ(塩漬けにした鱈)、アドック
  (塩漬け後に燻製にした鱈)」などとともに列挙されているものであり、
  いずれも塩辛いために、それらを調理したものを食べて消化をよくするた
  めに飲むワインの量が倍になった(p.681」とある。したがって、 lancerons
  marinezのマリネとは「海水あるいは塩水に漬けた」の意に解
  釈されよう。一方、ソミュールについては、11世紀末頃に、「保存のため
  漬け込む塩水」の意味で salmuire という語形が使用され、16世紀には
  「塩水およびその他の液体からなるもの」としてsaumureという現在とお
  なじ語形が記録されている。マリナードとソミュールが明確に分化したの
  はおそらく17世紀頃で、1651年刊ラ・ヴァレーヌ『フランス料理の本』に
  見られるマリナードの語には曖昧さを免れないものもあるが、例えば
  \emph{Poulets marinez}
  (鶏のマリネ)というレシピは「鶏を開いて叩き、しっ
  かり味付けしたヴィネガーに付ける。提供直前に小麦粉をまぶすか、卵と
  小麦粉で作った衣を付け、ラードで揚げる。揚がったらマリナードに戻し
  入れて軽く弱火で煮てから供する(p.36)」あるいは \emph{Longe de
  mouton} 
  (仔羊の腰肉のロースト)のレシピは、「よく熟成させてから棒状に切っ
  た豚背脂をラルデ針を使って刺し込み、串を刺してローストする。玉ねぎ、
  塩、こしょう、ごく少量のオレンジまたはレモンの外皮{[}ゼスト{]}とブイ
  ヨンとヴィネガーでマリナードを作る。肉に火が通ったら、ソース{[}マリ
  ナード{]}とともに弱火で煮込む。とろみ付けには前項同様にした小麦粉
  {[}小麦粉をラードで茶色くなるまで炒めたもの、すなわち『料理の手引き』
  の時代のルーの原型{]}を少々加える(p.80)」とあり、別の項目では「(串
  を刺した肉の下の受け皿にある)マリナードを小まめにかけながら{[}アロ
  ゼしながら{]}ローストする(p.106)」という表現がある。レシピ数からい
  うとラ・ヴァレーヌにおいてマリナードとは中世のドディーヌにヴィネガー
  を効かせたもののようにも受け取れるが、最初に見たように、「漬け込む」
  ものとしてヴィネガーを用いている点に注目すべきだろう。この流れは18、
  19世紀に引継がれる。1756年マラン『コモス神の贈り物』第1巻において、
  \emph{Cervelle de veau en
  marinade}(仔牛の脳のマリナード仕立て)などの
  レシピがあり、血抜きした仔牛の脳を豚背脂のシートで包みブイヨン少々
  で茹で、「冷ましてからヴィネガーもしくはレモン果汁に漬け込む。その
  後、水気をきって溶き卵に浸し、パン粉をつけて揚げる。小麦粉を溶いた
  揚げ衣に浸して揚げてもいい(p.206)」とある。19世紀のヴィアールでも
  同様の料理は見られる。『帝国料理の本』初版(1806年)において、
  \emph{Pieds d'agneau en marinade} 仔羊の足のマリナード仕立てなどいくつ
  かのmarinadeを冠するレシピが掲載されている。肝心のマリナードについ
  ての記述は欠落しているが、この版においてはよく見られる現象。なお、
  仔羊の足のマリナード仕立ては、マリナードがない場合には「塩、こしょ
  う、ビネガーに茹でた仔羊の足を漬けてから、揚げ衣を付けて揚げる
  (p.214)」となっている。1814年ボヴィリエ『調理技法』では「加熱マリ
  ナード」のレシピが掲載されている。これは、卵くらいの大きさのバター
  を鍋に入れ、輪切りにしたにんじん1、2本、同様にした玉ねぎ、ローリエ
  の葉1枚、にんにく1片、タイム、バジル、枝ごとのパセリ、シブール{[}≒
  葱{]}2〜3本を加えて強火で炒める。野菜が色付きはじめたら、約250mlの
  白ワインヴィネガーと約0.5 Lの水を注ぎ、塩、こしょうする。そのまま
  沸かして、漉し器で漉して、必要に応じて使う(pp.60-61)、というもの。
  もっとも、仔牛の脳のマリナード仕立てなどマランのレシピと大差ない揚
  げものが目に付く。また、1834年版のオドにおいても鶏のマリナードはラ・
  ヴァレーヌのものと同工異曲に留まっている。1837年版ではロースト用マ
  リナードの項が追加され、豚背脂とにんにく1片を細かく刻み、パセリ1つ
  まり、塩、こしょう、ヴィネガー大さじ1杯、油大さじ4杯を合わせてよく
  混ぜる(p.419)。1853年版ではマリネしたうなぎのグリル焼き、というレ
  シピが掲載される。これは、皮を剥いてぶつ切りにし、バターでソテーし
  たうなぎを深皿に並べ、塩、こしょうハーブ、マッシュルーム、細かく刻
  んだエシャロットとシブールを被せ、油大さじ1杯をかける。2〜3時間マ
  リネしたら、パン粉をまぶしてグリル焼きする(p.310)というもの。いっ
  ぽう、mariner(マリネ)という動詞については、オドの1834年版で既に、
  ノロ鹿の腿肉のローストにおいて、「オリーブオイルと塩で5〜6時間マリ
  ネする」(p.155)という記述が見られる。1867年刊グフェ『料理の本』に
  おいては、ヴィネガーをベースとしたソースとしてのマリナード(p.404)
  と仕立てとしてのマリナードがあるが、後者もこんにちの概念に近く、例
  えば \emph{Tête de veau en marinade}
  (仔牛の頭 マリナード仕立て)は、 仔牛の頭肉半分を3
  cm角に切り、下茹でしてから水にさらし、牛脂と小麦
  粉、香草類を加えたブランで茹でる。これを、塩、こしょう、油、ヴィネ
  ガーに1時間漬け込む。水気をきって揚げ衣を付けて油で揚げる、という
  もの(p.156)。ここでは肉を漬け込む液体としてmarinadeの語が用いられ
  ている。このように、marinadeという名詞とmariner「漬け込む」という
  動詞の用法にややずれが見られるため、『料理の手引き』におけるマリナー
  ドすなわちマリネ液、という概念は19世紀後半になってからのものと思わ
  れる。}}{マリナードとソミュール}}\label{ux30deux30eaux30caux30fcux30c9ux3068ux30bdux30dfux30e5ux30fcux30eb1}}

\hypertarget{marinades-et-saumucres}{%
\subsection{Marinades et Saumures}\label{marinades-et-saumucres}}

\index{marinade saumures@marinade et saumures} \index{marinade@marinade}
\index{saumure@saumure}

マリナードとソミュールにはいろいろな種類があるが、最終的な目的は同じで、

\begin{enumerate}
\def\labelenumi{\arabic{enumi}.}
\item
  素材に料理で使う香辛料やハーブの香りを浸み込ませる
\item
  ある種の肉を柔らかくさせる
\item
  場合によっては保存のために用いる。とりわけ温度と湿度で素材が駄目になってしまうような場合。さらに、目指す料理の仕上がりに合わせて素材の状態を調節する
\end{enumerate}
\begin{recette}
\hypertarget{sokuseki-marinade}{%
\subsubsection{即席マリナード}\label{sokuseki-marinade}}

\hypertarget{marinade-instantanee}{%
\subsubsection{Marinade instantanée}\label{marinade-instantanee}}

\index{marinade@marinade!marinade instantanee@marinade instantanée}
\index{まりなーと@マリナード!そくせき@即席---}

このマリナードはすぐに素材を使う場合、例えば赤身肉のグリル焼きや、ガランティーヌ、テリーヌ、パテのような冷製料理の補助材料\footnote{具体的には\protect\hyperlink{}{ファルス}のこと。}にする肉に用いる。

\begin{enumerate}
\def\labelenumi{\arabic{enumi}.}
\item
  グリル焼きにする肉の場合\ldots{}\ldots{}ごく薄くスライスしたエシャロットとパセリの枝、タイムの枝、ローリエの葉を肉の上に散らす。量は適宜加減すること。レモン果汁\undemi{}個分に対して油大さじ1杯の割合で、上からかけてやる。
\item
  仔牛、ジビエのフィレ肉、ハム、豚背脂などを細かく切ったもの\footnote{原文
    lardon (ラルドン)、通常は拍子木状に切ったものを言うが、こ
    こではファルスとして後で細かく挽くことになるので、形状はあまり問題
    にならない。}の場合\ldots{}\ldots{}塩こしょうしてから、白ワイン3、コニャック3、油1の割合のマリナードを上からかけてやる。
\end{enumerate}

ここで用いた風味付けの材料は、後でファルスにする際に加えることになる。

いずれの場合でも、マリナードに浸した肉を小まめに裏返してやり、マリナードがよく浸み込むようにしてやること。

\maeaki

\hypertarget{ushi-hitsuji-oogatajibie-youno-hikanetsu-marinade}{%
\subsubsection{牛、羊肉および大型ジビエ用の非加熱マリナード}\label{ushi-hitsuji-oogatajibie-youno-hikanetsu-marinade}}

\hypertarget{marinade-crue-pour-viandes-de-boucherie-ou-venaison}{%
\paragraph{Marinade crue pour viandes de boucherie ou
venaison}\label{marinade-crue-pour-viandes-de-boucherie-ou-venaison}}

\index{marinade@marinade!marinade crue viande boucherie venaison@marinade crue pour viande de boucherie ou venaison}
\index{まりなーと@マリナード!うしひつしおおかたしひえようひかねつ@牛、羊肉および大型ジビエ用非加熱---}

(仕上り2 L分)

\begin{itemize}
\item
  \textbf{香味素材}\ldots{}\ldots{}にんじん100 g、玉ねぎ100
  g、エシャロット40 g、セロリ30
  g、にんにく2片、パセリの枝3本、タイム1枝、ローリエの葉\undemi{}枚、大粒のこしょう6個、クローブ2本。
\item
  \textbf{使用する液体}\ldots{}\ldots{}白ワイン1\unquart{}
  L、ヴィネガー5 dl、油2\undemi{} dl。
\item
  \textbf{作業手順}\ldots{}\ldots{}マリネする素材に塩とこしょうを振る。にんじん、玉ねぎ、エシャロットを薄切り\footnote{émincer
    (エマンセ)薄切りにする、スライスする。}にし、半量を容器の底に敷く。容器の大きさは素材とマリナードがぴったり入る程度のものを用いること。素材を入れて、残りの香味野菜で蓋をするようにして、白ワインとヴィネガー、油を注ぎ入れる。
\end{itemize}

冷蔵し、マリネ液に漬かった素材を小まめに裏返してやること。

\maeaki

\hypertarget{ushi-hitsuji-oogatajibie-youno-kanetsu-marinade}{%
\subsubsection{牛、羊肉および大型ジビエ用の加熱マリナード}\label{ushi-hitsuji-oogatajibie-youno-kanetsu-marinade}}

\hypertarget{marinade-cuite-pour-viandes-de-boucherie-ou-venaison}{%
\paragraph{Marinade cuite pour viandes de boucherie ou
venaison}\label{marinade-cuite-pour-viandes-de-boucherie-ou-venaison}}

\index{marinade@marinade!marinade cuite viande boucherie venaison@marinade cuite pour viande de boucherie ou venaison}
\index{まりなーと@マリナード!うしひつしおおかたしひえようかねつ@牛、羊肉および大型ジビエ用加熱---}

(仕上り2 L分)

\begin{itemize}
\item
  \textbf{香味素材}\ldots{}\ldots{}非加熱マリナードと同じ材料で同じ分量
\item
  \textbf{使用する液体}\ldots{}\ldots{}白ワイン1\undemi{} L、ヴィネガー3
  dl、油2\undemi{} dl。
\item
  \textbf{作業手順}\ldots{}\ldots{}鍋に油を熱し、ごく薄くスライスしたにんじん、玉ねぎ、
  エシャロットおよびその他の香味素材を軽く色付くまで炒める。

  白ワインとヴィネガーを注ぎ、弱火で約30分間火を通す。

  必ず、マリナードが完全に冷めてからマリネする素材にかけること。
\end{itemize}

\maeaki

\hypertarget{toriwake-oogatano-jibieyou-hikanetsu-oyobi-kanetsu-marinade}{%
\subsubsection[とりわけ大型のジビエ用、非加熱および加熱マリナード]{\texorpdfstring{とりわけ大型のジビエ\footnote{具体的には鹿
  cerf(セール) の成獣など。ニホンジカやエゾジカは
  cerfに分類されるので、これを参考にするといいだろう。}用、非加熱および加熱マリナード}{とりわけ大型のジビエ用、非加熱および加熱マリナード}}\label{toriwake-oogatano-jibieyou-hikanetsu-oyobi-kanetsu-marinade}}

\hypertarget{marinade-crue-ou-cuite-pour-grosse-venaison}{%
\paragraph{Marinade crue ou cuite pour grosse
venaison}\label{marinade-crue-ou-cuite-pour-grosse-venaison}}

\index{marinade@marinade!marinade crue cuite grosse venaison@marinade crue ou cuite pour grosse venaison}
\index{まりなーと@マリナード!とりわけおおかたのしひえようひかねつおよひかねつ@とりわけ大型のジビエ用非加熱および加熱---}

(仕上り2 L分)

\begin{itemize}
\item
  \textbf{香味素材}\ldots{}\ldots{}牛、羊肉および大型ジビエ用のマリナードと同じだが、ローズマリー12
  gを追加する。
\item
  \textbf{使用する液体}\ldots{}\ldots{}ヴィネガー16 dl、油4 dl。
\item
  \textbf{作業手順}\ldots{}\ldots{}非加熱、加熱ともに作業手順は上記のレシピのとおり。
\end{itemize}

\maeaki

\hypertarget{ux7f8aux306eux30b7ux30e5ux30f4ux30ebux30a4ux30e6ux4ed5ux7acbux30666ux7528ux306eux52a0ux71b1ux30deux30eaux30caux30fcux30c9}{%
\subsubsection[羊のシュヴルイユ仕立て用の加熱マリナード]{\texorpdfstring{羊のシュヴルイユ仕立て\footnote{\protect\hyperlink{sauce-chevreuil}{ソース・シュヴルイユ}参照。}用の加熱マリナード}{羊のシュヴルイユ仕立て用の加熱マリナード}}\label{ux7f8aux306eux30b7ux30e5ux30f4ux30ebux30a4ux30e6ux4ed5ux7acbux30666ux7528ux306eux52a0ux71b1ux30deux30eaux30caux30fcux30c9}}

\hypertarget{marinade-cuite-pour-le-mouton-en-chevreuil}{%
\paragraph{Marinade cuite pour le mouton en
chevreuil}\label{marinade-cuite-pour-le-mouton-en-chevreuil}}

\index{marinade@marinade!marinade cuite mouton en chevreuil@marinade cuite pour le mouton en chevreuil}
\index{まりなーと@マリナード!ひつしのしゆうるいゆしたてようのかねつまりなーと@羊のシュヴルイユ仕立て用加熱---}

(仕上り2 L分)

\begin{itemize}
\item
  \textbf{香味素材}\ldots{}\ldots{}上記のとおりの分量の素材に、ジュニパーベリー\footnote{セイヨウネズの実。ジンの香り付けに用いられている。}10粒とバジル1つまみ、ローズマリー1つまみを足す。
\item
  \textbf{使用する液体}\ldots{}\ldots{}牛、羊および大型ジビエ用の加熱マリナードと同じ。
\item
  \textbf{作業手順}\ldots{}\ldots{}鍋に油を熱し、薄切りにしたにんじん、玉ねぎ、エシャロットおよびその他の香味素材を軽く色付くまで炒める。

  白ワインとヴィネガーを注ぎ、弱火で約30分間火を通す。
\end{itemize}

\maeaki

\hypertarget{ux7f8aux306eux30b7ux30e3ux30e2ux30efux4ed5ux7acbux30668ux7528ux306eux52a0ux71b1ux30deux30eaux30caux30fcux30c9}{%
\subsubsection[羊のシャモワ仕立て用の加熱マリナード]{\texorpdfstring{羊のシャモワ仕立て\footnote{オートザルプ県の山岳地帯およびピレネー山脈に生息する野生の山羊。
  ピレネー山脈のものは Isard (イザール)と呼ばれる。若い獣の肉は大
  型ジビエのなかでもとりわけ美味とされる。成獣の肉は固く、しっかりマ
  リネする必要があると言われている。しばしばノロ鹿と比較される。ここ
  では、羊肉を白ワインベースのマリナードに漬け込む仕立て、すなわちシュ
  ヴルイユ仕立てとの対比として、赤ワインでより強い風味のマリナードに
  漬け込むことで、シャモワ仕立てとしている。なお、本書においてシャモ
  ワ仕立てを料理名に謳ったレシピは掲載されていないので注意。基本的に
  はシュヴルイユ仕立てと同様に調理するといいだろう。}用の加熱マリナード}{羊のシャモワ仕立て用の加熱マリナード}}\label{ux7f8aux306eux30b7ux30e3ux30e2ux30efux4ed5ux7acbux30668ux7528ux306eux52a0ux71b1ux30deux30eaux30caux30fcux30c9}}

\hypertarget{marinade-cuite-pour-le-mouton-en-chamois}{%
\paragraph{Marinade cuite pour le mouton en
chamois}\label{marinade-cuite-pour-le-mouton-en-chamois}}

\index{marinade@marinade!marinade cuite mouton en chevreuil@marinade cuite pour le mouton en chevreuil}
\index{まりなーと@マリナード!ひつしのしやもわしたてようのかねつまりなーと@羊のシャモワ仕立て用加熱---}

(仕上り2 L分)

\begin{itemize}
\item
  \textbf{香味素材}\ldots{}\ldots{}非加熱マリナードと同じ分量の素材に、ジュニパーベリー\footnote{セイヨウネズの実。ジンの香り付けに用いられている。}15粒とバジル15
  g、ローズマリー15 gを足す。
\item
  \textbf{使用する液体}\ldots{}\ldots{}良質な赤ワイン1\undemi{}
  L、ヴィネガー3 dl、油2\undemi{} dl。
\item
  \textbf{作業手順}\ldots{}\ldots{}上記と同じ。

  このマリナードに上等な赤ワインを使える場合には、素材の量を次のように
  調整すること。赤ワイン12 dl、ワインヴィネガー6 dl、油は上記の分量と
  する。

  ワインの酸味の強さによっては、ヴィネガーの量をワインと同量にすることさえ可能。
\end{itemize}

\hypertarget{observation-sur-les-marinades}{%
\subparagraph{マリナードについての注意事項}\label{observation-sur-les-marinades}}

\ldots{}\ldots{} 1.
加熱マリナードを使用するのは、素材へのマリナードの浸透作用を促進するのが目的。\\
素材をマリナードに漬け込む時間は、加熱、非加熱ともに、素材の種類と大き
さ、気温、環境の変化を勘案して決めること。

\begin{enumerate}
\def\labelenumi{\arabic{enumi}.}
\setcounter{enumi}{1}
\tightlist
\item
  一般的な牛、羊肉と肉質の柔らかい大型ジビエに使うマリナードに純粋な
  酢酸を用いるのは絶対にやめておくこと。酢酸の腐食作用によって肉の風
  味が失なわれてしまうからだ。\\
  猪、鹿\footnote{cerf
    (セール)、ニホンジカやエゾジカ、ヨーロッパでは赤鹿を指す。}、トナカイなどの固い肉についても、純粋な酢酸だけを使うの
  は不可。
\end{enumerate}

\hypertarget{ux30deux30eaux30caux30fcux30c9ux306eux4fddux5b58ux65b9ux6cd5}{%
\subsubsection{マリナードの保存方法}\label{ux30deux30eaux30caux30fcux30c9ux306eux4fddux5b58ux65b9ux6cd5}}

\hypertarget{conservation-des-marinades}{%
\paragraph{Conservation des
marinades}\label{conservation-des-marinades}}

\index{marinade@marinade!conservation marinades@conservation des marinades}
\index{まりなーと@マリナード!ほそんほうほう@---の保存方法}

マリナードを長期間保存しておく必要がある場合には、とりわけ夏場は、本書
で示した分量に対して2〜3 gのホウ酸を加えるといい。

あえに、夏のあいだは2日に一度、冬季は4〜5日に一度、マリナードを沸騰さ
せ、冷めたら毎回そのマリナードに使っているのと同じワインを 2dlとヴィネ
ガー1 dlを足してやること。
\end{recette}
\hypertarget{ux30bdux30dfux30e5ux30fcux30eb}{%
\subsection{ソミュール}\label{ux30bdux30dfux30e5ux30fcux30eb}}

\vspace*{-2\zw}

\hypertarget{saumures}{%
\subsection{Saumures}\label{saumures}}
\begin{recette}
\hypertarget{ux5869ux6f2cux3051ux7528ux30bdux30dfux30e5ux30fcux30eb}{%
\subsubsection{塩漬け用ソミュール}\label{ux5869ux6f2cux3051ux7528ux30bdux30dfux30e5ux30fcux30eb}}

\hypertarget{saumure-au-sel}{%
\subsubsection{Saumure au sel}\label{saumure-au-sel}}

このソミュールは、グレーソルト\footnote{フランス語は sel gris
  (セルグリ)または gros gris (グログリ)。灰色がかった粗塩。}1
kgに対して硝石\footnote{硝酸カリウム。殺菌作用と、肉類を赤く発色させる効果を持つ。現代
  の日本では亜硝酸カリウム、亜硝酸ナトリウムが使われることが多い。い
  ずれも日本では劇物指定されているが、シャルキュトリ(豚肉加工品の製
  造)においては不可欠な薬品であり、食品添加物として使用限界量が厳密
  に定められている。}40
gの割合で作る。この硝石入りの塩の総量は、塩漬けにする肉の数と大きさで決まる。素材が完全に覆えて、重しが出来る分量とすること。
\end{recette}