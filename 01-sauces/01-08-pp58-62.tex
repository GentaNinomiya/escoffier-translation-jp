\hypertarget{ux30deux30eaux30caux30fcux30c9ux3068ux30bdux30dfux30e5ux30fcux30eb1}{%
\section[マリナードとソミュール]{\texorpdfstring{マリナードとソミュール\footnote{マリナードはマリネ液とも言う。marinade
  \textless{} mariner (マリネ)語源
  はラテン語のmare(海)。中世フランス語ではもっぱら「海で泳ぐ、海に
  潜る」の意で使われていたが、16世紀には既に、料理用語として用いられ
  ていたようだ。ラブレー『ガルガンチュアとパンタグリュエル』第四の書
  (1548年)において、lancerons marinez (マリネしたブロシェの幼魚)
  という表現が見られる。なおブロシェ brochet はノーザンパイク、和名
  キタカワカマス。川カマス属の淡水、汽水魚。この場面はパンタグリュエ
  ルに「小斉」のご馳走として捧げられた料理のリストの一部であり、「塩
  漬けのメルルーサ、卵料理各種、モリュ(塩漬けにした鱈)、アドック
  (塩漬け後に燻製にした鱈)」などとともに列挙されており、いずれも塩
  辛いために、それらを調理したものを食べて消化をよくするために飲むワ
  インの量が倍になった(p.681)とある。したがって、lancerons marinezの
  マリネとは「海水あるいは塩水に漬けた」の意に解釈されよう。一方、ソ
  ミュールについては、11世紀末頃に、「保存のため漬け込む塩水」の意味 で
  salmuire という語形が使用され、16世紀には「塩水およびその他の液
  体からなるもの」としてsaumureという現在とおなじ語形が記録されてい
  る。マリナードとソミュールが明確に分化したのはおそらく17世紀頃で、
  1651年刊ラ・ヴァレーヌ『フランス料理の本』に見られるマリナードの語
  には曖昧さを免れないものもあるが、例えば \emph{Poulets marinez} (鶏の
  マリネ)というレシピは「鶏を開いて叩き、しっかり味付けしたヴィネガー
  に漬ける。提供直前に小麦粉をまぶすか、卵と小麦粉で作った衣を付け、
  ラードで揚げる。揚がったらマリナードに戻し入れて軽く弱火で煮てから
  供する(p.36)」あるいは \emph{Longe de
  mouton} (仔羊の腰肉のロースト)
  のレシピは、「よく熟成させてから棒状に切った豚背脂をラルデ針を使っ
  て刺し込み、串を刺してローストする。玉ねぎ、塩、こしょう、ごく少量
  のオレンジまたはレモンの外皮{[}ゼスト{]}とブイヨンとヴィネガーでマリ
  ナードを作る。肉に火が通ったら、ソース{[}マリナード{]}とともに弱火で
  煮込む。とろみ付けには前項同様にした小麦粉{[}小麦粉をラードで茶色く
  なるまで炒めたもの、すなわち『料理の手引き』の時代のルーの原型{]}を
  少々加える(p.80)」とあり、別の項目では「(串を刺した肉の下の受け皿
  にある)マリナードを小まめにかけながら{[}アロゼしながら{]}ローストす
  る(p.106)」という表現がある。レシピ数からいうとラ・ヴァレーヌにお
  いてマリナードとは中世のドディーヌにヴィネガーを効かせたもののよう
  にも受け取れるが、最初に見たように、「漬け込む」ものとしてヴィネガー
  を用いている点に注目すべきだろう。この流れは18、19世紀に引継がれる。
  1756年マラン『コモス神の贈り物』第1巻において、\emph{Cervelle de veau
  en marinade}(仔牛の脳のマリナード仕立て)などのレシピがあり、血抜
  きした仔牛の脳を豚背脂のシートで包みブイヨン少々で茹で、「冷まして
  からヴィネガーもしくはレモン果汁に漬け込む。その後、水気をきって溶
  き卵に浸し、パン粉をつけて揚げる。小麦粉を溶いた揚げ衣に浸して揚げ
  てもいい(p.206)」とある。19世紀のヴィアールでも同様の料理は見られ
  る。『帝国料理の本』初版(1806年)において、\emph{Pieds d'agneau en
  marinade} 仔羊の足のマリナード仕立てなどいくつかのmarinadeを冠する
  レシピが掲載されている。肝心のマリナードについての記述は欠落してい
  るが、この版においてはよく見られる現象。なお、仔羊の足のマリナード
  仕立ては、マリナードがない場合には「塩、こしょう、ビネガーに茹でた
  仔羊の足を漬けてから、揚げ衣を付けて揚げる(p.214)」となっている。
  1814年ボヴィリエ『調理技法』では「加熱マリナード」のレシピが掲載さ
  れている。これは、卵くらいの大きさのバターを鍋に入れ、輪切りにした
  にんじん1、2本、同様にした玉ねぎ、ローリエの葉1枚、にんにく1片、タ
  イム、バジル、枝ごとのパセリ、シブール{[}≒葱{]}2〜3本を加えて強火で
  炒める。野菜が色付きはじめたら、約250mlの白ワインヴィネガーと約0.5
  Lの水を注ぎ、塩、こしょうする。そのまま沸かして、漉し器で漉し、必
  要に応じて使う(pp.60-61)、というもの。もっとも、仔牛の脳のマリナー
  ド仕立てなどマランのレシピと大差ない揚げものも同書では目に付く。ま
  た、1834年版のオドにおいても鶏のマリナードはラ・ヴァレーヌのものと
  同工異曲に留まっている。1837年版ではロースト用マリナードの項が追加
  され、豚背脂とにんにく1片を細かく刻み、パセリ1つまり、塩、こしょう、
  ヴィネガー大さじ1杯、油大さじ4杯を合わせてよく混ぜる(p.419)。1853
  年版ではマリネしたうなぎのグリル焼き、というレシピが掲載される。こ
  れは、皮を剥いてぶつ切りにし、バターでソテーしたうなぎを深皿に並べ、
  塩、こしょうハーブ、マッシュルーム、細かく刻んだエシャロットとシブー
  ルを被せ、油大さじ1杯をかける。2〜3時間マリネしたら、パン粉をまぶ
  してグリル焼きする(p.310)というもの。いっぽう、mariner(マリネ)と
  いう動詞については、オドの1834年版で既に、ノロ鹿の腿肉のローストに
  おいて、「オリーブオイルと塩で5〜6時間マリネする」(p.155)という記
  述が見られる。1867年刊グフェ『料理の本』においては、ヴィネガーをベー
  スとしたソースとしてのマリナード(p.404)と仕立てとしてのマリナー
  ドがあるが、後者はこんにちの概念に近く、例えば \emph{Tête de veau en
  marinade} (仔牛の頭 マリナード仕立て)は、仔牛の頭肉半分を3 cm角
  に切り、下茹でしてから水にさらし、牛脂と小麦粉、香草類を加えたブラ
  ンで茹でる。これを、塩、こしょう、油、ヴィネガーに1時間漬け込む。
  水気をきって揚げ衣を付けて油で揚げる、というもの(p.156)。ここでは
  肉を漬け込む液体としてmarinadeの語が用いられている。このように、
  marinadeという名詞とmariner「漬け込む」という動詞の用法にややずれ
  が見られるため、『料理の手引き』におけるマリナードすなわちマリネ液、
  という概念は19世紀後半になってからのものと思われる。}}{マリナードとソミュール}}\label{ux30deux30eaux30caux30fcux30c9ux3068ux30bdux30dfux30e5ux30fcux30eb1}}

\hypertarget{marinades-et-saumucres}{%
\subsection{Marinades et Saumures}\label{marinades-et-saumucres}}

\index{marinade saumures@marinade et saumures} \index{marinade@marinade}
\index{saumure@saumure}

マリナードとソミュールにはいろいろな種類があるが、最終的な目的は同じで、

\begin{enumerate}
\def\labelenumi{\arabic{enumi}.}
\item
  素材に料理で使う香辛料やハーブの香りを浸み込ませる
\item
  ある種の肉を柔らかくさせる
\item
  場合によっては保存のために用いる。とりわけ温度と湿度で素材が駄目になってしまうような場合。さらに、目指す料理の仕上がりに合わせて素材の状態を調節する
\end{enumerate}
\begin{recette}
\hypertarget{sokuseki-marinade}{%
\subsubsection{即席マリナード}\label{sokuseki-marinade}}

\hypertarget{marinade-instantanee}{%
\subsubsection{Marinade instantanée}\label{marinade-instantanee}}

\index{marinade@marinade!marinade instantanee@marinade instantanée}
\index{まりなーと@マリナード!そくせき@即席---}

このマリナードはすぐに素材を使う場合、例えば赤身肉のグリル焼きや、ガランティーヌ、テリーヌ、パテのような冷製料理の補助材料\footnote{具体的には\protect\hyperlink{}{ファルス}のこと。}にする肉に用いる。

\begin{enumerate}
\def\labelenumi{\arabic{enumi}.}
\item
  グリル焼きにする肉の場合\ldots{}\ldots{}ごく薄くスライスしたエシャロットとパセリの枝、タイムの枝、ローリエの葉を肉の上に散らす。量は適宜加減すること。レモン果汁\undemi{}個分に対して油大さじ1杯の割合で、上からかけてやる。
\item
  仔牛、ジビエのフィレ肉、ハム、豚背脂などを細かく切ったもの\footnote{原文
    lardon (ラルドン)、通常は拍子木状に切ったものを言うが、こ
    こではファルスとして後で細かく挽くことになるので、形状はあまり問題
    にならない。}の場合\ldots{}\ldots{}塩こしょうしてから、白ワイン3、コニャック3、油1の割合のマリナードを上からかけてやる。
\end{enumerate}

ここで用いた風味付けの材料は、後でファルスにする際に加えることになる。

いずれの場合でも、マリナードに浸した肉を小まめに裏返してやり、マリナードがよく浸み込むようにしてやること。

\maeaki

\hypertarget{ushi-hitsuji-oogatajibie-youno-hikanetsu-marinade}{%
\subsubsection{牛、羊肉および大型ジビエ用の非加熱マリナード}\label{ushi-hitsuji-oogatajibie-youno-hikanetsu-marinade}}

\hypertarget{marinade-crue-pour-viandes-de-boucherie-ou-venaison}{%
\paragraph{Marinade crue pour viandes de boucherie ou
venaison}\label{marinade-crue-pour-viandes-de-boucherie-ou-venaison}}

\index{marinade@marinade!marinade crue viande boucherie venaison@marinade crue pour viande de boucherie ou venaison}
\index{まりなーと@マリナード!うしひつしおおかたしひえようひかねつ@牛、羊肉および大型ジビエ用非加熱---}

(仕上り2 L分)

\begin{itemize}
\item
  \textbf{香味素材}\ldots{}\ldots{}にんじん100 g、玉ねぎ100
  g、エシャロット40 g、セロリ30
  g、にんにく2片、パセリの枝3本、タイム1枝、ローリエの葉\undemi{}枚、大粒のこしょう6個、クローブ2本。
\item
  \textbf{使用する液体}\ldots{}\ldots{}白ワイン1\unquart{}
  L、ヴィネガー5 dl、油2\undemi{} dl。
\item
  \textbf{作業手順}\ldots{}\ldots{}マリネする素材に塩とこしょうを振る。にんじん、玉ねぎ、エシャロットを薄切り\footnote{émincer
    (エマンセ)薄切りにする、スライスする。}にし、半量を容器の底に敷く。容器の大きさは素材とマリナードがぴったり入る程度のものを用いること。素材を入れて、残りの香味野菜で蓋をするようにして、白ワインとヴィネガー、油を注ぎ入れる。
\end{itemize}

冷蔵し、マリネ液に漬かった素材を小まめに裏返してやること。

\maeaki

\hypertarget{ushi-hitsuji-oogatajibie-youno-kanetsu-marinade}{%
\subsubsection{牛、羊肉および大型ジビエ用の加熱マリナード}\label{ushi-hitsuji-oogatajibie-youno-kanetsu-marinade}}

\hypertarget{marinade-cuite-pour-viandes-de-boucherie-ou-venaison}{%
\paragraph{Marinade cuite pour viandes de boucherie ou
venaison}\label{marinade-cuite-pour-viandes-de-boucherie-ou-venaison}}

\index{marinade@marinade!marinade cuite viande boucherie venaison@marinade cuite pour viande de boucherie ou venaison}
\index{まりなーと@マリナード!うしひつしおおかたしひえようかねつ@牛、羊肉および大型ジビエ用加熱---}

(仕上り2 L分)

\begin{itemize}
\item
  \textbf{香味素材}\ldots{}\ldots{}非加熱マリナードと同じ材料で同じ分量
\item
  \textbf{使用する液体}\ldots{}\ldots{}白ワイン1\undemi{} L、ヴィネガー3
  dl、油2\undemi{} dl。
\item
  \textbf{作業手順}\ldots{}\ldots{}鍋に油を熱し、ごく薄くスライスしたにんじん、玉ねぎ、
  エシャロットおよびその他の香味素材を軽く色付くまで炒める。

  白ワインとヴィネガーを注ぎ、弱火で約30分間火を通す。

  必ず、マリナードが完全に冷めてからマリネする素材にかけること。
\end{itemize}

\maeaki

\hypertarget{toriwake-oogatano-jibieyou-hikanetsu-oyobi-kanetsu-marinade}{%
\subsubsection[とりわけ大型のジビエ用、非加熱および加熱マリナード]{\texorpdfstring{とりわけ大型のジビエ\footnote{具体的には鹿
  cerf(セール) の成獣など。ニホンジカやエゾジカは
  cerfに分類されるので、これを参考にするといいだろう。}用、非加熱および加熱マリナード}{とりわけ大型のジビエ用、非加熱および加熱マリナード}}\label{toriwake-oogatano-jibieyou-hikanetsu-oyobi-kanetsu-marinade}}

\hypertarget{marinade-crue-ou-cuite-pour-grosse-venaison}{%
\paragraph{Marinade crue ou cuite pour grosse
venaison}\label{marinade-crue-ou-cuite-pour-grosse-venaison}}

\index{marinade@marinade!marinade crue cuite grosse venaison@marinade crue ou cuite pour grosse venaison}
\index{まりなーと@マリナード!とりわけおおかたのしひえようひかねつおよひかねつ@とりわけ大型のジビエ用非加熱および加熱---}

(仕上り2 L分)

\begin{itemize}
\item
  \textbf{香味素材}\ldots{}\ldots{}牛、羊肉および大型ジビエ用のマリナードと同じだが、ローズマリー12
  gを追加する。
\item
  \textbf{使用する液体}\ldots{}\ldots{}ヴィネガー16 dl、油4 dl。
\item
  \textbf{作業手順}\ldots{}\ldots{}非加熱、加熱ともに作業手順は上記のレシピのとおり。
\end{itemize}

\maeaki

\hypertarget{ux7f8aux306eux30b7ux30e5ux30f4ux30ebux30a4ux30e6ux4ed5ux7acbux30666ux7528ux306eux52a0ux71b1ux30deux30eaux30caux30fcux30c9}{%
\subsubsection[羊のシュヴルイユ仕立て用の加熱マリナード]{\texorpdfstring{羊のシュヴルイユ仕立て\footnote{\protect\hyperlink{sauce-chevreuil}{ソース・シュヴルイユ}参照。}用の加熱マリナード}{羊のシュヴルイユ仕立て用の加熱マリナード}}\label{ux7f8aux306eux30b7ux30e5ux30f4ux30ebux30a4ux30e6ux4ed5ux7acbux30666ux7528ux306eux52a0ux71b1ux30deux30eaux30caux30fcux30c9}}

\hypertarget{marinade-cuite-pour-le-mouton-en-chevreuil}{%
\paragraph{Marinade cuite pour le mouton en
chevreuil}\label{marinade-cuite-pour-le-mouton-en-chevreuil}}

\index{marinade@marinade!marinade cuite mouton en chevreuil@marinade cuite pour le mouton en chevreuil}
\index{まりなーと@マリナード!ひつしのしゆうるいゆしたてようのかねつまりなーと@羊のシュヴルイユ仕立て用加熱---}

(仕上り2 L分)

\begin{itemize}
\item
  \textbf{香味素材}\ldots{}\ldots{}上記のとおりの分量の素材に、ジュニパーベリー\footnote{セイヨウネズの実。ジンの香り付けに用いられている。}10粒とバジル1つまみ、ローズマリー1つまみを足す。
\item
  \textbf{使用する液体}\ldots{}\ldots{}牛、羊および大型ジビエ用の加熱マリナードと同じ。
\item
  \textbf{作業手順}\ldots{}\ldots{}鍋に油を熱し、薄切りにしたにんじん、玉ねぎ、エシャロットおよびその他の香味素材を軽く色付くまで炒める。

  白ワインとヴィネガーを注ぎ、弱火で約30分間火を通す。
\end{itemize}

\maeaki

\hypertarget{ux7f8aux306eux30b7ux30e3ux30e2ux30efux4ed5ux7acbux30668ux7528ux306eux52a0ux71b1ux30deux30eaux30caux30fcux30c9}{%
\subsubsection[羊のシャモワ仕立て用の加熱マリナード]{\texorpdfstring{羊のシャモワ仕立て\footnote{オートザルプ県の山岳地帯およびピレネー山脈に生息する野生の山羊。
  ピレネー山脈のものは Isard (イザール)と呼ばれる。若い獣の肉は大
  型ジビエのなかでもとりわけ美味とされる。成獣の肉は固く、しっかりマ
  リネする必要があると言われている。しばしばノロ鹿と比較される。ここ
  では、羊肉を白ワインベースのマリナードに漬け込む仕立て、すなわちシュ
  ヴルイユ仕立てとの対比として、赤ワインでより強い風味のマリナードに
  漬け込むことで、シャモワ仕立てとしている。なお、本書においてシャモ
  ワ仕立てを料理名に謳ったレシピは掲載されていないので注意。基本的に
  はシュヴルイユ仕立てと同様に調理するといいだろう。}用の加熱マリナード}{羊のシャモワ仕立て用の加熱マリナード}}\label{ux7f8aux306eux30b7ux30e3ux30e2ux30efux4ed5ux7acbux30668ux7528ux306eux52a0ux71b1ux30deux30eaux30caux30fcux30c9}}

\hypertarget{marinade-cuite-pour-le-mouton-en-chamois}{%
\paragraph{Marinade cuite pour le mouton en
chamois}\label{marinade-cuite-pour-le-mouton-en-chamois}}

\index{marinade@marinade!marinade cuite mouton en chevreuil@marinade cuite pour le mouton en chevreuil}
\index{まりなーと@マリナード!ひつしのしやもわしたてようのかねつまりなーと@羊のシャモワ仕立て用加熱---}

(仕上り2 L分)

\begin{itemize}
\item
  \textbf{香味素材}\ldots{}\ldots{}非加熱マリナードと同じ分量の素材に、ジュニパーベリー\footnote{セイヨウネズの実。ジンの香り付けに用いられている。}15粒とバジル15
  g、ローズマリー15 gを足す。
\item
  \textbf{使用する液体}\ldots{}\ldots{}良質な赤ワイン1\undemi{}
  L、ヴィネガー3 dl、油2\undemi{} dl。
\item
  \textbf{作業手順}\ldots{}\ldots{}上記と同じ。

  このマリナードに上等な赤ワインを使える場合には、素材の量を次のように
  調整すること。赤ワイン12 dl、ワインヴィネガー6 dl、油は上記の分量と
  する。

  ワインの酸味の強さによっては、ヴィネガーの量をワインと同量にすることさえ可能。
\end{itemize}

\hypertarget{observation-sur-les-marinades}{%
\subparagraph{マリナードについての注意事項}\label{observation-sur-les-marinades}}

\ldots{}\ldots{} 1.
加熱マリナードを使用するのは、素材へのマリナードの浸透作用を促進するのが目的。\\
素材をマリナードに漬け込む時間は、加熱、非加熱ともに、素材の種類と大き
さ、気温、環境の変化を勘案して決めること。

\begin{enumerate}
\def\labelenumi{\arabic{enumi}.}
\setcounter{enumi}{1}
\tightlist
\item
  一般的な牛、羊肉と肉質の柔らかい大型ジビエに使うマリナードに純粋な
  酢酸を用いるのは絶対にやめておくこと。酢酸の腐食作用によって肉の風
  味が失なわれてしまうからだ。\\
  猪、鹿\footnote{cerf
    (セール)、ニホンジカやエゾジカ、ヨーロッパでは赤鹿を指す。}、トナカイなどの固い肉についても、純粋な酢酸だけを使うの
  は不可。
\end{enumerate}

\hypertarget{ux30deux30eaux30caux30fcux30c9ux306eux4fddux5b58ux65b9ux6cd5}{%
\subsubsection{マリナードの保存方法}\label{ux30deux30eaux30caux30fcux30c9ux306eux4fddux5b58ux65b9ux6cd5}}

\hypertarget{conservation-des-marinades}{%
\paragraph{Conservation des
marinades}\label{conservation-des-marinades}}

\index{marinade@marinade!conservation marinades@conservation des marinades}
\index{まりなーと@マリナード!ほそんほうほう@---の保存方法}

マリナードを長期間保存しておく必要がある場合には、とりわけ夏場は、本書
で示した分量に対して2〜3 gのホウ酸を加えるといい。

あえに、夏のあいだは2日に一度、冬季は4〜5日に一度、マリナードを沸騰さ
せ、冷めたら毎回そのマリナードに使っているのと同じワインを 2dlとヴィネ
ガー1 dlを足してやること。
\end{recette}
\hypertarget{ux30bdux30dfux30e5ux30fcux30eb}{%
\subsection{ソミュール}\label{ux30bdux30dfux30e5ux30fcux30eb}}

\vspace*{-2\zw}

\hypertarget{saumures}{%
\subsection{Saumures}\label{saumures}}
\begin{recette}
\hypertarget{ux5869ux6f2cux3051ux7528ux30bdux30dfux30e5ux30fcux30eb}{%
\subsubsection{塩漬け用ソミュール}\label{ux5869ux6f2cux3051ux7528ux30bdux30dfux30e5ux30fcux30eb}}

\hypertarget{saumure-au-sel}{%
\subsubsection{Saumure au sel}\label{saumure-au-sel}}

このソミュールは、グレーソルト\footnote{フランス語は sel gris
  (セルグリ)または gros gris (グログリ)。灰色がかった粗塩。}1
kgに対して硝石\footnote{硝酸カリウム。殺菌作用と、肉類を赤く発色させる効果を持つ。現代
  の日本では亜硝酸カリウム、亜硝酸ナトリウムが使われることが多い。い
  ずれも日本では劇物指定されているが、シャルキュトリ(豚肉加工品の製
  造)においては不可欠な薬品であり、食品添加物として使用限界量が厳密
  に定められている。}40
gの割合で作る。この硝石入りの塩の総量は、塩漬けにする肉の数と大きさで決まる。素材が完全に覆えて、重しが出来る分量とすること。
\end{recette}