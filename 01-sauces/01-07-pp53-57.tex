\hypertarget{ux5408ux308fux305bux30d0ux30bfux30fc}{%
\section{合わせバター}\label{ux5408ux308fux305bux30d0ux30bfux30fc}}

\vspace{0\zw}

\hypertarget{ux30b0ux30eaux30ebux30bdux30fcux30b9ux306eux88dcux52a9ux6750ux6599ux30aaux30fcux30c9ux30d6ux30ebux7528}{%
\subsection{グリル、ソースの補助材料、オードブル用}\label{ux30b0ux30eaux30ebux30bdux30fcux30b9ux306eux88dcux52a9ux6750ux6599ux30aaux30fcux30c9ux30d6ux30ebux7528}}

\vspace*{-1.5\zw}

\hypertarget{beurres-composuxe9s-pour-adjuvants-de-sauces-et-hors-doeuvre}{%
\subsection{Beurres Composés pour Adjuvants de Sauces et
Hors-d'oeuvre}\label{beurres-composuxe9s-pour-adjuvants-de-sauces-et-hors-doeuvre}}

\index{あわせはたー@合わせバター} \index{はたー@バター ⇒ 合わせバター}
\index{ふーるこんほせ@ブール・コンポゼ ⇒ 合わせバター}
\index{みつくすはたー@ミックスバター ⇒ 合わせバター}
\index{beurre@beurre!beurres composes@Beurres Composés}

\hypertarget{ux6982ux8aac}{%
\subsection{概説}\label{ux6982ux8aac}}

本書においてレシピを掲載している合わせバター\footnote{beurre composé
  ブール・コンポゼ。ミックスバターとも。少なくとも
  バターは中世以来用長く用いられてきた食材だが、中世〜ルネサンスにお
  いては獣脂(もっぱらラード)のほうが多く用いられる傾向にあった。17
  世紀以降はたとえばラ・ヴァレーヌ『フランス料理の本』におけるアスパ
  ラガスの白いソース添え(\protect\hyperlink{sauce-hollandaise}{ソース・オランデーズ}
  訳注参照)のように、バターを料理に用いることが中世の料理書と比較す
  ると圧倒的に増えたのは事実である。ムノンの1741年刊『ブルジョワ屋敷
  に勤める女性料理人のための本』のバターの項には「良質のバターを用い
  ることは料理でとても重要なことであり、バターが匂いを放っているよう
  ではどんな素晴しい皿も台無しだ。料理担当の女中であればこのことをよ
  く理解しておくことと、良質なバターの価格を手に入れるのに金を惜しん
  ではならないことを肝に銘じておくこと。最良のバターは自然な黄色をし
  ており、白いものは大抵の場合、さして美味しくない。バルボットという
  植物から採った黄色で着色されたバターもある。こういうバターの色は、
  自然なバターの黄色よりもくすんだもので、慣れれば簡単に見分けること
  が出来る(p.320)」}のうちのほとんどは、
甲殻類の合わせバターを除いて、料理に直接用いられることがとても少な
い。とはいえ、合わせバターはさまざまなシチュエーションで役に立つ。
ポタージュでは野菜の合わせバターが、その他の合わせバターはソー
ス作りにおいて有用だ。ソースの風味と性格を明確に伝える決め手になるから
だ。

だから、読者である料理人諸君には、この概説に書いてあることを真剣に読みとっていただきたい。
\href{原文における内容矛盾。この後のパラグラフは甲殻類のバターについての注意点ばかりが目立つ}{}

甲殻類のバターについては、経験上、湯煎にかけながら煮出して\footnote{infuser
  アンフュゼ。}から、氷
水で冷やした陶製の容器に布で漉し入れるのがいい。そうすれば、冷たい状態
で作るよりも赤みがきれいに出る。逆に、熱によって風味の繊細さは失なわれ
てしまい、雑味さえも出てしまう。

この問題点を解決するために、我々は二種類の違うバターを作るという方式を
採ることにした。ひとつは甲殻類の胴のクリーム状の部分と切りくずあるいは
身そのものを生のバターとともに鉢ですり潰して、目の細かい網で裏漉しする
か、布で漉すというもの。このバターはソースに完璧もというべき風味を添え
てくれる。とりわけベシャメルソースをベースとしたソースの場合はそうだ。

もうひとつは、甲殻類の殻だけを用いて、熱して作るものだ。これは「色付け」
の役割しか持たない。この方式はまことに素晴しい結果を得られるので、ぜひ
とも実行していただきたい。

場合によっては、我々はバターを同様の上等な生クリームに代えることがある。
生クリームのほうがバターよりも、素材の持つ風味や香気をよく吸収する。こ
うすればソースやポタージュの仕上げに加えるのに文句ないクリ\footnote{Coulis
  水分のやや多いピュレをイメージするといい。}を作ることが 出来るわけだ。

色付け用のバターを使うと、ソースがきれいに色付き、個性的なソースとなる。
どんな場合でも、カルミン色素\footnote{コチニール色素ともいう。ラックカイガラムシなどを原料として抽出し
  た色素。かつて食品工業において多用されたが、アレルゲンとなることが
  わかり、使用は減っている。現在は代替品としてビーツから抽出したビー
  トレッドなどがの使用が増えている。}よりもずっといい。カルミン色素はソース
やポタージュにくすんだ、なさけない色合いしか与えてはくれないのだ。

合わせバターは一般的に、使う際にその都度作る\footnote{原文 au moment
  (オモモン)その都度、の意。à la minute (アラミニュッ
  ト)と呼ぶ調理現場もある。}ものだが、作り置き
しておかなければならない場合は、白い紙で円筒形に包んで冷蔵保管すること。
\begin{recette}
\hypertarget{ux306bux3093ux306bux304fux30d0ux30bfux30fc}{%
\subsubsection{にんにくバター}\label{ux306bux3093ux306bux304fux30d0ux30bfux30fc}}

\hypertarget{beurre-d-ail}{%
\paragraph{Beurre d'Ail}\label{beurre-d-ail}}

\index{はたー@バター!あわせはたー@合わせバター!にんにくはたー@にんにくバター}
\index{あわせはたー@合わせバター!にんにくはたー@にんにくバター}
\index{にんにく@にんにく!はたー@---バター}
\index{beurre@beurre!beurres composes@Beurres Composés!beurre d'ail@Beurre d'Ail}
\index{ail@ail!beurre@Beurre d'---}

皮を剥いたにんにく200 gを強火でしっかり茹でる\footnote{生のにんにくには胃腸を刺激する酵素が含まれているが、熱により不活
  性化するので、よく火を通す必要がある。}。よく湯をきってから、鉢
に入れてすり潰し、バター250 gと合わせ、布で漉す。

\maeaki

\hypertarget{ux30a2ux30f3ux30c1ux30e7ux30d3ux30d0ux30bfux30fc}{%
\subsubsection{アンチョビバター}\label{ux30a2ux30f3ux30c1ux30e7ux30d3ux30d0ux30bfux30fc}}

\hypertarget{beurre-d-anchois}{%
\paragraph{Beurre d'Anchois}\label{beurre-d-anchois}}

\index{はたー@バター!あわせはたー@合わせバター!あんちよひはたー@アンチョビバター}
\index{あわせはたー@合わせバター!あんちよひはたー@アンチョビバター}
\index{あんちよひ@アンチョビ!はたー@---バター}
\index{beurre@beurre!beurres composes@Beurres Composés!beurre d'anchois@Beurre d'Anchois}
\index{anchois@anchois!beurre@Beurre d'---}

アンチョビのフィレ200 gをよく洗い、しっかり水気を絞る。これを鉢に入れ
て細かくすり潰す。バター250 gを加えて布で漉す。

\maeaki

\hypertarget{ux30a2ux30fcux30e2ux30f3ux30c9ux30d0ux30bfux30fc}{%
\subsubsection{アーモンドバター}\label{ux30a2ux30fcux30e2ux30f3ux30c9ux30d0ux30bfux30fc}}

\hypertarget{beurre-d-amande}{%
\paragraph{Beurre d'Amande}\label{beurre-d-amande}}

\index{はたー@バター!あわせはたー@合わせバター!あーもんとはたー@アーモンドバター}
\index{あわせはたー@合わせバター!あーもんとはたー@アーモンドバター}
\index{あーもんと@アーモンド!はたー@---バター}
\index{beurre@beurre!beurres composes@Beurres Composés!beurre d'amande@Beurre d'Amande}
\index{amande@amande!beurre@Beurre d'---}

アーモンド\footnote{アーモンドには一般的なスイートアーモンド amandes
  doucesと、苦味 のあるビターアーモンドamande
  amèresの二種がある。後者はあまり多く
  使われることはないが、香りがいいためリキュールなどの香り付けにごく
  少量が用いられることがある。}150
gを湯むきしてよく洗い、すぐに水数滴を加えてすり潰し
てペースト状にする。これをバター250 gと混ぜ合わせ、布で漉す。

\maeaki

\hypertarget{ux30d6ux30fcux30ebux30c0ux30f4ux30eaux30fcux30cc8}{%
\subsubsection[ブール・ダヴリーヌ]{\texorpdfstring{ブール・ダヴリーヌ\footnote{アヴリーヌはヘーゼルナッツの仲間でセイヨウハシバミの大粒な変種。
  イタリア、ピエモンテ産やシチリア産が有名。}}{ブール・ダヴリーヌ}}\label{ux30d6ux30fcux30ebux30c0ux30f4ux30eaux30fcux30cc8}}

\hypertarget{beurre-d-aveline}{%
\paragraph{Beurre d'Aveline}\label{beurre-d-aveline}}

\index{はたー@バター!あわせはたー@合わせバター!ふーるたうりーぬ@ブール・ダヴリーヌ}
\index{あわせはたー@合わせバター!ふーるたうりーぬ@ブール・ダヴリーヌ}
\index{あうりーぬ@アヴリーヌ!ふーる@ブール・---}
\index{へーせるなつつ@ヘーゼルナッツ!ふーるたうりーぬ@ブール・ダヴリーヌ}
\index{beurre@beurre!beurres composes@Beurres Composés!beurre d'aveline@Beurre d'Aveline}
\index{aveline@aveline!beurre@Beurre d'---}

アヴリーヌ150 gを焙煎して丁寧に皮を剥く。油が浮いてこないよう水を数滴
加えてペースト状にすり潰す。これとバター250 gを混ぜ合わせる。目の細か
い網で裏漉しするか、布で漉す。

\maeaki

\hypertarget{ux30d6ux30fcux30ebux30d9ux30ebux30b7ux30fc9}{%
\subsubsection[ブール・ベルシー]{\texorpdfstring{ブール・ベルシー\footnote{\protect\hyperlink{sauce-bercy}{ソース・ベルシー}訳注参照。}}{ブール・ベルシー}}\label{ux30d6ux30fcux30ebux30d9ux30ebux30b7ux30fc9}}

\hypertarget{beurre-bercy}{%
\paragraph{Beurre Bercy}\label{beurre-bercy}}

\index{はたー@バター!あわせはたー@合わせバター!ふーるへるしー@ブール・ベルシー}
\index{あわせはたー@合わせバター!ふーるたへるしー@ブール・ベルシー}
\index{へるしー@ベルシー!ふーる@ブール・---}
\index{beurre@beurre!beurres composes@Beurres Composés!beurre bercy@Beurre Bercy}
\index{bercy@Bercy!beurre@Beurre ---}

白ワイン2 dlに細かく刻んだエシャロット大さじ1杯を加えて半量になるまで
煮詰める。生温い程度まで冷ましてから、ポマード状に柔らかくしたバター 200
gを混ぜ込む。牛骨髄500 gをさいの目に切って\footnote{原文 couper en
  dés。フランス語のまま「デにする(切る)」と表現することもある。}、沸騰しない程度の
湯で火を通し、よく湯ぎりをして加える。パセリのみじん切り大さじ1杯と塩8
g、挽きたてのこしょう1つまみ強とレモン\undemi{}個分の果汁を加えて仕上
げる。

\maeaki

\hypertarget{ux30adux30e3ux30d3ux30a2ux30d0ux30bfux30fc}{%
\subsubsection{キャビアバター}\label{ux30adux30e3ux30d3ux30a2ux30d0ux30bfux30fc}}

\hypertarget{beurre-de-caviar}{%
\paragraph{Beurre de Caviar}\label{beurre-de-caviar}}

\index{はたー@バター!あわせはたー@合わせバター!きやひあはたー@キャビアバター}
\index{あわせはたー@合わせバター!きやひあはたー@キャビアバター}
\index{きやひあ@キャビア!はたー@---バター}
\index{beurre@beurre!beurres composes@Beurres Composés!beurre caviar@Beurre de Caviar}
\index{caviar@caviar!beurre@Beurre de ---}

圧縮キャビア\footnote{もとはロシアで雪の中の樽で保存するために圧縮したもの。キャビア
  のグレードはベルガ、オセトラ、セヴルガが混ざっているのが多いという。}75
gを細かくすり潰す。パター250 gを加えて、布で漉す。

\maeaki

\hypertarget{ux30d6ux30fcux30ebux30b7ux30f4ux30ea12-ux30d6ux30fcux30ebux30e9ux30f4ux30a3ux30b4ux30c3ux30c813}{%
\subsubsection[ブール・シヴリ /
ブール・ラヴィゴット]{\texorpdfstring{ブール・シヴリ\footnote{\protect\hyperlink{sacue-chivry}{ソース・シヴリ}訳注参照。}
/ ブール・ラヴィゴット\footnote{\protect\hyperlink{sauce-ravigote}{ソース・ラヴィゴット}訳注参照。}}{ブール・シヴリ / ブール・ラヴィゴット}}\label{ux30d6ux30fcux30ebux30b7ux30f4ux30ea12-ux30d6ux30fcux30ebux30e9ux30f4ux30a3ux30b4ux30c3ux30c813}}

\hypertarget{beurre-chivry}{%
\paragraph{Beurre Chivry}\label{beurre-chivry}}

\index{はたー@バター!あわせはたー@合わせバター!しうり@ブール・シヴリ}
\index{あわせはたー@合わせバター!しうり@ブール・シヴリ}
\index{はたー@バター!あわせはたー@合わせバター!らういこつと@ブール・ラヴィゴット}
\index{あわせはたー@合わせバター!らういこつと@ブール・ラヴィゴット}
\index{しうり@シヴリ!ふーる@ブール・---}
\index{らういこつと@ラヴィゴット!ふーる@ブール・---}
\index{beurre@beurre!beurres composes@Beurres Composés!beurre chivry@Beurre Chivrya}
\index{beurre@beurre!beurres composes@Beurres Composés!beurre ravigote@Beurre Ravigote}
\index{chivry@Chivry!beurre@Beurre ---}
\index{ravigote@ravitote!beurre@Beurre ---}

パセリの葉とセルフイユ、エストラゴン、シヴレット、若摘みのサラダバーネッ
ト100 gを数分間下茹でし、水にさらしてから圧して余分な水気を絞る。エシャ
ロットのみじん切り25 gも下茹でする。これらを鉢に入れてすり潰す。

バター125 gを加え、布で漉す。

\maeaki

\hypertarget{ux30d6ux30fcux30ebux30b3ux30ebux30d9ux30fcux30eb14}{%
\subsubsection[ブール・コルベール]{\texorpdfstring{ブール・コルベール\footnote{\protect\hyperlink{sauce-colbert}{ソース・コルベール}本文および訳注参照。}}{ブール・コルベール}}\label{ux30d6ux30fcux30ebux30b3ux30ebux30d9ux30fcux30eb14}}

\hypertarget{beurre-colbert}{%
\paragraph{Beurre Colbert}\label{beurre-colbert}}

\index{はたー@バター!あわせはたー@合わせバター!ふーるこるへーる@ブール・コルベール}
\index{あわせはたー@合わせバター!ふーるこるへーる@ブール・コルベール}
\index{こるへーる@コルベール!ふーる@ブール・---}
\index{beurre@beurre!beurres composes@Beurres Composés!beurre colbert@Beurre Colbert}
\index{colbert@Colbert!beurre@Beurre ---}

\protect\hyperlink{beurre-maitre-d-hotel}{メートルドテルバター}200lgに、溶かした\protect\hyperlink{glace-de-viande}{グラス
ドヴィアンド}大さじ2杯と細かく刻んだエストラゴン小さ じ2杯を加える。
\end{recette}