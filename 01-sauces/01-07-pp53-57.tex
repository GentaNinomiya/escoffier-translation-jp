\hypertarget{ux5408ux308fux305bux30d0ux30bfux30fc}{%
\section{合わせバター}\label{ux5408ux308fux305bux30d0ux30bfux30fc}}

\vspace{0\zw}

\hypertarget{ux30b0ux30eaux30ebux30bdux30fcux30b9ux306eux88dcux52a9ux6750ux6599ux30aaux30fcux30c9ux30d6ux30ebux7528}{%
\subsection{グリル、ソースの補助材料、オードブル用}\label{ux30b0ux30eaux30ebux30bdux30fcux30b9ux306eux88dcux52a9ux6750ux6599ux30aaux30fcux30c9ux30d6ux30ebux7528}}

\vspace*{-1.5\zw}

\hypertarget{beurres-composuxe9s-pour-adjuvants-de-sauces-et-hors-doeuvre}{%
\subsection{Beurres Composés pour Adjuvants de Sauces et
Hors-d'oeuvre}\label{beurres-composuxe9s-pour-adjuvants-de-sauces-et-hors-doeuvre}}

\index{あわせはたー@合わせバター} \index{はたー@バター ⇒ 合わせバター}
\index{ふーるこんほせ@ブール・コンポゼ ⇒ 合わせバター}
\index{みつくすはたー@ミックスバター ⇒ 合わせバター}
\index{beurre@beurre!beurres composes@Beurres Composés}

\hypertarget{observation-sur-les-beurres-composes}{%
\subsection{概説}\label{observation-sur-les-beurres-composes}}

本書においてレシピを掲載している合わせバター\footnote{beurre composé
  ブール・コンポゼ。ミックスバターとも。少なくとも
  バターは中世以来用長く用いられてきた食材だが、中世〜ルネサンスにお
  いては獣脂(もっぱらラード)のほうが多く用いられる傾向にあった。17
  世紀以降はたとえばラ・ヴァレーヌ『フランス料理の本』におけるアスパ
  ラガスの白いソース添え(\protect\hyperlink{sauce-hollandaise}{ソース・オランデーズ}
  訳注参照)のように、バターを料理に用いることが中世の料理書と比較す
  ると圧倒的に増えたのは事実である。ムノンの1741年刊『ブルジョワ屋敷
  に勤める女性料理人のための本』のバターの項には「良質のバターを用い
  ることは料理でとても重要なことであり、バターが匂いを放っているよう
  ではどんな素晴しい皿も台無しだ。料理担当の女中であればこのことをよ
  く理解しておくことと、良質なバターの価格を手に入れるのに金を惜しん
  ではならないことを肝に銘じておくこと。最良のバターは自然な黄色をし
  ており、白いものは大抵の場合、さして美味しくない。バルボットという
  植物から採った黄色で着色されたバターもある。こういうバターの色は、
  自然なバターの黄色よりもくすんだもので、慣れれば簡単に見分けること
  が出来る(p.320)」}のうちのほとんどは、甲
殻類の合わせバターを除いて、料理に直接用いられることがとても少ない。だ
が、合わせバターはさまざまなシチュエーションで役に立つ。ポタージュでは
野菜の合わせバターが、その他の合わせバターはソース作りにおいて有用だ。
ソースの風味と性格を明確に伝える決め手になるからだ。

だから、読者である料理人諸君には、ここに書いてあることを真剣に読みとっ
ていただきたい。\href{原文における内容矛盾。この後のパラグラフは甲殻類の\%20バターについての注意点ばかりが目立つ}{}

甲殻類のバターについては、経験上、湯煎にかけながら煮出して\footnote{infuser
  アンフュゼ。}から、氷
水で冷やした陶製の容器に布で漉し入れるといい。そうすれば、冷たい状態で
作るよりも赤みがきれいに出る。だが逆に、熱によって風味の繊細さが失なわ
れてしまい、雑味さえも出てしまう。

この問題点を解決するために、我々は二種類の違うバターを作るという方式を
採ることにした。ひとつは甲殻類の胴のクリーム状の部分と切りくずあるいは
身そのものを生のバターとともに鉢ですり潰して、目の細かい網で裏漉しする
か、布で漉すというもの。このバターはソースに完璧ともいうべき風味を添え
てくれる。とりわけベシャメルソースをベースとしたソースの場合はそうだ。

もうひとつは、甲殻類の殻だけを用いて、熱して作るものだ。これは「色付け」
の役割しか持たない。この方式はまことに素晴しい結果を得られるので、ぜひ
とも実行していただきたい。

場合によっては、我々はバターを同様の上等な生クリームに代えることがある。
生クリームのほうがバターよりも、素材の持つ風味や香気をよく吸収する。こ
うすればソースやポタージュの仕上げに加えるのに文句ないクリ\footnote{Coulis
  水分のやや多いピュレをイメージするといい。}を作ることが 出来るわけだ。

色付け用のバターを使うと、ソースがきれいに色付き、個性的なソースとなる。
どんな場合でも、カルミン色素\footnote{コチニール色素ともいう。ラックカイガラムシなどを原料として抽出し
  た色素。ヨーロッパでは古代から中世にかけてケルメスカイガラムシから
  抽出され利用されてきた、非常に歴史の古い色素。とりわけルネサン期に
  は高級毛織物の染料として需要が高まった。また絵の具にも使用された。
  その後、ウチワサボテンでエンジムシを大量に養殖していた中南米を支配
  下に置いたスペインが、これを新大陸産のカルミンとしてヨーロッパ各国
  に売ることで巨万の富を得たという。かつて食品工業において多用された。
  1838年の『ラルース・ガストロノミック』初版では、「コチニールから抽
  出される鮮かな赤色色素で毒性はない。多くの食品に着色料として用いら
  れている」とある。現在は食物アレルギーの原因物質すなわちアレルゲン
  となり得ることがわかり、使用は減りつつある。現在は代替品としてビー
  ツから抽出したビートレッドなどの使用が増えてきている。また、この本
  文でカルミン色素の使用を「くすんだ、情けない色合いを与える」として
  否定的に扱っているのは、この色素がpHによって色調が変化し、なおかつ
  蛋白質を多く含む料理に加えると紫色に変化する(ソースやポタージュ全
  体が濁ったような色になる)ことがあるためだろう。}よりもずっといい。カルミン色素はソース
やポタージュにくすんだ、なさけない色合いしか与えてはくれないのだ。

合わせバターは一般的に、使う際にその都度作る\footnote{原文 au moment
  (オモモン)その都度、の意。à la minute (アラミニュッ
  ト)と呼ぶ調理現場もある。}ものだが、作り置き
しておかなければならない場合は、白い紙で円筒形に包んで冷蔵保管すること。

\vspace*{1.7\zw}
\begin{recette}
\hypertarget{ux306bux3093ux306bux304fux30d0ux30bfux30fc}{%
\subsubsection{にんにくバター}\label{ux306bux3093ux306bux304fux30d0ux30bfux30fc}}

\hypertarget{beurre-d-ail}{%
\paragraph{Beurre d'Ail}\label{beurre-d-ail}}

\index{はたー@バター!あわせはたー@合わせバター!にんにくはたー@にんにくバター}
\index{あわせはたー@合わせバター!にんにくはたー@にんにくバター}
\index{にんにく@にんにく!はたー@---バター}
\index{beurre@beurre!beurres composes@Beurres Composés!beurre d'ail@Beurre d'Ail}
\index{ail@ail!beurre@Beurre d'---}

皮を剥いたにんにく200 gを強火でしっかり茹でる\footnote{生のにんにくには胃腸を刺激する酵素が含まれているが、熱により不活
  性化するので、よく火を通す必要がある。}。よく湯をきってから、鉢
に入れてすり潰し、バター250 gと合わせ、布で漉す。

\maeaki

\hypertarget{ux30a2ux30f3ux30c1ux30e7ux30d3ux30d0ux30bfux30fc}{%
\subsubsection{アンチョビバター}\label{ux30a2ux30f3ux30c1ux30e7ux30d3ux30d0ux30bfux30fc}}

\hypertarget{beurre-d-anchois}{%
\paragraph{Beurre d'Anchois}\label{beurre-d-anchois}}

\index{はたー@バター!あわせはたー@合わせバター!あんちよひはたー@アンチョビバター}
\index{あわせはたー@合わせバター!あんちよひはたー@アンチョビバター}
\index{あんちよひ@アンチョビ!はたー@---バター}
\index{beurre@beurre!beurres composes@Beurres Composés!beurre d'anchois@Beurre d'Anchois}
\index{anchois@anchois!beurre@Beurre d'---}

アンチョビのフィレ200 gをよく洗い、しっかり水気を絞る。これを鉢に入れ
て細かくすり潰す。バター250 gを加えて布で漉す。

\maeaki

\hypertarget{ux30a2ux30fcux30e2ux30f3ux30c9ux30d0ux30bfux30fc}{%
\subsubsection{アーモンドバター}\label{ux30a2ux30fcux30e2ux30f3ux30c9ux30d0ux30bfux30fc}}

\hypertarget{beurre-d-amande}{%
\paragraph{Beurre d'Amande}\label{beurre-d-amande}}

\index{はたー@バター!あわせはたー@合わせバター!あーもんとはたー@アーモンドバター}
\index{あわせはたー@合わせバター!あーもんとはたー@アーモンドバター}
\index{あーもんと@アーモンド!はたー@---バター}
\index{beurre@beurre!beurres composes@Beurres Composés!beurre d'amande@Beurre d'Amande}
\index{amande@amande!beurre@Beurre d'---}

アーモンド\footnote{アーモンドには一般的なスイートアーモンド amandes
  doucesと、苦味 のあるビターアーモンドamande
  amèresの二種がある。後者はあまり多く
  使われることはないが、香りがいいためリキュールなどの香り付けにごく
  少量が用いられることがある。}150
gを湯むきしてよく洗い、すぐに水数滴を加えてすり潰し
てペースト状にする。これをバター250 gと混ぜ合わせ、布で漉す。

\maeaki

\hypertarget{ux30d6ux30fcux30ebux30c0ux30f4ux30eaux30fcux30cc8}{%
\subsubsection[ブール・ダヴリーヌ]{\texorpdfstring{ブール・ダヴリーヌ\footnote{アヴリーヌはヘーゼルナッツの仲間でセイヨウハシバミの大粒な変種。
  イタリア、ピエモンテ産やシチリア産が有名。}}{ブール・ダヴリーヌ}}\label{ux30d6ux30fcux30ebux30c0ux30f4ux30eaux30fcux30cc8}}

\hypertarget{beurre-d-aveline}{%
\paragraph{Beurre d'Aveline}\label{beurre-d-aveline}}

\index{はたー@バター!あわせはたー@合わせバター!ふーるたうりーぬ@ブール・ダヴリーヌ}
\index{あわせはたー@合わせバター!ふーるたうりーぬ@ブール・ダヴリーヌ}
\index{あうりーぬ@アヴリーヌ!ふーる@ブール・---}
\index{へーせるなつつ@ヘーゼルナッツ!ふーるたうりーぬ@ブール・ダヴリーヌ}
\index{beurre@beurre!beurres composes@Beurres Composés!beurre d'aveline@Beurre d'Aveline}
\index{aveline@aveline!beurre@Beurre d'---}

アヴリーヌ150 gを焙煎して丁寧に皮を剥く。油が浮いてこないよう水を数滴
加えてペースト状にすり潰す。これとバター250 gを混ぜ合わせる。目の細か
い網で裏漉しするか、布で漉す。

\maeaki

\hypertarget{ux30d6ux30fcux30ebux30d9ux30ebux30b7ux30fc9}{%
\subsubsection[ブール・ベルシー]{\texorpdfstring{ブール・ベルシー\footnote{\protect\hyperlink{sauce-bercy}{ソース・ベルシー}訳注参照。}}{ブール・ベルシー}}\label{ux30d6ux30fcux30ebux30d9ux30ebux30b7ux30fc9}}

\hypertarget{beurre-bercy}{%
\paragraph{Beurre Bercy}\label{beurre-bercy}}

\index{はたー@バター!あわせはたー@合わせバター!ふーるへるしー@ブール・ベルシー}
\index{あわせはたー@合わせバター!ふーるたへるしー@ブール・ベルシー}
\index{へるしー@ベルシー!ふーる@ブール・---}
\index{beurre@beurre!beurres composes@Beurres Composés!beurre bercy@Beurre Bercy}
\index{bercy@Bercy!beurre@Beurre ---}

白ワイン2 dlに細かく刻んだエシャロット大さじ1杯を加えて半量になるまで
煮詰める。生温い程度まで冷ましてから、ポマード状に柔らかくしたバター 200
gを混ぜ込む。牛骨髄500 gをさいの目に切って\footnote{原文 couper en
  dés。フランス語のまま「デにする(切る)」と表現することもある。}、沸騰しない程度の
湯で火を通し、よく湯ぎりをして加える。パセリのみじん切り大さじ1杯と塩8
g、挽きたてのこしょう1つまみ強とレモン\undemi{}個分の果汁を加えて仕上
げる。

\maeaki

\hypertarget{ux30adux30e3ux30d3ux30a2ux30d0ux30bfux30fc}{%
\subsubsection{キャビアバター}\label{ux30adux30e3ux30d3ux30a2ux30d0ux30bfux30fc}}

\hypertarget{beurre-de-caviar}{%
\paragraph{Beurre de Caviar}\label{beurre-de-caviar}}

\index{はたー@バター!あわせはたー@合わせバター!きやひあはたー@キャビアバター}
\index{あわせはたー@合わせバター!きやひあはたー@キャビアバター}
\index{きやひあ@キャビア!はたー@---バター}
\index{beurre@beurre!beurres composes@Beurres Composés!beurre caviar@Beurre de Caviar}
\index{caviar@caviar!beurre@Beurre de ---}

圧縮キャビア\footnote{もとはロシアで雪の中の樽で保存するために圧縮したもの。キャビア
  のグレードはベルガ、オセトラ、セヴルガが混ざっているのが多いという。}75
gを細かくすり潰す。パター250 gを加えて、布で漉す。

\maeaki

\hypertarget{ux30d6ux30fcux30ebux30b7ux30f4ux30ea12-ux30d6ux30fcux30ebux30e9ux30f4ux30a3ux30b4ux30c3ux30c813}{%
\subsubsection[ブール・シヴリ /
ブール・ラヴィゴット]{\texorpdfstring{ブール・シヴリ\footnote{\protect\hyperlink{sacue-chivry}{ソース・シヴリ}訳注参照。}
/ ブール・ラヴィゴット\footnote{\protect\hyperlink{sauce-ravigote}{ソース・ラヴィゴット}訳注参照。}}{ブール・シヴリ / ブール・ラヴィゴット}}\label{ux30d6ux30fcux30ebux30b7ux30f4ux30ea12-ux30d6ux30fcux30ebux30e9ux30f4ux30a3ux30b4ux30c3ux30c813}}

\hypertarget{beurre-chivry}{%
\paragraph{Beurre Chivry}\label{beurre-chivry}}

\index{はたー@バター!あわせはたー@合わせバター!しうり@ブール・シヴリ}
\index{あわせはたー@合わせバター!しうり@ブール・シヴリ}
\index{はたー@バター!あわせはたー@合わせバター!らういこつと@ブール・ラヴィゴット}
\index{あわせはたー@合わせバター!らういこつと@ブール・ラヴィゴット}
\index{しうり@シヴリ!ふーる@ブール・---}
\index{らういこつと@ラヴィゴット!ふーる@ブール・---}
\index{beurre@beurre!beurres composes@Beurres Composés!beurre chivry@Beurre Chivrya}
\index{beurre@beurre!beurres composes@Beurres Composés!beurre ravigote@Beurre Ravigote}
\index{chivry@Chivry!beurre@Beurre ---}
\index{ravigote@ravitote!beurre@Beurre ---}

パセリの葉とセルフイユ、エストラゴン、シヴレット、若摘みのサラダバーネッ
ト100 gを数分間下茹でし、水にさらしてから圧して余分な水気を絞る。エシャ
ロットのみじん切り25 gも下茹でする。これらを鉢に入れてすり潰す。

バター125 gを加え、布で漉す。

\maeaki

\hypertarget{ux30d6ux30fcux30ebux30b3ux30ebux30d9ux30fcux30eb14}{%
\subsubsection[ブール・コルベール]{\texorpdfstring{ブール・コルベール\footnote{\protect\hyperlink{sauce-colbert}{ソース・コルベール}本文および訳注参照。}}{ブール・コルベール}}\label{ux30d6ux30fcux30ebux30b3ux30ebux30d9ux30fcux30eb14}}

\hypertarget{beurre-colbert}{%
\paragraph{Beurre Colbert}\label{beurre-colbert}}

\index{はたー@バター!あわせはたー@合わせバター!ふーるこるへーる@ブール・コルベール}
\index{あわせはたー@合わせバター!ふーるこるへーる@ブール・コルベール}
\index{こるへーる@コルベール!ふーる@ブール・---}
\index{beurre@beurre!beurres composes@Beurres Composés!beurre colbert@Beurre Colbert}
\index{colbert@Colbert!beurre@Beurre ---}

\protect\hyperlink{beurre-maitre-d-hotel}{メートルドテルバター}200lgに、溶かした\protect\hyperlink{glace-de-viande}{グラス
ドヴィアンド}大さじ2杯と細かく刻んだエストラゴン小さ じ2杯を加える。

\maeaki

\hypertarget{ux8272ux4ed8ux3051ux7528ux306eux8d64ux3044ux30d0ux30bfux30fc}{%
\subsubsection{色付け用の赤いバター}\label{ux8272ux4ed8ux3051ux7528ux306eux8d64ux3044ux30d0ux30bfux30fc}}

\hypertarget{beurre-colorant-rouge}{%
\paragraph{Beurre Colorant rouge}\label{beurre-colorant-rouge}}

\index{はたー@バター!あわせはたー@合わせバター!いろつけようのあかいはたー@色付け用の赤いバター}
\index{あわせはたー@合わせバター!いろつけようのあかいはたー@色付け用の赤いバター}
\index{ちやくしよくそざい@着色素材!いろつけようのあかいはたー@色付け用の赤いバター}
\index{beurre@beurre!beurres composes@Beurres Composés!beurre colorant rouge@Beurre Colorant rouge}
\index{colorant@colorant!beurre rouge@Beurre --- rouge}

出来るだけ沢山の甲殻類の殻などの残りをまとめて用意する。殻の内側、外側
に張り付いている膜などをきれいに取り除く。よく乾燥させてから、鉢\footnote{伝統的には大理石製の鉢が用いられることが多かった。}
に入れて細かく粉砕して、同じ重さのバターを加える。これを湯煎にかけてよ
く混ぜながら溶かす。氷水を入れた陶製の器に、布で漉し入れる。固まったバ
ターをトーション\footnote{\protect\hyperlink{sauce-verte}{ソース・ヴェルト}訳注参照。}で包み、余計な水を絞り出す。

\hypertarget{ux539fux6ce8}{%
\subparagraph{【原注】}\label{ux539fux6ce8}}

この色付け用のバターを作るのに用いる甲殻類の殻がどうしてもない場合は、
\protect\hyperlink{beurre-de-paprika}{パプリカバター}を用いてもいいだろう。だがいずれに
せよ、どんなソースであっても、仕上りの色合いを決めるには、出来るだけ、
他の植物由来の赤色着色料の使用は避けることを勧める\footnote{この原注は第三版から。原文le
  rouge colorant végétal直訳すると
  「植物由来の赤色着色料」だが。ここではおそらくカルミン色素(コチニー
  ル色素)のことと思われる(\protect\hyperlink{observation-sur-les-beurres-composes}{本節「概説」参
  照})。他に赤系着色料として、
  ベニバナ色素、紅麹などもあるが、いずれも中国や日本において発達しこ
  とを考慮すると、両大戦間である1920年頃に「避けるべき」というほど普
  及していたのは、実際には昆虫由来であるコチニール色素と思われる。な
  お、ベニバナ色素も化学的にはカルミン酸色素。また、甲殻類の殻を茹で
  ると赤くなるが、この色素はアスタキサンチンといい、1938年に物質とし
  て「発見」された。もちろんエスコフィエをはじめとする料理人は経験上、
  甲殻類の殻を適度に加熱することで、タンパク質と結びついていたアスタ
  キサンチンがタンパク質の熱変性によって遊離して取り出せることを経験
  的によく知っており、それを利用してこの赤いバターを考案したと考えら
  れる。ちなみにサーモン、鮭の身の赤色もおなじアスタキサンチンによる
  もので、近縁種の鱒と同様に本来は白身。}。

\maeaki

\hypertarget{ux8272ux4ed8ux3051ux7528ux306eux7dd1ux306eux30d0ux30bfux30fc}{%
\subsubsection{色付け用の緑のバター}\label{ux8272ux4ed8ux3051ux7528ux306eux7dd1ux306eux30d0ux30bfux30fc}}

\hypertarget{beurre-colorant-vert}{%
\paragraph{Beurre Colorant vert}\label{beurre-colorant-vert}}

\index{はたー@バター!あわせはたー@合わせバター!いろつけようのみとりのはたー@色付け用の緑のバター}
\index{あわせはたー@合わせバター!いろつけようのみとりのはたー@色付け用の緑のバター}
\index{ちやくしよくそざい@着色素材!いろつけようのみとりのはたー@色付け用の緑のバター}
\index{beurre@beurre!beurres composes@Beurres Composés!beurre colorant vert@Beurre Colorant vert}
\index{colorant@colorant!beurre vert@Beurre --- vert}

ほうれんそうの葉1
kgをよく洗い、しっかり振って水気をきる。これを鉢に入れてすり潰す。トーション\footnote{\protect\hyperlink{sauce-verte}{ソース・ヴェルト}訳注参照。}で包んで緑の汁を絞り出す。これをソテー鍋に入れて湯煎にかけ、水分を蒸発させてペースト状にする\footnote{原文
  coaguler 凝固させる、の意。ここでは説明的に意訳した。なお、
  ほうれんそうに限らず、植物の緑色は葉緑素(クロロフィル)によるもの
  であり、葉緑素はマグネシウム(苦土)を核として窒素が周囲に結びつい
  た構造を持つ化学物質。ほうれんそうの緑が濃いのは土壌からのマグネシ
  ウム吸収能力が高いため。食品に含まれるマグネシウムはカルシウムの吸
  収を促す作用があり健康にいいとされている。}。

これを、ぴんと張ったナフキンの上に移し、さらに水気をきる。

パレットナイフを使って緑の色素を集め、鉢に入れてその倍の重さのバターを加えて練り込む。

\ldots{}\ldots{}布で漉し、冷蔵保存する。

\hypertarget{ux539fux6ce8-1}{%
\subparagraph{【原注】}\label{ux539fux6ce8-1}}

人工的な色素よりもこの緑の色素を用いたようが利点が大きい。

\maeaki

\hypertarget{ux30afux30ebux30f4ux30a7ux30c3ux30c8ux30d0ux30bfux30fc}{%
\subsubsection{クルヴェットバター}\label{ux30afux30ebux30f4ux30a7ux30c3ux30c8ux30d0ux30bfux30fc}}

\hypertarget{beurre-de-crevettes}{%
\paragraph{Beurre de crevettes}\label{beurre-de-crevettes}}

\index{はたー@バター!あわせはたー@合わせバター!くるうえつとはたー@クルヴェットバター}
\index{あわせはたー@合わせバター!くるうえつとはたー@クルヴェットバター}
\index{クルウエツト@クルヴェット!はたー@---バター}
\index{beurre@beurre!beurres composes@Beurres Composés!beurre de crevette@Beurre de Crevette}
\index{crevette@crevette!beurre@Beurre d'---}

クルヴェッット・グリーズ\footnote{フランスで好んで食される小海老の一種。\protect\hyperlink{sauce-aux-crevettes}{ソース・クルヴェット}訳注参照。}150
gを鉢に入れて細かくすり潰す。バター150 gを加えて、布で漉す。
\end{recette}