\hypertarget{ux51b7ux88fdux30bdux30fcux30b9}{%
\section{冷製ソース}\label{ux51b7ux88fdux30bdux30fcux30b9}}

\hypertarget{sauces-froides}{%
\subsection{Sauces Froides}\label{sauces-froides}}

\index{sauce@sauce!sauces froides@sauces froides}
\index{そーす@ソース!れいせいそーす@冷製ソース}
\begin{recette}
\hypertarget{ux30a2ux30a4ux30e8ux30ea2-ux30d7ux30edux30f4ux30a1ux30f3ux30b9ux30d0ux30bfux30fc}{%
\subsubsection[アイヨリ /
プロヴァンスバター]{\texorpdfstring{アイヨリ\footnote{ailloliとも綴るが、
  ail(にんにく)+
  oil(油)の合成語。19世前半紀には既にアカデミーフランセージの辞書に収録されており、広く知られていたようだ。ブイヤベースに添えるルイユとよく似ているが、ルイユがカイエンヌを加えるのに対して、こちらはにんにくと油、塩、レモン汁と少々の水だけで作る。用途も、茹でた塩鱈やじゃがいも、茹で卵、アーティチョーク、さやいんげん、などに合わせることが多い。}
/
プロヴァンスバター}{アイヨリ / プロヴァンスバター}}\label{ux30a2ux30a4ux30e8ux30ea2-ux30d7ux30edux30f4ux30a1ux30f3ux30b9ux30d0ux30bfux30fc}}

\hypertarget{sauce-aioli}{%
\paragraph{Sauce Aïoli, ou Beurre de Provence}\label{sauce-aioli}}

\index{そーす@ソース!れいせい@冷製---!あいより@アイヨリ}
\index{そーす@ソース!れいせい@冷製---!ふろふあんすはたー@プロヴァンスバター}
\index{あいより@アイヨリ}
\index{ふろふあんす@プロヴァンス!ふろふあんすはたー@プロヴァンスバター}
\index{はたー@バター!ふろふあんすはたー@プロヴァンスバター}
\index{sauce@sauce!sauce froide@sauce froide!aioli@--- Aïoli}
\index{sauce@sauce!sauce froide@sauce froide!beurre de provence@Beurre de Provence}
\index{aioli@Aïoli!sauce@Sauce ---}
\index{provence@Provence!Beurre de Provence (Aïoli)}
\index{beurre@beurre!beurre de provence@Beurre de Provence (Aïoli)}

にんにく4片(30 g)を鉢\footnote{この種の作業には、大理石製のものが伝統的によく用いられる。。}に入れて細かくすり潰す。ここに生の卵黄1個、塩1つまみを加える。混ぜながら、2\undemi{}
dlの油\footnote{原書ではとくに言及されていないが、プロヴァンス地方ではオリーブオイルを用いることが一般的。}を初めは1滴ずつ加えていき、ソースがまとまりはじめたら糸を垂らすようにして加える。この作業は鉢に入れたままで、棒をはげしく動かして行なう。

攪拌する作業の途中、レモン1個分の搾り汁と冷水大さじ\undemi{}杯を少しずつ加えて、ソースが固くなり過ぎないようにしてやること。

\hypertarget{ux539fux6ce8}{%
\subparagraph{【原注】}\label{ux539fux6ce8}}

このアイヨリソースが分離してしまいそうな時は、卵黄をさらに1個足して、
マヨネーズと場合と同様に修正すること。

\maeaki

\hypertarget{ux30a2ux30f3ux30c0ux30ebux30b7ux30a25ux98a8ux30bdux30fcux30b9}{%
\subsubsection[アンダルシア風ソース]{\texorpdfstring{アンダルシア\footnote{いうまでもなくスペインのアンダルシア地方のことだが、トマトやオリーブオイル、チョリソなどこの地方を「想起」させる食材が使われている料理などがこの名称になっている傾向がある。ところが、トマトにしろオリーブオイルにしろアンダルシア地方特有というわけではなく、アンダルシアが産地として有名なチョリソくらいしか、料理名の根拠となり得るものはない。逆に言えば、アンダルシア地方の食文化との関係は、そこに用いられている食材以外にはないものと考えてもいい。料理名に付けられた地方名がとりたてて根拠や由来のないものであることを示す一例。}風ソース}{アンダルシア風ソース}}\label{ux30a2ux30f3ux30c0ux30ebux30b7ux30a25ux98a8ux30bdux30fcux30b9}}

\hypertarget{sauce-andalouse}{%
\paragraph{Sauce Andalouse}\label{sauce-andalouse}}

\index{そーす@ソース!れいせい@冷製---!あんたるしあふう@アンダルシア風---}
\index{あんたるしあ@アンダルシア!そーす@---風ソース}
\index{そーす@ソース!あんたるしあふう@アンダルシア風---}
\index{sauce@sauce!sauce froide@sauce froide!Andalouse@--- Andalouse}
\index{sauce@sauce!andalouse@--- Andalouse}
\index{andalous@Andalous(e)!sauce@Sauce Andalouse}

ごく固く仕上げた\protect\hyperlink{mayonnaise}{ソース・マヨネーズ}\troisquarts{}
Lに、上等な赤いトマトピュレ2\undemi{}dlを加える。小さなさいの目に切ったポワヴロン\footnote{Poivron
  いわゆる日本で青果として輸入されているパプリカ(肉厚の辛くないピーマン)とほぼ同じものだが、香辛料として用いられる粉末のパプリカと混同を避けるため、あえてフランス語をそのままカタカナに訳した。}75
gを仕上げに加える。

\maeaki

\hypertarget{ux30bdux30fcux30b9ux30dcux30d8ux30dfux30a2ux306eux5a18}{%
\subsubsection{ソース・ボヘミアの娘}\label{ux30bdux30fcux30b9ux30dcux30d8ux30dfux30a2ux306eux5a18}}

\hypertarget{sauce-bohemienne}{%
\paragraph[Sauce Bohémienne]{\texorpdfstring{Sauce Bohémienne\footnote{アイルランド出身の作曲家マイケル・ウィリアム・バルフェMichael
  William Balfe (1808〜1870)のオペラ\emph{The Bohemien
  Girl}『ボヘミアの少女』のフランス語版タイトル\href{https://archive.org/details/labohmiennegrand00balf}{\emph{La
  Bohémienne}}『ラボエミエーヌ』にちなんだものと言われている。この作品はロンドンで1843年初演、1862年に四幕形式のフランス語版がパリのオペラ=コミック劇場で上演され、大ヒットしたという。この名を冠した料理はいくつかあるが、いずれもチェコのボヘミア地方とは何の関連性も認められないため、オペラの人気作品にあやかった料理名と考えるのが妥当だろう。}}{Sauce Bohémienne}}\label{sauce-bohemienne}}

\index{そーす@ソース!れいせい@冷製---!ほへみあのむすめ@---ボヘミアの娘}
\index{ほへみあ@ボヘミア!そーす@ソース・---の娘}
\index{そーす@ソース!ほへみあ@---・ボヘミアの娘}
\index{sauce@sauce!sauce froide@sauce froide!bohemienne@--- Bohémienne}
\index{sauce@sauce!bohemienne@--- Bohémienne}
\index{bohemien@bohémien(ne)!sauce@Sauce Bohémienne}

陶製の容器に、濃厚でよく冷やした\protect\hyperlink{sauce-bechamel}{ベシャメルソース}1\undemi{}
dlと卵黄4個、塩10 g、こしょう少々、ヴィネガー数滴を入れる。

泡立て器で全体をよく混ぜ、標準的なマヨネーズを作るのとまったく同じ要領で、油1
Lとエストラゴンヴィネガー大さじ2杯程を加える。

\ldots{}\ldots{}仕上げに、マスタード大さじ1杯を加える。

\maeaki

\hypertarget{ux30bdux30fcux30b9ux30b7ux30e3ux30f3ux30c6ux30a3ux30a47}{%
\subsubsection[ソース・シャンティイ]{\texorpdfstring{ソース・シャンティイ\footnote{パリ近郊の地名。詳しくはホワイト系派生ソースの\protect\hyperlink{sauce-chantilly}{ソース・シャンティイ}訳注参照。}}{ソース・シャンティイ}}\label{ux30bdux30fcux30b9ux30b7ux30e3ux30f3ux30c6ux30a3ux30a47}}

\hypertarget{sauce-chantilly-froide}{%
\paragraph{Sauce Chantilly}\label{sauce-chantilly-froide}}

\index{そーす@ソース!れいせい@冷製---!しやんていい@---・シャンティイ}
\index{しやんていい@シャンティイ!そーす@ソース・---(冷製)}
\index{そーす@ソース!しやんていい@---・シャンティイ}
\index{sauce@sauce!sauce froide@sauce froide!chantilly@--- Chantilly}
\index{sauce@sauce!chantilly@--- Chantilly (froide)}
\index{chantilly@Chantilly!sauce@Sauce --- (froide)}

酸味付けにレモンを用いて、固く仕上げた\protect\hyperlink{mayonnaise}{ソース・マヨネーズ}\troisquarts{}
Lを用意しておく。提供直前に、ごく固く泡立てた生クリーム大さじ4杯\footnote{大さじ1杯=15ccという概念にとらわれないよう注意。原文は、大きなスプーンで泡立てた生クリームをざっくりと4回加えるイメージで書かれている。本書における通常のソースの仕上り量が約1
  Lであることを考慮すると、最低でも100ml以上は加えることになるだろう。}を加える。その後、味を\ruby{調}{ととの}える。

\ldots{}\ldots{}もっぱら、アスパラガスの冷製、温製に添える。

\hypertarget{ux539fux6ce8-1}{%
\subparagraph{【原注】}\label{ux539fux6ce8-1}}

生クリームを加えるのは、このソースを使うまさにその時にすること。前もっ
て加えておくと、ソースが分離してしまう恐れがあるので注意。

\maeaki

\hypertarget{ux30b8ux30a7ux30ceux30f4ux30a1ux98a812ux30bdux30fcux30b9}{%
\subsubsection[ジェノヴァ風ソース]{\texorpdfstring{ジェノヴァ風\footnote{あまり明確な由来はないが、ジェノヴァが地中海に面した港町であり、このソースが魚料理用であるという点で一応の説明はつくだろう。}ソース}{ジェノヴァ風ソース}}\label{ux30b8ux30a7ux30ceux30f4ux30a1ux98a812ux30bdux30fcux30b9}}

\hypertarget{sauce-genoise-froids}{%
\paragraph{Sauce Génoise}\label{sauce-genoise-froids}}

\index{そーす@ソース!れいせい@冷製---!しえのうあふう@ジェノヴァ風---}
\index{しえのうあふう@ジェノヴァ風!そーす@ソース・---(冷製)}
\index{そーす@ソース!しえのうあふう@ジェノヴァ風---}
\index{sauce@sauce!sauce froide@sauce froide!genoise@--- Génoise}
\index{sauce@sauce!genoise@--- Génoise (froide)}
\index{genois@Génois(e)!sauce@Sauce ---e (froide)}

殻と皮を剥いたばかりのピスタチオ40 gと、松の実25
g、松の実がない場合はスイートアーモンド20
gを鉢に入れてよくすり潰し、冷めた\protect\hyperlink{sauce-bechamel}{ベシャメルソース}小さじ1杯程度を加えて練ってペースト状にする。これを目の細かい網で裏漉しする。陶製の容器に卵黄6個、塩1つまみ、こしょう少々を入れる。泡立て器でよく混ぜる。油1
Lと中位の大きさのレモン2個の搾り汁を少しずつ加えてよく混ぜて乳化させていく\footnote{明記されていないが、ソースをしっかりと乳化させるためには\protect\hyperlink{mayonnaise}{マヨネーズ}と同様に作業すること。}。仕上げにハーブのピュレ大さじ3杯を加える。これは、パセリの葉とセルフイユ、エストラゴン、時季が合えばサラダバーネットを同量ずつ用意し、強火で2分間下茹でしてから湯をきり、冷水にさらしてから水気を強く絞り、裏漉しして作っておく。

\ldots{}\ldots{}冷製の魚料理全般に合わせられる。

\maeaki

\hypertarget{ux30bdux30fcux30b9ux30b0ux30eaux30d3ux30c3ux30b7ux30e5}{%
\subsubsection{ソース・グリビッシュ}\label{ux30bdux30fcux30b9ux30b0ux30eaux30d3ux30c3ux30b7ux30e5}}

\hypertarget{sauce-gribiche13}{%
\paragraph[Sauce Gribiche]{\texorpdfstring{Sauce Gribiche\footnote{由来不明の語。ノルマンディ方言で「子どもを怖がらせるおばさん」
  の意味で用いられるということが分かっているのみ。19世紀後半以降に創
  案もしくは一般化したソースと思われる。本書初版には当然のように既に
  収録されており、その後の大きな異同もない。ただ、本書初版以前に出版
  された料理書においてこのソースのレシピはまだ見つかっていない。ファー
  ヴルは1905年刊『料理および食品衛生事典』第二版で「ある種のレムラー
  ドにレストランで付けられた名称」と定義し、掲載しているレシピは本書
  初版のものと大差ないが、「ウスターシャソース少々も加える」となって
  いるところが目を引く。また、1913年初版のプルーストの長編小説『失な
  われた時を求めて』の「スワン家の方へ」冒頭において「彼(=スワン)を
  招いていない夕食会のために、ソース・グリビッシュやパイナップルのサ
  ラダのレシピが必要になるや、ためらいもなく探しに行かせたりするのだっ
  た」(p.18)。もしこの語り手の記述が正確であるなら、19世紀末には広く
  知られたものであったと考えるべきだが、小説の場合は必ずしも歴史的事
  実と符号するわけではないので注意が必要。}}{Sauce Gribiche}}\label{sauce-gribiche13}}

\index{そーす@ソース!れいせい@冷製---!くりひつしゆ@---・グリビッシュ}
\index{くりひつしゆ@グリビッシュ!そーす@ソース・---(冷製)}
\index{そーす@ソース!くりひつしゆ@---・グリビッシュ}
\index{sauce@sauce!sauce froide@sauce froide!gribiche@--- Gribiche}
\index{sauce@sauce!gribiche@--- Gribiche (froide)}
\index{gribiche@gribiche!sauce@Sauce --- (froide)}

茹であがったばかりの固茹で卵の黄身6個を陶製のボウルに入れ、マスタード
小さじ1杯、塩1つまみ強、こしょう適量を加えてよく練り、滑らかなペースト
状にする。植物油\undemi{} Lとヴィネガー大さじ1\undemi{}杯を加えながら
よく混ぜて乳化させる。仕上げに、コルニションとケイパーのみじん切り計 100
gと、パセリとセルフイユ、エストラゴンのみじん切りのミックスを大さ
じ1杯、短かめの千切りにした固茹で卵の白身3個分を加える。

\ldots{}\ldots{}冷製の魚料理に添えるのが一般的。

\maeaki

\hypertarget{ux30ecux30d5ux30a9ux30fcux30ebux98a8ux5473ux306eux30bdux30fcux30b9ux30b0ux30edux30bcux30a4ux30e6}{%
\subsubsection{レフォール風味のソース・グロゼイユ}\label{ux30ecux30d5ux30a9ux30fcux30ebux98a8ux5473ux306eux30bdux30fcux30b9ux30b0ux30edux30bcux30a4ux30e6}}

\hypertarget{sauce-groseilles-au-raifort}{%
\paragraph{Sauce Groseilles au
Raifort}\label{sauce-groseilles-au-raifort}}

\index{そーす@ソース!れいせい@冷製---!れふおーるふうみくろせいゆ@レフォール風味の---・グロゼイユ}
\index{くろせいゆ@グロゼイユ!そーすれふおーる@レフォール風味のソース・---(冷製)}
\index{れふおーる@レフォール!そーすくろせいゆ@---風味のソース・グロゼイユ}
\index{そーす@ソース!れふおーるふうみくろせいゆ@レフォール風味の---・グロゼイユ}
\index{sauce@sauce!sauce froide@sauce froide!grroseilles raifort@--- Grroseilles au Rifort}
\index{sauce@sauce!groseille@--- Groseilles au Raifort (froide)}
\index{raiforg@raifort!sauce@Sauce Groseilles au --- (froide)}
\index{groseille@groseille!sauce@Sauce --- au Raifort (froide)}

ポルト酒1
dlにナツメグ、シナモン、塩、こしょう各1つまみを加え、を\deuxtiers{}量まで煮詰める。溶かした\protect\hyperlink{}{グロゼイユのジュレ}4
dlと細かくすりおろしたレフォール大さじ2杯を加える。

(さまざまな用途に使える)

\maeaki

\hypertarget{ux30a4ux30bfux30eaux30a2ux98a8ux30bdux30fcux30b9}{%
\subsubsection{イタリア風ソース}\label{ux30a4ux30bfux30eaux30a2ux98a8ux30bdux30fcux30b9}}

\hypertarget{sauce-italienne-froide}{%
\paragraph{Sauce Italienne}\label{sauce-italienne-froide}}

\index{そーす@ソース!れいせい@冷製---!いたりあふう@イタリア風---}
\index{いたりあふう@イタリア風!そーす@---ソース(冷製)}
\index{そーす@ソース!いたりあふうれいせい@イタリア風---(冷製)}
\index{sauce@sauce!sauce froide@sauce froide!italienne@--- Italienne}
\index{sauce@sauce!italienne@--- Italienne (froide)}
\index{italien@italien(ne)!sauce froide@Sauce ---ne (froide)}

仔牛の脳半分を、香草を効かせたクールブイヨンで火を通し、目の細かい網で
裏漉しする。同量の牛あるいは羊の脳でもいい。

裏漉ししたピュレを陶製の器に入れ、泡立て器で滑らかになるまで混ぜる。卵黄5個と塩10
g、こしょう1つまみ強、油1
Lとレモン果汁1個分でマヨネーズを作り、そこの脳のピュレを加える。パセリのみじん切り大さじ1杯強を加えて仕上げる。

\ldots{}\ldots{}このソースなどんな冷製の肉料理にも合う。

\hypertarget{ux30deux30e8ux30cdux30fcux30ba}{%
\subsubsection{マヨネーズ}\label{ux30deux30e8ux30cdux30fcux30ba}}

\hypertarget{mayonnaise}{%
\paragraph[Sauce Mayonnaise]{\texorpdfstring{Sauce Mayonnaise\footnote{このソース名の語源には諸説あり、未だ定説と呼べるものはない。
  Mayonnaise という綴りそのものは1806年のヴィアール『帝国料理の本』が初
  出で、Saumon à la Mayonnaise, Filet de Sole en Mayonnaise, Poulet en
  Mayonnaise の3つのレシピが掲載されている。そのうちのひとつ、サーモンの
  マヨネーズは、筒切りにしたサーモンを茹でて冷まし、ジュレを混ぜたマヨネー
  ズをかける、という内容であり、ソースについてはマヨネーズの項を参照となっ
  ているが、どういうわけかこの本にマヨネーズそのもののレシピはない。また、
  「鶏のマヨネーズ仕立て」におけるソースはどう見てもこんにち我々が理解し
  ているマヨネーズとまったく違い、鶏のゼラチン質を冷し固める要素として利
  用したものだ。同じヴィアールの改訂版ともいうべき『王国料理の本』(1822
  年)にはマヨネーズのレシピが掲載されている。興味深いことに「このソース
  にはいろいろな作り方がある。生の卵黄を使うもの、ジュレを使うもの、仔牛
  のグラスを使うものや仔牛の脳を使うもの」として、もっとも一般的な方法と
  して生の卵黄を使う方法が示されている。生の卵黄に攪拌しながら少しずつ油
  を加えていき、固くなってきたらヴィネガー少々を加えてコシをきる、という
  方法であり、こんにち我々の知るマヨネーズに非常に近いものとなっている。
  綴りについては、カレームはmagner(マニェ)捏ねる、という意味の動詞から
  派生したものだとして、magnonnaiseもしくはmagnionnaiseと綴るべきだと
  『パリ風料理の本』で力説している。グリモ・ド・ラ・レニェールは中世フラ
  ンス語で卵黄を意味するmoyeuの派生語としてmoyeunnaiseという綴りを使って
  いる。そのほかフランス大西洋岸の地名バイヨンヌの形容詞bayonnais(バヨ
  ネ)が語源だという説もある。綴りの起源についてある程度有力視されている
  のは、1756年にリシュリュー公爵が当時イギリスに占領されていたミノルカ島
  のマオン港 Mahon を奪取したことにちなんで、mahonnaise と名づけられたと
  いうもの。もっとも、卵黄とヴィネガーを植物油で乳化させたソースという点
  では、beurre de Provenceが1758年刊マラン『コモス神の贈り物』にPigeons,
  au beurre de Provence鳩のプロヴァンスバター添え、というレシピが掲載さ
  れている(t.2,
  pp.290-230)。これは本書『料理の手引き』における\protect\hyperlink{aioli}{アイヨリ
  / プロヴァンスバター}の作り方にやや近く、茹でたにんにくを鉢に
  入れてよくすり潰し、塩、こしょう、ケイパー、アンチョビを加えてさらにす
  り潰し、そこに油を加えて攪拌して濃度を出させる、つまり乳化させる、とい
  うもの。また、植物油ではなくバターを用いるものとして、\protect\hyperlink{sauce-hollandaise}{オランデーズソー
  ス}の原型ともいうべきレシピが1651年のラ・ヴァレー
  ヌ『フランス料理の本』に、Asperges à la Sauce blanche アスパラガスのホ
  ワイトソース添え(p.238)として掲載されていることや、卵黄をポタージュや
  ラグーのとろみ付けに使うことが古くから行なわれていたことなどを総合する
  と、良質のオリーブオイルやひまわり油を利用しやすい環境にある南フランス
  の方がどちらかといえば、卵黄と植物油の乳化作用を利用したソースの発達、
  普及しやすい環境にあったと想像される。なお、この『料理の手引き』では卵
  黄のみを用いたレシピとなっているが、全卵を用いる場合もある。日本の市販
  品でも卵黄のみを使うメーカーと全卵を使用しているメーカーが混在している。な
  お、マヨネーズの仕上りは、卵黄のみか全卵を用いるかという問題もあるが、
  どのような植物油を使うかにも大きく左右されるので注意。}}{Sauce Mayonnaise}}\label{mayonnaise}}

\index{そーす@ソース!れいせい@冷製---!まよねーす@マヨネーズ}
\index{まよねーす@マヨネーズ}
\index{そーす@ソース!まよねーす@マヨネーズ}
\index{sauce@sauce!sauce froide@sauce froide!mayonnaise@--- Mayonnaise}
\index{sauce@sauce!mayonnaise@--- Mayonnaise}
\index{mayonnaise@mayonnaise!sauce@Sauce ---}

冷製ソースのほとんどはマヨネーズの派生ソースだから、\protect\hyperlink{sauce-espagnole}{ソース・エスパニョ
ル}や\protect\hyperlink{veloute}{ヴルテ}と同様に基本ソースと見なされる。
マヨネーズの作り方はきわめてシンプルだが、以下に述べるポイントはしっか
り頭に入れておく必要がある。

\hypertarget{ux6750ux6599ux3068ux5206ux91cf}{%
\subparagraph{材料と分量}\label{ux6750ux6599ux3068ux5206ux91cf}}

\ldots{}\ldots{}卵黄6個、「からざ」は取り除いておくこと。油1 L。塩 10
g、白こしょう1
g、ヴィネガー大さじ1\undemi{}杯または、より白い仕上りを目指す場合にはヴィネガーと同等量のレモン果汁。

\begin{enumerate}
\def\labelenumi{\arabic{enumi}.}
\item
  塩、こしょう、ヴィネガーまたはレモン果汁ほんの少々を加えて、泡立て器で卵黄を溶く。
\item
  油を最初は1滴ずつ加えていき、滑らかにまとまっり始めたら、糸を垂らすようにして油を加えていく。
\item
  何回かに分けてヴィネガーもしくはレモン果汁を少量ずつ加え、コシを切ってやること\footnote{原文
    rompre le corps de la sauce ソースの粘り気をヴィネガーなど
    を加えることで「ゆるめる」あるいは「のばす」こと。ここでは「コシを
    きる」と訳したが、日本の調理用語なので注意。この作業は、一見乳化し
    たように見えてもまだ乳化が不完全であるため、何回かに分けて濃度を下
    げ、攪拌を続けることで乳化を促進させ安定したものにするのが目的。}。
\item
  最後に熱い湯を大さじ3杯加える。これは乳化をしっかりさせて、作り置きしておく必要がある場合でもソースが分離しないようにするため。
\end{enumerate}

\hypertarget{ux539fux6ce8-2}{%
\subparagraph{【原注】}\label{ux539fux6ce8-2}}

\noindent 1.
卵黄だけの段階で塩こしょうをするとソースが分離してしまうのではないかというのは思い込みに過ぎず、実際に調理現場で作業している者はそう考えていない。むしろ、塩を卵黄の水分に溶かし込んでおいた方が、卵黄がまとまりやすくなることは科学的に証明されている\footnote{当時の知見であることに注意。}。

\begin{enumerate}
\def\labelenumi{\arabic{enumi}.}
\setcounter{enumi}{1}
\item
  マヨネーズを作る際に、氷の上に容器を置いて作業するも間違いだ。事実はまったく逆で、冷気が伝わることがもっとも分離させてしまいやすい原因だ。寒い季節には、油はやや微温めか、せめて厨房の室温くらいにするべきだ\footnote{オリーブオイルのように、飽和温度が高い種類の油ではよく見られる現象。ひまわり油でさえも寒さで濁るので、この指摘は正しい。}。
\item
  マヨネーズが分離してしまう原因としては\ldots{}\ldots{}

  \begin{enumerate}
  \def\labelenumii{\arabic{enumii}.}
  \item
    最初に油を入れ過ぎてしまうこと。
  \item
    冷え過ぎた油を使うこと。
  \item
    卵黄の量に対して油の量が多過ぎること。卵黄1個につき油を乳化させることが出来るのは、作り置きするのには1\troisquarts{}
    dl、すぐに使う場合でも2 dlが限度\footnote{卵黄の乳化能力は含まれているレシチンの量で決まるので理論上はもっ
      と大量の油を乳化することが可能。風味や仕上りを考慮に入れて、
      この数字はあくまでも目安と考えたほうがいい。}。
  \end{enumerate}
\end{enumerate}

\maeaki

\hypertarget{ux30b3ux30fcux30c6ux30a3ux30f3ux30b0ux7528ux30deux30e8ux30cdux30fcux30ba}{%
\subsubsection{コーティング用マヨネーズ}\label{ux30b3ux30fcux30c6ux30a3ux30f3ux30b0ux7528ux30deux30e8ux30cdux30fcux30ba}}

\hypertarget{mayonnaise-collee}{%
\paragraph{Sauce Mayonnaise collée}\label{mayonnaise-collee}}

\index{そーす@ソース!れいせい@冷製---!こーていんくようまよねーす@コーティング用マヨネーズ}
\index{まよねーす@マヨネーズ!こーていんくよう@コーティング用---}
\index{そーす@ソース!こーていんくようまよねーす@コーティング用マヨネーズ}
\index{sauce@sauce!sauce froide@sauce froide!mayonnaise collee@--- Mayonnaise collée}
\index{sauce@sauce!mayonnaise collee@--- Mayonnaise collée}
\index{mayonnaise@mayonnaise!sauce collée@Sauce Mayonnaise collée}

コーティング用マヨネーズは、マヨネーズ7 dlに溶かしたジュレ3 dlを混ぜ込
んだもの。野菜サラダをあえるのに使う他、\protect\hyperlink{}{「ロシア風」ショフロワ}の
素材を覆うのにも使う。

\hypertarget{ux539fux6ce8-3}{%
\subparagraph{【原注】}\label{ux539fux6ce8-3}}

\protect\hyperlink{sauce-chaud-froid-maigre}{魚料理用ソース・ショフロワ}の項で述べたよ
うに、このコーティング用マヨネーズの代わりに魚料理用ソース・ショフロワ
を使う方がいい。その方がコーティング用マヨネーズを使う場合よりも風味も
見た目もよくなる。というのも、コーティング用マヨネーズは、冷気によって
ゼラチンが固まるとともに収縮し、マヨネーズに圧力がかかるために、ソース
で素材を覆った表面に油が浸み出してしまう\footnote{初版における原注は、「コーティング用マヨネーズで覆ったものは、
  数時間経つと、油の露で覆われたようになってしまうことがある。その原
  因は、冷気によってゼラチンが固まる際に収縮し、その結果マヨネーズに
  圧力がかかり、液体である油がソースを覆った表面に浸みだしてくること
  だ。これを避けるために、コーティング用マヨネーズはこんにちでは使わ
  れなくなっており、我々の場合だと、かなり以前から魚料理用ソース・ショ
  フロワを用いている(p.163)」。第二版以降、多少の異同はあるが、ほぼ
  第四版の記述と同様。いずれにしても、ジュレ(親水性アミノ酸であるコ
  ラーゲンが主体)を加えたことで、親水基と疎水基を併せ持つ卵黄レシチ
  ンの乳化作用が崩れてマヨネーズが分離した結果だということには気付い
  ていなかったと思われる。}。こういうふうに浸みが出
ることを防ぐには、どんな場合でも、このコーティング用マヨネーズではなく
魚料理用ソース・ショフロワを用いることをお勧めする\footnote{この『料理の手引き』ではジュレを加えたマヨネーズの使用に否定的
  だが、カレーム『19世紀フランス料理』ではSauce Magnonaiseとして、ま
  ず最初にジュレを加えるレシピが掲載されている。概略を示すと、氷の上
  に置いた陶製の容器に卵黄2個、塩、白こしょう少々、エストラゴンヴィ
  ネガー少々を入れる。木のさじで素早くかき混ぜる。まとまってきたら、
  エクス産の油大さじ1杯とヴィネガー少々を、少しずつ加えていく。容器
  の壁に叩きつけるようにしてソースを泡立てていく。この作業でマニョネー
  ズの白さが決まるという。また、油をごく少量ずつ加えていくことを強調
  している。粘度が出て滑らかになったら、最後に油をグラス二杯(≒2 dl)
  と\textbf{アスピック用ジュレ}をグラス\undemi{}杯、エストラゴンヴィネガー
  適量を加えて仕上げる、というもの(t.3, p.132. 強調は引用者による)。
  カレームはこのソースを白く仕上げることにこだわっているが、卵黄の色
  (薄黄色〜オレンジ)は飼料に含まれている色素の影響を大きく受けるの
  で、現代のとうもろこし中心の配合飼料にはパプリカ色素などが添加され
  ていることが多い(赤系の色素を添加すると濃い黄色あるいはオレンジに
  なりやすいため)。いっぽう、「おから」や飼料米などを主な飼料として
  いる場合(有機農業系の平飼い養鶏に多い)は、そういった色素を添加す
  るケースが少ないために、黄身の色が薄くなる傾向がある。19世紀には上
  記のような飼料への色素添加がまだ行なわれていなかったと思われるので、
  白い仕上がりを目指すのは納得のいくところだろう。また、カレームは卵
  黄に含まれるレシチンによって乳化作用が起きることを経験的にさえも理
  解していなかったようであり、卵黄を用いないマニョネーズのレシピも掲
  載されている。なかでも特徴的なのは、「ジュレ入りの白いマニョネーズ」
  のレシピで、これは氷の上に鍋を置き、大きなレードル2杯の白いジュレ
  と同量の油、レードル1杯のヴィネガー、塩、こしょうを入れて卵白用の
  泡立て器でよく混ぜ、途中何回かレモン果汁を少しずつ加えて白く仕上げ
  るようにする、というもの(\emph{ibid}.,
  p.133)。とりわけ舞踏会や格式ある
  大規模な宴席で魚のフィレや鶏のアスピックを飾るのに適していると述べ
  ている。カレームの時代のジュレは冷蔵技術が発達していなかったために、
  基本的にはかなりコラーゲン(ゼラチン質)の多い、固い仕上りのもので
  あったと考えらる。すなわち、固いジュレを油で「ゆるめた」ものがマニョ
  ネーズというソースとして成立し得たのだろう。構造としては現代のマヨ
  ネーズが卵黄や酢の水分の外側をレシチンが覆うようにして油との乳化を
  しているのと逆で、常温でもある程度の固さのあるゼラチンを泡立てて、
  その気泡のなかに油を閉じ込めているイメージだろうか。}。少なくとも、
そうするのが一般的になりつつある。

\maeaki 

\hypertarget{ux30edux30b7ux30a2ux98a8ux30dbux30a4ux30c3ux30d7ux30deux30e8ux30cdux30fcux30ba}{%
\subsubsection{ロシア風ホイップマヨネーズ}\label{ux30edux30b7ux30a2ux98a8ux30dbux30a4ux30c3ux30d7ux30deux30e8ux30cdux30fcux30ba}}

\hypertarget{mayonnaise-fouette-a-la-russe}{%
\paragraph{Sauce Mayonnaise fouettée, à la
Russe}\label{mayonnaise-fouette-a-la-russe}}

\index{そーす@ソース!れいせい@冷製---!ろしあふうほいつふまよねーす@ロシア風ホイップマヨネーズ}
\index{まよねーす@マヨネーズ!ろしあふうほいつふ@ロシア風ホイップ---}
\index{そーす@ソース!ろしあふうほいつふまよねーす@ロシア風ホイップマヨネーズ}
\index{ろしあふう@ロシア風!ほいつふまよねーす@---ホイップマヨネーズ}
\index{sauce@sauce!sauce froide@sauce froide!mayonnaise fouettee russe@--- fouettée à la Russe}
\index{sauce@sauce!mayonnaise fouettee russe@--- Mayonnaise fouettée à la Russe}
\index{mayonnaise@mayonnaise!sauce fouettée russe@Sauce --- fouettée à la Russe}
\index{russe@russe!sauce mayonnaise fouettee@Sauce Mayonnaise fouettée à la ---}

陶製かホーローの容器に、溶かしたジュレ4 dlとマヨネーズ3
dl、エストラゴンヴィネガー大さじ1杯、おろしてさらに細かく刻んだレフォール\footnote{ホースラディッシュ、西洋わさび。}大さじ1杯を入れる。

全体を混ぜ、容器を氷の上に置いて泡立て器でホイップする。ムース状になり、軽く固まり始めるまで、つまりこのソースを使うのに充分な流動性がある状態のところで作業をやめる\footnote{分量比率を考えると、構造的には前項の注で言及したカレームのジュ
  レを主体としたマニョネーズに近いものと思われる。}。\ldots{}\ldots{}主に、野菜のサラダを型に詰めて固めるのに用いる。

\maeaki

\hypertarget{ux30deux30e8ux30cdux30fcux30baux306eux30d0ux30eaux30a8ux30fcux30b7ux30e7ux30f3}{%
\subsubsection{マヨネーズのバリエーション}\label{ux30deux30e8ux30cdux30fcux30baux306eux30d0ux30eaux30a8ux30fcux30b7ux30e7ux30f3}}

\hypertarget{mayonnaises-divierses}{%
\paragraph{Sauce Mayonnaise diverses}\label{mayonnaises-divierses}}

\index{そーす@ソース!れいせい@冷製---!まよねーすのはりえーしよん@マヨネーズのバリエーション}
\index{まよねーす@マヨネーズ!はりえーしよん@---のバリエーション}
\index{そーす@ソース!まよねーすのはりえーしよん@マヨネーズのバリエーション}
\index{sauce@sauce!sauce froide@sauce froide!mayonnaises diverses@--- Mayonnaise diverses}
\index{sauce@sauce!mayonnaises dieverses@---s Mayonnaises diverses}
\index{mayonnaise@mayonnaise!sauces diverses@Sauces --- diverses}

オードブルや冷製料理に合わせるのに、大型甲殻類\footnote{homard
  オマール、langouste ラングースト(≒伊勢エビ)など。}およびエクルヴィス\footnote{ざりがにのこと。詳しくは\protect\hyperlink{sauce-bavaroise}{バイエルン風ソース}訳注参照。}の卵や
クリーム状の部分を用いたり、クルヴェット\footnote{小海老のこと。詳しくは\protect\hyperlink{sauce-aux-crevettes}{ソース・クルヴェット}訳注参照。}、キャビア、アンチョビなどを加
えることでマヨネーズにバリエーションを付けることが出来る。

上記の材料のいずれかをすり潰してから少量のマヨネーズを加えてピュレ状に
して布で漉す。これを適量のマヨネーズに混ぜ合わせればよい。

\maeaki

\hypertarget{ux30bdux30fcux30b9ux30e0ux30b9ux30afux30c6ux30fcux30eb}{%
\subsubsection{ソース・ムスクテール}\label{ux30bdux30fcux30b9ux30e0ux30b9ux30afux30c6ux30fcux30eb}}

\hypertarget{sauce-mousquetaire}{%
\paragraph[Sauce Mousquetaire]{\texorpdfstring{Sauce
Mousquetaire\footnote{マスケット銃兵、近衛騎兵、の意。日本でも子どもむけに翻案された
  もので有名な19世紀のアレクサンドル・デュマ(ペール)の小説 \emph{Les
  Trois Mousquetaires} 『三銃士』の「銃士」がこれに相当する。}}{Sauce Mousquetaire}}\label{sauce-mousquetaire}}

\index{そーす@ソース!れいせい@冷製---!むすくてーる@---・ムスクテール}
\index{むすくてーる@ムスクテール!そーす@ソース・---}
\index{そーす@ソース!むすくてーる@---・ムスクテール}
\index{sauce@sauce!sauce froide@sauce froide!mousquetaire@--- Mousquetaire}
\index{sauce@sauce!mousquetaire@--- Mousquetaire}
\index{mousquetaire@mousquetaire!sauce@Sauce ---}

マヨネーズ1 Lに以下を加える。ごく細かいみじん切りにしたエシャロット80g
を白ワイン1\undemi{}
dlに加えてほとんど煮詰めたもの。溶かした\protect\hyperlink{glace-de-viande}{グラスド
ヴィアンド}大さじ3杯、シブレット\footnote{チャイヴ。アサツキとも訳されることがあるが、日本のものとは風味が異なるので注意。}を細かく刻んだもの大
さじ1杯強。カイエンヌごく少量かミルで挽いたこしょう少々で風味を引き締
める。

\ldots{}\ldots{}羊、牛肉の冷製料理に添える。

\maeaki

\hypertarget{ux30afux30eaux30fcux30e0ux5165ux308aux30bdux30fcux30b9ux30e0ux30bfux30ebux30c9}{%
\subsubsection{クリーム入りソース・ムタルド}\label{ux30afux30eaux30fcux30e0ux5165ux308aux30bdux30fcux30b9ux30e0ux30bfux30ebux30c9}}

\hypertarget{sauce-moutarde-a-la-creme}{%
\paragraph{Sauce moutarde à la crème}\label{sauce-moutarde-a-la-creme}}

\index{そーす@ソース!れいせい@冷製---!くりーむいりむたると@クリーム入り---・ムタルド}
\index{そーす@ソース!むたるとくりーむいり@クリーム入り---・ムタルド(冷製)}
\index{むたると@ムタルド(マスタード)!そーすくりーむいり@クリーム入りソース・---(冷製)}
\index{ますたーと@マスタード(ムタルド)!そーすくりーむいり@クリーム入りソース・ムタルド}
\index{sauce@sauce!sauce froide@sauce froide!moutarde creme@--- moutarde à la crème}
\index{sauce@sauce!moutarde creme@--- moutarde à la crème (froide)}
\index{moutarde@moutarde!sauce creme@Sauce --- à la crème (froide)}

陶製の容器にマスタード大さじ3杯と塩1つまみ、こしょう少々とレモン果汁少々を入れて混ぜ合わせる。ここに少しずつ、マヨネーズを作る要領で、ごく新鮮なクレーム・エペス\footnote{乳酸醗酵させた、とても濃度のある生クリーム。}約2
dlを加える。

\ldots{}\ldots{}オードブル用。

\maeaki

\hypertarget{ux304fux308bux307fux5165ux308aux30bdux30fcux30b9ux30ecux30d5ux30a9ux30fcux30eb}{%
\subsubsection{くるみ入りソース・レフォール}\label{ux304fux308bux307fux5165ux308aux30bdux30fcux30b9ux30ecux30d5ux30a9ux30fcux30eb}}

\hypertarget{sauce-raifort-aux-noix}{%
\paragraph{Sauce Raifort aux noix}\label{sauce-raifort-aux-noix}}

\index{そーす@ソース!れいせい@冷製---!くるみいりれふおーる@くるみ入り---・レフォール}
\index{そーす@ソース!くるみいりれふおーる@くるみ入り---・レフォール(冷製)}
\index{れふおーる@レフォール(ホーシュラディッシュ)!くるみいりそーす@くるみ入りソース・---(冷製)}
\index{くるみ@くるみ!くるみいりれふおーる@---入りソース・レフォール}
\index{sauce@sauce!sauce froide@sauce froide!raifort noix@--- Raifort aux noix}
\index{sauce@sauce!raifort noix@--- Raifort aux noix (froide)}
\index{raifort@raifort!sauce noix@Sauce --- aux noix (froide)}
\index{noix@noix!sauce@sauce!raufort@Sauce Raifort aux --- (froide)}

陶製の器に、おろしたレフォール250gと皮を剥いて刻んだくるみ250 g、塩5 g、
砂糖15 g、クレーム・エペス3dlを入れて混ぜ合わせる。

\ldots{}\ldots{}オンブルシュヴァリエ\footnote{マス科の淡水魚。}の冷製用。

\maeaki

\hypertarget{ux30bdux30fcux30b9ux30e9ux30f4ux30a3ux30b4ux30c3ux30c830-ux30f4ux30a3ux30cdux30b0ux30ecux30c3ux30c831}{%
\subsubsection[ソース・ラヴィゴット /
ヴィネグレット]{\texorpdfstring{ソース・ラヴィゴット\footnote{ホワイト系派生ソースの\protect\hyperlink{sauce-ravigote}{ソース・ラヴィゴット}参照。}
/ ヴィネグレット\footnote{現代フランス語でvinaigrette(ヴィネグレット)はいわゆる「ドレッ
  シング」を指す。語源的にはヴィネガーを意味するvinaigre(ヴィネーグ
  ル)に縮小辞 -ette を付けたもの。ヴィネグレットという名称のレシピ
  としてもっとも古いのは14世紀に成立したとされる「タイユヴァン」のも
  ので、いわゆる「ヴァチカン写本」に収録されており、\textbf{Potaige
  Lyans}「とろみを付けた煮物」に分類されている。概要を示すと、
  menue-hasteムニュアット(豚の脾臓およびレバー半分と腎臓)をロースト
  する。火を通しすぎないよう注意。それを切り分けて、鍋にラード、輪切
  りにした玉ねぎとともに入れて、炭火にかけ、よく混ぜながら火を通す。
  全体によく火が通ったら、牛のブイヨンとワインを注いで沸かす。マニゲッ
  ト、サフランなどを鉢でよくすり潰したらヴィネガーでのばして加え、再
  度沸騰させる。全体にとろみがあって茶色に仕上げる、というもの
  (p.222)。これをほぼ書き写したと思われる14世紀末に書かれた『ル・メ
  ナジエ・ド・パリ』のレシピでは、肉の下処理としてよく洗ってから湯通
  しすること、とろみ付けの要素としてこんがり焼いたパンを香辛料ととも
  にすり潰してワインとヴィネガーで溶く、という指示が追加されている。
  また、こんがり焼いたパンを使わずに茶色に仕上げられるわけがない云々
  という『ル・メナジエ・ド・パリ』の筆者自身の感想も記されている。15
  世紀に書かれたシカールの『料理について』でも豚のレバーを焼いてから
  煮込みヴィネガーを加えるもので、細部は違うが基本的に似たものであり、
  中世においては豚レバーを煮込んでヴィネガーで味付けしたもの、という
  ことになる。これが変化したと思われるのは17世紀。1693年刊マシアロ
  『宮廷およびブルジョワ料理の本』にはBoeuf, Vinaigretteというレシピ
  があり、これは切った牛肉に背脂を刺して塩茹でして冷まし、ヴィネター
  をひと垂らししてレモンのスライスを添えるというとても単純なもの。と
  ころが、1694年のアカデミーフランセーズの辞書には既に「ヴィネガー、
  油、塩、こしょう、パセリ、シブール{[}葱{]}」で作る冷製ソース」という
  定義がなされている。こんにち我々がイメージするヴィネグレットの定義
  にほぼ近い。おおむね17世紀以降、とりわけ後半にヴィネガーと油、塩を
  合わせた冷製ソースというコンセンサスが形成されたと想像される。とこ
  ろで、料理とはまったく関係ないが、いわゆる日本語でいう「人力車」つ
  まり二輪で椅子があり、人力で引く車のこともvinaigretteという。ただ
  しこれは、ヴィネガー醸造業者vinaigriersの用いる小さな馬車と似てこ
  といるからそう呼ばれるようになったという。}}{ソース・ラヴィゴット / ヴィネグレット}}\label{ux30bdux30fcux30b9ux30e9ux30f4ux30a3ux30b4ux30c3ux30c830-ux30f4ux30a3ux30cdux30b0ux30ecux30c3ux30c831}}

\hypertarget{sauce-ravigote-froide}{%
\paragraph{Sauce Ravigote, ou Vinaigrette}\label{sauce-ravigote-froide}}

\index{そーす@ソース!れいせい@冷製---!らういこつと@ラヴィゴット(ヴィネグレット)}
\index{そーす@ソース!らういこつと@---・ラヴィゴット(冷製)}
\index{らういこつと@ラヴィゴット!そーす@ソース・---(冷製)}
\index{ういねくれつと@ヴィネグレット ⇒ソース・ラヴィゴット(冷製)}
\index{sauce@sauce!sauce froide@sauce froide!ravigote@--- ravigote, ou vinaigrette}
\index{sauce@sauce!ravigotte froide@--- Ravigote, ou Vinaigrette (froide)}
\index{ravigote@ravigote!sauce@Sauce --- , ou vinaigrette (froide)}
\index{vinaigrette@vinaigrette ⇒ sauce ravigote (froide)}

\hypertarget{ux6750ux6599}{%
\subparagraph{材料}\label{ux6750ux6599}}

\ldots{}\ldots{}油5 dl、ヴィネガー2
dl、小さめのケイパー小さじ2杯、パセリ50
g、セルフイユとエストラゴン、シブレットを刻んだもの40
g、細かくみじん切りにした玉ねぎ70g、塩4 g、こしょう1
g。以上をよく混ぜ合わせる。

\ldots{}\ldots{}仔牛の頭や足、羊の足などに合わせる。

\maeaki

\hypertarget{ux30bdux30fcux30b9ux30ecux30e0ux30e9ux30fcux30c934}{%
\subsubsection[ソース・レムラード]{\texorpdfstring{ソース・レムラード\footnote{ソース名としての初出はおそらくムノン『ブルジョワ屋敷勤めの女性
  料理人のための本』(1734)におけるSauce à la rémoladeだろう。レシ
  ピの概要は、エシャロット、パセリ、シブール、にんにく1片、アンチョ
  ビ、ケイパー、いずれもごく細かく刻んで鍋に入れ、塩、粗挽きこしょう
  を加え、マスタード少々と油、ヴィネガーでのばす、というもの。つまり、
  乳化ソースであるマヨネーズをベースにした本書のレムラードと、乳化さ
  せないという点が異なるのみで、基本的なところは共通していると見てい
  い。ヴィアール『帝国料理の本』第7版(1812年)にはRémouladeの綴りで、
  緑色のレムラード、レムラード、インド風レムラードと3種のレシピが掲
  載されている(この版にはまだマヨネーズのレシピは掲載されていない)。
  このうちのレムラードのレシピの概要は、グラス1杯のマスタードを器に
  入れ、エシャロットのみじん切り少々と香草少々を加える。油を大さじ6〜
  7杯、ヴィネガー大さじ3〜4杯、塩、粗挽きこしょうを加える。これらを
  よく混ぜ合わせ、生の卵黄2個を加えてさらによく混ぜる。ソースがよく
  まとまるように気をつけてしっかり綷。やや濃い仕上りにする、というも
  の(p.53)。手順的にはやや異なるが、卵黄を用いて乳化させようとしてい
  ることがわかる。緑のレムラードも生の卵黄を用いるなど、香草をすり潰
  すことと、ほうれんそうの緑の色素を用いる以外はレムラードと同様。な
  お、インド風レムラードの場合は固茹で卵の卵黄10個をよくすり潰して大
  さじ8杯の油を加えてさらによく混ぜる。唐辛子とターメリックの粉末、
  塩、こしょう、ヴィネガーを加える。出来るだけ粘りが出るようにする。
  これを布で漉して供する(id.)。}}{ソース・レムラード}}\label{ux30bdux30fcux30b9ux30ecux30e0ux30e9ux30fcux30c934}}

\hypertarget{sauce-remoulade}{%
\paragraph{Sauce Rémoulade}\label{sauce-remoulade}}

\index{そーす@ソース!れいせい@冷製---!れむらーと@---・レムラード}
\index{れむらーと@レムラード!そーす@ソース・---}
\index{そーす@ソース!れむらーと@---・レムラード}
\index{sauce@sauce!sauce froide@sauce froide!remoulade@--- Rémoulade}
\index{sauce@sauce!remoulade@--- Rémoulade}
\index{remoulade@rémoulade!sauce@Sauce ---}

\protect\hyperlink{mayonnaise}{マヨネーズ}1
Lに以下のものを加える。マスタード大さじ
1\undemi{}杯。コルニション100とケイパー50gを細かく刻んで、圧して余分な
水気を絞ったもの。パセリ、セルフイユ、エストラゴンのみじん切り大さじ1
杯。アンチョビエッセンス大さじ\undemi{}杯。

\maeaki

\hypertarget{ux30edux30b7ux30a2ux98a8ux30bdux30fcux30b9}{%
\subsubsection{ロシア風ソース}\label{ux30edux30b7ux30a2ux98a8ux30bdux30fcux30b9}}

\hypertarget{sauce-russe-froide}{%
\paragraph{Sauce Russe}\label{sauce-russe-froide}}

\index{そーす@ソース!れいせい@冷製---!ろしあふう@ロシア風---}
\index{ろしあふう@ロシア風!そーすれいせい@---ソース(冷製)}
\index{そーす@ソース!ろしあふうれいせい@ロシア風---(冷製)}
\index{sauce@sauce!sauce froide@sauce froide!russe@--- Russe}
\index{sauce@sauce!russe@--- Russe (froide)}
\index{russe@russe!sauce froide@Sauce --- (froide)}

鉢に、オマール\footnote{homard ロブスター。}かラングースト\footnote{langouste
  ≒ 伊勢エビ。}の胴のクリーム状の部分100 gとキャ ビア100 g\footnote{チョウザメの卵の塩蔵品のことだが、「高級」とされる順に、beluga
  (ベルガ)、osciètre, ossetra(オシエートル、オセトラ)、
  sevruga(セヴルガ)の種類がある(ここで示した読みがなはフランス語
  風のもの)。}、マヨネーズ大さじ2〜3杯を加えてよくすり潰す。これを目の
細かい漉し器で裏漉しする。こうして出来たピュレに、マヨネーズ
\troisquarts{} Lを加える。大さじ1杯強のマスタードと、同量のダービーソー
ス\footnote{初版では原注として、風味付けにマスタードを加えることを示唆して
  いるのみ。第二版では「マスタードとウスターシャソースを各大さじ1杯
  強」、第三版では「マスタードとエスコフィエソースを大さじ1杯強」と
  変遷している。なお、ダービーソースDerby Sauce の1946年の広告には、
  このブランド名でバーベキューソース、ステーキソース、ウスターシャソー
  ス、ホットソース、チャプスイソースのラインナップが記されている。現
  実問題として、もし加えるとするならリー\&ペリンのようなウスターシャ
  ソースということになろうか。}を加えて仕上げる。

\ldots{}\ldots{}魚および甲殻類の冷製料理に添える。

\maeaki

\hypertarget{ux30bfux30ebux30bfux30ebux30bdux30fcux30b9}{%
\subsubsection{タルタルソース}\label{ux30bfux30ebux30bfux30ebux30bdux30fcux30b9}}

\hypertarget{sauce-tartare}{%
\paragraph{Sauce Tartare}\label{sauce-tartare}}

\index{そーす@ソース!れいせい@冷製---!たるたる@タルタル---}
\index{たるたる@タルタル!そーすれいせい@---ソース(冷製)}
\index{そーす@ソース!たるたるれいせい@タルタル---(冷製)}
\index{sauce@sauce!sauce froide@sauce froide!tartare@--- Tartare}
\index{sauce@sauce!tartare@--- Tartare (froide)}
\index{tartare@tartare!sauce froide@Sauce --- (froide)}

固茹で卵の黄身8個をすり潰して滑らかになるまでよく練る。塩、挽きたての
こしょう各1つまみ強で味付けする。油1 Lとヴィネガー大さじ2杯を加えなが
らソースを立てていく\footnote{明記されていないが、\protect\hyperlink{mayonnaise}{マヨネーズ}や\protect\hyperlink{sauce-gribiche}{ソース・グリビッシュ}と同様に作業すること。}。若どりの玉ねぎ\footnote{いわゆる「オニオンヌーヴォー」だが、日本でこの名称で流通してい
  るものは黄色系の品種が多いのに対し、フランスでは白系品種(oignon blanc
  オニョンブロン)が一般的であり、風味が異なることに注意。}の葉またはシブレット20gをすり
潰してマヨネーズ大さじ2杯でのばし、目の細かい網で裏漉ししたものを加え
て仕上げる。

\ldots{}\ldots{}このソースは、冷製の家禽や肉料理、魚料理、甲殻類いずれにも合う。ま
た、「ディアーブル(悪魔風)」仕立ての肉料理、鶏料理にも用いられる。

\maeaki

\hypertarget{ux30bdux30fcux30b9ux30f4ux30a7ux30ebux30c841}{%
\subsubsection[ソース・ヴェルト]{\texorpdfstring{ソース・ヴェルト\footnote{緑のソース、の意。この名称のソースは中世からある。このレシピで
  はほうれんそうとクレソンが主体になっているが、時代とともにその材料
  には変遷がある。中世においては、麦の若葉をすり潰して用いるレシピが
  多かった。}}{ソース・ヴェルト}}\label{ux30bdux30fcux30b9ux30f4ux30a7ux30ebux30c841}}

\hypertarget{sauce-verte}{%
\paragraph{Sauce Verte}\label{sauce-verte}}

\index{そーす@ソース!れいせい@冷製---!うえると@---・ヴェルト}
\index{うえーる@ヴェール / ヴェルト!そーす@ソース・ヴェルト}
\index{そーす@ソース!うえると@---・ヴェルト}
\index{sauce@sauce!sauce froide@sauce froide!verte@--- Verte}
\index{sauce@sauce!verte@--- Verte}
\index{vert@vert(e)!sauce@Sauce ---e}

ほうれんそうの葉\footnote{日本では、ほうれんそうを葉のみではなく葉軸とともに利用するのが
  一般的だが、伝統的なフランス料理において葉軸は使われないのが普通。
  そもそも日本のほうれんそうは密植して葉が立つように仕立てて比較的若
  どりするのに対して、ヨーロッパ品種のほうれんそうは株間を充分にとっ
  てロゼッタ状に葉が広がるように栽培するのが伝統的な手法。この場合、
  葉は肉厚に仕上がるが、葉軸は太くて固いため可食部と見なされなかった。
  昔のフランスの八百屋の店先では軸を切り捨てる作業風景がよく見られた
  という。現代では機械収穫に適した立性の品種が増えており、専用の大型
  機械で株元近くから切り取り、自動的に軸をある程度除去して併走する巨
  大なコンテナに移すという収穫方法が普及している。なお、現代の日本で
  冬季に出回る「ちぢみほうれんそう」と呼ばれるものはヨーロッパの伝統
  的な栽培方法にやや近く、品種もヨーロッパ系の形質をほぼそのまま残し
  ていると思われるものがある。ただし、かつてのヨーロッパでは葉を「か
  き取り」しながら収穫していたのに対して、「ちぢみほうれんそう」は
  「株どり」を前提にして栽培し、さらに袋詰めしやすいようにある程度水
  分が抜けた状態で出荷されることが多いので、まったく同様というわけで
  はない。}50 gとクレソンの葉50 g、パセリの葉とセルフイユ、エスト
ラゴンを同量ずつ計50 gを、沸騰した湯に投入し、強火で5分間茹でる。水気
をきり、手早く冷水にさらす。しっかりと圧し絞って水気をきり、鉢に入れて
すり潰す。これをトーション\footnote{サービスタオルとも呼ばれる。いわゆるサービス用の布巾。やや厚め
  で網目が詰まっているものが多い。}でくるんできつく絞り、葉の濃い汁 を1
dl搾りだす。

固く立てて風味付けをした\protect\hyperlink{mayonnaise}{マヨネーズ}9
dlにこの緑の汁を加える。

\ldots{}\ldots{}冷製の魚料理や甲殻類に合わせる。
\end{recette}