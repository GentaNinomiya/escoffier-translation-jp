\hypertarget{ux51b7ux88fdux30bdux30fcux30b9}{%
\section{冷製ソース}\label{ux51b7ux88fdux30bdux30fcux30b9}}

\frsec{Sauces Froides}

\index{sauce@sauce!sauces froides@sauces froides}
\index{そーす@ソース!れいせいそーす@冷製ソース}
\begin{recette}
\hypertarget{sauce-aioli}{%
\subsubsection{アイヨリ / プロヴァンスバター}\label{sauce-aioli}}

\frsub{Sauce Aïoli, ou Beurre de Provence}\footnote{aïoli(アイヨリ)はailloliとも綴るが、
  ail(にんにく)+ oil(油)の合成語。19世紀前
  半には既にアカデミーフランセージの辞書に収録されており、広く知られ
  ていたようだ。ブイヤベースに添えるルイユとよく似ているが、ルイユが
  カイエンヌを加えるのに対して、こちらはにんにくと油、塩、レモン汁と
  少々の水だけで作る。用途も、茹でた塩鱈やじゃがいも、茹で卵、アーティ
  チョーク、さやいんげん、などに合わせることが多い。\\
  Beurre de Provence (ブールドプロヴォンス)の名称を持つレシピとし
  てもっとも古いと思われるものは、1758年刊マラン『コモス神の贈り物』 に
  \emph{Pigeons, au beurre de Provence} (鳩のプロヴァンスバター添え)
  だろう(t.2, pp.290-230)。ただし、このレシピは卵黄と油の乳化ソース
  ではない。古くは、北フランス(ラングドゥイユ圏)と南フランス(ラン
  グトック圏)の食文化は気候などにより大きく異なっており、その代表的
  なものに乳製品の扱いがある。オリーブオイルそのものを beurre de
  Provence プロヴァンスバターと呼ぶことも多かった。実際、オリーブオ
  イルは品質にもよるが、5℃くらいでほぼ固形化すること、とりわけエク
  ストラバージンと呼ばれるランクの製品は粘度の高いものが多いことから、
  南フランスの食文化においてはバターと似た位置を占めていた。}

\index{そーす@ソース!れいせい@冷製---!あいより@アイヨリ}
\index{そーす@ソース!れいせい@冷製---!ふろふあんすはたー@プロヴァンスバター}
\index{あいより@アイヨリ}
\index{ふろふあんす@プロヴァンス!ふろふあんすはたー@プロヴァンスバター}
\index{はたー@バター!ふろふあんすはたー@プロヴァンスバター}
\index{sauce@sauce!sauce froide@sauce froide!aioli@--- Aïoli}
\index{sauce@sauce!sauce froide@sauce froide!beurre de provence@Beurre de Provence}
\index{aioli@Aïoli!sauce@Sauce ---}
\index{provence@Provence!Beurre de Provence (Aïoli)}
\index{beurre@beurre!beurre de provence@Beurre de Provence (Aïoli)}

にんにく4片(30 g)を鉢\footnote{この種の作業には、大理石製のものが伝統的によく用いられる。。}に入れて細かくすり潰す。ここに生の卵黄1個、
塩1つまみを加える。混ぜながら、2\undemi{} dlの油\footnote{原書ではとくに言及されていないが、プロヴァンス地方ではオリーブオイルを用いることが一般的。}を初めは1滴ずつ加
えていき、ソースがまとまりはじめたら糸を垂らすようにして加える。この作
業は鉢に入れたままで、棒をはげしく動かして行なう。

攪拌する作業の途中、レモン1個分の搾り汁と冷水大さじ\undemi{}杯を少しずつ加えて、ソースが固くなり過ぎないようにしてやること。

\hypertarget{nota-sauce-aioli}{%
\subparagraph{【原注】}\label{nota-sauce-aioli}}

このアイヨリソースが分離してしまいそうな時は、卵黄をさらに1個足して、
マヨネーズと場合と同様に修正すること。

\maeaki

\hypertarget{sauce-andalouse}{%
\subsubsection{アンダルシア風ソース}\label{sauce-andalouse}}

\frsub{Sauce Andalouse}\footnote{いうまでもなくスペインのアンダルシア地方のことだが、トマトやオリー
  ブオイル、チョリソなどこの地方を「想起」させる食材が使われている料
  理などがこの名称になっている傾向がある。ところが、トマトにしろオリー
  ブオイルにしろアンダルシア地方特有というわけではなく、アンダルシア
  が産地として有名なチョリソくらいしか、料理名の根拠となり得るものは
  ない。逆に言えば、アンダルシア地方の食文化との関係は、そこに用いら
  れている食材以外にはないものと考えてもいい。料理名に付けられた地方
  名がとりたてて根拠や由来のないものであることを示す一例。}

\index{そーす@ソース!れいせい@冷製---!あんたるしあふう@アンダルシア風---}
\index{あんたるしあ@アンダルシア!そーす@---風ソース}
\index{そーす@ソース!あんたるしあふう@アンダルシア風---}
\index{sauce@sauce!sauce froide@sauce froide!Andalouse@--- Andalouse}
\index{sauce@sauce!andalouse@--- Andalouse}
\index{andalous@Andalous(e)!sauce@Sauce Andalouse}

ごく固く仕上げた\protect\hyperlink{mayonnaise}{ソース・マヨネーズ}\troisquarts{}
Lに、
上等な赤いトマトピュレ2\undemi{}dlを加える。小さなさいの目に切ったポワ
ヴロン\footnote{Poivron
  いわゆる日本で青果として輸入されているパプリカ(肉厚の辛
  くないピーマン)とほぼ同じものだが、香辛料として用いられる粉末のパ
  プリカと混同を避けるため、あえてフランス語をそのままカタカナに訳し
  た。}75 gを仕上げに加える。

\maeaki

\hypertarget{sauce-bohemienne}{%
\subsubsection{ソース・ボヘミアの娘}\label{sauce-bohemienne}}

\frsub{Sauce Bohémienne}\footnote{アイルランド出身の作曲家マイケル・ウィリアム・バルフェMichael
  William Balfe (1808〜1870)のオペラ\emph{The Bohemien Girl}『ボヘミア
  の少女』のフランス語版タイトル\href{https://archive.org/details/labohmiennegrand00balf}{\emph{La
  Bohémienne}}『
  ラボエミエーヌ』にちなんだものと言われている。この作品はロンドンで
  1843年初演、1862年に四幕形式のフランス語版がパリのオペラ=コミック
  劇場で上演され、大ヒットしたという。この名を冠した料理はいくつかあ
  るが、いずれもチェコのボヘミア地方とは何の関連性も認められないため、
  オペラの人気作品にあやかった料理名と考えるのが妥当だろう。}

\index{そーす@ソース!れいせい@冷製---!ほへみあのむすめ@---ボヘミアの娘}
\index{ほへみあ@ボヘミア!そーす@ソース・---の娘}
\index{そーす@ソース!ほへみあ@---・ボヘミアの娘}
\index{sauce@sauce!sauce froide@sauce froide!bohemienne@--- Bohémienne}
\index{sauce@sauce!bohemienne@--- Bohémienne}
\index{bohemien@bohémien(ne)!sauce@Sauce Bohémienne}

陶製の容器に、濃厚でよく冷やした\protect\hyperlink{sauce-bechamel}{ベシャメルソー
ス}1\undemi{} dlと卵黄4個、塩10 g、こしょう少々、ヴィ
ネガー数滴を入れる。

泡立て器で全体をよく混ぜ、標準的なマヨネーズを作るのとまったく同じ要領
で、油1 Lとエストラゴンヴィネガー大さじ2杯程を加える。

\ldots{}\ldots{}仕上げに、マスタード大さじ1杯を加える。

\maeaki

\hypertarget{sauce-chantilly-froide}{%
\subsubsection{ソース・シャンティイ}\label{sauce-chantilly-froide}}

\frsub{Sauce Chantilly}\footnote{パリ近郊の地名。詳しくはホワイト系派生ソースの\protect\hyperlink{sauce-chantilly}{ソース・シャンティ
  イ}訳注参照。}

\index{そーす@ソース!れいせい@冷製---!しやんていい@---・シャンティイ}
\index{しやんていい@シャンティイ!そーす@ソース・---(冷製)}
\index{そーす@ソース!しやんていい@---・シャンティイ}
\index{sauce@sauce!sauce froide@sauce froide!chantilly@--- Chantilly}
\index{sauce@sauce!chantilly@--- Chantilly (froide)}
\index{chantilly@Chantilly!sauce@Sauce --- (froide)}

酸味付けにレモンを用いて、固く仕上げた\protect\hyperlink{mayonnaise}{ソース・マヨネー
ズ}\troisquarts{} Lを用意しておく。提供直前に、ごく固く泡
立てた生クリーム大さじ4杯\footnote{大さじ1杯 = 15
  ccという概念にとらわれないよう注意。原文は、大き
  なスプーンで泡立てた生クリームをざっくりと4回加えるイメージで書か
  れている。本書における通常のソースの仕上がり量が約1 Lであることを
  考慮すると、最低でも100ml以上は加えることになるだろう。}を加える。その後、味を\ruby{調}{ととの}え
る。

\ldots{}\ldots{}もっぱら、アスパラガスの冷製、温製に添える。

\hypertarget{ux539fux6ce8}{%
\subparagraph{【原注】}\label{ux539fux6ce8}}

生クリームを加えるのは、このソースを使うまさにその時にすること。前もっ
て加えておくと、ソースが分離してしまう恐れがあるので注意。

\maeaki

\hypertarget{sauce-genoise-froids}{%
\subsubsection{ジェノヴァ風ソース}\label{sauce-genoise-froids}}

\frsub{Sauce Génoise}\footnote{あまり明確な由来はないが、ジェノヴァが地中海に面した港町であり、
  このソースが魚料理用であるという点で一応の説明はつくだろう。}

\index{そーす@ソース!れいせい@冷製---!しえのうあふう@ジェノヴァ風---}
\index{しえのうあふう@ジェノヴァ風!そーす@ソース・---(冷製)}
\index{そーす@ソース!しえのうあふう@ジェノヴァ風---}
\index{sauce@sauce!sauce froide@sauce froide!genoise@--- Génoise}
\index{sauce@sauce!genoise@--- Génoise (froide)}
\index{genois@Génois(e)!sauce@Sauce ---e (froide)}

殻と皮を剥いたばかりのピスタチオ40 gと、松の実25 g、松の実がない場合は
スイートアーモンド20
gを鉢に入れてよくすり潰し、冷めた\protect\hyperlink{sauce-bechamel}{ベシャメルソー
ス}小さじ1杯程度を加えて練ってペースト状にする。これ
を目の細かい網で裏漉しする。陶製の容器に卵黄6個、塩1つまみ、こしょう少々
を入れる。泡立て器でよく混ぜる。油1 Lと中位の大きさのレモン2個の搾り汁
を少しずつ加えてよく混ぜて乳化させていく\footnote{明記されていないが、ソースをしっかりと乳化させるためには\protect\hyperlink{mayonnaise}{マヨネー
  ズ}と同様に作業すること。}。仕上げにハーブのピュレ大
さじ3杯を加える。これは、パセリの葉とセルフイユ、エストラゴン、時季が
合えばサラダバーネットを同量ずつ用意し、強火で2分間下茹でしてから湯を
きり、冷水にさらしてから水気を強く絞り、裏漉しして作っておく。

\ldots{}\ldots{}冷製の魚料理全般に合わせられる。

\maeaki

\hypertarget{sauce-gribiche}{%
\subsubsection{ソース・グリビッシュ}\label{sauce-gribiche}}

\frsub{Sauce Gribiche}\footnote{由来不明の語。ノルマンディ方言で「子どもを怖がらせるおばさん」
  の意味で用いられるということが分かっているのみ。19世紀後半以降に創
  案もしくは一般化したソースと思われる。本書初版には当然のように既に
  収録されており、その後の大きな異同もない。ただ、本書初版以前に出版
  された料理書においてこのソースのレシピはまだ見つかっていない。ファー
  ヴルは1905年刊『料理および食品衛生事典』第二版で「ある種のレムラー
  ドにレストランで付けられた名称」と定義し、掲載しているレシピは本書
  初版のものと大差ないが、「ウスターシャソース少々も加える」となって
  いるところが目を引く。また、1913年初版のプルーストの長編小説『失な
  われた時を求めて』の「スワン家の方へ」冒頭において「彼(=スワン)を
  招いていない夕食会のために、ソース・グリビッシュやパイナップルのサ
  ラダのレシピが必要になるや、ためらいもなく探しに行かせたりするのだっ
  た」(p.18)。もしこの語り手の記述が正確であるなら、19世紀末には広く
  知られたものであったと考えるべきだが、小説の場合は必ずしも歴史的事
  実と符号するわけではないので注意が必要。}

\index{そーす@ソース!れいせい@冷製---!くりひつしゆ@---・グリビッシュ}
\index{くりひつしゆ@グリビッシュ!そーす@ソース・---(冷製)}
\index{そーす@ソース!くりひつしゆ@---・グリビッシュ}
\index{sauce@sauce!sauce froide@sauce froide!gribiche@--- Gribiche}
\index{sauce@sauce!gribiche@--- Gribiche (froide)}
\index{gribiche@gribiche!sauce@Sauce --- (froide)}

茹であがったばかりの固茹で卵の黄身6個を陶製のボウルに入れ、マスタード
小さじ1杯、塩1つまみ強、こしょう適量を加えてよく練り、滑らかなペースト
状にする。植物油\undemi{} Lとヴィネガー大さじ1\undemi{}杯を加えながら
よく混ぜて乳化させる。仕上げに、コルニションとケイパーのみじん切り計 100
gと、パセリとセルフイユ、エストラゴンのみじん切りのミックスを大さ
じ1杯、短かめの千切りにした固茹で卵の白身3個分を加える。

\ldots{}\ldots{}冷製の魚料理に添えるのが一般的。

\maeaki

\hypertarget{sauce-groseilles-au-raifort}{%
\subsubsection{レフォール風味のソース・グロゼイユ}\label{sauce-groseilles-au-raifort}}

\frsub{Sauce Groseilles au Raifort}

\index{そーす@ソース!れいせい@冷製---!れふおーるふうみくろせいゆ@レフォール風味の---・グロゼイユ}
\index{くろせいゆ@グロゼイユ!そーすれふおーる@レフォール風味のソース・---(冷製)}
\index{れふおーる@レフォール!そーすくろせいゆ@---風味のソース・グロゼイユ}
\index{そーす@ソース!れふおーるふうみくろせいゆ@レフォール風味の---・グロゼイユ}
\index{sauce@sauce!sauce froide@sauce froide!grroseilles raifort@--- Grroseilles au Rifort}
\index{sauce@sauce!groseille@--- Groseilles au Raifort (froide)}
\index{raifort@raifort!sauce@Sauce Groseilles au --- (froide)}
\index{groseille@groseille!sauce@Sauce --- au Raifort (froide)}

ポルト酒1 dlにナツメグ、シナモン、塩、こしょう各1つまみを加え、を
\deuxtiers{}量まで煮詰める。溶かした\protect\hyperlink{}{グロゼイユのジュレ}4
dlと細か くすりおろしたレフォール大さじ2杯を加える。

(さまざまな用途に使える)

\maeaki

\hypertarget{sauce-italienne-froide}{%
\subsubsection{イタリア風ソース}\label{sauce-italienne-froide}}

\frsub{Sauce Italienne}\footnote{このソースも温製のイタリア風ソースと同様に名称にとくに由来など
  はないと思われる。}

\index{そーす@ソース!れいせい@冷製---!いたりあふう@イタリア風---}
\index{いたりあふう@イタリア風!そーす@---ソース(冷製)}
\index{そーす@ソース!いたりあふうれいせい@イタリア風---(冷製)}
\index{sauce@sauce!sauce froide@sauce froide!italienne@--- Italienne}
\index{sauce@sauce!italienne@--- Italienne (froide)}
\index{italien@italien(ne)!sauce froide@Sauce ---ne (froide)}

仔牛の脳半分を、香草を効かせたクールブイヨンで火を通し、目の細かい網で
裏漉しする。同量の牛あるいは羊の脳でもいい。

裏漉ししたピュレを陶製の器に入れ、泡立て器で滑らかになるまで混ぜる。卵
黄5個と塩10 g、こしょう1つまみ強、油1 Lとレモン果汁1個分でマヨネーズを
作り、そこの脳のピュレを加える。パセリのみじん切り大さじ1杯強を加えて
仕上げる。

\ldots{}\ldots{}このソースなどんな冷製の肉料理にも合う。

\hypertarget{mayonnaise}{%
\subsubsection{マヨネーズ}\label{mayonnaise}}

\frsub{Sauce Mayonnaise}\footnote{このソース名の語源には諸説あり、未だ定説と呼べるものはない。
  Mayonnaise という綴りそのものは1806年のヴィアール『帝国料理の本』が初
  出で、Saumon à la Mayonnaise, Filet de Sole en Mayonnaise, Poulet en
  Mayonnaise の3つのレシピが掲載されている。そのうちのひとつ、サーモンの
  マヨネーズは、筒切りにしたサーモンを茹でて冷まし、ジュレを混ぜたマヨネー
  ズをかける、という内容であり、ソースについてはマヨネーズの項を参照となっ
  ているが、どういうわけかこの本にマヨネーズそのもののレシピはない。また、
  「鶏のマヨネーズ仕立て」におけるソースはどう見てもこんにち我々が理解し
  ているマヨネーズとまったく違い、鶏のゼラチン質を冷し固める要素として利
  用したものだ。同じヴィアールの改訂版ともいうべき『王国料理の本』(1822
  年)にはマヨネーズのレシピが掲載されている。興味深いことに「このソース
  にはいろいろな作り方がある。生の卵黄を使うもの、ジュレを使うもの、仔牛
  のグラスを使うものや仔牛の脳を使うもの」として、もっとも一般的な方法と
  して生の卵黄を使う方法が示されている。生の卵黄に攪拌しながら少しずつ油
  を加えていき、固くなってきたらヴィネガー少々を加えてコシをきる、という
  方法であり、こんにち我々の知るマヨネーズに非常に近いものとなっている。
  また、1814年刊ボヴィリエ『調理技法』のソース・マヨネーズは、焼き物の器
  に油大さじ3〜4杯とエストラゴンヴィネガー2杯を入れる。細かく刻んだエス
  トラゴン、エシャロット、サラダバーネットをたっぷり加え、ジュレ大さじ2、
  3杯を加える。ソースがまとまって、ポマード状になったら、味を調える
  (p.66)、というもの。ここでも卵黄と植物油の乳化ソースとはなっていない。
  綴りについては、カレームはmagner(マニェ)捏ねる、という意味の動詞から
  派生したものだとして、magnonnaiseもしくはmagnionnaiseと綴るべきだと
  『パリ風料理の本』で力説している。グリモ・ド・ラ・レニェールは中世フラ
  ンス語で卵黄を意味するmoyeuの派生語としてmoyeunnaiseという綴りを使って
  いる。そのほかフランス大西洋岸の地名バイヨンヌの形容詞bayonnais(バヨ
  ネ)が語源だという説もある。綴りの起源についてある程度有力視されている
  のは、1756年にリシュリュー公爵が当時イギリスに占領されていたミノルカ島
  のマオン港 Mahon を奪取したことにちなんで、mahonnaise と名づけられたと
  いうもの。ところで、植物油ではなくバターを用いるものとして、\protect\hyperlink{sauce-hollandaise}{オランデー
  ズソース}の原型ともいうべきレシピが1651年のラ・ヴァ
  レーヌ『フランス料理の本』に、Asperges à la Sauce blanche アスパラガス
  のホワイトソース添え(p.238)として掲載されていることや、卵黄をポタージュ
  やラグーのとろみ付けに使うことが古くから行なわれていたことなどを総合す
  ると、良質のオリーブオイルやひまわり油を利用しやすい環境にある南フラン
  スの方がどちらかといえば、卵黄と植物油の乳化作用を利用したソースの発達、
  普及しやすい環境にあったとも想像されよう。なお、この『料理の手引き』で
  は卵黄のみを用いたレシピとなっているが、全卵を用いる場合もある。日本の
  市販品でも卵黄のみを使うメーカーと全卵を使用しているメーカーが混在して
  いる。なお、マヨネーズの仕上がりは、卵黄のみか全卵を用いるかという問題も
  あるが、どのような植物油を使うかにも大きく左右されるので注意。}

\index{そーす@ソース!れいせい@冷製---!まよねーす@マヨネーズ}
\index{まよねーす@マヨネーズ}
\index{そーす@ソース!まよねーす@マヨネーズ}
\index{sauce@sauce!sauce froide@sauce froide!mayonnaise@--- Mayonnaise}
\index{sauce@sauce!mayonnaise@--- Mayonnaise}
\index{mayonnaise@mayonnaise!sauce@Sauce ---}

冷製ソースのほとんどはマヨネーズの派生ソースだから、\protect\hyperlink{sauce-espagnole}{ソース・エスパニョ
ル}や\protect\hyperlink{veloute}{ヴルテ}と同様に基本ソースと見なされる。
マヨネーズの作り方はきわめてシンプルだが、以下に述べるポイントはしっか
り頭に入れておく必要がある。

\hypertarget{proportions-mayonnaise}{%
\subparagraph{材料と分量}\label{proportions-mayonnaise}}

\ldots{}\ldots{}卵黄6個、「からざ」は取り除いておくこと。油1 L。塩 10
g、白こしょ う1
g、ヴィネガー大さじ1\undemi{}杯または、より白い仕上がりを目指す場
合にはヴィネガーと同等量のレモン果汁。

\begin{enumerate}
\def\labelenumi{\arabic{enumi}.}
\item
  塩、こしょう、ヴィネガーまたはレモン果汁ほんの少々を加えて、泡立て器で卵黄を溶く。
\item
  油を最初は1滴ずつ加えていき、滑らかにまとまっり始めたら、糸を垂らすようにして油を加えていく。
\item
  何回かに分けてヴィネガーもしくはレモン果汁を少量ずつ加え、コシを切ってやること\footnote{原文
    rompre le corps de la sauce ソースの粘り気をヴィネガーなど
    を加えることで「ゆるめる」あるいは「のばす」こと。ここでは「コシを
    きる」と訳したが、日本の調理用語なので注意。この作業は、一見乳化し
    たように見えてもまだ乳化が不完全であるため、何回かに分けて濃度を下
    げ、攪拌を続けることで乳化を促進させ安定したものにするのが目的。}。
\item
  最後に熱い湯を大さじ3杯加える。これは乳化をしっかりさせて、作り置きしておく必要がある場合でもソースが分離しないようにするため。
\end{enumerate}

\hypertarget{nota-mayonnaise}{%
\subparagraph{【原注】}\label{nota-mayonnaise}}

\noindent 1.
卵黄だけの段階で塩こしょうをするとソースが分離してしまうのではない
かというのは思い込みに過ぎず、実際に調理現場で作業している者はそう
考えていない。むしろ、塩を卵黄の水分に溶かし込んでおいた方が、卵黄
がまとまりやすくなることは科学的に証明されている\footnote{当時の知見であることに注意。}。

\begin{enumerate}
\def\labelenumi{\arabic{enumi}.}
\setcounter{enumi}{1}
\item
  マヨネーズを作る際に、氷の上に容器を置いて作業するも間違いだ。事実
  はまったく逆で、冷気が伝わることがもっとも分離させてしまいやすい原
  因だ。寒い季節には、油はやや微温めか、せめて厨房の室温くらいにする
  べきだ\footnote{オリーブオイルのように、飽和温度が高い種類の油ではよく見られる現象。ひまわり油でさえも寒さで濁るので、この指摘は正しい。}。
\item
  マヨネーズが分離してしまう原因としては\ldots{}\ldots{}

  \begin{enumerate}
  \def\labelenumii{\arabic{enumii}.}
  \tightlist
  \item
    最初に油を入れ過ぎてしまうこと。
  \item
    冷え過ぎた油を使うこと
  \item
    卵黄の量に対して油の量が多過ぎること。卵黄1個につき油を乳化させ
    ることが出来るのは、作り置きするのには1\troisquarts{} dl、すぐ
    に使う場合でも2 dlが限度\footnote{卵黄の乳化能力は含まれているレシチンの量で決まるので理論上はもっ
      と大量の油を乳化することが可能。風味や仕上がりを考慮に入れて、
      この数字はあくまでも目安と考えたほうがいい。}。
  \end{enumerate}
\end{enumerate}

\maeaki

\hypertarget{mayonnaise-collee}{%
\subsubsection{コーティング用マヨネーズ}\label{mayonnaise-collee}}

\frsub{Sauce Mayonnaise collée}

\index{そーす@ソース!れいせい@冷製---!こーていんくようまよねーす@コーティング用マヨネーズ}
\index{まよねーす@マヨネーズ!こーていんくよう@コーティング用---}
\index{そーす@ソース!こーていんくようまよねーす@コーティング用マヨネーズ}
\index{sauce@sauce!sauce froide@sauce froide!mayonnaise collee@--- Mayonnaise collée}
\index{sauce@sauce!mayonnaise collee@--- Mayonnaise collée}
\index{mayonnaise@mayonnaise!sauce collée@Sauce Mayonnaise collée}

コーティング用マヨネーズは、マヨネーズ7 dlに溶かしたジュレ3 dlを混ぜ込
んだもの。野菜サラダをあえるのに使う他、\protect\hyperlink{}{「ロシア風」ショフロワ}の
素材を覆うのにも使う。

\hypertarget{nota-mayonnaise-collee}{%
\subparagraph{【原注】}\label{nota-mayonnaise-collee}}

\protect\hyperlink{sauce-chaud-froid-maigre}{魚料理用ソース・ショフロワ}の項で述べたよ
うに、このコーティング用マヨネーズの代わりに魚料理用ソース・ショフロワ
を使う方がいい。その方がコーティング用マヨネーズを使う場合よりも風味も
見た目もよくなる。というのも、コーティング用マヨネーズは、冷気によって
ゼラチンが固まるとともに収縮し、マヨネーズに圧力がかかるために、ソース
で素材を覆った表面に油が浸み出してしまう\footnote{初版における原注は、「コーティング用マヨネーズで覆ったものは、
  数時間経つと、油の露で覆われたようになってしまうことがある。その原
  因は、冷気によってゼラチンが固まる際に収縮し、その結果マヨネーズに
  圧力がかかり、液体である油がソースを覆った表面に浸みだしてくること
  だ。これを避けるために、コーティング用マヨネーズはこんにちでは使わ
  れなくなっており、我々の場合だと、かなり以前から魚料理用ソース・ショ
  フロワを用いている(p.163)」。第二版以降、多少の異同はあるが、ほぼ
  第四版の記述と同様。いずれにしても、ジュレ(親水性アミノ酸であるコ
  ラーゲンが主体)を加えたことで、親水基と疎水基を併せ持つ卵黄レシチ
  ンの乳化作用が崩れてマヨネーズが分離した結果だということには気付い
  ていなかったと思われる。}。こういうふうに浸みが出
ることを防ぐには、どんな場合でも、このコーティング用マヨネーズではなく
魚料理用ソース・ショフロワを用いることをお勧めする\footnote{この『料理の手引き』ではジュレを加えたマヨネーズの使用に否定的
  だが、カレーム『19世紀フランス料理』ではSauce Magnonaiseとして、ま
  ず最初にジュレを加えるレシピが掲載されている。概略を示すと、氷の上
  に置いた陶製の容器に卵黄2個、塩、白こしょう少々、エストラゴンヴィ
  ネガー少々を入れる。木のさじで素早くかき混ぜる。まとまってきたら、
  エクス産の油大さじ1杯とヴィネガー少々を、少しずつ加えていく。容器
  の壁に叩きつけるようにしてソースを泡立てていく。この作業でマニョネー
  ズの白さが決まるという。また、油をごく少量ずつ加えていくことを強調
  している。粘度が出て滑らかになったら、最後に油をグラス二杯(≒2 dl)
  と\textbf{アスピック用ジュレ}をグラス\undemi{}杯、エストラゴンヴィネガー
  適量を加えて仕上げる、というもの(t.3, p.132. 強調は引用者による)。
  また、カレームは卵黄に含まれるレシチンによって乳化作用が起きること
  を経験的にさえも理解していなかったようであり、卵黄を用いないマニョ
  ネーズのレシピも掲載されている。なかでも特徴的なのは、「ジュレ入り
  の白いマニョネーズ」のレシピで、これは氷の上に鍋を置き、大きなレー
  ドル2杯の白いジュレと同量の油、レードル1杯のヴィネガー、塩、こしょ
  うを入れて卵白用の泡立て器でよく混ぜ、途中何回かレモン果汁を少しず
  つ加えて白く仕上げるようにする、というもの(\emph{ibid}.,
  p.133)。とりわ
  け舞踏会や格式ある大規模な宴席で魚のフィレや鶏のアスピックを飾るの
  に適していると述べている。}。少なくとも、
そうするのが一般的になりつつある。

\maeaki

\hypertarget{mayonnaise-fouette-a-la-russe}{%
\subsubsection{ロシア風ホイップマヨネーズ}\label{mayonnaise-fouette-a-la-russe}}

\frsub{Sauce Mayonnaise fouettée, à la Russe}

\index{そーす@ソース!れいせい@冷製---!ろしあふうほいつふまよねーす@ロシア風ホイップマヨネーズ}
\index{まよねーす@マヨネーズ!ろしあふうほいつふ@ロシア風ホイップ---}
\index{そーす@ソース!ろしあふうほいつふまよねーす@ロシア風ホイップマヨネーズ}
\index{ろしあふう@ロシア風!ほいつふまよねーす@---ホイップマヨネーズ}
\index{sauce@sauce!sauce froide@sauce froide!mayonnaise fouettee russe@--- fouettée à la Russe}
\index{sauce@sauce!mayonnaise fouettee russe@--- Mayonnaise fouettée à la Russe}
\index{mayonnaise@mayonnaise!sauce fouettée russe@Sauce --- fouettée à la Russe}
\index{russe@russe!sauce mayonnaise fouettee@Sauce Mayonnaise fouettée à la ---}

陶製かホーローの容器に、溶かしたジュレ4 dlとマヨネーズ3 dl、エストラゴ
ンヴィネガー大さじ1杯、おろしてさらに細かく刻んだレフォール\footnote{ホースラディッシュ、西洋わさび。}大さじ
1杯を入れる。

全体を混ぜ、容器を氷の上に置いて泡立て器でホイップする。ムース状になり、
軽く固まり始めるまで、つまりこのソースを使うのに充分な流動性がある状態
のところで作業をやめる\footnote{分量比率を考えると、構造的には前項の注で言及したカレームのジュ
  レを主体としたマニョネーズに近いものと思われる。}。\ldots{}\ldots{}主に、野菜のサラダを型に詰めて固める
のに用いる。

\maeaki

\hypertarget{mayonnaises-divierses}{%
\subsubsection{マヨネーズのバリエーション}\label{mayonnaises-divierses}}

\frsub{Sauce Mayonnaise diverses}

\index{そーす@ソース!れいせい@冷製---!まよねーすのはりえーしよん@マヨネーズのバリエーション}
\index{まよねーす@マヨネーズ!はりえーしよん@---のバリエーション}
\index{そーす@ソース!まよねーすのはりえーしよん@マヨネーズのバリエーション}
\index{sauce@sauce!sauce froide@sauce froide!mayonnaises diverses@--- Mayonnaise diverses}
\index{sauce@sauce!mayonnaises dieverses@---s Mayonnaises diverses}
\index{mayonnaise@mayonnaise!sauces diverses@Sauces --- diverses}

オードブルや冷製料理に合わせるのに、大型甲殻類\footnote{homard
  オマール、langouste ラングースト(≒伊勢エビ)など。}およびエクルヴィス\footnote{ざりがにのこと。詳しくは\protect\hyperlink{sauce-bavaroise}{バイエルン風ソース}訳注参照。}の卵や
クリーム状の部分を用いたり、クルヴェット\footnote{小海老のこと。詳しくは\protect\hyperlink{sauce-aux-crevettes}{ソース・クルヴェット}訳注参照。}、キャビア、アンチョビなどを加
えることでマヨネーズにバリエーションを付けることが出来る。

上記の材料のいずれかをすり潰してから少量のマヨネーズを加えてピュレ状に
して布で漉す。これを適量のマヨネーズに混ぜ合わせればよい。

\maeaki

\hypertarget{sauce-mousquetaire}{%
\subsubsection{ソース・ムスクテール}\label{sauce-mousquetaire}}

\frsub{Sauce Mousquetaire}\footnote{マスケット銃兵、近衛騎兵、の意。日本でも子どもむけに翻案された
  もので有名な19世紀のアレクサンドル・デュマ(ペール)の小説 \emph{Les
  Trois Mousquetaires} 『三銃士』の「銃士」がこれに相当する。}

\index{そーす@ソース!れいせい@冷製---!むすくてーる@---・ムスクテール}
\index{むすくてーる@ムスクテール!そーす@ソース・---}
\index{そーす@ソース!むすくてーる@---・ムスクテール}
\index{sauce@sauce!sauce froide@sauce froide!mousquetaire@--- Mousquetaire}
\index{sauce@sauce!mousquetaire@--- Mousquetaire}
\index{mousquetaire@mousquetaire!sauce@Sauce ---}

マヨネーズ1 Lに以下を加える。ごく細かいみじん切りにしたエシャロット80g
を白ワイン1\undemi{}
dlに加えてほとんど煮詰めたもの。溶かした\protect\hyperlink{glace-de-viande}{グラスド
ヴィアンド}大さじ3杯、シブレット\footnote{チャイヴ。アサツキとも訳されることがあるが、日本のものとは風味が異なるので注意。}を細かく刻んだもの大
さじ1杯強。カイエンヌごく少量かミルで挽いたこしょう少々で風味を引き締
める。

\ldots{}\ldots{}羊、牛肉の冷製料理に添える。

\maeaki

\hypertarget{sauce-moutarde-a-la-creme}{%
\subsubsection{生クリーム入りソース・ムタルド}\label{sauce-moutarde-a-la-creme}}

\frsub{Sauce moutarde à la crème}

\index{そーす@ソース!れいせい@冷製---!なまくりーむいりむたると@生クリーム入り---・ムタルド}
\index{そーす@ソース!むたるとなまくりーむいり@生クリーム入り---・ムタルド(冷製)}
\index{むたると@ムタルド(マスタード)!そーすなまくりーむいり@生クリーム入りソース・---(冷製)}
\index{ますたーと@マスタード(ムタルド)!そーすなまくりーむいり@生クリーム入りソース・ムタルド}
\index{sauce@sauce!sauce froide@sauce froide!moutarde creme@--- moutarde à la crème}
\index{sauce@sauce!moutarde creme@--- moutarde à la crème (froide)}
\index{moutarde@moutarde!sauce creme@Sauce --- à la crème (froide)}

陶製の容器にマスタード大さじ3杯と塩1つまみ、こしょう少々とレモン果汁少々
を入れて混ぜ合わせる。ここに少しずつ、マヨネーズを作る要領で、ごく新鮮
なクレーム・エペス\footnote{乳酸醗酵させた、とても濃度のある生クリーム。}約2
dlを加える。

\ldots{}\ldots{}オードブル用。

\maeaki

\hypertarget{sauce-raifort-aux-noix}{%
\subsubsection{くるみ入りソース・レフォール}\label{sauce-raifort-aux-noix}}

\frsub{Sauce Raifort aux noix}

\index{そーす@ソース!れいせい@冷製---!くるみいりれふおーる@くるみ入り---・レフォール}
\index{そーす@ソース!くるみいりれふおーる@くるみ入り---・レフォール(冷製)}
\index{れふおーる@レフォール(ホーシュラディッシュ)!くるみいりそーす@くるみ入りソース・---(冷製)}
\index{くるみ@くるみ!くるみいりれふおーる@---入りソース・レフォール}
\index{sauce@sauce!sauce froide@sauce froide!raifort noix@--- Raifort aux noix}
\index{sauce@sauce!raifort noix@--- Raifort aux noix (froide)}
\index{raifort@raifort!sauce noix@Sauce --- aux noix (froide)}
\index{noix@noix!sauce@sauce!raufort@Sauce Raifort aux --- (froide)}

陶製の器に、おろしたレフォール250gと皮を剥いて刻んだくるみ250 g、塩5 g、
砂糖15 g、クレーム・エペス3dlを入れて混ぜ合わせる。

\ldots{}\ldots{}オンブルシュヴァリエ\footnote{サケ科の淡水魚。体長20〜30
  cmのものが多く、最大で70 cmを越える
  ものもいるという。日本の岩魚に近い。フランスではアルプスのドイツお
  よびイタリアとの国境付近に生息するが、現代では養殖も多いという。}の冷製用。

\maeaki

\hypertarget{sauce-ravigote-froide}{%
\subsubsection{ソース・ラヴィゴット /
ヴィネグレット}\label{sauce-ravigote-froide}}

\frsub{Sauce Ravigote, ou Vinaigrette}\footnote{ラヴィゴットの意味などについてはホワイト系派生ソースの\protect\hyperlink{sauce-ravigote}{ソース・
  ラヴィゴット}参照。現代フランス語の
  vinaigrette(ヴィネグレット)はいわゆる「ドレッシング」を指す。語
  源的にはヴィネガーを意味するvinaigre(ヴィネーグル)に縮小辞 -ette
  を付けたもの。ヴィネグレットという名称のレシピとしてもっとも古いの
  は14世紀に成立したとされる「タイユヴァン」のもので、いわゆる「ヴァ
  チカン写本」に収録されており、\textbf{Potaige Lyans}「とろみを付けた煮
  物」に分類されている。概要を示すと、menue-hasteムニュアット(豚の
  脾臓およびレバー半分と腎臓)をローストする。火を通しすぎないよう注
  意。それを切り分けて、鍋にラード、輪切りにした玉ねぎとともに入れて、
  炭火にかけ、よく混ぜながら火を通す。全体によく火が通ったら、牛のブ
  イヨンとワインを注いで沸かす。マニゲット、サフランなどを鉢でよくす
  り潰したらヴィネガーでのばして加え、再度沸騰させる。全体にとろみが
  あって茶色に仕上げる、というもの(p.222)。これをほぼ書き写したと思
  われる14世紀末に書かれた『ル・メナジエ・ド・パリ』のレシピでは、肉
  の下処理としてよく洗ってから湯通しすること、とろみ付けの要素として
  こんがり焼いたパンを香辛料とともにすり潰してワインとヴィネガーで溶
  く、という指示が追加されている。また、こんがり焼いたパンを使わずに
  茶色に仕上げられるわけがない云々という『ル・メナジエ・ド・パリ』の
  筆者自身の感想も記されている。15世紀に書かれたシカールの『料理につ
  いて』でも豚のレバーを焼いてから煮込みヴィネガーを加えるもので、細
  部は違うが基本的に似たものであり、中世においては豚レバーを煮込んで
  ヴィネガーで味付けしたもの、ということになる。これが変化したと思わ
  れるのは17世紀。1693年刊マシアロ『宮廷およびブルジョワ料理の本』に
  はBoeuf, Vinaigretteというレシピがあり、これは切った牛肉に背脂を刺
  して塩茹でして冷まし、ヴィネターをひと垂らししてレモンのスライスを
  添えるというとても単純なもの。ところが、1694年のアカデミーフランセー
  ズの辞書には既に「ヴィネガー、油、塩、こしょう、パセリ、シブール
  {[}葱{]}」で作る冷製ソース」という定義がなされている。こんにち我々が
  イメージするヴィネグレットの定義にほぼ近い。おおむね17世紀以降、と
  りわけ後半にヴィネガーと油、塩を合わせた冷製ソースというコンセンサ
  スが形成されたと想像される。ところで、料理とはまったく関係ないが、
  いわゆる日本語でいう「人力車」つまり二輪で椅子があり、人力で引く車
  のこともvinaigretteという。ただしこれは、ヴィネガー醸造業者
  vinaigriersの用いる小さな馬車と似てこといるからそう呼ばれるように
  なったという。}

\index{そーす@ソース!れいせい@冷製---!らういこつと@ラヴィゴット(ヴィネグレット)}
\index{そーす@ソース!らういこつと@---・ラヴィゴット(冷製)}
\index{らういこつと@ラヴィゴット!そーす@ソース・---(冷製)}
\index{ういねくれつと@ヴィネグレット ⇒ソース・ラヴィゴット(冷製)}
\index{sauce@sauce!sauce froide@sauce froide!ravigote@--- ravigote, ou vinaigrette}
\index{sauce@sauce!ravigotte froide@--- Ravigote, ou Vinaigrette (froide)}
\index{ravigote@ravigote!sauce@Sauce --- , ou vinaigrette (froide)}
\index{vinaigrette@vinaigrette ⇒ sauce ravigote (froide)}

\hypertarget{ux6750ux6599}{%
\subparagraph{材料}\label{ux6750ux6599}}

\ldots{}\ldots{}油5 dl、ヴィネガー2
dl、小さめのケイパー小さじ2杯、パセリ50 g、セ
ルフイユとエストラゴン、シブレットを刻んだもの40 g、細かくみじん切りに
した玉ねぎ70g、塩4 g、こしょう1 g。以上をよく混ぜ合わせる。

\ldots{}\ldots{}仔牛の頭や足、羊の足などに合わせる。

\maeaki

\hypertarget{sauce-remoulade}{%
\subsubsection{ソース・レムラード}\label{sauce-remoulade}}

\frsub{Sauce Rémoulade}\footnote{ソース名としての初出はおそらくムノン『ブルジョワ屋敷勤めの女性
  料理人のための本』(1734)におけるSauce à la rémolade
  {[}sic.{]}だろう。レシ
  ピの概要は、エシャロット、パセリ、シブール、にんにく1片、アンチョ
  ビ、ケイパー、いずれもごく細かく刻んで鍋に入れ、塩、粗挽きこしょう
  を加え、マスタード少々と油、ヴィネガーでのばす、というもの。つまり、
  乳化ソースであるマヨネーズをベースにした本書のレムラードと、乳化さ
  せないという点が異なるのみで、基本的なところは共通していると見てい
  い。ヴィアール『帝国料理の本』第7版(1812年)にはRémouladeの綴りで、
  緑色のレムラード、レムラード、インド風レムラードと3種のレシピが掲
  載されている(この版にはまだマヨネーズのレシピは掲載されていない)。
  このうちのレムラードのレシピの概要は、グラス1杯のマスタードを器に
  入れ、エシャロットのみじん切り少々と香草少々を加える。油を大さじ6〜
  7杯、ヴィネガー大さじ3〜4杯、塩、粗挽きこしょうを加える。これらを
  よく混ぜ合わせ、生の卵黄2個を加えてさらによく混ぜる。ソースがよく
  まとまるように気をつけてしっかり綷。やや濃い仕上がりにする、というも
  の(p.53)。手順的にはやや異なるが、卵黄を用いて乳化させようとしてい
  ることがわかる。緑のレムラードも生の卵黄を用いるなど、香草をすり潰
  すことと、ほうれんそうの緑の色素を用いる以外はレムラードと同様。な
  お、インド風レムラードの場合は固茹で卵の卵黄10個をよくすり潰して大
  さじ8杯の油を加えてさらによく混ぜる。唐辛子とターメリックの粉末、
  塩、こしょう、ヴィネガーを加える。出来るだけ粘りが出るようにする。
  これを布で漉して供する(id.)。カレームに至るとさらにレシピは洗練さ
  れたものとなり、Sauce Rémoulade à la Ravigote(ソース・レムラード・
  アラ・ラヴィゴット)では、セルフイユとエストラゴン、サラダバーネッ
  ト、シブレットを茹がいて水にさらした後に水気を搾り、固茹で卵の卵黄
  を加えてよくすり潰し、塩、こしょう、ナツメグで調味して、上等のマス
  タードを加える。ここにエクス産の油とエストラゴンヴィネガーを少しず
  つ加えていく。最後に布で漉す(t.1, p.135)というもの。いずれにしても
  マヨネーズを基本ソースとして展開するという『料理の手引き』の発想、
  体系化にいたるまで100年近くを要したことになる。}

\index{そーす@ソース!れいせい@冷製---!れむらーと@---・レムラード}
\index{れむらーと@レムラード!そーす@ソース・---}
\index{そーす@ソース!れむらーと@---・レムラード}
\index{sauce@sauce!sauce froide@sauce froide!remoulade@--- Rémoulade}
\index{sauce@sauce!remoulade@--- Rémoulade}
\index{remoulade@rémoulade!sauce@Sauce ---}

\protect\hyperlink{mayonnaise}{マヨネーズ}1
Lに以下のものを加える。マスタード大さじ
1\undemi{}杯。コルニション100とケイパー50gを細かく刻んで、圧して余分な
水気を絞ったもの。パセリ、セルフイユ、エストラゴンのみじん切り大さじ1
杯。アンチョビエッセンス大さじ\undemi{}杯。

\maeaki

\hypertarget{sauce-russe-froide}{%
\subsubsection{ロシア風ソース}\label{sauce-russe-froide}}

\frsub{Sauce Russe}

\index{そーす@ソース!れいせい@冷製---!ろしあふう@ロシア風---}
\index{ろしあふう@ロシア風!そーすれいせい@---ソース(冷製)}
\index{そーす@ソース!ろしあふうれいせい@ロシア風---(冷製)}
\index{sauce@sauce!sauce froide@sauce froide!russe@--- Russe}
\index{sauce@sauce!russe@--- Russe (froide)}
\index{russe@russe!sauce froide@Sauce --- (froide)}

鉢に、オマール\footnote{homard ロブスター。}かラングースト\footnote{langouste
  ≒ 伊勢エビ。}の胴のクリーム状の部分100 gとキャ ビア100 g\footnote{チョウザメの卵の塩蔵品のことだが、「高級」とされる順に、beluga
  (ベルガ)、osciètre, ossetra(オシエートル、オセトラ)、
  sevruga(セヴルガ)の種類がある(ここで示した読みがなはフランス語
  風のもの)。}、マヨネーズ大さじ2〜3杯を加えてよくすり潰す。これを目の
細かい漉し器で裏漉しする。こうして出来たピュレに、マヨネーズ
\troisquarts{} Lを加える。大さじ1杯強のマスタードと、同量のダービーソー
ス\footnote{初版では原注として、風味付けにマスタードを加えることを示唆して
  いるのみ。第二版では「マスタードとウスターシャソースを各大さじ1杯
  強」、第三版では「マスタードとエスコフィエソースを大さじ1杯強」と
  変遷している。なお、ダービーソースDerby Sauce の1946年の広告には、
  このブランド名でバーベキューソース、ステーキソース、ウスターシャソー
  ス、ホットソース、チャプスイソースのラインナップが記されている。現
  実問題として、もし加えるとするならリー\&ペリンのようなウスターシャ
  ソースということになろうか。}を加えて仕上げる。

\ldots{}\ldots{}魚および甲殻類の冷製料理に添える。

\maeaki

\hypertarget{sauce-tartare}{%
\subsubsection{タルタルソース}\label{sauce-tartare}}

\frsub{Sauce Tartare}\footnote{タルタル(タタール)=フランス人から見て東方の蛮族、というイメー
  ジで語られがちだが、カレーム『19世紀フランス料理』にあるSauce
  Rémoulade à la Mogol {[}Mongoleの誤植と思われる{]}「モンゴル風ソース・
  レムラード」およびSauce à la Tartare「タルタル風ソース」のレシピを
  見るかぎり、誤解という可能性も感じられる。前者は固茹で卵の卵黄に塩、
  こしょう、ナツメグ、カイエンヌ、砂糖、油、エストラゴンヴィネガーを
  合わせてピュレ状にして布で漉し、サフランを煎じた汁で美しい黄色に染
  め、刻んだシブレットを加えて仕上げるというもの。後者はソース・アル
  マンドとマスタード同量に生の卵黄2個を加え、塩、こしょう、ナツメグ
  で調味してエクス産の油レードル2杯分とレードル\undemi{}杯のエストラ
  ゴンヴィネガーを少しずつ加えながら混ぜていく。みじん切りにして下茹
  でしたエシャロット少々とにんにく少々、エストラゴンとセルフイユのみ
  じん切りを大さじ1杯加える、というもの(pp.137-138)。少なくともこれ
  らのレシピにおいて、タルタルすなわち野蛮、というニュアンスを見出す
  ことは出来ないだろう。なお、Steak tartareタルタルステーキのレシピ
  は本書には掲載されておらず、1938年の『ラルース・ガストロノミック』
  初版が初出と思われる(p.1019)。}

\index{そーす@ソース!れいせい@冷製---!たるたる@タルタル---}
\index{たるたる@タルタル!そーすれいせい@---ソース(冷製)}
\index{そーす@ソース!たるたるれいせい@タルタル---(冷製)}
\index{sauce@sauce!sauce froide@sauce froide!tartare@--- Tartare}
\index{sauce@sauce!tartare@--- Tartare (froide)}
\index{tartare@tartare!sauce froide@Sauce --- (froide)}

固茹で卵の黄身8個をすり潰して滑らかになるまでよく練る。塩、挽きたての
こしょう各1つまみ強で味付けする。油1 Lとヴィネガー大さじ2杯を加えなが
らソースを立てていく\footnote{明記されていないが、\protect\hyperlink{mayonnaise}{マヨネーズ}や\protect\hyperlink{sauce-gribiche}{ソース・グリビッ
  シュ}と同様に作業すること。}。若どりの玉ねぎ\footnote{いわゆる「オニオンヌーヴォー」だが、日本でこの名称で流通してい
  るものは黄色系の品種が多いのに対し、フランスでは白系品種(oignon blanc
  オニョンブロン)が多く、風味が異なることに注意。}の葉またはシブレット20gをすり
潰してマヨネーズ大さじ2杯でのばし、目の細かい網で裏漉ししたものを加え
て仕上げる。

\ldots{}\ldots{}このソースは、冷製の家禽や肉料理、魚料理、甲殻類いずれにも合う。ま
た、「ディアーブル(悪魔風)」仕立ての肉料理、鶏料理にも用いられる。

\maeaki

\hypertarget{sauce-verte}{%
\subsubsection{ソース・ヴェルト}\label{sauce-verte}}

\frsub{Sauce Verte}\footnote{緑のソース、の意。この名称のソースは中世からある。このレシピで
  はほうれんそうとクレソンが主体になっているが、時代とともにその材料
  には変遷がある。中世においては、麦の若葉をすり潰して用いるレシピが
  多かった。}

\index{そーす@ソース!れいせい@冷製---!うえると@---・ヴェルト}
\index{うえーる@ヴェール / ヴェルト!そーす@ソース・ヴェルト}
\index{そーす@ソース!うえると@---・ヴェルト}
\index{sauce@sauce!sauce froide@sauce froide!verte@--- Verte}
\index{sauce@sauce!verte@--- Verte}
\index{vert@vert(e)!sauce@Sauce ---e}

ほうれんそうの葉\footnote{日本では、ほうれんそうを葉のみではなく葉軸とともに利用するのが
  一般的だが、伝統的なフランス料理において葉軸は使われないのが普通。
  そもそも日本のほうれんそうは密植して葉が立つように仕立てて比較的若
  どりするのに対して、ヨーロッパ品種のほうれんそうは株間を充分にとっ
  てロゼッタ状に葉が広がるように栽培するのが伝統的な手法。この場合、
  葉は肉厚に仕上がるが、葉軸は太くて固いため可食部と見なされなかった。
  昔のフランスの八百屋の店先では軸を切り捨てる作業風景がよく見られた
  という。現代では機械収穫に適した立性の品種が増えており、専用の大型
  機械で株元近くから切り取り、自動的に軸をある程度除去して併走する巨
  大なコンテナに移すという収穫方法が普及しており、量産品のピュレなど
  に使用されている。}50 gとクレソンの葉50
g、パセリの葉とセルフイユ、エスト ラゴンを同量ずつ計50
gを、沸騰した湯に投入し、強火で5分間茹でる。水気
をきり、手早く冷水にさらす。しっかりと圧し絞って水気をきり、鉢に入れて
すり潰す。これをトーション\footnote{綿などの天然素材で出来た調理場及びホール業務に用いられる布。サ
  イズは50〜55cm×70〜80cmのものが多い。}でくるんできつく絞り、葉の濃い汁
を1 dl搾りだす。

固く立てて風味付けをした\protect\hyperlink{mayonnaise}{マヨネーズ}9
dlにこの緑の汁を加える。

\ldots{}\ldots{}冷製の魚料理や甲殻類に合わせる。

\maeaki

\hypertarget{sauce-vincent}{%
\subsubsection{ソース・ヴァンサン}\label{sauce-vincent}}

\frsub{Sauce Vencent}\footnote{18世紀フランスを代表する料理人のひとり、Vincent
  La Chapelleヴァ
  ンサン・ラシャペル(1690または1703〜1745)の名を冠したソースと言わ
  れている。彼はチェスターフィールド伯フィリップ・スタンホープに仕え
  ていた頃に三巻からなる『近代料理』\emph{Modern Cook}英語版を1733年に上
  梓。そのフランス語版(全4巻)は1835年に\emph{Le Cuisinier moderne}のタ
  イトルでアムステルダムで刊行された。そして全5巻からなる第二版を
  1742年に自費で出版した。このソースはヴァンラサン・ラシャペル本人の
  考案したものとも言われるが、1742年版の著書には収録されていない。た
  だ、香草の扱いを得意としていたのは事実のようで、Sauce en Ravigote
  (ソース・オン・ラヴィゴット)だけでも5種のレシピが掲載されている。}

\index{そーす@ソース!れいせい@冷製---!うあんさん@---・ヴァンサン}
\index{うあんさん@ヴァンサン!そーす@ソース・---}
\index{そーす@ソース!うあんさん@---・ヴァンサン}
\index{sauce@sauce!sauce froide@sauce froide!vincent@--- Vincent}
\index{sauce@sauce!vincent@--- Vincent}
\index{vincent@Vincent!sauce@Sauce ---}

\hypertarget{ux4f5cux308aux65b91}{%
\subparagraph{作り方(1)}\label{ux4f5cux308aux65b91}}

\ldots{}\ldots{}オゼイユ\footnote{タデ科の葉菜。英語由来のソレルという名称もよく使われる。日本の
  スカンポに近く、そのように訳されることもあるが、オゼイユは野菜とし
  て品種の選抜育成が長期にわたって行なわれたために、同じとものとはは
  いい難い。}の葉とパセリの葉、セルフイユ、エストラゴン、シブレット、サラダバーネット\footnote{pimprenelle
  パンプルネル。}のごく若い葉をきっちり同量ずつ、計100 g、クレソンの葉60
gとほうれんそうの葉60 gを沸騰した湯で強火で2〜3分間茹がく。

湯をきって、冷水にさらす。しっかり水分を圧し絞って、鉢\footnote{伝統的には大理石製の鉢がこの種の作業には用いられた。}に入れる。茹であがったばかりの固茹で卵の黄身6個を加えて滑かになるまですり潰す。

これを布で漉し\footnote{このように濃度のあるものを布で漉す方法については\protect\hyperlink{veloute}{ヴルテ}訳注参照。}、陶製の容器に移す。塩1つまみ強とこしょう適量、生の卵黄5個を加える。油8dlとヴィネガー適量を加えながら混ぜ、滑らかに乳化させる。

風味付けにダービーソース\footnote{\protect\hyperlink{sauce-russe-froide}{ロシア風ソース}訳注参照。}大さじ1杯を加えて仕上げる。

\hypertarget{ux4f5cux308aux65b92}{%
\subparagraph{作り方(2)}\label{ux4f5cux308aux65b92}}

\ldots{}\ldots{}作り方(1)の香草と葉菜のピュレを作るところまでは同じ。

これに\protect\hyperlink{mayonnaise}{マヨネーズ}を加えて、同様に仕上げる。

\ldots{}\ldots{}冷製の魚料理、甲殻類にとりわけ合う。

\hypertarget{ux539fux6ce8-1}{%
\subparagraph{【原注】}\label{ux539fux6ce8-1}}

このソースは18世紀の偉大な料理人のひとり、ヴァンサン・ラシャペルが考案したもの\footnote{卵黄と植物油をベースとした乳化ソースとしてのマヨネーズの起源は
  判然としないところが多いが、19世紀初頭のヴィアールやカレームの記述
  を読むかぎりにおいて、卵黄レシチンによる油と水分の乳化作用について
  は経験レベルでさえはっきりとは認識されていなかったと考えられる。こ
  のソースあるいはこれに相当するレシピがヴァンサン・ラシャペルの著書
  に掲載されていないこと、ヴァンラン・ラシャペルがレストランの店主で
  はなく貴族に仕えていた料理人、メートルドテルであったことを考慮する
  と、このソースの考案者が彼である可能性も、自身の名をソース名に冠し
  た可能性もきわめて低いと言わざるを得ない。ただし、香草と葉菜を茹で
  てすり潰したピュレを上述の各種レムラードのいくつかで使用しているこ
  とから、後世にこの名称が付いた、あるいはこのソースの最大のポイント
  がヴァンサン・ラシャペル風の香草のピュレであると考えることも可能だ
  ろう。}。

\maeaki

\hypertarget{sauce-suedoise}{%
\subsubsection{スウェーデン風ソース}\label{sauce-suedoise}}

\frsub{Sauce Suédoise}\footnote{基本的にソース名はアルファベット順に掲載されているのだが、この
  ソースだけが後からとって付けたように末尾にある。実際、このレシピは
  第二版から掲載となっているが、ある程度組版が進んだ段階で急遽追加さ
  れたのだろうか。なお、1907年の英語版には掲載されていない。原注の最
  後「このソースはマスタードで風味付けしてもいい」は第四版で追加され
  たものだが、他は第二版からまったく異同がなく、掲載順も変化していな
  いのはいささか不思議なところ。}

\index{そーす@ソース!れいせい@冷製---!すうえーてんふう@スウェーデン風---}
\index{すうえーてんふう@スウェーデン風!そーすれいせい@---ソース(冷製)}
\index{そーす@ソース!すうえーてんふうれいせい@スウェーデン風---(冷製)}
\index{sauce@sauce!sauce froide@sauce froide!suedoise@--- Suédoise}
\index{sauce@sauce!suedoise@--- Suédoise (froide)}
\index{suedois@suédois(e)!sauce froide@Sauce ---e (froide)}

酸味のある固いリンゴを薄切りにして鍋にしっかり蓋をして煮る。普通の果肉
が甘いリンゴを使う場合にはレモン果汁数滴を加えること。リンゴを煮る際に
は、白ワインを大さじ数杯だけ加えればいい。リンゴを煮るというよりは蒸気
の圧力で溶かすイメージ。

これを目の細かい網で裏漉しする。このリンゴのピュレを2\undemi{} dlにな
るまで煮詰める。充分に冷ましてから、\protect\hyperlink{mayonnaise}{マヨネー
ズ}\troisquarts{} Lを加える。風味付けにおろした(または細
かく刻んだ)レフォール大さじ1\undemi{}杯を加えて仕上げる。

\ldots{}\ldots{}このソースはとりわけ豚肉の冷製に合う。がちょうのローストの冷製にもよく合う。

\hypertarget{nota-sauce-suedoise}{%
\subparagraph{【原注】}\label{nota-sauce-suedoise}}

リンゴの時季でない場合は、リンゴのピュレの代わりに房なりの緑のグロゼイ
ユ\footnote{すぐり。ここではホワイトカラントの若どりのものを指している。}またはグーズベリー\footnote{groseilles
  à maquereau (グロゼイユザマクロー)。}のピュレ2\undemi{}
dlを固く立てたマヨネー ズ1
Lに加える。このソースはマスタードで風味付けしてもいい。
\end{recette}
\hypertarget{sauces-froides-anglaises}{%
\section[イギリス風ソース(冷製)]{\texorpdfstring{イギリス風ソース(冷製)\footnote{この節に収録されているレシピは初版から第四版まで、表現の異同は
  あるが、項目に変化はない。興味深いことに、1907年刊の英語版\emph{A
  Guide to Modern Cookery}においても全て掲載されている。}}{イギリス風ソース(冷製)}}\label{sauces-froides-anglaises}}

\frsec{Sauces Froides Anglaises}

\index{そーす@ソース!いきりすふうれいせい@イギリス風---(冷製)}
\index{いきりすふう@イギリス風!そーすおんせい@---ソース(冷製)}
\index{sauce@sauce!froides anglaises@---s froides anglaises}
\index{anglais@anglais(e)!sauces froides@sauces froides ---es}
\begin{recette}
\hypertarget{cambridge-sauce}{%
\subsubsection[ケンブリッジソース]{\texorpdfstring{ケンブリッジ\footnote{イングランド東部のケンブリッジシャーの州都。大学都市として有名。}ソース}{ケンブリッジソース}}\label{cambridge-sauce}}

\frsub{Sauce Cambridge (*Cambridge-Sauce*)}

\index{いきりすふう@イギリス風!そーすれいせい@---ソース(冷製)!けんふりつし@ケンブリッジソース}
\index{そーす@ソース!いきりすふうれいせい@イギリス風---(冷製)!けんふりつし@ケンブリッジ---}
\index{けんふりつし@ケンブリッジ!そーす@---ソース}
\index{sauce@sauce!froides anglaises@---s froides anglaises!cambridge@--- Cambridge (Cambridge-Sauce)}
\index{cambridge@Cambridge!sauce@Sauce --- (Cambridge-Sauce)}
\index{anglais@anglais(e)!sauces froides@sauces ---es froides!cambridge@Sauce Cambridge (Cambridge-Sauce)}

固茹で卵の黄身6個と、よく洗ったアンチョビのフィレ4枚、小さめのケイパー
大さじ1杯、セルフイユとエストラゴンとシブレットのみじん切りを同量ずつ
計大さじ1杯を鉢に入れてよくすり潰す。マヨネーズを作る際の要領で、マス
タード小さじ1杯、油1\undemi{}dl\footnote{マヨネーズを作る際の要領で、と表現しているのに対して油の量が少
  なく思われるが、初版は「油1 dl」、第二版以降は「1\undemi{} dl」と
  なっている。}とヴィネガー大さじ1杯を加える。カ
イエンヌごく少量で風味を引き締める。ヘラでソースを混ぜながら布で漉し
\footnote{濃度のあるソースを布で漉す方法については\protect\hyperlink{veloute}{ヴルテ}訳注参照。}、ボウルに入れる。泡立て器で軽く混ぜて滑らかにしてやり、パセリの
みじん切り小さじ1杯を加えて仕上げる。

\maeaki

\hypertarget{cumberland-sauce}{%
\subsubsection{カンバーランドソース}\label{cumberland-sauce}}

\frsub{Sauce Cumberland} (\emph{Cumberland-Sauce}\footnote{イングランド北部の旧カウンティ(行政区分、ほぼ「州」と考えてい
  い)のひとつ。現在はウェストモーランド、ランカシャー、ヨークシャー
  の一部と統合され、カンブリアとなっている。})

\index{いきりすふう@イギリス風!そーすれいせい@---ソース(冷製)!かんはーらんと@カンバーランドソース}
\index{そーす@ソース!いきりすふうれいせい@イギリス風---(冷製)!かんはーらんと@カンバーランド---}
\index{かんはーらんと@カンバーランド!そーす@---ソース}
\index{sauce@sauce!froides anglaises@---s froides anglaises!cumberland@--- Cumberland (Cumberland-Sauce)}
\index{cumberland@Cumberland!sauce@Sauce --- (Cumberland-Sauce)}
\index{anglais@anglais(e)!sauces froides@sauces ---es froides!cumberland@Sauce Cumberland (Cumberland-Sauce)}

鍋に\protect\hyperlink{gelee-de-groseilles-a}{グロゼイユのジュレ}大さじ4杯を入れて溶かし、そこにポルト酒1
dl と細かいみじん切りにして下茹でして水気を絞ったエシャロット大さじ
\undemi{}杯、オレンジの表皮と\footnote{zeste
  ゼスト。柑橘類の硬い外皮をrâpe(ラプ)と呼ばれる器具を用いておろした場合にもこの語を用いる。}とレモンの表皮を薄く剥いてごく細い千
切りにしてしっかり下茹でしてよく水気をきって冷ましたもの各大さじ1杯、
オレンジ1個の搾り汁、レモン\undemi{}個分の搾り汁、マスタード小さじ1杯、
カイエンヌごく少量、粉末の生姜少々を加える。

全体をよく混ぜる。

\ldots{}\ldots{}大型ジビエの冷製に合わせる。

\maeaki

\hypertarget{gloucester-sauce}{%
\subsubsection{グロスターソース}\label{gloucester-sauce}}

\frsub{Sauce Gloucester} (\emph{Gloucester-Sauce}\footnote{イングランド南部、グロースターシャーの州都。})

\index{いきりすふう@イギリス風!そーすれいせい@---ソース(冷製)!くろすたー@グロスターソース}
\index{そーす@ソース!いきりすふうれいせい@イギリス風---(冷製)!くろすたー@グロスター---}
\index{くろすたー@グロスター!そーす@---ソース}
\index{sauce@sauce!froides anglaises@---s froides anglaises!gloucester@--- Gloucester (Gloucester-Sauce)}
\index{gloucester@Gloucester!sauce@Sauce --- (Gloucester-Sauce)}
\index{anglais@anglais(e)!sauces froides@sauces ---es froides!gloucester@Sauce Gloucester (Gloucester-Sauce)}

ごく固く立てた\protect\hyperlink{mayonnaise}{マヨネーズ}1
Lに、レモン\undemi{}個分の搾 り汁を加えたサワークリーム2
dlと、細かく刻んだフェンネル1つまみ、ダー ビーソース\footnote{初版と第二版は「ウスターシャソース数滴」、第三版は「エスコフィ
  エソース数滴」となっている。ダービーソースについては\protect\hyperlink{sauce-russe-froide}{ロシア風ソー
  ス}訳注も参照のこと。}大さじ2杯を加える。

\ldots{}\ldots{}主として肉の冷製料理に合わせる。

\maeaki

\hypertarget{mint-sauce}{%
\subsubsection{ミントソース}\label{mint-sauce}}

\frsub{Sauce Menthe} (\emph{Mint-Sauce})

\index{いきりすふう@イギリス風!そーすれいせい@---ソース(冷製)!みんと@ミントソース}
\index{そーす@ソース!いきりすふうれいせい@イギリス風---(冷製)!みんと@ミント---}
\index{みんと@ミント!そーす@---ソース}
\index{sauce@sauce!froides anglaises@---s froides anglaises!menthe@--- Menthe (Mint-Sauce)}
\index{menthe@menthe!sauce@Sauce --- (Mint-Sauce)}
\index{anglais@anglais(e)!sauces froides@sauces ---es froides!menthe@Sauce Menthe (Mint-Sauce)}

ミントの葉50
gをごく細い千切りか、みじん切りにする。これをボウルに入れて、白いカソナード\footnote{通常cassonadeすなわち粗糖は褐色のものが多い。}かパウダーシュガー25
gとヴィネガー1\undemi{}
dl、塩1つまみ、水大さじ4杯を加える。全体によく混ぜること。

\ldots{}\ldots{}仔羊\footnote{本書で仔羊agneau(アニョー)と言う場合はほぼ例外なく乳呑仔羊、
  agneau de lait(アニョードレ)を意味する。現代は仔羊という語の意味
  する範囲が広くなり、牧草および飼料によりある程度まで肥育した羊の赤
  身肉も「仔羊」として扱うが、乳呑仔羊は白身肉なので注意。}の温製、冷製に添える。

\maeaki

\hypertarget{oxford-sauce}{%
\subsubsection{オックスフォードソース}\label{oxford-sauce}}

\frsub{Sauce Oxford} (\emph{Oxford\footnote{イングランド東部、オックスフォードシャーの州都。英語圏では最古
  の大学であるオックスフォード大学を中心とした学園都市として有名。}
Sauce})

上述の\protect\hyperlink{cumberland-sauce}{カンバーランドソース}と同様に作るが、以下の2点を変更する\footnote{オレンジとレモンの皮の扱いと量を変えただけで別のソースとして扱
  うことに疑問はあるが、これについては初版から一貫してまったく説明が
  ない。何らかのエピソードがこれらのソース名にはあったと思われるが不明。}。

\begin{enumerate}
\def\labelenumi{\arabic{enumi}.}
\item
  オレンジとレモンの外皮は千切りにするのではなく、器具を用いておろすこと。
\item
  その量は半分にする。つまり、おろした外皮はそれぞれ大さじ\undemi{}杯にすること。
\end{enumerate}

\ldots{}\ldots{}用途はカンバーランドソースと同じ。

\maeaki

\hypertarget{cold-horseradish-sauce}{%
\subsubsection{ホースラディッシュソース}\label{cold-horseradish-sauce}}

\frsub{Sauce Raifort} (\emph{Cold horseradish sauce})

\index{いきりすふう@イギリス風!そーすれいせい@---ソース(冷製)!ほーすらていつしゆ@ホースラディッシュソース}
\index{そーす@ソース!いきりすふうれいせい@イギリス風---(冷製)!ほーすらていつしゆ@ホースラディッシュ---}
\index{ほーすらていつしゆ@ホースラディッシュ!そーす@---ソース}
\index{sauce@sauce!froides anglaises@---s froides anglaises06!raifort@--- Raifort (Cold horseradish sauce)}
\index{raifort@raifort!sauce@Sauce --- (Cold horseradish sauce)}
\index{anglais@anglais(e)!sauces froides@sauces ---es froides!raifort@Sauce Raifort (Cold horseradish sauce)}

陶製の器に、マスタード大さじ1杯、細かくおろしたレフォール50
g、パウダーシュガー50 g、塩1つまみ、生クリーム5
dl、牛乳に浸してからよく圧したパンの身250
g、ヴィネガー大さじ2杯を入れて混ぜ合わせる。

\ldots{}\ldots{}このソースは牛肉のブイイ\footnote{茹で肉、の意。肉単体あるいは野菜とともに煮たもので、もとはブイ
  ヨンをとった残りだったが、17世紀にこの食べ方が流行し、料理名として
  残った。}やローストに合わせる。よく冷やしてから供すること。

\hypertarget{ux539fux6ce8-2}{%
\subparagraph{【原注】}\label{ux539fux6ce8-2}}

ソースにヴィネガーを加えるのは作業の最後にすること。
\end{recette}