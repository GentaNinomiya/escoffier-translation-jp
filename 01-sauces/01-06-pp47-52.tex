\hypertarget{ux51b7ux88fdux30bdux30fcux30b9}{%
\section{冷製ソース}\label{ux51b7ux88fdux30bdux30fcux30b9}}

\hypertarget{sauces-froides}{%
\subsection{Sauces Froides}\label{sauces-froides}}

\index{sauce@sauce!sauces froides@sauces froides}
\index{そーす@ソース!れいせいそーす@冷製ソース}
\begin{recette}
\hypertarget{ux30a2ux30a4ux30e8ux30ea2-ux30d7ux30edux30f4ux30a1ux30f3ux30b9ux30d0ux30bfux30fc}{%
\subsubsection[アイヨリ /
プロヴァンスバター]{\texorpdfstring{アイヨリ\footnote{ailloliとも綴るが、
  ail(にんにく)+
  oil(油)の合成語。19世前半紀には既にアカデミーフランセージの辞書に収録されており、広く知られていたようだ。ブイヤベースに添えるルイユとよく似ているが、ルイユがカイエンヌを加えるのに対して、こちらはにんにくと油、塩、レモン汁と少々の水だけで作る。用途も、茹でた塩鱈やじゃがいも、茹で卵、アーティチョーク、さやいんげん、などに合わせることが多い。}
/
プロヴァンスバター}{アイヨリ / プロヴァンスバター}}\label{ux30a2ux30a4ux30e8ux30ea2-ux30d7ux30edux30f4ux30a1ux30f3ux30b9ux30d0ux30bfux30fc}}

\hypertarget{sauce-aioli}{%
\paragraph{Sauce Aïoli, ou Beurre de Provence}\label{sauce-aioli}}

\index{そーす@ソース!れいせい@冷製---!あいより@アイヨリ}
\index{そーす@ソース!れいせい@冷製---!ふろふあんすはたー@プロヴァンスバター}
\index{あいより@アイヨリ}
\index{ふろふあんす@プロヴァンス!ふろふあんすはたー@プロヴァンスバター}
\index{はたー@バター!ふろふあんすはたー@プロヴァンスバター}
\index{sauce@sauce!sauce froide@sauce froide!aioli@--- Aïoli}
\index{sauce@sauce!sauce froide@sauce froide!beurre de provence@Beurre de Provence}
\index{aioli@Aïoli!sauce@Sauce ---}
\index{provence@Provence!Beurre de Provence (Aïoli)}
\index{beurre@beurre!beurre de provence@Beurre de Provence (Aïoli)}

にんにく4片(30 g)を鉢\footnote{この種の作業には、大理石製のものが伝統的によく用いられる。。}に入れて細かくすり潰す。ここに生の卵黄1個、塩1つまみを加える。混ぜながら、2\undemi{}
dlの油\footnote{原書ではとくに言及されていないが、プロヴァンス地方ではオリーブオイルを用いることが一般的。}を初めは1滴ずつ加えていき、ソースがまとまりはじめたら糸を垂らすようにして加える。この作業は鉢に入れたままで、棒をはげしく動かして行なう。

攪拌する作業の途中、レモン1個分の搾り汁と冷水大さじ\undemi{}杯を少しずつ加えて、ソースが固くなり過ぎないようにしてやること。

\hypertarget{ux539fux6ce8}{%
\subparagraph{【原注】}\label{ux539fux6ce8}}

このアイヨリソースが分離してしまいそうな時は、卵黄をさらに1個足して、
マヨネーズと場合と同様に修正すること。

\maeaki

\hypertarget{ux30a2ux30f3ux30c0ux30ebux30b7ux30a25ux98a8ux30bdux30fcux30b9}{%
\subsubsection[アンダルシア風ソース]{\texorpdfstring{アンダルシア\footnote{いうまでもなくスペインのアンダルシア地方のことだが、トマトやオリーブオイル、チョリソなどこの地方を「想起」させる食材が使われている料理などがこの名称になっている傾向がある。ところが、トマトにしろオリーブオイルにしろアンダルシア地方特有というわけではなく、アンダルシアが産地として有名なチョリソくらいしか、料理名の根拠となり得るものはない。逆に言えば、アンダルシア地方の食文化との関係は、そこに用いられている食材以外にはないものと考えてもいい。料理名に付けられた地方名がとりたてて根拠や由来のないものであることを示す一例。}風ソース}{アンダルシア風ソース}}\label{ux30a2ux30f3ux30c0ux30ebux30b7ux30a25ux98a8ux30bdux30fcux30b9}}

\hypertarget{sauce-andalouse}{%
\paragraph{Sauce Andalouse}\label{sauce-andalouse}}

\index{そーす@ソース!れいせい@冷製---!あんたるしあふう@アンダルシア風---}
\index{あんたるしあ@アンダルシア!そーす@---風ソース}
\index{そーす@ソース!あんたるしあふう@アンダルシア風---}
\index{sauce@sauce!sauce froide@sauce froide!Andalouse@--- Andalouse}
\index{sauce@sauce!andalouse@--- Andalouse}
\index{andalous@Andalous(e)!sauce@Sauce Andalouse}

ごく固く仕上げた\protect\hyperlink{mayonnaise}{ソース・マヨネーズ}\troisquarts{}
Lに、上等な赤いトマトピュレ2\undemi{}dlを加える。小さなさいの目に切ったポワヴロン\footnote{Poivron
  いわゆる日本で青果として輸入されているパプリカ(肉厚の辛くないピーマン)とほぼ同じものだが、香辛料として用いられる粉末のパプリカと混同を避けるため、あえてフランス語をそのままカタカナに訳した。}75
gを仕上げに加える。

\maeaki

\hypertarget{ux30bdux30fcux30b9ux30dcux30d8ux30dfux30a2ux306eux5a18}{%
\subsubsection{ソース・ボヘミアの娘}\label{ux30bdux30fcux30b9ux30dcux30d8ux30dfux30a2ux306eux5a18}}

\hypertarget{sauce-bohemienne}{%
\paragraph[Sauce Bohémienne]{\texorpdfstring{Sauce Bohémienne\footnote{アイルランド出身の作曲家マイケル・ウィリアム・バルフェMichael
  William Balfe (1808〜1870)のオペラ\emph{The Bohemien
  Girl}『ボヘミアの少女』のフランス語版タイトル\href{https://archive.org/details/labohmiennegrand00balf}{\emph{La
  Bohémienne}}『ラボエミエーヌ』にちなんだものと言われている。この作品はロンドンで1843年初演、1862年に四幕形式のフランス語版がパリのオペラ=コミック劇場で上演され、大ヒットしたという。この名を冠した料理はいくつかあるが、いずれもチェコのボヘミア地方とは何の関連性も認められないため、オペラの人気作品にあやかった料理名と考えるのが妥当だろう。}}{Sauce Bohémienne}}\label{sauce-bohemienne}}

\index{そーす@ソース!れいせい@冷製---!ほへみあのむすめ@---ボヘミアの娘}
\index{ほへみあ@ボヘミア!そーす@ソース・---の娘}
\index{そーす@ソース!ほへみあ@---・ボヘミアの娘}
\index{sauce@sauce!sauce froide@sauce froide!bohemienne@--- Bohémienne}
\index{sauce@sauce!bohemienne@--- Bohémienne}
\index{bohemien@bohémien(ne)!sauce@Sauce Bohémienne}

陶製の容器に、濃厚でよく冷やした\protect\hyperlink{sauce-bechamel}{ベシャメルソース}1\undemi{}
dlと卵黄4個、塩10 g、こしょう少々、ヴィネガー数滴を入れる。

泡立て器で全体をよく混ぜ、標準的なマヨネーズを作るのとまったく同じ要領で、油1
Lとエストラゴンヴィネガー大さじ2杯程を加える。

\ldots{}\ldots{}仕上げに、マスタード大さじ1杯を加える。

\maeaki

\hypertarget{ux30bdux30fcux30b9ux30b7ux30e3ux30f3ux30c6ux30a3ux30a47}{%
\subsubsection[ソース・シャンティイ]{\texorpdfstring{ソース・シャンティイ\footnote{パリ近郊の地名。詳しくはホワイト系派生ソースの\protect\hyperlink{sauce-chantilly}{ソース・シャンティイ}訳注参照。}}{ソース・シャンティイ}}\label{ux30bdux30fcux30b9ux30b7ux30e3ux30f3ux30c6ux30a3ux30a47}}

\hypertarget{sauce-chantilly-froide}{%
\paragraph{Sauce Chantilly}\label{sauce-chantilly-froide}}

\index{そーす@ソース!れいせい@冷製---!しやんていい@---・シャンティイ}
\index{しやんていい@シャンティイ!そーす@ソース・---(冷製)}
\index{そーす@ソース!しやんていい@---・シャンティイ}
\index{sauce@sauce!sauce froide@sauce froide!chantilly@--- Chantilly}
\index{sauce@sauce!chantilly@--- Chantilly (froide)}
\index{chantilly@Chantilly!sauce@Sauce --- (froide)}

酸味付けにレモンを用いて、固く仕上げた\protect\hyperlink{mayonnaise}{ソース・マヨネーズ}\troisquarts{}
Lを用意しておく。提供直前に、ごく固く泡立てた生クリーム大さじ4杯\footnote{大さじ1杯=15ccという概念にとらわれないよう注意。原文は、大きなスプーンで泡立てた生クリームをざっくりと4回加えるイメージで書かれている。本書における通常のソースの仕上り量が約1
  Lであることを考慮すると、最低でも100ml以上は加えることになるだろう。}を加える。その後、味を\ruby{調}{ととの}える。

\ldots{}\ldots{}もっぱら、アスパラガスの冷製、温製に添える。

\hypertarget{ux539fux6ce8-1}{%
\subparagraph{【原注】}\label{ux539fux6ce8-1}}

生クリームを加えるのは、このソースを使うまさにその時にすること。前もっ
て加えておくと、ソースが分離してしまう恐れがあるので注意。

\maeaki

\hypertarget{ux30b8ux30a7ux30ceux30f4ux30a1ux98a812ux30bdux30fcux30b9}{%
\subsubsection[ジェノヴァ風ソース]{\texorpdfstring{ジェノヴァ風\footnote{あまり明確な由来はないが、ジェノヴァが地中海に面した港町であり、このソースが魚料理用であるという点で一応の説明はつくだろう。}ソース}{ジェノヴァ風ソース}}\label{ux30b8ux30a7ux30ceux30f4ux30a1ux98a812ux30bdux30fcux30b9}}

\hypertarget{sauce-genoise-froids}{%
\paragraph{Sauce Génoise}\label{sauce-genoise-froids}}

\index{そーす@ソース!れいせい@冷製---!しえのうあふう@ジェノヴァ風---}
\index{しえのうあふう@ジェノヴァ風!そーす@ソース・---(冷製)}
\index{そーす@ソース!しえのうあふう@ジェノヴァ風---}
\index{sauce@sauce!sauce froide@sauce froide!genoise@--- Génoise}
\index{sauce@sauce!genoise@--- Génoise (froide)}
\index{genois@Génois(e)!sauce@Sauce ---e (froide)}

殻と皮を剥いたばかりのピスタチオ40 gと、松の実25
g、松の実がない場合はスイートアーモンド20
gを鉢に入れてよくすり潰し、冷めた\protect\hyperlink{sauce-bechamel}{ベシャメルソース}小さじ1杯程度を加えて練ってペースト状にする。これを目の細かい網で裏漉しする。陶製の容器に卵黄6個、塩1つまみ、こしょう少々を入れる。泡立て器でよく混ぜる。油1
Lと中位の大きさのレモン2個の搾り汁を少しずつ加えてよく混ぜて乳化させていく\footnote{明記されていないが、ソースをしっかりと乳化させるためには\protect\hyperlink{mayonnaise}{マヨネーズ}と同様に作業すること。}。仕上げにハーブのピュレ大さじ3杯を加える。これは、パセリの葉とセルフイユ、エストラゴン、時季が合えばサラダバーネットを同量ずつ用意し、強火で2分間下茹でしてから湯をきり、冷水にさらしてから水気を強く絞り、裏漉しして作っておく。

\ldots{}\ldots{}冷製の魚料理全般に合わせられる。

\maeaki

\hypertarget{ux30bdux30fcux30b9ux30b0ux30eaux30d3ux30c3ux30b7ux30e5}{%
\subsubsection{ソース・グリビッシュ}\label{ux30bdux30fcux30b9ux30b0ux30eaux30d3ux30c3ux30b7ux30e5}}

\hypertarget{sauce-gribiche13}{%
\paragraph[Sauce Gribiche]{\texorpdfstring{Sauce Gribiche\footnote{由来不明の語。ノルマンディ方言で「子どもを怖がらせるおばさん」
  の意味で用いられるということが分かっているのみ。19世紀後半以降に創
  案もしくは一般化したソースと思われる。本書初版には当然のように既に
  収録されており、その後の大きな異同もない。ただ、本書初版以前に出版
  された料理書においてこのソースのレシピはまだ見つかっていない。ファー
  ヴルは1905年刊『料理および食品衛生事典』第二版で「ある種のレムラー
  ドにレストランで付けられた名称」と定義し、掲載しているレシピは本書
  初版のものと大差ないが、「ウスターシャソース少々も加える」となって
  いるところが目を引く。また、1913年初版のプルーストの長編小説『失な
  われた時を求めて』の「スワン家の方へ」冒頭において「彼(=スワン)を
  招いていない夕食会のために、ソース・グリビッシュやパイナップルのサ
  ラダのレシピが必要になるや、ためらいもなく探しに行かせたりするのだっ
  た」(p.18)。もしこの語り手の記述が正確であるなら、19世紀末には広く
  知られたものであったと考えるべきだが、小説の場合は必ずしも歴史的事
  実と符号するわけではないので注意が必要。}}{Sauce Gribiche}}\label{sauce-gribiche13}}

\index{そーす@ソース!れいせい@冷製---!くりひつしゆ@---・グリビッシュ}
\index{くりひつしゆ@グリビッシュ!そーす@ソース・---(冷製)}
\index{そーす@ソース!くりひつしゆ@---・グリビッシュ}
\index{sauce@sauce!sauce froide@sauce froide!gribiche@--- Gribiche}
\index{sauce@sauce!gribiche@--- Gribiche (froide)}
\index{gribiche@!gribiche!sauce@Sauce --- (froide)}

茹であがったばかりの固茹で卵の黄身6個を陶製のボウルに入れ、マスタード
小さじ1杯、塩1つまみ強、こしょう適量を加えてよく練り、滑らかなペースト
状にする。植物油\undemi{} Lとヴィネガー大さじ1\undemi{}杯を加えながら
よく混ぜて乳化させる。仕上げに、コルニションとケイパーのみじん切り計 100
gと、パセリとセルフイユ、エストラゴンのみじん切りのミックスを大さ
じ1杯、短かめの千切りにした固茹で卵の白身3個分を加える。

\ldots{}\ldots{}冷製の魚料理に添えるのが一般的。

\maeaki

\hypertarget{ux30ecux30d5ux30a9ux30fcux30ebux98a8ux5473ux306eux30bdux30fcux30b9ux30b0ux30edux30bcux30a4ux30e6}{%
\subsubsection{レフォール風味のソース・グロゼイユ}\label{ux30ecux30d5ux30a9ux30fcux30ebux98a8ux5473ux306eux30bdux30fcux30b9ux30b0ux30edux30bcux30a4ux30e6}}

\hypertarget{sauce-groseilles-au-raifort}{%
\paragraph{Sauce Groseilles au
Raifort}\label{sauce-groseilles-au-raifort}}

\index{そーす@ソース!れいせい@冷製---!れふおーるふうみくろせいゆ@レフォール風味の---・グロゼイユ}
\index{くろせいゆ@グロゼイユ!そーすれふおーる@レフォール風味のソース・---(冷製)}
\index{れふおーる@レフォール!そーすくろせいゆ@---風味のソース・グロゼイユ}
\index{そーす@ソース!れふおーるふうみくろせいゆ@レフォール風味の---・グロゼイユ}
\index{sauce@sauce!sauce froide@sauce froide!grroseilles raifort@--- Grroseilles au Rifort}
\index{sauce@sauce!groseille@--- Groseilles au Raifort (froide)}
\index{raiforg@raifort!sauce@Sauce Groseilles au --- (froide)}
\index{groseille@!groseille!sauce@Sauce --- au Raifort (froide)}

ポルト酒1
dlにナツメグ、シナモン、塩、こしょう各1つまみを加え、を\deuxtiers{}量まで煮詰める。溶かした\protect\hyperlink{}{グロゼイユのジュレ}4
dlと細かくすりおろしたレフォール大さじ2杯を加える。

(さまざまな用途に使える)

\maeaki

\hypertarget{ux30a4ux30bfux30eaux30a2ux98a8ux30bdux30fcux30b9}{%
\subsubsection{イタリア風ソース}\label{ux30a4ux30bfux30eaux30a2ux98a8ux30bdux30fcux30b9}}

\hypertarget{sauce-italienne-froide}{%
\paragraph{Sauce Italienne}\label{sauce-italienne-froide}}

\index{そーす@ソース!れいせい@冷製---!いたりあふう@イタリア風---}
\index{いたりあふう@イタリア風!そーす@---ソース(冷製)}
\index{そーす@ソース!いたりあふうれいせい@イタリア風---(冷製)}
\index{sauce@sauce!sauce froide@sauce froide!italienne@--- Italienne}
\index{sauce@sauce!italienne@--- Italienne (froide)}
\index{italien@italien(ne)!sauce froide@Sauce ---ne (froide)}

仔牛の脳半分を、香草を効かせたクールブイヨンで火を通し、目の細かい網で
裏漉しする。同量の牛あるいは羊の脳でもいい。

裏漉ししたピュレを陶製の器に入れ、泡立て器で滑らかになるまで混ぜる。卵黄5個と塩10
g、こしょう1つまみ強、油1
Lとレモン果汁1個分でマヨネーズを作り、そこの脳のピュレを加える。パセリのみじん切り大さじ1杯強を加えて仕上げる。

\ldots{}\ldots{}このソースなどんな冷製の肉料理にも合う。

\hypertarget{ux30deux30e8ux30cdux30fcux30ba}{%
\subsubsection{マヨネーズ}\label{ux30deux30e8ux30cdux30fcux30ba}}

\hypertarget{mayonnaise}{%
\paragraph[Sauce Mayonnaise]{\texorpdfstring{Sauce Mayonnaise\footnote{このソース名の語源には諸説あり、未だ定説と呼べるものはない。
  Mayonnaise という綴りそのものは1806年のヴィアール『帝国料理の本』が初
  出で、Saumon à la Mayonnaise, Filet de Sole en Mayonnaise, Poulet en
  Mayonnaise の3つのレシピが掲載されている。そのうちのひとつ、サーモンの
  マヨネーズは、筒切りにしたサーモンを茹でて冷まし、ジュレを混ぜたマヨネー
  ズをかける、という内容であり、ソースについてはマヨネーズの項を参照となっ
  ているが、どういうわけかこの本にマヨネーズそのもののレシピはない。また、
  「鶏のマヨネーズ仕立て」におけるソースはどう見てもこんにち我々が理解し
  ているマヨネーズとまったく違い、鶏のゼラチン質を冷し固める要素として利
  用したものだ。同じヴィアールの改訂版ともいうべき『王国料理の本』(1822
  年)にはマヨネーズのレシピが掲載されている。興味深いことに「このソース
  にはいろいろな作り方がある。生の卵黄を使うもの、ジュレを使うもの、仔牛
  のグラスを使うものや仔牛の脳を使うもの」として、もっとも一般的な方法と
  して生の卵黄を使う方法が示されている。生の卵黄に攪拌しながら少しずつ油
  を加えていき、固くなってきたらヴィネガー少々を加えてコシをきる、という
  方法であり、こんにち我々の知るマヨネーズに非常に近いものとなっている。
  綴りについては、カレームはmagner(マニェ)捏ねる、という意味の動詞から
  派生したものだとして、magnonnaiseもしくはmagnionnaiseと綴るべきだと
  『パリ風料理の本』で力説している。グリモ・ド・ラ・レニェールは中世フラ
  ンス語で卵黄を意味するmoyeuの派生語としてmoyeunnaiseという綴りを使って
  いる。そのほかフランス大西洋岸の地名バイヨンヌの形容詞bayonnais(バヨ
  ネ)が語源だという説もある。綴りの起源についてある程度有力視されている
  のは、1756年にリシュリュー公爵が当時イギリスに占領されていたミノルカ島
  のマオン港 Mahon を奪取したことにちなんで、mahonnaise と名づけられたと
  いうもの。もっとも、卵黄とヴィネガーを植物油で乳化させたソースという点
  では、beurre de Provenceが1758年刊マラン『コモス神の贈り物』にPigeons,
  au beurre de Provence鳩のプロヴァンスバター添え、というレシピが掲載さ
  れている(t.2,
  pp.290-230)。これは本書『料理の手引き』における\protect\hyperlink{aioli}{アイヨリ
  / プロヴァンスバター}の作り方にやや近く、茹でたにんにくを鉢に
  入れてよくすり潰し、塩、こしょう、ケイパー、アンチョビを加えてさらにす
  り潰し、そこに油を加えて攪拌して濃度を出させる、つまり乳化させる、とい
  うもの。また、植物油ではなくバターを用いるものとして、\protect\hyperlink{sauce-hollandaise}{オランデーズソー
  ス}の原型ともいうべきレシピが1651年のラ・ヴァレー
  ヌ『フランス料理の本』に、Asperges à la Sauce blanche アスパラガスのホ
  ワイトソース添え(p.238)として掲載されていることや、卵黄をポタージュや
  ラグーのとろみ付けに使うことが古くから行なわれていたことなどを総合する
  と、良質のオリーブオイルやひまわり油を利用しやすい環境にある南フランス
  の方がどちらかといえば、卵黄と植物油の乳化作用を利用したソースの発達、
  普及しやすい環境にあったと想像される。なお、この『料理の手引き』では卵
  黄のみを用いたレシピとなっているが、全卵を用いる場合もある。日本の市販
  品でも卵黄のみを使うメーカーと全卵を使用しているメーカーが混在している。
  全卵を用いた場合、当然ながら黄色というより白に近い色合いに仕上がる。な
  お、マヨネーズの仕上りは、卵黄のみか全卵を用いるかという問題もあるが、
  どのような植物油を使うかにも大きく左右されるので注意。}}{Sauce Mayonnaise}}\label{mayonnaise}}

\index{そーす@ソース!れいせい@冷製---!まよねーす@マヨネーズ}
\index{まよねーす@マヨネーズ}
\index{そーす@ソース!まよねーす@マヨネーズ}
\index{sauce@sauce!sauce froide@sauce froide!mayonnaise@--- Mayonnaise}
\index{sauce@sauce!italienne@--- Mayonnaise}
\index{mayonnaise@!mayonnaise!sauce@Sauce Mayonnaise}

冷製ソースのほとんどはマヨネーズの派生ソースだから、\protect\hyperlink{sauce-espagnole}{ソース・エスパニョ
ル}や\protect\hyperlink{veloute}{ヴルテ}と同様に基本ソースと見なされる。
マヨネーズの作り方はきわめてシンプルだが、以下に述べるポイントはしっか
り頭に入れておく必要がある。

\hypertarget{ux6750ux6599ux3068ux5206ux91cf}{%
\subparagraph{材料と分量}\label{ux6750ux6599ux3068ux5206ux91cf}}

\ldots{}\ldots{}卵黄6個、「からざ」は取り除いておくこと。油1 L。塩 10
g、白こしょう1
g、ヴィネガー大さじ1\undemi{}杯または、より白い仕上りを目指す場合にはヴィネガーと同等量のレモン果汁。

\begin{enumerate}
\def\labelenumi{\arabic{enumi}.}
\item
  塩、こしょう、ヴィネガーまたはレモン果汁ほんの少々を加えて、泡立て器で卵黄を溶く。
\item
  油を最初は1滴ずつ加えていき、滑らかにまとまっり始めたら、糸を垂らすようにして油を加えていく。
\item
  何回かに分けてヴィネガーもしくはレモン果汁を少量ずつ加え、コシを切ってやること\footnote{原文
    rompre le corps de la sauce ソースの粘り気をヴィネガーなど
    を加えることで「ゆるめる」あるいは「のばす」こと。ここでは「コシを
    きる」と訳したが、日本の調理用語もしくは調理現場のみで通用する用語。
    この作業は、一見乳化したように見えてもまだ不完全な場合があるため、
    いったん濃度を下げてさらに攪拌することで乳化をさらに促進させ安定し
    たものにするのが目的。}。
\item
  最後に熱い湯を大さじ3杯加える。これは乳化をしっかりさせて、作り置きしておく必要がある場合でもソースが分離しないようにするため。
\end{enumerate}

\hypertarget{ux539fux6ce8-2}{%
\subparagraph{【原注】}\label{ux539fux6ce8-2}}

\noindent 1.
卵黄だけの段階で塩こしょうをするとソースが分離してしまうのではないかというのは思い込みに過ぎず、実際に調理現場で作業している者はそう考えていない。むしろ、塩を卵黄の水分に溶かし込んでおいた方が、卵黄がまとまりやすくなることは科学的に証明されている\footnote{当時の知見であることに注意。ただ、塩が植物油に溶けにくいことは自明であるから、この方法そのものは正しいと言える。}。

\begin{enumerate}
\def\labelenumi{\arabic{enumi}.}
\setcounter{enumi}{1}
\item
  マヨネーズを作る際に、氷の上に容器を置いて作業するも間違いだ。事実はまったく逆で、冷気が伝わることがもっとも分離させてしまいやすい原因だ。寒い季節には、油はやや微温めか、せめて厨房の室温くらいにするべきだ\footnote{オリーブオイルのように、飽和温度が高い種類の油ではよく見られる現象。ひまわり油でさえも寒さで濁るので、この指摘は正しい。}。
\item
  マヨネーズが分離してしまう原因としては\ldots{}\ldots{}

  \begin{enumerate}
  \def\labelenumii{\arabic{enumii}.}
  \item
    最初に油を入れ過ぎてしまうこと。
  \item
    冷え過ぎた油を使うこと。
  \item
    卵黄の量に対して油の量が多過ぎること。卵黄1個につき油を乳化させることが出来るのは、作り置きする場合は1\troisquarts{}
    dl、すぐに使うのであれば2 dlが限度。
  \end{enumerate}
\end{enumerate}
\end{recette}