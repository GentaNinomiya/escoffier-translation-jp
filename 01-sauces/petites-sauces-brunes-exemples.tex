\documentclass[twoside,12Q,b5paper]{escoffierltjsbook}
\usepackage{amsmath}%数式
\usepackage{amssymb}
\usepackage[no-math]{fontspec}
%\usepackage{xunicode}


\usepackage[unicode=true]{hyperref}
\hypersetup{breaklinks=true,
             bookmarks=true,
             pdfauthor={},
             pdftitle={},
             colorlinks=true,
             citecolor=blue,
             urlcolor=blue,
             linkcolor=magenta,
             pdfborder={0 0 0}}
\urlstyle{same}

%%欧文フォント設定
\setmainfont[Ligatures=TeX,Scale=1.0]{Linux Libertine O}

%%Garamond
%\usepackage{ebgaramond-maths}
%\setmainfont[Ligatures=TeX,Scale=1.0]{EB Garamond}%fontspecによるフォント設定


%\setmainfont[Ligatures=TeX]{TeX Gyre Pagella}%ギリシャ語を用いる場合はこちら
%\setsansfont[Scale=MatchLowercase]{TeX Gyre Heros}  % \sffamily のフォント
\setsansfont[Ligatures=TeX, Scale=1]{Linux Biolinum O}     % Libertine/Biolinum
\setmonofont[Scale=MatchLowercase]{Inconsolata}       % \ttfamily のフォント

\usepackage[cmintegrals,cmbraces]{newtxmath}%数式フォント

\usepackage{luatexja}
\usepackage{luatexja-fontspec}
%\ltjdefcharrange{8}{"2000-"2013, "2015-"2025, "2027-"203A, "203C-"206F}
%\ltjsetparameter{jacharrange={-2, +8}}
\usepackage{luatexja-ruby}

%%%%和文仮名プロポーショナル
\usepackage[hiragino-pron,expert,deluxe]{luatexja-preset}
%\usepackage[ipaex,expert,deluxe]{luatexja-preset}
%\newopentypefeature{PKana}{On}{pkna} % "PKana" and "On" can be arbitrary string
%\setmainjfont[
%    JFM=prop,PKana=On,Kerning=On,
%    BoldFont={YuMincho-DemiBold},
%    ItalicFont={YuMincho-Medium},
%    BoldItalicFont={YuMincho-DemiBold}
%]{YuMincho-Medium}
%\setsansjfont[
%    JFM=prop,PKana=On,Kerning=On,
%    BoldFont={YuGothic-Bold},
%    ItalicFont={YuGothic-Medium},
%    BoldItalicFont={YuGothic-Bold}
%]{YuGothic-Medium}
%%%%和文仮名プロプーショナルここまで

\renewcommand{\bfdefault}{bx}%和文ボールドを有効にする
\renewcommand{\headfont}{\gtfamily\sffamily\bfseries}%和文ボールドを有効にする

\defaultfontfeatures[\rmfamily]{Scale=1.2}%効いていない様子
\defaultjfontfeatures{Scale=0.92487}%和文フォントのサイズ調整。デフォルトは 0.962212 倍%ltjsclassesでは不要?
%\defaultjfontfeatures{Scale=0.962212}
%\usepackage{libertineotf}%linux libertine font %ギリシア語含む
%\usepackage[T1]{fontenc}
%\usepackage[full]{textcomp}
%\usepackage[osfI,scaled=1.0]{garamondx}
%\usepackage{tgheros,tgcursor}
%\usepackage[garamondx]{newtxmath}
\usepackage{xfrac}

%レイアウト調整
\usepackage{layout}
%\setlength{\hoffset}{-1truein}
\setlength{\hoffset}{5mm}
\setlength{\oddsidemargin}{0pt}
\setlength{\evensidemargin}{-1cm}
%\setlength{\textwidth}{\fullwidth}%%ltjsclassesのみ有効
\setlength{\fullwidth}{13cm}
\setlength{\textwidth}{13cm}
\setlength{\marginparsep}{0pt}
\setlength{\marginparwidth}{0pt}

\def\tightlist{\itemsep1pt\parskip0pt\parsep0pt}

  
%\usepackage{fancyhdr}

%%%脚注番号のページ毎のリセット
%\makeatletter
%  \@addtoreset{footnote}{page}
%\makeatother
\usepackage[perpage,marginal,stable]{footmisc}

%レシピ本文
\usepackage{multicol}
\usepackage{setspace}

\newenvironment{recette}{\begin{small}\begin{spacing}{1}\begin{multicols}{2}}{\end{multicols}\end{spacing}\end{small}}
%\newenvironment{recette}{\begin{spacing}{1}\begin{multicols}{2}}{\end{multicols}\end{spacing}}

% PDF/X-1a
% \usepackage[x-1a]{pdfx}
% \Keywords{pdfTeX\sep PDF/X-1a\sep PDF/A-b}
% \Title{Sample LaTeX input file}
% \Author{LaTeX project team}
% \Org{TeX Users Group}

% \pdfcompresslevel=0
%\usepackage[cmyk]{xcolor}

%biblatex
%\usepackage[notes,strict,backend=biber,autolang=other,%
%                   bibencoding=inputenc,autocite=footnote]{biblatex-chicago}
%\addbibresource{hist-agri.bib}
\let\cite=\autocite

% % % % 
\date{}

\begin{document}
%%%%%\layout

%fancyhdr
%\pagestyle{fancy}
%\lhead[\thepage]{\thesection}
%\chead{}
%\rhead[\thechapter]{\thepage}
%\fancyhead{\gdef\headrulewidth{0pt}}
%\lfoot{}
%\cfoot{}
%\rfoot{}



\begin{center}
\subsection*{茶色い派生ソース PETITES SAUCES BRUNES
\index{COMPOSEES}COMPOSEES}\label{ux8336ux8272ux3044ux6d3eux751fux30bdux30fcux30b9-petites-sauces-brunes-composees}
\addcontentsline{toc}{subsection}{茶色い派生ソース PETITES SAUCES BRUNES
COMPOSEES}
\end{center}

\begin{recette}%二段組はじまり
  
\subsubsection*{\texorpdfstring{ソース・ビガラード\footnote{ビガラードは本来、南フランスで栽培されるビターオレンジの一種。}
Sauce
Bigarade}{ソース・ビガラード Sauce Bigarade}}\label{ux30bdux30fcux30b9ux30d3ux30acux30e9ux30fcux30c92-1-sauce-bigarade}
\addcontentsline{toc}{subsubsection}{ソース・ビガラード Sauce Bigarade}

\paragraph{仔鴨のブレゼ用}\label{ux4ed4ux9d28ux306eux30d6ux30ecux30bcux7528}
\addcontentsline{toc}{paragraph}{仔鴨のブレゼ用}

仔鴨をブレゼした際の煮汁を漉して浮き脂を取り除き、煮詰める。煮詰まった
らさらに目の細かい布で漉し、ソース1Lあたりオレンジ4個とレモン1個の搾り
汁でのばす。

\paragraph{\texorpdfstring{仔鴨のポワレ用\footnote{「肉料理」参照。}}{仔鴨のポワレ用}}\label{ux4ed4ux9d28ux306eux30ddux30efux30ecux75282-2}
\addcontentsline{toc}{paragraph}{仔鴨のポワレ用}

仔鴨をポワレのフォン\footnote{ここでのポワレは蒸し焼きの一種であるから、煮汁それ自体は野菜に
  含まれていた水分くらいしかない。実際には、火入れの終わった肉を取り
  出してから、鍋に適量のフォンを注いで火にかけ、残った香味野菜から風
  味を引き出したものを使う。}から浮き脂を取り除き、でんぷんで軽くとろみ付
けする。砂糖20gに大さじ1/2杯のヴィネガーを加えて火にかけカラメル状にし
たものを加える。ブレゼ用と同様に、オレンジとレモンの搾り汁でのばす。

仔鴨のブレゼ用、ポワレ用いずれの場合も、細かい千切りにしてよく下茹でし
ておいたオレンジの皮大さじ2とレモンの皮大さじ1を加えて仕上げる。

\subsubsection*{ボルドー風ソース Sauce
Bordelaise}\label{ux30dcux30ebux30c9ux30fcux98a8ux30bdux30fcux30b9-sauce-bordelaise}
\addcontentsline{toc}{subsubsection}{ボルドー風ソース Sauce Bordelaise}

赤ワイン3dlにエシャロットのみじん切り大さじ2、粗く砕いたこしょう、タ
イム、ローリエの葉1/2枚を加えて火にかけ、1/4量になるまで煮詰める。ソー
ス・エスパニョル1dlを加えて火にかけ、浮いてくる夾雑物を丁寧に取り除き
ながら弱火で15分間煮る。目の細かい布で漉す。

溶かしたグラスドヴィアンド大さじ1杯とレモン汁1/4個分、細かいさいの目
か輪切りにしてポシェしておいた牛骨髄を加えて仕上げる。

\ldots{}\ldots{}牛、羊の赤身肉のグリル用

【原注】こんにちではボルドー風ソースをこのように赤ワインを用いて作るが、
本来的には誤りである。もともとは白ワインが用いられていた。白ワインを用
いるものについては「ボルドー風ソース ボヌフォワ」として後述。

\subsubsection*{ブルゴーニュ風ソース Sauce
Bourgui-gnonne}\label{ux30d6ux30ebux30b4ux30fcux30cbux30e5ux98a8ux30bdux30fcux30b9-sauce-bourgui-gnonne}
\addcontentsline{toc}{subsubsection}{ブルゴーニュ風ソース Sauce
Bourgui-gnonne}

上質の赤ワイン11/2Lに、エシャロット5個の薄切りとパセリの枝、タイム、
ローリエの葉1/2枚、マッシュルームの切りくず\footnote{料理に使うマッシュルームは通常、トゥルネ(包丁を持った側の手は動
  かさずに材料を回して切ることからついた用語)すなわち螺旋状に切って
  供するが、その際に少なくない量の切りくずが出るのでこれを使う。}25gを加えて、半量になる
まで煮詰める。布で漉し、ブールマニエ80g(バター45gと小麦粉35g)を加え
てとろみを付ける。提供直前にバター150gを溶かし込み、カイエンヌ\footnote{赤唐辛子の粉末だが、カイエンヌは本来、品種名。日本でよく用いられ
  ているタカノツメなどと比べるとややスコビル値(辛さの指数)は低く、
  風味も異なる。}ごく 少量で加えて風味よく仕上げる。

\ldots{}\ldots{}いろいろな卵料理や、家庭料理に好適なソース。

\subsubsection*{ブルターニュ風ソース Sauce
Bretonne}\label{ux30d6ux30ebux30bfux30fcux30cbux30e5ux98a8ux30bdux30fcux30b9-sauce-bretonne}
\addcontentsline{toc}{subsubsection}{ブルターニュ風ソース Sauce
Bretonne}

中位の玉ねぎ2個をみじん切りにして、バターできつね色になるまで炒める。
白ワイン21/2dlを注ぎ、半量になるまで煮詰める。ここにソース・エスパニョ
ル31/2およびトマトソース同量を加える。7〜8分間煮立ててから、刻んだ
パセリを加えて仕上げる。

【原注】このソースは「白いんげん豆のブルターニュ風」以外にはほとんど使
われない。

\subsubsection*{\texorpdfstring{ソース・スリーズ\footnote{スリーズ
  cerises はさくらんぼのこと。このレシピでグロゼイユ(す
  ぐり)のジュレを用いるが、古くはさくらんぼを用いていたことからこの
  名称となった。} Sauce aux
cerises}{ソース・スリーズ Sauce aux cerises}}\label{ux30bdux30fcux30b9ux30b9ux30eaux30fcux30ba6-sauce-aux-cerises}
\addcontentsline{toc}{subsubsection}{ソース・スリーズ Sauce aux cerises}

ポルト酒2dlにイギリス風ミックススパイスひとつまみと、すりおろしたオレ
ンジの皮を大さじ1/2杯加えて2/3量になるまで煮詰める。グロゼイユのジュレ
21/2を加え、仕上げにオレンジ果汁を加える。

\ldots{}\ldots{}大型猟獣肉の料理用だが、鴨のポワレやブレゼにも用いられる。




\end{recette}%%2段組おわり

\end{document}
