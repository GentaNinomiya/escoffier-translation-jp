\href{原稿下準備20180414五島、連載からコピー}{} \href{訳と注釈}{}
\href{未、原文対照チェック}{} \href{未、日本語表現校正}{}
\href{未、その他修正}{} \href{未、原稿最終校正}{}

\hypertarget{ux30ddux30bfux30fcux30b8ux30e5ux3064ux304fux308aux306eux57faux790e}{%
\section{ポタージュつくりの基礎}\label{ux30ddux30bfux30fcux30b8ux30e5ux3064ux304fux308aux306eux57faux790e}}

\begin{center}
プチットマルミット、グランドブイヨン、いろいろなコンソメのクラリフィエの方法
\end{center}

\frsec{Précis des éléments nurtirifs, aromatiques}

\begin{center}

et de l'assaisonnement pour la Petite Marmite, les Grands Bouillons,
et la clarification des Consommé divers

\end{center}

\begin{recette}
\hypertarget{ux767dux3044ux30b3ux30f3ux30bdux30e1ux30b5ux30f3ux30d7ux30eb1}{%
\subsubsection[白いコンソメ・サンプル]{\texorpdfstring{白いコンソメ・サンプル\footnote{consommé
  simple「単純な(簡素な)コンソメ」の意。肉や魚、野菜を煮
  て漉しただけのもの。ここでは具体的な作業手順は記されていないが、モ
  ンタニェの『ラルース・ガストロノミーク』初版(1937年)の記述は概ね以
  下のとおり(材料はエスコフィエとほぼ同じ)。(a) 牛肉を紐で縛り、大鍋
  (陶製が良い)に入れて水7 Lを注ぐ。火にかけて沸騰したら、表面にアルブ
  ミンの軽く固まった膜が張るので、丁寧にこの膜を取り除く。鍋に野菜を
  加える。かすかに沸騰する火加減で5時間煮る。浮き脂を丁寧に取り除き、
  布または目の細かい漉し器で漉す。5時間以上煮込んではいけない。だが、
  5時間では骨に含まれているおいしさを全て抽出出来ないので、砕いた骨
  を長時間煮て第1のブイヨンをとり、これで肉と野菜を煮るようにすると
  良い。(b) 鍋に砕いた骨を入れ、水をかぶる程度注ぐ。沸騰させ、あくを
  引き、塩を加える。弱火で2時間半煮る。この「沸騰したブイヨン」に、
  骨を外して紐で縛った肉を入れる。再び沸騰させ、あくを引いて味を調え
  る。野菜を加え、弱火で約4時間煮る。塩は最初に全量を入れないこと。
  必要なら作業の最終段階でも塩を加える。}}{白いコンソメ・サンプル}}\label{ux767dux3044ux30b3ux30f3ux30bdux30e1ux30b5ux30f3ux30d7ux30eb1}}

(仕上がり 10 L分)

\begin{itemize}
\item
  主素材\ldots{}\ldots{}牛赤身肉4 kgと牛骨付きすね肉3 kg。
\item
  香味素材\ldots{}\ldots{}にんじん1.1 kg (5〜6本)、かぶ900
  g(5〜6ヶ)、ポワロー200 g、パース ニップ\footnote{panaisパネ。和名アメリカボウフウ。セリ科の根菜で、香りが良い。
    白く、にんじんに似た円錐形のため、俗に「白にんじん」と呼ばれること
    もあるが、にんじんとは別種。でんぷん質が豊富で、ピュレ等の調理にも
    適している。}200 g、玉ねぎ(中)2ヶ(200
  g)、クローブ3本、にんにく3片(20 g)、 セロリ120 g。
\item
  加える液体\ldots{}\ldots{}水14 L。
\item
  調味料\ldots{}\ldots{}粗塩70 g。
\item
  加熱時間\ldots{}\ldots{}5時間。
\end{itemize}

\hypertarget{ux4f5cux308aux65b9ux306bux95a2ux3059ux308bux88dcux8db33}{%
\paragraph[作り方に関する補足]{\texorpdfstring{作り方に関する補足\footnote{この部分は第二版で加筆された。}}{作り方に関する補足}}\label{ux4f5cux308aux65b9ux306bux95a2ux3059ux308bux88dcux8db33}}

\ldots{}\ldots{}コンソメ・サンプルを作る際、一般的には5時間かけて煮ることになっている。
肉汁を抽出するには充分な時間である。

しかし、骨の組織を壊して可溶性物質を確実に抽出するには5時間では絶対に
足りない。骨から可溶性物質を抽出することはとても重要だが、そのためには
弱火で12〜15時間煮る必要がある。

だから、グランド・キュイジーヌでは、粗く砕いた骨を12時間以上煮て第1の
コンソメをとるようになってきている。

この第1のコンソメを第2のコンソメをとる鍋に注ぐ。この鍋で肉を約4時間、
すなわち肉を煮るのに最低限必要な時間、火にかける。

肉を野菜を塊のままではなく細かく刻めば、2つめの作業時間をさらに短かく
することも可能だ。その場合は、通常のクラリフィエと同様の作業となる。
(「クラリフィエ」の項参照)。
\end{recette}
\hypertarget{ux30afux30e9ux30eaux30d5ux30a3ux30a84}{%
\subsection[クラリフィエ]{\texorpdfstring{クラリフィエ\footnote{原文
  clarifications クラリフィカシオン (動詞 clarifier「澄ませ
  る」の名詞形)。字義通りには「澄ませる作業」だが、実際にはコンソメ・
  ドゥーブルconsommé double (コンソメ・リッシュ consommé richeコンソ
  メ・クラリフィエ consommé clarifiéとも呼ばれる)を作ることを意味す
  る。本来はその工程のひとつであった「澄ませる作業」が作業全体を指す
  語として定着したのだろう。}}{クラリフィエ}}\label{ux30afux30e9ux30eaux30d5ux30a3ux30a84}}

\frsecb{Clarifications}
\begin{recette}
\hypertarget{ux901aux5e38ux306eux30b3ux30f3ux30bdux30e1}{%
\subsubsection{通常のコンソメ}\label{ux901aux5e38ux306eux30b3ux30f3ux30bdux30e1}}

(仕上がり4 L分)

\begin{itemize}
\item
  白いコンソメ・サンプル\ldots{}\ldots{}5 L。
\item
  主素材\ldots{}\ldots{}牛赤身肉1.5
  kg。丁寧に筋を除き、挽いておく\footnote{原文 hacher
    アシェ(細かく刻む)。語源は hacheアーシュ(斧)。日本
    語の「刻む」は包丁を用い、「挽く」はミートチョッパーのような器具を
    用いる場合を指すが、フランス語では区別せずどちらもhacher と表現す
    る。}。
\item
  香味素材\ldots{}\ldots{}にんじん100 g、ポワロー200
  g。小さなさいの目\footnote{brunoise ブリュノワーズ}に刻んでおく。
\item
  澄ませるための素材\ldots{}\ldots{}卵白2ヶ分。
\item
  所要時間\ldots{}\ldots{}1時間半。
\item
  作業\ldots{}\ldots{}片手鍋\footnote{casserole カスロール}または小ぶりの寸胴鍋\footnote{marmiteマルミート。一般的には、大型で深さが直径以上ある両手鍋を
    指す。}に牛挽肉、小さなさいの目に刻
  んだ野菜、卵白を入れ、全体をよく混ぜる。白いコンソメ・サンプルを注ぎ入
  れ、時々混ぜながら\footnote{ここは原文に忠実に訳したが、実際には常に混ぜ続けないと卵白が鍋
    底にくっついて無駄になってしまう。『ラルース・ガストロノミック』初
    版では「絶えず混ぜる」ように指示されている。}沸騰させる。軽く沸騰させながら1時間半煮る。
\end{itemize}

布で漉して仕上げる。

\hypertarget{ux9d8fux306eux30b3ux30f3ux30bdux30e1}{%
\subsubsection{鶏のコンソメ}\label{ux9d8fux306eux30b3ux30f3ux30bdux30e1}}

(仕上がり4 L分)**

*白いコンソメ・サンプル\ldots{}\ldots{}同上。

\begin{itemize}
\item
  主素材と香味素材\ldots{}\ldots{}同上に、以下を加える。オーヴンで軽く色づけた鶏1羽。
  鶏の首づる、手羽先、足など\footnote{原文
    abatisアバティ。鶏肉として食べられる以外の部位の総称。
    鶏の「内臓」と訳されることが多いが、とさか、頭、首づる、手羽先、足
    なども含まれる。}を刻んだもの6羽分。ロティールした鶏
  のがら\footnote{鶏のロティ(ローストチキン)を提供した際に出る「がら」。}2羽分。
\item
  澄ませるための素材、方法、時間は通常のコンソメと同様にする。
\end{itemize}
\end{recette}