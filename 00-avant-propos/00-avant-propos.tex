\hypertarget{ux5e8f}{%
\chapter{序}\label{ux5e8f}}

もう20年も前のことだ。本書の着想を我が尊敬する師、今は亡きユルバン・デュ
ボワ\footnote{Urbain Dubois (1818〜1901)。19世紀後半を代表する料理人。}先生に話したのは。先生は\ruby{是非}{ぜひ}とも実現させなさいと
強く勧めてくださった。けれども忙しさにかまけてしまい、\ruby{漸}{ようや}く
1898年になって、フィレアス・ジルベール\footnote{Philéas Gilbert
  (1857〜1942)。19世紀末から20世紀初頭に活躍した料
  理人。料理雑誌「ポトフ」を主宰した。}君と話し合い協力をとりつける
ことが出来た。ところがまもなく、カールトンホテル開業のために私はロンド
ンに呼び戻され、その厨房の準備や運営に忙殺されることとなった\footnote{エスコフィエはセザール・リッツの経営するホテルグループにおいて料
  理に関わる重要な役割を一手に担っていた。1890年〜1897年にかけてロン
  ドンのサヴォイホテルの総料理長を勤めた後、1898年にはパリのオテル・
  リッツの、1899年にはロンドンのカールトンホテルの開業に携わり、1920
  年までカールトンホテルで総料理長を務めた。}。本書
の計画を実現させるために落ち着いた時間を取り戻さねばならなくなってしまっ
た。

1898年から放置したままだった本書に再び着手出来たのは、多くの同僚たる料
理人諸君の助力と、友人でもあるフィレアス・ジルベール君とエミール・フェ
チュ\footnote{Emile Fétu 生没年不詳。}君の献身的な協力を得られたからに他ならない。この一大事業を完
成させることが出来たのは、ひとえに皆の励ましと、とりわけ辛抱強く、粘り
強く仕事を手伝ってくれた二人の共著者\footnote{ジルベールとフェチュを指しているが、初版には、この二人の他にも共
  著者として4人の名が挙げられている。第二版以降は共著者としてジルベー
  ルとフェチュの名しかクレジットされていない。第二版は初版から構成も
  含め大幅な改訂が行なわれた。その作業を実際に行なったのがジルベール
  とフェチュだったために、他の共著者のクレジットが抹消されたと考えら
  れる。なお、現行の第四版にはエスコフィエの名しかクレジットされてい
  ない。}のおかげだ。

私が作りたいと思ったのは立派な書物というよりはむしろ実用的な本だ。だか
ら、執筆協力者の皆には、作業手順を各自の考えにもとづいて自由にレシピを
書いてもらい、私自身は、40年にわたる現場経験に即して、少なくとも原理原
則、料理における伝統的基礎を明確に説明するのに専念した。

本書は、かつて私が構想したとおりとは言い難い出来だが、いずれはそうなる
べく努めねばなるまい。それでもなお、現状でも料理人諸君にとって大いに役
立つものと信じている。だからこそ、本書を誰にでも、とりわけ若い料理人に
も買える価格にした\footnote{1903年の初版の売価は、\href{http://gallica.bnf.fr/ark:/12148/bpt6k65768837}{フランス国立図書館
  蔵}のものの表紙に
  は、フランス国内で12フランと記したシールが貼られている。また、\href{https://archive.org/details/b21525912}{リー
  ズ大学図書館蔵の第二版}にも
  同様に国内売価12フランのシールが貼られている。1912年の第三版も同じ
  く12フランだった(\href{http://gallica.bnf.fr/ark:/12148/bpt6k96923116}{フランス国立図書館
  蔵}のものに価格を
  示すシールはないが、訳者個人蔵のものには12フランと記されたシールが
  貼られている)。なお、辻静雄は「1903年の初版発売当時は、800ページ
  でたった8フラン、全く破格の値段だった」(「エスコフィエ 偉大なる
  料理人の生涯」、『辻静雄著作集』、新潮社、1995年、729〜730頁)と記
  しているが、その数字の典拠は示されていない。現在と当時の通貨価値、
  物価の違いが分りにくいため、この「破格に安い」という言葉にはやや疑
  問が残るだろう。1900年当時の書籍広告において『料理の手引き』初版と
  同様の八折り版800ページの料理書が、フランス装10フラン、厚紙の表紙
  のものが11フランとあるため、初版の12フランという価格は、むしろ料理
  書としては一般的だったと考えられる。つまり、豪華本ではなく、普通に
  利用できる料理書だということを強調しているに過ぎないと解釈すべきと
  ころだろう。なお、八折り判というのは書籍の大きさを表す用語で、概ね
  縦20〜25cm、横12〜16cm程度。この序文でことさらに「実用性」や入手し
  やすい価格であることが強調されているのは、何度も言及されているデュ
  ボワとベルナールの名著『古典料理』が四折り判(概ね縦45cm、横30cm)
  の豪華本であったことを意識していたためとも推測されよう。}。そもそも若い料理人諸君にこそこの本を読んで
\ruby{貰}{もら}いたい。今はまだ初心者であったとしても、20年後には組織
のトップに立つべき人材なのだから。

私はこの本を豪華な装丁の\footnote{かつてフランスでは、大判の紙の両面に印刷して折ったものを糸で綴じ
  ただけの状態(いわゆる「フランス装」)で販売された本を、書店で買い
  求めた者が別途、業者に製本、装丁させることが一般的に行なわれていた。}、書棚の飾りのごときにはして欲しくない。そ
うではなく、いつでも、どんな時でも手元に置いて、分からないことを常に明
らかにしてくれる\ruby{盟友}{めいゆう}として欲しい。

本書には五千を越えるレシピが掲載されているが、それでも私は、この教本が
完全だとは思っていない。たとえ今この瞬間に完璧であったとしても、明日に
はそうではないかも知れぬ。料理は進化し、新しいレシピが日々創案されてい
る。まことにもって不都合だが、版を重ねる毎に新しい料理を採り
入れ、古くなってしまったものは改善せねばなるまい。

ユルバン・デュボワ、エミール・ベルナール\footnote{Emile Bernard
  (1827〜1897)。クラシンスキ将軍の料理人を務めた。}両氏の著作\footnote{デュボワとベルナールの共著は他にもあるが、ここでは『古典料理』
  (1856年)を指している。}に昔から慣
れ親しみ、その巨大な影がなおも料理の地平を覆い尽している現在、私として
は本書がその後継になって欲しいと思っている。カレーム以後、最高の料理の
高みに逹した二人に対し、ここであらためて心から敬意を表させていただき
いと思う。

調理現場を取り巻く諸事情により、私は、デュボワ、ベルナール両氏がもたら
したサービス(給仕)面での革新\footnote{\protect\hypertarget{service-russe}{19世紀後半に一般的となった「ロ
  シア式サービス」のこと。中世以来、正式な宴席では卓上に大皿の料理が
  一度に何種も並べられ、食べる者がそれぞれ好きなように取り分けていた。
  これを、献立を食べる順に1種ずつ、大皿料理の場合は食べ手に見せて回っ
  てから、給仕が取り分けて供する方式にしたのがロシア式サービスである。
  食卓に大皿を並べない代わりに、花を飾りナフキンを美しく折るなどの工
  夫により卓上が洗練さていった。19世紀パリに駐在していたロシア帝国の
  外交官クラーキンが提唱した。デュボワとベルナールの『古典料理』序文
  において詳述されている。}}に対し、こんにちのようなとりわけスピー
ドが重視される目まぐるしい生活リズムに合わせて、大きな変更を加えざるを
得なかった。そもそも物理的理由から、料理を載せる飾り台\footnote{\protect\hypertarget{socle}{socle ソークル。パンや米、ジュレな
  どで作った、料理を盛り付けるために銀の盆の上に据える飾り台。カレー
  ムの時代、つまり19世紀前半にはその装飾に凝ることが多かった。食べも
  ので作られてはいるが、料理の一部ではなく、あくまで装飾的要素でしか
  なかった。この飾り台はロシア式サービスの時代になってもな豪華絢爛た
  る宴席においては重要なものとして扱われており、デュボワとベ}ルナー
  ル『古典料理』でも相応のページ数を割いて説明がなされている。}をやめて、
シンプルな盛り付けにする新たなメソッドと新たな道具を考案する必要があっ
たのだ。デュボワ、ベルナール両氏が推奨した壮麗な盛り付けを私自身も行なっ
ていた頃はもちろん、今なお二方の思想にはまったく共感している。冗談でこ
んなことを言っているのではない。しかし、カレームを信奉する者たちは、装
飾の才があるが\ruby{故}{ゆえ}に、時代にもはや\ruby{似}{そぐ}わなくなっ
てしまった作品に対して改良を加えようとはしなかった。時代に合わせて改良
することこそ、まさに重要なのに。本書で奨励している盛り付けは、少なくと
もそれなりの期間、有用であり続けると思う。全ては変化する。姿を変える。
それなのに、装飾芸術の役割が変化しないと主張するなどとは\ruby{蒙昧}{も
うまい}ではないか。芸術は流行によって栄えるものだし、流行のように移ろ
いやすいものだ。

だが、カレームの時代にはこんにちと同じく\ruby{既}{すで}にあり、料理が
続く限りなくならないだろうものがある。それが料理のベースとなるフォンや
ストックだ。そもそも、料理が見た目にシンプルになっても料理そのものの価
値は失なわれないが、その逆はどうだろう? 人々の味覚は絶え間なく洗練さ
れ続け、それを満足させるために料理そのものも洗練されることになる。こん
にちの余剰活動が精神におよぼす悪影響に打ち\ruby{克}{か}つためには、料
理そのものがいっそう科学的な、正確なものとなるべきなのだ。

その意味で料理が進歩すればする程、我々料理人たちにとって、19世紀、料理
の行く末に大きく影響を与えた三人の料理人の存在は大きなものとなるだろう。
カレームとデュボワ、ベルナールはともすれば技術的側面ばかり評価されるが、
料理芸術の基礎において何よりも優れているのだ。

O既に物故した名だけ挙げるが、確かにグフェ\footnote{Jules Gouffé
  (1807〜1877)。著書も多く、代表作『料理の本(1867年)
  は前半が家庭料理、後半が高級料理の二部構成となっており、レシピの書
  き方も、まず材料表を掲げた後に調理手順を説明するという現代のものに
  近く、挿絵も比較的多くて分りやすい。フランス料理史における名著のひ
  とつ。}、ファーヴル\footnote{Joséph Favre
  (1849〜1903)。スイス生まれの料理人で、パリ、ドイツ、
  イギリス、ベルギー等において活躍した。著書『料理および食品衛生事典』
  (1884〜1895年)。}、エルー イ\footnote{Edouard
  Hélouis(生没年不詳)。イギリスのアルバート王配(ヴィク
  トリア女王の夫)(1819〜1861)やイタリアのヴィットーリオ・エマヌエー
  レ二世(1820〜1878)に仕えたという。著書『王室の晩餐』(1878年)。}、ルキュレ\footnote{『実践的料理』(1859年)の著者C.
  Reculetのこと。}はとても素晴らしい著作を残した。だが、『古典料理』
という\ruby{稀代}{きたい}の名著に\ruby{比肩}{ひけん}し得るものはひとつ
としてない。

料理人諸君に、新たに本書を使っていただくにあたり、言うべきことがある。
いろいろな料理書、雑誌を読み散らかすのもいいが、偉大な先達の不朽の名著
はしっかり読み込むように、と。\ruby{諺}{ことわざ}にあるように「知り過
ぎることなはい」のだ。学べば学ぶ程、さらに学ぶべきことは増えていく。そ
うすれば、柔軟な思考が出来るようになり、料理が上達するためのより効果的
な方法を知ることも出来るだろう。

本書を\ruby{上梓}{じょうし}するにあたって\ruby{唯}{ただ}ひとつ望むこと、
切に願う\ruby{唯一}{ゆいいつ}のことは、上記の点において、本書の対象た
る読者諸君が我が\ruby{言}{げん}に耳を傾け、実践するさまを見ることに尽
きる。\nopagebreak

\begin{flushright}
A. エスコフィエ \nopagebreak
\end{flushright}

1902年11月1日

\newpage

\hypertarget{ux7b2cux4e8cux7248ux5e8fux6587}{%
\section{第二版序文}\label{ux7b2cux4e8cux7248ux5e8fux6587}}

\vspace*{1.7\zw}

ここに第二版を上梓するに至ったわけだが、二人の共著者による熱意あふれる
仕事のおかげで、私の強い期待をさらに越える本書の成功が約束されたも同然
だろう。だからこそ、共著者両君および本書の読者諸君に心からの謝辞を申し
あげる次第だ。また、ありがたいことに、称賛の言葉を寄せてくださった方々
と、貴重な批判をくださった方々にも御礼申しあげる。批判については、それ
が正当なものと思われる場合については、本書に反映させるべく努めさせてい
ただいた。

かくも多くの人々に本書を受け入れていただけたことへの謝意を表するには、
本書における技術的な価値を高め、初版ではロジカルにレシピを分類しようと
したが故に生じた欠点を解消する他ないだろう。それは、調理理論とレシピを
損なうことなしに、本書の計画段階において簡単に済まさざるを得ないと思わ
れたテーマについて\ruby{能}{あた}う限り肉薄することでもある。私たち本
文の見直しをするとともに、多くのレシピを追加した。そのほとんどは調理法
と盛り付けにおいて、こんにちの顧客のニーズを鑑みて着想したものであり、
そのニーズが正当かつ実現可能な範囲において、顧客への給仕のペースが日増
しに加速していく傾向をも考慮に入れたものだ。こういった傾向は数年来まさ
しく際立ってきているが\ruby{故}{ゆえ}に、我々としも常に気を配っておか
ねばならぬ。

「料理芸術」というものは、その表現形態において、社会心理に左右されるも
のだ。社会から受ける衝撃に逆らわぬことも必要であり、\ruby{抗}{あらが}
えぬことでもある。快適で安楽な生活がいかなる心配事にも乱されることのな
いような社会であれば、未来が保証され、財をなす機会もいろいろあるような
社会であれば、料理芸術はたゆまぬことなく驚異的な進歩を遂げるだろう。料
理芸術とは、ひとが得られる悦びのうちでもっとも快適なもののひとつに寄与
しているのだから。

反対に、安穏とした生活の出来ぬ、商工業からもたらされる\ruby{数多}{あま
た}の不安で頭がいっぱいになるような社会において、料理芸術は心配事でいっ
ぱいの人々の心のごく限られた部分にしか美味しさを届けられない。ほとんど
の場合、諸事という渦巻きに巻き込まれた人々にとって、食事をするという必
要な行為はもはや悦びではなく、辛い義務でしかないのだ。

\ruby{斯}{か}くのごとき生活習慣は\ruby{嘆}{なげ}いていい、\ruby{否}{い
な}、嘆くべきことなのだ。食べ手の健康という観点からも、食べたものを胃
が受け付けないという結果になるとしたら、それは絶対に生活習慣が悪いのだ。
そういう結果を抑える力は私に出来る範囲を越えている。そういう場合に
調理科学が出来ることといえば、軽率な人々に\ruby{能}{あた}うかぎり最良
の食べものを与えるという対症療法だけなのだ。

顧客は料理を早く出せと言う。それに対して私たち料理人としては、ご満足い
ただけるようにするか、失望させてしまうことのどちらかしか出来ない。料理
を早く出せという顧客の要求を拒む方法があるとするなら、それ以上の方法で
顧客にご満足いただけるようにすることしかない。だから、私たちは顧客の気
まぐれの前に折れざるを得ないのだ。これまで私たちが慣れ親しんできた仕事
のやり方では、これまでの給仕のスタイルでは、顧客の気まぐれに応えること
が出来ぬ。意を決して仕事の方法を改革すべきなのだ。だがひとつだけ、変え
てはならぬ、手をつけてはならぬ領域がある。料理ひとつひとつのクオリティ
だ。それは、料理人にとって仕事のベースとなるフォンや事前に仕込んでおい
たストック類がもたらすゆたかな風味に他ならぬ。私たちは既に、盛り付けの
領域においては改革に着手した。足手まといにしかならぬ多くのものは既に姿
を消したか、いままさに消え去らんとしている。料理の飾り台\footnote{socle
  ソークル、\protect\hyperlink{socle}{序p.ii訳注4}参照。}、料理の
周囲の装飾\footnote{bordure
  ボルデュール。本書においてもガルニチュールの扱いにおい
  てこの指示はあるが、19世紀のものと比較するとかなりシンプルな内容に
  なっている。}、飾り串\footnote{hâtelet
  アトレ。一方の端に動物などの姿の装飾の施された銀製の串
  に、トリュフやクルヴェット(海老)などを事前に別の串(ブロシェット)
  で焼いてからこの飾り串に刺し直し、それを大きな塊肉や丸鶏、大型の魚
  1尾の料理に刺した。19世紀初頭、カレームの時代に全盛となり、その著
  書『パリ風料理』において詳述されている。19世紀末まではこの装飾がな
  されることが多かった。また、その飾り串そのものが美麗な装飾品である
  ためにコレクションの対象になっていた。}などのことだ。この方向性は推し進められると
思う。これについては後述しよう。私たちはシンプルであるということを極限
まで追究したい。それと同時に、料理の風味や栄養面での価値を増すことも目
指している。料理はより軽い、弱った胃にも優しいものにしたいと考えている。
私たちはこの点にのみ尽力したい。料理において役をなさない大部分はすっか
り剥ぎ取ってしまいたいと考えているのだ。一言でまとめると、料理は芸術で
あり続けつつも、より科学的なものとなるだろうし、その作り方はいまだ経験
則に基づいただけのものばかりであるが、ひとつのメソッド、偶然などに左右
されない正確なものになっていくことだろう。

こんにちは料理の過渡期にある。古典料理メソッドの愛好者はいまなお多く、
私たちもそれを理解し、その思想に心から共感するところもある。だが、食事
というものがセレモニーであり、かつパーティであった時代を懐しんでどうす
るというのだ? 古典料理がこんにちの美食家に至福の時を与えるために力を
発揮出来る場がどこにあるというのだ? いったいどうすれば、美食と宴の神
コモス\footnote{フランス語 Comus
  コミュス。ラテン語では同じ綴りでコムスと読む。
  ギリシア、ローマ神話における、悦びと美食の神。18世紀の料理本作家マ
  ランの主著は『コモス神の贈り物』がタイトル。}に捧げ物を供えるという幸せな機会を毎回得られるのだろうか?
だから私たちは本書において、個人的な創作よりむしろ伝統的なフランス料理
のレシピ集として、こんにちの料理のレパートリーから姿を消してしまったも
のも残すことに固執した。その名に値する料理人なら、機会さえ与えられたら
王侯貴族も近代の大ブルジョワもひとしく満足させるためには、知っておくべ
きものなのだ。時間のことなんぞ気にもせぬ穏かな美食家の方々にも、時こそ
全てと言わんばかりの金融家やビジネスマンたちにも満足していただくために。
だから、本書が新しいメソッドに偏ったものだという非難にはあたらない。私
はただ単に、料理芸術の進化の歩みをたどり、いまの時代に即しつつ、食べ手
すなわち食事会の主催者と招待客の皆様の意向を絶対的なものとして、それに
従いたいと願っているだけなのだ。食べ手の意向に対して私たち料理人は
\ruby{頭}{こうべ}を垂れて従うことしか出来ぬのだから。

私たちは、料理の美味しさを損なうことなくより早く料理を提供できるような
方法を、料理人各人が自らの嗜好を犠牲にすることなしに探求すべく
\ruby{誘}{いざな}うことこそが、料理人諸君にとって有益と信じている。全
体として、私たちのメソッドはまだまだ日々のルーチンワークに依存し過ぎて
いるものだ。顧客の求めに応えるため、私たちは既に仕事のやり方をシンプル
なものにせざるを得なかった。だが、残念ながらいまだ\ruby{途}{み
ち}\ruby{半}{なか}ばに過ぎぬと感じている。私たちは自己の信念をしっかり
堅持しており、どうしようもない場合にのみ自説を曲げることもある。だから、
装飾に満ちた飾り台を廃止した一方で、盛り付けに時間のかかる厄介で複雑な
ガルニチュールは残してある。こういったガルニチュールを濫用することはガ
ストロノミーの観点から言って、常に間違っているのは事実だが、残しておく
べきものと思われる。それを求める顧客あるいは食事会主催者に絶対に従う必
要のある場合はとりわけそうだ。ごく稀にとはいえ、料理の美味しさを損なう
ことなくそれらを実現可能なこともあるからだ。時間と金銭、広くてスタッフ
の充実した会場、という3つの本質的要素を最大限活用可能な場合のことだが。

通常の厨房業務においては、ガルニチュールをかなりシンプルな、せいぜい3〜
4種の構成要素からなるものに減らさざるを得なくなっている。そのガルニチュー
ルを添える料理がアントレであれルルヴェ\footnote{19世紀前半まで主流であった「フランス式サービス」つまり、一度に
  多くの料理の皿を食卓に並べるという給仕方式において、ポタージュを入
  れた大きな深皿が空くと、それを給仕が下げて、豪華な装飾を施した大き
  な塊肉の料理がポタージュを置いてあった場所に据えられた。これを
  relevéeルルヴェ(交代したもの、の意)と呼んだ。エスコフィエの時代
  にはフランス式サービスではなくロシア式サービスに移っており、大きな
  塊肉の料理や大型の魚1尾まるごとを大皿で出し、給仕が切り分けて配膳
  するようになっていたが、名称はそのまま残った。Entréeアントレ(もと
  は「入口」の意)は現代において「前菜」の意味で用いられているが、食
  卓に大皿で並べられた肉料理(場合によっては魚料理も含む)の総称とし
  てこの語が用いられていた。本書はそれを踏襲している。本書においてル
  ルヴェおよびアントレに分類されている料理の多くは現代においてコース
  料理の「メイン」に相当するものが多く、実際、英語ではコース料理のメ
  インのことを現在でもこの語で表わすことが多い(前菜はappetizerアペ
  タイザーと呼ぶ)。}であれ、牛・羊肉料理であれ、
家禽であれ魚料理であれ、そうせざるを得ない。そのようにして構成要素を減
らしたガルニチュールは、素早い皿出しが要求される場合には必ず、ソースと
同様に別添で供するのがいい。その場合、盛り付けは奇抜というくらいシンプ
ルなものとなるが\ldots{}\ldots{}メインの料理はより冷めない状態で、より早く、よりき
れいに供することが可能になる。給仕が料理を取り皿に分けてお客様に出すに
せよ、お客様が大皿を自分たちで受け渡して取り分けるにせよ、サービス担当
者は安心して仕事が出来るし、そのほうが容易だ。メインの大皿が山盛りにな
ることはないし、その上に盛り付けられたいろいろな素材のガルニチュールも
簡単に取ることが出来るからだ。

こんにちの他のシステムだと、料理を載せるための台や装飾のための飾り串を
作り、さらに料理の周囲にガルニチュールを配置するのに、看過出来ぬ程の時
間を要していた。こういう盛り付けというのは、料理そのものがさして大きく
ないものであっても、食べ手の人数が少ない場合であっても、大面積の皿を用
いる必要があった。だから、お客様が料理を自分たでぃで受け渡して取り分け
る必要がある場合などは、お客様にとっても、サービス担当者にとってもまこ
とに窮屈なものであった。これは、複雑な構成のガルニチュールの持つ大きな
欠点のひとつとして無視できないことだ。他の欠点というのは、あらかじめ盛
り付けを行なうことによって美味しさが減じてしまうこと、食べ手が少人数の
場合には必然的に、料理を見せて周る間に冷めてしまうこと、などがある。こ
ういう愉快とは言えぬことの結果は何とも情けないことになる。つまり、お客
様に大皿に盛り付けた料理をお見せするのはほんの一瞬だけ、お客様は多少な
りとも豪華で精密に盛り付けられた料理をちらりと見る暇があるかないか、と
いうことだ。昔日のごとき豪華壮麗な料理を供することの可能な場所もこんに
ちでは少なくなってきたが、それ以外のところでもこういった悪習が頑固なま
でに続けられているというのは、それが昔からの習慣だということでしか説明
がつかぬ。

給仕のスピードを容易に上げるために、大きな塊肉の料理でない場合には毎回、
下の図のごとき四角形の深皿を出来るだけ用いるよう是非ともお勧めしたい。
温かい料理でも、冷製の料理でも、この皿は非常に優れたものであるから、そ
の目的において厨房に備えておくべきものとして他の追随を許さないと言える
\footnote{この段落は、初版の序文の後にある「盛り付け方法をシンプルにする
  ことについて」という挿絵付きの節の内容を短かく縮めたために、ややわ
  かりにくいものになっている。ただし、第二版および第三版においては序
  文の最後に皿の挿絵が添えられている。}。

繰り返しになるが、本書が新しい方法を勧めているからといって、偏見で古典
的なものを悪いと断じているのでは決してない。私たちは、料理人諸君に、顧
客たちの生活習慣や味の好みを研究し、自らの仕事をそれらに適合させるよう
\ruby{誘}{いざな}いたいと思っているだけなのだ。我々料理人にとって高名
な師とも呼ぶべきカレームは、ある日、同業たる料理人のひとりとおしゃべり
をしていた際に、その料理人が仕えている主人の洗練さに欠けた食事の習慣や
下卑た味覚を苦々しげに語るのを聞かされたという。その食事の習慣と味覚に
憤慨して、自分が人生をかけて追究してきた知的な料理の原則を曲げてまで仕
え続けるくらいなら、いっそ辞めてしまいたいと思っている、と。カレームは
こう答えた。「そんなことをするのは君のほうが間違っているよ。料理におい
て原則なんていくつも存在しないんだ。あるのはひとつだけ、仕えているお方
に満足していただけるか、ということだけなんだよ」と。

今度は我々がその答を考える番だ。自分たちの習慣やこだわりを、料理を出す
相手に押しつけるなどと言い張るとしたら、まったくもって馬鹿げたことだ。
我々料理人は食べ手の味覚に合わせて料理することこそが第一でありもっとも
本質的なことなのだと、私たちは確信している。

私たちがかくも安易に顧客の気まぐれにおもねったり、過度なまでに盛り付け
をシンプルにするせいで、料理芸術の価値を下げ、単なる仕事のひとつにして
しまっている、と非難する向きもあるだろう。---だがそれは間違いだ。シン
プルであることは美しさを排するものではない。

ここで、本書の初版において盛り付けについて述べた部分を繰り返すことをお
許しいただきたい。

「どんなにささやかな作品にも自らの最高の印をつけられる才というのは、そ
の作品をエレガントで歪みのないものに見せられるわけで、技術というものに
不可欠だと私は信じている。

だが、職人が美しい盛り付けを行なうことで自らに課すべき目的とは、食材を
他に類のない方法で節度をもって用いつつ大胆に配置することによってのみ、
実現されるのだ。未来の盛り付けにおいて絶対に守るべきこととして、食べら
れないものを使わないこと、シンプルな趣味のよささこそが未来の盛り付けに
特徴的な原則となるだろうことを、認めるべきなのだ。

そのような仕事を成し遂げるために、能力ある職人にはいくつもの手段がある。
トリュフ、マッシュルーム、固茹で卵の白身、野菜、舌肉などの食べられるも
のだけを用いて、素晴らしい装飾を組み合わせ、無限に展開できるのだ。

王政復古期\footnote{1814年ナポレオンが退位して国外へ亡命、ルイ18世を戴く王政へ回帰
  した時期。1830年まで続いたが7月革命でブルボン家は断絶し、その後オ
  ルレアン朝による七月王政が1848年まで続いた。}に料理人たちによって流行した複雑な盛り付けの時代は終わっ
た。だが、特殊な例になるが、古い方法で盛り付けをしなければならない場合
もあり、そういう時は何よりもまず、盛り付けにかかる時間と利用できる手段
を見積らなくてはならない。土台の形状を犠牲にしなくても、装飾の繊細さを
忘れなくても、風味ゆたかな素材を軽んじたり劣化させてしまっては、価値の
ないものにしかならないのだ」。

以上の見解はずっと変わっていない。料理は進歩する(社会がそうであるよう
に)。だが常に芸術であり続けるのだ。

例えば、1850年から人々の生活習慣、習俗が変化したことを皆が認めるにやぶ
さかでないように、料理もまた変化するのだ。デュボワとベルナールの素晴し
い業績は当時のニーズに応えたものだ。だが、たとえ二人がその著書と同じく
永遠の存在であったとしても、彼らが称揚した形態は、料理の知識として、我々
の時代の要求に応えうるものではない。

私たちは二人の名著を尊重し、敬愛し、研究しなくてはならない。二人の著書は
カレームの著作とともに、我々料理人の仕事の基礎たるものだ。だが、そこに
書いてあることを盲目的に真似るのではなく、我々自身で新たな道を切り
\ruby{拓}{ひら}き、我々もまたこの時代の習俗や慣習に合わせた教本を残す
べきなのだと考える次第である。

\begin{flushright}
1907年2月1日
\end{flushright}

\newpage

\hypertarget{ux7b2cux4e09ux7248ux5e8fux6587}{%
\section{第三版序文}\label{ux7b2cux4e09ux7248ux5e8fux6587}}

\vspace*{1.7\zw}

『料理の手引き』第三版を同業たる料理人諸賢に向けて上梓するにあたり、絶
えず本書を好意的に支持してくださったことと、多くの方々から著者一同にお
寄せくださった励ましのお言葉に対し、あらためて深く御礼申しあげる次第だ。

第二版序文の内容につけ加えるべきことは何もない。というのも、第二版序文
で料理という仕事について申しあげたことは、1907年当時も今も変わっていな
い事実だからだし、今後も長くそうであり続けるだろう。とはいえ、この第三
版は内容を精査し、かなりの部分を改訂してある。かつては予測でしかなかっ
たことを実証し、この『料理の手引き』初版の序文においてエスコフィエ氏\footnote{この表現から、第三版序文がエスコフィエ自身ではなく、フィレアス・
  ジルベールかエミール・フェチュのいずれか、あるいは二人によって書か
  れたと判断される。}が
以下のように書かれた約束も果せたと思う。「本書には五千近くもの\footnote{初版において掲載レシピ数は5千に満たなかったため。初版および第二
  版では「五千近い」の表現となっており、第三版ににおいて「五千以上」
  と表現が変更された。}レ
シピが掲載されているが、それでも私は、この教本が完全だとは思っていない。
たとえ今この瞬間に完璧であったとしても、明日にはそうではないかも知れぬ。
料理は進化し、新しいレシピが日々創案されているのだ。まことにもって不都
合なことだが、版を重ねる毎に新しい料理を採り入れ、古くなってしまったも
のは改良を加えねばなるまい。」

この言葉が、前回の第二版から300ページを増やしたことの説明となっている
わけで、この新版でいくつかの変更を我々が必要と考えた理由でもある。

\begin{enumerate}
\def\labelenumi{\arabic{enumi}.}
\item
  判型の変更\ldots{}\ldots{}あえて判型を大きくすることで、より扱いやすいものとしたこと\footnote{初版および第二版はいわゆる「八折り版」約21.5cm×13.5cmであった
    のに対し、第三版は約24cm×16cm、つまり現代のB5版よりほんの少し小さ
    めの判型。}
\item
  巻末の目次の組みなおし\ldots{}\ldots{}当初は料理の種類別であったが、本書全体の
  項目をアルファベット順にまとめたこと\footnote{原文ではTable des
    Matière「目次」とあるが、これは巻頭の章を示す
    目次のことではなく、巻末の「索引」に相当するものを意味している。残
    念ながら「索引」としてはあまり使い勝手のよくない不完全なものに留まっ
    ている。}
\item
  時代遅れになったと思われるレシピを相当数削除し、その代わりとしてこ
  の数年の間に創案され好評を博したレシピを追加したこと
\end{enumerate}

既に大著であって本書にこれらの変更を加えるために、我々は第二版の巻末に
付されていた献立のページを削除せざるを得なかった。

献立についても内容を一新し、多くの献立例を追加して、『メニューの本』と
いう独立した書籍として、この第三版と同時に刊行する予定となっている。こ
の『メニューの本』において我々は献立とその説明文はもちろんのこと、大規
模な厨房における日々の業務配分を示す表を入れておいた。

このように別冊とすることで、献立の作成という非常に重要な問題を適切に展
開し、ゆとりを持って論じることが可能となったわけだ。

この新刊『メニューの本』は料理人諸賢だけではなくメートルドテル、食事施
設の責任者に必携のものとなった。さらには必要なものを奇抜なまでに単純化
してしまう家庭の主婦にとっても必携となろう。我々は上記の改良点が、これ
まで多くの好意的見解をお寄せくださった料理関わる皆様方に、好意的に受け
容れていただけると信じている。また、料理芸術の栄光のもと未来に続くモニュ
メントを建てるべく努めた我々のささやかなる尽力が、料理芸術に利をもたら
さんことを信じる次第だ。

\begin{flushright}
1912年5月1日
\end{flushright}

\hypertarget{ux7b2cux56dbux7248ux5e8fux6587}{%
\section{第四版序文}\label{ux7b2cux56dbux7248ux5e8fux6587}}

\vspace*{1.7\zw}

『料理の手引き』第三版刊行当時(1912年5月)から後、他の職業、産業と同
様に料理界もまた大いなる危機に見舞われた\footnote{第一次世界大戦(1914〜1918)による社会的影響を指している。フラ
  ンスは戦中から戦後にかけて激しいインフレに見舞われた。なお、この第
  四版から出版社がそれまでのラール・キュリネールからフラマリオン社に
  変わった。}。こんにちもなお料理は厳
しい試練にさらされている。しかしながら、料理界はその試練に耐えてきたし、
戦後のこの辛い時期に終止符を打ち、料理界がさらに前進し始めるのもさして
遠いことではないと信じている。だが、目下のところ、あらゆる食材の異常な
までの高騰により、料理長諸賢が責務を果すことがひどく難しくなっている。
料理長がその責務を果すということの困難さを経験上よく知っているからこそ、
今回の版において我々は、多くのレシピ、とりわけガルニチュールについて、
その本質的なところを曲げることなしに、よりシンプルなものにすることにこ
だわった。

さらに、もはやあまり興味を持たれないであろうレシピは全て削除して、その
代わりに近年創案されたレシピを収録することとした。

したがって、料理人諸賢および料理に関心を持つ皆様方に向けてこの『料理の
手引き』第四版を上梓するにあたり、旧版同様、皆様に温かく受け容れていた
だけると信じる次第である\footnote{原書の文体から、この序文も第三版序文と同様に、ジルベールとフェ
  チュによって書かれた可能性も考えられる。}。

\begin{flushright}
1921年1月
\end{flushright}

\small

\hypertarget{ux53c2ux8003ux76dbux308aux4ed8ux3051ux3092ux30b7ux30f3ux30d7ux30ebux306bux3059ux308bux3068ux3044ux3046ux3053ux3068ux521dux7248ux306eux307f}{%
\subsection{【参考】盛り付けをシンプルにするということ(初版のみ)}\label{ux53c2ux8003ux76dbux308aux4ed8ux3051ux3092ux30b7ux30f3ux30d7ux30ebux306bux3059ux308bux3068ux3044ux3046ux3053ux3068ux521dux7248ux306eux307f}}

本書では、かつては料理の盛り付けによく用いられた飾り串\footnote{hâtelet
  アトレ。}、縁飾り \footnote{bordure ボルデュール。}、クルトン\footnote{菱形やハート形にしたパンを揚げたもの。}、チョップ花\footnote{papillote
  パピヨット。紙製で、骨付き肉の先端を飾るもの。}などを使う指示がほとんど出てこな
い。著者としては、盛り付け方法を近代化すると同時に、ほぼ完全に上記のも
のどもを削除しなくしてしまいたいとさえ考えたくらいだ。

我らが先達が考えていたような盛り付けには、長所がたったひとつしかない。
皿を荘厳に、魅力的な姿にすることで、料理を味わう前に、食べ手の目を楽し
ませ、喜んでいただくということだ。

だが、そうした盛り付けの作業は複雑で難しいものであり、かなりの時間を必
要とする。比較的少人数の宴席でないかぎりは、こうした盛り付けは事前に用
意しておく必要がある。そのようにして作られた料理は、それを置いておく場
所のことを考えに入れないとしても、必ずといっていい程、冷めてしまってい
る。また、料理を載せる台や縁飾り、飾り串に費す時間も考えなくてはならな
いし、そういった装飾にかかる費用も考えなくてはならない。忘れてはならな
いことだが、そのように装飾した皿の見た目の調和がとれている時間というの
は、その皿をお客様にお見せする間だけなのだ。メートルドテルのスプーンが
料理に触れるやいなや、かくも無惨な姿となりお客様の目には不快なものとなっ
てしまう。こういう不都合はなんとしても改善しなければならなかったのだ。

ここで図に示すような四角形の皿を採用したことで、上記のような問題は解決
したと考えている。この皿はパリのリッツホテルで初めて用いられ、ロンドン
のカールトンホテルにおいて正式に採用されることとなったものだ。この皿を
用いることの利点は絶大で、これを用いない盛り付けなどもはや考えられない
程だ。この皿は場所をとらず、皿の内側に盛り付けられた料理は冷めることが
ない。蓋との距離が近いから保温されているわけだ。魚や肉の切り身は上に重
ねて盛るのではなく、ガルニチュールとともに並べて盛り付けることが出来る。
そうすることで、最初に給仕されるお客様から最後に給仕される方まで、料理
は美味しそうな見た目を保つことが出来るのだ。その結果、クルトンやチョッ
プ花、皿の上にしつらえる料理を載せる台や縁飾り、飾り串、昔の給仕で用い
られた面倒なクロッシュ\footnote{cloche
  主に金属製で半球形の保温を目的としたディッシュカバー。}は不要なものとなる。

この皿は冷製料理にもまた便利に使うことが出来る。周囲に氷を積み重ねて囲
うか、薄い氷のブロックの上に盛り付ければ、飾りには、ごく繊細なジュレだ
けていい。そのような繊細なジュレを使うのは昔の方法では不可能だった。か
くして、邪魔にさえ思える飾り台も、皿の底の飾りも、アトレも必要なくなっ
た。ショフロワは1切れずつ並べて、周囲を琥珀色のとろけるようなジュレで
満たしてやればいい。ムースはもはや「つなぎ」をまったく、あるいはほとん
ど必要としない。こういうことが、冷製料理の芸術的な見た目を、豪華さや美
しさという点でいっかな失なうことなく可能となるのだ。

この新式の什器とそれによって実現可能となる料理に習熟することについて料
理人諸君にお報せすることは我々の義務であると考える。利点がとても大きい
ので、あえて申しあげるが、これを使うことが、給仕を素早く、きれいに、経
済的に、そして文句ないまでに実践的なものにする唯一の方法である。

\hypertarget{ux53c2ux8003ux521dux7248ux306fux3057ux304cux304d}{%
\subsection{【参考】初版はしがき}\label{ux53c2ux8003ux521dux7248ux306fux3057ux304cux304d}}

本書はある特定の階層の料理人を対象としているものではなく、全ての料理人
が対象であるため、本書のレシピは、経済的観点や料理人が実際に利用可能な
手段に応じて、改変できるものだということを述べておきたい。

本書に収められたレシピはすべて、グランドメゾンでの仕事における原則にも
とづいて組み立てて調整してある。だから、より格下の店舗などでも、必然的
に量を減らせば作れるだろうし、適価で提供出来るようにもなるだろう。

ひとつひとつの項目において、いろいろな飲食を提供する形態を網羅するよう
にレシピを書くことが不可能だったということは理解されよう。料理人自身が
自主性をもって本書の内容を補えるし、そうすべきなのだ。ある者たちにとっ
て非常に大切なことが、大多数の者にとってはそこそこの興味しか引かず、一
般的に見たら無益で幼稚に思われることだってあるのだ。

だから、本書に収録したレシピは最大の分量でまとめられたものを考えるべき
であり、必要に応じて、各人の判断および物理的に出来る範囲に合わせて、量
を減らして作るといい。 \normalsize
