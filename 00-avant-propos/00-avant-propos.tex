\documentclass[twoside,12Q,b5j]{escoffierltjsbook}
%\documentclass[twoside,8pt,a5j]{escoffierltjsbook}
\usepackage{amsmath}%数式
\usepackage{amssymb}
\usepackage[no-math]{fontspec}
%\usepackage{xunicode}
\usepackage{geometry}
\usepackage{unicode-math}
\usepackage{xfrac}
\usepackage{luaotfload}
\usepackage{makeidx}


\usepackage[unicode=true]{hyperref}
\hypersetup{breaklinks=true,
             bookmarks=true,
             pdfauthor={},
             pdftitle={},
             colorlinks=true,
             citecolor=blue,
             urlcolor=blue,
             linkcolor=magenta,
             pdfborder={0 0 0}}
\urlstyle{same}

%%欧文フォント設定
\setmainfont[Ligatures=TeX,Scale=1.0]{Linux Libertine O}

%%Garamond
%\usepackage{ebgaramond-maths}
%\setmainfont[Ligatures=TeX,Scale=1.0]{EB Garamond}%fontspecによるフォント設定


%\setmainfont[Ligatures=TeX]{TeX Gyre Pagella}%ギリシャ語を用いる場合はこちら
%\setsansfont[Scale=MatchLowercase]{TeX Gyre Heros}  % \sffamily のフォント
\setsansfont[Ligatures=TeX, Scale=1]{Linux Biolinum O}     % Libertine/Biolinum
\setmonofont[Scale=MatchLowercase]{Inconsolata}       % \ttfamily のフォント
\unimathsetup{math-style=ISO,bold-style=ISO}
\setmathfont{xits-math.otf}
\setmathfont{xits-math.otf}[range={cal,bfcal},StylisticSet=1]

\usepackage[cmintegrals,cmbraces]{newtxmath}%数式フォント

\usepackage{luatexja}
\usepackage{luatexja-fontspec}
%\ltjdefcharrange{8}{"2000-"2013, "2015-"2025, "2027-"203A, "203C-"206F}
%\ltjsetparameter{jacharrange={-2, +8}}
\usepackage{luatexja-ruby}

%%%%和文仮名プロポーショナル
%\usepackage[yu-osx]{luatexja-preset}
\usepackage[hiragino-pron,90jis,expert,deluxe]{luatexja-preset}
%\usepackage[ipaex]{luatexja-preset}
%\newopentypefeature{PKana}{On}{pkna} % "PKana" and "On" can be arbitrary string
%\setmainjfont[
%    JFM=prop,PKana=On,Kerning=On,
%    BoldFont={YuMincho-DemiBold},
%    ItalicFont={YuMincho-Medium},
%    BoldItalicFont={YuMincho-DemiBold}
%]{YuMincho-Medium}
%\setsansjfont[
%    JFM=prop,PKana=On,Kerning=On,
%    BoldFont={YuGothic-Bold},
%    ItalicFont={YuGothic-Medium},
%    BoldItalicFont={YuGothic-Bold}
%]{YuGothic-Medium}
%%%%和文仮名プロプーショナルここまで

\renewcommand{\bfdefault}{bx}%和文ボールドを有効にする
\renewcommand{\headfont}{\gtfamily\sffamily\bfseries}%和文ボールドを有効にする

\defaultfontfeatures[\rmfamily]{Scale=1.2}%効いていない様子
\defaultjfontfeatures{Scale=0.92487}%和文フォントのサイズ調整。デフォルトは 0.962212 倍%ltjsclassesでは不要?
%\defaultjfontfeatures{Scale=0.962212}
%\usepackage{libertineotf}%linux libertine font %ギリシア語含む
%\usepackage[T1]{fontenc}
%\usepackage[full]{textcomp}
%\usepackage[osfI,scaled=1.0]{garamondx}
%\usepackage{tgheros,tgcursor}
%\usepackage[garamondx]{newtxmath}
\usepackage{xfrac}

\usepackage{layout}

	%レイアウト調整(B5,12Q,escoffierltjsbook.cls)
%
\setlength{\hoffset}{-1truein}
\setlength{\hoffset}{5mm}
\setlength{\oddsidemargin}{0pt}
\setlength{\evensidemargin}{-1cm}
\setlength{\textwidth}{\fullwidth}%%ltjsclassesのみ有効
\setlength{\fullwidth}{13cm}
\setlength{\textwidth}{13cm}
\setlength{\marginparsep}{0pt}
\setlength{\marginparwidth}{0pt}
\setlength{\footskip}{0pt}
\setlength{\textheight}{20.5cm}
%%%ベースライン調整
%\ltjsetparameter{yjabaselineshift=0pt,yalbaselineshift=-.75pt}

%レイアウト調整(8pt,a5j,escoffierltjsbook)
%\setlength{\voffset}{-.5cm}
%\setlength{\hoffset}{-.6cm}
%\setlength{\oddsidemargin}{0pt}
%\setlength{\evensidemargin}{\oddsidemargin}
%\setlength{\textwidth}{\fullwidth}%%ltjsclassesのみ有効
%\setlength{\fullwidth}{40\zw}
%\setlength{\textwidth}{40\zw}
%\setlength{\marginparsep}{0pt}
%\setlength{\marginparwidth}{0pt}
%\setlength{\footskip}{0pt}
%\setlength{\textheight}{17.5cm}
%%%ベースライン調整
%\ltjsetparameter{yjabaselineshift=0pt,yalbaselineshift=-.75pt}
%\setlength{\baselineskip}{15pt}


\def\tightlist{\itemsep1pt\parskip0pt\parsep0pt}

%リスト環境
\makeatletter
  \parsep   = 0pt
  \labelsep = 1\zw
  \def\@listi{%
     \leftmargin = 0pt \rightmargin = 0pt
     \labelwidth\leftmargin \advance\labelwidth-\labelsep
     \topsep     = 0pt%\baselineskip
     \topsep -0.1\baselineskip \@plus 0\baselineskip \@minus 0.1 \baselineskip
     \partopsep  = 0pt \itemsep       = 0pt
     \itemindent = 0pt \listparindent = 0\zw}
  \let\@listI\@listi
  \@listi
  \def\@listii{%
     \leftmargin = 1\zw \rightmargin = 0pt
     \labelwidth\leftmargin \advance\labelwidth-\labelsep
     \topsep     = 0pt \partopsep     = 0pt \itemsep   = 0pt
     \itemindent = 0pt \listparindent = 1\zw}
  \let\@listiii\@listii
  \let\@listiv\@listii
  \let\@listv\@listii
  \let\@listvi\@listii
\makeatother


  
%\usepackage{fancyhdr}

\usepackage{setspace}
\setstretch{1.15}


%レシピ本文
\usepackage{multicol}

\newenvironment{recette}{\begin{small}\begin{spacing}{1}\begin{multicols}{2}}{\end{multicols}\end{spacing}\end{small}}
%\newenvironment{recette}{\begin{multicols}{2}}{\end{multicols}}


%subsubsectionに連番をつける
%\usepackage{remreset}

\renewcommand{\thechapter}{}
\renewcommand{\thesection}{}
\renewcommand{\thesubsection}{}
\renewcommand{\thesubsubsection}{}
\renewcommand{\theparagraph}{}

%\makeatletter
%\@removefromreset{subsubsection}{subsection}
%\def\thesubsubsection{\arabic{subsubsection}.}
%\newcounter{rnumber}
%\renewcommand{\thernumber}{\refstepcounter{rnumber} }

\renewcommand{\prepartname}{\if@english Part~\else {}\fi}
\renewcommand{\postpartname}{\if@english\else {}\fi}
\renewcommand{\prechaptername}{\if@english Chapter~\else {}\fi}
\renewcommand{\postchaptername}{\if@english\else {}\fi}
\renewcommand{\presectionname}{}%  第
\renewcommand{\postsectionname}{}% 節

\makeatother



% PDF/X-1a
% \usepackage[x-1a]{pdfx}
% \Keywords{pdfTeX\sep PDF/X-1a\sep PDF/A-b}
% \Title{Sample LaTeX input file}
% \Author{LaTeX project team}
% \Org{TeX Users Group}
% \pdfcompresslevel=0
%\usepackage[cmyk]{xcolor}

%biblatex
%\usepackage[notes,strict,backend=biber,autolang=other,%
%                   bibencoding=inputenc,autocite=footnote]{biblatex-chicago}
%\addbibresource{hist-agri.bib}
\let\cite=\autocite

% % % % 
\date{}

%%%脚注番号のページ毎のリセット
%\makeatletter
%  \@addtoreset{footnote}{page}
%\makeatother
\usepackage[perpage,marginal,stable]{footmisc}
\makeatletter
\renewcommand\@makefntext[1]{%
  \advance\leftskip 1.5\zw
  \parindent 1\zw
  \noindent
  \llap{\@thefnmark\hskip0.5\zw}#1}


\renewenvironment{theindex}{% 索引を3段組で出力する環境
    \if@twocolumn
      \onecolumn\@restonecolfalse
    \else
      \clearpage\@restonecoltrue
    \fi
    \columnseprule.4pt \columnsep 2\zw
    \ifx\multicols\@undefined
      \twocolumn[\@makeschapterhead{\indexname}%
      \addcontentsline{toc}{chapter}{\indexname}]%変更点
    \else
      \ifdim\textwidth<\fullwidth
        \setlength{\evensidemargin}{\oddsidemargin}
        \setlength{\textwidth}{\fullwidth}
        \setlength{\linewidth}{\fullwidth}
        \begin{multicols}{3}[\chapter*{\indexname}
	\addcontentsline{toc}{chapter}{\indexname}]%変更点%
      \else
        \begin{multicols}{3}[\chapter*{\indexname}
	\addcontentsline{toc}{chapter}{\indexname}]%変更点%
      \fi
    \fi
    \@mkboth{\indexname}{\indexname}%
    \plainifnotempty % \thispagestyle{plain}
    \parindent\z@
    \parskip\z@ \@plus .3\p@\relax
    \let\item\@idxitem
    \raggedright
    \footnotesize\narrowbaselines
  }{
    \ifx\multicols\@undefined
      \if@restonecol\onecolumn\fi
    \else
      \end{multicols}
    \fi
    \clearpage
  }
\makeatother


\makeindex

\begin{document}

%\layout


%fancyhdr
%\pagestyle{fancy}
%\lhead[\thepage]{\thesection}
%\chead{}
%\rhead[\thechapter]{\thepage}
%\fancyhead{\gdef\headrulewidth{0pt}}
%\lfoot{}
%\cfoot{}
%\rfoot{}





\chapter{総序}\label{ux7dcfux5e8f}

もう20年も前のことだ。本書の着想を我が尊敬する師、今は亡きユルバン・デュ
ボワ\footnote{Urbain Dubois (1818〜1901)。19世紀後半を代表する料理人。}先生に話したのは。先生は\ruby{是非}{ぜひ}とも実現させなさいと
強く勧めてくださった。けれども忙しさにかまけてしまい、\ruby{漸}{ようや}く
1898年になって、フィレアス・ジルベール\footnote{Philéas Gilbert
  (1857〜1942)。19世紀末から20世紀初頭に活躍した料
  理人。料理雑誌「ポトフ」を主宰した。}君と話し合い協力をとりつける
ことが出来た。ところがまもなく、カールトンホテル開業のために私はロンド
ンに呼び戻され、その厨房の準備や運営に忙殺されることとなった\footnote{エスコフィエはセザール・リッツの経営するホテルグループにおいて料
  理に関わる重要な役割を一手に担っていた。1890年〜1997年にかけてロン
  ドンのサヴォイホテルの総料理長を勤めた後、1898年にはパリのオテル・
  リッツの、1899年にはロンドンのカールトンホテルの開業に携わり、1920
  年までカールトンホテルで総料理長を務めた。}。本書
の計画を実現させるために落ち着いた時間を取り戻さねばならなくなってしまっ
た。

1898年から放ったらかしにしてしまっていた本書に再び着手出来たのは、多く
の同僚たる料理人諸君の助力と、友人でもあるフィレアス・ジルベール君とエ
ミール・フェチュ\footnote{Emile Fétu 生没年不詳。}君の献身的な協力を得られたからに他ならない。この
一大事業を完成させることが出来たのは、ひとえに皆の励ましと、とりわけ辛
抱強く、粘り強く仕事を手伝ってくれた二人の共著者\footnote{ジルベールとフェチュを指しているが、初版には、この二人の他にも共著者
  として4人の名が挙げられている。第二版以降は共著者としてジルベールと
  フェチュの名しかクレジットされていない。第二版は初版から構成も含め
  大幅な改訂が行なわれた。その作業を実際に行なったのがジルベールとフェ
  チュだったために、他の共著者のクレジットが抹消されたと考えられる。}のおかげだ。

私が作りたいと思ったのは立派な書物というよりはむしろ実用的な本だ。だか
ら、執筆協力者の皆には、作業手順を各自の考えにもとづいて自由にレシピを
書いてもらい、私自身は、40年にわたる現場経験に即して、少なくとも原理原
則、料理における伝統的基礎を明確に説明するのに専念した。

本書は、かつて私が構想したとおりとは言い難い出来だが、いずれはそうなる
べく努めねばなるまい。それでもなお、現状でも料理人諸君にとって大いに役
立つものと信じている。だからこそ、本書を誰にでも、とりわけ若い料理人に
も買える価格にした\footnote{1903年の初版の売価は、\href{http://gallica.bnf.fr/ark:/12148/bpt6k65768837}{フランス国立図書館蔵}のものの表紙には、フランス国内で12フランと記したシールが貼られている。また、\href{https://archive.org/details/b21525912}{リーズ大学図書館蔵の第二版}にも同様に国内売価12フランのシールが貼られている。\\
  辻静雄は「1903年の初版発売当時は、800ページでたった8フラン、全く破格の値段だった」(「エスコフィエ 偉大なる料理人の生涯」、『辻静雄著作集』、新潮社、1995年、729〜730頁)と記しているが、8フランという価格の典拠は示されていない。現在と当時の通貨価値、物価の違いが分りにくいため、この「破格に安い」という言葉にはやや疑問が残るだろう。1900年当時の書籍広告において『料理の手引き』初版と同様の八折り版800ページの料理書が、フランス装10フラン、厚紙の表紙のものが11フランとある。初版の12フランという価格は、むしろ料理書としては一般的か、もしくはやや高めだったとも考えられる。なお、八折り判というのは書籍の大きさを表す用語で、概ね縦20〜25cm、横12〜16cm程度。\\
  この序文でことさらに「実用性」や入手しやすい価格であることが強調さ
  れているのは、何度も言及されているデュボワとベルナールの名著『古典料
  理』が四折り判(概ね縦45cm、横30cm)の豪華本であったことを意識して
  いたためとも推測されよう。}。そもそも若い料理人諸君にこそこの本を読んで
\ruby{貰}{もら}いたい。今はまだ初心者であったとしても、20年後には組織
のトップに立つべき人材なのだから。

私はこの本を豪華な装丁の\footnote{かつてフランスでは、大判の紙の両面に印刷して折ったものを糸で綴じ
  ただけの状態(いわゆる「フランス装」)で販売された本を、書店で買い
  求めた者が別途、業者に製本、装丁させることが一般的に行なわれていた。}、書棚の飾りのごときにはして欲しくない。そ
うではなく、いつでも、どんな時でも手元に置いて、分からないことを常に明
らかにしてくれる\ruby{盟友}{めいゆう}として欲しい。

本書には五千を越えるレシピが掲載されているが、それでも私は、この教本が
完全だとは思っていない。たとえ今この瞬間に完璧であったとしても、明日に
はそうではないかも知れぬ。料理は進化し、新しいレシピが日々創案されてい
るのだ。まことにもって不都合なことだが、版を重ねる毎に新しい料理を採り
入れ、古くなってしまったものは改良を加えねばなるまい。

ユルバン・デュボワ、エミール・ベルナール\footnote{Emile Bernard
  (1827〜1897)。クラシンスキ将軍の料理人を務めた。}両氏の著作\footnote{デュボワとベルナールの共著は他にもあるが、ここでは『古典料理』
  (1856年)を指している。}に昔から慣
れ親しみ、その巨大な影がなおも料理の地平を覆い尽している現在、私として
は本書がその後継になって欲しいと思っている。カレーム以後、最高の料理の
高みに逹した二人に対し、ここであらためて心から敬意を表させていただきた
いと思う。

調理現場を取り巻く諸事情により、私は、デュボワ、ベルナール両氏がもたら
したサービス(給仕)面での革新\footnote{19世紀後半に少しずつ一般的となった「ロシア式サービス」のこと。そ
  れまで中世以来、正式な宴席では卓上に大皿盛られた料理が一度に何種も
  並べられ、食べる者がそれぞれ好きなように取り皿に移して食べていた。
  これを、献立を食べる順に1種ずつ、大皿料理の場合は食べ手に見せて回っ
  てから、給仕が取り分けて供する方式に改めたものがロシア式サービスで
  ある。食卓に大皿を並べない代わりに、花を飾りナフキンを美しく折るな
  どの工夫により卓上が洗練されたものになっていった。19世紀パリに駐在
  していたロシア帝国の外交官クラーキンが提唱したと言われている。デュ
  ボワとベルナールの『古典料理』序文において詳述されている。}に対し、こんにちのようなとりわけスピー
ドが重視される目まぐるしい生活リズムに合わせて、大きな変更を加えざるを
得なかった。そもそも物理的理由から、料理を載せる飾り台\footnote{socle
  ソークル。パンや米、ジュレなどで作った、料理を盛り付ける
  ために銀の盆の上に据える飾り台。カレームの時代、つまり19世紀前半に
  はその装飾に凝ることが多かった。食べもので作られてはいるが、料理の
  一部ではなく、あくまで装飾的要素でしかなかった。}をやめて、
シンプルな盛り付けにする新たなメソッドと新たな道具を考案する必要があっ
たのだ。デュボワ、ベルナール両氏が推奨した壮麗な盛り付けを私自身も行なっ
ていた頃はもちろん、今なお二方の思想にはまったく共感している。冗談でこ
んなことを言っているのではない。しかし、カレームを信奉する者たちは、装
飾の才があるが\ruby{故}{ゆえ}に、時代にもはや\ruby{似}{そぐ}わなくなっ
てしまった作品に対して改良を加えようとはしなかった。時代に合わせて改良
することこそ、まさに重要なのに。本書で奨励している盛り付けは、少なくと
もそれなりの期間、有用であり続けると思う。全ては変化する。姿を変える。
それなのに、装飾芸術の役割が変化しないと主張するなどとは\ruby{蒙昧}{も
うまい}ではないか。芸術は流行によって栄えるものだし、流行のように移
ろいやすいものだ。

だが、カレームの時代にはこんにちと同じく\ruby{既}{すで}にあり、料理が
続く限りなくならないだろうものがある。それが料理のベースとなるフォンや
ストックだ。そもそも、料理が見た目にシンプルになっても料理そのものの価
値は失なわれないが、その逆はどうだろう? 人々の味覚は絶え間なく洗練さ
れ続け、それを満足させるために料理そのものも洗練されることになる。こん
にちの余剰活動が精神におよぼす悪影響に打ち\ruby{克}{か}つためには、料
理そのものがいっそう科学的な、正確なものとなるべきなのだ。

その意味で料理が進歩すればする程、我々料理人たちにとって、19世紀、料理
の行く末に大きく影響を与えた三人の料理人の存在は大きなものとなるだろう。
カレームとデュボワ、ベルナールはともすれば技術的側面ばかり評価されるが、
料理芸術の基礎において何よりも優れているのだ。

既に物故した名だけ挙げるが、確かにグフェ\footnote{Jules Gouffé
  (1807〜1877)。ナポレオン三世の料理人として知られる。
  著書も多く、代表作『料理の本(1867年) は前
  半が家庭料理、後半が高級料理(オートキュイジーヌ)の二部構成となっ
  ており、レシピの書き方も、まず材料表を掲げた後に調理手順を説明する
  という現代のそれに近いものになっている。}、ファーヴル\footnote{Joséph
  Favre (1849〜1903)。スイス生まれの料理人で、パリ、ドイ
  ツ、イギリス、ベルギー等において腕を活躍した。『料理および食品衛生
  事典』(1884〜1895年)は、こんにちでも事典としての有用性が失な
  われていない。}、エルーイ\footnote{Edouard Hélouis(生没年不詳)。
  イギリスのアルバート王配(ヴィ
  クトリア女王の夫)およびサルデーニャ国王ヴィットーリオ・エマヌエー
  レ二世の料理人を務めた。著書『王室の晩餐』(1878年)。}、ルキュレ\footnote{『実践的料理』(1859年)の著者C.
  Reculet を指していると思われる。}はとても素晴らしい著作を残した。だが、『古典料理』という\ruby{稀代}{きだい}の名著に\ruby{比肩}{ひけん}し得るものはひとつとしてない。

料理人諸君に、新たに本書を使っていただくにあたり、言うべきことがある。
いろいろな料理書、雑誌を読み散らかすのもいいが、偉大な先達の不朽の名著
はしっかり読み込むように、と。\ruby{諺}{ことわざ}にあるように「知り過
ぎることなはい」のだ。学べば学ぶ程、さらに学ぶべきことは増えていく。そ
うすれば、柔軟な思考が出来るようになり、料理が上達するためのより効果的
な方法を知ることも出来るだろう。

本書を\ruby{上梓}{じょうし}するにあたって\ruby{唯}{ただ}ひとつ望むこと、
切に願う\ruby{唯一}{ゆいいつ}のことは、上記の点において、本書の対象た
る読者諸君が我が\ruby{言}{げん}に耳を傾け、実践するさまを見ることに尽きる。

\begin{flushright} A. エスコフィエ\end{flushright}

1902年11月1日


{\printindex}



\end{document}
