\usepackage{okumacro}
\usepackage[utf8]{inputenc}
%\usepackage[french,japanese]{babel}
%\usepackage{multicol}
%\usepackage[deluxe]{otf}
%\usepackage[T1]{fontenc}
%\usepackage{palatino}
%\usepackage{makeidx}
%\usepackage{setspace}
%\usepackage{layout}
%\usepackage[stable]{footmisc}
%\usepackage[unicode=false]{hyperref}
\usepackage[unicode=true]{hyperref}
%\usepackage{pxjahyper}
%\usepackage[dvipdfmx]{graphicx}
%\usepackage{microtype}
\usepackage{longtable}

\makeindex
%\hyperrefマクロを無効化
%
%\renewcommand{\hyperref}[2][]{#2}%単に無効化. パッケージが有効の場合
%
%%%%% pandoc による EPUB 変換用 %%%%%%
\renewcommand{\hyperref}[2][]{\href{##1}{#2}}
%%%%%%%%%%%%%%%%%% markdown2latexのバグは直したのにlatex2markdownはそのままだった?
%%%%%%%%%%%%%%%%%% Cf. Releases ver.1.9.3
%
%%%%%以下は無効%%%%%%%
%\renewcommand{\hyperref}[2][]{<a href="##1">#2<a>}
%\renewcommand{\hyperref}[2][]{<a href="#1">#2<a>}%pandocでepubにして
				 %から正しく置換
%\renewcommand{hyperref}[2]{[#2](##1)}
%
%\newcommand{\hyperref}[2][]{#2}%上でhyperrefパッケージを無効化している場合
%
%
%%%%%%%%%%%%%%%%%%

\setlength{\fullwidth}{41zw}
\setlength{\textwidth}{\fullwidth}
\setlength{\oddsidemargin}{-14pt}
\setlength{\evensidemargin}{-35pt}
\voffset=-3zw
\setlength{\columnseprule}{0pt}
\setlength{\parskip}{0pt}
\setlength{\topskip}{\Cht}
\setlength{\textheight}{39\baselineskip}
\addtolength{\textheight}{1zh}


\makeatletter%% プリアンブルで定義する場合は必須

\def\@makechapterhead#1{\hbox{}%
  \vskip2\Cvs
  {\parindent\z@
%  \raggedright% オリジナルの定義(左揃え)
   \centering% 中央揃え
%  \raggedleft% 右揃え
   \reset@font\huge\sffamily
   \ifnum \c@secnumdepth >\m@ne
     \setlength\@tempdima{\linewidth}%
     \vtop{\hsize\@tempdima%
       \if@mainmatter% ← report クラスの場合この行不要
          \@chapapp\thechapter\@chappos\\%
       \fi% ← report クラスの場合この行不要
     #1}%
   \else
     #1\relax
   \fi}\nobreak\vskip3\Cvs}

\renewcommand{\sectionmark}[1]{\markright{#1}{}} 

\renewcommand{\thesection}{}

\renewcommand{\section}{%
  \@startsection{section}% #1 見出し
   {1}% #2 見出しのレベル
   {-1\Cvs}% #3 横組みの場合,見出し左の空き(インデント量)
   {1.5\Cvs \@plus.5\Cdp \@minus.2\Cdp}% #4 見出し上の空き
   {1.5\Cvs \@plus.5\Cdp}% #5 見出し下の空き (負の値なら見出し後の空き)
      %{.5\Cvs \@plus.3\Cdp}% #5 見出し下の空き (負の値なら見出し後の空き)
%  {\reset@font\Large\bfseries}% #6 見出しの属性
  % {\raggedright\reset@font\LARGE\sffamily}%
   {\centering\reset@font\LARGE\sffamily}% 中央揃え
%  {\raggedleft\reset@font\Large\bfseries}% 右揃え
}%

\renewcommand{\subsection}{%
  \@startsection{subsection}% #1 見出し
   {2}% #2 見出しのレベル
   {\z@}% #3 横組みの場合,見出し左の空き(インデント量)
 %  {1zw}
  {0\Cvs \@plus0\Cdp \@minus0\Cdp}% #4 見出し上の空き
    %{1\Cvs \@plus1\Cdp \@minus1\Cdp}% #4 見出し上の空き
    {0\Cvs \@plus0\Cdp}% #5 見出し下の空き (負の値なら見出し後の空き) 
 %a  {\z@}%{\z@}
  {\reset@font\normalsize\sffamily\bfseries}% #6 見出しの属性
%   {\centering\reset@font\large\sffamily}% 中央揃え
%   {\raggedright\reset@font\normalsize\bfseries}% 右揃え
}%

\renewcommand{\subsubsection}{%
  \@startsection{subsubsection}% #1 見出し
   {3}% #2 見出しのレベル
   {\z@}% #3 横組みの場合,見出し左の空き(インデント量)
 %  {1zw}
{\z@}
%   {0\Cvs \@plus0\Cdp \@minus0\Cdp}% #4 見出し上の空き
   {0\Cvs \@plus0\Cdp}% #5 見出し下の空き (負の値なら見出し後の空き) 
  % {\z@}%{\z@}
  {\reset@font\normalsize\itshape}% #6 見出しの属性
%   {\centering\reset@font\large\bfseries}% 中央揃え
%   {\raggedright\reset@font\small\itshape}% 右揃え
}%


%\renewcommand{\subsection}{\@startsection{subsection}{2}{\z@}%
%    {\z@}{\z@}%
%    {\normalfont\normalsize\headfont}}

\@addtoreset{footnote}{page}


\newcommand{\subsubsubsection}{\@startsection{subsubsubsection}{4}{\z@}%
   {1\Cvs \@plus.5\Cvs \@minus.2\Cvs}%
   %{\z@}%
   {-1\Cvs\@plus0\Cvs\@minus.2\Cvs}%
   {\reset@font\normalsize\bfseries}}


\makeatother%% プリアンブルで定義する場合は必須

\newcommand{\fr}[1]{\noindent\small\it{#1}\normalfont}

\newenvironment{maintext}{\begin{spacing}{1.6}\fontsize{11.3824pt}{11.3824pt}\selectfont}{\end{spacing}}

\newenvironment{recette}{\begin{normalsize}\end{normalsize}}

\newenvironment{nota}{\vspace{.3zw}\begin{small}\bfseries\selectfont\begin{spacing}{1}}{\end{spacing}\end{small}}

\renewcommand{\subsubsection}[1]{#1}

\newcommand{\ab}{}
\renewcommand{\ruby}[2]{#1}%ルビを出力しないようにする
\renewcommand{\textsf}[1]{#1}%ゴチ指定をベタで出力


%%分数の表記

\newcommand{\undemi}{½}
\newcommand{\untiers}{⅓}
\newcommand{\deuxtiers}{⅔}
\newcommand{\unquart}{¼}
\newcommand{\troisquarts}{¾}
\newcommand{\uncinquieme}{⅕}
\newcommand{\deuxcinquiemes}{⅖}
\newcommand{\troiscinquiemes}{⅗}
\newcommand{\quatrecinquiemes}{⅘}
\newcommand{\unsixieme}{⅙}
\newcommand{\cinqsixiemes}{⅚}
