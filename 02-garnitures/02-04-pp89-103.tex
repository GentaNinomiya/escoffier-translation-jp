\hypertarget{garnitures-recettes}{%
\subsection{ガルニチュールのレシピ}\label{garnitures-recettes}}

\frsecb{Garnitures}

\begin{center}
\medlarge(ここで示す分量はすべて仕上がり10人分)
\end{center}
\normalsize
\begin{recette}
\hypertarget{garniture-algerienne}{%
\subsubsection{ガルニチュール・アルジェリア風}\label{garniture-algerienne}}

\frsub{Garniture à l'Algérienne}

\index{garniture@garniture!algerienne@--- à l'Algérienne}
\index{algerien@algérien(nne)!garuniture à l'---ne}
\index{かるにちゆーる@ガルニチュール!あるしえりあふう@---・アルジェリア風}
\index{あるしえりあふう@アルジェリア風!かるにちゆーる@ガルニチュール・---}

(牛、羊の塊肉の料理に添える)

\begin{itemize}
\item
  ワインの栓の形にしたさつまいもの\protect\hyperlink{croquettes}{クロケット}10個。
\item
  小さなトマト10個は中をくり抜いて味付けをし、植物油少々で弱火で蒸し煮する。
\end{itemize}

\ul{ソース}\ldots{}\ldots{}薄く仕上げた\protect\hyperlink{sauce-tomate}{トマトソース}に、グリルして皮を剥き、細かい千切りにしたポワヴロン\footnote{いわゆる青果としてのパプリカ。}を加えたもの。

\hypertarget{garniture-alsacienne}{%
\subsubsection{ガルニチュール・アルザス風}\label{garniture-alsacienne}}

\frsub{Garniture à l'Alsacienne}

\index{garniture@garniture!alsacienne@--- à l'Alsacienne}
\index{alsacien@alsacien(ne)!garuniture à l'---ne}
\index{かるにちゆーる@ガルニチュール!あるさすふう@---・アルザス風}
\index{あるさすふう@アルザス風!かるにちゆーる@ガルニチュール・---}

(牛、羊の塊肉、牛フィレ、トゥルヌドに添える)

\begin{itemize}
\tightlist
\item
  ブレゼ\footnote{\protect\hyperlink{chou-braise}{キャベツのブレゼ}を参考にすること。}したシュークルート\footnote{専用品種の生食出来ないくらい固くて大きなキャベツを千切りにして香
    辛料とともに塩蔵、醗酵さたもの。ドイツのザワークラウトが原型だが、
    フランスとドイツで領土の取り合いとなったアルザス地方で独自に発展し
    た。温めたシュークルートにソーセージなどの豚肉加工品を添えた
    choucoûte barnie(シュークルートガルニ)はアルザスの名物料理のひと
    つ。なおシュークルート用の品種はQuintal d'Alsace(カンタルダルザ
    ス)が最良とされている。また、日本でも北海道で栽培され、鰊の漬物な
    どに使われる札幌大球甘藍という品種はこの系統が先祖らしい。。}を詰めてハムの脂身のないところを円く切ってのせたタルトレット10個。
\end{itemize}

\ul{ソース}\ldots{}\ldots{}\protect\hyperlink{jus-de-veau-lie}{とろみを付けた仔牛のジュ}。

\hypertarget{garniture-americaine}{%
\subsubsection[ガルニチュール・アメリケーヌ]{\texorpdfstring{ガルニチュール・アメリケーヌ\footnote{\protect\hyperlink{sauce-americaine}{ソース・アメリケーヌ}も参照されたい。}}{ガルニチュール・アメリケーヌ}}\label{garniture-americaine}}

\frsub{Garniture à l'Américaine}

\index{garniture@garniture!americaine@--- à l'Américaine}
\index{americain@américain(e)!garuniture à l'---e}
\index{かるにちゆーる@ガルニチュール!あめりけーぬ@---・アメリケーヌ}
\index{あめりかん@アメリカン/アメリケーヌ!かるにちゆーる@ガルニチュール・アメリケーヌ}

(魚料理に添える)

\begin{itemize}
\tightlist
\item
  このガルニチュールは必ず、\protect\hyperlink{homard-americaine}{オマール・アメリケー
  ヌ}の方法で調理した尾の身をやや斜めに1 cm程度の 薄切り\footnote{escalope
    (エスカロップ)肉などを筋線維と直角に、丸くスライスしたもの。}にして供する。
\end{itemize}

\ul{ソース}\ldots{}\ldots{}オマール・アメリケーヌのソース。

\hypertarget{garniture-andalouse}{%
\subsubsection[ガルニチュール・アンダルシア風]{\texorpdfstring{ガルニチュール・アンダルシア風\footnote{アンダルシア風、つまりスペイン風といいながら、ギリシャ風ライスを
  使うという点からも、料理名に付けられた地名がしばしば不確かで大雑把
  な理由さえないことが多いことが理解されよう。}}{ガルニチュール・アンダルシア風}}\label{garniture-andalouse}}

\frsub{Garniture à l'Andalouse}

\index{garniture@garniture!andalouse@--- à l'Andalouse}
\index{andalou@andalou(se)!garuniture à l'---se}
\index{かるにちゆーる@ガルニチュール!あんたるしあふう@---・アンダルシア風}
\index{あんたるしあふう@アンダルシア風!かるにちゆーる@ガルニチュール・---}

(牛、羊の塊肉料理や鶏料理に添える)

\begin{itemize}
\item
  中位の大きさのポワヴロン10個をグリル焼きして中をくり抜き、\protect\hyperlink{riz-grecque}{ギリシャ風ライス}を詰める。
\item
  なす\footnote{フランスで伝統的なタイプのなすはヘタが緑色で、風味や調理特性はい
    わゆる米なすに近いが、形状は比較的細長い。直径4〜6 cm、長さ25 cmく
    らいのものが多い。}を4
  cmの厚さの輪切りにして面取りをし、中に窪みをつくって油で揚げ、提供直前に油で炒めたトマトをのせる。
\end{itemize}

\ul{ソース}\ldots{}\ldots{}\protect\hyperlink{jus-de-veau-lie}{とろみを付けたジュ}。

\hypertarget{garniture-arlesienne}{%
\subsubsection[ガルニチュール・アルル風]{\texorpdfstring{ガルニチュール・アルル風\footnote{南フランスの都市
  Arles (アルル)の形容詞および名詞形。名詞の場
  合は「アルルの人」の意味になる。アルルはオランダ出身の印象派〜ポス
  ト印象派の画家フィンセント・ファン・ゴッホ Vincent van Gogh (フラ
  ンス語では昔からヴァンソンヴァンゴーグと呼ぶ習慣が付いてしまってお
  り、現代フランス語の原語発音尊重の風潮にもかかわらず、そのように発
  音されることは多いようだ)が1888年から1889年までアトリエを構え、
  「ひまわり」など多くの傑作を描いた。有名な、自分の耳を切り落すとい
  う「事件」を起こしたのもアルルでのことだ。この時期の作品のひとつに、
  「アルルの女(ジヌー夫人)」と呼ばれる一連のものがある。モデルはア
  ルルのカフェの経営者だといわれている。もっとも、フランスにおいて画
  家としてのゴッホおよび彼の作品は生前はほとんど評価されることがなく、
  生前に売れた絵は1枚だけだったとさえいわれている。このレシピは初版
  つまり1903年から収められているため、ゴッホの絵との関連はほぼないと
  考えていいだろう。むしろ、小説化アルフォンス・ドーデ原作を戯曲化し
  てジョルジュ・ビゼーが劇音楽を付けた『アルルの女』(1872年初演、
  1878年再演)との関連があると見るのがいいだろう。この作品は初演時点
  ではあまり好評ではなかったが、再演で大ヒットとなった。\protect\hyperlink{sauce-bohemienne}{ソース・ボ
  ヘミアの娘}のように、人気のある劇やオペラのタイ
  トルを料理名につけて、その人気にあやかろうという風潮が19世紀後半に
  は比較的多かった。そのため、トマトとなすという南フランスを思わせる
  食材を使ってはいてもアルルという土地に何の関係もないと思われる、内
  容的にも凡庸なこのガルニチュールに、当時の人気作品の名をつけて、い
  かにも流行のものであるかのように供したのが定着した、と考えることも
  可能だろう。その場合は「\ul{ガルニチュール・アルルの\\女}」と訳す
  べきかも知れない。なお、ビゼーが最初に作曲したのは27曲からなる舞台
  音楽であって、独立した音楽作品でもなければ、オペラでもなかったが、
  そのなかから数曲を選んで編曲し(あるいは作曲しなおし)、『アルルの
  女 組曲』としてこんにち広く知られている。第1組曲と第2組曲があり、
  前者はビゼー自身によるオーケストラ用編曲。後者はビゼーの死後1879年
  に友人エルネスト・ギローが完成させた。第1組曲の「メヌエット」や第2
  組曲の「ファランドール」など、曲名は知らずとも、メロディーを聴いた
  ことのある読者も少なくないとと思われる。}}{ガルニチュール・アルル風}}\label{garniture-arlesienne}}

\frsub{Garniture à l'Arlésienne}

\index{garniture@garniture!arlesienne@--- à l'Arlésienne}
\index{arlesien@arlésien(ne)!garuniture à l'---ne}
\index{かるにちゆーる@ガルニチュール!あるるふう@---・アルル風}
\index{あるるふう@アルル風!かるにちゆーる@ガルニチュール・---}

(トゥルヌドやノワゼットの料理に添える)

\begin{itemize}
\item
  なすは1
  cm程の厚さにスライスして塩こしょうをし、小麦粉をまぶして油で揚げる。
\item
  トマト皮を剥いてスライスし、バターでソテーする。
\item
  玉ねぎは輪切りにして指輪のようにばらばらにし、小麦粉をまぶして油で揚げ、花束のように盛る。
\end{itemize}

\ul{ソース}\ldots{}\ldots{}トマト風味の\protect\hyperlink{sauce-demi-glace}{ソース・ドゥミグラス}。

\hypertarget{garniture-banquiere}{%
\subsubsection[ガルニチュール・銀行家夫人風]{\texorpdfstring{ガルニチュール・銀行家夫人風\footnote{原文の
  à la Banquière をここでは文字通り訳した。料理名において {[}à la +
  形容詞の女性形{]}は通常、à la manière/façon 〜のmanièreも
  しくはfaçonが省力されたものと考えられている。これらmanière, façon
  いずれも女性名詞であるために、この後に付ける形容詞も女性形となる。
  ところが「〜風」」「〜を記念して/〜を称揚して」の意味で{[}à la +
  (固有)名詞{]}という用法もある。これは à la manière de + 名詞、の
  manière deが省略されたものと考える。Banquier(ボンキエ)は「銀行家」
  を意味する名詞であり、女性の場合はbanquièreとなり、女性銀行家ある
  いは銀行家夫人ということになる。そのため、従来は「銀行家風」と訳さ
  れていたが、あえて文法の原則に忠実に「銀行家夫人風」を訳した。さて、
  この料理名だが、日仏料理協会編『フランス 食の事典』(白水社、2000
  年)には「産業革命に伴う産業の隆盛を支えた銀行は、現代にいたるまで
  資本主義社会の根幹をなすもので、その経営者は19世紀において金持ちの
  代名詞ともなった。当時、「銀行家風」は王風、王妃風にかわる新しい表
  現だった(pp.162-163」と説明されている。ところが、料理書においてこ のà
  la Banquièreという表現は1856年のデュボワ、ベルナール共著『古典
  料理』以前には見つからない。しかも、「冷製料理用ガルニチュール・銀
  行家夫人風」Garniture à la banquière, pour froid (t.1, p.259)およ
  び「若鶏のガランティーヌ・銀行家夫人風」Galantine de poulet à la
  banquière (t.2, p.40)の2つでのみ料理名に使われているのみ。ガルニ
  チュールの概要は、オマール2尾の身をやや斜めの円形(エスカロップ)
  にスライスする。これをひとつずつ別々の陶製の器に入れ、小さなアーティ
  チョークの基底部を茹でたもの、大きな黒または白トリュフのスライス、
  マッシュルームのスライス、コルニションのスライスを盛り込み、塩、こ
  しょう、植物油、パセリとエストラゴンのみじん切りで味付けし、銘々に
  供する、というもの。本書のガルニチュールと温製、冷製の違いはあって
  も、同じ名称とは思い難いくらい異なった内容。その前後および以前につ
  いては、毎年のように版を重ねながら増補されたために料理の流行、変遷
  を見るのに非常に便利なヴィアールにもオドにも収録されておらず、グフェ
  『料理の本』(1867年)にも見あたらない。本書よりやや時代が下って、
  1838年の『ラルース・ガストロノミック』初版の「ガルニチュール・銀行
  家夫人風」は「鶏、仔牛胸腺肉(リドヴォー)の料理、ヴォロヴァン用。
  クネル、マッシュルーム、トリュフのスライス、ソース・バンキエール
  (p.136)」と定義されている。ソース・バンキエールsauce banauièreにつ
  いては「卵料理、鶏料理、牛や羊の副生物(リドヴォーなど)、ヴォロヴァ
  ン用。ソース・シュプレーム2 dlにマデイラ酒\undemi{} dlを加え、布で
  漉す。トリュフのみじん切り大さじ2杯を加えて仕上げる(p.959)」とある。
  2007年版の『ラルース・ガストロノミック』でもほぼ同様の内容だが、ソー
  ス・バンキエールのレシピはこの版では欠落している。また、20世紀につ
  いても、1950年に刊行されたレシピ集『フランス料理技法』(Flammarion)
  にソース・バンキエールのレシピは見られるが(p.147)、これはモンタニェ
  の『料理大全』\emph{Le grand livre de la cuisine},
  1929からの引用であり、
  ガルニチュール・バンキエールについては何も出ていない。1952年のペラ
  プラ『近代料理技術』にも、1953年のキュルノンスキー編『フランスの料
  理とワイン』にもこれらへの言及なない。ところが2018年現在、インター
  ネットで検索するとpoularde à la banquière 肥鶏・女銀行家風、のよう
  な、ここで見てきたものとはかなり内容の違うレシピが見つかる。「銀行
  家風」にしろ「女銀行家風」「銀行家夫人風」にしろ、銀行家という語に
  は肯定的な「富の象徴」というイメージがあると同時に、「\ruby{吝嗇}
  {りんしょく}\}家」あるいは「カネ貸し」のようなマイナスイメージが伴
  なわれ得ることもまた事実だろうし、銀行家が出席している宴席で「銀行
  家風」の料理を出す場合にはいろいろな誤解やトラブルの原因となる可能
  性さえあるかも知れない。このことから、『ラルース・ガストロノミック』
  が初版から2007年版までほぼ記述を変えなかった、つまり誰もこの名称の
  ガルニチュールに手を加えなかった、ということの証左ともなろう。}}{ガルニチュール・銀行家夫人風}}\label{garniture-banquiere}}

\frsub{Garniture à la Banquière}

\index{garniture@garniture!banauiere@--- à la Banquière}
\index{banquier@banquier(ère)!garuniture à la Banquière}
\index{かるにちゆーる@ガルニチュール!きんこうかふしんふう@---・銀行家夫人風}
\index{きんこうかふしんふう@銀行家夫人風!かるにちゆーる@ガルニチュール・---}

(肥鶏の料理に添える)

\begin{itemize}
\item
  ひばり\footnote{mauviette
    (モヴィエット)、ひばりの食材としての名称。生物とし
    てはalouette(アルエット)と呼ぶ。なお、オルレアネ地方の郷土料理に、
    pithiviers de mauviettes という、脳と鶏のファルスを詰めたひばりを
    折込みパイ生地で包んで焼いた料理があるが、pithiviers(ピティヴィエ)
    とだけ言う場合は、バターと砂糖、アーモンドパウダーなどを折込みパイ
    生地で包んで上部を渦巻模様に装飾したオルレアネ地方発祥の菓子を指す
    ので注意。}10羽を背側から開いて骨をすべて取り除き、\protect\hyperlink{farce-gratin-c}{ファルス・グラタ
  ン}を詰めて、表面を色よく焼き、カスロールで火を通す\footnote{en
    casserole (オンカスロール)カスロール仕立てと解釈も可能。
    \protect\hyperlink{sauce-smitane}{ソース・スミターヌ}訳注参照。}。
\item
  \protect\hyperlink{farce-b}{鶏のファルス}で小さなクネル10個。
\item
  トリュフのスライス10枚。
\end{itemize}

\ul{ソース}\ldots{}\ldots{}トリュフエッセンスを加えた\protect\hyperlink{sauce-demi-glace}{ソース・ドゥミグラス}

\hypertarget{garniture-berrichonne}{%
\subsubsection[ガルニチュール・ベリー風]{\texorpdfstring{ガルニチュール・ベリー風\footnote{berrichon(ベリション)
  はフランス中央部にある地方名 Berry の形 容詞。ここでは女性形
  berrichonne(ベリショーヌ)となる。山羊乳のチー
  ズで有名。なおフランス史関連の書物ににおいてよく見かける、ベリー公 duc
  de Berry (デュックドベリー)という公爵位はフランスの王族(つ
  まりその時の王の近縁者)に与えられた爵位で、その後フランス王となっ
  た者も多い。このため、いわゆる「世襲」はされてこなかった。また、中
  世フランスでもっとも豪華で美しい写本のひとつ『ベリー公のいとも豪華
  なる時祷書』\href{http://gallica.bnf.fr/ark:/12148/btv1b520004510}{\emph{Les
  Très Riches Heures du Duc de Berry}}(14世紀)
  は当時のベリー公ジャン1世が作成させたもので、美術史的にも重要なも の。}}{ガルニチュール・ベリー風}}\label{garniture-berrichonne}}

\frsub{Garniture à la Banquière}

\index{garniture@garniture!berrichonne@--- à la Berrichonne}
\index{berrichon@berrichon(ne)!garuniture à la Berrichonne}
\index{かるにちゆーる@ガルニチュール!へりーふう@---・ベリー風}
\index{へりーふう@ベリー風!かるにちゆーる@ガルニチュール・---}

(牛、羊肉の大がかりな料理\footnote{ルルヴェ relevé
  のこと。\protect\hyperlink{releve}{第二版序文訳注}参照。}に添える)

\begin{itemize}
\item
  卵の大きさにした\protect\hyperlink{chou-braise}{サヴォイキャベツ}のブレゼ20個。
\item
  キャベツとともに火を通した塩漬け豚バラ肉の小さなスライス10枚。
\item
  小玉ねぎ20個と大粒のマロン20個はこのガルニチュールを添える肉の煮汁で火を通す。
\end{itemize}

\ul{ソース}\ldots{}\ldots{}アロールート\footnote{Allow-root
  南米産クズウコンを原料とした良質のでんぷん。現代の日
  本ではコーンスターチで代用することがほとんど。}でとろみを付けた、ブレゼの煮汁。

\hypertarget{garniture-berny}{%
\subsubsection[ガルニチュール・ベルニ]{\texorpdfstring{ガルニチュール・ベルニ\footnote{ピエール・ド・ベルニPierre
  de Bernis (1715〜1794)のこと。なぜか
  料理名としてはBernyの綴りが一般的だが、個人名なのでもちろん誤り。
  29才でアカデミーフランセーズに入った俊才。ポンパドゥール夫人の庇護
  のもとルイ15世からも重用された。駐ヴェネツィア大使として食卓外交を
  展開し、フランス革命後、ローマで客死した。}}{ガルニチュール・ベルニ}}\label{garniture-berny}}

\frsub{Garniture à la Berny}

\index{garniture@garniture!berny@--- à la Berny}
\index{Berny@Berny (Bernis)!garuniture à la Berny}
\index{かるにちゆーる@ガルニチュール!へるに@---・ベルニ}
\index{へるに@ベルニ!かるにちゆーる@ガルニチュール・---}

(ジビエおよびマリネした牛、羊肉料理に添える)

\begin{itemize}
\item
  ワインの栓の形にしたじゃがいものクロケット・ベルニ\footnote{本書の温製オードブルの節に「クロケット・ベルニ」は掲載されてい
    ない。野菜料理の章にある「\protect\hyperlink{pommes-de-terre-berny}{じゃがいも・ベル
    ニ}」をアパレイユとしてクロケットを作るこ とになる。}10個。
\item
  空焼きしたタルトレットにバターを加えたマロンのピュレをドーム状に詰め、
  バターで軽くソテーして艶を出させたトリュフのスライスをタルトレットに
  1枚ずつのせる。
\end{itemize}

\ul{ソース}\ldots{}\ldots{}軽く仕上げた\protect\hyperlink{sauce-poivrade}{ソース・ポワヴラード}。

\hypertarget{garniture-bezontinne}{%
\subsubsection[ガルニチュール・ブザンソン風]{\texorpdfstring{ガルニチュール・ブザンソン風\footnote{Besonçon
  (ブゾンソン)フランス東部、ブルゴーニュ=フランシュ=コ
  ンテ圏の都市。形容詞は通常bisontin(e)(ビゾンタン/ビゾンティーヌ)
  だが、本書のようにbizontin(e)と綴ることもある。}}{ガルニチュール・ブザンソン風}}\label{garniture-bezontinne}}

\frsub{Garniture à la Bizontine}

\index{garniture@garniture!bizontinne@--- à la Bizontine}
\index{bizontin@bizontin(e) ⇒ bisontin(e)!garuniture à la ---e}
\index{かるにちゆーる@ガルニチュール!ふさんそんふう@---・ブザンソン風}
\index{ふさんそん@ブザンソン!かるにちゆーる@ガルニチュール・---風}

(牛、羊の塊肉料理およびトゥルヌドに添える)

\begin{itemize}
\item
  \protect\hyperlink{croustade-en-pomme-duchesse}{クルスタード・ポム・デュシェス}\footnote{\protect\hyperlink{pomme-de-terre-duchesse}{ポム・デュシェス}をバターを塗ったダ
    リオル型(小さな円筒形の型)を使って整形してからイギリス式パン粉衣
    を付けて油で揚げ、中をくり抜いてケースにする。詳細は温製オードブル
    の節参照。}10個 は提供直前にドリュール\footnote{色艶よく焼き上げるたに卵黄を溶いたもの、あるいは卵黄に水を加え
    て溶いたものをdorure(ドリュール)と呼び、それを塗ることを
    dorer(ドレ)という動詞で表現する。}を塗り、オーブンに入れて色よく焼く。生クリー
  ムを加えたカリフラワーのピュレを詰めてクルスタードの中に絞り袋を使って詰める。
  る。
\item
  半割りにした\protect\hyperlink{laitues-farcies-pour-garniture}{レチュの芯に近いところをファルシにしてブレゼしたも
  の}\footnote{本来は、日本で「サラダ菜」と呼ばれているものとおなじ品種系統の
    結球レタスの芯に近い部分を使う。日本のレタスとはまったく調理特性が
    異なるので注意。}10個。
\end{itemize}

\ul{ソース}\ldots{}\ldots{}バターを加えて仕上げた\protect\hyperlink{jus-de-veau-lie}{とろみを付けたジュ}。
\end{recette}