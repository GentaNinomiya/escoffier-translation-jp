\href{原稿下準備なし}{} \href{訳と注釈\%2020180405進行中}{}
\href{未、原文対照チェック}{} \href{未、日本語表現校正}{}
\href{未、その他修正}{} \href{未、原稿最終校正}{}

\hypertarget{garnitures-recettes}{%
\subsection{ガルニチュールのレシピ}\label{garnitures-recettes}}

\frsecb{Garnitures}

\begin{center}
\medlarge(ここで示す分量はすべて仕上がり10人分)
\end{center}
\normalsize
\begin{recette}
\hypertarget{garniture-algerienne}{%
\subsubsection{ガルニチュール・アルジェリア風}\label{garniture-algerienne}}

\frsub{Garniture à l'Algérienne}

\index{garniture@garniture!algerienne@--- à l'Algérienne}
\index{algerien@algérien(nne)!garuniture à l'---ne}
\index{かるにちゆーる@ガルニチュール!あるしえりあふう@---・アルジェリア風}
\index{あるしえりあふう@アルジェリア風!かるにちゆーる@ガルニチュール・---}

(牛、羊の塊肉\footnote{原文 Pour les pièce de boucherie
  より正確に訳すなら、「肉屋
  (boucherie)が伝統的に扱かってきた、白身肉を除く畜産精肉、具体的には牛、羊(馬も含まれる)の塊肉であり、牛の場合は基本的にランプ、イチボに相当する部位、羊の場合は鞍下肉から腿上部にかけての部位」を塊のまま調理したものを意味する。フランス語のpièceには「小さい細切れ」の意味もあるのだが、長い間の習慣として、pièce
  de boeuf は牛の大きな塊肉を意味する用語として一般化している。}の料理に添える)

\begin{itemize}
\item
  ワインの栓の形にしたさつまいもの\protect\hyperlink{croquettes}{クロケット}10個
\item
  小さなトマト10個は中をくり抜いて味付けをし、植物油少々で弱火で蒸し煮する
\item
  ソース\ldots{}\ldots{}薄く仕上げた\protect\hyperlink{sauce-tomate}{トマトソース}に、グリルして皮を剥き、細かい千切りにしたポワヴロン\footnote{いわゆる青果としてのパプリカ。}を加える
\end{itemize}

\hypertarget{garniture-alsacienne}{%
\subsubsection{ガルニチュール・アルザス風}\label{garniture-alsacienne}}

\frsub{Garniture à l'Alsacienne}

\index{garniture@garniture!alsacienne@--- à l'Alsacienne}
\index{alsacien@alsacien(ne)!garuniture à l'---ne}
\index{かるにちゆーる@ガルニチュール!あるさすふう@---・アルザス風}
\index{あるさすふう@アルザス風!かるにちゆーる@ガルニチュール・---}

(牛、羊の塊肉、牛フィレ、トゥルヌドに添える)

\begin{itemize}
\item
  ブレゼ\footnote{\protect\hyperlink{chou-braise}{キャベツのブレゼ}を参考にすること。}したシュークルート\footnote{生食出来ないくらい固くて大きな専用品種であるキャベツを千切りにして香辛料などとともに塩蔵、醗酵さたもの。ドイツのザワークラウトが原型だが、歴史的にフランスとドイツで領土の取り合いとなったアルザス地方で独自に発展した。温めたシュークルートにソーセージなどの豚肉加工品を添えたchoucoûte
    garnie(シュークルートガルニ)はアルザスの名物料理のひとつ。}を詰めてハムの脂身のないところを円く切ってのせたタルトレット10個
\item
  ソース\ldots{}\ldots{}\protect\hyperlink{jus-de-veau-lie}{とろみを付けた仔牛のジュ}
\end{itemize}

\hypertarget{garniture-americaine}{%
\subsubsection[ガルニチュール・アメリケーヌ]{\texorpdfstring{ガルニチュール・アメリケーヌ\footnote{\protect\hyperlink{sauce-americaine}{ソース・アメリケーヌ}も参照されたい。}}{ガルニチュール・アメリケーヌ}}\label{garniture-americaine}}

\frsub{Garniture à l'Américaine}

\index{garniture@garniture!americaine@--- à l'Américaine}
\index{americain@américain(e)!garuniture à l'---e}
\index{かるにちゆーる@ガルニチュール!あめりけーぬ@---・アメリケーヌ}
\index{あめりかん@アメリカン/アメリケーヌ!かるにちゆーる@ガルニチュール・アメリケーヌ}

(魚料理に添える)

\begin{itemize}
\item
  このガルニチュールは必ず、\protect\hyperlink{homard-americaine}{オマール・アメリケーヌ}の方法で調理した尾の身をやや斜めに1
  cm程度の薄切り\footnote{escalope
    (エスカロップ)肉などを筋線維と直角に、丸くスライスしたもの。}にして供する
\item
  ソース\ldots{}\ldots{}オマール・アメリケーヌのソース
\end{itemize}

\hypertarget{garniture-andalouse}{%
\subsubsection[ガルニチュール・アンダルシア風]{\texorpdfstring{ガルニチュール・アンダルシア風\footnote{アンダルシア風、つまりスペイン風といいながら、ギリシャ風ライスを使うという点からも、料理名に付けられた地名がしばしば不確かで大雑把な理由さえないことが多いことが理解されよう。}}{ガルニチュール・アンダルシア風}}\label{garniture-andalouse}}

\frsub{Garniture à l'Andalouse}

\index{garniture@garniture!andalouse@--- à l'Andalouse}
\index{andalou@andalou(se)!garuniture à l'---se}
\index{かるにちゆーる@ガルニチュール!あんたるしあふう@---・アンダルシア風}
\index{あんたるしあふう@アンダルシア風!かるにちゆーる@ガルニチュール・---}

(牛、羊の塊肉料理や鶏料理に添える)

\begin{itemize}
\item
  中位の大きさのポワヴロン10個をグリル焼きして中をくり抜き、\protect\hyperlink{riz-grecque}{ギリシャ風ライス}を詰める
\item
  なす\footnote{フランスで伝統的なタイプのなすはヘタが緑色で、風味や調理特性はいわゆる米なすに近いが、形状は比較的細長い。直径4〜6
    cm、長さ25 cmくらいのものが多い。}を4
  cmの厚さの輪切りにして面取りをし、中に窪みをつくって油で揚げ、提供直前に油で炒めたトマトをのせる
\item
  ソース\ldots{}\ldots{}\protect\hyperlink{jus-de-veau-lie}{とろみを付けたジュ}
\end{itemize}

\hypertarget{garniture-arlesienne}{%
\subsubsection[ガルニチュール・アルル風]{\texorpdfstring{ガルニチュール・アルル風\footnote{南フランスの都市
  Arles
  (アルル)の形容詞および名詞形。名詞の場合は「アルルの人」の意味になる。アルルはオランダ出身の印象派〜ポスト印象派の画家フィンセント・ファン・ゴッホ
  Vincent van Gogh
  (フランス語では昔からヴァンソンヴァンゴーグと呼ぶ習慣が付いてしまっており、現代フランス語の原語発音尊重の風潮にもかかわらず、そのように発音されることは多いようだ)が1888年から1889年までアトリエを構え、「ひまわり」など多くの傑作を描いた。有名な、自分の耳を切り落すという「事件」を起こしたのもアルルでのことだ。この時期の作品のひとつに、「アルルの女(ジヌー夫人)」と呼ばれる一連のものがある。モデルはアルルのカフェの経営者だといわれている。もっとも、フランスにおいて画家としてのゴッホおよび彼の作品は生前はほとんど評価されることがなく、生前に売れた絵は1枚だけだったとさえいわれている。このレシピは初版つまり1903年から収められているため、ゴッホの絵との関連はほぼないと考えていいだろう。むしろ、小説化アルフォンス・ドーデ原作を戯曲化してジョルジュ・ビゼーが劇音楽を付けた『アルルの女』(1872年初演、
  1878年再演)との関連があると見るのがいいだろう。この作品は初演時点ではあまり好評ではなかったが、再演で大ヒットとなった。\protect\hyperlink{sauce-bohemienne}{ソース・ボヘミアの娘}のように、人気のある劇やオペラのタイトルを料理名につけて、その人気にあやかろうという風潮が19世紀後半には比較的多かった。そのため、トマトとなすという南フランスを思わせる食材を使ってはいてもアルルという土地に何の関係もないと思われる、内容的にも凡庸なこのガルニチュールに、当時の人気作品の名をつけて、いかにも流行のものであるかのように供したのが定着した、と考えることも可能だろう。その場合は「\ul{ガルニチュール・アルルの\\女}」と訳すべきかも知れない。なお、ビゼーが最初に作曲したのは27曲からなる舞台音楽であって、独立した音楽作品でもなければ、オペラでもなかったが、そのなかから数曲を選んで編曲し(あるいは作曲しなおし)、『アルルの女 組曲』としてこんにち広く知られている。第1組曲と第2組曲があり、前者はビゼー自身によるオーケストラ用編曲。後者はビゼーの死後1879年に友人エルネスト・ギローが完成させた。第1組曲の「メヌエット」や第2
  組曲の「ファランドール」など、曲名は知らずとも、メロディーを聴いたことのある読者も少なくないとと思われる。}}{ガルニチュール・アルル風}}\label{garniture-arlesienne}}

\frsub{Garniture à l'Arlésienne}

\index{garniture@garniture!arlesienne@--- à l'Arlésienne}
\index{arlesien@arlésien(ne)!garuniture à l'---ne}
\index{かるにちゆーる@ガルニチュール!あるるふう@---・アルル風}
\index{あるるふう@アルル風!かるにちゆーる@ガルニチュール・---}

(トゥルヌドやノワゼットの料理に添える)

\begin{itemize}
\item
  なす\footnote{なす、トマト、玉ねぎの分量は記されていないので適宜判断すること。}は1
  cm程の厚さにスライスして塩こしょうをし、小麦粉をまぶして油で揚げる
\item
  トマト皮を剥いてスライスし、バターでソテーする
\item
  玉ねぎは輪切りにして指輪のようにばらばらにし、小麦粉をまぶして油で揚げ、花束のように盛る
\item
  ソース\ldots{}\ldots{}トマト風味の\protect\hyperlink{sauce-demi-glace}{ソース・ドゥミグラス}
\end{itemize}

\hypertarget{garniture-banquiere}{%
\subsubsection[ガルニチュール・銀行家夫人風]{\texorpdfstring{ガルニチュール・銀行家夫人風\footnote{原文の
  à la Banquière をここでは文字通り訳した。料理名において {[}à la +
  形容詞の女性形{]}は通常、à la manière/façon
  〜のmanièreもしくはfaçonが省力されたものと考えられている。これらmanière,
  façon
  いずれも女性名詞であるために、この後に付ける形容詞も女性形となる。ところが「〜風」」「〜を記念して/〜を称揚して」の意味で{[}à
  la + (固有)名詞{]}という用法もある。これは à la manière de + 名詞、の
  manière
  deが省略されたものと考える。Banquier(ボンキエ)は「銀行家」を意味する名詞であり、女性の場合はbanquièreとなり、女性銀行家あるいは銀行家夫人ということになる。そのため、従来は「銀行家風」と訳されていたが、あえて文法の原則に忠実に「銀行家夫人風」を訳した。さて、この料理名だが、日仏料理協会編『フランス 食の事典』(白水社、2000
  年)には「産業革命に伴う産業の隆盛を支えた銀行は、現代にいたるまで資本主義社会の根幹をなすもので、その経営者は19世紀において金持ちの代名詞ともなった。当時、「銀行家風」は王風、王妃風にかわる新しい表現だった(pp.162-163」と説明されている。ところが、料理書においてこのà
  la
  Banquièreという表現は1856年のデュボワ、ベルナール共著『古典料理』以前には見つからない。しかも、「冷製料理用ガルニチュール・銀行家夫人風」Garniture
  à la banquière, pour froid (t.1,
  p.259)および「若鶏のガランティーヌ・銀行家夫人風」Galantine de poulet
  à la banquière (t.2,
  p.40)の2つでのみ料理名に使われているのみ。ガルニチュールの概要は、オマール2尾の身をやや斜めの円形(エスカロップ)にスライスする。これをひとつずつ別々の陶製の器に入れ、小さなアーティチョークの基底部を茹でたもの、大きな黒または白トリュフのスライス、マッシュルームのスライス、コルニションのスライスを盛り込み、塩、こしょう、植物油、パセリとエストラゴンのみじん切りで味付けし、銘々に供する、というもの。本書のガルニチュールと温製、冷製の違いはあっても、同じ名称とは思い難いくらい異なった内容。その前後および以前については、毎年のように版を重ねながら増補されたために料理の流行、変遷を見るのに非常に便利なヴィアールにもオドにも収録されておらず、グフェ『料理の本』(1867年)にも見あたらない。本書よりやや時代が下って、
  1838年の『ラルース・ガストロノミック』初版の「ガルニチュール・銀行家夫人風」は「鶏、仔牛胸腺肉(リドヴォー)の料理、ヴォロヴァン用。クネル、マッシュルーム、トリュフのスライス、ソース・バンキエール
  (p.136)」と定義されている。ソース・バンキエールsauce
  banquièreについては「卵料理、鶏料理、牛や羊の副生物(リドヴォーなど)、ヴォロヴァン用。ソース・シュプレーム2
  dLにマデイラ酒 \(\frac{1}{2}\)
  dLを加え、布で漉す。トリュフのみじん切り大さじ2杯を加えて仕上げる(p.959)」とある。
  2007年版の『ラルース・ガストロノミック』でもほぼ同様の内容だが、ソース・バンキエールのレシピはこの版では欠落している。また、20世紀についても、1950年に刊行されたレシピ集『フランス料理技法』(Flammarion)
  にソース・バンキエールのレシピは見られるが(p.147)、これはモンタニェの『料理大全』(1929年)からの引用であり、ガルニチュール・バンキエールについては何も出ていない。1952年のペラプラ『近代料理技術』にも、
  1953年のキュルノンスキー編『フランスの料理とワイン』にもこれらへの言及なない。ところが2018年現在、インターネットで検索するとpoularde
  à la banquière
  「肥鶏 女銀行家風」のような、ここで見てきたものとはかなり内容の違うレシピが見つかる。「銀行家風」にしろ「女銀行家風」「銀行家夫人風」にしろ、銀行家という語には肯定的な「富の象徴」というイメージがあると同時に、「\ruby{吝嗇} {りんしょく}家」あるいは「カネ貸し」場合によっては「官僚主義的」のようなマイナスイメージが伴なわれ得ることもまた事実だろうし、銀行家が出席している宴席で「銀行家風」の料理を出す場合にはいろいろな誤解やトラブルの原因となる可能性さえあるかも知れない。このことから、『ラルース・ガストロノミック』が初版から2007年版までほぼ記述を変えなかった、つまり誰もこの名称のガルニチュールに手を加えなかった、ということの証左ともなろう。}}{ガルニチュール・銀行家夫人風}}\label{garniture-banquiere}}

\frsub{Garniture à la Banquière}

\index{garniture@garniture!banauiere@--- à la Banquière}
\index{banquier@banquier(ère)!garuniture à la Banquière}
\index{かるにちゆーる@ガルニチュール!きんこうかふしんふう@---・銀行家夫人風}
\index{きんこうかふしんふう@銀行家夫人風!かるにちゆーる@ガルニチュール・---}

(肥鶏の料理に添える)

\begin{itemize}
\item
  ひばり\footnote{mauviette
    (モヴィエット)、ひばりの食材としての名称。生物としてはalouette(アルエット)と呼ぶ。なお、オルレアネ地方の郷土料理に、
    pithiviers de mauviettes
    という、脳と鶏のファルスを詰めたひばりを折込みパイ生地で包んで焼いた料理があるが、pithiviers(ピティヴィエ)とだけ言う場合は、バターと砂糖、アーモンドパウダーなどを折込みパイ生地で包んで上部を渦巻模様に装飾したオルレアネ地方発祥の菓子を指すので注意。}10羽を背側から開いて骨をすべて取り除き\footnote{désosser
    (デゾセ)。日本の調理現場でも比較的よく使われる用語。この語に含まれるosは「骨」のこと、déは「反対、除去」などを意味する接頭辞、erは動詞であることを示す語尾。したがって、文字どおり「骨を取り除く」の意になる。}、\protect\hyperlink{farce-gratin-c}{ファルス・グラタン}を詰めて、表面を色よく焼き、カスロールで火を通す\footnote{en
    casserole
    (オンカスロール)カスロール仕立てと解釈も可能。\protect\hyperlink{sauce-smitane}{ソース・スミターヌ}訳注参照。}
\item
  \protect\hyperlink{farce-b}{鶏のファルス}で小さなクネル10個
\item
  トリュフのスライス10枚
\item
  ソース\ldots{}\ldots{}トリュフエッセンスを加えた\protect\hyperlink{sauce-demi-glace}{ソース・ドゥミグラス}
\end{itemize}

\hypertarget{garniture-berrichonne}{%
\subsubsection[ガルニチュール・ベリー風]{\texorpdfstring{ガルニチュール・ベリー風\footnote{berrichon(ne)(ベリション/ベリショーヌ)
  はフランス中央部にある地方名 Berry の形容詞。ここでは女性形
  berrichonne(ベリショーヌ)となる。山羊乳のチーズで有名。なおフランス史関連の書物ににおいてよく見かける、ベリー公
  duc de Berry
  (デュックドベリー)という公爵位はフランスの王族(つまりその時の王の近縁者)に与えられた爵位で、その後フランス王となった者も多い。このため、いわゆる「世襲」はされてこなかった。また、中世フランスでもっとも豪華で美しい写本のひとつ『ベリー公のいとも豪華なる時祷書』\href{http://gallica.bnf.fr/ark:/12148/btv1b520004510}{\emph{Les
  Très Riches Heures du Duc de
  Berry}}(14世紀)は当時のベリー公ジャン1世が作成させたもので、美術史的にも重要。}}{ガルニチュール・ベリー風}}\label{garniture-berrichonne}}

\frsub{Garniture à la Berrreichonne}

\index{garniture@garniture!berrichonne@--- à la Berrichonne}
\index{berrichon@berrichon(ne)!garuniture à la Berrichonne}
\index{かるにちゆーる@ガルニチュール!へりーふう@---・ベリー風}
\index{へりーふう@ベリー風!かるにちゆーる@ガルニチュール・---}

(牛、羊肉の大がかりな料理\footnote{ルルヴェ relevé
  のこと。\protect\hyperlink{releve}{第二版序文訳注}参照。}に添える)

\begin{itemize}
\item
  卵の大きさにした\protect\hyperlink{chou-braise}{サヴォイキャベツのブレゼ}20個
\item
  キャベツとともに火を通した塩漬け豚バラ肉の小さなスライス10枚
\item
  小玉ねぎ20個と大粒のマロン20個はこのガルニチュールを添える肉の煮汁で火を通す
\item
  ソース\ldots{}\ldots{}アロールート\footnote{Allow-root
    南米産クズウコンを原料とした良質のでんぷん。現代の日本ではコーンスターチで代用することがほとんど。}でとろみを付けた、ブレゼの煮汁
\end{itemize}

\hypertarget{garniture-berny}{%
\subsubsection[ガルニチュール・ベルニ]{\texorpdfstring{ガルニチュール・ベルニ\footnote{ピエール・ド・ベルニPierre
  de Bernis
  (1715〜1794)のこと。なぜか料理名としてはBernyの綴りが一般的だが、個人名なのでもちろん誤り。
  29才でアカデミーフランセーズに入った俊才。ポンパドゥール夫人の庇護のもとルイ15世からも重用された。駐ヴェネツィア大使として食卓外交を展開したが、フランス革命後、ローマで客死した。}}{ガルニチュール・ベルニ}}\label{garniture-berny}}

\frsub{Garniture à la Berny}

\index{garniture@garniture!berny@--- à la Berny}
\index{Berny@Berny (Bernis)!garuniture à la Berny}
\index{かるにちゆーる@ガルニチュール!へるに@---・ベルニ}
\index{へるに@ベルニ!かるにちゆーる@ガルニチュール・---}

(ジビエおよびマリネした牛、羊肉料理\footnote{シュヴルイユ仕立てのこと。\protect\hyperlink{sauce-porvrade}{ソース・ポワヴラード}および\protect\hyperlink{marinade-crue-pour-viandes-de-boucherie-ou-venaison}{マリナード}参照。}に添える)

\begin{itemize}
\item
  ワインの栓の形にしたじゃがいものクロケット・ベルニ\footnote{本書の温製オードブルの節に「クロケット・ベルニ」は掲載されていない。野菜料理の章にある「\protect\hyperlink{pommes-de-terre-berny}{じゃがいも・ベルニ}」をアパレイユとしてクロケットを作ることになる。}10個
\item
  空焼きしたタルトレット10個にバターを加えたマロンのピュレをドーム状に詰め、バターで軽くソテーして艶を出させたトリュフのスライスをタルトレットに1枚ずつのせる
\item
  ソース\ldots{}\ldots{}軽く仕上げた\protect\hyperlink{sauce-poivrade}{ソース・ポワヴラード}。
\end{itemize}

\hypertarget{garniture-bezontinne}{%
\subsubsection[ガルニチュール・ブザンソン風]{\texorpdfstring{ガルニチュール・ブザンソン風\footnote{Besonçon
  (ブゾンソン)フランス東部、ブルゴーニュ=フランシュ=コンテ圏の都市。形容詞は通常bisontin(e)(ビゾンタン/ビゾンティーヌ)だが、本書のようにbizontin(e)と綴ることもある。なお、1980年代に画期的といわれたフランス語教材\emph{C'est
  le
  printemps}の第1課においてはじめて出てくる地名がブザンソンだった。この教材は会話例のリアリティや題材としてdocuments
  authentiques(ドキュモンオトンティック=現実にあるドキュメントすなわち言語を用いたさまざまな書類、看板、広告など)を積極的に採用したこととともに、アプレ68(フランスの学生運動および現代思想における転換期のひとつとなった1968年の「五月革命」以後に多方面において展開された時代特有の雰囲気)が強く表われているのが特徴だった。同時期のフランス語教材の傑作とされる(やや保守的な傾向の)通称「カペル」\emph{Le
  français en direct}
  と並び、フランス語教育・教授法において現在のEUおよびフランスで定められ運用されている「外国語としての言語コミュニケーション能力」の概念形成の先駆けとなった。アプレ68的なものは食文化、料理の世界においても、ゴ\&ミヨの批評と店の格付けにおける、既存のミシュランのガイドブックのオルタナティヴとしての方向性、ヌーヴェルキュイジーヌ宣言などによく表われている。}}{ガルニチュール・ブザンソン風}}\label{garniture-bezontinne}}

\frsub{Garniture à la Bizontine}

\index{garniture@garniture!bizontinne@--- à la Bizontine}
\index{bizontin@bizontin(e) ⇒ bisontin(e)!garuniture à la ---e}
\index{かるにちゆーる@ガルニチュール!ふさんそんふう@---・ブザンソン風}
\index{ふさんそん@ブザンソン!かるにちゆーる@ガルニチュール・---風}

(牛、羊の塊肉料理およびトゥルヌドに添える)

\begin{itemize}
\item
  \protect\hyperlink{croustade-en-pomme-duchesse}{クルスタード・ポム・デュシェス}\footnote{\protect\hyperlink{pomme-de-terre-duchesse}{ポム・デュシェス}をバターを塗ったダリオル型(小さな円筒形の型)をに詰めて整形してからイギリス式パン粉衣を付けて油で揚げ、中をくり抜いてケースにする。詳細は温製オードブルの節参照。}10個は提供直前にドリュール\footnote{色艶よく焼き上げるために卵黄を溶いたもの、あるいは卵黄に水を加えて溶いたものをdorure(ドリュール)と呼び、それを塗ることをdorer
    (ドレ)という動詞で表現する。}を塗り、オーブンに入れて色よく焼く。生クリームを加えたカリフラワーのピュレを詰めてクルスタードの中に絞り袋を使って詰める
\item
  半割りにした\protect\hyperlink{laitues-farcies-pour-garniture}{ガルニチュール用レチュのファルシ}10個
\item
  ソース\ldots{}\ldots{}バターを加えて仕上げた\protect\hyperlink{jus-de-veau-lie}{とろみを付けたジュ}
\end{itemize}

\hypertarget{garniture-boulangere}{%
\subsubsection[ガルニチュール・ブランジェール]{\texorpdfstring{ガルニチュール・ブランジェール\footnote{boulanger/boulangère
  は「パン屋、パン職人」の意。}}{ガルニチュール・ブランジェール}}\label{garniture-boulangere}}

\frsub{Garniture à la Boulangère}

\index{garniture@garniture!boulangere@--- à la Boulangère}
\index{boulanger@boulanger/boulangère!garuniture à la ---ère}
\index{かるにちゆーる@ガルニチュール!ふらんしえーる@---・ブランジェール}
\index{ふらんしえ@ブランジェ/ブランジェール ⇒ パン屋!かるにちゆーる@ガルニチュール・ブランジェール}
\index{はんや@パン屋 ⇒ ブランジェ/ブランジェール!かるにちゆーる@ガルニチュール・ブランジェール}

(羊、乳呑み仔羊、鶏料理に添える)

\begin{enumerate}
\def\labelenumi{\arabic{enumi}.}
\item
  玉ねぎ250 gは薄切りにし\footnote{émincer (エマンセ)。}て、バターで色よく炒める
\item
  じゃがいも750 gは櫛切りか薄切りにする
\item
  塩15 gとこしょう5 g
\end{enumerate}

\begin{itemize}
\item
  1〜3を混ぜ合わせて、このガルニチュールを添える肉を油を熱したフライパンで表面を焼き固め\footnote{rissoler
    (リソレ)。}とともにオーヴンに入れて、一緒に火を通す
\item
  鶏の場合は、じゃがいもはオリーブ形に整形し\footnote{tourner
    (トゥルネ)}、小玉ねぎをあらかじめバターでこんがり焼き色を付けておく。
\item
  ソース\ldots{}\ldots{}美味しい肉汁(ジュ)少々
\end{itemize}

\hypertarget{garniture-bouquetiere}{%
\subsubsection[ガルニチュール・ブクティエール]{\texorpdfstring{ガルニチュール・ブクティエール\footnote{花売り娘、の意。}}{ガルニチュール・ブクティエール}}\label{garniture-bouquetiere}}

\frsub{Garniture à la Bouquetière}

\index{garniture@garniture!bouquetiere@--- à la Bouquetière}
\index{bouquetiere@bouquetière!garuniture à la ---}
\index{かるにちゆーる@ガルニチュール!ふくていえーる@---・ブクティエール}
\index{ふくていえーる@ブクティエール ⇒ 花売り娘!かるにちゆーる@ガルニチュール・ブクティエール}
\index{はなうりむすめ@花売り娘 ⇒ ブクティエール!かるにちゆーる@ガルニチュール・ブクティエール}

(牛、羊の大掛かりな仕立ての料理\footnote{ルルヴェ relevé
  のこと。\protect\hyperlink{releve}{第二版序文訳注}参照。}に添える)

\begin{itemize}
\item
  にんじん250 gと蕪250
  gはスプーンで中をくり抜いて下茹でし、バターで色艶よく炒める\footnote{glacer
    (グラセ)。}
\item
  小さなじゃがいも250 gはシャトー\footnote{長さ6
    cm程度の細長い樽の形状にすること。両端は切り落すので、ラグビーボール形ではない。}に整形する\footnote{いずれも適切に加熱調理するが、この節では細かく説明されていないので、対応する野菜のページを参照すること。}
\item
  プチポワ\footnote{petits pois
    (プティポワ)いわゆるグリンピースのことだが、日本でよく知られているものよりも若どりで小さく、風味も軽やかで甘みがある。}250
  gと、さいの目に切ったアリコヴェール\footnote{haricots verts
    さやいんげんのことだが、これも日本のものより若どりに適した品種が好まれる。}250
  g
\item
  カリフラワー250 gは花束の形状にバラしておく
\end{itemize}

以上の材料をそれぞれ加熱調理した後に、塊肉の周囲に、ブーケ状に、それぞれを離してニュアンスが明確になるように盛り付ける。カリフラワーのブーケには\protect\hyperlink{sauce-hollandaise}{オランデーズソース}を薄く塗ること。

\begin{itemize}
\tightlist
\item
  ソース\ldots{}\ldots{}塊肉を調理した際の肉汁の浮き脂を取り除き\footnote{dégraisser
    (デグレセ)。}、澄ませたもの
\end{itemize}

\hypertarget{garniture-bourgeoise}{%
\subsubsection[ガルニチュール・ブルジョワーズ]{\texorpdfstring{ガルニチュール・ブルジョワーズ\footnote{bourgeois(e)
  (ブルジョワ/ブルジョワーズ)。ブルジョワ風の意。中世においては都市に住む貴族ではないある種の特権階級を意味したが、
  19世紀以降は、肉体労働をせずに快適できわめて豊かな生活をおくれる社会階層、の意に変化した。社会が物質的に、経済的に豊かになるにともない
  petit bourgeois
  (プティブルジョワ)なる階層も出現したが、ブルジョワの本義はあくまでも「大金持ち」であり、現代日本語でいうところの「セレブ」に相当すると思っていい。}}{ガルニチュール・ブルジョワーズ}}\label{garniture-bourgeoise}}

\frsub{Garniture à la Bourgeoise}

\index{garniture@garniture!bourgeoise@--- à la Bourgeoise}
\index{bourgeois@bourgeois(e)!garuniture à la ---}
\index{かるにちゆーる@ガルニチュール!ふるしよわーす@---・ブルジョワーズ}
\index{ふるしよわーす@ブルジョワーズ!かるにちゆーる@ガルニチュール・ブルジョワーズ}
\index{ふるしよわふう@フルジョワ風 ⇒ ブルジョワーズ!かるにちゆーる@ガルニチュール・ブルジョワーズ}

(牛、羊の塊肉料理に添える)

\begin{itemize}
\item
  にんじん500 gは、にんにくのような形に整形して\footnote{tourner
    (トゥルネ)。}下茹でし、バターで色艶よく炒める\footnote{glacer
    (グラセ)。もともとは「鏡のようにする」ところから「艶を出す」の意となり、野菜の場合はもっぱら下茹でした後にバターで軽く炒めて艶を出すことをいうが、場合によっては茹でる段階で砂糖を煮含めたりもする。}
\item
  小玉ねぎ\footnote{日本のいわゆる「ペコロス」は黄色系品種が多いが、フランスの小さな玉ねぎはもっぱら白系品種であり、甘さや風味がまったく異なるので注意。}500
  gは下茹でした後にバターで色艶よく炒める
\item
  塩漬け豚バラ肉\footnote{原文 lard de poitrine
    (ラールドポワトリーヌ)は豚バラ肉のことだが、通常は塩蔵、熟成させたもの、およびそれを冷燻にかけたものを指す。しばしば「ベーコン」と誤訳されているが、日本語のいわゆるベーコンとは違うので注意。}125
  gはさいの目に切ってバターでこんがり炒める
\item
  このガルニチュールは、塊肉にほぼ火が通った段階で、鍋の中の肉の周囲に入れてやり、ブレゼの煮汁で火入れを完全にすること
\end{itemize}

\hypertarget{garniture-brabanconne}{%
\subsubsection[ガルニチュール・ブラバント風]{\texorpdfstring{ガルニチュール・ブラバント風\footnote{現在はベルギー中部の州ブラバントBrabantの、の意。なお、この名称のガルニチュールは『ラルース・ガストロノミック』初版にも掲載されているが、内容がまったく異なる。アンディーヴとじゃがいものピュレ、ホップの若芽を茹でてバターか生クリームであえたもので構成するという
  (p.239)。なおブラバントは中世においてブラバント公国として独立した国家であった。ベルギー王国成立後は、儀礼称号としてベルギー王家の法定推定相続人にブラバント公の称号が授けられるようになった。なお、エスコフィエによる\protect\hyperlink{peches-melba}{ピーチメルバ}創案のきっかけとなったといわれるワーグナーの楽劇『ローエングリン』においてネリー・メルバNellie
  Melba(1861〜1931)が演じていたエルザ・フォン・ブラバントはブラバント公国の公女という設定。}}{ガルニチュール・ブラバント風}}\label{garniture-brabanconne}}

\frsub{Garniture à la Brabançonne}

\index{garniture@garniture!brabanconne@--- à la Brabançonne}
\index{brabanconne@brabançon(ne)!garuniture à la ---ne}
\index{かるにちゆーる@ガルニチュール!ふらはんとふう@---・ブラバント風}
\index{ふらはんとふう@ブラバント風!かるにちゆーる@ガルニチュール・---}

(牛、羊の塊肉の料理に添える)

\begin{itemize}
\item
  空焼きしたタルトレット10個に、下茹でしてバターで蒸し煮した\footnote{étuver
    (エチュヴェ)。}芽キャベツ\footnote{芽キャベツはchoux de Bruxelles
    (シュドブリュクセル、ブリュッセルのキャベツの意)と呼ぶ。}をピュレにして詰め、\protect\hyperlink{sauce-mornay}{ソース・モルネー}を塗る
\item
  \protect\hyperlink{pommes-de-terre-duchesse}{ポムデュシェス}で作った小さな円盤形のクロケット10個
\item
  ソース\ldots{}\ldots{}\protect\hyperlink{jus-de-veau-lie}{とろみを付けたジュ}
\end{itemize}

\hypertarget{garniture-brehan}{%
\subsubsection[ガルニチュール・ブレオン]{\texorpdfstring{ガルニチュール・ブレオン\footnote{このガルニチュールについては、初版から掲載されているにもかかわらず、Bréhanがブルターニュ地方の町の名であることしかわかっていない。ファーヴルにもデュボワ、ベルナール『古典料理』にも言及は見られない。いささか疑問なのは、Bréhanの住人はbréhannaisという語で表わすことから、形容詞も同様であり、garniture
  à la bréhannaise
  (ガルニチュールアラブレアネーズ)の名称でもおかしくないのだが、第二版および第三版ではGarniture
  à la
  Bréhanとなっており、まるで人名のように扱われていることだろう。なお、ブルターニュ地方はアーティチョークの生産で有名だが旬は晩春から初夏にかけてであり、このガルニチュールの構成要素に初版はトリュフのスライスをそら豆のピュレを詰めたアーティチョークの上にのせる指示がある。カリフラワーも基本的には冬の野菜である。それに対してそら豆は乾物であれば1年中、フレッシュのものはやはり晩春から初夏が旬である。レシピには乾物を使うかフレッシュを使うかの指示がないが、「季節感」を演出するためには、フレッシュのそら豆を用いたいところだろう。}}{ガルニチュール・ブレオン}}\label{garniture-brehan}}

\frsub{Garniture Bréhan}

\index{garniture@garniture!brehan@--- Bréhan}
\index{brehan@Bréhan!garuniture ---}
\index{かるにちゆーる@ガルニチュール!ふれおん@---・ブレオン}
\index{ふれおん@ブレオン!かるにちゆーる@ガルニチュール・---}

(牛、仔牛の塊肉の料理に添える)

\begin{itemize}
\item
  小さなアーティチョークの基底部に、そら豆のピュレをドーム状に詰める。
\item
  カリフラワーの小房10個は\protect\hyperlink{sauce-hollandaise}{ソース・オランデーズ}を軽く塗っておく\footnote{茹でてよく水気をきっておくこと}
\item
  小さなじゃがいも10個はバターで火を通し、パセリのみじん切りを振る
\item
  ソース\ldots{}\ldots{}塊肉をブレゼした際の煮汁をソースに仕上げる
\end{itemize}

\hypertarget{garniture-bretonne}{%
\subsubsection{ガルニチュール・ブルターニュ風}\label{garniture-bretonne}}

\frsub{Garniture à la Bretonne}

\index{garniture@garniture!bretonne@--- à la Bretonne}
\index{breton@breton(ne)!garuniture à la ---ne}
\index{かるにちゆーる@ガルニチュール!ふるたーにゆふう@---・ブルターニュ風}
\index{ふるたーにゆふう@ブルターニュ風!かるにちゆーる@ガルニチュール・---}

(羊料理に添える)

\begin{itemize}
\item
  茹でた白いんげん豆またはフラジョレ\footnote{flageolet
    白いんげん豆の一種で、通常のものより小粒。}1
  Lを\protect\hyperlink{sauce-bretonne}{ブルターニュ風ソース}(ブラウン系の派生ソース参照)であえる、パセリのみじん切りを振りかける。
\item
  ソース\ldots{}\ldots{}塊肉の肉汁(ジュ)
\end{itemize}

\hypertarget{garniture-brillat-savarin}{%
\subsubsection[ガルニチュール・ブリヤサヴァラン]{\texorpdfstring{ガルニチュール・ブリヤサヴァラン\footnote{ジャン・アンテルム・ブリア=サヴァラン(Jean
  Anthelme
  Brillat-Savarin)(1755〜1826)。法律家であり、弁護士、一時はアメリカに亡命し、のちに裁判官として活躍したが、とりわけ、はじめ匿名で出版した『美味礼讃』\emph{Physiologie
  du
  Goût}(1825年、タイトルを直訳すれば「味覚の生理学」)で知られる。この著作は食をめぐる考察からなる随筆集だが、必ずしも生真面目な哲学的記述ばかりではない。むしろ「食をめぐる知的な面白読み物」ともいうすべき内容であり、のちに「生理学もの」というジャンルが流行する嚆矢となった。これにインスパイアされたバルザックが『結婚の生理学』(1829年)を出版し文筆家バルザックとして最初のヒット作となった。その後に続けとばかりに「○○の生理学」と題した書物が19世紀中頃まで数多く出版された。その多くはほとんど文学的にも省みられることのないもので、「丸わかり○○」あるいは「○○
  のすべて」的なものばかりだった。このため、「生理学もの」のうちで文学史において一般的に価値を認められている作品は『美味礼讃』および『結婚の生理学』くらいしかない。}}{ガルニチュール・ブリヤサヴァラン}}\label{garniture-brillat-savarin}}

\frsub{Garniture Bréhan}

\index{garniture@garniture!brillat-savarin@--- Brillat-Savarin}
\index{brillat-savarin@Brillat-Savarin!garuniture ---}
\index{かるにちゆーる@ガルニチュール!ふりやさうあらん@---・ブリヤサヴァラン}
\index{ふりやさうあらん@ブリヤサヴァラン!かるにちゆーる@ガルニチュール・---}

(鳥類のジビエ料理に添える)

\begin{itemize}
\item
  空焼きしたごく小さなタルトレットに、トリュフを加えた\protect\hyperlink{souffle-de-becasse}{ベカスのスフレ}\footnote{現行版の原書でベカスのスフレの項を見ると、\protect\hyperlink{becasse-favart}{ベカス・ファヴァール}と同じ、とある。なお、ファヴァールFavartというのは劇場の名称で、オペラコミック座が19世紀以来本拠地にしていたが、
    2度の火災に遭い、その度に再建された。19世紀にはイタリアオペラを主な演目とする「イタリア座」(テアトル・イタリアン)が間借りのようになかたちでファヴァール劇場を本拠にしていた時期もある。現在のファヴァール劇場は1898年に再建され、2005年以降国立となったオペラコミック座の本拠地となっている。}のアパレイユをピラミッド形に盛り、提供直前にやや低温のオーブンで焦がさないように火を通す。
\item
  大きなトリュフのスライス。
\item
  ソース\ldots{}\ldots{}このガルニチュールを添える\protect\hyperlink{fonds-de-gibier}{ジビエのフュメ}で作った上等な\protect\hyperlink{sauce-demi-glace}{ソース・ドゥミグラス}
\end{itemize}

\hypertarget{garniture-bristol}{%
\subsubsection[ガルニチュール・ブリストル]{\texorpdfstring{ガルニチュール・ブリストル\footnote{Bristol
  はイギリス西部の港湾都市。このガルニチュールの名称となった由来などは不明。}}{ガルニチュール・ブリストル}}\label{garniture-bristol}}

\frsub{Garniture Bristol}

\index{garniture@garniture!bristol@--- Bristol}
\index{bristol@Bristol!garniture@garuniture ---}
\index{かるにちゆーる@ガルニチュール!ふりすとる@---・ブリストル}
\index{ふりすとる@ブリストル!かるにちゆーる@ガルニチュール・---}

(牛、羊の塊肉料理に添える)

\begin{itemize}
\item
  アプリコットの形状、大きさの\protect\hyperlink{croquette-de-riz}{米のクロケット}10個。
\item
  茹でたフラジョレ\footnote{\protect\hyperlink{garniture-bretonne}{ガルニチュール・ブルターニュ風}訳注参照。}
  \(\frac{1}{2}\) Lを\protect\hyperlink{veloute}{ヴルテ}であえる。
\item
  くるみ大の丸い小さなじゃがいも20個はバターで火を通し、溶かした\protect\hyperlink{glace-de-viande}{グラスドヴィアンド}を塗る。
\item
  ソース\ldots{}\ldots{}塊肉をブレゼした煮汁をソースとして仕上げる
\end{itemize}

\hypertarget{garniture-bluxelloise}{%
\subsubsection[ガルニチュール・ブリュッセル風]{\texorpdfstring{ガルニチュール・ブリュッセル風\footnote{芽キャベツchoux
  de Bruxelles
  とアンディーヴendiveはいずれもベルギーで品種改良、開発された野菜であり、これらを組み合わせてブリュッセル風とするのはいささか安易なようにも思われる。}}{ガルニチュール・ブリュッセル風}}\label{garniture-bluxelloise}}

\frsub{Garniture à la Bruxelloise}

\index{garniture@garniture!bruxelloise@--- à la Bruxelloise}
\index{bruxellois@bruxellois(e)!garniture@garuniture à la ---e}
\index{かるにちゆーる@ガルニチュール!ふりゆつせるふう@---・フリュッセル風}
\index{ふりゆつせるふう@ブリュッセル風!かるにちゆーる@ガルニチュール・---}

(牛、羊の塊肉料理に添える)

\begin{itemize}
\item
  アンディーヴ10個は白さを保つようにしてブレゼする。
\item
  シャトー\footnote{\protect\hyperlink{garniture-bouquetiere}{ガルニチュール・ブクティエール}訳注参照。}に整形したじゃがいも10個。
\item
  芽キャベツ500gは下茹でした後バターで蒸し煮する\footnote{étuver
    (エチュヴェ)。下茹での段階で \(\frac{2}{3}\)〜
    \(\frac{3}{4}\)くらいまで火を通しておくこと。サヴォイキャベツもそうだが、下茹でにはアクを除去する意味もあり、エチュヴェの段階で変色してしまうことがあるため、アクを充分に取り除いてから比較的短時間でエチュヴェするのが望ましい。}。
\item
  ソース\ldots{}\ldots{}やや薄めのマデイラ酒風味の\protect\hyperlink{sauce-demi-glace}{ソース・ドゥミグラス}。
\end{itemize}

\hypertarget{garniture-cancalaise}{%
\subsubsection[ガルニチュール・カンカル風]{\texorpdfstring{ガルニチュール・カンカル風\footnote{ブルターニュ地方の地名Cancale(カンカール)の形容詞
  cancalais(e) (カンカレ/カンカレーズ)。牡蠣の産地として知られ、
  cancaleという牡蠣の品種もある。17世紀、ルイ14世は、ヴェルサイユ宮殿へカンカル産カキを取り寄せていたといわれている。なお、ブルターニュ地方とはいえノルマンディ地方に非常に近い位置にあるため、牡蠣を中心にしたこのガルニチュールにブルターニュの地名を冠し、ノルマンディ風ソースを合わせるのは、一種の洒落とも考えられなくもないが、ブルターニュが言語文化的にフランスにおいてやや異質な歴史を持っていることを考慮すると、無神経な命名ともとられかねない。}}{ガルニチュール・カンカル風}}\label{garniture-cancalaise}}

\frsub{Garniture à la Cancalaise}

\index{garniture@garniture!cancalaise@--- à la Cancalaise}
\index{cancalais@cancalais(e)!garniture@garuniture à la ---e}
\index{かるにちゆーる@ガルニチュール!かんかるふう@---・カンカル風}
\index{かんかるふう@カンカル風!かるにちゆーる@ガルニチュール・---}

(魚料理に添える)

\begin{itemize}
\item
  牡蠣20個の剥き身は、沸騰しない程度の温度の湯で火を通し、周囲をきれいに掃除する。殻を剥いたクルヴェットの尾125g
\item
  ノルマンディ風ソース
\end{itemize}

\hypertarget{garniture-cardinal}{%
\subsubsection[ガルニチュール・カルディナル]{\texorpdfstring{ガルニチュール・カルディナル\footnote{カトリック教会における枢機卿のこと。枢機卿の衣が真紅であることからオマールを用いた料理に付けられた名称とも、オマールが「海の枢機卿」と呼ばれるから、ともいわれている。なお、\ul{à la + 男性名詞}
  の形態は、固有名詞の場合および、対応する女性名詞がない場合にも成立する。これは
  \ul{à la manière de + 名詞} のmanière de
  が省略されたものと解釈される。さらに、料理名において à la
  も省略される傾向にあるため、garuniture Cardinal あるいは garniture
  cardinal という表現も\ul{料理名においては}正しいとされている。}}{ガルニチュール・カルディナル}}\label{garniture-cardinal}}

\frsub{Garniture à la Cardinal}

\index{garniture@garniture!cardinal@--- à la Cardinal}
\index{cardinal@cardinal!garniture@garuniture à la ---}
\index{かるにちゆーる@ガルニチュール!かるていなる@---・カルディナル}
\index{かるていなる@カルディナル!かるにちゆーる@ガルニチュール・---}
\index{すうききよう@枢機卿 ⇒ カルディナル!かるにちゆーる@ガルニチュール・カルディナル}

(魚料理に添える)

\begin{itemize}
\item
  立派なオマールの尾の身をやや斜めに厚さ1cm程度にスライスしたもの10枚。
\item
  真黒なトリュフのスライス10枚。
\item
  さいの目に切ったオマールの身60 gとトリュフ50 g。
\item
  \protect\hyperlink{sauce-cardinal}{ソース・カルディナル}
\end{itemize}

\hypertarget{garniture-castillane}{%
\subsubsection[ガルニチュール・カスティリア風]{\texorpdfstring{ガルニチュール・カスティリア風\footnote{Castilla
  (カスティーリャ、カスティージャ)はスペイン中部の地域で、中世はカスティリア王国だった。「カステラ」の語源ともいわれる。}}{ガルニチュール・カスティリア風}}\label{garniture-castillane}}

\frsub{Garniture à la Castillane}

\index{garniture@garniture!castillane@--- à la Castillane}
\index{castillan@castillan(e)!garniture@garuniture à la ---e}
\index{かるにちゆーる@ガルニチュール!かすていりあふう@---・カスティリア風}
\index{かすていりあふう@カスティリア風!かるにちゆーる@ガルニチュール・---}

(トゥルヌド、ノワゼットに添える)

\begin{itemize}
\item
  \protect\hyperlink{pommes-de-terre-duchesse}{ポム・デュシェス}で作ったた小さなケースにドリュールを塗ってオーブンで焼き色を付ける。そこに、軽くにんにく風味を効かせた\protect\hyperlink{portugaise}{トマトのフォンデュ}を詰める。
\item
  皿の周囲に、輪切りにして塩こしょうし、小麦粉をまぶして油で揚げた玉ねぎを飾る。
\item
  トマト風味を加えたデグラセした肉汁(ジュ)\footnote{トゥルヌド、ノワゼットをフライパンでソテーし、デグラセしてトマトピュレまたは本文にあるトマトのフォンデュを加えてソースにするということ。}
\end{itemize}

\hypertarget{garniture-chambord}{%
\subsubsection[ガルニチュール・シャンボール]{\texorpdfstring{ガルニチュール・シャンボール\footnote{シャンボールとは16世紀、ロワール河の近くに建てられた瀟洒な城の名。このガルニチュールを添えた場合、料理名にシャンボールが冠される。鯉、サーモンが代表的だが、とりわけ19世紀は鯉が好まれ、カレーム『19
  世紀フランス料理』第2巻では鯉のシャンボールだけで近代風、ロヤイヤル、レジャンスの3種の仕立てについて詳述されている(pp.181-189)。なお、このガルニチュールの構成も時代や料理人によって多少の変化があり、『ラルース・ガストロノミック』初版では、魚でつくった大小のクネル、マッシュルーム、舌びらめのフィレ、バターでソテーした白子、オリーヴ形に整形したトリュフ、クールブイヨンで火を通したエクルヴィス、揚げたクルトン、となっている(p.516)。}}{ガルニチュール・シャンボール}}\label{garniture-chambord}}

\frsub{Garniture Chambord}

\index{garniture@garniture!chambord@--- Chambord}
\index{chambord@Chambord!garniture@garuniture ---}
\index{かるにちゆーる@ガルニチュール!しやんほーる@---・シャンボール}
\index{しやんほーる@シャンボール!かるにちゆーる@ガルニチュール・---}

(魚のブレゼの大掛かりな仕立てに添える\footnote{ルルヴェのこと。\protect\hyperlink{releve}{第二版序文訳注}参照。19世紀前半くらいまではカトリックの習慣としての「小斉」が比較的厳格に守られており、料理人たちは四旬節やその他の小斉の日の献立としていかに豪華で美味な魚料理を提供するかに腐心していたのが、17〜18世紀の料理書を読むとよくわかる。カレームの著書にも魚の大掛かりな仕立てのレシピが数多く収められている。})

\begin{itemize}
\item
  トリュフを加えてスプーンで整形した魚のファルスで作ったクネル10個。
\item
  長卵形の大きな、表面に装飾を施したクネル4個。
\item
  渦巻模様を付けた\footnote{原文 canneler (カヌレ)。この場合はtourner
    (トゥルネ)とほぼ同義だが、凹凸の刻み模様を付けた、の意。}小さなマッシュルーム200
  g。
\item
  鯉の白子を1
  cm程度の厚さにスライスして塩こしょうし、小麦粉をまぶしてソテーしたもの10枚。
\item
  オリーブ形に整形した\footnote{tourner
    (トゥルネ)。原義は「回す」。野菜などを包丁ではなく材料を回すようにして皮を剥いたり整形するところから。}トリュフ200
  g。
\item
  エクルヴィス\footnote{ecrevisse ヨーロッパザリガニ。}6尾は\protect\hyperlink{courtbouillon-a}{クールブイヨン}\footnote{court-bouillon
    (クールブイヨン)。court
    は少量の意。つまり、原則としてはできるだけ少量の液体を煮汁として魚介類その他を加熱調理するのに用いる。また、とりわけ魚介類の場合は沸騰しない程度の温度で火を通す(pocher
    ポシェする)のが原則。たんなる水、塩水だけでなく、ワインや香味野菜、香辛料などを加えて風味付け(および場合によっては臭みのマスキング効果)も兼ねて事前に用意しておくこともある。ただしこれらはあくまでも原則論にすぎない。詳細は\protect\hyperlink{poissons}{魚料理}の\protect\hyperlink{serie-de-courts-bouillons-de-poisson}{クールブイヨン}および\protect\hyperlink{ecrevisse-a-la-nage}{エクルヴィス・ナージュ}参照のこと。なお、エクルヴィスの場合は上記の「少量」にあまりこだわらず、後ではさみを背に回しやすくなるように鍋に入れて加熱すればいいだろう。エクルヴィスはジストマ(寄生虫)のリスクがあるためしっかり加熱すること。またエクルヴィスは腕が取れやすいが、その場合でも可食部である尾の身には問題がないので装飾以外の利用はもちろん可能であり、装飾用としてはロス分を見込んで用意しておくのがいいだろう。}で火を通し、はさみを背に回すように整形する\footnote{trousser
    (トゥルセ)。}(しなくてもよい)。
\item
  食パンを鶏のとさかの形に切りバターで揚げたクルトン6枚。
\item
  魚をブレゼした際の煮汁をベースにしたソース。
\end{itemize}

\hypertarget{garniture-chatelaine}{%
\subsubsection[ガルニチュール・シャトレーヌ]{\texorpdfstring{ガルニチュール・シャトレーヌ\footnote{châtelain(e)
  (シャトラン/シャトレーヌ)。城館の主の意。城館に住む者を思わせる豪華な、の意で料理名として使われるようになったようだ。}}{ガルニチュール・シャトレーヌ}}\label{garniture-chatelaine}}

\frsub{Garniture Châtelaine}

\index{garniture@garniture!chatelaine@--- Châtelaine}
\index{chatelaine@Châtelaine!garniture@garuniture ---}
\index{かるにちゆーる@ガルニチュール!しやとれーぬ@---・シャトレーヌ}
\index{しやとれーぬ@シャトレーヌ!かるにちゆーる@ガルニチュール・---}

(牛、羊の塊肉や鶏料理に添える)

\begin{itemize}
\item
  アーティチョークの基底部10個に、固く作った\protect\hyperlink{sauce-soubise}{スビーズ}を詰める。
\item
  殻を剥いて塊肉をブレゼした煮汁で蒸し煮したマロン30個。
\item
  \protect\hyperlink{pommes-de-terre-noisette}{じゃがいものノワゼット}300
  g。
\item
  ブレゼした煮汁を加えた\protect\hyperlink{sauce-madere}{ソース・マデール}
\end{itemize}

\hypertarget{garniture-chipolata}{%
\subsubsection[ガルニチュール・シポラタ]{\texorpdfstring{ガルニチュール・シポラタ\footnote{もとはイタリアで玉ねぎとソーセージを煮込んだ料理(cipollata
  チポッラータ \textless{} cipolla
  チポッラ=玉ねぎ)を意味していたが、フランスに伝わった際に、語本来の意味に含まれていた玉ねぎが脱落して、羊腸に豚挽肉を詰めた小さなソーセージをこう呼ぶようになったといわれている。}}{ガルニチュール・シポラタ}}\label{garniture-chipolata}}

\frsub{Garniture à la Chipolata}

\index{garniture@garniture!chipolata@--- à la Chipolata}
\index{chipolata@chipolata!garniture@garuniture à la ---}
\index{かるにちゆーる@ガルニチュール!しほらた@---・シポラタ}
\index{しほらた@シポラタ!かるにちゆーる@ガルニチュール・---}

(牛、羊の塊肉および鶏料理に添える)

\begin{itemize}
\item
  小玉ねぎ20個は下茹でしてバターで色艶よく炒める\footnote{glacer
    (グラセ)。本文下のにんじんも同様の指示。}。
\item
  シポラタソーセージ10本。
\end{itemize}

コンソメで煮たマロン10個。

塩漬け豚バラ肉125 gはさいの目に切って、強火でこんがり炒める。

\begin{itemize}
\item
  オリーブ形に整形して下茹でし、バターで色艶よく炒めたにんじん20個(なくてもよい)。
\item
  ソース\ldots{}\ldots{}このガルニチュールを添える料理の煮汁を加えた\protect\hyperlink{sauce-demi-glace}{ソース・ドゥミグラス}
\end{itemize}

\hypertarget{garniture-choisy}{%
\subsubsection[ガルニチュール・ショワジー]{\texorpdfstring{ガルニチュール・ショワジー\footnote{パリのセーヌ川上流(=東側)約12
  kmのところにある Choisy-le-Roi
  の地名に由来。17世紀にショワジー城が建てられ、18世紀にこれを相続したルイ15世が狩りの際に使う邸宅として利用し、現在の名称ショワジールロワになった。その後、ポンパドゥール夫人がここに移り住み、豪華な夕食会がしばしば開かれたという。ショワジーの名称はレチュを用いた料理に付けられることが多い。}}{ガルニチュール・ショワジー}}\label{garniture-choisy}}

\frsub{Garniture Choisy}

\index{garniture@garniture!choisy@--- Choisy}
\index{choisy@Choisy!garniture@garuniture ---}
\index{かるにちゆーる@ガルニチュール!しよわしー@---・ショワジー}
\index{しよわしー@ショワジー!かるにちゆーる@ガルニチュール・---}

(トゥルヌドおよびノワゼットに添える)

\begin{itemize}
\item
  半割りにした\protect\hyperlink{laitue-braise}{レチュのブレゼ}10個。
\item
  シャトーに整形した小さなじゃがいも20個。
\item
  ソース\ldots{}\ldots{}バターを加えた\protect\hyperlink{glace-de-viande}{グラスドビアンド}
\end{itemize}

\hypertarget{garniture-choron}{%
\subsubsection[ガルニチュール・ショロン]{\texorpdfstring{ガルニチュール・ショロン\footnote{19世紀にあったパリの有名レストラン、ヴォワザンの料理長の名。\protect\hyperlink{sauce-bearnaise-tomatee}{ソース・ショロン}も参照。}}{ガルニチュール・ショロン}}\label{garniture-choron}}

\frsub{Garniture Choron}

\index{garniture@garniture!choron@--- Choron}
\index{choron@Choron!garniture@garuniture ---}
\index{かるにちゆーる@ガルニチュール!しよろん@---・ショロン}
\index{しよろん@ショロン!かるにちゆーる@ガルニチュール・---}

(トゥルヌドおよびノワゼットに添える)

\begin{itemize}
\item
  中位か小さいアーティチョークの基底部をにバターであえたアスパラガスの穂先を詰める。アスパラガスがなければ、バターであえた小粒のプチポワでもいい。
\item
  \protect\hyperlink{pommes-de-terre-noisette}{じゃがいものノワゼット}30個。
\item
  \protect\hyperlink{sauce-bearnaise-tomatee}{トマト入りソース・ベアルネーズ}。
\end{itemize}

\hypertarget{garniture-clamart}{%
\subsubsection[ガルニチュール・クラマール]{\texorpdfstring{ガルニチュール・クラマール\footnote{パリ郊外の町の名。プチポワを使った料理にこの名が冠されるものがいくつかある。}}{ガルニチュール・クラマール}}\label{garniture-clamart}}

\frsub{Garniture à la Clamart}

\index{garniture@garniture!clamart@--- à la Clamart}
\index{clamart@Clamart!garniture@garuniture à la ---}
\index{かるにちゆーる@ガルニチュール!くらまーる@---・クラマール}
\index{くらまーる@クラマール!かるにちゆーる@ガルニチュール・---}

(牛、羊の塊肉の料理に添える)

\begin{itemize}
\item
  \protect\hyperlink{petits-pois-francaise}{プチポワ・アラフランセーズ}に細かく刻んだレチュの葉を加えてバターであえ、空焼きしたタルトレット10個に詰める。
\item
  \protect\hyperlink{pommes-de-terre-macaire}{じゃがいものマケール}で作った円形の小さな台の上に、タルトレットをひとつずつのせる。
\item
  ソース\ldots{}\ldots{}\protect\hyperlink{jus-de-veau-lie}{とろみを付けたジュ}\footnote{このガルニチュールを添える料理がポワレ(\protect\hyperlink{sauce-bigarade}{ソース・ビガラード}訳注および\protect\hyperlink{releves-et-entrees}{肉料理}参照)の場合には、鍋に残った香味野菜(マティニョン)にフォン少量を注いで風味を引き出し、それにコーンスターチでとろみを付けることになるだろう。}
\end{itemize}

\hypertarget{garniture-de-compote}{%
\subsubsection[ガルニチュール・コンポート]{\texorpdfstring{ガルニチュール・コンポート\footnote{compote
  (コンポート)。果物のシロップ煮のイメージが強いが、肉や野菜をばらばらになるまで煮込んだ料理のことも指す。}}{ガルニチュール・コンポート}}\label{garniture-de-compote}}

\frsub{Garniture de Compote}

\index{garniture@garniture!compote@--- de Compote}
\index{compote@compote!garniture@garuniture de ---}
\index{かるにちゆーる@ガルニチュール!こんほーと@---・コンポート}
\index{こんほーと@コンポート!かるにちゆーる@ガルニチュール・---}

(鳩およびプレ・ド・グラン\footnote{poulet de grain
  鶏の大きさや飼育方法による区別については\protect\hyperlink{sauce-chaud-froid-vert-pre}{ソース・ショフロワ・ヴェールプレ}参照。}に添える)

\begin{itemize}
\item
  塩漬け豚バラ肉\footnote{lard de poitrine
    (ラールドポワトリーヌ)、lard maigre
    (ラールメーグル)あるいは原文のように合わせて lard de poitrine
    maigre
    (ラールドポワトリーヌメーグル)とも呼ぶが、塩漬けにして熟成させた豚バラ肉のこと。通常、lard
    だけの場合は lard gras
    (ラールグラ)すなわち豚背脂のことを意味するので注意。}250は拍子木\footnote{lardon
    (ラルドン)。たんに lardon
    というだけで、この豚バラ肉の塩漬けを拍子木状に切ったものを意味することはごく一般的で、塩漬け後に冷燻にかけた
    lard de poitrine fumé を拍子木に切ったものは lardon fumé
    (ラルドンフュメ)と呼ばれる。}に切り、下茹でしてからバターでこんがり炒める\footnote{rissoler
    (リソレ)油脂を熱して、素材の表面をこんがり焼くこと。語源は中世からある
    rissole
    (リソール)という円形または半円形、塩味または甘い焼き菓子(揚げ菓子)---
    つまり時代や地域とともに非常にバリエーションに富むものだが、こんがりとした色合いに仕上げるのは共通している。}。
\item
  小玉ねぎ300 gは下茹でしてバターで色艶よく炒める\footnote{glacer
    (グラセ)。}。
\item
  小さなマッシュルーム300 gは生のまま2つに切り、バターで炒める。
\item
  これらは鳩とともに火入れを仕上げ、供する際には鳩を覆うようにガルニチュールを盛る。
\end{itemize}

\hypertarget{garniture-conti}{%
\subsubsection[ガルニチュール・コンティ]{\texorpdfstring{ガルニチュール・コンティ\footnote{ブルボン王家のひとつCondé(コンデ)家の傍流(いわゆる分家筋)で、代々のコンティ大公
  le prince de Conti
  (ルプランスドコンティ)がいる。もとはピカルディ地方アミアンの近くにあるContyというところを領地にしていたのが家名の起源。コンティの名は本書でも、レンズ豆のポタージュ、\protect\hyperlink{puree-conti}{ピュレ・コンティ}が収められている。このガルニチュールは18世紀のコンティ大公ルイ・フランソワ・ド・ブルボン(1727〜
  1776)の料理人が考案したと伝えられているが、実際にそうだったとしても、調理法としてはあまりにシンプルなものなので、ガルニチュールとして供することを考え出した、というのがせいぜいのところだろう。}}{ガルニチュール・コンティ}}\label{garniture-conti}}

\frsub{Garniture Conti}

\index{garniture@garniture!conti@--- Conti}
\index{conti@Conti!garniture@garuniture de ---}
\index{かるにちゆーる@ガルニチュール!こんてい@---・コンティ}
\index{こんてい@コンティ!かるにちゆーる@ガルニチュール・---}

(牛、羊の塊肉のブレゼに添える)

\begin{itemize}
\item
  レンズ豆\footnote{lentille
    (ロンティーユ)。西アジア原産で旧約聖書の「創世記」にも出てくる。アブラハムの息子イサクの双子のうちのひとりであるエサウはすぐれた狩人に、もうひとりのヤコブは「穏かな人で天幕の周りで働くのを常として。(中略)ある日のこと、ヤコブが煮物をしていると、エサウが疲れきって野原から帰ってきた。エサウはヤコブに言った。『お願いだ、その赤いもの(アドム)、そこの赤いものを食べさせてほしい。わたしは疲れきっているんだ。』(中略)エサウは誓い、長子の権利をヤコブに譲ってしまった。ヤコブはエサウにパンとレンズ豆の煮物を与えた。(中略)こうしてエサウは長子の権利を軽んじた。(「創世記」25-27〜34、新共同訳『聖書』)」。レンズ豆はおそらく農耕がはじまったごく初期からの作物であり、エジプトや地中海沿岸で多く栽培されていた。このため温暖な気候に向いた作物であり、その意味ではフランス北部に縁があるコンティ大公の名はふさわしくないかも知れない。ただ、Condé
    すなわちコンデ大公の名を冠した料理、とりわけポタージュ\protect\hyperlink{puree-conde}{ピュレ・コンデ}が赤いんげん豆のポタージュであることとの釣り合いはとれている考えられよう。}のピュレ750
  g。
\item
  脂身のほとんどない豚バラ肉の塩漬け250
  gは長方形に切って、レンズ豆を煮る際に一緒に煮ておく。
\item
  ソース\ldots{}\ldots{}ブレゼの煮汁をソースとして仕上げたもの。
\end{itemize}

\hypertarget{garniture-a-la-commodore}{%
\subsubsection[ガルニチュール・コモドール]{\texorpdfstring{ガルニチュール・コモドール\footnote{もとは英語
  commodore であり、イギリスでは艦隊司令官、アメリカでは准将の意。}}{ガルニチュール・コモドール}}\label{garniture-a-la-commodore}}

\frsub{Garniture à la Commodore}

\index{garniture@garniture!commodore@--- à la Commodore}
\index{commodore@commodore!garniture@garuniture de ---}
\index{かるにちゆーる@ガルニチュール!こもとーる@---・コモドール}
\index{こもとーる@コモドール!かるにちゆーる@ガルニチュール・---}

(魚の大掛かりな仕立てに添える)

\begin{itemize}
\item
  エクルヴィスの尾の身を入れた小さなグラタン皿10個。
\item
  メルラン\footnote{merlan (メルロン)タラ科の海水魚。}の\protect\hyperlink{farce-a}{ファルス}に\protect\hyperlink{beurre-d-ecrevissse}{エクルヴィスバター}を加え、スプーンで整形したクネル10個。
\item
  大きな\protect\hyperlink{moules-a-la-villeroy}{ムール貝のヴィルロワ}10個。
\item
  仕上げにエクルヴィスバターを加えた\protect\hyperlink{sauce-normande}{ノルマンディ風ソース}。
\end{itemize}

\hypertarget{garniture-cussy}{%
\subsubsection[ガルニチュール・キュシー]{\texorpdfstring{ガルニチュール・キュシー\footnote{キュシー侯爵(1767〜1841)。\protect\hyperlink{osbservation-sur-la-sauce}{基本ソース 概説}訳注参照。}}{ガルニチュール・キュシー}}\label{garniture-cussy}}

\frsub{Garniture Cussy}

\index{garniture@garniture!cussy@--- Cussuy}
\index{cussy@Cussy (marquis de)!garniture@garuniture ---}
\index{かるにちゆーる@ガルニチュール!きゆしー@---・キュシー}
\index{きゆしー@キュシー!かるにちゆーる@ガルニチュール・---}

(トゥルヌド、ノワゼット、鶏料理に添える)

\begin{itemize}
\item
  マロンのピュレを詰めてグリル焼きした大きなマッシュルーム10個。
\item
  完全に球形に整形し、マデイラ酒風味で火を通した小さなトリュフ10個。
\item
  大きな雄鶏のロニョン\footnote{rognon
    仔牛などでは腎臓のこと。鶏の場合はrognon
    blanc(ロニョンブロン)とも呼び、精巣のこと。この場合は後者。もちろんきちんと加熱調理したものをガルニチュールの構成要素とする。}20個。
\item
  \protect\hyperlink{sauce-madere}{ソース・マデール}
\end{itemize}

\hypertarget{garniture-Daumont}{%
\subsubsection[ガルニチュール・ドモン]{\texorpdfstring{ガルニチュール・ドモン\footnote{ドモン公爵家duché
  d'Aumon(デュシェドモン)にちなんだ名称をいわれている。}}{ガルニチュール・ドモン}}\label{garniture-Daumont}}

\frsub{Garniture Daumont}

\index{garniture@garniture!daumont@--- Daumont}
\index{daumont@Daumont!garniture@garuniture ---}
\index{かるにちゆーる@ガルニチュール!ともん@---・ドモン}
\index{ともん@ドモン!かるにちゆーる@ガルニチュール・---}

(魚料理に添える)

\begin{itemize}
\item
  バターで鍋に蓋をして弱火で火を通した\footnote{étuver au beurre
    (エチュヴェオブール)}マッシュルーム10個に、それぞれエクルヴィスの尾の身を半分に切ったもの6枚ずつ添える。
\item
  \protect\hyperlink{farce-c}{生クリームを加えてつくった魚のファルス}で小さな球形にし、トリュフで装飾を施したクネル10個。
\item
  厚さ1cm程にスライスした\footnote{escalope (エスカロップ)肉や魚を1〜2
    cmの厚さで、筋腺維と直角にスライスした円形または楕円形にスライスしたもの。}白子10枚はイギリス式パン粉衣\footnote{paner
    à l'anglaise
    (パネアロングレーズ)。素材に小麦粉をまぶしてから、卵液に浸し、細かいパン粉で衣を付けること。日本でフライなどをつくる際に一般的な方法とよく似ているが、日本では粗いパン粉が好まれるのに対して、フランスやイギリスでは細かいパン粉を使うのが一般的。}を付けて油で揚げる。
\item
  \protect\hyperlink{sauce-nantua}{ソース・ナンチュア}
\end{itemize}

\hypertarget{garniture-a-la-dauphine}{%
\subsubsection[ガルニチュール・ドフィーヌ]{\texorpdfstring{ガルニチュール・ドフィーヌ\footnote{à
  la Dauphine
  (アラドフィーヌ)王太子妃風、の意。この料理名には由来や理由がないことがほとんど。あえていえば「豪華」であるという程度だが、存外、簡素な仕立ての料理にも付けられることがある。}}{ガルニチュール・ドフィーヌ}}\label{garniture-a-la-dauphine}}

\frsub{Garniture à la Dauphine}

\index{garniture@garniture!dauphine@--- à la Dauphine}
\index{dauphin@dauphin(e)!garniture@garuniture ---e}
\index{かるにちゆーる@ガルニチュール!とふいぬ@---・ドフィーヌ}
\index{とふあん@ドファン/ドフィーヌ!かるにちゆーる@ガルニチュール・ドフィーヌ}

(牛、羊の塊肉の料理に添える)

\begin{itemize}
\item
  \protect\hyperlink{pomme-de-terres-dauphine}{じゃがいものドフィーヌ}をアパレイユにした\protect\hyperlink{croquettes}{クロケット}20個。大きな塊肉料理に添える場合はコルクの栓の形状に、トゥルヌドやノワゼットに添えるときは平たい円盤の形にする。
\item
  マデイラ酒風味の\protect\hyperlink{sauce-demi-glace}{ソース・ドゥミグラス}。
\end{itemize}

\hypertarget{garniture-a-la-dieppoise}{%
\subsubsection[ガルニチュール・ディエープ風]{\texorpdfstring{ガルニチュール・ディエープ風\footnote{dieppois(e)
  (ディエポワ/ディエポワーズ) \textless{} Dieppe
  (ディエープ)ノルマンディ地方の港町の名。}}{ガルニチュール・ディエープ風}}\label{garniture-a-la-dieppoise}}

\frsub{Garniture à la Dieppoise}

\index{garniture@garniture!dieppoise@--- à la Dieppoise}
\index{dieppois@dieppois(e)!garniture@garuniture ---e}
\index{かるにちゆーる@ガルニチュール!ていえーふふう@---・ディエープ風}
\index{ていえーふふう@ディエープ風!かるにちゆーる@ガルニチュール・---}

(魚料理に添える)

\begin{itemize}
\item
  殻を剥いたクルヴェット\footnote{小海老。生の状態で灰色のcrevette
    grise(クルヴェットグリーズ)とやや大きめのcrevette
    rose(クルヴェットローズ)が代表的。}の尾の身100g。
\item
  \(\frac{3}{4}\)
  L(約30個)のムール貝は白ワインを加えた湯で沸騰させない程度の温度で火を通し\footnote{pocher
    (ポシェ)}、周囲をきれいに掃除する\footnote{ébarber
    (エバルベ)貝類の身の周囲をきれいにすること。帆立貝の場合は「ひもを取る」ともいう。}。
\item
  このガルニチュールを添える魚の煮汁を煮詰めて加えた\protect\hyperlink{sauce-vin-blanc}{白ワインソース}。
\end{itemize}

\hypertarget{garniture-doria}{%
\subsubsection[ガルニチュール・ドリア]{\texorpdfstring{ガルニチュール・ドリア\footnote{原書現行版ではDorlaとなっているが初版〜第三版はDoria。19世紀パリのカフェ・アングレの顧客として知られていた名家ドリアの名を冠したといわれている。このドリア家は12世紀ジェノヴァの
  de Auria (ラテン語の filiis
  Auriaeすなわちアウリアの子孫の意)から発する由緒ある家系として有名。なお、日本の洋食のドリアは1930年頃横浜ホテルニューグランド総料理長サリー・ワイルが発案したものといわれており、上記のドリア家とはまったく関係がない。また古代ギリシア時代の民族ドーリア人とも関係がない。ちなみに、バルザックの小説『幻滅』におなじ発音の名の
  Dauriatという登場人物がいる。}}{ガルニチュール・ドリア}}\label{garniture-doria}}

\frsub{Garniture Doria}

\index{garniture@garniture!doria@--- Doria}
\index{doria@Doria!garniture@garuniture ---e}
\index{かるにちゆーる@ガルニチュール!とりあ@---・ドリア}
\index{とりあ@ドリア!かるにちゆーる@ガルニチュール・---}

(魚料理に添える)

\begin{itemize}
\item
  オリーブ形に整形したきゅうり\footnote{concombre
    (コンコンブル)日本で一般的なきゅうりと品種系統も異なるものが多く、サイズも太さ4〜5
    cm、長さ30〜45
    cmで収穫する(品種によって異なる)。青臭さがなく、加熱調理することが多い。}30個をバターとともに鍋に入れて蓋をして弱火で蒸し煮する\footnote{étuver
    (エチュヴェ)。}。
\item
  表皮を剥いて種を取り除いたレモンのスライスを魚の上に並べる。魚は\protect\hyperlink{meuniere}{ムニエル}にしたもの。
\end{itemize}

\hypertarget{garniture-dubarry}{%
\subsubsection[ガルニチュール・デュバリー]{\texorpdfstring{ガルニチュール・デュバリー\footnote{Madame
  du Barry (マダムデュバリー)デュバリー夫人
  (1743〜1793)のこと。ルイ15世の公妾であり、フランス革命により断頭台に送られ命を落したことで知られる。もとはシャンパーニュ地方の貧しい家庭の生まれ。パリに出てのち「お針子」などの仕事や娼婦をしていたが、デュ・バリー子爵に囲われ、いわゆるdemi-mondaine(ドゥミモンデーヌ)、
  courtisane
  (クルティザーヌ)すなわち高等娼婦として知られるようになる。その後、ポンパドゥール夫人を亡くしたルイ15世が彼女を妾にすることにし、形式上、デュ・バリー子爵の弟と結婚したことにして、正式な社交界デビューを果たした。フランス史において「女性」であることを最大限利用して社会的にのしあがった典型例のひとつ。フランス革命のさなか、捕えられて断頭台へ連れていかれる際に、ほとんどの貴族の女性が取り乱さず凛として死に臨んだのに対し、デュバリー夫人ただひとりだけが狂乱し泣き叫んで命乞いした、という逸話が残っている。ただし、それはロベスピエールによる恐怖政治への警鐘になり得たという見解も少なくない。}}{ガルニチュール・デュバリー}}\label{garniture-dubarry}}

\frsub{Garniture Dubarry}

\index{garniture@garniture!dubarry@--- Dubarry}
\index{dubarry@Dubarry!garniture@garuniture ---}
\index{かるにちゆーる@ガルニチュール!とゆはりー@---・デュバリー}
\index{とゆはりー@デュバリー!かるにちゆーる@ガルニチュール・---}

(牛、羊の塊肉の料理やノワゼット、トゥルヌドに添える)

\begin{itemize}
\item
  カリフラワーの花房を小さく分け、小さなボウルに詰め半球形になるようまとめて裏返し、\protect\hyperlink{sauce-mornay}{ソース・モルネー}で覆ったもの10個。おろした\footnote{râper
    (ラペ)}チーズを振りかけて高温のオーブンでこんがり焼く\footnote{gratiner
    (グラティネ)。また、原文moulés en
    boulesを文字通りに読むと「完全な球形」にするようにも解釈出来なくはないが、そのためには強力な「つなぎ」が必要になる。ソース・モルネー以外に「つなぎ」の役割を果たすものの指定がないため、これでは球形を維持する「つなぎ」として熱に弱過ぎるだろう。ここは\protect\hyperlink{chou-fleur-au-gratin}{カリフラワーのグラタン}にあるように
    moulé dans un bol
    「ボウルに詰める」と同様と解釈していいと思われる。。}。
\item
  \protect\hyperlink{pommes-de-terre-fondantes}{じゃがいものフォンダント}10個。
\item
  塊肉をブレゼあるいはポワレした際のフォン、もしくはノワゼットやトゥルヌドをソテした後にデグラセしてソースに仕上げる。
\end{itemize}

\hypertarget{garniture-a-la-duchesse}{%
\subsubsection[ガルニチュール・デュシェス]{\texorpdfstring{ガルニチュール・デュシェス\footnote{duc
  (デュック=公爵)、duchesse(デュシェス=公爵夫人)。ここではたんに、\protect\hyperlink{pommes-de-terre-duchesse}{じゃがいものデュシェス}を用いるからこの名称になっているが、デュシェスそれ自体にも料理名としての由来や根拠はほとんどない。}}{ガルニチュール・デュシェス}}\label{garniture-a-la-duchesse}}

\frsub{Garniture à la Duchesse}

\index{garniture@garniture!duchesse@--- à la Duchesse}
\index{duc@duc / duchesse!garniture@garuniture à la Duchesse}
\index{かるにちゆーる@ガルニチュール!てゆせす@---・デュシェス}
\index{てゆしえす@デュシェス!かるにちゆーる@ガルニチュール・---}

(牛、羊の塊肉の料理やノワゼット、トゥルヌドに添える)

\begin{itemize}
\item
  じゃがいものデュシェスを舟形または円盤状、あるいはブリオシュ型に詰めて整形し、溶き卵\footnote{dorer
    (ドレ)\textless{} dorure
    (ドリュール)焼いた際に艶を出すために塗る溶き卵。水や牛乳などを混ぜることもある。}を塗って、提供直前にオーブンでこんがり焼いたもの20個。
\item
  \protect\hyperlink{sauce-madere}{ソース・マデール}
\end{itemize}

\hypertarget{garniture-a-la-favorite}{%
\subsubsection[ガルニチュール・ラファヴォリータ]{\texorpdfstring{ガルニチュール・ラファヴォリータ\footnote{『愛の妙薬』や『ランメルモールのルチア』で知られる作曲家ガエタノ・ドニゼッティ(1797〜1848)のグランドオペラ\emph{La
  Favorite}(1840
  年初演)にあやかって付けられた名称。グランドオペラ(grand opéra
  グロントペラ、複数形grands
  opérasグロンゾペラ)とは19世紀前半から中葉にかけて、パリのオペラ座において、豪華な舞台装置と派手な演出、大編成のオーケストラ、歴史的題材などをテーマとしたわかりやすい悲劇的筋書きなどを特徴としたオペラ作品の様式のこと。ジャコモ・マイアベーア『悪魔ロベール』(1831年)や『ユグノー教徒』(1836年)などが代表的。なお、ロッシーニはこの様式が流行る前のオペラ作曲家と位置付けられていることが多いが、『ウィリアム・テル』(1829年。フランス語原題
  \emph{Guillaume
  Tell}ギヨーム・テル)あるいはそれに先立つ1827年の『モーセとファラオン』をこのジャンルの嚆矢と見なす場合もある。その他の代表的なグランドオペラの作曲家にダニエル=フランソワ・オーベール(1782〜
  1871)やジャック=フロマンタル・アレヴィ(1799〜1862)がいる。ドニゼッティのこの作品もロッシーニやマイアベーアの諸作品同様、フランス語の台本、歌詞であり、原題もフランス語で
  \emph{La
  Favorite}(ラファヴォリット)だが、どういうわけか、こんにちの日本ではイタリア語式に直した『ラファヴォリータ』と呼ばれることが多いためにここではそれに合わせた。なおこのオペラのリブレット(台本、歌詞)はアルフォンス・ロワイエとギュスターヴ・ヴァエズによるものだが、18世紀バキュラール・ダルノー(1718〜1805)の戯曲『不幸な恋人たち』を原作としている。さらにいえば、バキュラール・ダルノーの戯曲もまた、クロディーヌ・ゲラン・ド・トンサン(1682〜1749)の小説『コマンジュ伯爵の手記』を翻案したもの。『ロメオとジュリエット』の物語のバリエーションのひとつともいえるこの小説は18世紀に大きな反響を呼び、多くの小説、戯曲に影響を与えた。ダルノーの戯曲はその代表例。}}{ガルニチュール・ラファヴォリータ}}\label{garniture-a-la-favorite}}

\frsub{Garniture à la Favorite}

\index{garniture@garniture!favorite@--- à la Favorite}
\index{favorite@Favorige (La)!garniture@garuniture à la Favorite}
\index{かるにちゆーる@ガルニチュール!らふあうおりーた@---・ラファヴォリータ}
\index{らふあうおりーた@ラファヴォリータ!かるにちゆーる@ガルニチュール・---}

(ノワゼット、トゥルヌドに添える)

\begin{itemize}
\item
  小さめのフォワグラを厚さ1 cm程にスライス\footnote{éscalope
    (エスカロップ)。}し、塩こしょうしてから小麦粉をまぶしてバターでソテーしたもの10枚。
\item
  大きなトリュフのスライスをソテーしたフォラグラに1枚ずつのせる。
\item
  アスパラガスの穂先を束にしたもの。
\item
  ソース\ldots{}\ldots{}\protect\hyperlink{jus-de-veau-lie}{とろみを付けたジュ}。
\end{itemize}

\hypertarget{garniture-a-la-fermiere}{%
\subsubsection[ガルニチュール・フェルミエール]{\texorpdfstring{ガルニチュール・フェルミエール\footnote{日本語にすれば「農場主風」。野菜を厚さ1
  mmくらい、長さ1 cm程度の四角形に切ることを détailler en paysanne
  (デタイエオンペイザーヌ)というが、そのpaysanneとはpaysan(ペイゾン=農民)の女性形であり、このガルニチュールでは野菜をすべてそのように切るところにかけての名称。}}{ガルニチュール・フェルミエール}}\label{garniture-a-la-fermiere}}

\frsub{Garniture à la Fernière}

\index{garniture@garniture!fermiere@--- à la Fermière}
\index{fermier@fermier/fermière!garniture@garuniture à la Fermière}
\index{かるにちゆーる@ガルニチュール!ふえるみえーる@---・フェルミエール}
\index{ふえるみえ@フェルミエ/フェルミエール!かるにちゆーる@ガルニチュール・フェルミエール}

(鶏料理に添える)

\begin{itemize}
\item
  にんじん150 gと蕪150 gは厚さ1 mm程度、長さ1
  cm程度の四角形に切る\footnote{原文émincer en paysanne
    (エマンセオンペイザーヌ)。ペイザーヌに切る場合、動詞にはémincer
    薄くスライスする、も使われる。}。
\item
  玉ねぎ50 gとセロリ50 gも同様に切る。
\item
  これらを鍋に入れてバターと、塩3g、粉砂糖5
  gを加えて蓋をして弱火で軽く蒸し煮する\footnote{étuver (エチュヴェ)。}。
\item
  野菜を鶏の周囲に盛り、野菜の火入れを仕上げる\footnote{このガルニチュールは\protect\hyperlink{poulet-saute-a-la-fermiere}{鶏のソテー・フェルミエール}に添えるという前提がある。表面に焼き色を付けた鶏を、あらかじめ軽く蒸し煮しておいたこのガルニチュール・フェルミエールとともに陶製の鍋に入れて、さいの目に切ったハムを加え、蓋をしてオーブンに入れて鶏と野菜の火入れを仕上げることになる。}。
\end{itemize}

\hypertarget{garniture-a-la-financiere}{%
\subsubsection[ガルニチュール・フィナンシエール]{\texorpdfstring{ガルニチュール・フィナンシエール\footnote{徴税官風の意。名称について詳しくは\protect\hyperlink{sauce-financiere}{ソース・フィナンシエール}訳注参照のこと。なお、カレームは『19世紀フランス料理』で、多少の違いはあるが、これらの具材とソースを合わせることで「ラグー・フィナンシエール」と呼んでいる(t.3,
  pp.146-148)。これはつまり、ソース・フィナンシエールの訳注でも述べたように、もともとはガルニチュールとソースが別々のものではなく、一体化したものとして調理されていたことを示唆している。実際、このフィナンシエールという料理名の初出と思われる1755年ムノン『宮廷の晩餐』第2巻「肥鶏・フィナンシエール」および第3巻「鯉・フィナンシエール」は19世紀のものと内容、素材は違えどラグーとして扱われている。前者は肥鶏を掃除して中抜きした後、背から開いて骨を取り除き、大きなトリュフ4個とフォワグラとマッシュルーム、おろした豚背脂、卵黄、粒こしょう、バジルの粉末を混ぜて詰める。これを豚背脂のシートで包んで鍋に入れ、液体は注がずに熱い灰の上に鍋を置く。加熱していると肉汁などが出てくる。そこにビガラード(南フランスのビターオレンジ)の搾り汁と塩、こしょうで調味する(p.280)。後者は大きな鯉を掃除し、舌を残すようにしてエラは取り除く。片側の包丁を入れていない面の皮を剥がし、細かく刻んだ豚背脂を表面に刺す。鯉の中に詰めるラグーを作る。仔牛胸腺肉、トリュフ、フォワグラ、マッシュルームを鍋にバター1片とともに入れ、パセリ、シブール、にんにく、クローブ、バジルのブーケガルニを加える。鍋を火にかけて、小麦粉を振りかけ、シャンパーニュをグラス2杯注ぐ。塩こしょうで調味し、具材に火を通す。浮き脂を取り除き、冷めたら鯉の腹に詰め、ラグーが出てこないようしっかり鯉の腹を縫う。鯉の大きさにぴったりの魚用鍋に、ハムのスライスとたっぷりの仔牛腿肉のスライスを敷き、その上に鯉をのせる。豚背脂のシートで多い、玉ねぎのスライス、根菜の皮、パセリ、シブール、にんにく、クローブ、タイム、ローリエの葉、バルジのブーケガルニを入れる。中火にかけて汗をかかせるイメージで少し火を通し、それから上等のブイヨンとシャンパーニュを同量ずつ、鯉が液体に浸るまで注ぐ。塩こしょう。弱火にかける。火が通ったら鍋から鯉を取り出して水気をきる。縫った糸を取り除き、仔牛のグラスを塗って艶を出す。周囲にはお好みでエクルヴィスやまるごとのトリュフ、大きな鶏のとさか、鶏の胸肉、ペルドロー、こんがり焼いた鳩などをセンスよく配する。ソース・エスパニョルを添えて供する(pp.43-44)。その後、19世紀になるとヴィアール『帝国料理の本』1806年において「鳩・フィナンシエール」のレシピが掲載される。概要は、鳩6羽をバター、塩こしょう、レモン果汁を入れた鍋でさっと表面を色付けないように焼き固めたら、豚背脂で包んで鍋に入れ、ポワル(ここではソースの一種と考えていい)を注ぎ、柔らかく火を通す。提供直前に水気をきり、皿の周囲に鳩を配置する。その中央に雄鶏のとさかとロニョン、フォワグラ、トリュフのラグーを流し入れる
  (p.332)。残念ながらヴィアールの1806、1807年版は不完全なもののため、このラグーのレシピそのものは1820の第10版でようやく掲載に至る。概要は、マッシュルーム大24個、ボール形にしたトリュフ24個は辛口のマデイラ酒\(\frac{1}{2}\)瓶とともに鍋に入れ、唐辛子2本、トマト少々、仔牛のグラス1オンスを加えて火にかけてほとんどシロップ状になるまで煮詰める。それからソース・エスパニョルをレードル4杯、仔牛のブロンド(ソース)スプーン2杯を注いでよく混ぜる。沸騰させたら火の弱い場所に移して浮き脂を取り除き、煮詰めていく。このソースを布で漉し、きれいな鍋にマッシュルームとトリュフを移し入れて漉したソースを注ぐ。ここに雄鶏のとさかとロニョン24個ずつ、スプーンで整形したクネル24個、仔羊または仔牛の胸腺肉のスライス24枚を入れる(p.67)。この時点つまり1820年頃には、カレームが記した「ラグー・フィナンシエール」とほぼ同じ内容になっていることが注目されよう。カレームの「ラグー・フィナンシエール」は、トリュフ500
  gを円く整形し、マデイラ酒を加えて10分間弱火で蒸し煮する。ここにソース・フィナンシエールを注ぐ。ひと煮立ちさせたら、マデイラ酒から出たアクを取り除き、マッシュルーム12個、鶏のとさか12個(マデイラ酒少々を煮立たせて火を通しておく)、雄鶏のロニョン
  12個を加える。ひと煮立ちさせたらバター少々、鶏のクネル、フォラグラの1〜2
  cmのスライス、仔羊胸腺肉を加える。このラグーの半量はこれを添える料理上に盛り、周囲に白い立派な鶏のとさかとロニョンを配する。ソースが皿の上の料理をのせる台からはみ出ないようにしないと優美さが失なわれ、皿の縁飾りが乱れてしまうことに注意。ラグーの残りはソース入れで別添で供する(pp.146-147)。カレームはもうひとつ、「フォワグラのラグー・フィナンシエール」というレシピも残している(pp.147-148)。デュボワ、ベルナール共著『古典料理』(1868年)では「ガルニチュール・フィナンシエール」の項目は見られず、「サルピコン・フィナンシエ」
  (p.65)\ldots{}\ldots{}
  これは黒トリュフ、鶏胸肉、赤く漬けた舌肉、マッシュルームに火を通して小さなさいの目に切り、茶色いソース・フィナンシエールであえたもの、となっている。この他、「仔牛の耳・フィナンシエール
  (p.167)」「ほろほろ鳥のフィレ・フィナンシエール(p.179)」「ベカシーヌのグラタン・フィナンシエール(p.187)」「雉のクネル・フィナンシエール(p.190)」「温製パイ包み焼き・フィナンシエール(p.210)」「うずら・フィナンシエール(p.228)」「鳩・フィナンシエール(p.223)」といったレシピが収録されている。これらのレシピを見ると「ガルニチュール・フィナンシエール」を用いる指示になっているものがほとんどのため「ガルニチュール・フィナンシエール」の項が抜けているのは執筆あるいは何らかのミスによるものに過ぎないだろうと思われる。}}{ガルニチュール・フィナンシエール}}\label{garniture-a-la-financiere}}

\frsub{Garniture à la Financière}

\index{garniture@garniture!financiere@--- à la Financière}
\index{financier@financier/financère!garniture@garuniture à la Financière}
\index{かるにちゆーる@ガルニチュール!ふいなんしえーる@---・フィナンシエール}
\index{ふいなんしえ@フィナンシエ/フィナンシエール!かるにちゆーる@ガルニチュール・フィナンシエール}

(牛、羊の塊肉あるいは鶏料理に添える)

\begin{itemize}
\item
  仔牛か鶏のファルスでつくった標準的なクネル20個。ファルスに仔牛を使うか鶏を使うかは、このガルニチュールを添える料理に合わせて決めること。
\item
  渦巻状の刻み模様を入れた小さなマッシュルーム150 g。
\item
  雄鶏のとさかとロニョン\footnote{rognonは通常「腎臓」を指すが、雄鶏の場合はrognon
    blanc(ロニョンブロン)=
    testicule(テスティキュル)すなわち精巣のこと。}100 g。
\item
  トリュフのスライス50 g。
\item
  皮を剥いて下茹でしたオリーブ12個。
\item
  \protect\hyperlink{sauce-financiere}{ソース・フィナンシエール}
\end{itemize}

\hypertarget{garniture-a-la-flamande}{%
\subsubsection[ガルニチュール・フランドル風]{\texorpdfstring{ガルニチュール・フランドル\footnote{flamand(e)
  (フラモン/フラモンド) \textless{} Flandre
  (フロンドル)フランドル地方 =
  現在のベルギー西部からフランス北部にかけての北海に面する地域。フランダース。ただし『フランダースの犬』はイギリスの児童文学なので、フランスおよびベルギーではあまり知られていない。}風}{ガルニチュール・フランドル風}}\label{garniture-a-la-flamande}}

\frsub{Garniture à la Flamande}

\index{garniture@garniture!flamande@--- à la Flamande}
\index{flamand@flamand(e)!garniture@garuniture à la ---e}
\index{かるにちゆーる@ガルニチュール!ふらんとるふう@---・フランドル風}
\index{ふらんとるふう@フランドル風!かるにちゆーる@ガルニチュール・---}

\begin{itemize}
\item
  球形に整形した小さな\protect\hyperlink{chou-braise}{サヴォイキャベツのブレゼ}10個。
\item
  オリーブ形に整形し、コンソメで煮たにんじんと蕪、各10個ずつ。
\item
  \protect\hyperlink{pommes-de-terres-a-l-anglaise}{じゃがいものアラングレーズ}\footnote{à
    l'anglaise
    イギリス風、の意だが、必ずしもイギリス料理に由来するとは限らない。野菜の場合、アラングレーズとはすなわち「塩を加えた湯で茹でる」ことを意味するが、本書の該当個所にも、イギリスでは塩を加えない、とある。}小10個。
\item
  塩漬け豚バラ肉250
  gは10枚の長方形の板状に切り、キャベツとともにブレゼする。
\item
  輪切りにしたソシソン\footnote{熟成、乾燥させてつくる太いソーセージ。多くの場合、調理せず薄切りにして食べる。}10枚(150
  g)。
\item
  塊肉をブレゼした煮汁\footnote{このガルニチュールは\protect\hyperlink{piece-de-boeuf-a-la-flammande}{牛塊肉・フランドル風}に添えるのを前提に書かれているため、ブレゼと特定出来るが、本書における\protect\hyperlink{les-poeles}{ポワレ}の手法でももちろん可能だろう。}をソースに仕上げる。
\end{itemize}

\hypertarget{garniture-a-la-florentine}{%
\subsubsection[ガルニチュール・フィレンツェ風]{\texorpdfstring{ガルニチュール・フィレンツェ風\footnote{florentin(e)
  (フロロンタン/フロロンティーヌ)\textless{} Florence
  (フロロンス)フィレンツェのこと。}}{ガルニチュール・フィレンツェ風}}\label{garniture-a-la-florentine}}

\frsub{Garniture à la Florentine}

\index{garniture@garniture!florentine@--- à la Florentine}
\index{florentin@florentin(e)!garniture@garuniture à la ---e}
\index{かるにちゆーる@ガルニチュール!ふいれんつえふう@---・フィレンツェ風}
\index{ふいれんつえふう@フィレンツェ風!かるにちゆーる@ガルニチュール・---}

(魚料理に添える場合)

\begin{itemize}
\item
  ほうれんそうの葉250 gは下茹でしてから、バターで蒸し煮する\footnote{étuver
    au beurre (エチュヴェオブール)}。
\item
  このほうれんそうを皿の底に敷き、その上に煮立たせないように茹で\footnote{pocher
    (ポシェ)。}
  て火を通した魚をのせ、\protect\hyperlink{sauce-mornay}{ソース・モルネー}を覆いかける。高温のオーブンに入れて焼き色を付ける。
\end{itemize}

(牛、羊の塊肉の料理に添える場合)

\begin{itemize}
\item
  \protect\hyperlink{sucric-d-epinards}{ほうれんそうのシュブリック}10個。
\item
  セモリナ粉を獣脂で加熱し、卵とおろしたチーズを混ぜ込んだアパレイユで円盤状につくった小さな\protect\hyperlink{croquettes}{クロケット}10個。
\item
  トマト風味を効かせ、ただしよく澄んだ状態の\protect\hyperlink{sauce-demi-glace}{ソース・ドゥミグラス}。
\end{itemize}

\hypertarget{garniture-Florian}{%
\subsubsection[ガルニチュール・フローリアン]{\texorpdfstring{ガルニチュール・フローリアン\footnote{ヴェネツィア、サンマルコ広場にある18世紀からあるカフェ。}}{ガルニチュール・フローリアン}}\label{garniture-Florian}}

\frsub{Garniture Florian}

\index{garniture@garniture!florian@--- Florian}
\index{florian@Florian!garniture@garuniture ---}
\index{かるにちゆーる@ガルニチュール!ふろーりあん@---・フローリアン}
\index{ふろーりあん@フローリアン!かるにちゆーる@ガルニチュール・---}

(乳呑仔羊\footnote{本書で
  agneauという場合には、いわゆるプレサレ(agneau de pré-salé
  アニョドプレサレ)はmouton(ムトン=羊の成獣)に準ずる扱いであり、それ以外は基本的にagneau
  de lait (アニョドレ)乳呑仔羊を指すことに留意。}の料理に添える\footnote{\protect\hyperlink{epaule-d-agneau-florian}{乳呑仔羊肩肉・フローリアン}に添えるのを前提としたガルニチュールであることに留意。。})

\begin{itemize}
\item
  大きめのレチュ3個は四つ割りにして外葉を取り除き、ブレゼする\footnote{\protect\hyperlink{laitue-braisee}{レチュのブレゼ}参照。}。
\item
  オリーブの大きさと形にしたにんじん20個は下茹でしてバターで色艶よく火を通す\footnote{glacer
    (グラセ)。これらの野菜の場合は下茹でして半ば火を通しておくことと、必要に応じて砂糖を加える場合があることに留意。}。
\item
  小玉ねぎ20個は下茹でしてバターで色艶よく炒める。
\item
  小さな\protect\hyperlink{pommes-de-terre-fondantes}{じゃがいものフォンダント}10個。
\item
  ソース\ldots{}\ldots{}仔羊の肉汁\footnote{原文ではfondsとなっているが、前提となっている仕立て「乳呑仔羊肩肉・フローリアン」の場合はバターをかけながらローストするので、いわゆる「ジュ」と考えていい。}。
\end{itemize}

\hypertarget{garniture-a-la-Forestiere}{%
\subsubsection[ガルニチュール・フォレスティエール]{\texorpdfstring{ガルニチュール・フォレスティエール\footnote{forestier
  (フォレスティエ)形容詞は森林の、の意。名詞の場合は森林管理人。一般には「森番風」などと訳されることが多いようだ。}}{ガルニチュール・フォレスティエール}}\label{garniture-a-la-Forestiere}}

\frsub{Garniture à la Forestière}

\index{garniture@garniture!forestiere@--- à la Forestière}
\index{forestier@forestier/forestière!garniture@garuniture à la Forestière}
\index{かるにちゆーる@ガルニチュール!ふおれすていえーる@---・フォレスティエール}
\index{ふおれすていえーる@フォレスティエール!かるにちゆーる@ガルニチュール・---}

(牛、羊の塊肉や鶏の料理に添える)

\begin{itemize}
\item
  モリーユ\footnote{茸の一種。和名アミガサタケ。生食出来ないので注意。}300
  gはバターと植物油同量ずつでソテーする。
\item
  脂身の少ない豚バラ肉の塩漬け125
  gは拍子木に切って下茹でし、バターでこんがりと焼く\footnote{rissoler
    (リソレ)。油脂を鍋に熱し、高温で素材の表面に焼き色を付けること。}。
\item
  じゃがいも300 gは大きめのさいの目に切ってバターでソテーする。
\item
  ブレゼの煮汁あるいはデグラセした液体を加えた\protect\hyperlink{sauce-duxelles}{ソース・デュクセル}
\end{itemize}

\hypertarget{garniture-frascati}{%
\subsubsection[ガルニチュール・フラスカーティ]{\texorpdfstring{ガルニチュール・フラスカーティ\footnote{フラスカーティは古代ローマの避暑地として有名だったところ。19世紀にナポリ出身のアイスクリーム職人ガルキがパリのブルヴァール・イタリアンにフラスカーティという店名のカジノ、パティスリを開き盛況だったというが、1837年のカジノ禁止令により閉店を余儀なくされたという。このガルニチュールおよび「牛フィレ肉・フラスカーティ」がどちらに由来しているかは不明。ちなみに、ロラン・バルトがバルザックの短編『サラジーヌ』の徹底した構造分析を試みた『S/Z』において、分析の山場ともなる『サラジーヌ』の場面、去勢歌手(カストラート)ザンビネッラを女性と思い込んで恋をした彫刻家サラジーヌが劇団員らとフラスカーティに遠足に出かけた際に、突然現われた蛇に驚いたザンビネッラの姿にやはり女性であると主人公が確信を深める場面(この場面について蛇を男根の象徴とバルトは分析している)がある。なお、バルザックが『サラジーヌ』をものした時代にはカストラートはほとんど存在せず、オペラにおいてその役割はコントラルトが担うようになっていた。したがって、カストラートが「女性的」な見た目であるという一種の偏見のようなものは小説家の念頭にあった可能性は否定出来ない。もっとも、18世紀のカサノヴァ『回想録』においてもカストラートを女性として扱い、口説く場面があり、他にもいわゆる艶書の類においてそういった筋立のものは存在したので、バルザックの想像が突飛でオリジナルなものだったとはいいきれない。また、シェイクスピアの時代など、舞台の役者が皆男性で、女性役も若い少年俳優が演じるなど珍しくはなかったし、日本の歌舞伎では現代でもその伝統は続けられている。また、女性のみで構成される劇団も存在するし、リヒャルト・シュトラウスのオペラ『薔薇の騎士』での主人公である青年は当初からコントラルト(つまり女性歌手)が歌う前提で作曲されている。つまり、舞台芸術においては「性」(ジェンダーおよびセクシャリティの両方)のある種の「ゆらぎ」は決して珍しいものではない。ただそれを「小説」という形式で、しかもいわゆる「艶書」ではなく文学的にすぐれたものとして仕上げたところにバルザックの天才を見いだすべきだろう。こうしたことから、物語を徹底的に分解して文化的および社会的なコード体系にひとつひとつ当て嵌めて分析していったバルトの試みとその手腕はきわめて高く評価されるべきものだろうが、文学作品の分析としては不完全なものと断ぜざるを得ない。}}{ガルニチュール・フラスカーティ}}\label{garniture-frascati}}

\frsub{Garniture Frascati}

\index{garniture@garniture!forestiere@--- Frascati}
\index{frascati@Frascati!garniture@garuniture ---}
\index{かるにちゆーる@ガルニチュール!ふらすかーてい@---・フラスカーティ}
\index{ふらすかーてい@フラスカーティ!かるにちゆーる@ガルニチュール・---}

(牛、羊の塊肉の豪華な仕立てに添える\footnote{直訳すると「牛、羊の塊肉のルルヴェ用」(ルルヴェについては「\protect\hyperlink{releve}{第二版序文訳注}」参照)だが、本書では「\protect\hyperlink{filet-de-boeur-frascati}{牛フィレ肉・フラスカーティ}」くらいしか目ぼしいレシピがない。})

\begin{itemize}
\item
  厚さ1〜2 cm程度スライスした\footnote{escalope (エスカロップ)。}フォワグラ(出来るだけ生のものがいい)10枚に小麦粉をまぶし、バターでソテーする。
\item
  アスパラガスの穂先300 gは茹でてからバターであえる。
\item
  小さめの真っ白なマッシュルーム10個は軸を落とし、渦巻状に飾り模様を入れる。
\item
  大きめのオリーブくらいのサイズに整形したトリュフ10個はバターでかるく炒めて艶を出す。
\item
  トリュフ風味にした\protect\hyperlink{pommes-de-terre-duchesse}{じゃがいものデュシェス}をアパレイユにして細長く作ったクロワッサン10個は提供直前に溶き卵を塗り、オーブンで焼いて艶を出す。このクロワッサンを並べてでガルニチュールの外枠にする。
\item
  軽くとろみを付けた肉汁(ジュ)\footnote{「牛フィレ肉・フラスカーティ」の場合は\protect\hyperlink{les-poeles}{ポワレ}するので、適量のフォンを肉を加熱する際鍋の底に敷いたマティニョンに注ぎ、肉汁の風味をひき出してから布で漉し、でんぷんでとろみを付けることになる。}。
\end{itemize}

\hypertarget{garniture-a-la-gastronome}{%
\subsubsection[ガルニチュール・ガストロノーム]{\texorpdfstring{ガルニチュール・ガストロノーム\footnote{美食家、食通、の意。}}{ガルニチュール・ガストロノーム}}\label{garniture-a-la-gastronome}}

\frsub{Garniture à la Gastronome}

\index{garniture@garniture!gastronome@--- à la Gastronome}
\index{gastronome@gastronome!garniture@garuniture à la ---}
\index{かるにちゆーる@ガルニチュール!かすとろのーむ@---・ガストロノーム}
\index{かすとろのーむ@ガストロノーム!かるにちゆーる@ガルニチュール・---}
\index{ひしよくかふう@美食家風 ⇒ ガストロノーム!かるにちゆーる@ガルニチュール・---}

(牛、羊の塊肉および鶏の料理に添える)

\begin{itemize}
\item
  大きめのマロン20個は、皮を剥いてコンソメで煮、小玉ねぎのようにバターで色艶よく炒める\footnote{glacer
    (グラセ)。}。
\item
  中位のサイズのトリュフ10個はシャンパーニュ風味に茹でる。
\item
  立派な雄鶏のロニョン\footnote{rognons de coq
    (ロニョンドコック)ここでは鶏の睾丸のこと。}20個はブロンド色の\protect\hyperlink{glace-de-viande}{グラスドヴィアンド}でコーティングする。
\item
  大きなモリーユ\footnote{morille
    茸の一種。和名アミガサタケ。生食不可なのでよく加熱する必要がある。}は縦二つ割りにし、バターでソテーする。
\item
  トリュフエッセンスを加えた\protect\hyperlink{sauce-demi-glace}{ソース・ドゥミグラス}。
\end{itemize}

\hypertarget{garniture-Godard}{%
\subsubsection[ガルニチュール・ゴダール]{\texorpdfstring{ガルニチュール・ゴダール\footnote{18世紀の徴税官(つまりフィナンシエ)であり作家としても活動したクロード・ゴダール・ドクール
  Claude Godard d'Aucour (1716〜
  1795)の名を冠したものと考えられる。\protect\hyperlink{sauce-godard}{ソース・ゴダール}も参照のこと。本書のレシピだけを見ていると\protect\hyperlink{garniture-a-la-financiere}{ガルニチュール・フィナンシエール}と非常によく似ているけれどもソースの違うパターン、くらいにしか見えないかも知れぬが、このガルニチュールのほうが圧倒的に大掛かりで豪華な仕立てにすることを前提としており、ガルニチュール・ゴダールあるいはゴダールという名称の仕立てはフィナンシエールの完成形というべきか、究極の到達点のひとつだったのではないか?
  19世紀前半をとおして版を重ね、そのたびに増補されたヴィアールの本を版ごとに見ていくと、初版から
  1817年の第9版まではフィナンシエールのみ。1820年の第10版以降から「牛アロワイヨ・ゴダール」というレシピが登場する。長いレシピなので要点だけ見ると、約7〜8
  kgの牛アロワイヨ(日本語では「腰肉」すなわちフィレを含むサーロインからランプ、イチボにかけての部分)を四角形に切り整えて骨は取り除き、紐で縛ってからマデイラ酒を加えて6時間ブレゼする。肉を取り出したら煮汁を漉して、卵白でクラリフィエし、さら布で漉して煮詰める。その煮汁の半分にコンソメを足して、肉を鍋に戻し入れてさらに2時間弱火で煮込む。肉を皿に盛り付け、周囲に若鳩4羽、拍子木に切った豚背脂やトリュフ、赤く漬けた舌肉を表面に刺して装飾した仔羊胸腺肉4枚、スプーンで整形した大きなクネル8個、大きなエクルヴィス8尾、鶏またはその他の揚げものを刺した飾り串8本をアロワイヨの上から刺す。ドゥミグラス半量と合わせたラグー・フィナンシエールをかける。強火のオーブンで照りを付け、熱々を供する(p.101)。これを見るかぎり、本質的にはフィナンシエールの変形もしくは豪華版と考えていいだろう。カレーム『19世紀フランス料理』では「牛アロワイヨのブレゼ・ゴダール」に2種のレシピが記述されており、ひとつは上記と似たアロワイヨ全体をブレゼしたもの。もうひとつはアロワイヨの一部をブレゼし、残りはローストにした仕立てになっている。いずれにしても、非常に豪華な仕立てであり、ものすごいコストがかかるため、きわめて格式の高い荘厳な宴席でしか出来ないだろうが「食卓外交に携わる料理人はこうした料理の知識を大切にして、これらの料理を供すべきだ」と述べている(t.3,
  p.327)。デュボワ、ベルナール共著『古典料理』に至るとむしろゴダールという仕立ては簡略される方向に向かい、本書と同様に「ルルヴェ用ガルニチュール・ゴダール」として記述される。「このガルニチュールはトリュフで装飾を施した大きなクネル、表面に装飾をしてソースをかけてオーブンで焼き色を付けた仔牛胸腺肉、トリュフ、マッシュルームで構成される。これらを各まとまりごとに料理の周囲に添える。クネルとマッシュルームには軽くソースをかけてやり、トリュフと仔牛胸腺肉には艶を出させてやる(グラセ)こと」(p.94)とある。こうしたことから、フィナンシエールおよびその発展形としての仕立てであるゴダールが19世紀後半にむかってだんだんと広まっていき、盛んに作られるようになったが、ゴダールについてはそのコストゆえに簡素化していく傾向にあった。いずれにしても両者ともにきわめて19世紀的なソースとガルニチュールの組み合わせ、あるいは料理の仕立てといえよう。}}{ガルニチュール・ゴダール}}\label{garniture-Godard}}

\frsub{Garniture Godard}

\index{garniture@garniture!godard@--- Godard}
\index{godard@Godard!garniture@garuniture ---}
\index{かるにちゆーる@ガルニチュール!こたーる@---・ゴダール}
\index{こたーる@ゴダール!かるにちゆーる@ガルニチュール・---}

(牛、羊、鶏の大掛かりで豪華な仕立てに添える)

\begin{itemize}
\item
  マッシュルームとトリュフのみじん切りを加えた、\protect\hyperlink{farce-a}{バター入りのファルス}をスプーンで整形したクネル10個。
\item
  トリュフと赤く漬けた舌肉で装飾を施した大きな楕円形のクネル4個。
\item
  小さめのマッシュルーム10個は軸を除き、螺旋状に切れ込み模様を付ける。
\item
  雄鶏のとさかとロニョン125 g。
\item
  上等の仔羊胸腺肉200
  gは高温のオーブンで焼き色を付ける。または仔牛胸腺肉の喉側を高温のオーブンで焼き色を付け、スライスする
\item
  オリーブ形に整形したトリュフ10個。
\item
  \protect\hyperlink{sauce-godard}{ソース・ゴダール}
\end{itemize}

\hypertarget{garniture-grand-duc}{%
\subsubsection[ガルニチュール・グランデュック]{\texorpdfstring{ガルニチュール・グランデュック\footnote{grand-duc
  (グロンデュック)大公およびロシアの皇太子の意。
  Prince(プランス)大公とほぼ同義だが使われる場面などで違いがある。料理においてはアスパラガスの穂先とトリュフを用いた料理に付されることが多い。}}{ガルニチュール・グランデュック}}\label{garniture-grand-duc}}

\frsub{Garniture Grand-Duc}

\index{garniture@garniture!grand-duc@--- Grand-Duc}
\index{grand-duc@grand-duc!garniture@garuniture ---}
\index{かるにちゆーる@ガルニチュール!くらんていゆつく@---・グランデュック}
\index{くらんていゆすく@グランデュック!かるにちゆーる@ガルニチュール・---}
\index{たいこう@大公(風)⇒グランデュック!かるにちゆーる@ガルニチュール・---}

(魚料理に添える)

\begin{itemize}
\item
  アスパラガスの穂先200gは下茹でしてバターであえる。
\item
  殻をむいたエクルヴィスの尾の身10。
\item
  大きなトリュフのスライス10枚。
\end{itemize}

\hypertarget{garniture-a-la-grecque}{%
\subsubsection[ガルニチュール・ギリシア風]{\texorpdfstring{ガルニチュール・ギリシア風\footnote{grec
  / grecque
  は「ギリシアの」の意。ここではあえて「ギリシア風」訳したが、ギリシアに起源あるいは縁のないと思われる調理が少なくないので注意。}}{ガルニチュール・ギリシア風}}\label{garniture-a-la-grecque}}

\frsub{Garniture à la Grecque}

\index{garniture@garniture!grecque@--- à la  Grecque}
\index{grec@grec/grecque!garniture@garuniture à la grecque}
\index{かるにちゆーる@ガルニチュール!くれつく@---・グレック}
\index{くれいつく@グレック!かるにちゆーる@ガルニチュール・---}
\index{きりしあふう@ギリシャ風⇒グレック!かるにちゆーる@ガルニチュール・---}

(乳呑仔羊および鶏料理に添える)

\begin{itemize}
\item
  \protect\hyperlink{riz-a-la-grecque}{ギリシア風ライス}\footnote{ピラフの一種だが、実際のところまったくギリシア風ではないことに注意。}250
  g(\protect\hyperlink{riz}{野菜料理「米」の項}参照)。
\item
  \protect\hyperlink{sauce-tomate}{トマトソース}
\end{itemize}

\hypertarget{garniture-henri-iv}{%
\subsubsection{ガルニチュール・アンリ4世亭風}\label{garniture-henri-iv}}

\frsub{Garniture Henri IV }

\index{garniture@garniture!henri iV@--- Henri IV}
\index{henri iv@Henri IV!garniture@garuniture ---}
\index{かるにちゆーる@ガルニチュール!あんりよんせいていふう@---・アンリ4世亭風}
\index{あんりよんせいていふう@アンリ4世亭風!かるにちゆーる@ガルニチュール・---}

(ノワゼットやトゥルヌドに添える)

\begin{itemize}
\tightlist
\item
  肉に合わせて中くらいから小さめのアーティチョークの芯\footnote{比較的小ぶりであっても完熟のアーティチョーク(開花がやや近い状態のもの)は下茹で後に花萼をすべて取り除く。この状態を
    fond d'artichaut (フォンダルティショー)またはcul d'artichaut
    (キュダルティショー)と呼ぶ。とりわけ大きなアーティチョークは花萼部が完全に固いことが多いために、上半分よりやや下で切り離して、繊毛を取り除いてから下茹でする。花萼を全て剥いて皿のような形状の基底部のみを取り出す。丸い皿のような底面になるので、そこに詰め物をすることが多い。小さく比較的若どりのアーティチョークは花萼を全部は取り除かず、周囲の固いところを2周くらい剥いて使う。これを
    coeur d'artichaut
    (クールダルティショー)と呼ぶ。サイズによっては縦半分または四つ割りにして繊毛を取り除いてから下茹でする。四つ割りの場合はquartiers
    d'artichaut
    (カルティエダルティショー)と呼ぶ。若どりのアーティチョークの内側の花萼は柔らかく火が通り、とても美味。また、生食できるくらい若どりのアーティチョークはpoivrade(ポワヴラード)とも呼ばれる。ただし若どりであればそれだけ、アーティチョーク特有の風味は弱い。}に、溶かした\protect\hyperlink{glace-de-viande}{グラスドヴィアンド}の中に入れて転がしてグラスをコーティングさせた小さな\protect\hyperlink{pommes-de-terre-noisette}{じゃがいものノワゼット}を詰める。
\end{itemize}

\hypertarget{garniture-a-la-hongroise}{%
\subsubsection[ガルニチュール・ハンガリー風]{\texorpdfstring{ガルニチュール・ハンガリー風\footnote{この名称の根拠となっているのはパプリカを使用していることのみ。\protect\hyperlink{sauce-hongroise}{ハンガリー風ソース}訳注も参照。}}{ガルニチュール・ハンガリー風}}\label{garniture-a-la-hongroise}}

\frsub{Garniture à la Hongroise}

\index{garniture@garniture!hongroise@--- à la Hongroise}
\index{hongrois@hongrois(e)!garniture@garuniture à la ---e}
\index{かるにちゆーる@ガルニチュール!はんかりーふう@---・ハンガリー風}
\index{はんかりーふう@ハンガリー風!かるにちゆーる@ガルニチュール・---}

(いろいろな料理に添えられる)

\begin{itemize}
\item
  小房に分けたカリフラワーをクリームであえて、いくつかの小さな容器に詰め、バターを塗ったグラタン皿に裏返して並べ、上からおろしたチーズを振りかけ、みじん切りにしたハムを加えたパプリカ風味の\protect\hyperlink{sauce-mornay}{ソース・モルネー}で覆い、高温のオーブンに入れてこんがりと焼く。
\item
  パプリカで風味付けした軽いソースを添える。
\end{itemize}

\hypertarget{garniture-a-l-italienne}{%
\subsubsection{ガルニチュール・イタリア風}\label{garniture-a-l-italienne}}

\frsub{Garniture à l'Italienne}

\index{garniture@garniture!italienne@--- à l'Italienne}
\index{italien@italien(ne)!garniture@garuniture à l' ---ne}
\index{かるにちゆーる@ガルニチュール!いたりあふう@---・イタリア風}
\index{いたりあふう@イタリア風!かるにちゆーる@ガルニチュール・---}

(牛、羊の塊肉および鶏料理に添える)

\begin{itemize}
\item
  小さなアーティチョークを縦4つに切って、\protect\hyperlink{quartiers-d-artichauts-a-l-italienne}{イタリア風}に調理する(野菜料理「\protect\hyperlink{artichauts}{アーティチョーク}」参照)20個。
\item
  茹でたマカロニにたっぷりチーズをあえて円盤型にしたクロケット10個。
\item
  \protect\hyperlink{sauce-italienne}{イタリア風ソース}。
\end{itemize}

\hypertarget{garniture-a-l-indienne}{%
\subsubsection{ガルニチュール・インド風}\label{garniture-a-l-indienne}}

\frsub{Garniture à l'Indienne}

\index{garniture@garniture!indienne@--- à l'Indienne}
\index{indien@indien(ne)!garniture@garuniture à l' ---ne}
\index{かるにちゆーる@ガルニチュール!いんとふう@---・インド風}
\index{いんとふう@インド風!かるにちゆーる@ガルニチュール・---}

(魚、牛、羊の塊肉や鶏料理に添える)

\begin{itemize}
\item
  \protect\hyperlink{riz-a-l-indienne}{インド風に調理}したパトナ米\footnote{パトナはコメの品種名。いわゆる「長粒種」だがバスマティのような香り米ではない。}125
  g(野菜料理「\protect\hyperlink{riz}{米}」参照)。
\item
  \protect\hyperlink{sauce-a-l-indienne}{インド風ソース}。
\end{itemize}

\hypertarget{garniture-a-la-japonaise}{%
\subsubsection[ガルニチュール・日本風]{\texorpdfstring{ガルニチュール・日本風\footnote{このガルニチュールが「日本風」であるのは、ちょろぎを用いているから。中国原産のシソ科の根菜で、現代日本では慶事などの際に用いられる程度だが、どういうわけか日本原産と誤解されたまま19世紀にフランスで栽培されるようになり、以来、日本風の名を付けた料理にはほとんど必ずといっていい程、ちょろぎが用いられる。}}{ガルニチュール・日本風}}\label{garniture-a-la-japonaise}}

\frsub{Garniture à la Japonaise}

\index{garniture@garniture!japonaise@--- à la Japonaise}
\index{japonais@japonais(e)!garniture@garuniture à la ---e}
\index{かるにちゆーる@ガルニチュール!にほんふう@---・日本風}
\index{にほんふう@日本風!かるにちゆーる@ガルニチュール・---}

(牛、羊の塊肉の料理に添える)

\begin{itemize}
\item
  ちょろぎ625
  gは\protect\hyperlink{veloute}{ヴルテ}であえ、ブリオシュ型でつくりオーブンでこんがり焼いた\protect\hyperlink{croustades}{クルスタード}に詰める。
\item
  米の\protect\hyperlink{croquettes}{クロケット}10個\footnote{初版〜第三版は「\protect\hyperlink{croquettes-de-pommes-de-terre}{じゃがいものクロケット}10個」となっている。「じゃがいものクロケット」は初版からレシピが掲載されているが、第四版すなわち現行版において「米のクロケット」のレシピは掲載されておらず、米をアパレイユに用いるのは「\protect\hyperlink{croquettes-a-l-indienne}{インド風クロケット}」のみ。ほとんどのレシピが本書の内部参照のみで成立しうるなかではやや珍しいケースともいえる。}。
\end{itemize}

\hypertarget{garniture-a-la-jardiniere}{%
\subsubsection[ガルニチュール・菜園風]{\texorpdfstring{ガルニチュール・菜園風\footnote{jardinier/jardinière
  (ジャルディニエ、ジャルディニエール)には名詞で「庭師」の意味もあるが、ここでは
  jardin potager (ジャルダンポタジェ)すなわち野菜畑、菜園、の意。}}{ガルニチュール・菜園風}}\label{garniture-a-la-jardiniere}}

\frsub{Garniture à la Jardinière}

\index{garniture@garniture!jardiniere@--- à la Jardinière}
\index{jardinier@jardinier/jardinière!garniture@garuniture à la Jardinière}
\index{かるにちゆーる@ガルニチュール!さいえんふう@---・菜園風}
\index{さいえんふう@菜園風!かるにちゆーる@ガルニチュール・---}

(牛、羊の塊肉の料理に添える)

\begin{itemize}
\item
  にんじん125 gと蕪125
  gは、プレーンな、あるいは刻み模様の入ったスプーンでくり抜く。あるいは円柱形にしてもいい。これをコンソメで煮て、最後にバターで色艶よく炒める。
\item
  プチポワ125 g。小さなフラジョレ125 g。アリコ・ヴェール125
  gは小さな菱形に切る。これらを別々にバターであえる\footnote{しっかり加熱調理してからバターであえること。}。
\item
  提供食前に茹であがったばかりのカリフラワーの小房10個。
\item
  以上の構成要素を肉の周囲に、別々にはっきりとニュアンスが代わるように配置する。カリフラワーの小房は\protect\hyperlink{sauce-hollandaise}{オランデーズソース}小さじ1杯程度をそれぞれに塗ってやる。
\item
  ソース\ldots{}\ldots{}澄んだジュ(肉汁。
\end{itemize}

\hypertarget{garniture-joinville}{%
\subsubsection[ガルニチュール・ジョワンヴィル]{\texorpdfstring{ガルニチュール・ジョワンヴィル\footnote{フランソワ・ドルレアン・ジョワンヴィル海軍大将(1818〜
  1900)のこと。\protect\hyperlink{sauce-joinville}{ソース・ジョワンヴィル}も参照。}}{ガルニチュール・ジョワンヴィル}}\label{garniture-joinville}}

\frsub{Garniture Joinville}

\index{garniture@garniture!joinville@--- Joinville}
\index{joinville@Joinille!garniture@garuniture ---}
\index{かるにちゆーる@ガルニチュール!しよわんういる@---・ジョワンヴィル}
\index{しよわんういる@ジョワンヴィル!かるにちゆーる@ガルニチュール・---}

(魚料理に添える)

\begin{itemize}
\item
  以下のものを5 mm角くらいの小さな角切り\footnote{salpicon
    (サルピコン)。}か短かい拍子木状\footnote{julienne courte
    (ジュリエーヌクルト)。}に刻む\ldots{}\ldots{}茹でたマッシュルーム125
  g、トリュフ50 g。これにクルヴェットの尾の身125
  gを加え、スプーン数杯の\protect\hyperlink{sauce-joinville}{ソース・ジョワンヴィル}であえる。
\item
  追加として\ldots{}\ldots{}トリュフのスライス10枚。白くて大きなマッシュルームに殻をむいたクルヴェット8尾を刺す。
\item
  ソース・ジョワンヴィル。
\end{itemize}

\hypertarget{garniture-judic}{%
\subsubsection[ガルニチュール・ジュディック]{\texorpdfstring{ガルニチュール・ジュディック\footnote{女優アンナ・ジュディック(1849〜1911)の名を冠したもの。}}{ガルニチュール・ジュディック}}\label{garniture-judic}}

\frsub{Garniture Judic}

\index{garniture@garniture!judic@--- Judic}
\index{judic@Judic!garniture@garuniture ---}
\index{かるにちゆーる@ガルニチュール!しゆていつく@---・ジュディック}
\index{しゆていつく@ジュディック!かるにちゆーる@ガルニチュール・---}

(ノワゼットやトゥルヌド、鶏料理に添える)

\begin{itemize}
\item
  小さめのレチュを縦半割りにしてきれいに掃除し、ブレゼしたもの10個。
\item
  大きな雄鶏のロニョン10個。
\item
  トリュスのスライス10枚。
\item
  上等な仕上りの\protect\hyperlink{sauce-demi-glace}{ソース・ドゥミグラス}。
\end{itemize}

\hypertarget{garniture-languedocienne}{%
\subsubsection[ガルニチュール・ラングドック風]{\texorpdfstring{ガルニチュール・ラングドック風\footnote{languedocien(ne)
  (ラングドスィヤン、ラングドスィエーヌ)\textless{} Languedoc
  ラングドック地方。フランス南西部の地方名。もとは「オック語」langue
  d'oc から。中世プロヴァンス語と考えていい。オック oc
  とは古語で、現代フランス語の oui
  に相当する肯定の語。ラテン語の格変化の消失が比較的遅かった。バスク地方を除く(バスク語は別言語として扱われる)ロワール河以南で話された諸語の総称。これに対し、オイル語
  langue d'oil (ラングドイル)があり、oil
  が肯定の語であるというところが代表的な違い。主としてロワール河以北で話された諸語の総称。現代フランス語は後者の系統にあたるが、語彙の面などではラングドックの影響を大きく受けている。}}{ガルニチュール・ラングドック風}}\label{garniture-languedocienne}}

\frsub{Garniture Languedocienne}

\index{garniture@garniture!languedocienne@--- Languedocienne}
\index{languedocien@languedocien(ne)!garniture@garuniture ---}
\index{かるにちゆーる@ガルニチュール!らんくとつくふう@---・ラングドック風}
\index{らんくとつくふう@ラングドック風!かるにちゆーる@ガルニチュール・---}

(牛、羊の塊肉、鶏料理に添える)

\begin{itemize}
\item
  なすは1 cm厚の輪切りを10枚用意し、小麦粉をまぶして油で揚げる。
\item
  セープ\footnote{cèpe
    (セープ)、茸の一種、和名ヤマドリタケ。イタリアのポルチーニと同種だが、フランス産、イタリア産で風味や調理特性が異なる。また日本に多いのはヤマドリタケモドキという種で、食用できるが風味などはまったく及ばないという。類似のものにウツロイグチ、ドクヤマドリという毒茸があるので注意が必要。}400
  gはスライスして植物油でソテーする。
\item
  トマト400
  gは河を剥いて圧しつぶし、粗く刻んで、にんにく1片を加えて油でソテーする。
\item
  パセリのみじん切り。
\item
  ソース\ldots{}\ldots{}\protect\hyperlink{jus-de-veau-lie}{とろみを付けたジュ}。
\end{itemize}
\end{recette}