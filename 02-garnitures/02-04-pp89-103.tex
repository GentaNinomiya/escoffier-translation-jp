\hypertarget{garnitures-recettes}{%
\subsection{ガルニチュールのレシピ}\label{garnitures-recettes}}

\frsecb{Garnitures}

\begin{center}
\medlarge(ここで示す分量はすべて仕上がり10人分)
\end{center}
\normalsize
\begin{recette}
\hypertarget{garniture-algerienne}{%
\subsubsection{ガルニチュール・アルジェリア風}\label{garniture-algerienne}}

\frsub{Garniture à l'Algérienne}

\index{garniture@garniture!algerienne@--- à l'Algérienne}
\index{algerien@algérien(nne)!garuniture à l'---ne}
\index{かるにちゆーる@ガルニチュール!あるしえりあふう@---・アルジェリア風}
\index{あるしえりあふう@アルジェリア風!かるにちゆーる@ガルニチュール・---}

(牛、羊の塊肉の料理に添える)

\begin{itemize}
\item
  ワインの栓の形にしたさつまいもの\protect\hyperlink{croquettes}{クロケット}10個。
\item
  小さなトマト10個は中をくり抜いて味付けをし、植物油少々で弱火で蒸し煮する。
\end{itemize}

\ul{ソース}\ldots{}\ldots{}薄く仕上げた\protect\hyperlink{sauce-tomate}{トマトソース}に、グリルして皮を剥き、細かい千切りにしたポワヴロン\footnote{いわゆる青果としてのパプリカ。}を加えたもの。

\hypertarget{garniture-alsacienne}{%
\subsubsection{ガルニチュール・アルザス風}\label{garniture-alsacienne}}

\frsub{Garniture à l'Alsacienne}

\index{garniture@garniture!alsacienne@--- à l'Alsacienne}
\index{alsacien@alsacien(ne)!garuniture à l'---ne}
\index{かるにちゆーる@ガルニチュール!あるさすふう@---・アルザス風}
\index{あるさすふう@アルザス風!かるにちゆーる@ガルニチュール・---}

(牛、羊の塊肉、牛フィレ、トゥルヌドに添える)

\begin{itemize}
\tightlist
\item
  ブレゼ\footnote{\protect\hyperlink{chou-braise}{キャベツのブレゼ}を参考にすること。}したシュークルート\footnote{専用品種の生食出来ないくらい固くて大きなキャベツを千切りにして香
    辛料とともに塩蔵、醗酵さたもの。ドイツのザワークラウトが原型だが、
    フランスとドイツで領土の取り合いとなったアルザス地方で独自に発展し
    た。温めたシュークルートにソーセージなどの豚肉加工品を添えた
    choucoûte barnie(シュークルートガルニ)はアルザスの名物料理のひと
    つ。なおシュークルート用の品種はQuintal d'Alsace(カンタルダルザ
    ス)が最良とされている。また、日本でも北海道で栽培され、鰊の漬物な
    どに使われる札幌大球甘藍という品種はこの系統が先祖らしい。。}を詰めてハムの脂身のないところを円く切ってのせたタルトレット10個。
\end{itemize}

\ul{ソース}\ldots{}\ldots{}\protect\hyperlink{jus-de-veau-lie}{とろみを付けた仔牛のジュ}。

\hypertarget{garniture-americaine}{%
\subsubsection[ガルニチュール・アメリケーヌ]{\texorpdfstring{ガルニチュール・アメリケーヌ\footnote{\protect\hyperlink{sauce-americaine}{ソース・アメリケーヌ}も参照されたい。}}{ガルニチュール・アメリケーヌ}}\label{garniture-americaine}}

\frsub{Garniture à l'Américaine}

\index{garniture@garniture!americaine@--- à l'Américaine}
\index{americain@américain(e)!garuniture à l'---e}
\index{かるにちゆーる@ガルニチュール!あめりけーぬ@---・アメリケーヌ}
\index{あめりかん@アメリカン/アメリケーヌ!かるにちゆーる@ガルニチュール・アメリケーヌ}

(魚料理に添える)

\begin{itemize}
\tightlist
\item
  このガルニチュールは必ず、\protect\hyperlink{homard-americaine}{オマール・アメリケー
  ヌ}の方法で調理した尾の身をやや斜めに1 cm程度の 薄切り\footnote{escalope
    (エスカロップ)肉などを筋線維と直角に、丸くスライスしたもの。}にして供する。
\end{itemize}

\ul{ソース}\ldots{}\ldots{}オマール・アメリケーヌのソース。

\hypertarget{garniture-andalouse}{%
\subsubsection[ガルニチュール・アンダルシア風]{\texorpdfstring{ガルニチュール・アンダルシア風\footnote{アンダルシア風、つまりスペイン風といいながら、ギリシャ風ライスを
  使うという点からも、料理名に付けられた地名がしばしば不確かで大雑把
  な理由さえないことが多いことが理解されよう。}}{ガルニチュール・アンダルシア風}}\label{garniture-andalouse}}

\frsub{Garniture à l'Andalouse}

\index{garniture@garniture!andalouse@--- à l'Andalouse}
\index{andalou@andalou(se)!garuniture à l'---se}
\index{かるにちゆーる@ガルニチュール!あんたるしあふう@---・アンダルシア風}
\index{あんたるしあふう@アンダルシア風!かるにちゆーる@ガルニチュール・---}

(牛、羊の塊肉料理や鶏料理に添える)

\begin{itemize}
\item
  中位の大きさのポワヴロン10個をグリル焼きして中をくり抜き、\protect\hyperlink{riz-grecque}{ギリシャ風ライス}を詰める。
\item
  なす\footnote{フランスで伝統的なタイプのなすはヘタが緑色で、風味や調理特性はい
    わゆる米なすに近いが、形状は比較的細長い。直径4〜6 cm、長さ25 cmく
    らいのものが多い。}を4
  cmの厚さの輪切りにして面取りをし、中に窪みをつくって油で揚げ、提供直前に油で炒めたトマトをのせる。
\end{itemize}

\ul{ソース}\ldots{}\ldots{}\protect\hyperlink{jus-de-veau-lie}{とろみを付けたジュ}。

\hypertarget{garniture-arlesienne}{%
\subsubsection[ガルニチュール・アルル風]{\texorpdfstring{ガルニチュール・アルル風\footnote{南フランスの都市
  Arles (アルル)の形容詞および名詞形。名詞の場
  合は「アルルの人」の意味になる。アルルはオランダ出身の印象派〜ポス
  ト印象派の画家フィンセント・ファン・ゴッホ Vincent van Gogh (フラ
  ンス語では昔からヴァンソンヴァンゴーグと呼ぶ習慣が付いてしまってお
  り、現代フランス語の原語発音尊重の風潮にもかかわらず、そのように発
  音されることは多いようだ)が1888年から1889年までアトリエを構え、
  「ひまわり」など多くの傑作を描いた。有名な、自分の耳を切り落すとい
  う「事件」を起こしたのもアルルでのことだ。この時期の作品のひとつに、
  「アルルの女(ジヌー夫人)」と呼ばれる一連のものがある。モデルはア
  ルルのカフェの経営者だといわれている。もっとも、フランスにおいて画
  家としてのゴッホおよび彼の作品は生前はほとんど評価されることがなく、
  生前に売れた絵は1枚だけだったとさえいわれている。このレシピは初版
  つまり1903年から収められているため、ゴッホの絵との関連はほぼないと
  考えていいだろう。むしろ、小説化アルフォンス・ドーデ原作を戯曲化し
  てジョルジュ・ビゼーが劇音楽を付けた『アルルの女』(1872年初演、
  1878年再演)との関連があると見るのがいいだろう。この作品は初演時点
  ではあまり好評ではなかったが、再演で大ヒットとなった。\protect\hyperlink{sauce-bohemienne}{ソース・ボ
  ヘミアの娘}のように、人気のある劇やオペラのタイ
  トルを料理名につけて、その人気にあやかろうという風潮が19世紀後半に
  は比較的多かった。そのため、トマトとなすという南フランスを思わせる
  食材を使ってはいてもアルルという土地に何の関係もないと思われる、内
  容的にも凡庸なこのガルニチュールに、当時の人気作品の名をつけて、い
  かにも流行のものであるかのように供したのが定着した、と考えることも
  可能だろう。その場合は「\ul{ガルニチュール・アルルの\\女}」と訳す
  べきかも知れない。なお、ビゼーが最初に作曲したのは27曲からなる舞台
  音楽であって、独立した音楽作品でもなければ、オペラでもなかったが、
  そのなかから数曲を選んで編曲し(あるいは作曲しなおし)、『アルルの
  女 組曲』としてこんにち広く知られている。第1組曲と第2組曲があり、
  前者はビゼー自身によるオーケストラ用編曲。後者はビゼーの死後1879年
  に友人エルネスト・ギローが完成させた。第1組曲の「メヌエット」や第2
  組曲の「ファランドール」など、曲名は知らずとも、メロディーを聴いた
  ことのある読者も少なくないとと思われる。}}{ガルニチュール・アルル風}}\label{garniture-arlesienne}}

\frsub{Garniture à l'Arlésienne}

\index{garniture@garniture!arlesienne@--- à l'Arlésienne}
\index{arlesien@arlésien(ne)!garuniture à l'---ne}
\index{かるにちゆーる@ガルニチュール!あるるふう@---・アルル風}
\index{あるるふう@アルル風!かるにちゆーる@ガルニチュール・---}

(トゥルヌドやノワゼットの料理に添える)

\begin{itemize}
\item
  なすは1
  cm程の厚さにスライスして塩こしょうをし、小麦粉をまぶして油で揚げる。
\item
  トマト皮を剥いてスライスし、バターでソテーする。
\item
  玉ねぎは輪切りにして指輪のようにばらばらにし、小麦粉をまぶして油で揚げ、花束のように盛る。
\end{itemize}

\ul{ソース}\ldots{}\ldots{}トマト風味にした\protect\hyperlink{sauce-demi-glace}{ソース・デミグラス}。
\end{recette}