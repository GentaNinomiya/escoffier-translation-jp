\hypertarget{serie-des-appareiles-et-preparations-diverses-pour-garnitures-froides}{%
\section[冷製ガルニチュール用アパレイユなど]{\texorpdfstring{冷製ガルニチュール用アパレイユなど\footnote{この節は、初版で「冷製料理」の章の冒頭に概説としてまとめられて
  いたものを、第二版の改訂時に、ほぼそのままの内容で現在の位置に移動
  させられている。もちろん順序および内容の加筆も行なわれており、異同
  は少なくない。}}{冷製ガルニチュール用アパレイユなど}}\label{serie-des-appareiles-et-preparations-diverses-pour-garnitures-froides}}

\frsec{Série des Appareils et Préparations diverses pour Garnitures froides}

\index{garniture@garniture!appareils garnitures froides@appareils et préparations diverses pour garnitures froides}
\index{appareil@appareil!garnitures froides@--- et préparations diverses pour garnitures froides}
\index{かるにちゆーる@ガルニチュール!あはれいゆれいせい@冷製ガルニチュールのためのアパレイユなど}
\index{あはれいゆ@アパレイユ!れいせいかるにちゆーる@冷製ガルニチュールのための---など}

\hypertarget{mousses-mousselines-et-souffles-froids}{%
\subsection{冷製のムース、ムスリーヌ、スフレ}\label{mousses-mousselines-et-souffles-froids}}

\frsecb{Mousse, Moussseline, et Soufflé froids}

\index{mousse@mousse!froide@--- froide}
\index{mousseline@mousseline!froide@--- froide}
\index{souffle@soufflé!froid@--- froid}
\index{むーす@ムース!れいせい@冷製の---}
\index{むすりーぬ@ムスリーヌ!れいせい@冷製の---}
\index{すふれ@スフレ!れいせい@冷製の---}

温製の場合でも冷製の場合でも、\ul{ムースとムスリーヌはどちらも同じ材料から作られる}。

ムースとムスリーヌの違いは、温製でも冷製でも、通常は10人分が入る大きな
型に詰めて作るのが\ul{ムース}と呼ばれ、いっぽう、\ul{ムスリーヌ}はスプー
ンで整形したり絞り袋を使ったり、あるいは大きなクネルの形をした専用の型
に入れたりして作るが、基本的に\ul{1つ}で1人分と決まっている。スフレは
小さなスフレ型に詰める。
\begin{recette}
\hypertarget{composition-de-l-appareil-pour-mousses-et-mousseline-froides}{%
\subsubsection{冷製のムースとムスリーヌのアパレイユ}\label{composition-de-l-appareil-pour-mousses-et-mousseline-froides}}

\frsub{Composition de l'Appareil pour Mousses et Mousseline froides}

\index{garniture@garniture!appareils garnitures froides@appareils et préparations diverses pour garnitures froides!appareil mousses mousselines froides@composition de l'appareil pour mousses et mousselines froides}
\index{appareil@appareil!garnitures froides@--- et préparations diverses pour garnitures froides!appareil mousses mousselines froides@composition de l'appareil pour mousses et mousselines froides}
\index{mousse@mousse!froide@froide!composition appareil@Composition de l'appareil pour mousses et mousseline froides}
\index{mousseline@mousseline!froide@froide!composition appareil@Composition de l'appareil pour mousses et mousseline froides}
\index{かるにちゆーる@ガルニチュール!あはれいゆれいせい@冷製ガルニチュールのためのアパレイユなど!れいせいのむーすとむすりーぬのあぱれいゆ@冷製のムースとムスリーヌのアパレイユ}
\index{あはれいゆ@アパレイユ!れいせいかるにちゆーる@冷製ガルニチュールのための---など!れいせいのむーすとむすりーぬのあぱれいゆ@冷製のムースとムスリーヌのアパレイユ}
\index{むーす@ムース!れいせい@冷製!むーすとむすりーぬのあぱれいゆ@ムースとムスリーヌのアパレイユ}
\index{むすりーぬ@ムスリーヌ!れいせい@冷製!むーすとむすりーぬのあぱれいゆ@ムースとムスリーヌのアパレイユ}

\begin{itemize}
\tightlist
\item
  \textbf{材料}\ldots{}\ldots{}主素材のピュレ\footnote{本書では加熱した肉や魚、甲殻類のピュレを作る方法への言及はないが、
    \textbf{本章冒頭にある\protect\hyperlink{farce-mousseline}{ファルス・ムスリーヌ}をそのま
    ま使おうなどと考えてはいけない。ここで説明されている冷製のムース、
    ムスリーヌ、スフレの作り方に加熱の工程がまったく含まれていないのは、
    主素材のピュレが既に加熱済みであることを当然の前提としている}から
    だ。つまりここで材料として示されているピュレは\textbf{すべて加熱済みのも
    のをピュレにしたものだ}と考えなければならない。『料理の手引き』の
    当時はローストするか茹でるなどの加熱後に、鉢に入れてすり潰し、裏漉
    ししてから何らかのソース(ここではヴルテ)を加えて漉さ(固さ)を調
    節するなどしていた。現代ではフードプロセッサーや冷凍粉砕調理機など
    を利用すればより容易に滑らかなピュレを作ることが可能だろう。また、
    第3章ポタージュに\protect\hyperlink{les-purees}{ポタージュ・ピュレ}についての概説が
    あるが、そこではポタージュにすることを前提として「つなぎ」の使用が
    作業のプロセスに組込まれて説明されているために、あくまで参考程度に
    読むのがいいだろう。}1 Lすなわち鶏のピュレ、ジビエ、フォワグラ
  や魚、甲殻類のピュレ。溶かした\protect\hyperlink{gelees-ordinaires}{ジュレ}2\undemi{}
  dl、\protect\hyperlink{veloute}{ヴルテ}4 dl、生クリーム4
  dlはちょうどいい固さに立てて6 dl相当にしておく。
\end{itemize}

素材の特性によって、これらの分量比率は多少変更してもいい。同様に、ある
種のムースを作る際にはジュレまたはヴルテのどちらかしか用いなくてもいい。

\begin{itemize}
\tightlist
\item
  \textbf{作業手順}\ldots{}\ldots{}まずベースとなるピュレを入れたボウルを氷の上に置いて、軽
  く混ぜながら、ジュレとヴルテを加える(どちらかしか使わない場合は使う
  もののみ)。次に泡立てた生クリームを加える。
\end{itemize}

味付けを確認する。これは冷製料理ではとても重要なことだ。いつも気をつけ
て確認し、修正を加えるようにすること。

\hypertarget{nota-composition-de-l-appareil-pour-mousses-et-mousseline-froides}{%
\subparagraph{【原注】}\label{nota-composition-de-l-appareil-pour-mousses-et-mousseline-froides}}

生クリームは五分立てにしておくこと。完全に立ててしまっていたら、ムース
は滑らかさが失なわれてパサついた仕上りになってしまう。

\hypertarget{moulage-des-mousses-froides}{%
\subsubsection{冷製ムースの型詰め}\label{moulage-des-mousses-froides}}

\frsub{Moulage des Mousses froides}

\index{garniture@garniture!appareils garnitures froides@appareils et préparations diverses pour garnitures froides!moulage mousses froides@moulage des mousses froides}
\index{appareil@appareil!garnitures froides@--- et préparations diverses pour garnitures froides!moulage mousses froides@moulage des mousses froides}
\index{mousse@mousse!froide@froide!moulage@moulage des mousses froides}
\index{かるにちゆーる@ガルニチュール!あはれいゆれいせい@冷製ガルニチュールのためのアパレイユなど!れいせいむーすのかたつめ@冷製ムースの型詰め}
\index{あはれいゆ@アパレイユ!れいせいかるにちゆーる@冷製ガルニチュールのための---など!れいせいむーすのかたつめ@冷製ムースの型詰め}
\index{むーす@ムース!れいせい@冷製!むーすのかたつめ@ムースの型詰め}

いまもそうしている料理人は少なくないようだが、かつては、プレーンな型あ
るいは浮き彫り模様の付いた型の中に透明なジュレを流して層をつくってやり\footnote{chemiser
  (シュミゼ)ジュレなどを型の内側に流して薄い層を作ること。}、
ムースの主素材と関連あるものを装飾要素として貼り付けていた。

こんにちでは次の方法がむしろ好ましい。銀製のタンバル型\footnote{timbale
  (タンバル)円筒形の比較的浅い型。野菜料理用の深皿もこの語で呼ぶので注意。}の底面だけに
透明なジュレの薄い層をつくる。型の側面の外側に紙の帯を冷たいバターで貼
り付ける。型の\ruby{縁}{ふち}から2〜3 cmくらい高くなるようにすること。
そうするとスフレのような見た目のムースになる。紙の帯は型の内側に貼り付
けてもいい。この紙の帯は提供直前に、ぬるま湯で濡らしてナイフの刃を使っ
てムースからそっと引き剥してやる。

タンバル型の用意が整ったら、ムースを詰めて冷やす。アイスクリーム用の冷
凍庫に入れるほうがいいだろう。この方法は、小さな銀製のスフレ型に詰めて
やってもいいが、それは冷製のスフレにとっておいたほうがいいだろう。アパ
レイユの構成が同じであるにもかかわらず、冷製ムースと冷製スフレの違いを
はっきりさせることが出来るからだ。

とりわけジビエのムースやフォワグラのムースについては、近代的な料理の提
供方法に合わせて作られた銀製かガラス製の容器を用いてもいい。その場合は、
型の底面だけジュレの層をつくってやり、アパレイユをそのまま流し込めばい
い。表面はパレットナイフなどで丁寧に滑らかにならしてやってから、ムース
を冷やす。その後\footnote{型から出して、ということだろう。}、ムースに直接装飾を施し、ジュレをかけて艶を出させる。

ジビエのムースの場合には、そのジビエの胸肉を冷やして、ムースの周囲に飾
るようにする。

\hypertarget{moulage-des-mousselines-froides}{%
\subsubsection{冷製ムスリーヌの整形}\label{moulage-des-mousselines-froides}}

\frsub{Moulage des Mousselines froides}

\index{garniture@garniture!appareils garnitures froides@appareils et préparations diverses pour garnitures froides!moulage mousselines froides@moulage des mousselines froides}
\index{appareil@appareil!garnitures froides@--- et préparations diverses pour garnitures froides!moulage mousselines froides@moulage des mousselines froides}
\index{mousseline@mousseline!froide@froide!moulage@moulage des mousselines froides}
\index{かるにちゆーる@ガルニチュール!あはれいゆれいせい@冷製ガルニチュールのためのアパレイユなど!れいせいむすりーぬのかたつめ@冷製ムスリーヌの型詰め}
\index{あはれいゆ@アパレイユ!れいせいかるにちゆーる@冷製ガルニチュールのための---など!れいせいむすりーぬのかたつめ@冷製ムスリーヌの型詰め}
\index{むすりーぬ@ムスリーヌ!れいせい@冷製!むすりーぬのかたつめ@ムスリーヌの型詰め}

冷製ムスリーヌの型詰めには2つの方法がある。たんに、型にジュレの層を作っ
てやるか、ソース・ショフロワの層を作ってやるかの違いでしかない。どちら
の場合でも、卵形の型に詰めるか、大きなクネルの形状のものにするか、とい
うことになる。

\hypertarget{procede-un-moulage-des-mousselines-froides}{%
\subparagraph{方法1\ldots{}\ldots{}}\label{procede-un-moulage-des-mousselines-froides}}

型の内側に透明なジュレを流して薄い層を作ってやる\footnote{chemiser
  (シュミゼ)。}。その上にアパレイ
ユを張るように塗り、アパレイユのベースとなっている素材とおなじもの---
鶏、ジビエ、甲殻類の身など、とトリュフ---で構成された\protect\hyperlink{salpicons-divers}{サルピコ
ン}を盛り込む。その上からアパレイユを塗って覆い、パ
レットナイフなどを使ってドーム形に滑らかにならす。冷蔵庫に入れて冷し固
める。

\hypertarget{procede-deux-moulage-des-mousselines-froides}{%
\subparagraph{方法2\ldots{}\ldots{}}\label{procede-deux-moulage-des-mousselines-froides}}

型の内側にアパレイユを詰め、さらにサルピコンをその内側に射込む。アパレ
イユで覆って、冷し固める。

型から外す。ムスリーヌのアパレイユの素材と関連性のある\protect\hyperlink{sauce-chaud-froid-ordinaire}{ソース・ショフ
ロワ}を表面を覆うように塗る\footnote{napper
  (ナペ)。覆いかける(ように塗る)こと。}。トリュフおよびそ
の他の素材(これもムスリーヌと関連性があること)を装飾用に細工したもの
を飾り付ける。装飾が剥れないように、上からジュレを塗って艶を出させる。

銀製またはガラス製の深皿の底に透明なジュレの層を作り、その上にムスリー
ヌを並べる。再度ジュレを上からかけてやり、冷蔵庫に入れて提供するまで保
管しておく。

\hypertarget{souffles-froids}{%
\subsubsection{冷製スフレ}\label{souffles-froids}}

\frsub{Soufflés froids}

\index{garniture@garniture!appareils garnitures froides@appareils et préparations diverses pour garnitures froides!souffles froids@soufflés froids}
\index{appareil@appareil!garnitures froides@--- et préparations diverses pour garnitures froides!souffles froids@soufflés froids}
\index{souffle@soufflé!froid@---s froids}
\index{かるにちゆーる@ガルニチュール!あはれいゆれいせい@冷製ガルニチュールのためのアパレイユなど!れいせいすふれ@冷製スフレ}
\index{あはれいゆ@アパレイユ!れいせいかるにちゆーる@冷製ガルニチュールのための---など!れいせいすふれ@冷製スフレ}
\index{すふれ@スフレ!れいせい@冷製---}

冷製スフレはムースそのものに他ならない。だから構成はまったく同じだ。た
だ、先に見たようにスフレが10人分\footnote{1 service
  (アンセルヴィス)、格式のある宴席料理などを作る際の単位。基本は10人分。}を確保できるだけの大きな型に詰める
のに対して、スフレはそもそも、小さなスフレ型に入れてひとり1つ宛で作る
ものだ。

アパレイユを型に詰める方法はムースの場合と同様、つまり、スフレ型の底に
ジュレの層を敷いてその上にアパレイユを盛り、型の縁より高くなるように周
囲に巻いた紙の帯を利用して縁より高くアパレイユを盛る。そうすると、冷や
し固めた後で紙の帯を取り除けば、まるで温製のスフレのように見えることに
なる。

\hypertarget{nota-souffles-froids}{%
\subparagraph{【原注】}\label{nota-souffles-froids}}

ここまで述べた3種の作り方の基礎はおなじだから、ポイントは次のようにまとめられる。

\begin{enumerate}
\def\labelenumi{\arabic{enumi}.}
\item
  ムースは「スフレ」の名称で供してもいいものだが、混同されるのを避けるために「ムース」の名称で約10人分をひとつの型に入れて作る。
\item
  ムスリーヌはサルピコンを射込んだものであってもそうでなくても、大きなクネルであって、ひとりあたり1つにする。
\item
  スフレは小さなムースであって、スフレ型あるいは似たような型に詰めて、これもひとりあたり1つとする。
\end{enumerate}
\end{recette}
\hypertarget{aspics}{%
\subsection{アスピック}\label{aspics}}

\frsecb{Aspics}

\index{garniture@garniture!appareils garnitures froides@appareils et préparations diverses pour garnitures froides!aspics@aspics}
\index{appareil@appareil!garnitures froides@--- et préparations diverses pour garnitures froides!aspics@aspics}
\index{aspics@aspics (généralité)}
\index{かるにちゆーる@ガルニチュール!あはれいゆれいせい@冷製ガルニチュールのためのアパレイユなど!あすぴつく@アスピック(概説)}
\index{あはれいゆ@アパレイユ!れいせいかるにちゆーる@冷製ガルニチュールのための---など!あすぴつく@アスピック(概説)}
\index{あすぴつく@アスピック(概説)}

アスピックを作る際に、肝に銘じておくべき第一のポイントは、どんなアスピッ
クでも、ジュレがジューシー\footnote{原文succulent(スュキュロン)はsuc(スュック=肉汁)から派生した
  形容詞で、もともとは「汁気の多い」の意味だったが、そこから転じて
  「美味な、滋味に富んだ」の意味で一般的に用いられている。ここでは、
  両方のニュアンスで表現されていると解釈できる。}で美味しく、完全に透き通ったもので、ちょうど
いい加減に固まっていなければならないことである。

アスピックを作る際には、昔もそうだったが現代でも、中央に穴の空いたアス
ピック型\footnote{moule à douille
  (ムーラドゥイユ)サヴァラン型のような中央に穴 が空いた型。
  現代では「アスピック型」というと楕円形で中央に穴のな
  いものを指すことが多いが、それとは異なる。クグロフ型のようなものを
  イメージするとわかりやすいだろう。19世紀、アスピックには高さのある
  型が多く用いられたようである。なお、現代では一般にサヴァラン型とい
  うと、型の高さや穴の大きさ等さまざまなタイプのものをまとめて指すこ
  とになるので注意。高さのない(低い)、中央の穴が大きな型について、
  エスコフィエはボルデュール型 moule à bordure (ムーラボルデュール)
  と呼んで区別している}でプレーンなもの、波模様等の装飾のあるものが用いられている。

ボルデュール型\footnote{moule à bordure
  (ムーラボルデュール)蛇の目形に料理の縁り飾り
  を作るための、やや丈が低く中央の穴が大きいリング型。}も使われることがあるが、一般的に、アスピックの中心にガル
ニチュールを盛り込む場合のみである。

アスピックを型に入れる時には、まず、型の底と周囲に装飾をする。

そのために、型は砕いた氷の中に入れてよく冷やしておく。やや固まりかけた
ジュレ少量を流し入れ、型を氷の上で転がしながらジュレを周囲に貼り付かせ
る(シュミゼ)。次に、装飾するパーツを、固まらない程度に冷たいジュレに浸
してからすぐに貼り付ける。装飾については料理人のセンスとアイデア次第な
ので、ここで明確に述べておくべきことはほとんどない。ひとつだけ言えるの
は、常に正確な作業を期して、型からアスピックを出したときに装飾がはっき
りと見えるようにすべき、ということだけだ。

装飾に用いる素材はアスピックの主素材と関連性のあるものでなくてはならな
い。一般的には、トリュフ、ポシェした卵白、コルニション\footnote{cornichon
  主としてピクルスにする小型のきゅうり、およびそのピク
  ルスのこと。日本では、ハンバーガーによく用いられているドイツ系のピ
  クルス用品種であるガーキンス(英 gherkins 独 Einlegegurken)と混同さ
  れることがあるが、コルニションはより小さなサイズで収穫し、フレッシュ
  な状態では「いぼ」が尖っているのが特徴。}、ケイパー、
いろいろな香草の葉先、ラディッシュの薄い輪切り、オマールのコライユ\footnote{胴の背側にあるオレンジ色がかった「内子」。}、
\protect\hyperlink{saumure-liquide-pour-langues}{赤く漬けた舌肉}、等。

アスピックのガルニチュールが種々のエスカロップ\footnote{escalope
  (エスカロップ)筋線維とは垂直方向に、厚さ1〜2 cmに薄
  切りにした仔牛などの肉や魚の薄い切り身。}や長方形に切ったフォワグ
ラ等で、型の大きさから何度も並べなければならない場合、ジュレの層と交互
に重ねて型に入れていく。新しい層を並べる際には先に入れたジュレがある程
度固まってからにする。

アスピックの型入れでは常に、最後のジュレの層を充分な厚みにする。できる
だけ、型を氷に埋めるようにしながらジュレを流し込んでいくが、早く冷やす
ために氷に塩を加えてはいけない。塩を使うとジュレの透明さが損なわれるか
らである。

\noindent\textbf{型から外す方法}\ldots{}\ldots{}型を湯につけてただちに水気を拭い、折り
ただんだナフキンや彫刻した氷のブロック等に、アスピックを裏返してあける。

菱形や正方形に切ったジュレのクルトン\footnote{パンで作るクルトンと同様に、菱形やさいの目に切った冷製料理装飾用
  のジュレもクルトンと呼ぶ。}、またはアシェしたジュレで周囲を飾 る。

\hypertarget{nota-aspics}{%
\subparagraph{【原注】}\label{nota-aspics}}

アスピックを型に入れて作るには、必然的に、ジュレが相当に固いものでなけ
ればならないが、これはまことに具合がよろしくない。というのも、固いジュ
レは口あたりがよくないからだ。だから現代の調理現場では、以下のような方
法を採っている。タンバル型か、氷に嵌め込むようにした銀やガラスあるいは
陶製の深皿の底に予めジュレの層を作って固めておき、その上にアスピックの
素材を並べる。次に、固まりかけのジュレをたっぷり覆いかける。この方法で
は、装飾をしなければならない場合は、アスピックの調理をおこなう前に、主
素材にじかに装飾することになる。

\hypertarget{chauds-froids}{%
\subsection[ショフロワ]{\texorpdfstring{ショフロワ\footnote{chaud-froid
  このchaud(熱い)とfroid(冷たい)の合成語の複数形は、それぞれ
  にsを付ける、chauds-froidsとなる。合成語の複数形はいろいろなパターンがあるので、必要が
  出たらその都度覚えるようにしたほうがいい。}}{ショフロワ}}\label{chauds-froids}}

\frsecb{Chauds-froids}

\index{garniture@garniture!appareils garnitures froides@appareils et préparations diverses pour garnitures froides!chauds-froids@chauds-froids}
\index{appareil@appareil!garnitures froides@--- et préparations diverses pour garnitures froides!chauds-froids@chauds-froids}
\index{chauds-froids@chauds-froids (généralité)}
\index{かるにちゆーる@ガルニチュール!あはれいゆれいせい@冷製ガルニチュールのためのアパレイユなど!しよふろわ@ショフロワ(概説)}
\index{あはれいゆ@アパレイユ!れいせいかるにちゆーる@冷製ガルニチュールのための---など!しよふろわ@ショフロワ(概説)}
\index{しよふろわ@ショフロワ(概説)}

\protect\hyperlink{sauce-chaud-froid-ordinaire}{ソース・ショフロワ}には大抵の場合、切り
分けた素材を浸す。が、時として大きな塊肉全体をソース・ショフロワで覆わ
なくてはならない場合もある。ただ、そういう仕立てにする場合には、別の料
理名となっている。

ショフロワが複数のばらばらのパーツからなる場合には、それらをソース・ショ
フロワに漬けたら網の上に並べておく。ソースが冷えたら、それぞれのパーツ
に装飾をし、ジュレを覆いかけて艶を出してやる。さらに盛り付けの際にはみ
出す余分なソースについてはきれいに取り除いておくこと。

大きな塊肉の場合は、よく冷えてはいるけれどまだ流動性のある状態のソース・
ショフロワを一気に塗りつけて、その後に装飾をし、ジュレを塗って艶出しす
ること。

切り分けた素材からなるショフロワの盛り付けは、\protect\hyperlink{fonds-de-plats}{皿の上の
台}の上に盛り付けてもいいし、縁飾りの内側に、パンまた
は米、セモリナ粉で作った台を置いてその上に盛り付けてもいい。あるいは、
銀製か陶製、ガラス製の深皿に盛り付けてもいい。

大きな塊肉のショフロワの場合、皿の上の台にのせてもいいし、あるいは、氷
のブロックに料理が嵌まるようにブロックを削ってからそこに盛り付けるのも
いい。

ショフロワ仕立ての鶏やジビエについては、正確に切り分けて\footnote{基本的に鶏および鳥類のジビエの可食部は胸肉のみとされていたこと
  に留意。}皮は剥いでおく
こと。手羽や下腿肉は使わないので、別の用途に取り置いておくといい。

細かく切った素材のショフロワ仕立ての場合、添えてやるマッシュルームや雄
鶏のとさかとロニョン\footnote{rognon
  (ロニョン)牛、羊などの場合は腎臓だが、雄鶏の場合は精巣
  のこと。高級食材として珍重された。}にもソース・ショフロワを塗ってやること。トリュ
フはただジュレをかけて艶を出すだけでいい。
