\hypertarget{serie-des-appareiles-et-preparations-diverses-pour-garnitures-froides}{%
\section{冷製ガルニチュール用アパレイユなど}\label{serie-des-appareiles-et-preparations-diverses-pour-garnitures-froides}}

\frsec{Série des Appareils et Préparations diverses pour Garnitures froides}

\index{garniture@garniture!appareils garnitures froides@appareils et préparations diverses pour garnitures froides}
\index{appareil@appareil!garnitures froides@--- et préparations diverses pour garnitures froides}
\index{かるにちゆーる@ガルニチュール!あはれいゆれいせい@冷製ガルニチュールのためのアパレイユなど}
\index{あはれいゆ@アパレイユ!れいせいかるにちゆーる@冷製ガルニチュールのための---など}

\hypertarget{mousses-mousselines-et-souffles-froids}{%
\subsection{冷製のムース、ムスリーヌ、スフレ}\label{mousses-mousselines-et-souffles-froids}}

\frsecb{Mousse, Moussseline, et Soufflé froids}

\index{mousse@mousse!froide@--- froide}
\index{mousseline@mousseline!froide@--- froide}
\index{souffle@soufflé!froid@--- froid}
\index{むーす@ムース!れいせい@冷製の---}
\index{むすりーぬ@ムスリーヌ!れいせい@冷製の---}
\index{すふれ@スフレ!れいせい@冷製の---}

温製の場合でも冷製の場合でも、\ul{ムースとムスリーヌはどちらも同じ材料から作られる}。

ムースとムスリーヌの違いは、温製でも冷製でも、通常は10人分が入る大きな
型に詰めて作るのが\ul{ムース}と呼ばれ、いっぽう、\ul{ムスリーヌ}はスプー
ンで整形したり絞り袋を使ったり、あるいは大きなクネルの形をした専用の型
に入れたりして作るが、基本的に\ul{1つ}で1人分と決まっている。スフレは
小さなスフレ型に詰める。
\begin{recette}
\hypertarget{composition-de-l-appareil-pour-mousses-et-mousseline-froides}{%
\subsubsection{冷製のムースとムスリーヌのアパレイユ}\label{composition-de-l-appareil-pour-mousses-et-mousseline-froides}}

\frsub{Composition de l'Appareil pour Mousses et Mousseline froides}

\index{garniture@garniture!appareils garnitures froides@appareils et préparations diverses pour garnitures froides!appareil mousses mousselines froides@composition de l'appareil pour mousses et mousselines froides}
\index{appareil@appareil!garnitures froides@--- et préparations diverses pour garnitures froides!appareil mousses mousselines froides@composition de l'appareil pour mousses et mousselines froides}
\index{mousse@mousse!froide@froide!composition appareil@Composition de l'appareil pour mousses et mousseline froides}
\index{mousseline@mousseline!froide@froide!composition appareil@Composition de l'appareil pour mousses et mousseline froides}
\index{かるにちゆーる@ガルニチュール!あはれいゆれいせい@冷製ガルニチュールのためのアパレイユなど!れいせいのむーすとむすりーぬのあぱれいゆ@冷製のムースとムスリーヌのアパレイユ}
\index{あはれいゆ@アパレイユ!れいせいかるにちゆーる@冷製ガルニチュールのための---など!れいせいのむーすとむすりーぬのあぱれいゆ@冷製のムースとムスリーヌのアパレイユ}
\index{むーす@ムース!れいせい@冷製!むーすとむすりーぬのあぱれいゆ@ムースとムスリーヌのアパレイユ}
\index{むすりーぬ@ムスリーヌ!れいせい@冷製!むーすとむすりーぬのあぱれいゆ@ムースとムスリーヌのアパレイユ}

\begin{itemize}
\tightlist
\item
  \textbf{材料}\ldots{}\ldots{}主素材のピュレ\footnote{本書では加熱した肉や魚、甲殻類のピュレを作る方法への言及はないが、
    \textbf{本章冒頭にある\protect\hyperlink{farce-mousseline}{ファルス・ムスリーヌ}をそのま
    ま使おうなどと考えてはいけない。ここで説明されている冷製のムース、
    ムスリーヌ、スフレの作り方に加熱の工程がまったく含まれていないのは、
    主素材のピュレが既に加熱済みであることを当然の前提としている}から
    だ。つまりここで材料として示されているピュレは\textbf{すべて加熱済みのも
    のをピュレにしたものだ}と考えなければならない。『料理の手引き』の
    当時はローストするか茹でるなどの加熱後に、鉢に入れてすり潰し、裏漉
    ししてから何らかのソース(ここではヴルテ)を加えて漉さ(固さ)を調
    節するなどしていた。現代ではフードプロセッサーや冷凍粉砕調理機など
    を利用すればより容易に滑らかなピュレを作ることが可能だろう。また、
    第3章ポタージュに\protect\hyperlink{les-purees}{ポタージュ・ピュレ}についての概説が
    あるが、そこではポタージュにすることを前提として「つなぎ」の使用が
    作業のプロセスに組込まれて説明されているために、あくまで参考程度に
    読むのがいいだろう。}1 Lすなわち鶏のピュレ、ジビエ、フォワグラ
  や魚、甲殻類のピュレ。溶かした\protect\hyperlink{gelees-ordinaires}{ジュレ}2\undemi{}
  dl、\protect\hyperlink{veloute}{ヴルテ}4 dl、生クリーム4
  dlはちょうどいい固さに立てて6 dl相当にしておく。
\end{itemize}

素材の特性によって、これらの分量比率は多少変更してもいい。同様に、ある
種のムースを作る際にはジュレまたはヴルテのどちらかしか用いなくてもいい。

\begin{itemize}
\tightlist
\item
  \textbf{作業手順}\ldots{}\ldots{}まずベースとなるピュレを入れたボウルを氷の上に置いて、軽
  く混ぜながら、ジュレとヴルテを加える(どちらかしか使わない場合は使う
  もののみ)。次に泡立てた生クリームを加える。
\end{itemize}

味付けを確認する。これは冷製料理ではとても重要なことだ。いつも気をつけ
て確認し、修正を加えるようにすること。

\hypertarget{nota-composition-de-l-appareil-pour-mousses-et-mousseline-froides}{%
\subparagraph{【原注】}\label{nota-composition-de-l-appareil-pour-mousses-et-mousseline-froides}}

生クリームは五分立てにしておくこと。完全に立ててしまっていたら、ムース
は滑らかさが失なわれてパサついた仕上りになってしまう。

\hypertarget{moulage-des-mousses-froides}{%
\subsubsection{冷製ムースの型詰め}\label{moulage-des-mousses-froides}}

\frsub{Moulage des Mousses froides}

\index{garniture@garniture!appareils garnitures froides@appareils et préparations diverses pour garnitures froides!moulage mousses froides@moulage des mousses froides}
\index{appareil@appareil!garnitures froides@--- et préparations diverses pour garnitures froides!moulage mousses froides@moulage des mousses froides}
\index{mousse@mousse!froide@froide!moulage@moulage des mousses froides}
\index{かるにちゆーる@ガルニチュール!あはれいゆれいせい@冷製ガルニチュールのためのアパレイユなど!れいせいむーすのかたつめ@冷製ムースの型詰め}
\index{あはれいゆ@アパレイユ!れいせいかるにちゆーる@冷製ガルニチュールのための---など!れいせいむーすのかたつめ@冷製ムースの型詰め}
\index{むーす@ムース!れいせい@冷製!むーすのかたつめ@ムースの型詰め}

いまもそうしている料理人は少なくないようだが、かつては、プレーンな型あ
るいは浮き彫り模様の付いた型の中に透明なジュレを流して層をつくってやり\footnote{chemiser
  (シュミゼ)ジュレなどを型の内側に流して薄い層を作ること。}、
ムースの主素材と関連あるものを装飾要素として貼り付けていた。

こんにちでは次の方法がむしろ好ましい。銀製のタンバル型\footnote{timbale
  (タンバル)円筒形の比較的浅い型。野菜料理用の深皿もこの語で呼ぶので注意。}の底面だけに
透明なジュレの薄い層をつくる。型の側面の外側に紙の帯を冷たいバターで貼
り付ける。型の\ruby{縁}{ふち}から2〜3 cmくらい高くなるようにすること。
そうするとスフレのような見た目のムースになる。紙の帯は型の内側に貼り付
けてもいい。この紙の帯は提供直前に、ぬるま湯で濡らしてナイフの刃を使っ
てムースからそっと引き剥してやる。

タンバル型の用意が整ったら、ムースを詰めて冷やす。アイスクリーム用の冷
凍庫に入れるほうがいいだろう。この方法は、小さな銀製のスフレ型に詰めて
やってもいいが、それは冷製のスフレにとっておいたほうがいいだろう。アパ
レイユの構成が同じであるにもかかわらず、冷製ムースと冷製スフレの違いを
はっきりさせることが出来るからだ。

とりわけジビエのムースやフォワグラのムースについては、近代的な料理の提
供方法に合わせて作られた銀製かガラス製の容器を用いてもいい。その場合は、
型の底面だけジュレの層をつくってやり、アパレイユをそのまま流し込めばい
い。表面はパレットナイフなどで丁寧に滑らかにならしてやってから、ムース
を冷やす。その後\footnote{型から出して、ということだろう。}、ムースに直接装飾を施し、ジュレをかけて艶を出させる。

ジビエのムースの場合には、そのジビエの胸肉を冷やして、ムースの周囲に飾
るようにする。

\hypertarget{moulage-des-mousselines-froides}{%
\subsubsection{冷製ムスリーヌの整形}\label{moulage-des-mousselines-froides}}

\frsub{Moulage des Mousselines froides}

\index{garniture@garniture!appareils garnitures froides@appareils et préparations diverses pour garnitures froides!moulage mousselines froides@moulage des mousselines froides}
\index{appareil@appareil!garnitures froides@--- et préparations diverses pour garnitures froides!moulage mousselines froides@moulage des mousselines froides}
\index{mousseline@mousseline!froide@froide!moulage@moulage des mousselines froides}
\index{かるにちゆーる@ガルニチュール!あはれいゆれいせい@冷製ガルニチュールのためのアパレイユなど!れいせいむすりーぬのかたつめ@冷製ムスリーヌの型詰め}
\index{あはれいゆ@アパレイユ!れいせいかるにちゆーる@冷製ガルニチュールのための---など!れいせいむすりーぬのかたつめ@冷製ムスリーヌの型詰め}
\index{むすりーぬ@ムスリーヌ!れいせい@冷製!むすりーぬのかたつめ@ムスリーヌの型詰め}

冷製ムスリーヌの型詰めには2つの方法がある。たんに、型にジュレの層を作っ
てやるか、ソース・ショフロワの層を作ってやるかの違いでしかない。どちらの場合でも、卵形の型に詰めるか、大きなクネルの形状のものにするか、ということになる。

\hypertarget{procede-un-moulage-des-mousselines-froides}{%
\subparagraph{方法1\ldots{}\ldots{}}\label{procede-un-moulage-des-mousselines-froides}}

型の内側に透明なジュレを流して薄い層を作ってやる\footnote{chemiser
  (シュミゼ)。}。その上にアパレイ
ユを張るように塗り、アパレイユのベースとなっている素材とおなじもの---
鶏、ジビエ、甲殻類の身など、とトリュフ---で構成された\protect\hyperlink{salpicons-divers}{サルピコ
ン}を盛り込む。その上からアパレイユを塗って覆い、パレットナイフなどを使ってドーム形に滑らかにならす。冷蔵庫に入れて冷し固める。

\hypertarget{procede-deux-moulage-des-mousselines-froides}{%
\subparagraph{方法2\ldots{}\ldots{}}\label{procede-deux-moulage-des-mousselines-froides}}

型の内側にアパレイユを詰め、さらにサルピコンをその内側に射込む。アパレイユで覆って、冷し固める。

型から外す。ムスリーヌのアパレイユの素材と関連性のある\protect\hyperlink{sauce-chaud-froid}{ソース・ショフ
ロワ}を表面を覆うように塗る\footnote{napper
  (ナペ)。覆いかける(ように塗る)こと。}。トリュフおよびそ
の他の素材(これもムスリーヌと関連性があること)を装飾用に細工したもの
を飾り付ける。装飾が剥れないように、上からジュレを塗って艶を出させる。

銀製またはガラス製の深皿の底に透明なジュレの層を作り、その上にムスリー
ヌを並べる。再度ジュレを上からかけてやり、冷蔵庫に入れて提供するまで保
管しておく。
\end{recette}