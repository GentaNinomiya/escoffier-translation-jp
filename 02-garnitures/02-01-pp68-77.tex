\hypertarget{ii.ux30acux30ebux30cbux30c1ux30e5ux30fcux30ebgarnitures}{%
\chapter{II. ガルニチュール Garnitures}\label{ii.ux30acux30ebux30cbux30c1ux30e5ux30fcux30ebgarnitures}}

\vspace*{1.7\zw}

料理においてガルニチュール\footnote{garniture
  一般的には「付け合せ」と訳すが、本書におけるガルニチュー
  ルはたんなる料理の「付け合わせ」にとどまらず、それ自体がひとつの料
  理として成立し得るものも多い。そのため、あえて片仮名でガルニチュー
  ルとした。なお、「付け合わせ」の意味で「ガルニ」または「ガロニ」な
  どというスラングを用いる調理現場もある。}は重要なものであり、現場に立つ料理人は
誰であってもガルニチュールの役割を過小評価してはいけない。どのような構
成のガルニチュールにするかは、添える料理の素材との関係性で決まる。いか
なる気まぐれや不自然なものは絶対に、ガルニチュールから追放すべきなのだ。

ガルニチュールの構成要素は、場合により、とりわけガルニチュールを添える
大きな塊肉の種類によって決まる。具体的には、野菜料理やパスタ、ファルス
でさまざまな形状に作ったクネル、あるいは雄鶏のとさかとロニョン\footnote{\protect\hyperlink{garniture-financiere}{ガルニチュール・フィナンシエール}やその
  バリエーションともいえる\protect\hyperlink{garniture-godard}{ガルニチュール・ゴダー
  ル}で必須の素材。ロニョンrognonは通常なら腎臓を
  意味するが、この場合のロニョンは rognon blanc ロニョンブラン(白い
  ロニョン)とも呼ばれるもので、雄鶏の精巣のこと。}、さ
まざまな種類の茸、オリーブとトリュフ、イカや貝および甲殻類、場合によっ
ては卵、小魚、牛や羊の副生物\footnote{正肉以外の部分。例えば内臓や骨髄など。ris
  de veau リドヴォー(仔牛胸腺肉)などもこれに含まれる。}。

その昔、ガルニチュールというのは、マトロットやコンポート、ブルゴーニュ
風料理などのように味付けのために用いた素材が結果として添えられたもので
あった。

ガルニチュールにする野菜は、どういう皿にするのかということによって役割
が決まり、それに従って切って形状を整え、調理する。とはいえ、野菜の調理
自体は、その野菜を「野菜料理」として調理する方法と同じだ。

パスタやイカ、貝類、甲殻類についても同様のことが言える。

この章では、それぞれのガルニチュールを構成する素材とその分量を示すに留
めるので、各素材の調理の仕方についてはその素材に対応する章を参照するこ
と。

\hypertarget{ux30d5ux30a1ux30ebux30b9}{%
\section{ファルス}\label{ux30d5ux30a1ux30ebux30b9}}

\hypertarget{suxe9rie-des-farces-diverses}{%
\subsection{Série des farces
diverses}\label{suxe9rie-des-farces-diverses}}

ガルニチュールの多くは、その構成要素にファルスあるいはファルスから作ら
れる「クネル」が含まれており、ファルスはまた、多くの大きな仕立ての料理
にも使われる。ここではまずファルスの材料および作り方を示し、使い途につ
いては後で述べることにする。

ファルスは大きく5種に分類される。

\begin{enumerate}
\def\labelenumi{\arabic{enumi}.}
\item
  仔牛肉と脂で作るもの。すなわち古典料理における\textbf{ゴディヴォー}。
\item
  基本となる材料はさまざまだが、「つなぎ」に主としてパナードを使うもの。
\item
  近代的な手法で、生クリームを用いてふんわり泡立てたファルス。ムース、ムスリーヌに用いる。
\item
  レバーをベースとした\textbf{グラタン}と呼ばれる特殊なファルス。いろいろな種類があるが、作り方は常に同じ。
\item
  \protect\hyperlink{}{ガランティーヌ}、\protect\hyperlink{}{パテ}、\protect\hyperlink{}{テリーヌ}などの冷製料理に主として用いるシンプルなファルス。
\end{enumerate}

\hypertarget{ux30d5ux30a1ux30ebux30b9ux7528ux306eux30d1ux30caux30fcux30c9}{%
\subsection{ファルス用のパナード}\label{ux30d5ux30a1ux30ebux30b9ux7528ux306eux30d1ux30caux30fcux30c9}}

\vspace*{-1.7\zw}

\hypertarget{les-panades-pour-farces}{%
\subsection{Les Panades pour Farces}\label{les-panades-pour-farces}}
