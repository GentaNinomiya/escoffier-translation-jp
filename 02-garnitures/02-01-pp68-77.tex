\hypertarget{garnitures}{%
\chapter{II. ガルニチュール Garnitures}\label{garnitures}}

\index{garnitures@garnitures} \index{かるにちゆーる@ガルニチュール}

料理においてガルニチュール\footnote{garniture
  一般的には「付け合せ」と訳すが、本書におけるガルニチュー
  ルはたんなる料理の「付け合わせ」にとどまらず、こんにちではそれ自体
  がひとつの料理として成立し得るものも多い。そのため、あえて片仮名で
  ガルニチュールとした。なお、「付け合わせ」の意味で「ガルニ」または
  「ガロニ」などというスラングを用いる調理現場もある。}は重要なものだから、料理人は決してガルニ
チュールの役割を軽視してはいけない。ガルニチュールの構成をどうするかは、
添える料理の主素材との関係性で決まる。気まぐれ的なものや不自然なもの
は絶対にいけない。

ガルニチュールの構成要素は、場合によりけりだが、もっぱらどんな種類の料
理に添えるかで決まる。具体的には、野菜料理やパスタ、ファルスでさ
まざまな形状に作ったクネル\footnote{quenelle
  仔牛肉や鶏肉、豚肉などと獣脂をすり潰して、しばしば「つ
  なぎ」として後述のパナードを加えて練り、スプーンなどを用いて整形し、
  沸騰しない程度の温度で茹でる{[}ポシェ{]}またはオーブンで焼いたもの。
  スプーンを2つ使ってラグビーボールに似た形状にしたものが代表的だが、
  他にもいろいろな形状、大きさにする。}、あるいは雄鶏のとさかとロニョン\footnote{\protect\hyperlink{garniture-financiere}{ガルニチュール・フィナンシエール}やその
  バリエーションともいえる\protect\hyperlink{garniture-godard}{ガルニチュール・ゴダー
  ル}で必須の素材。ロニョンrognonは通常なら腎臓を
  意味するが、この場合のロニョンは rognon blanc ロニョンブラン(白い
  ロニョン)とも呼ばれるもので、雄鶏の精巣のこと。}、さ
まざまな種類の茸、オリーブとトリュフ、イカや貝および甲殻類、場合によっ
ては卵、小魚、牛や羊の副生物\footnote{正肉以外の部分。例えば内臓や骨髄など。Ris
  de vea(リドヴォー)仔牛胸腺肉などはこれに含まれる。}など。

その昔、ガルニチュールというのは、マトロットやコンポート、ブルゴーニュ
風料理などのように風味付けのために用いた素材がそのまま添えられたもので
あった。

ガルニチュールにする野菜は、どういう仕立ての皿にするかで役割が決まり、
それに合うように切って形状を整え、調理する。ただし、野菜の調理法は「野
菜料理」として調理する場合と同じだ。

パスタやイカ、貝類、甲殻類についても同様のことが言える。

この章では、それぞれのガルニチュールを構成する素材とその分量を示すに留
めるので、各素材の調理法ついてはその素材に対応する章を参照すること。

\hypertarget{serie-des-farces-diverses}{%
\section[ファルス]{\texorpdfstring{ファルス\footnote{本来は「詰め物」の意で、鶏のローストの内臓を抜いた空洞部分に詰めたり、ガランティーヌやパテアンクルートの内部の詰め物などの用途に用いられる。この意味はこんにちでも変化がないが、本文にあるように、クネルにしてガルニチュールの一部にするなど、用途は多岐にわたる。本書ではファルスとして用いられるもののうち、肉および魚肉をベースにしたものをこの節にまとめて分類、説明している。したがって、ここでファルスとして挙げられていないファルスも料理によっては多い(例えば丸鶏の空洞部分に米などを詰めるのもファルス)ことに注意。}}{ファルス}}\label{serie-des-farces-diverses}}

\frsec{Série des farces diverses}

\index{garnitures@garnitures!farces@farces} \index{farce@farce}
\index{かるにちゆーる@ガルニチュール!ふあるす@ファルス}
\index{ふあるす@ファルス}

ガルニチュールの多くは、その構成要素にファルスあるいはファルスで作った
「クネル」が含まれている。ファルスはまた、多くの大きな仕立ての料理にも
使われる。ここではまずファルスの材料および作り方を示し、使い途について
は後で述べることにする。

ファルスは大きく5種に分類される。

\begin{enumerate}
\def\labelenumi{\arabic{enumi}.}
\item
  仔牛肉と脂で作るもの。すなわち古典料理における\ul{ゴディヴォ}。
\item
  基本となる材料はさまざまだが、「つなぎ」に主としてパナードを使うもの。
\item
  近代的な手法で、生クリームを用いてふんわり泡立てたファルス。ムース、ムスリーヌに用いる。
\item
  レバーをベースとした「ファルス・\ul{グラタン}」。種類はいろいろだが作り方は常に同じ。
\item
  \ruby{主}{おも}に\protect\hyperlink{}{ガランティーヌ}、\protect\hyperlink{}{パテアンクルート}、\protect\hyperlink{}{テリーヌ}などの冷製料理に用いるシンプルなファルス。
\end{enumerate}

\newpage

\hypertarget{les-panades-pour-farces}{%
\subsection[ファルス用のパナードについて]{\texorpdfstring{ファルス用のパナードについて\footnote{パナードは本来、パンと水、バターを弱火で時間をかけて煮た粥のようなものを意味した。本書ではその意味を拡大して肉や魚肉をベースとしたファルスを加熱する際に崩れないようにする「つなぎ」として、この語を用いている。そのため、必ずしもパンを材料としていないものが含まれている。}}{ファルス用のパナードについて}}\label{les-panades-pour-farces}}

\frsecb{Les Panades pour Farces}

\index{garnitures@garnitures!farces@farces}
\index{farce@farce!panade@les panades pour farces}
\index{かるにちゆーる@ガルニチュール!ふあるす@ファルス!はなーと@パナード}
\index{ふあるす@ファルス!はなーと@---用パナード}

ファルスに用いるパナードにはいくつもの種類がある。ファルスの種類や、そ
のファルスを添える料理の性質によって使い分けることとなる。

原則として、パナードの分量は、ファルスのベースとする素材が何であれ、そ
の半量を越えないようにすること。

卵とバターを用いるパナードの場合はレシピの分量どおりに作らなければなら
ないから、それを合わせて作るファルスの全体量のほうを調節してやること。

パナードE以外のパナードは使用する際には必ず完全に冷めた状態になってい
ること。パナードが出来上がったら、バターを塗った平皿か天板に流し広げ、
早く冷めるようにする。このとき、バターを塗った紙で蓋をするか、表面にバ
ターのかけらをいくつか置いてやり、パナードが直接空気に触れないようにし
てやること。

以下のパナードのレシピは仕上がり重量が正味500
gになるように調整してある。

したがって、必要な量のパナードを作るのに材料を増やしたり減らしたりする
のも難しくはないだろう\footnote{原文では、Rien de plus simple, donc, que
  \ldots{}
  となっており、直訳すると「これ以上に簡単なことはない」と言いきっているが、都度計算しなければならないことに変わりはないので、多少ニュアンスを柔らげて訳した。}。

\hypertarget{panades}{%
\subsection{パナード}\label{panades}}

\frsecb{Panades}

\index{panade} \index{garniture!panade} \index{garniture!farce!panade}
\index{かるにちゆーる@ガルニチュール!ふあるす@ファルス!はなーと@パナード}
\index{はなーと@パナード}
\begin{recette}
\hypertarget{panade-a}{%
\subsubsection{A. パンのパナード}\label{panade-a}}

\frsub{Panade au pain}

\index{garnitures@garnitures!farces@farces!panade a@panade A}
\index{farce@farce!panade@les panades pour farces!panade a@panade A}
\index{panade!a pain@A. --- au pain}
\index{かるにちゆーる@ガルニチュール!ふあるす@ファルス!はなーとa@パナードA. パンの---}
\index{ふあるす@ファルス!はなーと@---用パナード!a@A. パンのパナード}
\index{はなーと@パナード!a@A. パンの---}

(魚を素材にした固めのファルス用)

\begin{itemize}
\item
  材料\ldots{}\ldots{}沸かした牛乳3 dL、固くなった白パン\footnote{ここではいわゆるバゲットのようなパンの外側を削り落した白い部分、
    あるいは食パンの「耳」を切り落した白い部分を使う、ということ。なお、
    フランスのパンは使う小麦粉の精白度や種類によって、pain complet
    (パンコンプレ)全粒粉パン、pain de sègle(パンドセーグル)ライ麦
    パン、精白度の高い小麦粉と食塩、塩、パン種だけで作るバゲットなどの
    pain と、バターや砂糖を加えて作るヴィエノワズリ(クロワッサンやパ
    ンオショコラ、ブリオシュなど)に分けられる。イギリスやアメリカのい
    わゆる食パン(フランス語 pain de mie パンドミ)は小麦粉、バター、
    塩、イースト菌、牛乳などで作られている。また、現代フランスでバゲッ
    トなどのパンに用いられている小麦粉の精白度は、T-55と呼ばれる灰分
    (小麦粉を燃やした際に残る炭水化物およびタンパク質以外の要素)0.5〜
    0.6%のものが主流であり、いわゆる食パンpain de mie(パンドミ)やヴィ
    エノワズリにはT-45(灰分0.5%以下)が多く用いられている。このほか
    T-65(灰分0.62〜0.75%)およびT-80(灰分0.75〜0.9%)、T-110(灰分
    1.0〜1.2%)、T-150(灰分1.4%前後、いわゆる全粒粉)のように種類が
    ある。このうちT-45およびT-55はfarine blanche(ファリーヌブロンシュ)
    と呼ばれ、T-150はfarine complète(ファリーヌコンプレット)と通称さ
    れている。灰分が高くなればそれだけ不純物が多いわけだから、粉は薄い
    茶色あるいはグレーがかった色合いになり、パンを焼く場合などはグルテ
    ン形成が難しくなりやすい。そのいっぽうで、香りゆたかなパンを実現し
    やすいという面もある。結果として、例えば全粒粉パンは香りはいいが固
    い仕上がりになりやすい。かつては精白度の低い(すなわち灰分の多い)
    粉ほど重量あたりの価格が安く、パンの価格もそれに比例していた。中世
    においてはパンの価格は基本的に1ドゥニエ(通貨単位)で、精白度の高
    いものは200〜300 g、精白度の低いものは700〜800 g程と大きな差があった
    という。ところで、本書では基本的に小麦粉を使う場合にその精白度について
    の指示はないが、概ねT-55またはT-45相当のもの考えていいだろう。なお、
    日本に輸入されている小麦は北米産のものがほとんどで、硬質小麦を粉に
    したものが「強力粉」、軟質小麦の場合は「薄力粉」と呼ばれ、精白度合
    いによる分類は通常なされていないが、製品としては概ねT-45相当あるい
    はそれ以上の精白度のものが多い。}の身250 g、塩5 g。
\item
  作業手順\ldots{}\ldots{}パンの身を牛乳に浸して完全にもどす。強火にかけて、ペー
  スト状になったパンがヘラから簡単に取れるくらいまで水気をとばす。バター
  を塗った平皿か天板に広げ、冷ます。
\end{itemize}

\maeaki

\hypertarget{panade-b}{%
\subsubsection{B. 小麦粉のパナード}\label{panade-b}}

\frsub{Panade à la farine}

\index{garnitures@garnitures!farces@farces!panade b@panade B}
\index{farce@farce!panade@les panades pour farces!panade b@panade B}
\index{panade!b farine@B. --- à la farine}
\index{かるにちゆーる@ガルニチュール!ふあるす@ファルス!はなーとb@パナードB. 小麦粉の---}
\index{ふあるす@ファルス!はなーと@---用パナード!b@パナードB. 小麦粉の---}
\index{はなーと@パナード!b@B. 小麦粉の---}

(肉、魚などあらゆるファルスに用いられる)

\begin{itemize}
\item
  材料\ldots{}\ldots{}水3 dL、塩2 g、バター50 g、篩にかけた小麦粉150 g。
\item
  作業手順\ldots{}\ldots{}片手鍋に水、塩、バターを入れて火にかけ、沸騰させる。
  火から外して小麦粉を加えて混ぜる。再度火にかけて、\protect\hyperlink{pate-a-chou}{シュー生地}を
  作る要領で余計な水分をとばす。上記パナードAと同様にして冷ます。
\end{itemize}

\maeaki

\hypertarget{panade-c}{%
\subsubsection[C. パナード・フランジパーヌ]{\texorpdfstring{C.
パナード・フランジパーヌ\footnote{フランジパーヌとは製菓で用いられる、小麦粉、砂糖、卵を混ぜて牛
  乳とバニラを加えて煮、砕いたマカロンmacaronを加えたクリーム。本文
  にあるように、このパナード・フランジパーヌにはマカロンは加えないの
  で、作り方のプロセスが途中まで似ていることからの命名だろう。なお、
  本来のクレーム・フランジパーヌに用いられるマカロンは、現代日本でよ
  く知られているタイプとは異なり、すり潰したアーモンドと卵白、砂糖を
  混ぜた生地を紙の上にクルミ大に絞り出してオーブンで焼いただけのシン
  プルなもの。Macaron craquelé(マカロンクラクレ)はこのタイプの代表
  的なもので、焼く際に膨らんで割れ目が出来ることからクラクレ(裂け目
  のある)の名称が付けられた。ところで、日本にマカロンが伝わった時期
  は不明だが、このタイプのものが太平洋戦争前には、アーモンドを落花生
  に代え、「まころん」の名称でいくつかの製菓会社で製造されるようにな
  り、現在も生産されている。フランス語 macaron の初出はボッカッチョ
  『デカメロン』のフランス語訳で、原文 maccheroni の訳語として現われ
  る。ただ、このフランス語訳は異本も多く、そのうちの写本のひとつに
  macaronという語が見られるに過ぎない点で、フランス語への影響という
  意味では微妙なところだ。むしろ既にフランス語として存在したmacaron
  と音が似ているからというだけの理由で訳語としてあてた可能性さえある。
  ボッカッチョの原書におけるマッケーローニはこんにちのそれ(マカロニ)
  とは違い、ニョッキのようなものだったと解釈されるのが定説であり、
  「マッケローニやラヴィオリを去勢鶏のブロードで煮る」という文脈で出
  てくる。次にmacaronという語がフランス語の文献で現われるのは16世紀
  フランソワ・ラブレーの小説『ガルガンチュアとパンタグリュエル』の
  「第四の書」であり、長い献立リストの一部に組込まれた河面だあ」あ
  (p.678)。ただしこのリストでは``Poupelin, Macaron. Tartres vingt
  sortes.''「ププラン(パティスリの一種)、マカロン、20種ものタルト」
  と並んでいることから、ボッカッチョのマッケローニとはまったく違うも
  のであることがわかる。また、17世紀には上述のようなマカロンの存在は
  知られていたという説があり、さらにフランス革命期にカルメル会修道女
  たちが隠れて作っていたというmacaron des soeurs(マカロン・クラクレ
  のタイプで平たい形状)はナンシーの名物としてこんにちも有名。なおこ
  のmacaron des soeurs のsoeurs は「姉妹たち」の意味ではなく「(修道
  女である)シスター」のことなので間違えないよう注意。}}{C. パナード・フランジパーヌ}}\label{panade-c}}

\frsub{Panade à la Frangipane}

\index{garnitures@garnitures!farces@farces!panade c@panade C}
\index{farce@farce!panade@les panades pour farces!panade c@panade C}
\index{panade!c frangipane@C. --- à la Frangipane}
\index{かるにちゆーる@ガルニチュール!ふあるす@ファルス!はなーとc@パナードC}
\index{ふあるす@ファルス!はなーと@---用パナード!はなーとC@パナードC}
\index{はなーと@パナード!c@C. ---・フランジパーヌ}

(鶏のファルス、魚のファルス用)

\begin{itemize}
\item
  材料\ldots{}\ldots{}小麦粉125 g、卵黄4個、溶かしバター90 g、塩2
  g、こしょう1 g、おろしたナツメグの粉ごく少量、牛乳2\undemi{} dL。
\item
  作業手順\ldots{}\ldots{}片手鍋に小麦粉と卵黄を入れてよく練る。溶かしバター、
  塩、こしょう、ナツメグを加える。沸かした牛乳で少しずつ溶きのばしてい
  く。
\end{itemize}

\protect\hyperlink{creme-frangipane}{標準的なフランジパーヌ}と同様に、火にかけて5〜6分間、泡立て器で混
ぜながら煮る。ちょうどいい漉さになったら、バットに移して\footnote{débarasser
  (デバラセ)バットなどに移す、片付ける、の意。とりわけ前者の意味に注意。}冷ます。

\maeaki

\hypertarget{panade-d}{%
\subsubsection{D. 米のパナード}\label{panade-d}}

\frsub{Panade au Riz}

\index{garnitures@garnitures!farces@farces!panade d@panade D}
\index{farce@farce!panade@les panades pour farces!panade d@panade D}
\index{panade!d riz@D. --- au Riz}
\index{かるにちゆーる@ガルニチュール!ふあるす@ファルス!はなーとd@パナードD. 米の---}
\index{ふあるす@ファルス!はなーと@---用パナード!はなーとd@D. 米のパナード}
\index{はなーと@パナード!d@D. 米の---}

(いろいろなファルスに用いられる)

\begin{itemize}
\item
  材料\ldots{}\ldots{}米200 gすなわち2
  dLあるいは大さじ8杯。\protect\hyperlink{}{白いコンソメ}6 dL、バター20
  g。
\item
  作業手順\ldots{}\ldots{}米を入れた鍋にコンソメを注ぎ、バターを加える。火にかけて沸騰させたら、オーブンに入れて40〜45分間加熱する。この間、米に触れないようにすること。
\end{itemize}

オーブンから出したら、米粒がよく潰れるようにヘラでしっかりと混ぜる。その後、冷ます。

\maeaki

\hypertarget{panade-e}{%
\subsubsection{E. じゃがいものパナード}\label{panade-e}}

\frsub{Panade à la pomme de terre}

\index{garnitures@garnitures!farces@farces!panade e@panade E}
\index{farce@farce!panade@les panades pour farces!panade e@panade E}
\index{panade!e riz@E. --- à la pomme de terre}
\index{かるにちゆーる@ガルニチュール!ふあるす@ファルス!はなーとe@パナードE}
\index{ふあるす@ファルス!はなーと@---用パナード!はなーとe@パナードE}
\index{はなーと@パナード!e@E. じゃがいもの---}

(仔牛および他の白身肉の、詰め物\footnote{fourrré
  (フレ)詰め物をした。farci (ファルシ)も同様に「詰め
  物をした」の意だが、後者はより一般的で、前者はオムレツやクレープに
  中身を詰めて「包む」のが本来の意味。すなわち、このパナードを加えた
  ファルスで、何らかの素材を「包む」と解釈してもいい。とりわけこの
  fourréには日本料理の用語「射込む」をあてる場合もある。}をする大きなクネルに用いられる)

\begin{itemize}
\item
  材料\ldots{}\ldots{}茹でて皮を剥いたばかりの中位のサイズのじゃがいも2個、牛
  乳3 dL、塩 2 g、白こしょう\undemi{} g、ナツメグ少々、バター20 g。
\item
  作業手順\ldots{}\ldots{}牛乳を2.5 dLになるまで煮詰める\footnote{原文は
    réduire le lait d'un sixième 直訳すると「牛乳を
    \unsixieme{}量だけ煮詰める」すなわち「\cinqsixiemes{}量まで煮詰め
    る」のだが、かえって分かりにくいため、具体的な数字に直して訳した。
    分量を代えて作る場合には85%まで煮詰めるくらいと考えてもいいだろう。
    そもそも、じゃがいもの重さが曖昧なのだから、あまり細かい数字にこだ
    わらず臨機応変に考えること。}。バター、調味料、
  薄く輪切りにしたじゃがいもを加え、15分間程加熱する。
\end{itemize}

このパナードはまだ少し\ruby{温}{ぬる}いくらいで使用すること。完全に
冷めてからではいけない。完全に冷めてから練ると粘りが出てしまうからだ。
\end{recette}
\hypertarget{farces}{%
\subsection{ファルス}\label{farces}}

\frsecb{Farces}

\index{farce} \index{garniture!farce}
\index{かるにちゆーる@ガルニチュール!ふあるす@ファルス}
\index{ふあるす@ファルス}

ベースとなる素材が\ul{仔牛}、\ul{鶏}、\ul{ジビエ}あるいは\ul{甲殻類}で
あっても、分量と作業手順はどんなファルスでも同じだ。そのベースにする素
材を代えればいいのだから、ここでは各種ファルスの典型的なレシピを示せば
充分だろう。料理で用いられるファルスひとつひとつを説明するのに一章をあ
てる必要はないと思われる。
\begin{recette}
\hypertarget{farce-a}{%
\subsubsection{A. パナードとバターを用いるファルス}\label{farce-a}}

\frsub{Farce à la Panade et au beurre}

\index{farce!a@A. --- à la Panade et au beurre}
\index{garniture!farce!a@A. Farce à la Panade et au beurre}
\index{かるにちゆーる@ガルニチュール!ふあるす@ファルス!a@A. パナードとバターを用いるファルス}
\index{ふあるす@ファルス!a@A. パナードとバターを用いる---}

(標準的なクネル、肉料理\footnote{原文 entrée
  (アントレ)、現代では「前菜」の意味で用いられるが、 本書では Relevé
  et Entrée 「ルルヴェとアントレ」すなわち肉料理の章
  に収録されているレシピ、仕立てのこと。これらのうちとりわけ大掛かり
  な仕立てのものをルルヴェ、それ以外をアントレと考えていい。本来ルル
  ヴェもアントレも魚を主素材にした仕立てが少なからずあったり、17世紀〜
  19世紀前半にかけての料理書では、いかに魚料理を大掛かりでゴージャス
  な仕立てでしかも美味なものにするか、が大きなテーマを占めていた。本
  書ではこれら四旬節の際などの「小斉」すなわち「肉断ちの料理」にあまりこだ
  わらない傾向があるために「魚料理」としてまとめられている。アントレ
  の場合は、概ね10人前を一皿に盛ったものを指し、現代でも立派にメイン
  の料理として通用するものがほとんど。実際、英語での前菜は hors-d'oeuvre
  または appetizer の語を用い、メインデュッシュには entree
  (またはフランス語のまま entrée)の語が現代でもあてられてい る。}の縁飾り
etc.)

\begin{itemize}
\item
  材料\ldots{}\ldots{}ていねいに筋取りをした肉1
  kg、\protect\hyperlink{panade-b}{パナードB} 500 g、塩12 g、こしょう2
  g、全卵4個、卵黄8個。
\item
  作業手順\ldots{}\ldots{}肉をさいの目に切って鉢に入れ、調味料を加えてすり潰す。
  いったん肉を取り出して、パナードをよくすり潰しながらバターを加える。
  肉を戻し入れ、すりこ木\footnote{pilon
    (ピロン)形状は日本のすりこ木をやや異なるのが多い。裏漉
    し用の漉し器(tamis タミ)とともに用いるピロンの場合は、棒の端に円
    盤状のやや厚い板を付けた形状のものが多かった。現代の手動式のポテト
    マッシャーのようなイメージだろうか。なお、ここでは大理石の鉢もしく
    は陶製のボウルを用いて作業していることに注意。現代ではフードプロセッ
    サなどを用いるところだろうが、かつては人力で、力を込めて丁寧に作業
    していたということは頭に留めておきたい。}で力強く練って全体をまとめる。
\end{itemize}

次に全卵と卵黄を加えて混ぜ合わせる。これは2回に分けても1回でやってもい
い。裏漉しして陶製の容器に入れる。さらに泡立て器で滑かになるまで混ぜる。

\hypertarget{nota-farce-a}{%
\subparagraph{【原注】}\label{nota-farce-a}}

どんな種類のファルスを作る場合でも、必ず少量を沸騰しない程度の温度で茹
でて\footnote{pocher (ポシェ)。}テストしてから、クネルの整形に取りかかること。

\maeaki

\hypertarget{farce-b}{%
\subsubsection{B. パナードと生クリームを用いるファルス}\label{farce-b}}

\frsub{Farce à la Panade et à la Crème}

\index{farce!b@B. --- à la Panade et à la crème}
\index{garniture!farce!b@B. Farce à la Panade et à la crème}
\index{かるにちゆーる@ガルニチュール!ふあるす@ファルス!b@B. パナードと生クリームを用いるファルス}
\index{ふあるす@ファルス!b@B. パナードと生クリームを用いる---}

(滑らかな仕上がりのクネル用)

\begin{itemize}
\item
  材料\ldots{}\ldots{}筋取りをした肉1
  kg、\protect\hyperlink{panade-c}{パナードC} 400 g、卵白5 個分、塩15
  g、白こしょう2 g、ナツメグ1 g、クレーム・ドゥーブル \footnote{乳酸発酵させた濃い生クリーム。フランスの生クリームについては\protect\hyperlink{sauce-supreme}{ソー
    ス・シュプレーム}訳注参照。}1\undemi{} L。
\item
  作業手順\ldots{}\ldots{}どんな肉を使う場合でも、卵白を少しずつ加えながらしっ
  かりとすり潰すこと。
\end{itemize}

パナードを加え、すりこ木でしっかり練り、二つの素材がよくよく混ざ
り合うようにする。

目の細かい網で裏漉しし、鍋にファルスを入れる。ヘラで滑らかになるよう混
ぜ、鍋を氷の上に置いて一時間ほど休ませる。

生クリームの\untiers{}量を少しずつ加えながら、のばしていく。最終的に残
りの\deuxtiers{}の生クリームも加えるが、これは先に泡立て器で軽く立てておくこと。

生クリームを全部加えた時点で、ファルスは真っ白で滑らかでしかも、ふんわりとし
た仕上がりにならなくてはいけない。

\hypertarget{nota-farce-b}{%
\subparagraph{【原注】}\label{nota-farce-b}}

手に入った生クリームが必ずしも最上級のものでない場合には、パナードC を
用いて\protect\hyperlink{farce-a}{バターを用いたファルス}を作った方がまだいい。

\maeaki

\hypertarget{farce-c}{%
\subsubsection{C. 生クリームを用いる滑らかなファルス /
ファルス・ムスリーヌ}\label{farce-c}}

\frsub{Farce à la Crème, ou Mousseline}

\index{farce!c@B. --- fine à la crème, ou Mousseline}
\index{garniture!farce!c@C. Farce fine à la crème, ou Mousseline}
\index{mousseline!farce mousseline}
\index{かるにちゆーる@ガルニチュール!ふあるす@ファルス!c@C. 生クリームを用いる滑らかなファルス / ファルス・ムスリーヌ}
\index{ふあるす@ファルス!c@C. 生クリームを用いる滑らかな--- / ---・ムスリーヌ}
\index{むすりーぬ@ムスリーヌ!ふあるす@ファルス・---}

(ムース、ムスリーヌ、ポタージュ用クネルなど)

\begin{itemize}
\item
  材料\ldots{}\ldots{}丁寧に掃除をして筋取りをした肉1
  kg、卵白4個分、クレーム・ エペス\footnote{crème épaisse fraîche
    低温殺菌の後、乳酸醗酵させたとても濃い生
    クリーム。前出のクレーム・ドゥーブルよりも濃い。}1\undemi{} L、塩18
  g、白こしょう3 g。
\item
  作業手順\ldots{}\ldots{}肉と調味料を鉢に入れて細かくすり潰す。卵白を少量ずつ
  加えていく。目の細かい網で裏漉しする。
\end{itemize}

これをソテー鍋に入れ、ヘラで滑らかになるまで混ぜたら、たっぷりの氷で鍋
を囲むようにして2時間冷やす。

次に、生クリームを少しずつ加えながらファルスをのばしていく。丁寧に練っ
ていくこと。またこの作業は鍋底を常に氷にあてた状態で行なうこと。

\hypertarget{nota-farce-c}{%
\subparagraph{【原注】}\label{nota-farce-c}}

\ldots{}\ldots{} 1.
上で示した生クリームの分量は平均的な数字だ。ファルスのベースとなっ
ている素材つまり肉、魚、甲殻類によってそれぞれタンパク質の特性が違
うのだから、素材に吸収される生クリームの量には多少の違いがでてくる
わけだ。

\begin{enumerate}
\def\labelenumi{\arabic{enumi}.}
\setcounter{enumi}{1}
\item
  ここで示したファルスの作り方は、滑らかな仕上がりのファルスの典型で
  あって、これを越える繊細さを出せるものはないから、ファルスに出来る
  材料すべて、つまり各種の肉、ジビエ、鶏、魚、甲殻類などに適用してい
  い。
\item
  卵白の量は、ファルスのベースと素材によって調整する必要がある。鶏や
  仔牛肉のようにアルブミンが多く含まれていて\footnote{当時の知見であることに注意。卵白が主としてアルブミンで出来てい
    るのは事実だが、肉については現代の知見と大きなズレがある。本書にお
    いて、赤身肉は「オスマゾーム」という架空の、茶褐色をした美味しさの
    エキスのようなものが豊富に含まれており、仔牛などの白身肉はアルブミ
    ンが主体であるとする考え方が随所に認められる。現代ならイノシン酸の
    「うま味」とテクスチュア、焼いた場合はメイラード反応による香気成分
    などが美味しさを感じさせる要素であると考えるところだが、フランス料
    理は長い歴史においてイノシン酸というアミノ酸の一種が「うま味」成分
    であるということを知らずに、けれども経験的にイノシン酸の比率が増え
    るようにブイヨンあるいはフォン、ソースなどの味を追究していった。イ
    ノシン酸やグルタミン酸、グアニル酸などのアミノ酸による「うま味」の
    概念そのものが、20世紀末になってようやく認知されるようになったに過
    ぎない。あくまでも「経験則」にもとづいて美味しさの探求が行なわれて
    きたと言える。}新鮮な肉であれば、成獣の
  固くなった肉を使う場合よりも量は少なくて済む。つまり、捌いたばかり
  でまだ温かい若鳥の胸肉を使ってこのファルス・ムスーズを作るのであれ
  ば、卵白は省略してもいい。
\item
  良質の生クリームが入手できる環境にあるなら、他のファルスを作るより
  もこのファルスの方がいいだろう。とりわけ、甲殻類をベースとしたファ
  ルスについては重要なことだ。
\end{enumerate}
\end{recette}
\hypertarget{godiveau}{%
\subsection[ゴディヴォ /
仔牛肉とケンネ脂のファルス]{\texorpdfstring{ゴディヴォ\footnote{ゴディヴォgodiveau
  はフランソワ・ラブレーの小説『ガルガンチュア
  とパンタグリュエル』の「第三の書」(1546年)が初出。原書の綴りは
  guodiveaulx。これは「アンドゥイエット(のようなもの)」と一般に解
  釈されている。ラブレーはこれに先立つ1534年「ガルガンチュア」(=第
  一の書)において gaudebillaux という表現を用いている。これについて
  は「ゴドビヨとは、たっぷり肥育した牛のトリップ(胃と腸)のこと」と
  本文で説明している。これらを敷衍すると、ゴディヴォはもともと牛など
  の胃や腸を刻んで詰めた腸詰すなわちアンドゥイエットのことだった、と
  考えたくなっても不思議はない。しかし、たとえ16世紀のラブレーにおけ
  るゴディヴォが当時アンドゥイエットと呼ばれるものとほぼ同じだったと
  しても、アンドゥイエット andouilette がアンドゥイユ andouille に縮
  小辞を付したものであることから、中世のアンドゥイユを確認する必要が
  出てくる。14世紀末に書かれた『ル・メナジエ・ド・パリ』においてアン
  ドゥイユは確かに「細かく刻んだ胃や腸を、腸詰にする」という説明がま
  ず出てくるが、その他に、牛の第1胃だけを詰めるもの、豚のコトレット
  を切り出した端肉を材料にするもの、胸腺肉やレバーを掃除した残りの肉
  を材料にするもの、が挙げられている(t.2,p.127)。これに従うなら、中
  世におけるアンドゥイユとは素材の定義があまりはっきりしていなかった
  もの、言える。ところが17世紀、ピエール・ド・リュヌ『新料理の本』
  (1660年)に「スペイン風アンドゥイエット」というレシピがある。概要
  を記すと、仔牛肉を細かく刻む。豚背脂少々、香草、卵黄、塩、こしょう、
  ナツメグ、粉にしたシナモンを加える。豚背脂のシートで巻いてアンドゥ
  イエットの形状にする。串を刺してローストする。ローストする際に滴り
  落ちてくる肉汁は受け皿で受ける。火が通ったらその肉汁をかける。茹で
  卵の黄身8〜10個分と細かくおろしたパン粉を順につけて、しっかりした
  衣を作る。提供時にレモン汁と羊のジュをかけ、揚げたパセリを添える、
  というものだ。1693年刊マシアロ『宮廷および大ブルジョワ料理の本』で
  は豚のアンドゥイユ、仔牛のアンドゥイユとともに、仔牛のアンドゥイエッ
  トというレシピが掲載されている。最後のものには材料として「細かく刻
  んだ仔牛肉、豚背脂、香草、卵黄、塩、こしょう、ナツメグ、シナモンを
  加えて作る」とある(pp.108-109)。また、1750年に出版された『食品、ワ
  イン、リキュール事典』でも、アンドゥイエットは「細かく刻んだ仔牛肉
  を楕円形に巻いたもの」と定義されている。実際、17、18世紀の料理書に
  出てくるアンドゥイエットは腸詰であるかどうかは別にしても、仔牛肉を
  主材料にしたものが多い。18世紀ヴァンサン・ラ・シャペル『近代料理』
  第1巻のアンドゥイエットも細かく刻んだ仔牛肉を豚の腸に詰めて作る。
  さて、ゴディヴォに戻ると、17世紀、1653年刊の『フランスのパティスリ
  の本』(ラ・ヴァレーヌが著者だと言われている)にはFaire un pasté de
  gaudiueau 「ゴディヴォのパテの作り方」という節があり、仔牛腿肉
  あるいは他の肉と脂身を細かく刻んだもの、をパテ(≒パイ包み焼き)に
  入れる。つまりここでも「仔牛腿肉」の使用が前提となっている。したがっ
  て、これら勘案すれば、ラブレーのゴディヴォもまた仔牛肉を材料にして
  いたものだった可能性は充分に考えられるだろう。 もちろんゴドビヨと
  いう別の巻で出てくる名詞との関連性は無視出来ないものだが、中世〜ル
  ネサンス期において、食にかかわる名詞、概念がしばしば曖昧だったこと
  を考えると、多少のわかりにくさは許容せざるを得ない。したがって、本
  書において仔羊腿肉とケンネ脂を使うゴディヴォを「古典的」なファルス
  として扱っているのはまことに正鵠を射ていると言えよう。} /
仔牛肉とケンネ脂のファルス}{ゴディヴォ / 仔牛肉とケンネ脂のファルス}}\label{godiveau}}

\frsecb{Farce de Veau à la Graisse de boeuf, ou Godiveau}

\index{farce!veau graisse de boeuf@--- de veau à la graisse de boeuf}
\index{garniture!farce!veau graisse de boeuf@Farce de veau à la graisse de boeuf}
\index{farce!veau glodiveau@Godiveau}
\index{garniture!farce!godiveau@Godiveau}
\index{かるにちゆーる@ガルニチュール!ふあるす@ファルス!こうしにくとけんねあふらのふあるす@仔牛肉とケンネ脂のファルス / ゴディヴォ}
\index{ふあるす@ファルス!こうしにくとけんねあふらのふあるす@牛仔牛肉とケンネ脂の--- / ゴディヴォ}
\index{かるにちゆーる@ガルニチュール!ふあるす@ファルス!こていうお@ゴディヴォ}
\index{ふあるす@ファルス!こていうお@ゴディヴォ} \index{godiveau}
\index{こていうお@ゴディヴォ}
\begin{recette}
\hypertarget{godiveau-mouille-a-la-glace}{%
\subsubsection[A. 氷を入れて作るゴディヴォ]{\texorpdfstring{A.
氷を入れて作るゴディヴォ\footnote{氷を入れて作る方法についてはカレームが1815年刊『パリ風パティ
  スリの本』の「シブレット入りゴディヴォ」原注において詳しく論じてい
  る。「不思議なことだが、氷を入れることでゴディヴォが滑らかなテクス
  チュアになり、素晴しくふんわりとしてとてもいい柔らかさに仕上がる。
  ゴディヴォが変質してしまうと、部分的とはいえそのクオリティはまった
  く失なわれてしまう。これは夏によく起こる事で、あまりに暑いとその熱
  で牛脂が仔牛肉としっかりつながらなくなってしまうからだ。一方(仔牛
  肉)は水分を含んでいて、もう一方(牛脂)は脂質そのものだからだ。だ
  から、夏の暑い時期には必ず氷を加えて作るべきであり、逆に冬場はそこ
  までする必要はない(p.142)」。ほぼ同時期のヴィアール『王国料理の本』
  1817年版においてゴディヴォのレシピの末尾に、「夏に、水の代わりに少
  量でも氷を使えるならそのほうがずっといい仕上がりになる(p.145)」と
  書かれている。これは、製氷機、冷凍庫が実用化されるのが19世紀中頃な
  ので、それよりやや早い時代ということになり、カレームの主たる活躍の
  舞台であった食卓外交というものが、いかに贅沢だったかを示していると
  も言えよう。言うまでもなく、17〜18世紀の料理書、パティスリの本にお
  いてゴディヴォのレシピは多く見られるが、氷の使用について言及したも
  のはいまのところ見つかっていない。}}{A. 氷を入れて作るゴディヴォ}}\label{godiveau-mouille-a-la-glace}}

\frsub{Godiveau mouillé à la glace}

\index{farce@farce!godiveau a@Godiveau A. Godeiveau mouillé à la glace}
\index{ふあるす@ファルス!こていうお@ゴディヴォ!a@A. 氷を入れて作るゴディヴォ}
\index{godiveau@godiveau!a@A. --- mouillé à la glace}
\index{こていうお@ゴディヴォ!a@A. 氷を入れて作る---}

\begin{itemize}
\item
  材料\ldots{}\ldots{}筋をきれいに取り除いた仔牛腿肉1
  kg、\ul{水気を含んでいない}牛ケンネ脂\footnote{腎臓の周囲を厚く覆っている脂肪。融解温度が低く、精製して牛脂(ヘッ
    ト)の原料となる。}1.5 kg、全卵8個、塩25 g、白こしょう5 g、ナツメグ1
  g、 透明な氷7〜800 gまたは氷水7〜8 dL。
\item
  作業手順\ldots{}\ldots{}はじめに、仔牛肉とケンネ脂を別々に、細かく刻む。仔牛
  肉はさいの目に切り、調味料と合わせておく。牛脂は細かくして、薄皮は筋
  はきれいに取り除いておく。
\end{itemize}

仔牛肉と牛脂を別々の鉢に入れて、それぞれすり潰す。次にこれらを合わせて
から、完全に混ざり合って一体化するまでよくすり潰し、卵を一個ずつ、すり
潰す作業を止めずに加えていく。

裏漉しして、平皿に\footnote{大きなバット。}広げ、氷の上に置いて翌日まで休ませる。

翌日になったら、再度ファルスをすり潰す。この時、小さく割った氷を少しず
つ加えていき、よく混ぜ合わせる。

ゴディヴォに氷を加え終わったら、必ずテスト\footnote{少量を、沸騰しない程度の温度で火を通し(ポシェ)て様子を見ること。}を行ない、必要に応じて
修正する。固すぎるようなら水を少々加え、柔らかすぎるようなら卵白を少し
加えること。

\hypertarget{nota-godiveau-a}{%
\subparagraph{【原注】}\label{nota-godiveau-a}}

ゴディヴォで作ったクネルはもっぱら、\protect\hyperlink{vol-au-vent}{ヴォロヴァン}の詰め物\footnote{原文
  garniture ガルニチュールの意味が広いことに注意。}にし
たり、牛、羊の塊肉の料理に添える\protect\hyperlink{garniture-financiere}{ガルニチュール・フィナンシエール}
に用いられる。

他のクネルがどれもそうであるように、沸騰しない程度の温度で茹でて\footnote{pocher
  (ポシェ)。}
火を通せばいいが、一般的には手で整形して塩を加えた沸騰しない程度の温度
の湯で茹でる。

だが、「ポシャジャセック\footnote{pochage à sec
  直訳すると「乾燥した状態でポシェすること」。つま
  り水(湯)を用いずに、pocher と同様に低めの温度で加熱することを指
  している。}」と呼ばれる技法、すなわち弱火のオーブ
ンで焼くのがいちばんいい。

以下に示す方法はとても短時間で出来るので特にお勧めだ。

ゴディヴォは充分に氷を加えて水気を含んだ状態にしておく。オーブンの天板
に敷いたバターを塗った紙の上に、丸口金を付けた絞り袋から絞り出す。オー
ブンの天板にもバターを塗っておくこと。絞り出したクネルは触れ合うように
していい。

これを低温のオーブンに入れて加熱する。

7〜8分すると、クネルの表面に脂が水滴状に浸み出してくる。これが、ちょう
どいい具合に火が通った合図だ。オーブンから出して、クネルを別の銀製の盆
か大理石の板の上に裏返しに広げる。クネルが\ruby{微温}{ぬる}くなるまで
冷めたら、敷いてあった紙を端のほうから引き剥して取り除く。

クネルは完全に冷めるまで放置し、その後に皿に移すか、可能なら柳編みのすのこに
載せてやるのがいい。

\maeaki

\hypertarget{godiveau-a-la-creme}{%
\subsubsection{B. 生クリーム入りゴディヴォ}\label{godiveau-a-la-creme}}

\frsub{Godiveau à la crème}

\index{farce@farce!godiveau b@Godiveau B. Godeiveau  à la crème}
\index{ふあるす@ファルス!こていうお@ゴディヴォ!b@B. 生クリーム入りゴディヴォ}
\index{godiveau@godiveau!b@B. --- à la crème}
\index{こていうお@ゴディヴォ!b@B. 生クリーム入り---}

\begin{itemize}
\item
  材料\ldots{}\ldots{}筋をきれいに取り除いた極上の白さの仔牛腿肉1
  kg、水気を含んでいない牛ケンネ脂1 kg、全卵4個、卵黄3個、生クリーム7
  dL、塩25 g、白こしょう5 g、ナツメグ1 g。
\item
  作業手順\ldots{}\ldots{}仔牛肉とケンネ脂は別々に、細かく刻む。これらを鉢に入れて合わせ、調味料、全卵、卵黄をひとつずつ加えながら、力強く全体をすり潰し、完全に一体化させる。
\end{itemize}

裏漉しして、天板に広げる。氷の上にのせて翌日まで休ませる。

翌日になったら、あらかじめ中に氷を入れて冷やしておいた鉢で再度すり潰す。この際に生クリームを少量ずつ加えていく。

クネルを整形する前にテストをして、必要があれば固さなどを修正してやること。

\maeaki

\hypertarget{godiveau-lyonnais}{%
\subsubsection[C.
リヨン風ゴディヴォ/ケンネ脂入りブロシェのファルス]{\texorpdfstring{C.
リヨン風ゴディヴォ\footnote{このレシピは第二版以降。このファルスが仔牛肉が材料ではなくパナー
  ドも使うにもかかわらずゴディヴォの名称である根拠はおそらく、ケンネ
  脂を用いていることだろう。なお、これを用いたブロシェのクネルの起源
  については、リヨンのシャルキュトリ(豚肉加工業者)であるオ・プチ・
  ヴァテルの店主ルイ・レグロスが1907年に創案したものだという説がある。
  しかしこの説は、1907年の本書第二版にファルスとクネル両方のレシピが
  収録されているといることで否定されよう。また、19世紀前半にオーヴェ
  ルニュ・ローヌ・アルプ地方にある宿屋の主J.-F. モワーヌなる人物が、
  宿泊客を呼び込むための料理としてブロシェの身と卵、小麦粉で作ったク
  ネルを創案し、これがリヨンに伝わったという説もある。ただしこれは信
  憑性がさほど高くないうえに、そもそもケンネ脂を使わないのであれば、
  その後のリヨン風ゴディヴォとは似て非なるものということになろう。魚
  のすり身をクネルにすることはローマ時代後期の『アピキウス』(この場
  合は人物ではなく料理書の意)以来、ヨーロッパにおいてごくあたりまえ
  のように行なわれてきたことだ。いずれにしても、本書では\protect\hyperlink{quenelles-de-brochet-lyonnaise}{ブロシェの
  クネル リヨン風}のレシピでのみこ
  のファルスが用いられることになる。その意味でも、「ブロシェのクネル
   リヨン風」という料理が20世紀初頭に大流行したものだったことは間違
  いなく、そのことが理由で第二版においてレシピが追加されたと考えら
  れる。}/ケンネ脂入りブロシェのファルス}{C. リヨン風ゴディヴォ/ケンネ脂入りブロシェのファルス}}\label{godiveau-lyonnais}}

\frsub{Godiveau Lyonnais ou Farce de Brochet à la graisse}

\index{farce@farce!godiveau c@Godiveau C. Godeiveau Lyonnais ou Farce de Brochet à la graisse}
\index{ふあるす@ファルス!こていうお@ゴディヴォ!c@C. リヨン風ゴディヴォ / ケンネ脂入りブロシェのファルス}
\index{godiveau@godiveau!c@C. --- Lyonnais ou Farce de Brochet à la graisse}
\index{こていうお@ゴディヴォ!c@C. リヨン風--- / ケンネ脂入りブロシェのファルス}

\begin{itemize}
\item
  材料\ldots{}\ldots{}皮とアラをきれいに取り除いたブロシェ\footnote{brochet
    ノーザンパイク、和名キタカワカマス。カワカマス属の淡水、汽水魚。}の身(正味重量)500
  g、筋を取り除き細かく刻んだ水気を含んでいない牛ケンネ脂500 g(または
  ケンネ脂と白い牛骨髄半量ずつ)、\protect\hyperlink{panade-c}{パナード
  C}500 g、卵白4 個分、塩15 g、こしょう4 g、ナツメグ1 g。
\item
  作業手順\ldots{}\ldots{}まず鉢でブロシェの身をすり潰す。これを取り出して、次
  にケンネ脂にパナード(よく冷やしたもの)を加えてすり潰し、卵白を少し
  ずつ加えていく。ブロシェの身と調味料を入れ戻す。すりこ木で力強く練り、
  裏漉しする。
\end{itemize}

陶製の器に移し、ヘラで滑らかになるまで練る。使うまで、氷の上に置いてお
く。

次のように作ってもいい。ブロシェの身を調味料とともにすり潰し、そこにパ
ナードを加える。裏漉しして、鉢に戻す。すりこ木で力強く練ってまとまるよ
うになったらケンネ脂を少しずつ加えるか、溶かしたケンネ脂と牛骨髄を加え
て、よくまとめる。陶製の器に移し、氷の上に置いておく。

\maeaki

\hypertarget{farce-de-veau-pour-bordures}{%
\subsubsection{盛り付けの縁飾りおよび底に敷いたり、詰め物をしたクネルに用いる仔牛のファルス}\label{farce-de-veau-pour-bordures}}

\frsub{Farce de veau pour Bordures de dressage, fonds, quenelles fourrées etc.}

\index{farce!veau@--- de veau pour Bordures de dressage, fonds, quenelles fourrées, etc.}
\index{garniture!farce!veau@Farce de veau pour Bordures de dressage, fonds, quenelles fourrées etc.}
\index{かるにちゆーる@ガルニチュール!ふあるす@ファルス!こうし@盛り付けの縁飾りおよび底に敷いたり、詰め物をしたクネルに用いる仔牛の---}
\index{ふあるす@ファルス!こうし@盛り付けの縁飾りおよび底に敷いたり、詰め物をしたクネルに用いる仔牛の---}

\begin{itemize}
\item
  材料\ldots{}\ldots{}筋をきれいに取り除いた\ul{極上の白さの\\仔牛腿肉}1
  kg、 \protect\hyperlink{panade-e}{パナード E} 500 g、バター300
  g、全卵5個、卵黄8個、濃い冷
  えた\protect\hyperlink{sauce-bechamel}{ベシャメルソース}大さじ2杯、塩20
  g、白こしょう 3 g、ナツメグ1 g。
\item
  作業手順\ldots{}\ldots{}鉢に仔牛肉と調味料を入れてて細かくすり潰す。これを鉢
  から取り出す。
\end{itemize}

まだ温い状態のじゃがいものパナードを入れ、すりこ木でペースト状になるま
で練り、だいたい冷めた頃に、先にすり潰した仔牛肉を戻し入れる。全体によ
く混ぜながら、バター、全卵、卵黄をひとつずつ加えていき、最後に冷たいベ
シャメルソースを加える。

裏漉しして、陶製の器に入れ、充分に滑らかになるまでヘラで練る\footnote{装飾用、あるいは中に別の食材を射込んだ大きなクネルを作る目的な
  ので、加熱後はしっかりとしたテクスチュアとなる。クネルにする場合も、
  アトレと呼ばれる飾り串を刺して料理を飾るのが主要な目的で、トリュフ
  を射込むなど、料理としてきちんと成立していた。なおアトレはエスコフィ
  エの時代にフランスではほぼ用いられなくなっていたが、アメリカ経由で
  19世紀中頃のフランス料理をベースに始まった日本の西洋料理では、20世
  紀になってからも使われ続けていたという。}。

\maeaki

\index{garniture@garniture!farce@farce!gratin@Farce Gratin}
\index{farce@farce!gratin@--- gratin}
\index{gratin@gratin!farce@farce ---}
\index{かるにちゆーる@ガルニチュール!ふあるす@ファルス!くらたん@ファルス・グラタン}
\index{ふあるす@ファルス!くらたん@ファルス・グラタン}
\index{くらたん@グラタン!ふあるす@ファルス・---}

\hypertarget{farce-gratin-a}{%
\subsubsection[ファルス・グラタン
A]{\texorpdfstring{ファルス・グラタン\footnote{ここでゴディヴォのように小見出しがあって然るべきところだが、初
  版には小見出しの類が一切なかったので、第二版改訂の際に見落とされて
  そのままになったのだろう。本書におけるファルス・グラタンの定義が、
  決して「グラタン用」ファルスではないことに注意。語源的には gratin
  \textless{} gratter
  (グラテ)引っ掻く、であり、元来はbouillie(ブイイ)とい
  う粥のようなものの鍋底や隅に貼り付いた部分のことをグラタンと呼んだ。
  18世紀マラン『コモス神の贈り物』には「グラタン」という名称のファル
  スがある。これは、鶏胸肉、レバー、牛の骨髄、香草などと卵黄をすり潰
  して練ったもの(t.1, p.143)。また仕立てとしてのグラタンは深皿にこの
  ファルスを敷き詰め、その上に別途調理した素材をのせてソースをかけ、
  フルノーの端でファルスが容器に貼り付く程度に加熱する(仔牛の耳のグ
  ラタン(id., p.209)、エクルヴィスのグラタン(id., pp.171-172)がある。
  その後、グラタンという名称のファルスは他の料理書に記されなかったが、
  1868年のデュボワとベルナールの『古典料理』においてfarce à gratin de
  gibier, farce à gratin de foie-grasの2つのレシピが掲載され
  (p.125)、その約半世紀後『料理の手引き』において完全に復活したが、
  その頃にはグラタンという仕立てがまったく別の、こんにち我々がよく知っ
  ているものへと変わってしまっていた。このため、本書におけるグラ
  タンの説明(原書pp.405-407)においてもこれらのファルス・グラタンは用
  いられない。} A}{ファルス・グラタン A}}\label{farce-gratin-a}}

\frsub{Frace Gratin A}

\index{farce@farce!gratin a@--- Gratin A}
\index{garniture@garniture!farce@farce!gratin a@Farce Gratin A}
\index{かるにちゆーる@ガルニチュール!ふあるす@ファルス!くらたんa@---・グラタン A}
\index{ふあるす@ファルス!くらたんa@---・グラタン A}

(標準的な温製パテ\footnote{pâté
  とは本来、生地で素材を包んで焼いたもの全般を指す。こんにち
  ではその意味が失なわれつつあり「パイ包み」のような表現をとることも
  多い。決して英語のpatty(小型のミートパイ、ハンバーガーのパティなど)と
  混同しないこと。}、大皿料理\footnote{Entrée
  アントレ。\protect\hyperlink{panade-a}{パナードとバターを用いるファルス}訳注参照。}の縁飾りなど)

\begin{itemize}
\item
  ファルス1 kg分の材料\ldots{}\ldots{}豚背脂250
  g、筋をきれいに取り除いた極上 の白さの仔牛腿肉1
  kg、出来るだけ白い仔牛のレバー250 g、バター150 g、
  マッシュルームの切りくず40 g、トリュフの切りくず(可能なら生のもの)
  25 g、卵黄6個、ローリエの葉\undemi{}枚、タイム1枝、エシャロット4個、
  塩20 g、こしょう4 g、ミックススパイス\footnote{原文は初版から一貫して、2
    grammes d'épices 直訳すると「香辛料2 g」
    としか記されていないが、フランスでもっともポピュラーなミックススパ
    イスであるquatre-épicesカトルエピスの場合は、こしょう、ナツメグ、
    クローブ、シナモンの粉末のミックス。また「オールスパイス」単独を意
    味することもある。なお、1907年の英語版には、ローリエ5オンス、タイ
    ム3オンス、コリアンダー3オンス、シナモン4オンス、ナツメグ6オンス、
    クローブ4オンス、ジンジャーパウダー3オンス、メース3オンス、黒こしょ
    うと白こしょう同量ずつ計10オンス、カイエンヌ1オンス、を粉末にして
    保存すべし(p.75)、とあるが、フランス語原書にこのミックススパイスの
    レシピはいずれの版でも記されていない。}2 g、マデイラ酒1\undemi{}
  dL、
  \protect\hyperlink{sauce-espagnole}{ソース・エスパニョル}1\undemi{}
  dL(よく煮詰めて あって、冷やしてあること)。
\item
  作業手順\ldots{}\ldots{}豚背脂をさいの目に切る。ソテー鍋に50gのバターを熱し、強火で色よく焼く。
\end{itemize}

背脂が色付いたらすぐに取り出して余分な脂をきり、同じ鍋で、大きめのさい
の目に切った仔牛肉を色よく焼く。同様してに余分な脂はきる。

同じく強火で、仔牛肉と同様に切ったレバーを色よく焼く。仔牛肉と背脂を鍋
に戻し入れ、マッシュルームの切りくず、トリュフの切りくず、タイム、ロー
リエの葉、みじん切りにしたエシャロットと調味料を加える。2分程火にかけ
たままにし、バットにあける。ソテー鍋にマデイラ酒を注いでデグラセ\footnote{肉を焼く際に肉から浸み出た肉汁が濃縮して鍋底に貼り付いているの
  を、何らかの液体を注いで溶かし出すこと。意味としては「焦げ」を取る
  ことではないので注意。}する。

鉢に背脂、仔牛肉、レバーなどを入れて細かくすり潰しながら、バターの残り
(100 g)と卵黄をひとつずつ加えていく。さらに煮詰めたソース・エスパニョ
ルとデグラセしたマデイラ酒を加える。裏漉しして、陶製の容器に入れ、ヘラで
滑らかになるまで練る。

\hypertarget{nota-farce-gratin-a}{%
\subparagraph{【原注】}\label{nota-farce-gratin-a}}

このファルスのレシピでの仔牛のレバーは鶏や鴨、がちょう、七面鳥のレバー
に代えてもいい。その場合は、胆汁および胆汁で汚れた部分を丁寧に取り除く
必要がある。

\maeaki

\hypertarget{farce-gratin-b}{%
\subsubsection{ファルス・グラタン B}\label{farce-gratin-b}}

\frsub{Frace Gratin B}

(ジビエの温製パテ用)

\index{farce@farce!gratin b@--- Gratin B}
\index{garniture@garniture!farce!farce gratin b@Farce Gratin B}
\index{かるにちゆーる@ガルニチュール!ふあるす@ファルス!くらたんb@---・グラタン B}
\index{ふあるす@ファルス!くらたんb@---・グラタン B}

\begin{itemize}
\item
  ファルス1 kg分の材料\ldots{}\ldots{}塩漬け豚バラ肉250
  g、穴うさぎの\footnote{lapin de garenne
    (ラパンドガレーヌ)、野生の穴うさぎ。いわ
    ゆる野うさぎlièvre(リエーヴル)とは肉質も違い、まったく別のものとして扱われる。
    この穴うさぎを家畜化したものが、いわゆるlapin(ラパン)。}肉
  (正味重量)250 g、鶏とジビエのレバー250 g、マッシュルーム、トリュフ、
  タイム、ローリエ、エシャロット、塩こしょうは\protect\hyperlink{farce-gratin-a}{ファルス・グラタン
  A}と同じ。バター50 g、生あるいは加熱済みのフォワグラ 100
  g、卵黄6個、マデイラ酒1\undemi{}
  dL、ジビエで作った\protect\hyperlink{sauce-espagnole}{ソース・エスパ
  ニョル}または\protect\hyperlink{sauce-salmis}{ソース・サルミ}をよく
  煮詰めて冷ましたもの1\undemi{} dL。
\item
  作業手順\ldots{}\ldots{}前項で説明したように、バターで3種の素材、つまり豚バ
  ラ、うさぎ肉、レバーを別々に色よく焼く。これらをソテー鍋に調味料、香
  辛料とともに入れ、軽く炒めたらマデイラ酒を注ぎ蓋をして弱火で5分程蒸し
  煮\footnote{étuver (エチュヴェ)。}する。よく水気をきってから鉢に入れてすり潰す。充分に滑らかに
  なったら、フォワグラと卵黄、冷めたソースとマデイラ酒を加える。裏漉しし
  て、ヘラで滑らかになるまで混ぜる。
\end{itemize}

\maeaki

\hypertarget{farce-gratin-c}{%
\subsubsection{ファルス・グラタン C}\label{farce-gratin-c}}

\frsub{Frace Gratin C}

\index{farce@farce!gratin c@--- Gratin C}
\index{garniture@garniture!farce!farce gratin c@Farce Gratin C}
\index{かるにちゆーる@ガルニチュール!ふあるす@ファルス!くらたんc@---・グラタン C}
\index{ふあるす@ファルス!くらたんc@---・グラタン C}

(詰め物をしたクルトン、カナペ、小型ジビエ、仔鴨用)

\begin{itemize}
\item
  ファルス1 kg分の材料\ldots{}\ldots{}生のフレッシュな豚背脂\footnote{塩漬けなどの加工をしていないということ。なお、lard
    (gras) (ラー ル グラ)は「豚背脂」を意味し、lard maigre
    (ラールメーグル)また はlard de
    poitrine(ラールドポワトリーヌ)は塩漬け豚ばら肉およびそ
    れを冷燻したものを意味する。後者はしばしば日本語で「ベーコン」と誤
    訳されるが、日本語でいう「ベーコン」は温燻、熱燻されたものであり、
    風味などが大きく異なるので注意。近年は「生ベーコン」という商品名の
    ものもあるらしく、紛らわしいので注意が必要だろう。いずれにしても、
    豚背脂は薄いシート状または長い棒状、拍子木状にして、素材の油脂分と
    風味を補う目的で使われることが多く、豚ばら肉の塩漬けおよびその冷燻
    品は拍子木状に切って(lardon ラルドン)各種料理に使われる。既に拍
    子木状にカットされたものがごく一般的に市販されており、それぞれ
    lardon(ラルドン)、lardon fumé(ラルドンフュメ)と呼ばれ非常にポ
    ピュラーな食材。}を器具を用い ておろしたもの\footnote{râper (ラペ)
    \textless{} râpe (ラープ)という器具を用いておろすこと。 Mandeline
    (マンドリーヌ)と呼ばれる野菜スライサーにこの機能が付属
    しているものは非常に多い。}300 g、鶏レバー600
  g、エシャロット4〜5個の薄切り \footnote{émincé \textless{} émincer
    (エマンセ)薄切りにする、スライスする。}、マッシュルームの切りく\footnote{マッシュルームは通常、料理として提供する際にはtourner
    (トゥル
    ネ)と呼ばれる、螺旋状の切れ込みを入れて装飾したものが使われる。こ
    の際に少なくない量の切りくずが発生するのでそれを利用する。なお
    tournerの原義は「回す」であり、包丁を持った側の手は動かさずに材料
    を回すようにして切れ目を入れたり皮を剥いたりすることを意味する料理
    用語。}ず25 g、ローリエの葉\undemi{}枚、 タイム1枝、塩18
  g、こしょう3 g、ミックススパイス3 g。
\item
  作業手順\ldots{}\ldots{}ソテー鍋に豚背脂を熱して溶かす。レバーと香辛料、調味
  料を加え、強火で\ul{色付かないように}炒める。
\end{itemize}

いま、\ul{色付かないように}\footnote{原文 raidir ou saisir
  (レディール ウ セジール)。前者は油脂を
  熱したフライパン等で、材料が色付かないように表面を焼き固めること。
  後者「セジール」は焼く、炒める、茹でるなど方法は問わないが、熱によっ
  て表面だけを固める(タンパク質の熱変性)ことを指す。}と書いたように、焼き色を付けないように
することがポイント。レバーはレアな焼き加減で血が滴るくらいにすると、バラ色
のきれいなファルスに仕上がる\footnote{現代の衛生学的知見からすると、充分に加熱調理していないレバーに
  は食中毒あるいは肝炎などのリスクがあるので注意。}。

材料がだいたい冷めたら鉢に入れてすり潰す。裏漉しして、陶製の容器に移し
てヘラで練って滑らかにする。バターを塗った紙で蓋をして冷蔵する。
\end{recette}
\hypertarget{farce-pour-les-pieces-froides}{%
\subsection{冷製料理用のファルス}\label{farce-pour-les-pieces-froides}}

\vspace{-1.5\zw}
\begin{center}
\textbf{(ガランティーヌ、パテアンクルート、テリーヌ)}
\end{center}
\vspace{.5\zw}
\frsecb{Farces pour Pièces froides}
\begin{center}
\vspace{-1\zw}
\hspace{1\zw}\textbf{(Galantines --- Pâtés --- Terrines)}
\end{center}

\index{farce@farce!piece froides@--- pour les pièces froides}
\index{ふあるす@ファルス!れいせいりようりよう@冷製料理用の---}

\normalfont
\begin{recette}
\hypertarget{assaisonnement-et-liaison}{%
\subsubsection{味付けと「つなぎ」}\label{assaisonnement-et-liaison}}

\frsub{Assaisonnement et Liaison}

ガランティーヌや、パテアンクルート、テリーヌに用いる標準的なファルスは、
ファルス1 kgあたり25〜30 gのスパイスソルトで調味する。最後に、肉1 kgあた
りコニャック1\undemi{} dLを振りかける。

冷製料理用のファルスは以下のように3つに分類される。これらは前述の滑ら
かな口あたりのファルスやファルス・グラタンとはまったく違うものである。

「つなぎ」が必要な場合には、ファルス1 kgあたり全卵2個を加えて調整する。

\hypertarget{sel-epice}{%
\subsubsection{スパイスソルト}\label{sel-epice}}

\frsub{Sel épicé}

\index{sel epice@sel épicé} \index{すぱいすそると@スパイスソルト}
\index{こうしんりよういりしお@香辛料入りの塩 ⇒ スパイスソルト}

スパイスソルトはよく乾燥した細かい塩100 gと、こしょう20 g、ミックスス
パイス\footnote{\protect\hyperlink{farce-gratin-a}{ファルス・グラタン
  A}訳注参照。}20 gを混ぜて作る。

すぐに使わない場合は、密閉できる缶に入れて乾燥した場所で保存すること。

\hypertarget{farce-froide-a}{%
\subsubsection{ファルス A (豚肉)}\label{farce-froide-a}}

\frsub{Farce A (Porc)}

\index{farce@farce!froide a@--- pour les pièces froides A (Porc)}
\index{garniture@garniture!farce!farce froides a@Farce pour les pièces froides A (Porc)}
\index{かるにちゆーる@ガルニチュール!ふあるす@ファルス!れいせいa@冷製料理用--- A}
\index{ふあるす@ファルス!れいせいa@冷製料理用--- A}

これは豚肉の脂身のない部分と、フレッシュな背脂を同量ずつ用いる。別々に
細かく刻むこと。それを鉢に入れて合わせてすり潰し、調味と風味付けを上記
の分量比率で行なう。

ごく標準的なパテアンクルートやテリーヌに用いらる。

これは「\protect\hypertarget{chair-a-saucisse}{ソーセージ用の挽肉}」\footnote{chair
  à saucisses
  (シェラソシス)。料理書によってはよく出てくる表現なので覚えておくといいだろう。}としても使われる。\label{chair-a-saucisse}

\maeaki

\hypertarget{farce-froide-b}{%
\subsubsection{ファルス B (仔牛肉と豚肉)}\label{farce-froide-b}}

\frsub{Farce A (Porc)}

\index{farce@farce!froide b@--- pour les pièces froides B (Veau et Porc)}
\index{garniture@garniture!farce!farce froides b@Farce pour les pièces froides B (Veau et Porc)}
\index{かるにちゆーる@ガルニチュール!ふあるす@ファルス!れいせいb@冷製料理用--- B}
\index{ふあるす@ファルス!れいせいb@冷製料理用--- B}

\begin{itemize}
\item
  材料\ldots{}\ldots{}仔牛腿肉の輪切り250
  g、さいの目に切った豚肉の脂身を含ま ない部分250
  g、フレッシュな豚背脂500 g、全卵2個、調味料とコニャック
  は上記のとおり。
\item
  作業手順\ldots{}\ldots{}仔牛肉、豚肉、背脂を別々に細かく刻む。調味料とともに
  鉢に入れてよくすり潰し、最後に、火を点けてアルコールをとばした\footnote{flamber
    (フロンベ)。フランベする。鍋に入れて火にかけるとコニャッ
    クのようにアルコール度の高い酒類はすぐにアルコール分が揮発して非常
    に燃えやすくなる。} コニャックを加える。裏漉しする。
\end{itemize}

このファルスは主としてガランティーヌに使うが、パテアンクルートやテリー
ヌに用いてもいい。

\hypertarget{farce-froide-c}{%
\subsubsection{ファルス C (鶏とジビエ)}\label{farce-froide-c}}

\frsub{Farce C (Volaille et Gibier)}

\index{farce@farce!froide c@--- pour les pièces froides C (Volaille et Gibier)}
\index{garniture@garniture!farce!farce froides c@Farce pour les pièces froides C (Volaille et Gibier)}
\index{かるにちゆーる@ガルニチュール!ふあるす@ファルス!れいせいc@冷製料理用--- C}
\index{ふあるす@ファルス!れいせいc@冷製料理用--- C}

このファルスの素材はいろいろだから、分量比率は使用する鶏とジビエの肉の
正味重量\footnote{poids net (ポワネット)。}から調節することになる。

例えば、中抜きしただけの丸鶏の重量\footnote{廃棄分なども含めた全重量は
  poids brut (ポワブリュット)。}が1.5 kgの場合、ガルニチュール
に使うフィレの量は500〜600 gに減ってしまうことになる。そのため、ファル
スの材料の分量比率は以下のようになる。

鶏肉550 g、きれいに筋取りした仔牛肉200 g、豚肉の脂身のないところ200 g、
生の豚背脂900
g、全卵4個、\protect\hyperlink{sel-epice}{スパイスソルト}50〜60
g、コニャッ ク3 dL。

作業手順\ldots{}\ldots{}肉と背脂は別々にして、それぞれ細かく刻む。これを鉢に入
れて合わせ、調味料を加える。細かくすり潰しながら卵を一個ずつ加えていく。
コニャックは最後に加えること。裏漉しする。

ジビエのファルスも同様の材料の比率で、同じように作る。

\hypertarget{observation-sur-les-farces}{%
\subsubsection{冷製料理用ファルスの補足}\label{observation-sur-les-farces}}

場合によっては、ファルスB(仔牛と豚)およびファルスC(鶏)に、ファルス 1
kgあたりフォワグラ125 gを加えることがある。その場合フォワグラは出来
るだけ新鮮なものを用いて、裏漉しして加えること。あるいはトリュフのみじ
ん切り50 gを加えることもある。

ジビエのファルスCを極上の滑らかな仕上りにするには、\unquart{}量の\protect\hyperlink{farce-gratin-b}{ファ
ルス・グラタンB}と、ファルスのベースにしたジビエのフュ
メをよく煮詰めて少量加えるといい。
\end{recette}
\hypertarget{farces-speciales-pour-garnir-les-poissons-braises}{%
\subsection[魚のブレゼのガルニチュール用ファルス]{\texorpdfstring{魚のブレゼ\footnote{本書において魚を「煮る」あるいは「茹でる」場合、通常はクールブ
  イヨンか塩水で沸騰させない程度の温度で火入れをする(ポシェ)。料理
  の仕立てとしての「ブレゼ」は基本が牛、羊の赤身肉であり、仔羊、仔牛、
  家禽などはやや例外的な位置付けとして「ブレゼ」が存在する。同様に、
  サーモン、大型のトラウト、チュルボ、チュルボタンなどについても「ブ
  レゼ」という仕立ての方法が\protect\hyperlink{cuisson-des-poissons-par-le-braisage}{「第6章魚料
  理」}において説明されている ので併せて読んでおきたい。}のガルニチュール用ファルス}{魚のブレゼのガルニチュール用ファルス}}\label{farces-speciales-pour-garnir-les-poissons-braises}}

\frsecb{Farces spéciales pour garnir les Poissons Braisés}
\begin{recette}
\hypertarget{farces-poissons-braises-a}{%
\subsubsection{ファルス A}\label{farces-poissons-braises-a}}

\frsub{Farce A}

\index{farce@farce!poissons braises a@---s spéciales pour les poissons braisés A}
\index{garniture@garniture!farce!farce poissons braises a@Farce spéciale pour les poissons braisés A}
\index{かるにちゆーる@ガルニチュール!ふあるす@ファルス!さかなのふれせa@魚のブレゼのガルニチュール用--- A}
\index{ふあるす@ファルス!さかなのふれせa@魚のブレゼのガルニチュール用--- A}

\begin{itemize}
\item
  材料\ldots{}\ldots{}細かく刻んだ生の白子\footnote{laitance
    (レトンス)。伝統的な高級料理では鯉の白子が一般的に使
    用された。他に鯖や鰊の白子も食用とするが、日本のようにスケトウダラ
    の白子を食材とするケースはほとんどないと思われる。}250
  g、白いパンの身180 gを牛乳 に浸して絞ったもの、塩5g、こしょう1
  g、ナツメグごく少量、シブレット 10gとパセリの葉5 g、セルフイユ20
  gをみじん切りにしたもの。バター50 g、 全卵1個、卵黄3個。
\item
  作業手順\ldots{}\ldots{}陶製の鉢に材料をすべて入れ、木のヘラで全体をよく練り、
  完全にまとまるようにする。
\end{itemize}

\hypertarget{farces-poissons-braises-b}{%
\subsubsection{ファルス B}\label{farces-poissons-braises-b}}

\frsub{Farce B}

\index{farce@farce!poissons braises b@---s spéciales pour les poissons braisés B}
\index{garniture@garniture!farce!farce poissons braises b@Farce spéciale pour les poissons braisés B}
\index{かるにちゆーる@ガルニチュール!ふあるす@ファルス!さかなのふれせb@魚のブレゼのガルニチュール用--- B}
\index{ふあるす@ファルス!さかなのふれせb@魚のブレゼのガルニチュール用--- B}

\begin{itemize}
\item
  材料\ldots{}\ldots{}白いパンの身200
  gを牛乳に浸して絞ったもの。玉ねぎ50 gとエシャロット25
  gを細かいみじん切りにしてバターで炒めたもの。ごく新鮮なマッシュルームをみじん切りにし、圧して余分な水分を絞ったもの。パセリのみじん切り大さじ1杯、叩き潰したにんにく1片、全卵1個、卵黄3個、塩8
  g、こしょう2 g、ナツメグごく少量。
\item
  作業手順\ldots{}\ldots{}ファルスAと同じ。
\end{itemize}
\end{recette}
\hypertarget{ux30afux30cdux30eb54}{%
\subsection[クネル]{\texorpdfstring{クネル\footnote{ローマ時代後期に成立した料理書アピキウスにも甲殻類やイカをはじ
  めとした各種素材のすり身を丸めて作るクネルとも呼ぶべきレシピが多く
  見られるように、とても古くからある調理だが、フランス語のquenelleと
  いう語それ自体は意外と新しく、18世紀頃に定着したと思われる。語源は
  ドイツ語の Knödel (クヌーデル)すなわちボール状にした食べものを意
  味する語からの移入と考えられている。荘厳で華麗な装飾を施した大掛か
  りな仕立てがとりわけ好まれた17、18世紀の宮廷料理においてその装飾の
  一部としてクネルの利用が広まり、発達したのだろう。また、\protect\hyperlink{godiveau}{ゴディ
  ヴォ}の訳注において触れたように、ピエール・ド・リュヌの
  アンドゥイエットなどは仔牛肉をすり潰したものを棒状にして豚背脂で包
  んで焼くという、まさしく本書におけるゴディヴォの調理法に近いもので
  あり、これもまた一種のクネルと言えるだろう。}}{クネル}}\label{ux30afux30cdux30eb54}}

\frsecb{Quenelles diverses}

クネルは大きさや形状がさまざま。

\begin{enumerate}
\def\labelenumi{\arabic{enumi}.}
\item
  粉を打った台の上で転がして小さな円筒形にする
\item
  絞り袋に詰めてバターを塗った天板に絞り出す
\item
  スプーンを使って整形する
\item
  指で丸めて、雄鶏のロニョン\footnote{ロニョンrognonは通常は腎臓のことだが、rognon
    de coq は精巣のこ と。高級食材として珍重された。}のような形状にする
\end{enumerate}

クネルの作り方のその他の詳細はよく知られていることだから、本書ではこれ
以上は述べないことにする。加熱方法についても同様としたい。

ただ、以下の点には留意していただきたい。\protect\hyperlink{garniture-financiere}{フィナンシエー
ル}や\protect\hyperlink{garniture-toulouse}{トゥールーズ}といっ
た標準的なガルニチュールに加えるクネルはコーヒースプーンを使って整形す
るか、丸口あるいは刻み模様が入る口金を使って絞り出すこと。

こうやって作る場合のクネルは平均で、ひとつ12〜15 g程度となる。

\protect\hyperlink{garniture-godard}{ガルニチュール・ゴダール}や\protect\hyperlink{garniture-regence}{レジャンス}、\protect\hyperlink{garniture-chambole}{シャ
ンボール}に使うような大きなクネルの場合は、必ずス
プーンを用いて整形し、20〜22 gの大きさにすること。

上記のような大がかりなガルニチュールでよく用いられる、装飾を施したクネ
ルの場合、大きさは40〜50 g、球形か卵形、あるいはやや長い卵形にすること。

装飾に用いる素材は、ほとんど常にトリュフ、\protect\hyperlink{saumure-liquide-pour-langues}{赤く漬けた舌
肉}のどちらか、あるいは両方を用いて、
生の卵白でクネルに貼り付けて固定する。

\protect\hyperlink{godiveau}{ゴディヴォ}のクネルは茹でずに低めの温度のオーブンで加熱し
ていいが、それ以外は1 Lあたり10 gの塩を加えた湯で、沸騰しない程度の温
度で茹でること。整形したクネルを並べたソテー鍋や天板に、沸騰した塩湯を
注ぎ、沸騰寸前の温度を保つようにして火を通すこと。
