\hypertarget{garniture}{%
\chapter{II. ガルニチュール Garnitures}\label{garniture}}

\index{garnitures@garnitures} \index{かるにちゆーる@ガルニチュール}

\vspace*{1.7\zw}

料理においてガルニチュール\footnote{garniture
  一般的には「付け合せ」と訳すが、本書におけるガルニチュー
  ルはたんなる料理の「付け合わせ」にとどまらず、こんにちではそれ自体
  がひとつの料理として成立し得るものも多い。そのため、あえて片仮名で
  ガルニチュールとした。なお、「付け合わせ」の意味で「ガルニ」または
  「ガロニ」などというスラングを用いる調理現場もある。}は重要なものだから、料理人は決してガルニ
チュールの役割を軽視してはいけない。ガルニチュールの構成をどうするかは、
添える料理の主素材との関係性で決まる。気まぐれ的なものや不自然なもの
は絶対にいけない。

ガルニチュールの構成要素は、場合によりけりだが、もっぱらどんな種類の料
理に添えるかで決まる。具体的には、野菜料理やパスタ、ファルスでさ
まざまな形状に作ったクネル\footnote{quenelle
  仔牛肉や鶏肉、豚肉などと獣脂をすり潰して、しばしば「つ
  なぎ」として後述のパナードを加えて練り、スプーンなどを用いて整形し、
  沸騰しない程度の温度で茹でる{[}ポシェ{]}またはオーブンで焼いたもの。
  スプーンを2つ使ってラグビーボールに似た形状にしたものが代表的だが、
  他にもいろいろな形状、大きさにする。}、あるいは雄鶏のとさかとロニョン\footnote{\protect\hyperlink{garniture-financiere}{ガルニチュール・フィナンシエール}やその
  バリエーションともいえる\protect\hyperlink{garniture-godard}{ガルニチュール・ゴダー
  ル}で必須の素材。ロニョンrognonは通常なら腎臓を
  意味するが、この場合のロニョンは rognon blanc ロニョンブラン(白い
  ロニョン)とも呼ばれるもので、雄鶏の精巣のこと。}、さ
まざまな種類の茸、オリーブとトリュフ、イカや貝および甲殻類、場合によっ
ては卵、小魚、牛や羊の副生物\footnote{正肉以外の部分。例えば内臓や骨髄など。Ris
  de vea(リドヴォー)仔牛胸腺肉などはこれに含まれる。}など。

その昔、ガルニチュールというのは、マトロットやコンポート、ブルゴーニュ
風料理などのように風味付けのために用いた素材がそのまま添えられたもので
あった。

ガルニチュールにする野菜は、どういう仕立ての皿にするかで役割が決まり、
それに合うように切って形状を整え、調理する。ただし、野菜の調理法は「野
菜料理」として調理する場合と同じだ。

パスタやイカ、貝類、甲殻類についても同様のことが言える。

この章では、それぞれのガルニチュールを構成する素材とその分量を示すに留
めるので、各素材の調理法ついてはその素材に対応する章を参照すること。

\hypertarget{ux30d5ux30a1ux30ebux30b9-5}{%
\section[ファルス ]{\texorpdfstring{ファルス \footnote{本来は「詰め物」の意で、鶏のローストの内臓を抜いた空洞部分に詰めたり、ガランティーヌやパテアンクルートの内部の詰め物などの用途に用いられる。この意味はこんにちでも変化がないが、本文にあるように、クネルにしてガルニチュールの一部にするなど、用途は多岐にわたる。本書ではファルスとして用いられるもののうち、肉および魚肉をベースにしたものをこの節にまとめて分類、説明している。したがって、ここでファルスとして挙げられていないファルスも料理によっては多い(例えば丸鶏の空洞部分に米などを詰めるのもファルス)ことに注意。}}{ファルス }}\label{ux30d5ux30a1ux30ebux30b9-5}}

\hypertarget{serie-des-farces-diverses}{%
\subsection{Série des farces diverses}\label{serie-des-farces-diverses}}

\index{garnitures@garnitures!farces@farces} \index{farce@farce}
\index{かるにちゆーる@ガルニチュール!ふあるす@ファルス}
\index{ふあるす@ファルス}

ガルニチュールの多くは、その構成要素にファルスあるいはファルスで作った
「クネル」が含まれている。ファルスはまた、多くの大きな仕立ての料理にも
使われる。ここではまずファルスの材料および作り方を示し、使い途について
は後で述べることにする。

ファルスは大きく5種に分類される。

\begin{enumerate}
\def\labelenumi{\arabic{enumi}.}
\item
  仔牛肉と脂で作るもの。すなわち古典料理における\textbf{ゴディヴォ}。
\item
  基本となる材料はさまざまだが、「つなぎ」に主としてパナードを使うもの。
\item
  近代的な手法で、生クリームを用いてふんわり泡立てたファルス。ムース、ムスリーヌに用いる。
\item
  レバーをベースとした「ファルス・\textbf{グラタン}」。種類はいろいろだが作り方は常に同じ。
\item
  \ruby{主}{おも}に\protect\hyperlink{}{ガランティーヌ}、\protect\hyperlink{}{パテアンクルート}、\protect\hyperlink{}{テリーヌ}などの冷製料理に用いるシンプルなファルス。
\end{enumerate}

\hypertarget{ux30d5ux30a1ux30ebux30b9ux7528ux306eux30d1ux30caux30fcux30c9ux306bux3064ux3044ux30666}{%
\subsection[ファルス用のパナードについて]{\texorpdfstring{ファルス用のパナードについて\footnote{パナードは本来、パンと水、バターを弱火で時間をかけて煮た粥のようなものを意味した。本書ではその意味を拡大して肉や魚肉をベースとしたファルスを加熱する際に崩れないようにする「つなぎ」として、この語を用いている。そのため、必ずしもパンを材料としていないものが含まれている。}}{ファルス用のパナードについて}}\label{ux30d5ux30a1ux30ebux30b9ux7528ux306eux30d1ux30caux30fcux30c9ux306bux3064ux3044ux30666}}

\vspace*{-1.7\zw}

\hypertarget{les-panades-pour-farces}{%
\subsection{Les Panades pour Farces}\label{les-panades-pour-farces}}

\index{garnitures@garnitures!farces@farces}
\index{farce@farce!panade@les panades pour farces}
\index{かるにちゆーる@ガルニチュール!ふあるす@ファルス!はなーと@パナード}
\index{ふあるす@ファルス!はなーと@---用パナード}

ファルスに用いるパナードにはいくつもの種類がある。ファルスの種類や、そ
のファルスを添える料理の性質によって使い分けることとなる。

原則として、パナードの分量は、ファルスのベースとする素材が何であれ、そ
の半量を越えないようにすること。

卵とバターを用いるパナードの場合はレシピの分量どおりに作らなければなら
ないから、それを合わせて作るファルスの全体量のほうを調節してやること。

パナードE以外のパナードは使用する際には必ず完全に冷めた状態になってい
ること。パナードが出来上がったら、バターを塗った平皿か天板に流し広げ、
早く冷めるようにする。このとき、バターを塗った紙で蓋をするか、表面にバ
ターのかけらをいくつか置いてやり、パナードが直接空気に触れないようにし
てやること。

以下のパナードのレシピは仕上がり重量が正味500
gになるように調整してある。

したがって、必要な量のパナードを作るのに材料を増やしたり減らしたりする
のも難しくはないだろう\footnote{原文では、Rien de plus simple, donc, que
  \ldots{}
  となっており、直訳すると「これ以上に簡単なことはない」と言いきっているが、都度計算しなければならないことに変わりはないので、多少ニュアンスを柔らげて訳した。}。

\hypertarget{ux30d1ux30caux30fcux30c9}{%
\subsection{パナード}\label{ux30d1ux30caux30fcux30c9}}

\vspace*{-1.7\zw}

\hypertarget{panades}{%
\subsection{Panades}\label{panades}}

\index{panade} \index{garniture!panade} \index{garniture!farce!panade}
\index{かるにちゆーる@ガルニチュール!ふあるす@ファルス!はなーと@パナード}
\index{はなーと@パナード}
\begin{recette}
\hypertarget{a.-ux30d1ux30f3ux306eux30d1ux30caux30fcux30c9}{%
\subsubsection{A.
パンのパナード}\label{a.-ux30d1ux30f3ux306eux30d1ux30caux30fcux30c9}}

\hypertarget{panade-a}{%
\paragraph{Panade au pain}\label{panade-a}}

\index{garnitures@garnitures!farces@farces!panade a@panade A}
\index{farce@farce!panade@les panades pour farces!panade a@panade A}
\index{panade!a pain@A. --- au pain}
\index{かるにちゆーる@ガルニチュール!ふあるす@ファルス!はなーとa@パナードA. パンの---}
\index{ふあるす@ファルス!はなーと@---用パナード!a@A. パンのパナード}
\index{はなーと@パナード!a@A. パンの---}

\ldots{}\ldots{}\textbf{魚を素材にした固めのファルス用}

\begin{itemize}
\item
  \textbf{材料}\ldots{}\ldots{}沸かした牛乳3
  dl、固くなった白パン\footnote{ここでいわゆるバゲットのようなパンの外側を削り落した白い部分、あ
    るいは食パンの「耳」を切り落した白い部分を使う、ということ。なお、
    パンは使う小麦粉の精白度や種類によって、pain complet (パンコンプ
    レ)全粒粉パン、pain de sègle(パンドセーグル)ライ麦パン、一般的
    な小麦粉と食塩、塩、パン種だけで作るバゲットなどの pain と、バター
    や砂糖を加えて作るヴィエノワズリ(クロワッサンやパンオショコラ、ブ
    リオシュなど)に分けられる。イギリスやアメリカのいわゆる食パン(フ
    ランス語 pain de mie パンドミ)は小麦粉、バター、塩、イースト菌、
    牛乳などで作られている。また、現代フランスでバゲットなどのパンに用
    いられている小麦粉の精白度は、T-55と呼ばれる灰分(小麦粉を燃やした
    際に残る炭水化物およびタンパク質以外の要素)0.5〜0.6%のものが主流
    であり、いわゆる食パンpain de mie(パンドミ)やヴィエノワズリには
    T-45(灰分0.5%以下)が多く用いられている。このほかT-65(灰分0.62〜
    0.75%)およびT-80(灰分0.75〜0.9%)、T-110(灰分1.0〜1.2%)、
    T-150(灰分1.4%前後、いわゆる全粒粉)のように種類がある。このうち
    T-45およびT-55はfarine blanche(ファリーヌブロンシュ)と呼ばれ、
    T-150はfarine complète(ファリーヌコンプレット)と通称されている。
    灰分が高くなればそれだけ不純物が多いわけだから、粉は薄い茶色あるい
    はグレーがかった色合いになり、パンを焼く場合などはグルテン形成が難
    しくなりやすい。ただし、香りゆたかなパンを実現しやすいという面もあ
    る。結果として、例えば全粒粉パンは香りはいいが固い仕上がりになる。か
    つては精白度の低い(すなわち灰分の多い)粉ほど重量あたりの価格が安
    く、パンの価格もそれに比例していた。また、本書では基本的に小麦粉を
    使う場合にその精白度についての指示はないが、概ねT-55またはT-45相当
    のもの考えていいだろう。なお、日本に輸入されている小麦は北米産のも
    のがほとんどで、硬質小麦を粉にしたものが「強力粉」、軟質小麦の場合
    は「薄力粉」と呼ばれ、精白度合いによる分類は通常なされていないが、
    製品としては概ねT-45相当あるいはそれ以上の精白度のものが多い。}の身250
  g、塩5 g。
\item
  \textbf{作業手順}\ldots{}\ldots{}パンの身を牛乳に浸して完全にもどす。強火にかけて、ペー
  スト状になったパンがヘラから簡単に取れるくらいまで水気をとばす。バター
  を塗った平皿か天板に広げ、冷ます。
\end{itemize}

\maeaki

\hypertarget{b.-ux5c0fux9ea6ux7c89ux306eux30d1ux30caux30fcux30c9}{%
\subsubsection{B.
小麦粉のパナード}\label{b.-ux5c0fux9ea6ux7c89ux306eux30d1ux30caux30fcux30c9}}

\hypertarget{panade-b}{%
\paragraph{Panade à la farine}\label{panade-b}}

\index{garnitures@garnitures!farces@farces!panade b@panade B}
\index{farce@farce!panade@les panades pour farces!panade b@panade B}
\index{panade!b farine@B. --- à la farine}
\index{かるにちゆーる@ガルニチュール!ふあるす@ファルス!はなーとb@パナードB. 小麦粉の---}
\index{ふあるす@ファルス!はなーと@---用パナード!b@パナードB. 小麦粉の---}
\index{はなーと@パナード!b@B. 小麦粉の---}

\ldots{}\ldots{}\textbf{肉、魚などあらゆるファルスに用いられる}

\begin{itemize}
\item
  \textbf{材料}\ldots{}\ldots{}水3 dl、塩2 g、バター50
  g、篩にかけた小麦粉150 g。
\item
  \textbf{作業手順}\ldots{}\ldots{}片手鍋に水、塩、バターを入れて火にかけ、沸騰させる。
  火から外して小麦粉を加えて混ぜる。再度火にかけて、\protect\hyperlink{}{シュー生地}を
  作る要領で余計な水分をとばす。上記パナードAと同様にして冷ます。
\end{itemize}

\maeaki

\hypertarget{c.-ux30d1ux30caux30fcux30c9ux30d5ux30e9ux30f3ux30b8ux30d1ux30fcux30cc12}{%
\subsubsection[C. パナード・フランジパーヌ]{\texorpdfstring{C.
パナード・フランジパーヌ\footnote{フランジパーヌとは製菓で用いられる、小麦粉、砂糖、卵を混ぜて牛
  乳とバニラを加えて煮、砕いたマカロンmacaronを加えたクリーム。なお、
  これに用いられるマカロンは、現代日本でよく知られているタイプとは違
  い、すり潰したアーモンドと卵白、砂糖を混ぜた生地を紙の上にクルミ大
  に絞り出してオーブンで焼いただけもの。macaron craquelé(マカロンク
  ラクレ)はこのタイプの代表的なもので、焼く際に膨らんで割れ目が出来
  ることからクラクレの名称が付けられた。ところで、日本にマカロンが伝
  わった時期は判然としないが、このタイプのものが太平洋戦争前には、アー
  モンドを落花生に代え、「まころん」の名称でいくつかの製菓会社で製造
  されるようになり、現在も生産されている。}}{C. パナード・フランジパーヌ}}\label{c.-ux30d1ux30caux30fcux30c9ux30d5ux30e9ux30f3ux30b8ux30d1ux30fcux30cc12}}

\hypertarget{panade-c}{%
\paragraph{Panade à la Frangipane}\label{panade-c}}

\index{garnitures@garnitures!farces@farces!panade c@panade C}
\index{farce@farce!panade@les panades pour farces!panade c@panade C}
\index{panade!c frangipane@C. --- à la Frangipane}
\index{かるにちゆーる@ガルニチュール!ふあるす@ファルス!はなーとc@パナードC}
\index{ふあるす@ファルス!はなーと@---用パナード!はなーとC@パナードC}
\index{はなーと@パナード!c@C. ---・フランジパーヌ}

\ldots{}\ldots{}\textbf{鶏のファルス、魚のファルス用}

\begin{itemize}
\item
  \textbf{材料}\ldots{}\ldots{}小麦粉125 g、卵黄4個、溶かしバター90
  g、塩2 g、こしょう1 g、おろしたナツメグの粉ごく少量、牛乳2\undemi{}
  dl。
\item
  \textbf{作業手順}\ldots{}\ldots{}片手鍋に小麦粉と卵黄を入れてよく練る。溶かしバター、
  塩、こしょう、ナツメグを加える。沸かした牛乳で少しずつ溶きのばしてい
  く。
\end{itemize}

\protect\hyperlink{}{標準的なフランジパーヌ}と同様に、火にかけて5〜6分間、泡立て器で混
ぜながら煮る。ちょうどいい漉さになったら、バットに移して\footnote{débarasser
  (デバラセ)バットなどに移す、片付ける、の意。とりわけ前者の意味に注意。}冷ます。

\maeaki

\hypertarget{d.-ux7c73ux306eux30d1ux30caux30fcux30c9}{%
\subsubsection{D.
米のパナード}\label{d.-ux7c73ux306eux30d1ux30caux30fcux30c9}}

\hypertarget{panade-d}{%
\paragraph{Panade au Riz}\label{panade-d}}

\index{garnitures@garnitures!farces@farces!panade d@panade D}
\index{farce@farce!panade@les panades pour farces!panade d@panade D}
\index{panade!d riz@D. --- au Riz}
\index{かるにちゆーる@ガルニチュール!ふあるす@ファルス!はなーとd@パナードD. 米の---}
\index{ふあるす@ファルス!はなーと@---用パナード!はなーとd@D. 米のパナード}
\index{はなーと@パナード!d@D. 米の---}

\ldots{}\ldots{}\textbf{いろいろなファルスに用いられる}

\begin{itemize}
\item
  \textbf{材料}\ldots{}\ldots{}米200 gすなわち2
  dlあるいは大さじ8杯。\protect\hyperlink{}{白いコンソメ}6 dl、バター20
  g。
\item
  \textbf{作業手順}\ldots{}\ldots{}米を入れた鍋にコンソメを注ぎ、バターを加える。火にかけて沸騰させたら、オーブンに入れて40〜45分間加熱する。この間、米に触れないようにすること。
\end{itemize}

オーブンから出したら、米粒がよく潰れるようにヘラでしっかりと混ぜる。その後、冷ます。

\maeaki

\hypertarget{e.-ux3058ux3083ux304cux3044ux3082ux306eux30d1ux30caux30fcux30c9}{%
\subsubsection{E.
じゃがいものパナード}\label{e.-ux3058ux3083ux304cux3044ux3082ux306eux30d1ux30caux30fcux30c9}}

\hypertarget{panade-e}{%
\paragraph{Panade à la pomme de terre}\label{panade-e}}

\index{garnitures@garnitures!farces@farces!panade e@panade E}
\index{farce@farce!panade@les panades pour farces!panade e@panade E}
\index{panade!e riz@E. --- à la pomme de terre}
\index{かるにちゆーる@ガルニチュール!ふあるす@ファルス!はなーとe@パナードE}
\index{ふあるす@ファルス!はなーと@---用パナード!はなーとe@パナードE}
\index{はなーと@パナード!e@E. じゃがいもの---}

\ldots{}\ldots{}\textbf{仔牛および他の白身肉の、詰め物\footnote{fourrré
  (フレ)詰め物をした。farci (ファルシ)も同様に「詰め
  物をした」の意だが、後者はより一般的で、前者はオムレツやクレープに
  中身を詰めて「包む」のが本来の意味。すなわち、このパナードを加えた
  ファルスで、何らかの素材を「包む」と解釈してもいい。}をする大きなクネルに用いられる}

\begin{itemize}
\item
  \textbf{材料}\ldots{}\ldots{}茹でて皮を剥いたばかりの中位のサイズのじゃがいも2個、牛
  乳3 dl、塩 2g、白こしょう\undemi{} g、ナツメグ少々、バター20 g。
\item
  \textbf{作業手順}\ldots{}\ldots{}牛乳を2.5
  dlになるまで煮詰める\footnote{原文は réduire le lait d'un sixième
    直訳すると「牛乳を
    \unsixieme{}量だけ煮詰める」すなわち「\cinqsixiemes{}量まで煮詰め
    る」のだが、かえって分かりにくいだろうから、ここでは具体的な数字に
    直して訳した。分量を代えて作る場合には85%まで煮詰めるくらいと考え
    てもいいだろう。そもそも、じゃがいもの重さが曖昧なのだから、あまり
    細かい数字にこだわらず臨機応変に考えること。}。バター、調味料、
  薄く輪切りにしたじゃがいもを加え、15分間程加熱する。
\end{itemize}

このパナードはまだ少し\ruby{微温}{ぬる}いくらいで使用すること。完全に
冷めてからではいけない。完全に冷めてから練ると粘りが出てしまうからだ。
\end{recette}
\hypertarget{ux30d5ux30a1ux30ebux30b9}{%
\subsection{ファルス}\label{ux30d5ux30a1ux30ebux30b9}}

\vspace*{-1.7\zw}

\hypertarget{farces}{%
\subsection{Farces}\label{farces}}

\index{farce} \index{garniture!farce}
\index{かるにちゆーる@ガルニチュール!ふあるす@ファルス}
\index{ふあるす@ファルス}

ベースとなる素材が\textbf{仔牛}、\textbf{鶏}、\textbf{ジビエ}あるいは\textbf{甲殻類}であっても、分量と作
業手順はどんなファルスでも同じだ。そのベースにする素材を代えればいいの
だから、ここでは各種ファルスの典型的なレシピを示せば充分だろう。料理で
用いられるファルスひとつひとつを説明するのに一章をあてる必要はないと思
われる。
\begin{recette}
\hypertarget{a.-ux30d1ux30caux30fcux30c9ux3068ux30d0ux30bfux30fcux3092ux7528ux3044ux308bux30d5ux30a1ux30ebux30b9}{%
\subsubsection{A.
パナードとバターを用いるファルス}\label{a.-ux30d1ux30caux30fcux30c9ux3068ux30d0ux30bfux30fcux3092ux7528ux3044ux308bux30d5ux30a1ux30ebux30b9}}

\hypertarget{farce-uxe0-la-panade-et-au-beurre}{%
\paragraph{Farce à la Panade et au
beurre}\label{farce-uxe0-la-panade-et-au-beurre}}

\index{farce!a@A. --- à la Panade et au beurre}
\index{garniture!farce!a@A. Farce à la Panade et au beurre}
\index{かるにちゆーる@ガルニチュール!ふあるす@ファルス!a@A. パナードとバターを用いるファルス}
\index{ふあるす@ファルス!a@A. パナードとバターを用いる---}

(標準的なクネル、肉料理\footnote{原文 entrée
  (アントレ)、現代では「前菜」の意味で用いられるが、
  本書では概ね10人前を一皿に盛ったものを指し、現代では立派にメインの
  料理として通用するものが多くある。}の縁飾り etc.)

\begin{itemize}
\item
  \textbf{材料}\ldots{}\ldots{}ていねいに筋取りをした肉1
  kg、\protect\hyperlink{panade-b}{パナードB} 500 g、塩12 g、こしょう2
  g、全卵4個、卵黄8個。
\item
  \textbf{作業手順}\ldots{}\ldots{}肉をさいの目に切って鉢に入れ、調味料を加えてすり潰す。
  いったん肉を取り出して、パナードをよくすり潰しながらバターを加える。
  肉を戻し入れ、すりこ木で力強く練って全体をまとめる。
\end{itemize}

次に全卵と卵黄を加えて混ぜ合わせる。これは2回に分けても1回でやってもい
い。裏漉しして陶製の容器に入れる。さらに泡立て器で滑かになるまで混ぜる。

\hypertarget{ux539fux6ce8}{%
\subparagraph{【原注】}\label{ux539fux6ce8}}

どんな種類のファルスを作る場合でも、必ず少量を沸騰しない程度の温度で茹
でて\footnote{pocher (ポシェ)。}テストしてから、クネルの整形に取りかかること。
\end{recette}