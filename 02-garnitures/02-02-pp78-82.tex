\hypertarget{serie-des-appareiles-et-preparations-diverses-pour-garnitures-chaudes}{%
\section[温製ガルニチュール用アパレイユなど]{\texorpdfstring{温製ガルニチュール用アパレイユ\footnote{料理用語としての
  appareil アパレイユとは、具体的な何かを指す言葉
  ではなく、\textbf{ある料理を作る過程において用いられる、複数の材料を組み
  合わせたもの}、という一種の概念。現実には、キッシュのアパレイユ
  (生クリームと卵、塩漬け豚バラ肉など)、クレーム・ブリュレのアパレ
  イユ(卵黄、砂糖、生クリーム、牛乳)というように用いられるが、概ね、
  加熱して凝固する液体または半液状のもの、およびそれらを「つなぎ」と
  して固形物をあえたものを指す、と考えていい。アパレイユの概念として
  は、まったくの固形物である、3〜4 mmのさいの目に切った香味野菜(場
  合によってはハムも入る)である\protect\hyperlink{matignon}{マティニョン}も
  appareil à matignon
  と表現されることはフランスの料理書においては珍しくない
  し、本節の\protect\hyperlink{duxelles-seche}{デュクセル・セッシュ}もまたアパレイユ
  の一種に含められる。実際のところ、アパレイユという語はそれぞれの調
  理現場および料理人によって使い方がさまざまであり、概念としての理解
  も必ずしも共通しているとは限らない。本書では基本的に、上述のように
  加熱凝固する液体の場合と、半固形状あるいはクリーム状のものを指す場
  合がほとんど。この節のタイトルを直訳すると「温製ガルニチュール用の
  アパレイユおよびその他の仕込み」となるが、概念のレベルでいえば、こ
  の節に収められているレシピはほぼ全て「アパレイユ」と呼び得るものに
  他ならない。ただ、そう言い切ってしまうと現実問題として理解出来ない
  であろうことを想定したのか、やや曖昧な表現になっているのだと思われ
  る。また、既出の\protect\hyperlink{sauce-villeroy}{ソース・ヴィルロワ}などもまた、
  ソースというよりはむしろアパレイユと呼んでおかしくないものと言える。}など}{温製ガルニチュール用アパレイユなど}}\label{serie-des-appareiles-et-preparations-diverses-pour-garnitures-chaudes}}

\frsec{Série des Appareils et Préparations diverses pour Garnitures chaudes}

\index{garniture@garniture!appareils garnitures chaudes@appareils et préparations diverses pour garnitures chaudes}
\index{appareil@appareil!garnitures chaudes@--- et préparations diverses pour garnitures chaudes}
\index{かるにちゆーる@ガルニチュール!あはれいゆおんせい@温製ガルニチュールのためのアパレイユなど}
\index{あはれいゆ@アパレイユ!おんせいかるにちゆーる@温製ガルニチュールのための---など}
\begin{recette}
\hypertarget{appareils-a-cromesquis-et-a-croquettes}{%
\subsubsection{クロメスキとクロケットのアパレイユ}\label{appareils-a-cromesquis-et-a-croquettes}}

\frsub{Appareils à Cromesquis et à Croquettes}\footnote{クロケットは日本のコロッケの原型となったもので、細かく切った素材をじゃがいものピュレや\protect\hyperlink{sauce-bechamel}{ベシャメルソース}であえて円盤または円筒形に整形してパン粉衣を付けて揚げたもの。クロメスキは正六面体(サイコロ形)にすることが多く、コロッケとアパレイユが共通のため、形状が違うだけでクロケットのバリエーションという見方もあるが、ポーランド語のkromesk(薄く切ったもの)が語源とされる。}

\index{garniture@garniture!appareil@appareil!cromesquis croquettes@appareils à cromesquis et à croqeuttes}
\index{appareil@appareil!cromesquis croquettes@---s à cromesquis et à croquettes}
\index{cromesqui@cromesqui!appareil@appareils à --- et à croquettes}
\index{croquette@croquette!appareil@appareils à cromesquis et à ---}
\index{かるにちゆーる@ガルニチュール!あはれいゆ@アパレイユ!くろめすきとくろけつと@クロケットとクロメスキのアパレイユ}
\index{あはれいゆ@アパレイユ!くろめすきとくろけつと@クロケットとクロメスキの---}
\index{くろめすき@クロメスキ!あはれいゆ@---とクロケットのアパレイユ}
\index{くろけつと@クロケット!あはれいゆ@クロメスキと---のアパレイユ}

⇒ \protect\hyperlink{hors-d-oeuvres-chauds}{温製オードブル}の章を参照。

\hypertarget{appareils-a-pomme-dauphine-duchesse-marquise}{%
\subsubsection{じゃがいものドフィーヌ、デュシェス、マルキーズのアパレイユ}\label{appareils-a-pomme-dauphine-duchesse-marquise}}

\frsub{Appareils à pomme Dauphine, Duchesse et Marquise}\footnote{dauphin(王太子)、dauphine(王太子妃)、duc(公爵)、
  duchesse(公爵夫人)、mariquis(侯爵)、marquise(侯爵夫人)。いず
  れも王家、貴族の位階(爵位)を表わす語だが、特に理由もなく料理名に
  付けられることが非常に多い。}

\index{garniture@garniture!appareil@appareil!pomme dauphine@appareils à pomme Dauphine, Duchesse et Marquise}
\index{appareil@appareil!pomme dauphine@---s à pomme Dauphine, Duchesse et Marquise}
\index{dauphin@dauphin(e)!appareil@appareil à pomme ---e}
\index{duc@duc / duchesse!appareil@appareil à pomme duchesse}
\index{marquis@marquis(s)!appareil@appareil à pomme ---e}
\index{かるにちゆーる@ガルニチュール!あはれいゆ@アパレイユ!しやかいものとふいーぬ@じゃがいものドフィーヌ、デュシェス、マルキーズのアパレイユ}
\index{あはれいゆ@アパレイユ!しやかいものとふいーぬと@じゃがいものドフィーヌ、デュシェス、マルキーズの---}
\index{とふいーぬ@ドフィーヌ!あはれいゆ@アパレイユ!しやかいものとふいーぬ@じゃがいもの---、デュシェス、マルキーズのアパレイユ}
\index{てゆしえす@デュシェス!あはれいゆ@アパレイユ!しやかいものてゆしえす@じゃがいものドフィーヌ、---、マルキーズのアパレイユ}
\index{まるきーす@マルキーズ!あはれいゆ@アパレイユ!しやかいものまるきーす@じゃがいものドフィーヌ、デュシェス、---のアパレイユ}

⇒ \protect\hyperlink{legumes}{野菜料理}の章、\protect\hyperlink{pommes-de-terre}{じゃがいも}の項を参照。

\hypertarget{appareil-maintenon}{%
\subsubsection{アパレイユ・マントノン}\label{appareil-maintenon}}

\frsub{Appareils Maintenon}\footnote{マントノン夫人(出生名フランソワーズ・ドビニェ
  1635〜1719)。は
  じめはマントノン侯爵夫人としてルイ14世とモンテスパン夫人の間に生ま
  れた子どもたちの非公式な教育係となり、モンテスパン夫人の死後、ルイ
  14世と結婚した。彼女の名を冠した料理はここで言及されている\protect\hyperlink{cotelettes-maintenon}{羊のコ
  トレット マントノン}の他、卵料理、菓子など
  にある。「羊のコトレット マントノン」は彼女自身が考案したとも、ル
  イ14世付の料理人の考案ともいわれているが、いずれも憶測の域を出ない。
  なお、côtelette(コトレット)とは仔牛、羊の背肉を骨付きで肋骨1本ず
  つに切り分けたもの。日本語では、仔羊の場合ラムチョップと呼ばれるこ
  とも多い。}

\index{garniture@garniture!appareil@appareil!maintenon@appareil Maintenon}
\index{appareil@appareil!maintenon@--- Maintenon}
\index{maintenon@Maintenon!appareil@appareil ---}
\index{かるにちゆーる@ガルニチュール!あはれいゆ@アパレイユ!まんとのん@アパレイユ・マントノン}
\index{あはれいゆ@アパレイユ!まんとのん@---・マントノン}
\index{まんとのん@マントノン!あはれいゆ@アパレイユ・---}

(\protect\hyperlink{cotelettes-maintenon}{羊のコトレット マントノン}用)

\protect\hyperlink{sauce-bechamel}{ベシャメルソース}4
dlと\protect\hyperlink{sauce-soubise}{スビーズ}1
dlを半量になるまで煮詰める。

卵黄3個を加えてとろみを付ける。あらかじめマッシュルーム100 gを薄切りに
してバターでごく弱火で鍋に蓋をして蒸し煮\footnote{étuver
  (エチュヴェ)。}したものを加える。

\hypertarget{appareil-montglas}{%
\subsubsection{アパレイユ・モングラ}\label{appareil-montglas}}

\frsub{Appareils à la Montglas}\footnote{Salpicon à la
  Monglas(サルピコンアラモングラ)とも呼ばれものと
  ほぼ同じ。サルピコンはせいぜい5 mm角くらいの小さなさいの目に切った
  もののこと。羊のコトレット モングラ以外の用途としては、ブシェ(パ
  イ生地で作ったケースに詰め物をしたもの。本書ではオードブルに分類さ
  れている)やタルトレット(小さなタルト)の\textbf{アパレイユ}にする。17
  世紀のモングラ侯爵 François Clermont Marquis de Montglas (生年不
  詳〜1675)の名を冠したものらしいが、由来などは不明。}

\index{garniture@garniture!appareil@appareil!montglas@appareil à la Montglas}
\index{appareil@appareil!montglas@--- à l Montglas}
\index{montglas@Montglas!appareil@appareil à la ---}
\index{かるにちゆーる@ガルニチュール!あはれいゆ@アパレイユ!もんくら@アパレイユ・モングラ}
\index{あはれいゆ@アパレイユ!もんくら@---・モングラ}
\index{もんくら@モングラ!あはれいゆ@アパレイユ・---}

(\protect\hyperlink{cotelettes-mongras}{羊のコトレット モングラ}その他に用いられる)

以下の材料を通常より太めで短かい千切り\footnote{julienne
  (ジュリエーヌ)。}にする。\protect\hyperlink{saumure-liquide-pour-langue}{赤く漬けた舌
肉}150 g、フォワグラ150 g、茹でたマッシュ ルーム100 g、トリュフ100 g。

これらを、マデイラ酒風味の充分に煮詰めた\protect\hyperlink{sauce-demi-glace}{ソース・ドゥミグラ
ス}2\undemi{} dlであえる。バターを塗った平皿に広げ、
使うまでそのまま冷ましておく。

\hypertarget{appareil-provencal}{%
\subsubsection{プロヴァンス風アパレイユ}\label{appareil-provencal}}

\frsub{Appareils à la Provençale}

\index{garniture@garniture!appareil@appareil!provencal@appareil à la provençale}
\index{appareil@appareil!provencale@--- à la Provençale}
\index{provençal@provençale(e)!appareil@appareil à la ---e}
\index{かるにちゆーる@ガルニチュール!あはれいゆ@アパレイユ!ふろうあんすふう@プロヴァンス風アパレイユ}
\index{あはれいゆ@アパレイユ!ふろうあんすふう@プロヴァンス風---}
\index{ふろうあんすふう@プロヴァンス風!あはれいゆ@---アパレイユ}

(\protect\hyperlink{cotelettes-provencale}{羊のコトレット プロヴァンス風}用)\footnote{\protect\hyperlink{appareil-maintenon}{アパレイユ・マントノン}からここまでの3種の
  アパレイユはいずれも、羊のコトレット(ラムチョップ)の片面だけを焼
  いて、その表面をよく拭い、まだ焼いていない面を下にして、焼いた側の
  面にこれらのアパレイユを塗る、あるいは盛り上げてからオーブンに入れ
  るという同工異曲とも言うべき手法をとっている。ここで、アパレイン・
  マントノンとこのプロヴァンス風アパレイユの「用途」の部分の原文には
  動詞farcirあるいはその過去分詞farci(es)が用いられているのはとても
  興味深いと言えよう。farcirを日本語の「詰め物をする」と等価と考えて
  はうまく理解できない例のひとつだろう。farcirの原義は「ファルスで満
  たす」であって、中に詰めることではない。なお、\href{http://cnrtl.fr/definition/farce}{TLFiによるファルス
  の定義}は、「肉などと他の材料
  (香草や茸、細かく刻んだマロンなど)を混ぜ合わせ、スパイスを加えた
  りして、一般的にはソースや卵、パナードでつないだもの。これを牛や羊
  の肉や家禽あるいは魚や野菜に、加熱前に加えて使用する」となっている。
  すなわち、詰めることは詰めるけれども、必ずしも空洞になっている部分
  というわけではないというのが言葉のうえでの意味。}

\protect\hyperlink{sauce-soubise}{ソース・スビーズ}5
dlを充分に固くなるまで煮詰める。潰
したにんにく1片を加え、卵黄3個を加えてとろみを付ける。
\end{recette}