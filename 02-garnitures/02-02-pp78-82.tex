\hypertarget{serie-des-appareiles-et-preparations-diverses-pour-garnitures-chaudes}{%
\section[温製ガルニチュール用アパレイユなど]{\texorpdfstring{温製ガルニチュール用アパレイユ\footnote{料理用語としての
  appareil アパレイユとは、具体的な何かを指す言葉
  ではなく、\textbf{ある料理を作る過程において用いられる、複数の材料を組み
  合わせたもの}、という一種の概念。現実には、キッシュのアパレイユ
  (生クリームと卵、塩漬け豚バラ肉など)、クレーム・ブリュレのアパレ
  イユ(卵黄、砂糖、生クリーム、牛乳)というように用いられるが、概ね、
  \textbf{加熱して凝固する液体または半液状のもの、およびそれらを「つなぎ」
  として固形物をあえたものを指す}、と考えていい。アパレイユの概念と
  しては、まったくの固形物である、3〜4 mmのさいの目に切った香味野菜
  (場合によってはハムも入る)である\protect\hyperlink{matignon}{マティニョン}も
  appareil à matignon と表現されることはフランスの料理書においては珍
  しくないし、本節の\protect\hyperlink{duxelles-seche}{デュクセル・セッシュ}もまたア
  パレイユの一種に含められる。実際のところ、アパレイユという語はそれ
  ぞれの調理現場および料理人によって使い方がさまざまであり、概念とし
  ての理解も必ずしも共通しているとは限らない。本書では基本的に、上述
  のように加熱凝固する液体の場合と、半固形状あるいはクリーム状のもの
  を指す場合がほとんど。この節のタイトルを直訳すると「温製ガルニチュー
  ル用のアパレイユおよびその他の仕込み」となるが、概念のレベルでいえ
  ば、この節に収められているレシピのうちかなりの数が「アパレイユ」と
  呼び得るものに他ならない。ただ、そう言い切ってしまうと現実問題とし
  て理解出来ないであろうことを想定したのか、やや曖昧な表現になってい
  るのだと思われる。また、既出の\protect\hyperlink{sauce-villeroy}{ソース・ヴィルロワ}な
  どもまた、ソースというよりはむしろアパレイユと呼んでおかしくないも
  のと言える。}など}{温製ガルニチュール用アパレイユなど}}\label{serie-des-appareiles-et-preparations-diverses-pour-garnitures-chaudes}}

\frsec{Série des Appareils et Préparations diverses pour Garnitures chaudes}

\index{garniture@garniture!appareils garnitures chaudes@appareils et préparations diverses pour garnitures chaudes}
\index{appareil@appareil!garnitures chaudes@--- et préparations diverses pour garnitures chaudes}
\index{かるにちゆーる@ガルニチュール!あはれいゆおんせい@温製ガルニチュールのためのアパレイユなど}
\index{あはれいゆ@アパレイユ!おんせいかるにちゆーる@温製ガルニチュールのための---など}
\begin{recette}
\hypertarget{appareils-a-cromesquis-et-a-croquettes}{%
\subsubsection{クロメスキとクロケットのアパレイユ}\label{appareils-a-cromesquis-et-a-croquettes}}

\frsub{Appareils à Cromesquis et à Croquettes}\footnote{クロケットは日本のコロッケの原型となったもので、細かく切った素材をじゃがいものピュレや\protect\hyperlink{sauce-bechamel}{ベシャメルソース}であえて円盤または円筒形に整形してパン粉衣を付けて揚げたもの。クロメスキは正六面体(サイコロ形)にすることが多く、コロッケとアパレイユが共通のため、形状が違うだけでクロケットのバリエーションという見方もあるが、ポーランド語のkromesk(薄く切ったもの)が語源とされる。}

\index{garniture@garniture!appareil@appareil!cromesquis croquettes@appareils à cromesquis et à croqeuttes}
\index{appareil@appareil!cromesquis croquettes@---s à cromesquis et à croquettes}
\index{cromesqui@cromesqui!appareil@appareils à --- et à croquettes}
\index{croquette@croquette!appareil@appareils à cromesquis et à ---}
\index{かるにちゆーる@ガルニチュール!あはれいゆ@アパレイユ!くろめすきとくろけつと@クロケットとクロメスキのアパレイユ}
\index{あはれいゆ@アパレイユ!くろめすきとくろけつと@クロケットとクロメスキの---}
\index{くろめすき@クロメスキ!あはれいゆ@---とクロケットのアパレイユ}
\index{くろけつと@クロケット!あはれいゆ@クロメスキと---のアパレイユ}

⇒ \protect\hyperlink{hors-d-oeuvres-chauds}{温製オードブル}の章を参照。

\maeaki

\hypertarget{appareils-a-pomme-dauphine-duchesse-marquise}{%
\subsubsection{じゃがいものドフィーヌ、デュシェス、マルキーズのアパレイユ}\label{appareils-a-pomme-dauphine-duchesse-marquise}}

\frsub{Appareils à pomme Dauphine, Duchesse et Marquise}\footnote{dauphin(王太子)、dauphine(王太子妃)、duc(公爵)、
  duchesse(公爵夫人)、mariquis(侯爵)、marquise(侯爵夫人)。いず
  れも王家、貴族の位階(爵位)を表わす語だが、特に理由もなく料理名に
  付けられることが非常に多い。}

\index{garniture@garniture!appareil@appareil!pomme dauphine@appareils à pomme Dauphine, Duchesse et Marquise}
\index{appareil@appareil!pomme dauphine@---s à pomme Dauphine, Duchesse et Marquise}
\index{dauphin@dauphin(e)!appareil@appareil à pomme ---e}
\index{duc@duc / duchesse!appareil@appareil à pomme duchesse}
\index{marquis@marquis(s)!appareil@appareil à pomme ---e}
\index{かるにちゆーる@ガルニチュール!あはれいゆ@アパレイユ!しやかいものとふいーぬ@じゃがいものドフィーヌ、デュシェス、マルキーズのアパレイユ}
\index{あはれいゆ@アパレイユ!しやかいものとふいーぬと@じゃがいものドフィーヌ、デュシェス、マルキーズの---}
\index{とふいーぬ@ドフィーヌ!あはれいゆ@アパレイユ!しやかいものとふいーぬ@じゃがいもの---、デュシェス、マルキーズのアパレイユ}
\index{てゆしえす@デュシェス!あはれいゆ@アパレイユ!しやかいものてゆしえす@じゃがいものドフィーヌ、---、マルキーズのアパレイユ}
\index{まるきーす@マルキーズ!あはれいゆ@アパレイユ!しやかいものまるきーす@じゃがいものドフィーヌ、デュシェス、---のアパレイユ}

⇒ \protect\hyperlink{legumes}{野菜料理}の章、\protect\hyperlink{pommes-de-terre}{じゃがいも}の項を参照。

\maeaki

\hypertarget{appareil-maintenon}{%
\subsubsection{アパレイユ・マントノン}\label{appareil-maintenon}}

\frsub{Appareils Maintenon}\footnote{マントノン夫人(出生名フランソワーズ・ドビニェ
  1635〜1719)。は
  じめはマントノン侯爵夫人としてルイ14世とモンテスパン夫人の間に生ま
  れた子どもたちの非公式な教育係となり、モンテスパン夫人の死後、ルイ
  14世と結婚した。彼女の名を冠した料理はここで言及されている\protect\hyperlink{cotelettes-maintenon}{羊のコ
  トレット マントノン}の他、卵料理、菓子など
  にある。「羊のコトレット マントノン」は彼女自身が考案したとも、ル
  イ14世付の料理人の考案ともいわれているが、いずれも憶測の域を出ない。
  なお、côtelette(コトレット)とは仔牛、羊の背肉を骨付きで肋骨1本ず
  つに切り分けたもの。日本語では、仔羊の場合ラムチョップと呼ばれるこ
  とも多い。}

\index{garniture@garniture!appareil@appareil!maintenon@appareil Maintenon}
\index{appareil@appareil!maintenon@--- Maintenon}
\index{maintenon@Maintenon!appareil@appareil ---}
\index{かるにちゆーる@ガルニチュール!あはれいゆ@アパレイユ!まんとのん@アパレイユ・マントノン}
\index{あはれいゆ@アパレイユ!まんとのん@---・マントノン}
\index{まんとのん@マントノン!あはれいゆ@アパレイユ・---}

(\protect\hyperlink{cotelettes-maintenon}{羊のコトレット マントノン}用)

\protect\hyperlink{sauce-bechamel}{ベシャメルソース}4
dlと\protect\hyperlink{sauce-soubise}{スビーズ}1
dlを半量になるまで煮詰める。

卵黄3個を加えてとろみを付ける。あらかじめマッシュルーム100 gを薄切りに
してバターでごく弱火で鍋に蓋をして蒸し煮\footnote{étuver
  (エチュヴェ)。}したものを加える。

\maeaki

\hypertarget{appareil-montglas}{%
\subsubsection{アパレイユ・モングラ}\label{appareil-montglas}}

\frsub{Appareils à la Montglas}\footnote{Salpicon à la
  Monglas(サルピコンアラモングラ)とも呼ばれものと
  ほぼ同じ。サルピコンはせいぜい5 mm角くらいの小さなさいの目に切った
  もののこと。羊のコトレット モングラ以外の用途としては、ブシェ(パ
  イ生地で作ったケースに詰め物をしたもの。本書ではオードブルに分類さ
  れている)やタルトレット(小さなタルト)の\textbf{アパレイユ}にする。17
  世紀のモングラ侯爵 François Clermont Marquis de Montglas (生年不
  詳〜1675)の名を冠したものらしいが、由来などは不明。}

\index{garniture@garniture!appareil@appareil!montglas@appareil à la Montglas}
\index{appareil@appareil!montglas@--- à l Montglas}
\index{montglas@Montglas!appareil@appareil à la ---}
\index{かるにちゆーる@ガルニチュール!あはれいゆ@アパレイユ!もんくら@アパレイユ・モングラ}
\index{あはれいゆ@アパレイユ!もんくら@---・モングラ}
\index{もんくら@モングラ!あはれいゆ@アパレイユ・---}

(\protect\hyperlink{cotelettes-monglas}{羊のコトレット モングラ}その他に用いられる)

以下の材料を通常より太めで短かい千切り\footnote{julienne
  (ジュリエーヌ)。}にする。\protect\hyperlink{saumure-liquide-pour-langue}{赤く漬けた舌
肉}150 g、フォワグラ150 g、茹でたマッシュ ルーム100 g、トリュフ100 g。

これらを、マデイラ酒風味の充分に煮詰めた\protect\hyperlink{sauce-demi-glace}{ソース・ドゥミグラ
ス}2\undemi{} dlであえる。バターを塗った平皿に広げ、
使うまでそのまま冷ましておく。

\maeaki

\hypertarget{appareil-provencal}{%
\subsubsection{プロヴァンス風アパレイユ}\label{appareil-provencal}}

\frsub{Appareils à la Provençale}

\index{garniture@garniture!appareil@appareil!provencal@appareil à la provençale}
\index{appareil@appareil!provencale@--- à la Provençale}
\index{provençal@provençale(e)!appareil@appareil à la ---e}
\index{かるにちゆーる@ガルニチュール!あはれいゆ@アパレイユ!ふろうあんすふう@プロヴァンス風アパレイユ}
\index{あはれいゆ@アパレイユ!ふろうあんすふう@プロヴァンス風---}
\index{ふろうあんすふう@プロヴァンス風!あはれいゆ@---アパレイユ}

(\protect\hyperlink{cotelettes-provencale}{羊のコトレット プロヴァンス風}用)\footnote{\protect\hyperlink{appareil-maintenon}{アパレイユ・マントノン}からこれまでの3種の
  アパレイユはいずれも、羊のコトレット(ラムチョップ)の片面だけを焼
  いて、その表面をよく\ruby{拭}{ぬぐ}い、まだ焼いていない面を下にし
  て、焼いた側の面にこれらのアパレイユを塗る、あるいは盛り上げてから
  オーブンに入れるという同工異曲とも言うべき仕立てに用いられる。ここ
  で、アパレイン・マントノンとこのプロヴァンス風アパレイユの「用途」
  の部分の原文には動詞farcirあるいはその過去分詞farci(es)が用いられ
  ているのはとても興味深いと言えよう。farcirを日本語の「詰め物をする」
  と等価と考えてはうまく理解できない例のひとつだろう。farcirの原義は
  「ファルスで満たす」であって、中に詰めることではない。なお、\href{http://cnrtl.fr/definition/farce}{TLFi
  によるファルスの定義}は、「肉な
  どと他の材料(香草や茸、細かく刻んだマロンなど)を混ぜ合わせ、スパ
  イスを加えたりして、一般的にはソースや卵、パナードでつないだもの。
  これを牛や羊の肉や家禽あるいは魚や野菜に、加熱前に加えて使用する」
  となっている。すなわち、詰めることは詰めるけれども、必ずしも空洞に
  なっている部分に詰めるというわけではないというのが言葉のうえでの意
  味。例えばマッシュルームのカサの裏側にファルス、もしくは何らかのア
  パレイユを「詰める」(日本語としては「盛る」のほうが適切かも知れな
  い)と、champignon farci (シャンピニョンファルシ)となる。}

\protect\hyperlink{sauce-soubise}{ソース・スビーズ}5
dlを充分に固くなるまで煮詰める。潰
したにんにく1片を加え、卵黄3個を加えてとろみを付ける。

\maeaki

\hypertarget{bordures-en-farce}{%
\subsubsection{ファルスで作る縁飾り}\label{bordures-en-farce}}

\frsub{Bordures en farce}

\index{garniture@garniture!appareil@appareil!bordures farce@bordures en farce}
\index{appareil@appareil!bordures farce@bordures en farce}
\index{bordure@bordure!farce@--- en farce}
\index{farce@farce!bordures@bordures en ---}
\index{かるにちゆーる@ガルニチュール!あはれいゆ@アパレイユ!ふあるすふちかさり@ファルスで作る縁飾り}
\index{あはれいゆ@アパレイユ!ふあるすふちかさり@ファルスで作る縁飾り}
\index{ふあるす@ファルス!ふちかざり@---で作る縁飾り}
\index{ほるてゆーる@ボルデュール ⇒ 縁飾り!ふぁるす@ファルスで作る縁飾り}
\index{ふちかさり@縁飾り!ふあるす@ファルスで作る---}

この縁飾りは、飾り付ける料理の素材とおなじ材料を中心にしたファルス\footnote{本文に指定はないが、原則としては\protect\hyperlink{farce-a}{ファルス
  A}か、\protect\hyperlink{farce-de-veau-pour-bordures}{盛り
  付けの縁飾りおよび底に敷いたり、詰め物をしたクネルに用いる仔牛のファ
  ルス}を用いることになるだろう。もっ
  とも、料理において厳密な規定ではないので、実現可能な範囲で他のタイ
  プのファルスを用いてみるのもいいだろう。}を使
う。縁飾り用の形\footnote{moule à
  bordure(ムーラボルデュール)、ボルデュール型ともいう。
  大きなリング型で、表面に山形の刻み目(浮き彫り模様)の入ったタイプ
  (moule historié ムールイストリエ、またはmoule cannelé ムールカヌ
  レ)と、特に模様の入っていないプレーンなもの(moule uni ムールユニ)
  の2種に大別される。}はプレーンなものでも浮き彫り模様の入ったものでもい
いが、たっぷりとバターを塗ってからファルスを詰めて低めの温度で火を通す
\footnote{原文pocher(ポシェ)。ここまでにも何度も出てきた表現だが、茹で
  る場合は「沸騰しない程度の温度で加熱すること」であり、このように型
  に詰めた場合には湯をはった天板に型をのせてやや低温のオーブンに入れ
  てゆっくり加熱することになる。}。

プレーンな型を使う場合は、きれいに切ったトリュフのスライスやポシェした
\footnote{原文 oeuf poché をそのまま訳したが、表面に飾りとして用いるのは
  固茹で卵の白身をスライスして型抜きあるいはナイフできれいに切ったも
  のを使うことが多い。}卵の白身、\protect\hyperlink{saumure-liquide-pour-langues}{赤く漬けた舌肉}、ピスタ
チオなどで表面を装飾するといい。

浮き彫り模様の型を使う場合は上記のような装飾は省いていい。

このようなファルスで作った縁飾りを使うのはとりわけ、鶏肉料理、魚料理、牛や羊肉のソテーなど。

\maeaki

\hypertarget{bordures-en-legumes}{%
\subsubsection{野菜で作る縁飾り}\label{bordures-en-legumes}}

\frsub{Bordures en Légumes}

\index{garniture@garniture!appareil@appareil!bordures legumes@bordures en légumes}
\index{appareil@appareil!bordures legume@bordures en légumes}
\index{bordure@bordure!legumes@--- en légumes}
\index{legume@légume!bordures@bordures en ---s}
\index{かるにちゆーる@ガルニチュール!あはれいゆ@アパレイユ!やさいふちかさり@野菜で作る縁飾り}
\index{あはれいゆ@アパレイユ!やさいふちかさり@野菜で作る縁飾り}
\index{やさい@野菜!ふちかざり@---で作る縁飾り}
\index{ほるてゆーる@ボルデュール ⇒ 縁飾り!やさい@野菜で作る縁飾り}
\index{ふちかさり@縁飾り!やさい@野菜で作る---}

プレーンなボルデュール型の内側にたっぷりとバターを塗り、下拵えしたさ
まざまな野菜を型の底面と側面にシャルトルーズ\footnote{chartreuse
  本文にあるように、野菜を装飾に用いた仕立てのひとつ。
  日本では「ペルドローのシャルトルーズ」perdreaux en chartreuseが有
  名だろうか。シャルトル会修道院で作られている同名のスピリッツがある
  が、料理におけるシャルトルーズ仕立てもシャルトル会修道院に由来して
  いるという。シャルトル会は大斉、小斉の決まりに厳格で、野菜を多く食
  べる修道生活を送っていたことで有名。そのことにちなんだ仕立ての名称
  と言われている。この仕立ての文献上の初出は1914年刊ボヴィリエ『調理
  技術』第2巻の「りんごのシャルトルーズ仕立て」と思われる。これは今
  でいうデザートに位置するもので、りんごをサフランやアンゼリカととも
  に煮て黄色や緑に染め、もとの白い果肉、皮の赤など、それら色合いを組
  み合わせて美しく型の底面を側面に貼り付け、内部をりんごのマーマレー
  ド(≒ジャム)で満たす、というもの(t.2, pp.149-150)。このボヴィリ
  エのシャルトルーズは「原型」というよりはむしろ「バリエーション」的
  なものであることが、レシピ本文の文面から伺える。そのため、いつごろ
  成立した仕立てなのかは不明だが、いずれにしてもシャルトルーズはカレー
  ムが「アントレの女王」と呼んだ程に手の込んだ華やかな仕立てとして19
  世紀前半には定着していた。基本的には、円筒形の型に拍子木に切ってそ
  れぞれ下茹でしたにんじん、さやいんげん、かぶ、などの野菜をびっしり
  と貼りつけて崩れないようにファルスで塗り固める。その内側に、「ペル
  ドリのシャルトルーズ」の場合は、下茹でしたサヴォイキャベツとペルド
  リ(ペルドロー≒山うずら、の成鳥)をブレゼしたものを詰め、型の上面
  (提供するときは底面になる)に蓋をするようにファルスを塗ってから、
  湯煎にかけてファルスに火を通して固める。裏返して型から外して供する、
  というもの。野菜の配置、配色が重要で技術のいる仕立て(ヘリンボーン
  のようなパターンが比較的多かったようだ)。なお、「ペルドローのシャ
  ルトルーズ」と「ペルドリとサヴォイキャベツのブレゼ」を混同している
  ケースが日本でよく見られるが、シャルトルーズとはあくまでも数種類の
  野菜とファルスで表面を装飾する仕立てを意味しているので注意。}状に貼り付けるように
敷き詰める。型の中にやや固めに作った\protect\hyperlink{farce-de-veau-pour-bordures}{じゃがいもを「つなぎ」にした仔牛
のファルス}をいっぱいに詰める(\protect\hyperlink{farce-de-veau-pour-bordures}{「縁飾り
用仔牛のファルス」}参照)。低めのオーブ ンで湯煎焼きして火を通す。

この縁飾りはもっぱら、牛、羊肉の料理で野菜のガルニチュールをともなうも
のに使う。

\maeaki

\hypertarget{bordures-en-pate-blanche}{%
\subsubsection[白い生地で作る縁飾り]{\texorpdfstring{白い生地で作る縁飾り\footnote{おなじ縁飾り(ボルデュール)でも、美味しく出来るものと、食べも
  ので出来てはいるけれども実際には食べないことを前提とした装飾では、
  本書において明らかに扱いが異なる。この「白い生地で作る縁飾り」およ
  び次項「ヌイユ生地で作る縁飾り」は後者にあたるため、さして重きを置
  いた説明になっていない。}}{白い生地で作る縁飾り}}\label{bordures-en-pate-blanche}}

\frsub{Bordures en pâte blanche}

\index{garniture@garniture!appareil@appareil!bordures pate blanche@bordures en pâte blanche}
\index{appareil@appareil!bordures pate blanche@bordures en pâte blanche}
\index{bordure@bordure!pate blanche@--- en pâte blanche}
\index{pate blanche@pâte blanche!bordures@bordures en ---s}
\index{かるにちゆーる@ガルニチュール!あはれいゆ@アパレイユ!しろいきしふちかさり@白い生地で作る縁飾り}
\index{あはれいゆ@アパレイユ!しろいきしふちかさり@白い生地で作る縁飾り}
\index{きし@生地!しろ@白!ふちかざり@白い生地で作る縁飾り}
\index{ほるてゆーる@ボルデュール ⇒ 縁飾り!しろいきし@白い生地で作る縁飾り}
\index{ふちかさり@縁飾り!しろいきし@白い生地で作る野菜---}
\index{はーと@パート ⇒ 生地!ふらんしゅ@ブランシュ!ふちかざり@白い生地で作る縁飾り}

片手鍋に水1 dlと塩5 g、ラード\footnote{saindoux
  (サンドゥー)精製した豚の脂}30gを入れ、火にかけて沸騰させる。ふ
るった小麦粉100 gを加えて、余分な水分をとばし、大理石板の上に広げる。

捏ねながらでんぷん\footnote{原文では fécule
  (フェキュール)すなわち「でんぷん」としか指示 がないが、fécule de
  maïs (フェキュールドマイス)コンスターチがいいだろう。}を練り込んでいく。10回生地を折ってから、生地を休ませる。

生地を厚さ7 mm程度にのす。これを専用の抜き型で抜いて飾りのパーツをつく
る。エチューヴ\footnote{野菜などを乾燥させるための専用の低温で用いるオーブンの一種。}に入れて乾燥させる。これを卵白に小麦粉を加えた糊\footnote{repère
  (ルペール)ここでは小麦粉を卵白に加えて混ぜた糊のこと。
  通常は銀などの金属製の皿に装飾を貼り付ける際に用いる。この場合は
  事前に皿を熱しておき、手早く装飾のパーツを貼る。現代ではほとんど行
  なわれていない手法。小麦粉と水で作り鍋の蓋に目張りをするための生地
  も同じ用語だが、いずれのケースについても「ルペール」という用語は現
  代の日本の調理現場であまり多用されていない。}で皿の縁に貼り付ける。

\maeaki

\hypertarget{bordures-en-pate-a-nouille}{%
\subsubsection{ヌイユ生地で作る縁飾り}\label{bordures-en-pate-a-nouille}}

\frsub{Bordures en pâte à nouille}

\index{garniture@garniture!appareil@appareil!bordures pate nouille@bordures en pâte à nouille}
\index{appareil@appareil!bordures pate nouille@bordures en pâte à nouille}
\index{bordure@bordure!pate nouille@--- en pâte à nouille}
\index{nouille@nouille!bordures@bordures en pâte à ---}
\index{かるにちゆーる@ガルニチュール!あはれいゆ@アパレイユ!ぬいゆきしふちかさり@ヌイユ生地で作る縁飾り}
\index{あはれいゆ@アパレイユ!ぬいゆきしきしふちかさり@ヌイユ生地で作る縁飾り}
\index{きし@生地!ぬいゆ@ヌイユ!ふちかざり@ヌイユ生地で作る縁飾り}
\index{ほるてゆーる@ボルデュール ⇒ 縁飾り!ぬいゆきし@ヌイユ生地で作る縁飾り}
\index{ふちかさり@縁飾り!ぬいゆきし@ヌイユ生地で作る野菜---}
\index{はーと@パート ⇒ 生地!ぬいゆ@ヌイユ!ふちかざり@白い生地で作る縁飾り}

ごく固めに捏ねた\protect\hyperlink{nouilles}{ヌイユ生地}を用いて作る縁飾り。上記のよう
に抜き型で抜いてもいいし、あるいは厚さ6〜7 mmで高さ4〜5 cmの帯状に切っ
てもいい。この場合は、「エヴィドワール」と呼ばれる専用の小さな抜き型を
用いて模様をつけた帯状の生地を皿の縁にしっかりと貼り付ける。

どちらの方法をとる場合でも、ヌイユ生地を用いた縁飾りには溶いた卵黄を塗ってから、乾燥させる。

\maeaki

\hypertarget{croutons}{%
\subsubsection{クルトン}\label{croutons}}

\frsub{Croûtons}

\index{garniture@garniture!appareil@appareil!croutons@croûtons}
\index{appareil@appareil!croutons@croûtons}
\index{かるにちゆーる@ガルニチュール!あはれいゆ@アパレイユ!くるとん@クルトン}
\index{あはれいゆ@アパレイユ!くるとん@クルトン}
\index{くるとん@クルトン}

クルトンはいわゆる食パン\footnote{フランス語でpain(パン)とだけ言う場合はバゲットに代表されるリー
  ンなパンを指すのが普通で、イギリス式およびアメリカ式の「食パン」は
  pain de mie(パンドミー)と呼ばれて区別される。}で作る。形状や大きさは、どんな料理に合わ
せるかで決まってくる。これを澄ましバター\footnote{バターには少なからずカゼインなどの不純物が含まれており、それら
  が焦げや色むらの原因となるので、充分よく澄んだバターを使うこと。}で揚げるが、揚げるのは必ず提供
直前にすること。

\maeaki

\hypertarget{duxelles-seche}{%
\subsubsection[デュクセル・セッシュ]{\texorpdfstring{デュクセル\footnote{俗説としては、17世紀にユクセル侯爵
  Marquis d'Uxelles(マルキ デュ
  クセル)に料理長として仕えていたラ・ヴァレーヌが創案し、主人の名を
  付けたと言われている。d' は de + 母音の短縮形(フランス語文法では
  エリジオンという)。貴族の場合は領地名の前に de (≒ of, from) を付
  ける慣習があり、爵位 de 領地名、というのが正式な呼び名として用いら
  れていた。Uxellesは母音で始まるからd'Uxellesとなり、それが料理用語
  としてひとつの単語となりduxellesとして定着したという。しかし、
  duxelles (デュクセル)あるいはそれに類似する名称が用いられるよう
  になったのは19世紀以降であり、文献によって綴りも安定していない。19
  世紀末のファーヴル『料理および食品衛生事典』では duxel という綴り
  で項目が立てられている(なお、ファーヴルはデュクセルをアパレイユの
  一種と明確に定義している)。さらに時代を遡っていくと、オドの1858年
  版ではDurcelle(デュルセル)またはDuxelleという名称で呼ばれている
  と記述がある(p.167)。1856年刊グフェ『料理の本』ではd'Uxelles
  (p.72)。1833年刊カレーム『19世紀フランス料理』第3巻には、sauce à la
  Duxelle「ソース・デュクセル」が掲載されている。これはあくまでも
  「ソース」ではあるが、ベースとしてマッシュルームのみじん切りを使っ
  ていることは他と同様。さらにヴィアール『王国料理の本』1820年版
  (p.74)および1814年刊ボヴィリエ『調理技術』(p.73)には、のちのデュク
  セル・セッシュとほぼ同様のものがDurcelleの綴りで掲載されている。カ
  レームは\protect\hyperlink{mayonnaise}{マヨネーズ}の訳注でも見たとおり、料理名の綴
  りに独自のこだわりを持つ傾向が強かったので、あるいはカレームが
  durcelleからduxelleへの転換点として存在している可能性はある。
  Durcelleの語としての成り立ちは不明だが、人名(名字)に時折見られる
  ものであるため、何らかの由来があったことまでは推察される。このよう
  に、19世紀を通して綴りが安定しなかった点を考慮すると、ユクセル侯爵
  の名を冠したという説がどんなに早くとも19世紀中葉以降のものだとわか
  る。逆にいえばフランス語の/R/と/k/の音がやや似て聞こえることがある
  ために、はじめdurcelleと呼ばれていたアパレイユがduxelleとなり、ひ
  いては歴史上の人物Marquis d'Uxellesユクセル侯爵に結びつけられるよ
  うになった、と考えられよう。とはいえ、17世紀はいわゆるマッシュルー
  ム(和名バフンタケ)の人工栽培が実用化され、食材として流行した時代
  でもあった。そのため、ラ・ヴァレーヌおよび彼が仕えていたユクセル侯
  爵の名称をこのアパレイユに関連付けたはまったくの見当違いとも言えな
  いだろう。ちなみにラ・ヴァレーヌの『フランス料理の本』にはデュクセ
  ルに相当するアパレイユあるいは調理、仕込みの類の記述は見当らない。}・セッシュ}{デュクセル・セッシュ}}\label{duxelles-seche}}

\frsub{Duxelles sèche}\footnote{sec / sèche (セック /
  セッシュ)乾燥した、水気のない、の意。}

\index{garniture@garniture!appareil@appareil!duxelles seche@duxelles sèche}
\index{appareil@appareil!duxelles seche@duxelles sèche}
\index{champignon@champisnon!duxelles seche@duxelles sèche}
\index{duxelles@duxelles!seche@--- sèche}
\index{かるにちゆーる@ガルニチュール!あはれいゆ@アパレイユ!てゆくせるせつしゆ@デュクセル・セッシュ}
\index{あはれいゆ@アパレイユ!てゆくせるせつしゆ@デュクセル・セッシュ}
\index{てゆくせる@デュクセル!せつしゆ@---・セッシュ}
\index{まつしゆるーむ@マッシュルーム!てゆくせる@デュクセル!せつしゆ@デュクセル・セッシュ}

デュクセルはベースとして必ず、みじん切りにした茸を用いるが、食用のものならどんな茸でも構わない。

バター30 gと植物油30 gを鍋に熱し、玉ねぎのみじん切りとエシャロットのみ
じん切りを各大さじ1杯ずつ入れて、軽く炒める。マッシュルームの切りくず
と軸を細かくみじん切りにしたもの250 gを加え、よく圧して水気を出させる。
水分が完全に蒸発するまで強火で炒め続ける。塩こしょうで調味し、パセリの
みじん切り1つまみを加えて仕上げる。陶製の器に移し入れ、バターを塗った紙で蓋をする。

このデュクセル・セッシュは多くの料理で使われる。

\maeaki

\hypertarget{duxelles-pour-legumes-farcis}{%
\subsubsection[野菜のファルシ用デュクセル]{\texorpdfstring{野菜のファルシ\footnote{farci
  (ファルシ)詰め物をした、の意。}用デュクセル}{野菜のファルシ用デュクセル}}\label{duxelles-pour-legumes-farcis}}

\frsub{Duxelles sèche}

\index{garniture@garniture!appareil@appareil!duxelles legumes farcis@duxelles pour légumes farcis}
\index{appareil@appareil!duxelles legumes farcis@duxelles pour légumes farcis}
\index{champignon@champisnon!duxelles legumes farcis@duxelles pour légumes farcis}
\index{duxelles@duxelles!legumes farcis@--- pour légumes farcis}
\index{かるにちゆーる@ガルニチュール!あはれいゆ@アパレイユ!やさいのふあるしようてゆくせる@野菜のファルシ用デュクセル}
\index{あはれいゆ@アパレイユ!やさいのふあるしようてゆくせる@野菜のファルシ用デュクセル}
\index{てゆくせる@デュクセル!やさいのふあるしよう@野菜のファルシ用---}
\index{まつしゆるーむ@マッシュルーム!てゆくせる@デュクセル!やさいのふあるしよう@野菜のファルシ用デュクセル}

(トマト、茸などの詰め物用)

\protect\hyperlink{duxelles-seche}{デュクセル・セッシュ}100
g、すなわち大さじ\footnote{大さじ1杯 = 15 cc
  は第二次大戦後に日本で普及した単位にすぎない
  ことに注意。本書における「大さじ1杯」の表現は非常にあいまいで、ざっ
  くりとした分量表示。「大きなスプーン」と訳してもいいのだが、本文が
  あまりに冗長になるために「大さじ」としていることに留意していただき
  たい。}4杯を 用意する。白ワイン\undemi{}
dlを加えてほぼ完全に煮詰める。次に、トマト
をしっかり効かせた\protect\hyperlink{sauce-demi-glace}{ソース・ドゥミグラス}1
dlと小さ めのにんにく1片をつぶしたもの、パンの身25 gを加える。

ごく弱火にかけて煮込み、詰め物をするのにちょうどいい固さになるまで煮詰める。

\maeaki

\hypertarget{duxelles-pour-farnitures-diverses}{%
\subsubsection{ガルニチュール用デュクセル}\label{duxelles-pour-farnitures-diverses}}

\frsub{Duxelles pour garnitures diverses}

\index{garniture@garniture!appareil@appareil!duxelles ganiture diverses@duxelles pour garniture diverses}
\index{appareil@appareil!duxelles garnitures diverses@duxelles pour garnitures diverses}
\index{champignon@champisnon!duxelles pour garnitures diverses@duxelles pour garnitures diverses}
\index{duxelles@duxelles!garnitures diverses@--- pour garnitures diverses}
\index{かるにちゆーる@ガルニチュール!あはれいゆ@アパレイユ!かるにちゆーるようてゆくせる@ガルニチュール用デュクセル}
\index{あはれいゆ@アパレイユ!かるにちゆーるようてゆくせる@ガルニチュール用デュクセル}
\index{てゆくせる@デュクセル!かるにちゆーるよう@ガルニチュール用---}
\index{まつしゆるーむ@マッシュルーム!てゆくせる@デュクセル!かるにちゆーるよう@ガルニチュール用デュクセル}

(タルトレット、玉ねぎ\footnote{玉ねぎには、完熟、乾燥させた際に表皮が黄色いタイプと白いもの、
  赤紫色の3系統がある。黄色系統の玉ねぎはフォンなどに用いられること
  が多い(日本ではこのタイプがほとんど。また「泉州黄」という品種はフ
  ランスの野菜栽培の専門書でも言及がある程に栽培特性とクオリティが高
  く評価されて、フランスでも栽培されている)。白玉ねぎ(oignon blanche
  オニョンブロンシュ)は生食やその他の調理、とりわけ小さいも
  のは下茹でしてからバターで色よく炒めて(グラセ)ガルニチュールに用
  いられる。火が通りやすく、甘いものが多い。赤紫のものは品種によって
  特性が違うが、加熱調理、生食いずれにも用いられる。}、きゅうり\footnote{20世紀末頃から日本の種苗メーカーが育種した品種も栽培されるよう
  になってきているため、あえて「きゅうり」と訳したが、伝統的な concombre
  (コンコンブル)は太さ4〜5cm、長さ30〜45cm程度まで大きく
  するのが一般的で、日本の現代品種と異なり表皮は固く、苦味やアクは少ない。
  種の部分をスプーンなどで取り除いて、そこに詰め物して加熱調理する。
  また、ビストロなどでは生のまま輪切りにして、crudités (クリュディ
  テ)すなわち野菜盛り合わせの前菜に加えることも多い。ただし、日本の
  ように「サラダ野菜」という認識はあまり持たれておらず、加熱調理する
  のが前提の野菜というイメージのほうが強い。}、などの詰め物用)

\protect\hyperlink{duxelles-seche}{デュクセル・セッシュ}100
gに、\protect\hyperlink{farce-c}{ファルス・ムスリー
ヌ}または\protect\hyperlink{farce-a}{パナードを用いたファルス}もしくは\protect\hyperlink{farce-gratin-a}{ファル
ス・グラタン}60 gのいずれかを料理に合わせて加える。

このデュクセルを野菜の詰め物として用いた場合は、表面を焦がさないように\footnote{表面に焦げ目を付けることを
  gratiner (グラティネ)という。}、
低温のオーブンに入れて加熱すること\footnote{pocher
  (ポシェ)。本来は沸騰しない程度の温度で茹でることを指す
  が、この場合は比較的低温のオーブンで加熱調理するという意味。}。

\maeaki

\hypertarget{duxelles-bonne-femme}{%
\subsubsection[デュクセル・ボヌファム]{\texorpdfstring{デュクセル・ボヌファム\footnote{bonne
  femme(ボヌファム)は「おばさん」くらいの意。家庭風あるいは田舎風の比較的素朴さを感じさせる料理に付けられる名称。}}{デュクセル・ボヌファム}}\label{duxelles-bonne-femme}}

\frsub{Duxelles à la bonne femme}

\index{garniture@garniture!appareil@appareil!duxelles bonne femme@duxelles à la bonne femme}
\index{appareil@appareil!duxelles legumes bonne femme@duxelles à la bonne femme}
\index{champignon@champisnon!duxelles bonne femme@duxelles à la bonne femme}
\index{duxelles@duxelles!bonne femme@--- à la bonne femme}
\index{bonne femme@bonne femme!duxelles@duxelles à la ---}
\index{かるにちゆーる@ガルニチュール!あはれいゆ@アパレイユ!てゆくせるほぬふあむ@デュクセル・ボヌファム}
\index{あはれいゆ@アパレイユ!てゆくせるほぬふあむ@デュクセル・ボヌファム}
\index{てゆくせる@デュクセル!ほぬふあむ@---・ボヌファム}
\index{ほぬふあむ@ボヌファム!てゆくせる@デュクセル・---}

(家庭料理用)

生のデュクセルに、しっかり味付けをした\protect\hyperlink{chair-a-saucisse}{ソーセージ用挽肉}を同量加えるだけ。

\maeaki

\hypertarget{essence-de-tomate}{%
\subsubsection{トマトエッセンス}\label{essence-de-tomate}}

\frsub{Essence de tomate}

\index{garniture@garniture!appareil@appareil!essence tomate@essence de tomate}
\index{appareil@appareil!essence tomate@essence de tomate}
\index{tomate@tomate!essence@essence de ---}
\index{essence@essence!tomate@--- de tomate}
\index{かるにちゆーる@ガルニチュール!あはれいゆ@アパレイユ!とまとえつせんす@トマトエッセンス}
\index{あはれいゆ@アパレイユ!とまとえつせんす@トマトエッセンス}
\index{とまと@トマト!えつせんす@---エッセンス}
\index{えつせんす@エッセンス!トマト---}

よく熟したトマトのジュースを漉し器で漉す。これを片手鍋に入れて、弱火に
かけてゆっくりと、シロップ状になるまで煮詰める。

布で漉るが、圧したり絞らないこと。保存しておく。

\hypertarget{nota-essence-de-tomate}{%
\subparagraph{【原注】}\label{nota-essence-de-tomate}}

このトマトエッセンスはブラウン系の派生ソースの仕上げに色合いを調節する
のにとても便利だ\footnote{ブラウン系の派生ソースの節で、明示的にこのトマトエッセンスの使
  用に言及しているレシピは2つのみだが、必ずしもそのことにこだわず、適
  宜、必要に応じて使うのがいい。}。

\maeaki

\hypertarget{fonds-de-plats}{%
\subsubsection{皿に敷いて料理をのせる台、タンポン、クルスタード}\label{fonds-de-plats}}

\frsub{Fonds de plats, Tampons et Croustades}

\index{garniture@garniture!appareil@appareil!fonds plats tampons croustades@fonds de plats, tampons et croustades}
\index{appareil@appareil!fonds plats tampons croustades@fonds de plats, tampons et croustades}
\index{fonds@fonds!plats@--- de plats} \index{tampon@tampon}
\index{croustade@croustade}
\index{かるにちゆーる@ガルニチュール!あはれいゆ@アパレイユ!さらにしいてりようりをのせるたい@皿に敷いて料理をのせる台、タンポン、クルスタードトマトエッセンス}
\index{あはれいゆ@アパレイユ!さらにしいてりようりをのせるたい@皿に敷いて料理をのせる台}
\index{たい@台!さらにしいてりようりをのせる@皿に敷いて料理をのせる---}
\index{たんぽん@タンポン} \index{くるすたーと@クルスタード}

皿に敷いて料理をのせる台、タンポン、クルスタードの重要性は日々ますます
失なわれつつある。新しいサーヴィスの方式ではこれらをほぼ完全に用いなて
はいない。これらの装飾的な台はパンや、一番多いケースは米を材料に作られ
る\footnote{実際、本書においてこれらを用いる指示は非常に少ないが、まったく
  ないわけでもない。ただ、エスコフィエが乗り越えたいと願ったデュボワ
  とベルナールの『古典料理』がこれらの装飾的な台の作り方にかなりのペー
  ジを割いていることと比較すると、驚くほどに素気なく短かい説明で終わっ
  ている。}。

パンを使った台は、固くなったパンの身を切って作る。これをバターで揚げ
{[}\^{}31{]}、小麦粉を卵白に加えて作った糊\footnote{説明的に訳したが、原文は
  repèreの1語。\protect\hyperlink{bordures-en-pate-blanche}{白い生地で作る縁飾
  り}および訳注参照。}で皿の底に貼り付ける。

\hypertarget{ux7c73ux3067ux4f5cux308bux30bfux30f3ux30ddux30f3ux3068ux30afux30ebux30b9ux30bfux30fcux30c9}{%
\subparagraph{米で作るタンポンとクルスタード}\label{ux7c73ux3067ux4f5cux308bux30bfux30f3ux30ddux30f3ux3068ux30afux30ebux30b9ux30bfux30fcux30c9}}

\ldots{}\ldots{}パトナ米2kgを、水が完全に澄むまでよく洗う。

たっぷりの水に入れて火にかけ、5分間茹でる。鍋の湯を捨て、別の湯に漬け
て米を洗う。再度湯をきる。大きな片手鍋に丈夫で清潔な布または豚背脂のシー
トを敷き、入れてみょうばん10 gを加え、布または豚背脂のシートを折り畳ん
で米を包む。鍋に蓋をして、弱火のオーブンかエチューヴ\footnote{étuve
  主として野菜の乾燥などを目的とした低温専用のオーブン。}に入れ、3時間
加熱する。

その後、米を力をこめてすり潰す。ラードを塗った布のナフキンで包んで揉み、
ラードを塗った器に手早く詰めて、冷ます。

充分に冷めたら、米の塊を彫って装飾する。みょうばんを加えた水に漬けて、
こまめに水を替えてやれば長期保存も可能だ。

\maeaki

\hypertarget{portugaise}{%
\subsubsection{ポルチュゲーズ / トマトのフォンデュ}\label{portugaise}}

\frsub{Fondue de tomate ou Portugaise}\footnote{portugais(e)
  (ポルチュゲ/ポルチュゲーズ)は形容詞の場合は「ポ
  ルトガルの」の意。名詞の場合はポルトガル人。ここでは大文字で書き出
  していることから名詞と考えられる(なお現代フランス語の正書法では文
  頭以外の語は固有名詞のみ大文字で始めることになっており、普通名詞を
  文中で大文字にすることはないが、料理名などの場合は比較的自由に大文
  字を使う傾向にある)。すなわち「ポルトガルの女」くらいの意味にとる
  ことが可能。ちなみに、このレシピとはまったく関係ないが、、\emph{Lettres
  Portugaises} (レットルポルチュゲーズ)『ぽるとがる\ruby{文}{ぶみ}』
  という題名の本が17世紀にフランスで出版され人々の感動を誘った。リル
  ケや佐藤春夫が自国語に翻訳、翻案したものも有名。実在したポルトガル
  の修道女マリアナ・アルコフォラドがフランス軍人に宛てた恋文をまとめ
  た、事実にもとづく書簡集と考えられていたが、20世紀になってから、ガ
  ブリエル・ド・ギユラーグという男性文筆家によるまったくの創作である
  ことが証明された。いわゆる「書簡体小説」である。とはいえ作品の文学
  的価値はまったく減じることのない名作。書簡体小説という形式は18世紀
  に流行し、ゲーテ『若きウェルテルの悩み』やラクロ『危険な関係』、ル
  ソー『新エロイーズ』などの名作がある。19世紀前半にはその流行も落ち
  着き、バルザック『二人の若妻の手記』などはこの小説形式の流行の最後
  を飾る名作のひとつとして名高い。なお、トマトは16世紀に既にフランス
  にもたらされており、16世紀末に出版されたオリヴィエ・ド・セール『農
  業経営論』では「美しいが食べても美味しくない」と記されている。食材
  として広く普及したのは19世紀以降であり、爆発的な流行現象とさえいえ
  るほどだった。第二帝政期を代表する小説家のひとりフロベールの遺作
  『ブヴァールとペキュシェ』にも農業に挑戦した2人の主人公がトマトの
  芽掻きをする必要があることを知らなかったために失敗したエピソードが
  描かれている。「オマール・アメリケーヌ」や「舌びらめ デュグレレ」
  などトマトが重要な役割を果している料理が多く創案され、フランス料理
  の歴史において時代を象徴する食材のひとつともいえる。}

\index{garniture@garniture!appareil@appareil!fondue tomate@fondue de tomate ou Portugaise}
\index{appareil@appareil!fondue tomate@fondue de tomate ou Portugaise}
\index{tomate@tomate!fondue@fondue de --- ou Portugaise}
\index{portugais@portugais(e)!fondue de tomate@foudue de tomate ou Portugaise}
\index{かるにちゆーる@ガルニチュール!あはれいゆ@アパレイユ!とまとのふおんてゆ@トマトのフォンデュ/ポルチュゲーズ}
\index{あはれいゆ@アパレイユ!とまとのふおんてゆ@トマトのフォンデュ/ポルチュゲーズ}
\index{かるにちゆーる@ガルニチュール!あはれいゆ@アパレイユ!ほるちゆけーす@ポルチュゲーズ/トマトのフォンデュ}
\index{あはれいゆ@アパレイユ!ほるちゆけーす@ポルチュゲーズ/トマトのフォンデュ}
\index{とまと@トマト!ふおんてゆ@---のフォンデュ/ポルチュゲーズ}
\index{ふおんてゆ@フォンデュ!とまと@トマトの---/ポルチュゲーズ}
\index{ほるちゆけーす@ポルチュゲーズ/トマトのフォンデュ}
\index{ほるとかるふう@ポルトガル風!ほるちゆけーす@ポルチュゲーズ/トマトのフォンデュ}

玉ねぎ大1個をみじん切りにしてバターまたは植物油で炒める。トマト500 gは
皮を剥いて潰し、粗みじん切りにして鍋に加える。潰したにんにく1片と塩、
こしょうを加える。弱火にかけて水分がすっかりなくなるまで煮詰める\footnote{トマトは品種にもよるが、混ぜずに弱火で加熱すると固形物が沈殿し、
  水分が上澄みになる。ここでは濃縮トマトペーストになるほどは煮詰めず、
  その上澄みがなくなるまで、という解釈でいいだろう。}。

時季によっては、つまりトマトの熟し具合によっては、粉砂糖をほんの1つま
み加えてやるといい。

\maeaki

\hypertarget{kache-de-sarrazin-pour-potage}{%
\subsubsection[ポタージュ用そば粉のカシュ]{\texorpdfstring{ポタージュ用そば粉のカシュ\footnote{Kacha
  (カシャ)とも。日本語ではカーシャと呼ばれるほうが多いようだ。
  もとはロシアをはじめとするスラブ諸国における粥の総称でロシア語では
  \ltjsetparameter{jacharrange={-2}}каша\ltjsetparameter{jacharrange={+2}}
  。フランス料理に取り入れられ、そば粉またはセモリナ粉でつくったクレー
  プのようなものを意味するようになった。このカシュはポタージュのガル
  ニチュールそのもの、つまり「浮き実」となるので、アパレイユには分類
  されないが、次項のクリビヤック用のカシュはクリビヤックを作る際のパー
  ツのひとつとなるのでアパレイユに属すると考えていい。}}{ポタージュ用そば粉のカシュ}}\label{kache-de-sarrazin-pour-potage}}

\frsub{Kache de Sarrazin pour Potages}

\index{garniture@garniture!appareil@appareil!kache sarrazin potages@Kache de Sarrazin pour Potages}
\index{appareil@appareil!kache sarrazin potage@Kache de Sarrazin pour Poatages}
\index{kache@kache!sarrazin@--- de Sarrazin pour Potages}
\index{sarrazin@sarrazin!kache@Kache de --- pour Potages}
\index{かるにちゆーる@ガルニチュール!あはれいゆ@アパレイユ!ほたーしゆようそはこのかしゆ@ポタージュ用そば粉のカシュ}
\index{あはれいゆ@アパレイユ!ほたーしゆようそはこのかしゆ@ポタージュ用そば粉のカシュ}
\index{かしゆ@カシュ!そば粉@ポタージュ用そば粉の---}
\index{そはこ@そば粉!かしゆ@ポタージュ用---のカシュ}

(仕上がり約10人分\footnote{原文 pour un service (プランセルヴィス)
  フランス宮廷料理の時
  代から、ロシア式サービスの普及しはじめた頃まで、宴席などの料理を作
  る際の単位としてserviceが用いられた。1 service は概ね10人分。現実
  には8〜12人くらいの間で融通を効かせて運用されていたようだ。本書の
  レシピの分量は多くが1 serviceすなわち約10人分で書かれている。})

粗挽きのそば粉1 kgに塩を加えたゆるま湯を7〜8
dl加えてデトランプ\footnote{ここでは動詞 détremper
  (デトロンペ)が使われているが、faire un détrempe
  (フェランデトロンプ)と同義で粉が吸水して捏ねる前の状態(塊)のこと。}を
作ってまとめる。これを深手の片手鍋\footnote{casserole russe
  (カスロールリュス)直訳すると「ロシアの片手鍋」だが、通常は深い片手鍋をそう呼ぶ。}に入れて押し潰す。高温のオーブ
ンに入れて約2時間加熱する。

オーブンから出したら、表面の固くなった皮の部分は取り除く。鍋の中のパン
状になったものを、鍋の周囲にこびりついた焦げの部分に触れないようにして
取り出す。

これにバター100 gを加えて捏ねる。厚さ1 cmになるように重しをして冷ます。
直径25 mm位の\footnote{原文 un emporte-pièce rond de la grandeur d'une
  pièce de 2 francs
  「2フラン硬貨の大きさの円形の抜き型」。フランはヨーロッパ通
  貨統合前のフランスの通貨単位。2フラン硬貨は概ね26〜27 mm。}の丸い型抜きで抜く。これを澄ましバターで色よく
焼く。オードブル皿か、ナフキンに盛り付けて供する。

\hypertarget{ux539fux6ce8}{%
\subparagraph{【原注】}\label{ux539fux6ce8}}

このカシュをオーブンから出してそのままの状態で供してもいい。その場合は専用の容器に盛りつける。

\maeaki

\hypertarget{kache-de-semoule-pour-coulibiac}{%
\subsubsection[クリビヤック用セモリナ粉のカシュ]{\texorpdfstring{クリビヤック用\footnote{サーモンなどをブリオシュ生地で包んで焼いた料理。これを作る際に、
  厚さ1cmくらいに切ったサーモンの身とこのカシュまたは米を互いに層に
  なるようにして、ブリオシュ生地で包んで焼く。カレームがロシアからこ
  の料理のレシピを持ち帰ってきたことからロシア料理起源と解されること
  も多いし、実際にロシア料理風に作るが、料理の起源としては、ドイツ料
  理に端を発しているという。}セモリナ粉のカシュ}{クリビヤック用セモリナ粉のカシュ}}\label{kache-de-semoule-pour-coulibiac}}

\frsub{Kache de Semoule pour le Coulibiac}

\index{garniture@garniture!appareil@appareil!kache semoule coulibiac@Kache de Semoule pour le Coulibiac}
\index{appareil@appareil!kache semoule coulibiac@Kache de Semoule pour le Coulibiac}
\index{kache@kache!semoule@--- de Semoule pour le Coulibiac}
\index{semoule@semoule!kache@Kache de --- pour le Coulibiac}
\index{かるにちゆーる@ガルニチュール!あはれいゆ@アパレイユ!くりひやつくようせもりなこのかしゆ@クリビヤック用セモリナ粉のカシュ}
\index{あはれいゆ@アパレイユ!くりひやつくようせもりなこのかしゆ@クリビヤック用セモリナ粉のカシュ}
\index{かしゆ@カシュ!セモリナ粉@クリビヤック用セモリナ粉の---}
\index{せもりなこ@セモリナ粉!かしゆ@クリビヤック用セモリナ粉---のカシュ}

(仕上がり約10人分)

大粒のセモリナ粉200gに溶き卵1個をよく混ぜる。天板の上に広げて弱火で乾燥させる。

これを目の粗い漉し器で裏漉しする。コンソメに入れて約20分間、沸騰しない
程度の温度で加熱する\footnote{pocher (ポシェ)。}。気をつけて水気をきる。
\end{recette}