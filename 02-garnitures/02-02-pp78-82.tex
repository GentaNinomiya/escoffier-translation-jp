\hypertarget{serie-des-appareiles-et-preparations-diverses-pour-garnitures-chaudes}{%
\section[温製ガルニチュール用アパレイユなど]{\texorpdfstring{温製ガルニチュール用アパレイユ\footnote{料理用語としての
  appareil
  アパレイユとは、具体的な何かを指す言葉ではなく、\ul{ある料理を作る過程において用いられる、\\複数の材料を組み
  合わせたもの}、という一種の概念。現実には、キッシュのアパレイユ(生クリームと卵、塩漬け豚バラ肉など)、クレーム・ブリュレのアパレイユ(卵黄、砂糖、生クリーム、牛乳)というように用いられるが、概ね、
  \ul{加熱して凝固する液体または半液状のもの、およびそれらを「つなぎ」
  として固形物をあえたものを指す}、と考えていい。アパレイユの概念としては、まったくの固形物である、3〜4
  mmのさいの目に切った香味野菜(場合によってはハムも入る)である\protect\hyperlink{matignon}{マティニョン}も
  appareil à matignon
  と表現されることはフランスの料理書においては珍しくないし、本節の\protect\hyperlink{duxelles-seche}{デュクセル・セッシュ}もまたアパレイユの一種に含められる。実際のところ、アパレイユという語はそれぞれの調理現場および料理人によって使い方がさまざまであり、概念としての理解も必ずしも共通しているとは限らない。本書では基本的に、上述のように加熱凝固する液体の場合と、半固形状あるいはクリーム状のものを指す場合がほとんど。また、既出の\protect\hyperlink{sauce-villeroy}{ソース・ヴィルロワ}などもまた、ソースというよりはむしろアパレイユと呼んでおかしくないものだろう。}など}{温製ガルニチュール用アパレイユなど}}\label{serie-des-appareiles-et-preparations-diverses-pour-garnitures-chaudes}}

\frsec{Série des Appareils et Préparations diverses pour Garnitures chaudes}

\index{garniture@garniture!appareils garnitures chaudes@appareils et préparations diverses pour garnitures chaudes}
\index{appareil@appareil!garnitures chaudes@--- et préparations diverses pour garnitures chaudes}
\index{かるにちゆーる@ガルニチュール!あはれいゆおんせい@温製ガルニチュールのためのアパレイユなど}
\index{あはれいゆ@アパレイユ!おんせいかるにちゆーる@温製ガルニチュールのための---など}
\begin{recette}
\hypertarget{appareils-a-cromesquis-et-a-croquettes}{%
\subsubsection[クロメスキとクロケットのアパレイユ]{\texorpdfstring{クロメスキとクロケット\footnote{クロケットは日本のコロッケの原型となったもので、細かく切った素材をじゃがいものピュレや\protect\hyperlink{sauce-bechamel}{ベシャメルソース}であえて円盤または円筒形に整形してパン粉衣を付けて揚げたもの。クロメスキは正六面体(サイコロ形)にすることが多く、コロッケとアパレイユが共通のため、形状が違うだけでクロケットのバリエーションという見方もあるが、ポーランド語のkromesk(薄く切ったもの)が語源とされる。}のアパレイユ}{クロメスキとクロケットのアパレイユ}}\label{appareils-a-cromesquis-et-a-croquettes}}

\frsub{Appareils à Cromesquis et à Croquettes}

\index{garniture@garniture!appareil@appareil!cromesquis croquettes@appareils à cromesquis et à croqeuttes}
\index{appareil@appareil!cromesquis croquettes@---s à cromesquis et à croquettes}
\index{cromesqui@cromesqui!appareil@appareils à --- et à croquettes}
\index{croquette@croquette!appareil@appareils à cromesquis et à ---}
\index{かるにちゆーる@ガルニチュール!あはれいゆ@アパレイユ!くろめすきとくろけつと@クロケットとクロメスキのアパレイユ}
\index{あはれいゆ@アパレイユ!くろめすきとくろけつと@クロケットとクロメスキの---}
\index{くろめすき@クロメスキ!あはれいゆ@---とクロケットのアパレイユ}
\index{くろけつと@クロケット!あはれいゆ@クロメスキと---のアパレイユ}

⇒ \protect\hyperlink{hors-d-oeuvres-chauds}{温製オードブル}の章を参照。

\hypertarget{appareils-a-pomme-dauphine-duchesse-marquise}{%
\subsubsection[じゃがいものドフィーヌ、デュシェス、マルキーズのアパレイユ]{\texorpdfstring{じゃがいものドフィーヌ、デュシェス、マルキーズ\footnote{dauphin(王太子)、dauphine(王太子妃)、duc(公爵)、
  duchesse(公爵夫人)、mariquis(侯爵)、marquise(侯爵夫人)。いずれも王家、貴族の位階(爵位)を表わす語だが、特に理由もなく料理名に付けられることが非常に多い。}のアパレイユ}{じゃがいものドフィーヌ、デュシェス、マルキーズのアパレイユ}}\label{appareils-a-pomme-dauphine-duchesse-marquise}}

\frsub{Appareils à pomme Dauphine, Duchesse et Marquise}

\index{garniture@garniture!appareil@appareil!pomme dauphine@appareils à pomme Dauphine, Duchesse et Marquise}
\index{appareil@appareil!pomme dauphine@---s à pomme Dauphine, Duchesse et Marquise}
\index{dauphin@dauphin(e)!appareil@appareil à pomme ---e}
\index{duc@duc / duchesse!appareil@appareil à pomme duchesse}
\index{marquis@marquis(s)!appareil@appareil à pomme ---e}
\index{かるにちゆーる@ガルニチュール!あはれいゆ@アパレイユ!しやかいものとふいーぬ@じゃがいものドフィーヌ、デュシェス、マルキーズのアパレイユ}
\index{あはれいゆ@アパレイユ!しやかいものとふいーぬと@じゃがいものドフィーヌ、デュシェス、マルキーズの---}
\index{とふいーぬ@ドフィーヌ!あはれいゆ@アパレイユ!しやかいものとふいーぬ@じゃがいもの---、デュシェス、マルキーズのアパレイユ}
\index{てゆしえす@デュシェス!あはれいゆ@アパレイユ!しやかいものてゆしえす@じゃがいものドフィーヌ、---、マルキーズのアパレイユ}
\index{まるきーす@マルキーズ!あはれいゆ@アパレイユ!しやかいものまるきーす@じゃがいものドフィーヌ、デュシェス、---のアパレイユ}

⇒ \protect\hyperlink{legumes}{野菜料理}の章、\protect\hyperlink{pommes-de-terre}{じゃがいも}の項を参照。

\hypertarget{appareil-maintenon}{%
\subsubsection[アパレイユ・マントノン]{\texorpdfstring{アパレイユ・マントノン\footnote{マントノン夫人(出生名フランソワーズ・ドビニェ
  1635〜1719)。はじめはマントノン侯爵夫人としてルイ14世とモンテスパン夫人の間に生まれた子どもたちの非公式な教育係となり、モンテスパン夫人の死後、ルイ
  14世と結婚した。彼女の名を冠した料理はここで言及されている\protect\hyperlink{cotelettes-maintenon}{羊のコトレット マントノン}の他、卵料理、菓子などにある。「羊のコトレット マントノン」は彼女自身が考案したとも、ルイ14世付の料理人の考案ともいわれているが、いずれも憶測の域を出ない。なお、côtelette(コトレット)とは仔牛、羊の背肉を骨付きで肋骨1本ずつに切り分けたもの。日本語では、仔羊の場合ラムチョップと呼ばれることも多い。}}{アパレイユ・マントノン}}\label{appareil-maintenon}}

\frsub{Appareils Maintenon}

\index{garniture@garniture!appareil@appareil!maintenon@appareil Maintenon}
\index{appareil@appareil!maintenon@--- Maintenon}
\index{maintenon@Maintenon!appareil@appareil ---}
\index{かるにちゆーる@ガルニチュール!あはれいゆ@アパレイユ!まんとのん@アパレイユ・マントノン}
\index{あはれいゆ@アパレイユ!まんとのん@---・マントノン}
\index{まんとのん@マントノン!あはれいゆ@アパレイユ・---}

(\protect\hyperlink{cotelettes-maintenon}{羊のコトレット マントノン}用)

\protect\hyperlink{sauce-bechamel}{ベシャメルソース}4
dLと\protect\hyperlink{sauce-soubise}{スビーズ}1
dLを半量になるまで煮詰める。

卵黄3個を加えてとろみを付ける。あらかじめマッシュルーム100
gを薄切りにしてバターでごく弱火で鍋に蓋をして蒸し煮\footnote{étuver
  (エチュヴェ)。}したものを加える。

\hypertarget{appareil-montglas}{%
\subsubsection[アパレイユ・モングラ]{\texorpdfstring{アパレイユ・モングラ\footnote{Salpicon
  à la
  Monglas(サルピコンアラモングラ)とも呼ばれものとほぼ同じ。サルピコンはせいぜい5
  mm角くらいの小さなさいの目に切ったもののこと。他の用途としては、ブシェ(パイ生地で作ったケースに詰め物をしたもの。本書ではオードブルに分類されている)やタルトレット(小さなタルト)の\ul{アパレイユ}にする。カレーム『19世紀フランス料理』にはプロフィットロール(小さな丸いパンの中身を刳り貫いたもの)にフォワグラと赤く漬けた舌肉とマッシュルームのサルピコンを詰めた「プロフィットロールのポタージュ モングラ」が掲載されている(t.1,
  p.180)。また1806年刊のヴィアール『帝国料理の本』にはローストしたペルドローの胸肉とマッシュルーム、トリュフのサルピコンをソース・エスパニョルなどであえた「ペルドロー モングラ」が掲載されている
  (pp.265-266)。それ以前の主な料理書にこの料理名は見当たらないが、17
  世紀のモングラ侯爵 François Clermont Marquis de Montglas
  (生年不詳〜1675)の名を冠したものらしい。}}{アパレイユ・モングラ}}\label{appareil-montglas}}

\frsub{Appareils à la Montglas}

\index{garniture@garniture!appareil@appareil!montglas@appareil à la Montglas}
\index{appareil@appareil!montglas@--- à l Montglas}
\index{montglas@Montglas!appareil@appareil à la ---}
\index{かるにちゆーる@ガルニチュール!あはれいゆ@アパレイユ!もんくら@アパレイユ・モングラ}
\index{あはれいゆ@アパレイユ!もんくら@---・モングラ}
\index{もんくら@モングラ!あはれいゆ@アパレイユ・---}

(\protect\hyperlink{cotelettes-monglas}{羊のコトレット モングラ}用など)

\protect\hyperlink{saumure-liquide-pour-langues}{赤く漬けた舌肉}150
g、フォワグラ150 g、茹でたマッシュルーム100 g、トリュフ100
gを通常より太めで短かい千切り\footnote{julienne (ジュリエーヌ)。}にする。

これらを、マデイラ酒風味の充分に煮詰めた\protect\hyperlink{sauce-demi-glace}{ソース・ドゥミグラス}2\undemi{}
dLであえ、バターを塗った平皿に広げて使うまでそのまま冷ます。

\hypertarget{appareil-provencal}{%
\subsubsection[プロヴァンス風アパレイユ]{\texorpdfstring{プロヴァンス風アパレイユ\footnote{\protect\hyperlink{appareil-maintenon}{アパレイユ・マントノン}からこれまでの3種のアパレイユはいずれも、羊のコトレット(ラムチョップ)の片面だけを焼いて、その表面をよく拭い、まだ焼いていない面を下にして、焼いた側の面にこれらのアパレイユを塗る、あるいは盛り上げてからオーブンに入れるという同工異曲とも言うべき仕立てに用いられる。ここで、アパレイン・マントノンとこのプロヴァンス風アパレイユの「用途」の部分の原文には動詞farcirあるいはその過去分詞farci(es)が用いられているのはとても興味深いと言えよう。farcirを日本語の「詰め物をする」と等価と考えてはうまく理解できないケースのひとつで、日本語としてはこの場合「盛る」のほうがむしろ適切だろう。}}{プロヴァンス風アパレイユ}}\label{appareil-provencal}}

\frsub{Appareils à la Provençale}

\index{garniture@garniture!appareil@appareil!provencal@appareil à la provençale}
\index{appareil@appareil!provencale@--- à la Provençale}
\index{provençal@provençale(e)!appareil@appareil à la ---e}
\index{かるにちゆーる@ガルニチュール!あはれいゆ@アパレイユ!ふろうあんすふう@プロヴァンス風アパレイユ}
\index{あはれいゆ@アパレイユ!ふろうあんすふう@プロヴァンス風---}
\index{ふろうあんすふう@プロヴァンス風!あはれいゆ@---アパレイユ}

(\protect\hyperlink{cotelettes-provencale}{羊のコトレット プロヴァンス風}用)

\protect\hyperlink{sauce-soubise}{ソース・スビーズ}5
dLを充分に固くなるまで煮詰める。潰したにんにく1片を加え、卵黄3個を加えてとろみを付ける。

\hypertarget{bordures-en-farce}{%
\subsubsection{ファルスで作る縁飾り}\label{bordures-en-farce}}

\frsub{Bordures en farce}

\index{garniture@garniture!appareil@appareil!bordures farce@bordures en farce}
\index{appareil@appareil!bordures farce@bordures en farce}
\index{bordure@bordure!farce@--- en farce}
\index{farce@farce!bordures@bordures en ---}
\index{かるにちゆーる@ガルニチュール!あはれいゆ@アパレイユ!ふあるすふちかさり@ファルスで作る縁飾り}
\index{あはれいゆ@アパレイユ!ふあるすふちかさり@ファルスで作る縁飾り}
\index{ふあるす@ファルス!ふちかざり@---で作る縁飾り}
\index{ほるてゆーる@ボルデュール ⇒ 縁飾り!ふぁるす@ファルスで作る縁飾り}
\index{ふちかさり@縁飾り!ふあるす@ファルスで作る---}

この縁飾りは、飾り付ける料理の素材とおなじ材料を中心にしたファルス\footnote{本文に指定はないが、原則としては\protect\hyperlink{farce-a}{ファルス
  A}か、\protect\hyperlink{farce-de-veau-pour-bordures}{盛り付けの縁飾りおよび底に敷いたり、詰め物をしたクネルに用いる仔牛のファルス}を用いることになるだろう。}を使う。縁飾り用の型\footnote{moule
  à
  bordure(ムーラボルデュール)、ボルデュール型ともいう。大きなリング型で、表面に山形の刻み目(浮き彫り模様)の入ったタイプ(moule
  historié ムールイストリエ、またはmoule cannelé
  ムールカヌレ)と、特に模様の入っていないプレーンなもの(moule uni
  ムールユニ) の2種に大別される。}はプレーンなものでも浮き彫り模様のあるものでもいいが、たっぷりとバターを塗ってからファルスを詰めて低めの温度で火を通す\footnote{原文pocher(ポシェ)。ここまでにも何度も出てきた表現だが、茹でる場合は「沸騰しない程度の温度で加熱すること」であり、このように型に詰めた場合には湯をはった天板に型をのせてやや低温のオーブンに入れてゆっくり加熱することになる。}。

プレーンな型を使う場合、きれいに切ったトリュフのスライスやポシェした
\footnote{原文 oeuf poché
  をそのまま訳したが、表面に飾りとして用いるのは固茹で卵の白身をスライスして型抜きあるいはナイフできれいに切ったものを使うことが多い。}卵の白身、\protect\hyperlink{saumure-liquide-pour-langues}{赤く漬けた舌肉}、ピスタチオなどで表面を装飾するといい。

浮き彫り模様の型を使うなら、上記のような装飾は省いていい。

このようなファルスで作った縁飾りを使うのはとりわけ、鶏肉料理、魚料理、牛や羊肉のソテーなど。

\hypertarget{bordures-en-legumes}{%
\subsubsection{野菜で作る縁飾り}\label{bordures-en-legumes}}

\frsub{Bordures en Légumes}

\index{garniture@garniture!appareil@appareil!bordures legumes@bordures en légumes}
\index{appareil@appareil!bordures legume@bordures en légumes}
\index{bordure@bordure!legumes@--- en légumes}
\index{legume@légume!bordures@bordures en ---s}
\index{かるにちゆーる@ガルニチュール!あはれいゆ@アパレイユ!やさいふちかさり@野菜で作る縁飾り}
\index{あはれいゆ@アパレイユ!やさいふちかさり@野菜で作る縁飾り}
\index{やさい@野菜!ふちかざり@---で作る縁飾り}
\index{ほるてゆーる@ボルデュール ⇒ 縁飾り!やさい@野菜で作る縁飾り}
\index{ふちかさり@縁飾り!やさい@野菜で作る---}

プレーンなボルデュール型の内側にたっぷりとバターを塗り、下拵えしたさまざまな野菜を型の底面と側面にシャルトルーズ\footnote{chartreuse
  本文にあるように、野菜を装飾に用いた仕立てのひとつ。シャルトル会修道院で作られている同名のスピリッツがあるが、料理におけるシャルトルーズ仕立てもシャルトル会修道院に由来しているという。シャルトル会は大斉、小斉の決まりに厳格で、野菜を多く食べる修道生活を送っていたことで有名。そのことにちなんだ仕立ての名称と言われている。この仕立ての文献上の初出は1914年刊ボヴィリエ『調理技術』第2巻の「りんごのシャルトルーズ仕立て」と思われる。これは今でいうデザートに位置するもので、りんごをサフランやアンゼリカとともに煮て黄色や緑に染め、もとの白い果肉、皮の赤など、それら色合いを組み合わせて美しく型の底面を側面に貼り付け、内部をりんごのマーマレード(≒ジャム)で満たす、というもの(t.2,
  pp.149-150)。このボヴィリエのシャルトルーズは「原型」というよりはむしろ「バリエーション」的なものであることが、レシピ本文の文面から伺える。そのため、いつごろ成立した仕立てなのかは不明だが、いずれにしてもシャルトルーズはカレームが「アントレの女王」と呼んだ程に手の込んだ華やかな仕立てとして19世紀前半には定着していた。基本的には、円筒形の型に拍子木に切ってそれぞれ下茹でしたにんじん、さやいんげん、かぶ、などの野菜をびっしりと貼りつけて崩れないようにファルスで塗り固める。その内側に、「ペルドリのシャルトルーズ」の場合は、下茹でしたサヴォイキャベツとペルドリ(ペルドロー
  ≒山うずら、の成鳥)をブレゼしたものを詰め、型の上面(提供するときは底面になる)に蓋をするようにファルスを塗ってから、湯煎にかけてファルスに火を通して固める。裏返して型から外して供する、というもの。野菜の配置、配色が重要で技術のいる仕立て(ヘリンボーンのようなパターンが比較的多かったようだ)。なお、「ペルドローのシャルトルーズ」と「ペルドリとサヴォイキャベツのブレゼ」を混同しているケースが日本でよく見られるが、シャルトルーズとはあくまでも数種類の野菜とファルスで表面を装飾する仕立てを意味しているので注意。}状に貼り付けるように敷き詰める。型の中にやや固めに作った\protect\hyperlink{farce-de-veau-pour-bordures}{じゃがいもを「つなぎ」にした仔牛のファルス}をいっぱいに詰める(\protect\hyperlink{farce-de-veau-pour-bordures}{「縁飾り用仔牛のファルス」}参照)。低めのオーブンで湯煎焼きして火を通す。

この縁飾りはもっぱら、牛、羊肉の料理で野菜のガルニチュールをともなうものに使う。

\hypertarget{bordures-en-pate-blanche}{%
\subsubsection[白い生地で作る縁飾り]{\texorpdfstring{白い生地で作る縁飾り\footnote{おなじ縁飾り(ボルデュール)でも、美味しく出来得るものと、食べもので出来てはいるけれども実際には食べないことを前提とした装飾では、本書において明らかに扱いが異なる。この「白い生地で作る縁飾り」および次項「ヌイユ生地で作る縁飾り」は後者にあたるため、さして重きを置いた説明になっていない。}}{白い生地で作る縁飾り}}\label{bordures-en-pate-blanche}}

\frsub{Bordures en pâte blanche}

\index{garniture@garniture!appareil@appareil!bordures pate blanche@bordures en pâte blanche}
\index{appareil@appareil!bordures pate blanche@bordures en pâte blanche}
\index{bordure@bordure!pate blanche@--- en pâte blanche}
\index{pate blanche@pâte blanche!bordures@bordures en ---s}
\index{かるにちゆーる@ガルニチュール!あはれいゆ@アパレイユ!しろいきしふちかさり@白い生地で作る縁飾り}
\index{あはれいゆ@アパレイユ!しろいきしふちかさり@白い生地で作る縁飾り}
\index{きし@生地!しろ@白!ふちかざり@白い生地で作る縁飾り}
\index{ほるてゆーる@ボルデュール ⇒ 縁飾り!しろいきし@白い生地で作る縁飾り}
\index{ふちかさり@縁飾り!しろいきし@白い生地で作る野菜---}
\index{はーと@パート ⇒ 生地!ふらんしゅ@ブランシュ!ふちかざり@白い生地で作る縁飾り}

片手鍋に水1 dLと塩5 g、ラード\footnote{saindoux
  (サンドゥー)精製した豚の脂}30
gを入れ、火にかけて沸騰させる。ふるった小麦粉100
gを加えて、余分な水分をとばし、大理石板の上に広げる。

捏ねながらでんぷん\footnote{原文では fécule
  (フェキュール)すなわち「でんぷん」としか指示がないが、fécule de maïs
  (フェキュールドマイス)コンスターチがいいだろう。}を練り込んでいく。10回生地を折ってから、生地を休ませる。

生地を厚さ7
mm程度にのす。これを専用の抜き型で抜いて飾りのパーツをつくる。エチューヴ\footnote{野菜などを乾燥させるためなどの目的で使用する低温で用いるオーブンの一種。}に入れて乾燥させる。これを卵白に小麦粉を加えた糊\footnote{repère
  (ルペール)ここでは小麦粉を卵白に加えて混ぜた糊のこと。通常は銀などの金属製の皿に装飾を貼り付ける際に用いる。この場合は事前に皿を熱しておき、手早く装飾のパーツを貼る。現代ではほとんど行なわれていない手法。小麦粉と水で作り鍋の蓋に目張りをするための生地も同じ用語だが、いずれのケースについても「ルペール」という用語は現代の日本の調理現場であまり多用されていない。}で皿の縁に貼り付ける。

\hypertarget{bordures-en-pate-a-nouille}{%
\subsubsection{ヌイユ生地で作る縁飾り}\label{bordures-en-pate-a-nouille}}

\frsub{Bordures en pâte à nouille}

\index{garniture@garniture!appareil@appareil!bordures pate nouille@bordures en pâte à nouille}
\index{appareil@appareil!bordures pate nouille@bordures en pâte à nouille}
\index{bordure@bordure!pate nouille@--- en pâte à nouille}
\index{nouille@nouille!bordures@bordures en pâte à ---}
\index{かるにちゆーる@ガルニチュール!あはれいゆ@アパレイユ!ぬいゆきしふちかさり@ヌイユ生地で作る縁飾り}
\index{あはれいゆ@アパレイユ!ぬいゆきしきしふちかさり@ヌイユ生地で作る縁飾り}
\index{きし@生地!ぬいゆ@ヌイユ!ふちかざり@ヌイユ生地で作る縁飾り}
\index{ほるてゆーる@ボルデュール ⇒ 縁飾り!ぬいゆきし@ヌイユ生地で作る縁飾り}
\index{ふちかさり@縁飾り!ぬいゆきし@ヌイユ生地で作る野菜---}
\index{はーと@パート ⇒ 生地!ぬいゆ@ヌイユ!ふちかざり@白い生地で作る縁飾り}

ごく固めに捏ねた\protect\hyperlink{nouilles}{ヌイユ生地}を用いて作る縁飾り。上記のように抜き型で抜いてもいいし、あるいは厚さ6〜7
mmで高さ4〜5
cmの帯状に切ってもいい。後者は「エヴィドワール」と呼ばれる専用の小さな抜き型を用いて模様をつけた帯状の生地を皿の縁にしっかりと貼り付ける。

どちらの方法でも、ヌイユ生地を用いた縁飾りには溶いた卵黄を塗ってから、乾燥させる。

\hypertarget{croutons}{%
\subsubsection{クルトン}\label{croutons}}

\frsub{Croûtons}

\index{garniture@garniture!appareil@appareil!croutons@croûtons}
\index{appareil@appareil!croutons@croûtons}
\index{かるにちゆーる@ガルニチュール!あはれいゆ@アパレイユ!くるとん@クルトン}
\index{あはれいゆ@アパレイユ!くるとん@クルトン}
\index{くるとん@クルトン}

クルトンはいわゆる食パン\footnote{フランス語でpain(パン)とだけ言う場合はバゲットに代表されるリーンなパンを指すのが普通で、イギリス式およびアメリカ式の「食パン」は
  pain de mie(パンドミー)と呼ばれて区別される。}で作る。形状や大きさは、どんな料理に合わせるかで決まってくる。これを澄ましバター\footnote{バターには少なからずカゼインなどの不純物が含まれており、それらが焦げや色むらの原因となるので、充分よく澄んだバターを使うこと。}で揚げるが、揚げるのは必ず提供直前にすること。

\hypertarget{duxelles-seche}{%
\subsubsection[デュクセル・セッシュ]{\texorpdfstring{デュクセル\footnote{俗説では17世紀にユクセル侯爵
  Marquis d'Uxelles(マルキ
  デュクセル)に料理長として仕えていたラ・ヴァレーヌが創案し、主人の名を付けたとされている。d'
  は de +
  母音の短縮形(フランス語文法ではエリジオンという)。貴族の場合は領地名の前に
  de (≒ of, from) を付ける慣習があり、爵位 de
  領地名、というのが正式な呼び名として用いられていた。Uxellesは母音で始まるからd'Uxellesとなり、それが料理用語としてひとつの単語となりduxellesとして定着したという。しかし、
  duxelles(デュクセル)あるいはそれに類似する名称が用いられるようになったのは19世紀以降であり、文献によって綴りも安定していない。19世紀末のファーヴル『料理および食品衛生事典』では
  duxel
  という綴りで項目が立てられている(なお、ファーヴルはデュクセルをアパレイユの一種と明確に定義している)。さらに時代を遡っていくと、オドの1858年版ではDurcelle(デュルセル)またはDuxelleという名称で呼ばれていると記述がある(p.167)。1856年刊グフェ『料理の本』ではd'Uxelles
  (p.72)。 1833年刊カレーム『19世紀フランス料理』第3巻には、sauce à la
  Duxelle「ソース・デュクセル」が掲載されている。これはあくまでも「ソース」ではあるが、ベースとしてマッシュルームのみじん切りを使っている点は他と同様。さらにヴィアール『王国料理の本』1820年版(p.74)
  および1814年刊ボヴィリエ『調理技術』(p.73)には、のちのデュクセル・セッシュとほぼ同様のものがDurcelleの綴りで掲載されている。カレームは\protect\hyperlink{mayonnaise}{マヨネーズ}の訳注でも見たとおり、料理名の綴りに独自のこだわりを持つ傾向が強かったので、あるいはカレームがdurcelleからduxelleへの転換点として存在している可能性はある。Durcelleの語としての成り立ちは不明だが、人名(名字)に時折見られる綴りのため、何かの由来があったことまでは推察される。以上を考慮すると、ユクセル侯爵の名を冠したという説がどんなに早くとも19世紀中葉以降のものだとわかる。フランス語の/R/と/k/の音がやや似て聞こえることがあるために、はじめdurcelleと呼ばれていたアパレイユがduxelleとなり、ひいては歴史上の人物Marquis
  d'Uxellesユクセル侯爵に結びつけられるようになった、と考えられよう。とはいえ、17世紀はいわゆるマッシュルームの人工栽培が実用化され、食材として流行した時代でもあっため、ラ・ヴァレーヌとユクセル侯爵をこのアパレイユに関連付けたもまったくの見当違いでとは言えまい。}・セッシュ\footnote{sec
  / sèche (セック/セッシュ)乾燥した、水気のない、の意。}}{デュクセル・セッシュ}}\label{duxelles-seche}}

\frsub{Duxelles sèche}

\index{garniture@garniture!appareil@appareil!duxelles seche@duxelles sèche}
\index{appareil@appareil!duxelles seche@duxelles sèche}
\index{champignon@champisnon!duxelles seche@duxelles sèche}
\index{duxelles@duxelles!seche@--- sèche}
\index{かるにちゆーる@ガルニチュール!あはれいゆ@アパレイユ!てゆくせるせつしゆ@デュクセル・セッシュ}
\index{あはれいゆ@アパレイユ!てゆくせるせつしゆ@デュクセル・セッシュ}
\index{てゆくせる@デュクセル!せつしゆ@---・セッシュ}
\index{まつしゆるーむ@マッシュルーム!てゆくせる@デュクセル!せつしゆ@デュクセル・セッシュ}

デュクセルはベースとして必ず、みじん切りにした茸を用いるが、食用のものならどんな茸でも構わない。

バター30 gと植物油30
gを鍋に熱し、玉ねぎのみじん切りとエシャロットのみじん切りを各大さじ1杯ずつ入れて、軽く炒める。マッシュルームの切りくずと軸を細かくみじん切りにしたもの250
gを加え、よく圧して水気を出させる。水分が完全に蒸発するまで強火で炒め続ける。塩こしょうで調味し、パセリのみじん切り1つまみを加えて仕上げる。陶製の器に移し入れ、バターを塗った紙で蓋をする。

デュクセル・セッシュは多くの料理で使われる。

\hypertarget{duxelles-pour-legumes-farcis}{%
\subsubsection[野菜のファルシ用デュクセル]{\texorpdfstring{野菜のファルシ\footnote{farci
  (ファルシ)詰め物をした、の意。}用デュクセル}{野菜のファルシ用デュクセル}}\label{duxelles-pour-legumes-farcis}}

\frsub{Duxelles pour légumes farcis}

\index{garniture@garniture!appareil@appareil!duxelles legumes farcis@duxelles pour légumes farcis}
\index{appareil@appareil!duxelles legumes farcis@duxelles pour légumes farcis}
\index{champignon@champisnon!duxelles legumes farcis@duxelles pour légumes farcis}
\index{duxelles@duxelles!legumes farcis@--- pour légumes farcis}
\index{かるにちゆーる@ガルニチュール!あはれいゆ@アパレイユ!やさいのふあるしようてゆくせる@野菜のファルシ用デュクセル}
\index{あはれいゆ@アパレイユ!やさいのふあるしようてゆくせる@野菜のファルシ用デュクセル}
\index{てゆくせる@デュクセル!やさいのふあるしよう@野菜のファルシ用---}
\index{まつしゆるーむ@マッシュルーム!てゆくせる@デュクセル!やさいのふあるしよう@野菜のファルシ用デュクセル}

(トマト、茸などの詰め物用)

\protect\hyperlink{duxelles-seche}{デュクセル・セッシュ}100
g、すなわち大さじ\footnote{本書における「大さじ1杯」の表現は非常にあいまいで、ざっくりとした分量表示であることに注意。}4杯を用意する。白ワイン\undemi{}
dLを加えてほぼ完全に煮詰める。次に、トマトを\protect\hyperlink{sauce-demi-glace}{ソース・ドゥミグラス}1
dLと小さめのにんにく1片をつぶしたもの、パンの身25 gを加える。

ごく弱火にかけて煮込み、詰め物をするのにちょうどいい固さになるまで煮詰める。

\hypertarget{duxelles-pour-farnitures-diverses}{%
\subsubsection{ガルニチュール用デュクセル}\label{duxelles-pour-farnitures-diverses}}

\frsub{Duxelles pour garnitures diverses}

\index{garniture@garniture!appareil@appareil!duxelles ganiture diverses@duxelles pour garniture diverses}
\index{appareil@appareil!duxelles garnitures diverses@duxelles pour garnitures diverses}
\index{champignon@champisnon!duxelles pour garnitures diverses@duxelles pour garnitures diverses}
\index{duxelles@duxelles!garnitures diverses@--- pour garnitures diverses}
\index{かるにちゆーる@ガルニチュール!あはれいゆ@アパレイユ!かるにちゆーるようてゆくせる@ガルニチュール用デュクセル}
\index{あはれいゆ@アパレイユ!かるにちゆーるようてゆくせる@ガルニチュール用デュクセル}
\index{てゆくせる@デュクセル!かるにちゆーるよう@ガルニチュール用---}
\index{まつしゆるーむ@マッシュルーム!てゆくせる@デュクセル!かるにちゆーるよう@ガルニチュール用デュクセル}

(タルトレット、玉ねぎ\footnote{玉ねぎには、完熟、乾燥させた際に表皮が黄色いタイプと白いもの、赤紫色の3系統がある。黄色系統の玉ねぎはフォンなどに用いられることが多い(日本ではこのタイプがほとんど。また「泉州黄」という品種はフランスの野菜栽培の専門書でも言及がある程に栽培特性とクオリティが高く評価されて、フランスでも栽培されている)。白玉ねぎ(oignon
  blanche
  オニョンブロンシュ)は生食やその他の調理、とりわけ小さいものは下茹でしてからバターで色よく炒めて(グラセ)ガルニチュールに用いられる。火が通りやすく、甘いものが多い。赤紫のものは品種によって特性が違うが、加熱調理、生食いずれにも用いられる。}、きゅうり\footnote{20世紀末頃から日本の種苗メーカーが育種した品種も栽培されるようになってきているため、あえて「きゅうり」と訳したが、伝統的な
  concombre (コンコンブル)は太さ4〜5 cm、長さ30〜45
  cm程度まで大きくするのが一般的で、日本の現代品種と異なり表皮は固く、苦味やアクは少ない。種の部分をスプーンなどで取り除いて、そこに詰め物して加熱調理する。また、生のまま輪切りにして食べることも多い。}、などの詰め物用)

\protect\hyperlink{duxelles-seche}{デュクセル・セッシュ}100
gに、\protect\hyperlink{farce-c}{ファルス・ムスリーヌ}または\protect\hyperlink{farce-a}{パナードを用いたファルス}もしくは\protect\hyperlink{farce-gratin-a}{ファルス・グラタン}60
gのいずれかを料理に合わせて加える。

このデュクセルを野菜の詰め物として用いた場合は、表面を焦がさないように\footnote{表面に焦げ目を付けることを
  gratiner (グラティネ)という。}、低温のオーブンに入れて加熱すること\footnote{pocher
  (ポシェ)。本来は沸騰しない程度の温度で茹でることを指すが、この場合は比較的低温のオーブンで加熱調理するという意味。}。

\hypertarget{duxelles-bonne-femme}{%
\subsubsection[デュクセル・ボヌファム]{\texorpdfstring{デュクセル・ボヌファム\footnote{bonne
  femme(ボヌファム)は「おばさん」くらいの意。家庭風、田舎風の素朴さを感じさせる料理に付けられる名称。}}{デュクセル・ボヌファム}}\label{duxelles-bonne-femme}}

\frsub{Duxelles à la bonne femme}

\index{garniture@garniture!appareil@appareil!duxelles bonne femme@duxelles à la bonne femme}
\index{appareil@appareil!duxelles legumes bonne femme@duxelles à la bonne femme}
\index{champignon@champisnon!duxelles bonne femme@duxelles à la bonne femme}
\index{duxelles@duxelles!bonne femme@--- à la bonne femme}
\index{bonne femme@bonne femme!duxelles@duxelles à la ---}
\index{かるにちゆーる@ガルニチュール!あはれいゆ@アパレイユ!てゆくせるほぬふあむ@デュクセル・ボヌファム}
\index{あはれいゆ@アパレイユ!てゆくせるほぬふあむ@デュクセル・ボヌファム}
\index{てゆくせる@デュクセル!ほぬふあむ@---・ボヌファム}
\index{ほぬふあむ@ボヌファム!てゆくせる@デュクセル・---}

(家庭料理用)

生のデュクセルに、しっかり味付けをした\protect\hyperlink{chair-a-saucisse}{ソーセージ用挽肉}を同量加えるだけ。

\hypertarget{essence-de-tomate}{%
\subsubsection{トマトエッセンス}\label{essence-de-tomate}}

\frsub{Essence de tomate}

\index{garniture@garniture!appareil@appareil!essence tomate@essence de tomate}
\index{appareil@appareil!essence tomate@essence de tomate}
\index{tomate@tomate!essence@essence de ---}
\index{essence@essence!tomate@--- de tomate}
\index{かるにちゆーる@ガルニチュール!あはれいゆ@アパレイユ!とまとえつせんす@トマトエッセンス}
\index{あはれいゆ@アパレイユ!とまとえつせんす@トマトエッセンス}
\index{とまと@トマト!えつせんす@---エッセンス}
\index{えつせんす@エッセンス!トマト---}

よく熟したトマトのジュースを漉し器で漉す。これを片手鍋に入れて、弱火にかけてゆっくりと、シロップ状になるまで煮詰める。

布で漉るが、圧したり絞らないこと。保存しておく。

\hypertarget{nota-essence-de-tomate}{%
\subparagraph{【原注】}\label{nota-essence-de-tomate}}

このトマトエッセンスはブラウン系の派生ソースの仕上げに色合いを調節するのにとても便利だ\footnote{ブラウン系の派生ソースの節で、明示的にこのトマトエッセンスの使用に言及しているレシピは2つのみだが、必ずしもそのことにこだわず、適宜、必要に応じて使うのがいい。}。

\hypertarget{fonds-de-plats}{%
\subsubsection{皿に敷いて料理をのせる台、トンポン、クルスタード}\label{fonds-de-plats}}

\frsub{Fonds de plats, Tampons et Croustades}

\index{garniture@garniture!appareil@appareil!fonds plats tampons croustades@fonds de plats, tampons et croustades}
\index{appareil@appareil!fonds plats tampons croustades@fonds de plats, tampons et croustades}
\index{fonds@fonds!plats@--- de plats} \index{tampon@tampon}
\index{croustade@croustade}
\index{かるにちゆーる@ガルニチュール!あはれいゆ@アパレイユ!さらにしいてりようりをのせるたい@皿に敷いて料理をのせる台、トンポン、クルスタードトマトエッセンス}
\index{あはれいゆ@アパレイユ!さらにしいてりようりをのせるたい@皿に敷いて料理をのせる台}
\index{たい@台!さらにしいてりようりをのせる@皿に敷いて料理をのせる---}
\index{とんぽん@トンポン} \index{くるすたーと@クルスタード}

皿に敷いて料理をのせる台、トンポン、クルスタードの重要性は日々ますます失なわれつつある。新しいサーヴィスの方式ではこれらをほぼ完全に用いなてはいない。これらの装飾的な台はパンや、一番多いケースは米を材料に作られる\footnote{実際、本書においてこれらを用いる指示は非常に少ないが、まったくないわけでもない。ただ、エスコフィエが乗り越えたいと願ったデュボワとベルナールの『古典料理』がこれらの装飾的な台の作り方にかなりのページを割いていることと比較すると、驚くほどに素気なく短かい説明で終わっている。}。

パンを使った台は、固くなったパンの身を切って作る。これをバターで揚げ
{[}\^{}31{]}、小麦粉を卵白に加えて作った糊\footnote{説明的に訳したが、原文は
  repèreの1語。\protect\hyperlink{bordures-en-pate-blanche}{白い生地で作る縁飾り}および訳注参照。}で皿の底に貼り付ける。

\hypertarget{ux7c73ux3067ux4f5cux308bux30c8ux30f3ux30ddux30f3ux3068ux30afux30ebux30b9ux30bfux30fcux30c9}{%
\subparagraph{米で作るトンポンとクルスタード}\label{ux7c73ux3067ux4f5cux308bux30c8ux30f3ux30ddux30f3ux3068ux30afux30ebux30b9ux30bfux30fcux30c9}}

\ldots{}\ldots{}パトナ米2 kgを、水が完全に澄むまでよく洗う。

たっぷりの水に入れて火にかけ、5分間茹でる。鍋の湯を捨て、別の湯に漬けて米を洗う。再度湯をきる。大きな片手鍋に丈夫で清潔な布または豚背脂のシートを敷き、入れてみょうばん10
gを加え、布または豚背脂のシートを折り畳んで米を包む。鍋に蓋をして、弱火のオーブンかエチューヴ\footnote{étuve
  主として野菜の乾燥などを目的とした低温専用のオーブン。}に入れ、3時間加熱する。

その後、米を力をこめてすり潰す。ラードを塗った布のナフキンで包んで揉み、ラードを塗った器に手早く詰めて、冷ます。

充分に冷めたら、米の塊を彫って装飾する。みょうばんを加えた水に漬けて、こまめに水を替えてやれば長期保存も可能だ。

\hypertarget{portugaise}{%
\subsubsection[ポルチュゲーズ/トマトのフォンデュ]{\texorpdfstring{ポルチュゲーズ\footnote{portugais(e)
  (ポルチュゲ/ポルチュゲーズ)は形容詞の場合は「ポルトガルの」の意。名詞の場合はポルトガル人。ここでは大文字で書き出していることから名詞と考えられる(なお現代フランス語の正書法では文頭以外の語は固有名詞のみ大文字で始めることになっており、普通名詞を文中で大文字にすることはないが、料理名などの場合は比較的自由に大文字を使う傾向にある)。すなわち「ポルトガルの女」くらいの意味にとることが可能。ちなみに、このレシピとはまったく関係ないが、、\emph{Lettres
  Portugaises}
  (レットルポルチュゲーズ)『ぽるとがる\ruby{文}{ぶみ}』という題名の本が17世紀にフランスで出版され人々の感動を誘った。リルケや佐藤春夫が自国語に翻訳、翻案したものも有名。実在したポルトガルの修道女マリアナ・アルコフォラドがフランス軍人に宛てた恋文をまとめた、事実にもとづく書簡集と考えられていたが、20世紀になってから、ガブリエル・ド・ギユラーグという男性文筆家によるまったくの創作であることが証明された。いわゆる「書簡体小説」である。とはいえ作品の文学的価値はまったく減じることのない名作。書簡体小説という形式は18世紀に流行し、ゲーテ『若きウェルテルの悩み』やラクロ『危険な関係』、ルソー『新エロイーズ』などの名作がある。19世紀前半にはその流行も落ち着き、バルザック『二人の若妻の手記』などはこの小説形式の流行の最後を飾る名作のひとつとして名高い。なお、トマトは16世紀に既にフランスにもたらされており、16世紀末に出版されたオリヴィエ・ド・セール『農業経営論』では「美しいが食べても美味しくない」と記されている。食材として広く普及したのは19世紀以降であり、爆発的な流行現象とさえいえるほどだった。第二帝政期を代表する小説家のひとりフロベールの遺作『ブヴァールとペキュシェ』にも農業に挑戦した2人の主人公がトマトの芽掻きをする必要があることを知らなかったために失敗したエピソードが描かれている。「オマール・アメリケーヌ」や「舌びらめ デュグレレ」などトマトが重要な役割を果している料理が多く創案され、フランス料理の歴史において19世紀という時代を象徴する食材のひとつともいえる。}/トマトのフォンデュ}{ポルチュゲーズ/トマトのフォンデュ}}\label{portugaise}}

\frsub{Fondue de tomate ou Portugaise}

\index{garniture@garniture!appareil@appareil!fondue tomate@fondue de tomate ou Portugaise}
\index{appareil@appareil!fondue tomate@fondue de tomate ou Portugaise}
\index{tomate@tomate!fondue@fondue de --- ou Portugaise}
\index{portugais@portugais(e)!fondue de tomate@foudue de tomate ou Portugaise}
\index{かるにちゆーる@ガルニチュール!あはれいゆ@アパレイユ!とまとのふおんてゆ@トマトのフォンデュ/ポルチュゲーズ}
\index{あはれいゆ@アパレイユ!とまとのふおんてゆ@トマトのフォンデュ/ポルチュゲーズ}
\index{かるにちゆーる@ガルニチュール!あはれいゆ@アパレイユ!ほるちゆけーす@ポルチュゲーズ/トマトのフォンデュ}
\index{あはれいゆ@アパレイユ!ほるちゆけーす@ポルチュゲーズ/トマトのフォンデュ}
\index{とまと@トマト!ふおんてゆ@---のフォンデュ/ポルチュゲーズ}
\index{ふおんてゆ@フォンデュ!とまと@トマトの---/ポルチュゲーズ}
\index{ほるちゆけーす@ポルチュゲーズ/トマトのフォンデュ}
\index{ほるとかるふう@ポルトガル風!ほるちゆけーす@ポルチュゲーズ/トマトのフォンデュ}

玉ねぎ大1個をみじん切りにしてバターまたは植物油で炒める。トマト500
gは皮を剥いて潰し、粗みじん切りにして鍋に加える。潰したにんにく1片と塩、こしょうを加える。弱火にかけて水分がすっかりなくなるまで煮詰める\footnote{トマトは品種にもよるが、混ぜずに弱火で加熱すると固形物が沈殿し、水分が上澄みになる。ここでは濃縮トマトペーストになるほどは煮詰めず、その上澄みがなくなるまで、という解釈でいいだろう。}。

時季、つまりトマトの熟し具合に応じて必要なら粉砂糖をほんの1つまみ加えるといい。

\hypertarget{kache-de-sarrazin-pour-potage}{%
\subsubsection[ポタージュ用そば粉のカシュ]{\texorpdfstring{ポタージュ用そば粉のカシュ\footnote{Kacha
  (カシャ)とも。日本語ではカーシャと呼ばれるほうが多いようだ。もとはロシアをはじめとするスラブ諸国における粥の総称でロシア語では
  \ltjsetparameter{jacharrange={-2}}каша\ltjsetparameter{jacharrange={+2}}
  。フランス料理に取り入れられ、そば粉またはセモリナ粉でつくったクレープのようなものを意味するようになった。このカシュはポタージュのガルニチュールそのもの、つまり「浮き実」となる。}}{ポタージュ用そば粉のカシュ}}\label{kache-de-sarrazin-pour-potage}}

\frsub{Kache de Sarrazin pour Potages}

\index{garniture@garniture!appareil@appareil!kache sarrazin potages@Kache de Sarrazin pour Potages}
\index{appareil@appareil!kache sarrazin potage@Kache de Sarrazin pour Poatages}
\index{kache@kache!sarrazin@--- de Sarrazin pour Potages}
\index{sarrazin@sarrazin!kache@Kache de --- pour Potages}
\index{かるにちゆーる@ガルニチュール!あはれいゆ@アパレイユ!ほたーしゆようそはこのかしゆ@ポタージュ用そば粉のカシュ}
\index{あはれいゆ@アパレイユ!ほたーしゆようそはこのかしゆ@ポタージュ用そば粉のカシュ}
\index{かしゆ@カシュ!そば粉@ポタージュ用そば粉の---}
\index{そはこ@そば粉!かしゆ@ポタージュ用---のカシュ}

(仕上がり約10人分\footnote{原文 pour un service (プランセルヴィス)
  フランス宮廷料理の時代から、ロシア式サービスの普及しはじめた頃まで、格式ある宴席での料理を作る際の単位としてserviceが用いられた。1
  service
  は概ね10人分。現実には8〜12人くらいの間で融通を効かせて運用されていたようだ。本書のレシピの分量は多くが1
  serviceすなわち約10人分で書かれている。})

粗挽きのそば粉1 kgに塩を加えたゆるま湯を7〜8
dL加えてデトランプ\footnote{ここでは動詞 détremper
  (デトロンペ)が使われているが、faire un détrempe
  (フェランデトロンプ)と同義で粉が吸水して捏ねる前の状態(塊)のこと。}を作ってまとめる。これを深手の片手鍋\footnote{casserole
  russe
  (カスロールリュス)直訳すると「ロシアの片手鍋」だが、通常は深い片手鍋をそう呼ぶ。}に入れて押し潰す。高温のオーブンに入れて約2時間加熱する。

オーブンから出したら、表面の固くなった皮の部分は取り除く。鍋の中のパン状になったものを、鍋の周囲にこびりついた焦げの部分に触れないようにして取り出す。

これにバター100 gを加えて捏ねる。厚さ1
cmになるように重しをして冷ます。直径26〜27 mm位の\footnote{原文 un
  emporte-pièce rond de la grandeur d'une pièce de 2 francs
  「2フラン硬貨の大きさの円形の抜き型」。フランはヨーロッパ通貨統合前のフランスの通貨単位。2フラン硬貨は概ね26〜27
  mm。}の丸い型抜きで抜く。これを澄ましバターで色よく焼く。オードブル皿か、ナフキンに盛り付けて供する。

\hypertarget{ux539fux6ce8}{%
\subparagraph{【原注】}\label{ux539fux6ce8}}

このカシュをオーブンから出してそのままの状態で供してもいい。その場合は専用の容器に盛りつける。

\hypertarget{kache-de-semoule-pour-coulibiac}{%
\subsubsection[クリビヤック用セモリナ粉のカシュ]{\texorpdfstring{クリビヤック\footnote{サーモンなどをブリオシュ生地で包んで焼いた料理。これを作る際に、厚さ1
  cmくらいに切ったサーモンの身とこのカシュまたは米を互いに層になるようにして、ブリオシュ生地で包んで焼く。}用セモリナ粉のカシュ}{クリビヤック用セモリナ粉のカシュ}}\label{kache-de-semoule-pour-coulibiac}}

\frsub{Kache de Semoule pour le Coulibiac}

\index{garniture@garniture!appareil@appareil!kache semoule coulibiac@Kache de Semoule pour le Coulibiac}
\index{appareil@appareil!kache semoule coulibiac@Kache de Semoule pour le Coulibiac}
\index{kache@kache!semoule@--- de Semoule pour le Coulibiac}
\index{semoule@semoule!kache@Kache de --- pour le Coulibiac}
\index{かるにちゆーる@ガルニチュール!あはれいゆ@アパレイユ!くりひやつくようせもりなこのかしゆ@クリビヤック用セモリナ粉のカシュ}
\index{あはれいゆ@アパレイユ!くりひやつくようせもりなこのかしゆ@クリビヤック用セモリナ粉のカシュ}
\index{かしゆ@カシュ!セモリナ粉@クリビヤック用セモリナ粉の---}
\index{せもりなこ@セモリナ粉!かしゆ@クリビヤック用セモリナ粉---のカシュ}

(仕上がり約10人分)

大粒のセモリナ粉200
gに溶き卵1個をよく混ぜる。天板の上に広げて弱火で乾燥させる。

これを目の粗い漉し器で裏漉しする。コンソメに入れて約20分間、沸騰しない程度の温度で加熱する\footnote{pocher
  (ポシェ)。}。気をつけて水気をきる。

\hypertarget{matignon}{%
\subsubsection[マティニョン]{\texorpdfstring{マティニョン\footnote{1935年以来首相官邸として使われているマティニョン館を18世紀に所有していたジャック・ド・マティニョンの料理人が創案したものといわれているが真偽は不明。料理用語としての初出はおそらく1856年刊デュボワ、ベルナール共著『古典料理』。ここでは「マティニョンのフォン」として、「器具でお削りおろした豚背脂、同量のバター、生ハムのスライス数枚、薄切りにしたにんじんと玉ねぎ、マッシュリュームの切りくずを弱火にかけて軽く色付くまで炒め、ローリエの葉、塩、こしょうを加えてからマデイラ酒かソテルヌのワインをひたひたに注ぎ、強火でグラス状になるまで煮詰める。これを串焼きあるいはオーブン焼きにする塊肉を覆うのに使う。魚料理の場合には豚背脂とハムをバターか植物油に代える(p.71)」となっている。これ以前の主要な料理書にmatignonの語はまったく見られないが、
  1858年版のオドにおいて「ミルポワとマティニョンは野菜と豚背脂、ハムを煮てグラス状に煮詰めたガルニチュール。鶏やジビエを串刺しでローストする際にこれで覆ってさらにバターを塗った紙で包む。高級料理でしかほとんど用いられない(p.167)」とされている。}}{マティニョン}}\label{matignon}}

\frsub{Matignon}

\index{garniture@garniture!appareil@appareil!matignon@matignon}
\index{appareil@appareil!matignon@matignon} \index{matignon@matignon}
\index{かるにちゆーる@ガルニチュール!あはれいゆ@アパレイユ!まていによん@マティニョン}
\index{あはれいゆ@アパレイユ!まていによん@マティニョン}
\index{まていによん@マティニョン}

にんじん125 g、玉ねぎ125 g、セロリ50 g、生ハム100 gを1
cm弱のさいの目に刻む。ローリエの葉1枚とタイム1枝とともに鍋に入れて、バターで弱火にかけ蓋をして蒸し煮し、少量の白ワインでデグラセする。

\hypertarget{mirepoix}{%
\subsubsection[ミルポワ]{\texorpdfstring{ミルポワ\footnote{18世紀にガストン・ピエール・レヴィ・ミルポワ公爵(1699〜1757)の料理人が考案したといわれているが真偽は不明。料理書における初出はおそらく1814年刊ボヴィリエ『調理技術』(p.61)だが、非常に厄介な問題を含んでいる。というのも、まずpoêle(ポワル)という名称のソースがあり(これがのちのà
  la poêle \textgreater{} poêlé という調理の歴史につながる)、 ~
  それは、さいの目に切った仔牛腿肉2 kgとハム750
  g、器具を使っておろすかさいの目に刻んだ豚背脂750
  g、さいの目に切ったにんじん5〜6本、玉ねぎ8個は切らずにそのまま、ブーケガルニとしてパセリ、シブール(≒葱)、クローブ、ローリエの葉2枚、タイム、バジル少々と外皮を剥いて種を取り除いたレモンのスライスを、500
  gのバターで弱火で炒め、ブイヨンかコンソメを注ぎ、4〜5時間アクを引きながら煮込み、漉す、というもの。そして、ミルポワとはこのポワルにブイヨンの\unquart{}量をシャンパーニュか上等の白ワインにして作ったもの、となっている。1833
  年のカレーム『19世紀フランス料理』第1巻におけるミルポワも同工異曲であり、さいの目に切った材料をブイヨンで煮込んで布で絞り漉したもの。さて、1856年のデュボワ、ベルナール共著『古典料理』においては「ミルポワのフォン」としてソースのベースとして掲載されている。概要は、器具を用いておろすか細かく刻んだ豚背脂300
  gを鍋に入れて溶かし、玉ねぎ1個とにんじん1本の薄切りを加え、弱火でゆっくり色付かないよう炒める。さらに大きめのさいの目に切ったハム250
  gとブーケガルニ、パセリ、マッシュルームの切りくず、にんにく、クローブを加えて2
  Lのブイヨンと\undemi{}
  Lの白ワインを注ぐ。強火にかけ沸騰したら端に寄せて弱火にし、沸騰状態を保ったまま\deuxtiers{}量まで煮詰める。最後に漉し器で漉す、というもの(pp.70-71)。1867年のグフェ『料理の本』においても「ミルポワすなわち肉と野菜のエッセンス」となっている(p.406)。つまり、デュボワとベルナールあるいはグフェの頃、つまり19世紀後半まで、ミルポワとは「出汁」の一種あるいは液体調味料のようなものだったと考えていい。前項のマティニョンの訳注でも見たように、1858年版のオドでやや違った認識がされていることは注目に値しよう。19世紀末のファーヴルの『料理および食品衛生事典』ではアパレイユとして定義している。さいの目に刻んだハム、にんじん、玉ねぎを白こしょう、タイム、バジル、ローリエの葉、クローブとともに色付くまで炒め、ソースやブレゼの調理に用いる、とある。おそらくはファーヴルの示したミルポワがもっとも本書のものに近いが、ファーヴルはマティニョンに言及していないため、さいの目に刻む大きさによって呼び名を変えているのは本書が文献上最初のものと思われる。なお、ファーヴルはミルポワ公爵の料理人が創案したという説をとっている。いずれにしてもミルポワという言葉の指す内容、用途が19世紀後半の30年くらいの間に大きく変化したと考えていいだろう。なお、現代日本の調理現場ではミルポワとマティニョンを厳密に区別することなく、また、豚背脂やハムは用いず、にんじんや玉ねぎなどの香味野菜を細かいさいの目に刻んだものをミルポワの用語で統一しているケースも多いようだ。}}{ミルポワ}}\label{mirepoix}}

\frsub{Mirepoix}

\index{garniture@garniture!appareil@appareil!mirepoix@mirepois}
\index{appareil@appareil!mirepoix@mirepoix} \index{mirepoix@mirepoix}
\index{かるにちゆーる@ガルニチュール!あはれいゆ@アパレイユ!みるほわ@ミルポワ}
\index{あはれいゆ@アパレイユ!みるほわ@ミルポワ}
\index{みるほわ@ミルポワ}

材料は\protect\hyperlink{matignon}{マティニョン}とまったく同じだが、より小さなさいの目
\footnote{brunoise (ブリュノワーズ)厳密には1〜2
  mmのさいの目に刻んだものを指す。}に刻むことと、ハムではなく塩漬け豚バラ肉の脂身の少ないところをさいの目に切って下茹でしたものを使う場合もある。

バターで色よく炒める\footnote{原文 faire revenir
  (フェールルヴニール)熱した油脂で色付くまで焼く、炒める ≒ rissoler
  (リソレ)。}。

\hypertarget{mirepoix-fine}{%
\subsubsection{ボルドー風ミルポワ}\label{mirepoix-fine}}

\frsub{Mirepoix fine, dite à la Bordelaise}

\index{garniture@garniture!appareil@appareil!mirepoix fine@mirepoix fine, dite à la Bordelaise}
\index{appareil@appareil!mirepoix fine@mirepoix fine, dite à la Bordelaise}
\index{mirepoix@mirepoix!fine@ --- fine, dite à la Bordelaise}
\index{bordelais@bordelais(e)!mirepoix fine@mirepoix fine, dite à la ---e}
\index{かるにちゆーる@ガルニチュール!あはれいゆ@アパレイユ!ほるとーふうみるほわ@ボルドー風ミルポワ}
\index{あはれいゆ@アパレイユ!ほるとーふうみるほわ@ボルドー風ミルポワ}
\index{みるほわ@ミルポワ!ほるとーふう@ボルドー風ミルポワ}
\index{ほるとーふう@ボルドー風!みるほわ@---ミルポワ}

標準的な大きさに刻んだミルポワを料理に加えると、普通は即座にその料理にふさわしい香り付けが出来るが、ボルドー風ミルポワはとりわけエクルヴィス
\footnote{ecrevisse ヨーロッパザリガニ。}やオマール\footnote{homard
  ロブスター。}の料理の風味付けにいい。これはあらかじめ用意しておくべきもので、次のように作業する。

にんじん125 gと玉ねぎ125 g、パセリ1枝を出来るだけ細かいさいの目に刻む
\footnote{原文 brunoise excessivement fine
  直訳すると「過度なまでに細かいブリュノワーズ(1〜2
  mm角のさいの目)」。}。これにタイム1つまみと粉末にしたローリエの葉1つまみを加える。

材料をバター50
gとともに片手鍋に入れ、完全に火が通るまで蓋をして弱火で蒸し煮する\footnote{étuver
  (エチュヴェ)。}。

小さな陶製の器に広げ、フォークの背を使って器に押し込む。バターを塗った白い円形の紙で蓋をして、使用するまで保存する。

\hypertarget{ux539fux6ce8-1}{%
\subparagraph{【原注】}\label{ux539fux6ce8-1}}

より細かいミルポワを作るには、材料をみじん切りにして、トーション\footnote{\protect\hyperlink{sauce-verte}{ソース・ヴェルト}訳注参照。}
の端で材料を強く圧して野菜の水気を出してしまうだけでいい。こうすると蒸し煮している間にその水分は蒸発しきれないで残る。ただし、こうしてミルポワに残った水分は、長い時間保存する場合にはカビや腐敗の原因になるので注意すること。

\hypertarget{orge-perle-pour-volailles-farcies}{%
\subsubsection[丸鶏の詰め物その他に用いる真珠麦]{\texorpdfstring{丸鶏の詰め物その他に用いる真珠麦\footnote{このレシピは第四版のみ。}}{丸鶏の詰め物その他に用いる真珠麦}}\label{orge-perle-pour-volailles-farcies}}

\frsub{Orge perlé pour volailles farcies et autres usages}

\index{garniture@garniture!appareil@appareil!orge perle@orge perlé pour volailles farcies et autres usages}
\index{appareil@appareil!orge perle@orge perlé pour volailles farcies et autres usages}
\index{orge perle@orge perlé}
\index{farce@farce!orge perle@orge perlé pour volailles farcies et autres usages}
\index{かるにちゆーる@ガルニチュール!あはれいゆ@アパレイユ!しんしゆむき@丸鶏の詰め物その他に用いるの真珠麦}
\index{あはれいゆ@アパレイユ!しんしゆむき@丸鶏の詰め物その他に用いる真珠麦}
\index{しんしゆむき@真珠麦!まるとりのつめもの@丸鶏の詰め物その他に用いる真珠麦}
\index{おおむぎ@大麦!せいはく@精白--- ⇒ 丸鶏の詰め物その他に用いる真珠麦}
\index{まるむぎ@丸麦 ⇒ 真珠麦!まるとりのつめもの@丸鶏の詰め物その他に用いる真珠麦}

玉ねぎのみじん切り75
gをバターでブロンド色になるまで炒める。皮を剥いて洗い、水気をきってさらに布で水気を取り除いた大麦250
gを加える。木のヘラで混ぜながら炒める。沸かした\protect\hyperlink{consomme-blanc}{白いブイヨン}\footnote{原文どおりに訳したが、第四版に
  bouillon blanc
  は掲載されていない。ここでは「白いコンソメ」すなわちコンソメ・サンプルと解釈するのがいいだろう。}\troisquarts{}
Lを注ぐ。こしょう1つまみを加えたら蓋をしてごく弱火のオーブンで約2時間加熱する。焦がしバター50
gをかけて仕上げる。

\hypertarget{pate-a-chou-d-office}{%
\subsubsection{調理用シュー生地}\label{pate-a-chou-d-office}}

\frsub{Pâte à chou d'office}

\index{garniture@garniture!appareil@appareil!pate chou office@pâte à chou d'office}
\index{appareil@appareil!pate chous@pâte à chou d'office}
\index{pate@pâte!chou@chou!office@pâte à chou d'office}
\index{choupate@chou (pâte)!office@pâte à --- d'office}
\index{かるにちゆーる@ガルニチュール!あはれいゆ@アパレイユ!ちようりようしゆうきし@調理用シュー生地}
\index{あはれいゆ@アパレイユ!ちようりようしゆうきし@調理用シュー生地}
\index{しゆうきし@シュー生地!ちようりよう@調理用---}
\index{きし@生地!しゆう@シュー!ちようりよう@調理用シュー生地}

水1 Lとバター200 g、塩 10
gを片手鍋に入れて火にかけ、沸騰したら火から外す。ふるった小麦粉625
gを加える。強火にかけて混ぜながら余計な水分をとばす。次に、卵の大きさによって12〜14個の全卵を生地に加える。

このシュー生地は\protect\hyperlink{pomme-dauphine}{じゃがいものドフィーヌ}やニョッキなどのアパレイユとして使用されるのがほとんどなので、通常のシュー生地よりも固く作らなくてはいけない。

\hypertarget{pate-a-frire-pour-beignet-de-cervelles}{%
\subsubsection[脳、白子のベニェやフリトー用の揚げ衣]{\texorpdfstring{脳、白子のベニェやフリトー\footnote{かえるの腿、牡蠣、ムール貝、サーモン、鶏のレバーなどをマリネして揚げ衣を付けて油で揚げた料理。friteau
  (フリトー)とも綴る。とくにfrite(s)(フリット)と混同しないように注意したい。名詞としての
  fritesはフライドポテトのこと。過去分詞(形容詞)としての
  frit(e)(フリ/フリット)は「油で揚げた」の意。例えばcourgette
  frite(クルジェットフリット)は油で揚げたズッキーニのこと(friteは形容詞)だが、steak
  frites(ステックフリット)フライドポテト添えのステーキを意味する(この場合のfritesは名詞)。}用の揚げ衣}{脳、白子のベニェやフリトー用の揚げ衣}}\label{pate-a-frire-pour-beignet-de-cervelles}}

\frsub{Pâte à frire pour Beignets de cervelles et de laitances, fritots, etc.}

\index{garniture@garniture!appareil@appareil!pate frire beignets cervelles@pâte à frire pour Beignets de cervelles et de laitances, fritots, etc.}
\index{appareil@appareil!pate frire beignets cervelles@pâte à frire pour Beignets de cervelles et de laitances, fritots, etc.}
\index{pate@pâte!frire@--- à frire!beignet cervelles@pâte à frire pour Beignets de cervelles et de laitances, fritots, etc.}
\index{frire@frire!pate@pâte à ---!beignet cervelles@pâte à frire pour Beignet de cervelles et de laitances, fritots; etc.}
\index{fritot@fritot!pate frire@pâte à frire pour Beignet de cervelles et de laitances, fritots; etc.}
\index{cervelle@cervelle!pate frire@pâte à frire pour Beignet de cervelles et de laitances, fritots; etc.}
\index{laitance@laitance!pate frire@pâte à frire pour Beignet de cervelles et de laitances, fritots; etc.}
\index{かるにちゆーる@ガルニチュール!あはれいゆ@アパレイユ!のうしらこのへにえ@脳、白子のベニェやフリト用の揚げ衣}
\index{あはれいゆ@アパレイユ!のうしらこのへにえ@脳、白子のベニェやフリト用の揚げ衣}
\index{あけころも@揚げ衣!のうしらこのへにえ@脳、白子のベニェやフリト用の揚げ衣}
\index{ふりと@フリト!あけころも@揚げ衣!のうしらこのへにえ@脳、白子のベニェやフリト用の揚げ衣}
\index{のう@脳!あけころも@揚げ衣!のうしらこのへにえ@脳、白子のベニェやフリト用の揚げ衣}
\index{しらこ@白子!あけころも@揚げ衣!のうしらこのへにえ@脳、白子のベニェやフリト用の揚げ衣}

陶製の器に、ふるった小麦粉125 g、塩1つまみ、植物油か溶かしバター大さじ
2杯、微温湯2
dLを入れる。木のヘラで生地を持ち上げながら混ぜる。すぐに使う場合は決して生地を捏ねまわさないこと。弾力が出てしまい、揚げる具材を漬けたときに生地が上手く付かなくなってしまうからだ。事前に用意しておく場合には、捏ねまわしても大丈夫。生地を休ませている間に弾力性は失なわれる\footnote{いったん形成されたグルテンはそうそう崩れないので、内容としてはやや疑問に思う部分だが、日本の「てんぷら」の常識をここで適用すべきではない。実際のところ、フリトーの衣はグルテンが形成されていてもまったく問題ないだろう。}。

この生地は、使う直前に、ふんわりと泡立てた卵白2個分を加える。

\hypertarget{pate-a-frire-pour-legumes}{%
\subsubsection{野菜用の揚げ衣}\label{pate-a-frire-pour-legumes}}

\frsub{Pâte à frire pour Légumes}

\index{garniture@garniture!appareil@appareil!pate frire legumes@pâte à frire pour Légumes}
\index{appareil@appareil!pate frire legumes@pâte à frire pour Légumes}
\index{pate@pâte!frire@--- à frire!legumes@pâte à frire pour Légumes}
\index{frire@frire!pate@pâte à ---!legumes@pâte à frire pour Légumes}
\index{legumes@legumes!pate frire@pâte à frire Légumes}
\index{salsifis@salsifis!pate frire@pâte à frire pour Légumes}
\index{celeris@céleris!pate frire@pâte à frire pour Légumes}
\index{crosnes@crosnes!pate frire@pâte à frire pour Légumes}
\index{かるにちゆーる@ガルニチュール!あはれいゆ@アパレイユ!やさいようのあけころも@野菜用の揚げ衣}
\index{あはれいゆ@アパレイユ!やさいようの@野菜用の揚げ衣}
\index{あけころも@揚げ衣!やさいよう@野菜用の揚げ衣}
\index{さすしふい@サルシフィ!あけころも@揚げ衣!やさいよう@野菜用の揚げ衣}
\index{せろり@セロリ!あけころも@揚げ衣!やさいよう@野菜用の揚げ衣}
\index{ちよろき@チョロギ!あけころも@揚げ衣!やさいよう@野菜用の揚げ衣}

(サルシフィ\footnote{salsifis
  キク科の根菜、見た目は牛蒡に似ているが風味や調理特性はまったく異なる。}、セロリ、クローヌ\footnote{crosne
  ちょろぎ。シソ科の根菜(正確には塊茎が食用となる)。中国原産で日本には江戸時代に伝わった。同様に中国からヨーロッパにも伝わり、フランスでは最初に栽培された地名からcrosne、あるいは日本由来のものとしてcrosne
  du
  Japon(クローヌデュジャポン)と呼ばれる。絵画などの分野でジャポニスムが流行したこともあって、日本の食材として注目を浴びたためか、(à
  la) japonaise (アラ
  ジャポネーズ)「日本風」を冠したものには、このちょろぎを用いた料理が多い。}など)

陶製の器に小麦粉125
gと塩1つまみ、溶かしバター大さじ2杯、全卵1個、水適量を混ぜて薄めの衣をつくる。

出来るだけ、1時間前に用意しておくこと。

\hypertarget{riz-pour-farcir-les-volailles-servies-en-releve-ou-en-entree}{%
\subsubsection{大皿仕立ての丸鶏に詰める米}\label{riz-pour-farcir-les-volailles-servies-en-releve-ou-en-entree}}

\frsub{Riz pour farcir les volailles servies en Relevé ou en Entrée}

\index{garniture@garniture!appareil@appareil!riz farcir@riz pour farcir les volailles servies en relevé ou en entrée}
\index{appareil@appareil!riz farcir@riz pour farcir les volailles servies en relevé ou en entrée}
\index{riz@riz!farcir@--- pour farcir les volailles servies en Relevé ou en Entrée}
\index{farce@farce!riz@pour farcir les volailles servies en Relevé ou en Entrée}
\index{かるにちゆーる@ガルニチュール!あはれいゆ@アパレイユ!おおさらしたてのまるとりにつめるこめ@大皿仕立ての丸鶏に詰める米}
\index{あはれいゆ@アパレイユ!おおさらしたてのまるとりにつめるこめ@大皿仕立ての丸鶏に詰める米}
\index{ふあるす@ファルス!おおさらしたてのまるとりにつめるこめ@大皿仕立ての丸鶏に詰める米}
\index{こめ@米!つめもの@大皿仕立ての丸鶏に詰める米}

玉ねぎ\undemi{}個のみじん切りをバター50
gでさっと炒める。カロライナ米またはパトナ米250
gを加え、米が白くなるまで混ぜながら炒める。

\protect\hyperlink{consomme-blanc}{白いコンソメ}\undemi{}
Lを注ぎ、蓋をして15分間煮る。生クリーム1\undemi{}
dLとフォワグラの脂\footnote{フォワグラのテリーヌなどを作る際に余分な脂が出るのでそれを利用するといい。}またはバター125
g、\protect\hyperlink{sauce-supreme}{ソース・シュプレーム}大さじ数杯と、この米を詰める鶏料理に添えることになっているガルニチュールの一部を加える。

\hypertarget{ux539fux6ce8-2}{%
\subparagraph{【原注】}\label{ux539fux6ce8-2}}

米は鶏を焼いている間に完全に火が通るよう、詰め物をする段階では\troisquarts{}程度に火が通っているようにする。鶏に詰めた米は膨らむので、きっちりとは鶏に詰め込まないこと。

\hypertarget{salpicons-divers}{%
\subsubsection[サルピコン]{\texorpdfstring{サルピコン\footnote{この項は第二版で全面的に書き換えられ、分量も大幅に増えた。初版の記述は以下のとおり。「この用語は一般的に火を通した肉、フアルス、マッシュルーム、トリュフなどをさいの目切りにしたもののこと。大きさは合わせる料理に応じて加減する。たった1種類の肉、あるいは野菜のさいの目切りにしたものでもサルピコンと呼ぶ。(例)フォワグラのサルピコン、トリュフのサルピコン、など(p.188)」。}}{サルピコン}}\label{salpicons-divers}}

\frsub{Salpicons divers}

\index{garniture@garniture!appareil@appareil!salpicons divers@salpicons divers}
\index{appareil@appareil!salpicon divers@salpicons divers}
\index{salpicon@salpicon}
\index{かるにちゆーる@ガルニチュール!あはれいゆ@アパレイユ!さるひこん@サルピコン}
\index{あはれいゆ@アパレイユ!さるひこん@サルピコン}
\index{さるひこん@サルピコン}

サルピコンという用語は普通、ある調理の種類を指すものと理解されよう。

サルピコンにはサンプルとコンポゼ\footnote{simple
  (サンプル)単一の、シンプルな。composé(e) (コンポゼ)組み合わせた。}がある。

素材が1種類だけの場合はサンプルと呼ぶ。例えば鶏やジビエの肉、羊や牛の肉、仔牛胸腺肉\footnote{ris
  de veau (リドヴォ))}、あるいはフォワグラ、魚、甲殻類、ハム、舌肉など。

素材が複数からなる場合はコンポゼと呼ぶ。本書に掲載されている組み合わせのほか、相性のよさそうなものの組み合わせ、マッシュルームやトリュフで嵩を増したもの、などがそうだ。

サルピコンの作り方は、各種の素材を、小さな規則正しいサイズ、すなわち一辺が0.5
cm程度のさいの目に刻む。

各種サルピコンのレシピ集を作るとしたら\footnote{原文は直説法現在という時制で書かれており「事実を述べる」ニュアンスだが、本書にサルピコンのレシピをまとめた章も節もないため、やや仮定法的に訳した。なお『ラルース・ガストロノミック』初版には
  salpiconの項に代表的なレシピがまとめられている。}、上記のような素材の組み合わせから始まり、それによって使い途も名称も決まることになる。例えば\footnote{ここに挙げられている例が第二版での加筆者の「思い付き」かそれとも本書の全体の構想にかかわるものだったかは不明だが、結果として本書第四版にはおろか肝心の第二版にさえ具体的な素材が記されていない例が含まれている。「ロワイヤル」と「シャスール」がそれにあたる。以下、ひとつずつ見ていくと、(1)ロワイヤルroyale
  王宮風、王家風、の意で、これほど料理そのものと関連なく料理名に濫用されている語も珍しいとさえ言えるが、サルピコン・ロワイヤルsalpicon
  à la
  royaleの場合は『ラルース・ガストロノミック』初版によると「トリュフとマッシュルームを鶏のピュレであえたもの」を指す。(2)フィナンシーエル
  salpicon à la
  financière\ldots{}\ldots{}クネル、雄鶏のとさかとロニョン(精巣)、マッシュルーム、トリュフ、すなわち\protect\hyperlink{garniture-financiere}{ガルニチュール・フィナンシエール}の構成素材をさいの目に切ってを煮詰めたソース・フィナンシエールであえたもの。(3)パリ風
  salpicon à la
  parisienneは\protect\hyperlink{garniture-parisienne}{パリ風ガルニチュール}参照。(4)
  モングラsalpicon à la
  Monglasは本節冒頭の\protect\hyperlink{appareil-montglas}{アパレイユ・モングラ}そのもの。(5)シャスールsalpicon
  chasseur\ldots{}\ldots{}さいの目に切ってバターで炒めた鶏のレバーとマッシュルームを、煮詰めた\protect\hyperlink{sauce-chasseur}{ソース・シャスール}であえたもの。}\ul{ロワイヤル}、\ul{フィ
ナンシエール}、\ul{パリ風}、\ul{モングラ}、\ul{シャスール}など。

\hypertarget{ux30d4ux30edux30b7ux30adux7528ux30c8ux30f4ux30a1ux30edux30fcux30b072}{%
\subsubsection[ピロシキ用トヴァローグ]{\texorpdfstring{ピロシキ用トヴァローグ\footnote{このレシピは第三版から。なお第四版=現行版の綴りはtawrogueになっているが、明らかに第三版にあるtwarogueの誤植。ロシア語の綴りは
  \ltjsetparameter{jacharrange={-2}}творог\ltjsetparameter{jacharrange={+2}}。}}{ピロシキ用トヴァローグ}}\label{ux30d4ux30edux30b7ux30adux7528ux30c8ux30f4ux30a1ux30edux30fcux30b072}}

\frsub{Twatogue pour Piroguis}

\index{garniture@garniture!appareil@appareil!twarogue piroguis@twarogue pour piroguis}
\index{appareil@appareil!twarogue piroguis@twarogue pour piroguis}
\index{twarogue@twarogue}
\index{cuisine russe@cuisine russe!twarogue@twarogue pour piroguis}
\index{かるにちゆーる@ガルニチュール!あはれいゆ@アパレイユ!ひろしきようとうあろーく@ピロシキ用トヴァローグ}
\index{あはれいゆ@アパレイユ!ひろしきようとうあろーく@ピロシキ用トヴァローグ}
\index{とうあろーく@トヴァローグ}
\index{ろしあふう@ロシア風!ひろしきようとうあろーく@ピロシキ用トヴァローグ}

よく水気をきったフロマージュ・ブラン\footnote{ヨーグルトに見た目のよく似た半固形チーズ。デザートなどとして砂糖をかけて食べるなどが一般的。}250
gをナフキンでしっかり絞る。これを陶製の器に入れ、ヘラで滑らかになるまで練る。あらかじめ捏ねてポマード状に柔らかくしておいたバター250
gと全卵1個を加える。

塩、こしょうで調味する。
\end{recette}