\hypertarget{ux30a8ux30b9ux30b3ux30d5ux30a3ux30a8ux306eux65b0ux89e3ux91c8-ux53c2ux8003ux4f8b}{%
\chapter{エスコフィエの新解釈 --- 参考例
---}\label{ux30a8ux30b9ux30b3ux30d5ux30a3ux30a8ux306eux65b0ux89e3ux91c8-ux53c2ux8003ux4f8b}}

\hypertarget{ux524dux83dc}{%
\section{前菜}\label{ux524dux83dc}}
\begin{recette}
\hypertarget{bouchees}{%
\subsubsection{ブシェ}\label{bouchees}}

\frsub{Bouchées}

通常、ブシェをメニューの「温製オードブル」に位置付ける場合には、標準的なブシェよりも小さいサイズのものにしなくてはいけない。そのうえで、「かわいらしいブシェ」のように明記される。形状はどんな仕上りのものにするかでいろいろに変えてやり、大きなブシェを切った場合とは全然違うものであるとわかるようにすること。

場合によっては、ブシェの蓋の部分は残して蓋にするが、スライスしたまま、あるいは飾り切りをしたトリュフを蓋にすることもあるし、また別の場合には、詰めものの一部を蓋として利用することもある。

ブシェは必ずナフキンの上にのせて供すること。

\hypertarget{bouchee-a-la-reine}{%
\subsubsection{ブシェ・王妃風}\label{bouchee-a-la-reine}}

\frsub{Bouchée à la Reine}

この種のブシェの、クラシックな、本来の詰め物は生クリーム入りの鶏のピュレが用いられていた。だが、こんにちでは鶏胸肉とマッシュルーム、トリュフを1〜2mm角の細かいみじん切りにしたものを\protect\hyperlink{sauce-allemande}{ソース・アルマンド}であえたもので代用されている。ほとんど全ての調理現場では詰め物に後者を用いるようになってしまった。このブシェの形状は必ず円形で、縁に波形の模様が入ったものであること。

\begin{center}\rule{0.5\linewidth}{\linethickness}\end{center}

\hypertarget{epinards-a-la-viroflay}{%
\subsubsection[ほうれんそう・ヴィロフレー]{\texorpdfstring{ほうれんそう・ヴィロフレー\footnote{パリ郊外南西のヴェルサイユ近くの地名。ほうれんそうの栽培で有名で、ヴィロフレーという名称の伝統品種もある。}}{ほうれんそう・ヴィロフレー}}\label{epinards-a-la-viroflay}}

\frsub{Epinards à la Viroflay}

布の上に下茹でしたほうれんそうの葉(大)を広げる。それぞれの葉の中心に「ほうれんそうのシュブリック」を置く。このシュブリックにはパンの身をバターで揚げた小さなクルトンを混ぜ込んでおくこと。シュブリックをほうれんそうの葉で丸くなるように包む。これをバターを塗ったグラタン皿に並べ、\protect\hyperlink{sauce-mornay}{ソース・モルネー}を覆いかける。上からおろしたチーズを振りかけ、溶かしバターをかけてやり、高温のオーブンでこんがり焼く。

\hypertarget{subric-d-epinards}{%
\subsubsection{ほうれんそうのシュブリック}\label{subric-d-epinards}}

\frsub{Subric d'épinards}

ほうれんそうは上述のとおり\footnote{「ほうれんそうのクリームあえ」参照。ほうれんそうは下茹でして水にはなしてから、水気を絞り、みじん切りにするか裏漉ししてから、ほうれんそう500
  gあたりバター60
  gとともにソテー鍋に入れて強火にかけ、余計な水分をとばす。}にバターを加えて強火にかけて水気をとばす。鍋を火からはずし、ほうれんそう500
gあたり、濃い\protect\hyperlink{sauce-bechamel}{ベシャメルソース}1
dL、クレーム・エペス大さじ2杯、溶きほぐした全卵1
個と卵黄3個、塩、こしょう、ナツメグを加える。フライパンにたっぷりのバターを熱して充分な量の澄ましバターを用意する。

ほうれんそうでつくったタネをスプーンで掬いとり、指で押し出すようにして澄ましバターの中に落としていく。シュブリックの成形をこのまま続けていくが、隣り同士で触れ合わないように注意すること。1分程焼いたら、パレットナイフかフォークで反対側の面にも焼き色を付けてやる。これをメインの料理の皿か野菜料理用の皿に盛り、ソース・クレームを別添で供する。

\hypertarget{nota-subric-d-epinards}{%
\subparagraph{【原注】}\label{nota-subric-d-epinards}}

シュブリックのタネには別の作り方もある。バターを加えてほうれんそうの水気をとばしたら、ほうれんそうと同量の、やや濃い目に作ったクレープ生地を混ぜ込む。
\end{recette}
\begin{center}\rule{0.5\linewidth}{\linethickness}\end{center}

\hypertarget{Potages}{%
\section{ポタージュ}\label{Potages}}
\begin{recette}
\hypertarget{consomme-rabelais}{%
\subsubsection[コンソメ・ラブレー]{\texorpdfstring{コンソメ・ラブレー\footnote{フランスのルネサンス期を代表する人文主義者、小説家であり医師でもあったフランソワ・ラブレー(?〜1553)のこと。なおこのレシピは第四版のみで、初版は「ジビエのコンソメにヴヴレ産白ワイン2
  dLを煮詰めて加える(コンソメ4
  Lあたり)。浮き実は小さな棒状にしたトリュフ風味のひばりの小さなクネルと、セロリの千切りをコンソメで軽く煮たもの
  (p.23)」。第二版では「ジビエのコンソメに、1
  Lあたりヴヴレ産白ワイン\(\frac{1}{2}\)
  dLを煮詰めて加える。浮き実\ldots{}\ldots{}トリュフを加えたひばりのファルスを刻み模様の付いた口金で絞り出したクネル。セロリの千切りをコンソメで軽く煮たもの(p.170)」となっているが、第三版にこの名称のレシピは掲載されていない。なお、ラブレーはシノン郊外の生まれであるため、トゥーレーヌ産のワイン(とりわけシノンの赤)が引き合いに出されることが多い。}}{コンソメ・ラブレー}}\label{consomme-rabelais}}

\begin{itemize}
\item
  鶏のコンソメにペルドローのフュメを加える。
\item
  浮き実\ldots{}\ldots{}\protect\hyperlink{farce-c}{生クリーム入りペルドローのファルス}をコーヒースプーンで成形し、提供直前にやや沸騰しない程度の温度で火を通した\footnote{pocher
    (ポシェ)。}クネル。マデイラ酒風味で火を通したトリュフの細い千切り\footnote{fine
    julienne (フィーヌジュリエンヌ)。}。
\item
  別添\ldots{}\ldots{}パルメザン風味の小さなプロフィットロール。
\end{itemize}
\end{recette}