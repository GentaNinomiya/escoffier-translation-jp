\hypertarget{ux30a8ux30b9ux30b3ux30d5ux30a3ux30a8ux306eux65b0ux89e3ux91c8-ux53c2ux8003ux4f8b}{%
\chapter{エスコフィエの新解釈 --- 参考例
---}\label{ux30a8ux30b9ux30b3ux30d5ux30a3ux30a8ux306eux65b0ux89e3ux91c8-ux53c2ux8003ux4f8b}}

\hypertarget{ux524dux83dc}{%
\section{前菜}\label{ux524dux83dc}}
\begin{recette}
\hypertarget{bouchees}{%
\subsubsection{ブシェ}\label{bouchees}}

\frsub{Bouchées}

通常、ブシェをメニューの「温製オードブル」に位置付ける場合には、標準的なブシェよりも小さいサイズのものにしなくてはいけない。そのうえで、「かわいらしいブシェ」のように明記される。形状はどんな仕上りのものにするかでいろいろに変えてやり、大きなブシェを切った場合とは全然違うものであるとわかるようにすること。

場合によっては、ブシェの蓋の部分は残して蓋にするが、スライスしたまま、あるいは飾り切りをしたトリュフを蓋にすることもあるし、また別の場合には、詰めものの一部を蓋として利用することもある。

ブシェは必ずナフキンの上にのせて供すること。

\hypertarget{bouchee-a-la-reine}{%
\subsubsection{ブシェ・王妃風}\label{bouchee-a-la-reine}}

\frsub{Bouchée à la Reine}

この種のブシェの、クラシックな、本来の詰め物は生クリーム入りの鶏のピュレが用いられていた。だが、こんにちでは鶏胸肉とマッシュルーム、トリュフを1〜2
mm角の細かいみじん切りにして\protect\hyperlink{sauce-allemande}{ソース・アルマンド}であえたもので代用されている。ほとんど全ての調理現場では詰め物に後者を用いるようになってしまった。このブシェの形状は必ず円形で、縁に波形の模様が入ったものであること。

\begin{center}\rule{0.5\linewidth}{\linethickness}\end{center}

\hypertarget{epinards-a-la-viroflay}{%
\subsubsection[ほうれんそう・ヴィロフレー]{\texorpdfstring{ほうれんそう・ヴィロフレー\footnote{パリ郊外南西のヴェルサイユ近くの地名。ほうれんそうの栽培で有名で、ヴィロフレーという名称の伝統品種もある。}}{ほうれんそう・ヴィロフレー}}\label{epinards-a-la-viroflay}}

\frsub{Epinards à la Viroflay}

布の上に下茹でしたほうれんそうの葉(大)を広げる。それぞれの葉の中心に「ほうれんそうのシュブリック」を置く。このシュブリックにはパンの身をバターで揚げた小さなクルトンを混ぜ込んでおくこと。シュブリックをほうれんそうの葉で丸くなるように包む。これをバターを塗ったグラタン皿に並べ、\protect\hyperlink{sauce-mornay}{ソース・モルネー}を覆いかける。上からおろしたチーズを振りかけ、溶かしバターをかけてやり、高温のオーブンでこんがり焼く。

\hypertarget{subric-d-epinards}{%
\subsubsection{ほうれんそうのシュブリック}\label{subric-d-epinards}}

\frsub{Subric d'épinards}

ほうれんそうは上述のとおり\footnote{「ほうれんそうのクリームあえ」参照。ほうれんそうは下茹でして水にはなしてから、水気を絞り、みじん切りにするか裏漉ししてから、ほうれんそう500
  gあたりバター60
  gとともにソテー鍋に入れて強火にかけ、余計な水分をとばす。}にバターを加えて強火にかけて水気をとばす。鍋を火からはずし、ほうれんそう500
gあたり、濃い\protect\hyperlink{sauce-bechamel}{ベシャメルソース}1
dL、クレーム・エペス大さじ2杯、溶きほぐした全卵1
個と卵黄3個、塩、こしょう、ナツメグを加える。フライパンにたっぷりのバターを熱して充分な量の澄ましバターを用意する。

ほうれんそうでつくったアパレイユをスプーンで掬いとり、指で押し出すようにして澄ましバターの中に落としていく。シュブリックの成形をそのまま続けていくが、隣り同士で触れ合わないように注意すること。1分程焼いたら、パレットナイフかフォークで反対側の面にも焼き色を付けてやる。これをメインの料理の皿か野菜料理用の皿に盛り、ソース・クレームを別添で供する。

\hypertarget{nota-subric-d-epinards}{%
\subparagraph{【原注】}\label{nota-subric-d-epinards}}

シュブリックのアパレイユには別の作り方もある。バターを加えてほうれんそうの水気をとばしたら、ほうれんそうと同量の、やや濃い目に作ったクレープ生地を混ぜ込む。
\end{recette}
\newpage

\hypertarget{Potages}{%
\section{ポタージュ}\label{Potages}}
\begin{recette}
\hypertarget{consomme-rabelais}{%
\subsubsection[コンソメ・ラブレー]{\texorpdfstring{コンソメ・ラブレー\footnote{フランスのルネサンス期を代表する人文主義者、小説家であり医師でもあったフランソワ・ラブレー(?〜1553)のこと。なおこのレシピは第四版のみで、初版は「ジビエのコンソメにヴヴレ産白ワイン2
  dLを煮詰めて加える(コンソメ4
  Lあたり)。浮き実は小さな棒状にしたトリュフ風味のひばりの小さなクネルと、セロリの千切りをコンソメで軽く煮たもの
  (p.23)」。第二版では「ジビエのコンソメに、1
  Lあたりヴヴレ産白ワイン\(\frac{1}{2}\)
  dLを煮詰めて加える。浮き実\ldots{}\ldots{}トリュフを加えたひばりのファルスを刻み模様の付いた口金で絞り出したクネル。セロリの千切りをコンソメで軽く煮たもの(p.170)」となっているが、第三版にこの名称のレシピは掲載されていない。なお、ラブレーはシノン郊外の生まれであるため、トゥーレーヌ産のワイン(とりわけシノンの赤)が引き合いに出されることが多い。}}{コンソメ・ラブレー}}\label{consomme-rabelais}}

\begin{itemize}
\item
  鶏のコンソメにペルドローのフュメを加える。
\item
  浮き実\ldots{}\ldots{}\protect\hyperlink{farce-c}{生クリーム入りペルドローのファルス}をコーヒースプーンで成形し、提供直前に沸騰しない程度の温度で火を通した\footnote{pocher
    (ポシェ)。}クネル。マデイラ酒風味で火を通したトリュフの細い千切り\footnote{fine
    julienne (フィーヌジュリエンヌ)。}。
\item
  別添\ldots{}\ldots{}パルメザン風味の小さなプロフィットロール。
\end{itemize}

\hypertarget{puree-conde}{%
\subsubsection[ピュレ・コンデ]{\texorpdfstring{ピュレ・コンデ\footnote{ブルボン王家の支流にあたる
  Prince de Condée
  (プランスドコンデ)コンデ大公のこと。赤いんげん豆のポタージュにコンデの名称を冠したのは文献上はおそらくヴィアール『帝国料理の本』(1806年)が初出。
  (Potage) A la Condé
  となっている。また、18世紀以前の料理書において赤いんげん豆のポタージュはほとんど見つからない。よく知られているように、いんげん豆はアメリカ大陸原産で16世紀くらいにはフランスに伝えられていたはずだが、広まるのに時間がかかったようだ。さて、ヴィアールのレシピの概要は、1リトロン(≒0.8
  L)の赤いんげん豆をブイヨンで煮る。にんじん2本、玉ねぎ2個、ポタージュの浮き脂少々、クローブ2本を加える。豆が煮えたら裏漉しして滑らかなピュレにする。これをバターで揚げたパンのクルートの上に注いで供する(p.8)。この本にはレンズ豆のピュレのポタージュも続けて掲載されているが、そこにコンティの名はなく、たんに「レンズ豆のピュレのポタージュ」と称されているのみ。作り方上述のコンデとほぼ同じ。ヴィアールでは『料理の手引き』に近い非常にシンプルなレシピだが、カレーム『19世紀フランス料理』第1巻(1833年)の「赤いんげん豆のピュレのポタージュ・コンデ風」は、1
  \(\frac{1}{2}\)
  Lの赤いんげん豆の殻を剥いて洗う。これを大鍋に入れて、ペルドリ1羽、バイヨンヌの生ハム1切れ、にんじん2本、玉ねぎ2個、ブイヨン適量を加える。火にかけて煮ながらアクを取る。ペルドリに火が通ったらすぐに、ハムや他の根菜とともに取り出す。豆は煮汁ごと布で漉す。このピュレをごく標準的な鶏のコンソメに流し入れ、粗く砕いたこしょう
  1つまみ加えて弱火で煮込む。フルノーの端に鍋を置いて弱火で2時間程、アクを取りながら煮込む。その後スープ入れに移し、バターで揚げたクルトンを入れておいた各自のスープ皿に供する(p.144)。この本では赤いんげん豆のポタージュには「コンデ風」の名称が付けられているが、その次のレシピ「白いんげん豆のピュレのポタージュ」はというだけの単純な名称になっている。ブルジョワ料理の本として19世紀から20世紀初頭まで版を重ねたオド『女性料理人のための本』第15版(1834年)では早くも「ポタージュ・コンデ風」として簡単にだが赤いんげん豆のピュレのポタージュのレシピが掲載されている。その一方で、レンズ豆を用いたポタージュについては1909年の第97版に至るまでレンズ豆のピュレのポタージュは掲載されているが「コンティ」の名は冠されていない。}}{ピュレ・コンデ}}\label{puree-conde}}

\frsub{Purée Condé}

赤いんげん豆は塩18 gを加えた冷水1 \(\frac{1}{2}\)
Lに入れて火にかける。沸騰したら、しっかりアクを取り\footnote{écumer
  (エキュメ)浮いてくる泡を取り除く、が原義。}、赤ワイン2
\(\frac{1}{2}\)
dLを沸かしてから加える。ブーケガルニ、クローブを刺した玉ねぎ1個、四つ割りに切ったにんじん1本を加えて弱火にして煮込む。いんげん豆がよく煮えたら、煮汁から出して、ブーケガルニと玉ねぎ、にんじんは取り除く。いんげん豆を丁寧にすり潰す。煮汁でのばしてから布で漉し、提供直前にバターを加える。

\hypertarget{puree-conti}{%
\subsubsection[ピュレ・コンティ]{\texorpdfstring{ピュレ・コンティ\footnote{上記コンデ大公家のさらに傍流。王家の分家の分家という扱いになるが、
  Prince de Conti
  (プランスドコンティ)の称号を持つ。ポタージュにコンティの名が冠されたのは、上記コンデの名よりずっと早く、ムノン『宮廷の晩餐』(1755年)第1巻に「ポタージュ・コンティ風」とある。ただしこれはレンズ豆を材料にしたポタージュではなく、スライスした玉ねぎを炒めて煮込み、スープ入れの底にバターで揚げたパン(クルート)を敷いてその上に盛り、刻んだアンチョビを玉ねぎに散らすというもの
  (pp.91-92)。ボヴィリエの『調理技術』(1814年)第1巻では「レンズ豆のピュレのポタージュ・王妃風」と出ている。作り方はえんどう豆のピュレのポタージュと同様にするが、赤レンズ豆を用いて「王妃風」を謳う場合は、上手に煮込んできれいな赤色に仕上げるべし、とある(p.22)。「レンズ豆のポタージュ・コンティ風」の名称が出てくるのはカレーム『19世紀フランス料理』第1巻。1
  \(\frac{1}{2}\) Lの赤レンズ豆 (lentilles à la
  reine)の殻を剥いて洗う。下茹でしたハム、ペルドリ1羽、にんじん2
  本、蕪1個、玉ねぎ2個、ポワロー2本を束ねたものとセロリの根元1株を加え、適量のブイヨンを注いで煮る。アクを取り、3時間弱火で煮込む。根菜、ペルドリ、ハムを取り出してから、レンズ豆を布で漉す。このピュレを普段のとおり作ったコンソメに加える。沸騰したらフルノーの端に鍋を寄せて、浮いてくるアク油脂を取り除きながら澄ませていく。提供直前に、スープ入れに移し、バターで揚げた小さなクルトンを散らす(p.142)。デュボワ、ベルナール『古典料理』(1856年)にはピュレ・コンデもピュレ・コンティも掲載されていないが、グフェ『料理の本』(1867年)では「赤いんげん豆のポタージュ・ピュレ・コンデ」(p.369)と「レンズ豆のピュレ・コンティ」(p.371)がともに掲載されている。このように、ポタージュにおけるコンデとコンティはまったく別々に命名されたものと考えられるため、ブルボン王家の傍流とそのさらに傍流を揶揄したようなものではないと思われる。また、レンズ豆のピュレ自体の歴史は非常に古く、1660年ピエール・ド・リュヌ『新料理の本』においてPotage
  de nantilles
  としてレンズ豆を煮込んで潰したもののレシピが掲載されている(p.315)。
  nantilles
  という表現は誤植ではなく、17、18世紀の料理書においてしばしば見られる表現で、もちろんレンズ豆を意味する。裕福な、の意である形容詞nantiをレンズ豆lentillesをかけた造語であり、レンズ豆の形状が硬貨に似ているところから連想されたものと思われる。また、レンズ豆は地中海世界で農業が始まった頃からの古い作物であり、聖書にも出てくる。詳しくは\protect\hyperlink{garniture-conti}{ガルニチュール・コンティ}訳注参照。}}{ピュレ・コンティ}}\label{puree-conti}}

レンズ豆は欠けたものや割れたものを取り除いて大きさを揃え、
\(\frac{3}{4}\)
Lを軽い\protect\hyperlink{consomme-blanc-simple}{コンソメ}1
Lにさいの目に切って下茹でした塩漬け豚バラ肉を加えて煮込む。乾燥豆を煮る際の標準的な香味野菜を加える。レンズ豆を取り出して水気をきり、香味野菜は取り除く。レンズ豆をすり潰して、茹で汁でピュレをのばし、布で漉す。

\protect\hyperlink{consomme-ordinaire}{コンソメ}2 \(\frac{1}{2}\)
dLを加えて丁度いい濃度にし、提供直前にバターを加え、セルフイユ1つまみで仕上げる。
\end{recette}
\begin{center}\rule{0.5\linewidth}{\linethickness}\end{center}

\hypertarget{ux9b5aux6599ux7406}{%
\section{魚料理}\label{ux9b5aux6599ux7406}}

\hypertarget{ux820cux3073ux3089ux3081ux30c7ux30e5ux30b0ux30ecux30ec-sole-dugruxe9ruxe9-p.336}{%
\subsubsection{舌びらめ・デュグレレ Sole Dugréré
(p.~336)}\label{ux820cux3073ux3089ux3081ux30c7ux30e5ux30b0ux30ecux30ec-sole-dugruxe9ruxe9-p.336}}

\hypertarget{ux30b5ux30fcux30e2ux30f3ux306eux30afux30eaux30d3ux30e4ux30c3ux30af-coulibiac-de-saumon-pp.301-303}{%
\subsubsection{サーモンのクリビヤック Coulibiac de Saumon
(pp.301-303)}\label{ux30b5ux30fcux30e2ux30f3ux306eux30afux30eaux30d3ux30e4ux30c3ux30af-coulibiac-de-saumon-pp.301-303}}

\hypertarget{ux8089ux6599ux7406}{%
\section{肉料理}\label{ux8089ux6599ux7406}}

\hypertarget{ux30d6ux30d5ux30a2ux30e9ux30e2ux30fcux30c9}{%
\subsubsection{ブフアラモード}\label{ux30d6ux30d5ux30a2ux30e9ux30e2ux30fcux30c9}}

\hypertarget{ux7f8aux306eux30a2ux30eaux30b3-haricot-de-mouton-p.519}{%
\subsubsection{羊のアリコ Haricot de mouton
(p.519)}\label{ux7f8aux306eux30a2ux30eaux30b3-haricot-de-mouton-p.519}}

\hypertarget{ux9d8fux80f8ux8089ux30a2ux30ebux30d3ux30e5ux30d5ux30a7ux30e9-supruxeame-de-volaille-albufuxe9ra-p.576}{%
\subsubsection{鶏胸肉・アルビュフェラ Suprême de volaille Albuféra
(p.~576)}\label{ux9d8fux80f8ux8089ux30a2ux30ebux30d3ux30e5ux30d5ux30a7ux30e9-supruxeame-de-volaille-albufuxe9ra-p.576}}

\hypertarget{ux30c7ux30b6ux30fcux30c8}{%
\section{デザート}\label{ux30c7ux30b6ux30fcux30c8}}

\hypertarget{ux3055ux304fux3089ux3093ux307cux306eux30b8ux30e5ux30d3ux30ec-cerises-jubiluxe9e-p.821}{%
\subsubsection{さくらんぼのジュビレ Cerises Jubilée
(p.~821)}\label{ux3055ux304fux3089ux3093ux307cux306eux30b8ux30e5ux30d3ux30ec-cerises-jubiluxe9e-p.821}}

\hypertarget{ux30bfux30f3ux30d0ux30eb}{%
\subsubsection{タンバル}\label{ux30bfux30f3ux30d0ux30eb}}
