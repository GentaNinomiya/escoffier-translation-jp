\hypertarget{ux30a8ux30b9ux30b3ux30d5ux30a3ux30a8ux306eux65b0ux89e3ux91c8-ux53c2ux8003ux4f8b}{%
\chapter{エスコフィエの新解釈 --- 参考例
---}\label{ux30a8ux30b9ux30b3ux30d5ux30a3ux30a8ux306eux65b0ux89e3ux91c8-ux53c2ux8003ux4f8b}}

\hypertarget{les-hors-d-oeuvres}{%
\section{前菜}\label{les-hors-d-oeuvres}}
\begin{recette}
\hypertarget{bouchees}{%
\subsubsection{ブシェ}\label{bouchees}}

\frsub{Bouchées}

通常、ブシェをメニューの「温製オードブル」に位置付ける場合には、標準的なブシェよりも小さいサイズのものにしなくてはいけない。そのうえで、「かわいらしいブシェ」のように明記される。形状はどんな仕上りのものにするかでいろいろに変えてやり、大きなブシェを切った場合とは全然違うものであるとわかるようにすること。

場合によっては、ブシェの蓋の部分は残して蓋にするが、スライスしたまま、あるいは飾り切りをしたトリュフを蓋にすることもあるし、また別の場合には、詰めものの一部を蓋として利用することもある。

ブシェは必ずナフキンの上にのせて供すること。

\hypertarget{bouchee-a-la-reine}{%
\subsubsection{ブシェ・王妃風}\label{bouchee-a-la-reine}}

\frsub{Bouchée à la Reine}

この種のブシェの、クラシックな、本来の詰め物は生クリーム入りの鶏のピュレが用いられていた。だが、こんにちでは鶏胸肉とマッシュルーム、トリュフを1〜2
mm角の細かいみじん切りにして\protect\hyperlink{sauce-allemande}{ソース・アルマンド}であえたもので代用されている。ほとんど全ての調理現場では詰め物に後者を用いるようになってしまった。このブシェの形状は必ず円形で、縁に波形の模様が入ったものであること。

\begin{center}\rule{0.5\linewidth}{\linethickness}\end{center}

\hypertarget{epinards-a-la-viroflay}{%
\subsubsection[ほうれんそう・ヴィロフレー]{\texorpdfstring{ほうれんそう・ヴィロフレー\footnote{パリ郊外南西のヴェルサイユ近くの地名。ほうれんそうの栽培で有名で、ヴィロフレーという名称の伝統品種もある。}}{ほうれんそう・ヴィロフレー}}\label{epinards-a-la-viroflay}}

\frsub{Epinards à la Viroflay}

布の上に下茹でしたほうれんそうの葉(大)を広げる。それぞれの葉の中心に「ほうれんそうのシュブリック」を置く。このシュブリックにはパンの身をバターで揚げた小さなクルトンを混ぜ込んでおくこと。シュブリックをほうれんそうの葉で丸くなるように包む。これをバターを塗ったグラタン皿に並べ、\protect\hyperlink{sauce-mornay}{ソース・モルネー}を覆いかける。上からおろしたチーズを振りかけ、溶かしバターをかけてやり、高温のオーブンでこんがり焼く。

\hypertarget{subric-d-epinards}{%
\subsubsection{ほうれんそうのシュブリック}\label{subric-d-epinards}}

\frsub{Subric d'épinards}

ほうれんそうは上述のとおり\footnote{「ほうれんそうのクリームあえ」参照。ほうれんそうは下茹でして水にはなしてから、水気を絞り、みじん切りにするか裏漉ししてから、ほうれんそう500
  gあたりバター60
  gとともにソテー鍋に入れて強火にかけ、余計な水分をとばす。}にバターを加えて強火にかけて水気をとばす。鍋を火からはずし、ほうれんそう500
gあたり、濃い\protect\hyperlink{sauce-bechamel}{ベシャメルソース}1
dL、クレーム・エペス大さじ2杯、溶きほぐした全卵1
個と卵黄3個、塩、こしょう、ナツメグを加える。フライパンにたっぷりのバターを熱して充分な量の澄ましバターを用意する。

ほうれんそうでつくったアパレイユをスプーンで掬いとり、指で押し出すようにして澄ましバターの中に落としていく。シュブリックの成形をそのまま続けていくが、隣り同士で触れ合わないように注意すること。1分程焼いたら、パレットナイフかフォークで反対側の面にも焼き色を付けてやる。これをメインの料理の皿か野菜料理用の皿に盛り、ソース・クレームを別添で供する。

\hypertarget{nota-subric-d-epinards}{%
\subparagraph{【原注】}\label{nota-subric-d-epinards}}

シュブリックのアパレイユには別の作り方もある。バターを加えてほうれんそうの水気をとばしたら、ほうれんそうと同量の、やや濃い目に作ったクレープ生地を混ぜ込む。
\end{recette}
\begin{center}\rule{0.5\linewidth}{\linethickness}\end{center}

\newpage

\hypertarget{Potages}{%
\section{ポタージュ}\label{Potages}}
\begin{recette}
\hypertarget{consomme-rabelais}{%
\subsubsection[コンソメ・ラブレー]{\texorpdfstring{コンソメ・ラブレー\footnote{フランスのルネサンス期を代表する人文主義者、小説家であり医師でもあったフランソワ・ラブレー(?〜1553)のこと。なおこのレシピは第四版のみで、初版は「ジビエのコンソメにヴヴレ産白ワイン2
  dLを煮詰めて加える(コンソメ4
  Lあたり)。浮き実は小さな棒状にしたトリュフ風味のひばりの小さなクネルと、セロリの千切りをコンソメで軽く煮たもの
  (p.23)」。第二版では「ジビエのコンソメに、1
  Lあたりヴヴレ産白ワイン\(\frac{1}{2}\)
  dLを煮詰めて加える。浮き実\ldots{}\ldots{}トリュフを加えたひばりのファルスを刻み模様の付いた口金で絞り出したクネル。セロリの千切りをコンソメで軽く煮たもの(p.170)」となっているが、第三版にこの名称のレシピは掲載されていない。なお、ラブレーはシノン郊外の生まれであるため、トゥーレーヌ産のワイン(とりわけシノンの赤)が引き合いに出されることが多い。}}{コンソメ・ラブレー}}\label{consomme-rabelais}}

\begin{itemize}
\item
  鶏のコンソメにペルドローのフュメを加える。
\item
  浮き実\ldots{}\ldots{}\protect\hyperlink{farce-c}{生クリーム入りペルドローのファルス}をコーヒースプーンで成形し、提供直前に沸騰しない程度の温度で火を通した\footnote{pocher
    (ポシェ)。}クネル。マデイラ酒風味で火を通したトリュフの細い千切り\footnote{fine
    julienne (フィーヌジュリエンヌ)。}。
\item
  別添\ldots{}\ldots{}パルメザン風味の小さなプロフィットロール。
\end{itemize}

\hypertarget{puree-conde}{%
\subsubsection[ピュレ・コンデ]{\texorpdfstring{ピュレ・コンデ\footnote{ブルボン王家の支流にあたる
  Prince de Condée
  (プランスドコンデ)コンデ大公のこと。赤いんげん豆のポタージュにコンデの名称を冠したのは文献上はおそらくヴィアール『帝国料理の本』(1806年)が初出。
  (Potage) A la Condé
  となっている。また、18世紀以前の料理書において赤いんげん豆のポタージュはほとんど見つからない。よく知られているように、いんげん豆はアメリカ大陸原産で16世紀くらいにはフランスに伝えられていたはずだが、広まるのに時間がかかったようだ。さて、ヴィアールのレシピの概要は、1リトロン(≒0.8
  L)の赤いんげん豆をブイヨンで煮る。にんじん2本、玉ねぎ2個、ポタージュの浮き脂少々、クローブ2本を加える。豆が煮えたら裏漉しして滑らかなピュレにする。これをバターで揚げたパンのクルートの上に注いで供する(p.8)。この本にはレンズ豆のピュレのポタージュも続けて掲載されているが、そこにコンティの名はなく、たんに「レンズ豆のピュレのポタージュ」と称されているのみ。作り方上述のコンデとほぼ同じ。ヴィアールでは『料理の手引き』に近い非常にシンプルなレシピだが、カレーム『19世紀フランス料理』第1巻(1833年)の「赤いんげん豆のピュレのポタージュ・コンデ風」は、1
  \(\frac{1}{2}\)
  Lの赤いんげん豆の殻を剥いて洗う。これを大鍋に入れて、ペルドリ1羽、バイヨンヌの生ハム1切れ、にんじん2本、玉ねぎ2個、ブイヨン適量を加える。火にかけて煮ながらアクを取る。ペルドリに火が通ったらすぐに、ハムや他の根菜とともに取り出す。豆は煮汁ごと布で漉す。このピュレをごく標準的な鶏のコンソメに流し入れ、粗く砕いたこしょう
  1つまみ加えて弱火で煮込む。フルノーの端に鍋を置いて弱火で2時間程、アクを取りながら煮込む。その後スープ入れに移し、バターで揚げたクルトンを入れておいた各自のスープ皿に供する(p.144)。この本では赤いんげん豆のポタージュには「コンデ風」の名称が付けられているが、その次のレシピは「白いんげん豆のピュレのポタージュ」というだけの単純な名称になっている。ブルジョワ料理の本として19世紀から20世紀初頭まで版を重ねたオド『女性料理人のための本』第15版(1834年)では早くも「ポタージュ・コンデ風」として簡単にだが赤いんげん豆のピュレのポタージュのレシピが掲載されている。その一方で、レンズ豆を用いたポタージュについては1909年の第97版に至るまでレンズ豆のピュレのポタージュは掲載されているが「コンティ」の名は冠されていない。}}{ピュレ・コンデ}}\label{puree-conde}}

\frsub{Purée Condé}

赤いんげん豆は塩18 gを加えた冷水1 \(\frac{1}{2}\)
Lに入れて火にかける。沸騰したら、しっかりアクを取り\footnote{écumer
  (エキュメ)浮いてくる泡を取り除く、が原義。}、赤ワイン2
\(\frac{1}{2}\)
dLを沸かしてから加える。ブーケガルニ、クローブを刺した玉ねぎ1個、四つ割りに切ったにんじん1本を加えて弱火にして煮込む。いんげん豆がよく煮えたら、煮汁から出して、ブーケガルニと玉ねぎ、にんじんは取り除く。いんげん豆を丁寧にすり潰す。煮汁でのばしてから布で漉し、提供直前にバターを加える。

\hypertarget{puree-conti}{%
\subsubsection[ピュレ・コンティ]{\texorpdfstring{ピュレ・コンティ\footnote{上記コンデ大公家のさらに傍流。王家の分家の分家という扱いになるが、
  Prince de Conti
  (プランスドコンティ)の称号を持つ。ポタージュにコンティの名が冠されたのは、上記コンデの名よりずっと早く、ムノン『宮廷の晩餐』(1755年)第1巻に「ポタージュ・コンティ風」とある。ただしこれはレンズ豆を材料にしたポタージュではなく、スライスした玉ねぎを炒めて煮込み、スープ入れの底にバターで揚げたパン(クルート)を敷いてその上に盛り、刻んだアンチョビを玉ねぎに散らすというもの
  (pp.91-92)。ボヴィリエの『調理技術』(1814年)第1巻では「レンズ豆のピュレのポタージュ・王妃風」と出ている。作り方はえんどう豆のピュレのポタージュと同様にするが、赤レンズ豆を用いて「王妃風」を謳う場合は、上手に煮込んできれいな赤色に仕上げるべし、とある(p.22)。「レンズ豆のポタージュ・コンティ風」の名称が出てくるのはカレーム『19世紀フランス料理』第1巻。1
  \(\frac{1}{2}\) Lの赤レンズ豆 (lentilles à la
  reine)の殻を剥いて洗う。下茹でしたハム、ペルドリ1羽、にんじん2
  本、蕪1個、玉ねぎ2個、ポワロー2本を束ねたものとセロリの根元1株を加え、適量のブイヨンを注いで煮る。アクを取り、3時間弱火で煮込む。根菜、ペルドリ、ハムを取り出してから、レンズ豆を布で漉す。このピュレを普段のとおり作ったコンソメに加える。沸騰したらフルノーの端に鍋を寄せて、浮いてくるアク油脂を取り除きながら澄ませていく。提供直前に、スープ入れに移し、バターで揚げた小さなクルトンを散らす(p.142)。デュボワ、ベルナール『古典料理』(1856年)にはピュレ・コンデもピュレ・コンティも掲載されていないが、グフェ『料理の本』(1867年)では「赤いんげん豆のポタージュ・ピュレ・コンデ」(p.369)と「レンズ豆のピュレ・コンティ」(p.371)がともに掲載されている。このように、ポタージュにおけるコンデとコンティはまったく別々に命名されたものと考えられるため、ブルボン王家の傍流とそのさらに傍流を揶揄したようなものではないと思われる。また、レンズ豆のピュレ自体の歴史は非常に古く、1660年ピエール・ド・リュヌ『新料理の本』においてPotage
  de nantilles
  としてレンズ豆を煮込んで潰したもののレシピが掲載されている(p.315)。
  nantilles
  という表現は誤植ではなく、17、18世紀の料理書においてしばしば見られる表現で、もちろんレンズ豆を意味する。裕福な、の意である形容詞nantiをレンズ豆lentillesをかけた造語であり、レンズ豆の形状が硬貨に似ているところから連想されたものと思われる。また、レンズ豆は地中海世界で農業が始まった頃からの古い作物であり、聖書にも出てくる。詳しくは\protect\hyperlink{garniture-conti}{ガルニチュール・コンティ}訳注参照。}}{ピュレ・コンティ}}\label{puree-conti}}

レンズ豆は欠けたものや割れたものを取り除いて大きさを揃え、
\(\frac{3}{4}\)
Lを軽い\protect\hyperlink{consomme-blanc-simple}{コンソメ}1
Lにさいの目に切って下茹でした塩漬け豚バラ肉を加えて煮込む。乾燥豆を煮る際の標準的な香味野菜を加える。レンズ豆を取り出して水気をきり、香味野菜は取り除く。レンズ豆をすり潰して、茹で汁でピュレをのばし、布で漉す。

\protect\hyperlink{consomme-ordinaire}{コンソメ}2 \(\frac{1}{2}\)
dLを加えて丁度いい濃度にし、提供直前にバターを加え、セルフイユ1つまみで仕上げる。
\end{recette}
\begin{center}\rule{0.5\linewidth}{\linethickness}\end{center}

\hypertarget{les-poissons}{%
\section{魚料理}\label{les-poissons}}
\begin{recette}
\hypertarget{sole-duglere}{%
\subsubsection[舌びらめ・デュグレレ]{\texorpdfstring{舌びらめ・デュグレレ\footnote{アドルフ・デュグレレ
  Adolphe Dugléré
  (1805〜1884)。カレームのもとで学び、カフェ・アングレやトロワ・フレール・プロヴオンソーで料理長を務めた。\protect\hyperlink{pommes-de-terre-anna}{ポム・アンナ}、\protect\hyperlink{potage-germiny}{ポタージュ・ジェルミニ}、この舌びらめ・デュグレレなどの料理を考案したことで知られる。とりわけこの料理は19世紀中葉に食材として大流行していたトマトを用いている点で、時代性をよく表わしている。また、小説家アレクサンドル・デュマ(1802〜1870)の『料理事典』(1882年版と1883年版があるが、いずれも死後出版。前者は「選集」。他の著作からの無断引用が多く、食文化史の史料としてはあまり重要視されていない)の編纂に助力したとも言われている。}}{舌びらめ・デュグレレ}}\label{sole-duglere}}

\frsub{Sole Dugléré}

\index{sole@sole!duglere@--- Dugléré}
\index{duglere@Dugléré!sole@sole ---}
\index{したひらめ@舌びらめ!てゆくれれ@--- ・デュグレレ}
\index{てゆくれれ@デュグレレ!したひらめ@舌びらめ・---}
\index{そーる@ソール ⇒ 舌びらめ!てゆくれれ@---・デュグレレ}

基本的に、この調理をする魚はトロンソン\footnote{tronçon
  筒切り、の意で、うなぎなどは文字通りにやや長めのぶつ切りにすることを言うが、チュルボなどのような平らな魚の場合には、まず縦2つに切ってから、骨の方向に添うようにいくつかに切り分ける。}に切っておくべきなのだが、舌びらめの場合は例外的に丸ごと1尾で調理してかまわない\footnote{このレシピは舌びらめをフィレではなく丸ごと1尾で調理する節に含まれていることに注意。}。

舌びらめはバターを塗った平鍋に入れる。玉ねぎ \(\frac{1}{2}\)
個とエシャロット2個はみじん切りにし、トマト2個は皮を剥いて潰してからざく切りにして加える。パセリのみじん切り少々と塩、こしょう、白ワイン大さじ数杯を加える。弱火で沸騰させないよう火を通し\footnote{pocher
  (ポシェ)。【参考】\textbf{ごく少量のクールブイヨンを用いたポシェ}\ldots{}\ldots{}この火入れの方法は主としてチュルボタン、バルビュ、舌びらめ丸ごとでもフィレでも用いらる。バターを塗った天板あるいはソテー鍋に魚丸ごとあるいはそのフィレを置き、軽く塩をして、所要量の魚のフュメかマッシュルームの煮汁を注ぐ。フュメとマッシュルームの煮汁を合わせたものを用いる場合もある。蓋をして、中温のオーヴンに入れる。魚丸ごとの場合は時折煮汁をかけてやる(原書
  pp.279-280)。}、皿に盛り付ける。

舌びらめの煮汁を煮詰める。これに\protect\hyperlink{veloute-de-poisson}{魚料理用ヴルテ}大さじ2〜3杯を加えてとろみを付ける。仕上げにバター30
gとレモン果汁少々を加え、舌びらめに覆いかける。

\hypertarget{coulibiac-de-saumon-a}{%
\subsubsection{サーモンのクリビヤック A}\label{coulibiac-de-saumon-a}}

\frsub{Coulibiac de Saumon A}

\index{saumon@saumon!coulibiac a@Coulibiac de --- A}
\index{coulibiac@coulibiac!saumon a@--- de Saumon A}
\index{さけ@鮭 ⇒ サーモン}
\index{さーもん@サーモン!くりひやつくa!サーモンのクリビヤックA}
\index{くりひやつく@クリビヤック!さーもんa!サーモンの--- A}

(材料)

\begin{itemize}
\item
  砂糖を加えずにやや固めに作った標準的なブリオシュ生地約1
  kg(\protect\hyperlink{pate-a-brioche}{ブリオシュ生地}参照)。
\item
  サーモン650 gは線維と垂直に1
  cm程度の厚さにスライスし、バターで色付かないよう表面を焼き固め\footnote{raidir
    (レディール)素材の表面を色付けないように強火でさっと焼いて表面を焼き固めること。語義としては「焼く」限定されるものではなく、熱湯などの液体を用いる場合もある。}て冷ましておく。
\item
  マッシュルーム 75
  gと玉ねぎ(中)はみじん切りにし、バターで炒めて冷ましておく。パセリのみじん切り大さじ1杯強を加えておく。
\item
  \protect\hyperlink{kache-de-semoule-pour-coulibiac}{セモリナ粉のカーシャ}200
  gまたはコンソメで茹でた米200
  g(「\protect\hyperlink{garnitures}{ガルニチュール}」\protect\hyperlink{kache-de-semoule-pour-coulibiac}{カシャ}参照)。
\item
  固茹で卵2個のみじん切り。卵白、卵黄は分けなくていい。
\item
  茹でたヴェジガ(後述参照)500 g(乾燥状態のヴェジガ90
  gが必要)。乾燥ヴェジガは最低5時間冷水に漬けてもどし、\protect\hyperlink{consomme-blanc-simple}{白いコンソメ}か湯で3時間半茹でてから、粗くみじん切りにしておく。
\end{itemize}

(作業手順)

ブリオシュ生地を長さ32〜35
cm、幅18〜20cmの長方形に\ruby{伸}{の}す。中央に「パンタン\footnote{一般的には板などで出来た色とりどりの操り人形のことだが、料理においては、豚肉のファルスを詰めた正方形または楕円型の小さなパイ包み焼きのこと。ファルスにはトリュフを混ぜ込むこともある。ただし、ここではサーモンを1cm厚程度の薄切りにしているため、前者のイメージのほうが正しく伝わると思われる。}」のように具を詰めていく。カーシャまたは米とサーモン、みじん切りにしたヴェジガ、卵、マッシュルームと玉ねぎの層を順に重ねていくわけだ。最後はカーシャの層になるようにする。

生地の端を軽く濡らして、生地の両端が詰め物の中心に来るようにしてつなぎ合わせる。こうして成形したクリビヤックを裏返して、継ぎ目が下になるように天板にのせる。

暖い場所に置いて、25分間生地を醗酵させる。

最後に、溶かしバターを刷毛でクリビヤックに塗り、細かいパン粉を上から振りかける。加熱中に蒸気が抜けるように上面に切れ目を入れて穴を空けてやる。中温のオーブンでとりわけ炉床の温度の強い状態で焼く。

焼成時間\ldots{}\ldots{}45分間。

クーリビヤックをオーブンから出したら、溶かしバターをスプーン数杯、中に流し込んでやること。

\hypertarget{note-sur-vesiga}{%
\subparagraph{【ヴェジガについて】}\label{note-sur-vesiga}}

ヴェジガとはすなわちチョウザメの脊髄のことで、ロシア料理のいくつかの品でしか用いられないものだ。これは市場で入手可能で、リボン状のゼラチンのような見た目で、質感は魚膠のような感じだ。いろいろな方法で水で戻して火を通して試した結果、

\begin{enumerate}
\def\labelenumi{\arabic{enumi}.}
\item
  ヴェジガを冷水に漬けて普通にもどすのにかかる時間は5時間程度。
\item
  その程度の時間水でもどすと、だいたい5倍の量になる。さらに長い時間漬けておけばもっと嵩も重さも増すが、実際のところ5時間で充分。
\item
  乾燥ヴェジガ10 gは水で戻すと52〜55 gということになる。
\item
  乾燥ヴェジガを水でもどしてから茹でるのに必要な液体の量は、ヴェジガ
  260〜270gにつき 3 L必要。加熱はごく弱火で、蓋をしてすること。
\item
  ヴェジガの小さな切れ端を茹でる場合はせいぜい3時間半〜4時間半でいい。
\end{enumerate}
\end{recette}
\begin{center}\rule{0.5\linewidth}{\linethickness}\end{center}

\hypertarget{ux8089ux6599ux7406}{%
\section{肉料理}\label{ux8089ux6599ux7406}}
\begin{recette}
\hypertarget{boeuf-a-la-mode}{%
\subsubsection[ブフアラモード]{\texorpdfstring{ブフアラモード\footnote{à
  la mode
  (アラモード)元来は「流行の、おしゃれな」の意だが、この料理名については日本語の「プリンアラモード」と同様に、本来の意味が失なわれて、料理名として定着していると考えるのがいいだろう。
  Boeuf à la bourgeoise
  (ブフアラブルジョワーズ)ブルジョワ風とも呼ばれる。後者の料理名から考えると、産業革命の進展につれてブルジョワ階級が台頭してきた時代、すなわち18世紀末〜19世紀初頭の「流行」と見ることも出来なくはないが、その後も料理内容にほぼ変化がないままこの名称で作られ続けているので、上述のように料理名本来の意味は失なわれていると見るべき。牛イチボ肉の塊と小玉ねぎを用いるこの料理の原型とも言えるべきものは18世紀ムノン『ブルジョワ料理』に見出せるが,料理名に「ア・ラ・モード」の表現はない。一方、同じく18世紀マラン『食の贈り物』には「ブフ・ア・ラ・モード」の料理名が見られる。カレーム『19世紀フランス料理』の「
  ブフ・ア・ラ・モード ブルジョワ風」は『ル・ギード・キュリネール』のものと非常に近い内容であり、遅くともカレームの時点で料理としてほぼ完成していると考えられる。}}{ブフアラモード}}\label{boeuf-a-la-mode}}

\frsub{Boeuf à la mode}

\index{boeuf@boeuf!mode@--- à la mode}
\index{piece de boeuf@pièce de boeuf!mode@Boeuf à la mode}
\index{mode@mode (à la)!boeuf@Boeuf à la mode}
\index{うしかたまりにく@牛塊肉!もーと@ブフアラモード}
\index{ふふ@ブフ!もーと@ブルアラモード}
\index{もーと@モード!ふふあらもーと@ブフアラモード}
\index{あらもーと@アラモード!ふふ@ブフ---}

作業しやすいよう、2.5〜3kgを越えない程度のイチボ肉を用いる。この重量で約20人分となる。

豚背脂350
gをコニャックで20分間マリネし、こしょう、香辛料で味つけし、直前に刻んだパセリをまぶす。これをラルデ針でイチボ肉に刺し込む。

塩、こしょう、ナツメグ少量を肉にすり込む。これを赤ワイン \(\frac{1}{2}\)
本とコニャック1 dLで5〜6時間マリネする。

通常の方法で\protect\hyperlink{les-braises}{ブレゼ}するが、煮汁にマリネ液を加える。さらに仔牛の足を小さいものなら3本、中位のものなら2本、骨を外して下茹でし、紐で縛って、鍋に入れる。

\(\frac{3}{4}\)
程度火が通ったら、ひとまわり小さな鍋に肉を移す。小さなさいの目か長方形に切った仔牛の足と、バターで色よく炒めた小玉ねぎ400
g、オリーヴ形に整形し固めに茹でたにんじん600
gを肉の周囲に入れる。煮汁をシノワで漉してから浮き脂を取り除く。これを肉の入った鍋に注ぎ、弱火で火入れを仕上げる。

塊肉を皿に盛り、周囲につけあわせの野菜と仔牛の足を種類ごとにまとめてブーケのように飾る。ほどよく煮詰めた煮汁をかける。

\hypertarget{haricot-de-mouton}{%
\subsubsection[羊のアリコ]{\texorpdfstring{羊のアリコ\footnote{haricot
  (アリコ)は現代フランス語ではもっぱら、いんげん豆、さやいんげんを意味するが、中世フランス語においてはある種の「煮込み料理」を意味した。14世紀末に成立されたとされる手稿本『ル・メナジエ・ド・パリ』における「羊のアリコ」のレシピには当然ながらいんげん豆は使われていない。そもそもいんげん豆はアメリカ大陸原産なので、フランスに伝播して広まるのは16世紀以降のこと。にもかかわらず、かつて豆の代表であったえんどう豆が現代ではもっぱら若どりのプチポワでの利用が中心となった一方で、いんげん豆は若どりのさやいんげんも乾燥豆、さらに半乾燥のものも非常に好まれる食材となっている。}}{羊のアリコ}}\label{haricot-de-mouton}}

\frsub{Haricot de Mouton}

\index{haricot@haricot!mouton@--- de mouton}
\index{mouton@mouton!haricot@Haricot de ---}
\index{ありこ@アリコ!ひつし@羊の---}
\index{ひつし@羊!ありこ@---のアリコ}

豚ばら肉の塩漬け250
gは大きめのさいの目に切って下茹でし、小玉ねぎ20個とともにラードでこんがり炒める\footnote{faire
  revenir (フェールルヴニール)。}。これらを取り出して、同じ鍋で羊の胸肉、首肉、肩肉をラグー用に切ったもの2
kgを色よく焼く\footnote{risoller
  (リソレ)油脂を熱した鍋などで肉の表面にこんがり焼き色を付けること。≒
  faire revenir}。

肉の表面ががこんがり焼けたら、鍋の脂の半分は取り出す。潰したにんにく3
片と小麦粉40 gを加えてさらに加熱する。

水1
Lを注ぎ入れ、塩こしょうで調味し、ブーケガルニを加える。混ぜながら沸騰させた後、弱火で30分程煮込む。

肉を別の鍋に移して、先に炒めた塩漬け豚ばら肉と小玉ねぎを加える。半ば火を通した状態の白いんげん豆1
Lを加える。先の煮汁を全体にかけてやり、弱火のオーブンに入れて火入れを仕上げる。

小さな陶製の器に入れて供する。

\hypertarget{poularde-albufera}{%
\subsubsection{肥鶏 アルビュフェラ}\label{poularde-albufera}}

\frsub{Poularde Albuféra}

\index{poularde@poularde!albufera@Albuféra}
\index{albufera@Albuféra!poularde@Poularde ---}
\index{ひとり@肥鶏!あるひゆふえら@---・アルビュフェラ}
\index{ふーらると@プーラルド ⇒ 肥鶏!あるひゆふえら@---・アルビュフェラ}
\index{あるひゆふえら@アルビュフェラ@ひとり!肥鶏・---}

フォワグラと大きめのさいの目に切ったトリュフを米と合わせ、肥鶏に詰め物する。肥鶏を\protect\hyperlink{les-poches}{ポシェ}する。

皿に盛り、ソース・アルビュフェラを塗る。

周囲に次のものを盛り込む。くり抜きスプーンで丸く抜いたトリュフ、同様に丸く整形した鶏のクネル、小さめのマッシュルーム、雄鶏のロニョン。これらの\protect\hyperlink{garniture-albufera}{ガルニチュール}は\protect\hyperlink{sauce-albufera}{ソース・アルビュフェラ}であえておく。

\protect\hyperlink{saumure-liquide-pour-langues}{赤く漬けた舌肉}を鶏のとさか形に切って皿の縁を飾る。
\end{recette}
\begin{center}\rule{0.5\linewidth}{\linethickness}\end{center}

\hypertarget{ux30c7ux30b6ux30fcux30c8}{%
\section{デザート}\label{ux30c7ux30b6ux30fcux30c8}}
\begin{recette}
\hypertarget{cerises-jubilee}{%
\subsubsection[さくらんぼのジュビレ]{\texorpdfstring{さくらんぼのジュビレ\footnote{戴冠式、の意。さくらんぼに
  Napoléon という品種があるので、それを使って Jubilée de Napoléon
  と洒落た名称にすることも可能だろう。}}{さくらんぼのジュビレ}}\label{cerises-jubilee}}

\frsub{Cerises Jubilée}

大きさの揃った立派ななさくらんぼの種を抜く。シロップでやや低めの温度で火を通し\footnote{pocher
  (ポシェ)}、銀製の深皿に盛る。シロップを煮詰め、少量の冷水で溶いたアロールート\footnote{南米産のクズウコンから採れる良質のでんぷん。一般的にはコーンスターチで代用する。}を加えてとろみを付ける。比率はシロップ3
dLに対してアロールートがスプーン\(\frac{1}{2}\)杯。もしくは\protect\hyperlink{gelee-de-groseilles-a}{グロゼイユのジュレ}を用いる。

さくらんぼにとろみを付けたシロップをかけ、デザートスプーン1杯の温めたキルシュ酒を注ぎ、提供直前に火を点ける。

\hypertarget{timbale-d-arenberg}{%
\subsubsection[タンバル・アーレンベルク]{\texorpdfstring{タンバル\footnote{もとは「小太鼓」を意味する語で、円筒形の仕立てによく命名される。あくまでも形状を指す言葉であって、料理やパティスリの種類を意味しているわけではないことに注意。また、野菜料理を盛る深皿もタンバルと呼ばれ、混同しやすいので注意。}・アーレンベルク\footnote{Arenberg
  とも綴り、現在のドイツ東部の地名。または18世紀末までアーレンベルク公国を治めていたアーレンベルク家のこと。}}{タンバル・アーレンベルク}}\label{timbale-d-arenberg}}

\frsub{Timbale d'Aremberg}

バターを塗ったシャルロット型\footnote{型の口(上部)がやや広くなった円筒形の型。側面に刻み模様や波形模様の付いたものもある。}に、やや固めに作ったブリオシュ生地を敷き詰める。

四つ割りにしてバニラ風味のシロップで少し固めに煮た洋梨とアプリコットのマーマレードの層を交互に敷き詰めていく。

同じブリオシュ生地でタンバルに蓋をする。周囲を軽く湿らせてからしっかり生地を貼り付かせる。中央に、蒸気抜きの小さな穴を空けておく。中温のオーブンで約40分間焼く。

オーブンから出したら、型から外して皿に盛り、マラスキーノ酒\footnote{marasquin
  (マラスカン)。マラスカという品種のさくらんぼで作ったリキュール。}風味のアプリコットソースをかけて供する。

\hypertarget{sauce-a-l-abricot}{%
\subsubsection{アプリコットソース}\label{sauce-a-l-abricot}}

\frsub{Sauce à l'Abricot}

よく熟したアプリコットを目の細かい網で裏漉しする。またはアプリコットのマーマレードを使う。28°Béのシロップでアプリコットのピュレをのばす。沸騰させて浮いてくる泡を丁寧に取り除く。スプーンをコーティングする程度の漉さになったら火から外し、アーモンドミルクかマデイラ酒、マラスキーノ酒で香り付けする。(pp.793-794)
\end{recette}