\documentclass[twoside,12Q,b5j]{escoffierltjsbook}
%\documentclass[twoside,8pt,a5j]{escoffierltjsbook}
\usepackage{amsmath}%数式
% \let\equation\gather
%  \let\endequation\endgather
\usepackage{amssymb}
\usepackage[no-math]{fontspec}
%\usepackage{xunicode}
\usepackage{geometry}
\usepackage{unicode-math}
\usepackage{xfrac}
\usepackage{luaotfload}
\usepackage{graphicx}






%%欧文フォント設定
\setmainfont[Ligatures=TeX,Scale=1.0]{Linux Libertine O}

%%Garamond
%\usepackage{ebgaramond-maths}
%\setmainfont[Ligatures=TeX,Scale=1.0]{EB Garamond}%fontspecによるフォント設定

%\usepackage{qpalatin}%palatino

%\setmainfont[Ligatures=TeX]{TeX Gyre Pagella}%ギリシャ語を用いる場合はこちら
%\setsansfont[Scale=MatchLowercase]{TeX Gyre Heros}  % \sffamily のフォント
\setsansfont[Ligatures=TeX, Scale=1]{Linux Biolinum O}     % Libertine/Biolinum
%\setmonofont[Scale=MatchLowercase]{Inconsolata}       % \ttfamily のフォント
%\unimathsetup{math-style=ISO,bold-style=ISO}
%\setmathfont{xits-math.otf}
%\setmathfont{xits-math.otf}[range={cal,bfcal},StylisticSet=1]

%\index{\usepackage}\usepackage[cmintegrals,cmbraces]{newtxmath}%数式フォント

\usepackage{luatexja}
\usepackage{luatexja-fontspec}
%\ltjdefcharrange{8}{"2000-"2013, "2015-"2025, "2027-"203A, "203C-"206F}
%\ltjsetparameter{jacharrange={-2, +8}}
\usepackage{luatexja-ruby}

%%%%和文仮名プロポーショナル
%\usepackage[sourcehan,expert]{luatexja-preset}%Adobe源ノ明朝、ゴチ
\usepackage[hiragino-pron,jis2004,expert,deluxe]{luatexja-preset}
%\usepackage[ipa]{luatexja-preset}
%\newopentypefeature{PKana}{On}{pkna} % "PKana" and "On" can be arbitrary string
%\setmainjfont[
%    JFM=prop,PKana=On,Kerning=On,
%    BoldFont={YuMincho-DemiBold},
%    ItalicFont={YuMincho-Medium},
%    BoldItalicFont={YuMincho-DemiBold}
%]{YuMincho-Medium}
%\setsansjfont[
%    JFM=prop,PKana=On,Kerning=On,
%    BoldFont={YuGothic-Bold},
%    ItalicFont={YuGothic-Medium},
%    BoldItalicFont={YuGothic-Bold}
%]{YuGothic-Medium}
%%%%和文仮名プロプーショナルここまで

\renewcommand{\bfdefault}{bx}%和文ボールドを有効にする
\renewcommand{\headfont}{\gtfamily\sffamily\bfseries}%和文ボールドを有効にする

\defaultfontfeatures[\rmfamily]{Scale=1.2}%効いていない様子
\defaultjfontfeatures{Scale=0.92487}%和文フォントのサイズ調整。デフォルトは 0.962212 倍%ltjsclassesでは不要?
%\defaultjfontfeatures{Scale=0.962212}
%\usepackage{libertineotf}%linux libertine font %ギリシア語含む
%\usepackage[T1]{fontenc}
%\usepackage[full]{textcomp}
%\usepackage[osfI,scaled=1.0]{garamondx}
%\usepackage{tgheros,tgcursor}
%\usepackage[garamondx]{newtxmath}
\usepackage{xfrac}

\usepackage{layout}

%レイアウト調整(B5,12Q,escoffierltjsbook.cls)
%
\setlength{\hoffset}{-1truein}
\setlength{\hoffset}{-0.5mm}
\setlength{\oddsidemargin}{0pt}
\setlength{\evensidemargin}{-1cm}
%\setlength{\textwidth}{\fullwidth}%%ltjsclassesのみ有効
\setlength{\fullwidth}{14cm}
\setlength{\textwidth}{14cm}
\setlength{\marginparsep}{0pt}
\setlength{\marginparwidth}{0pt}
\setlength{\footskip}{0pt}
\setlength{\textheight}{20.5cm}
%%%ベースライン調整
%\ltjsetparameter{yjabaselineshift=0pt,yalbaselineshift=-.75pt}

%レイアウト調整(8pt,a5j,escoffierltjsbook)
%\setlength{\voffset}{-.5cm}
%\setlength{\hoffset}{-.6cm}
%\setlength{\oddsidemargin}{0pt}
%\setlength{\evensidemargin}{\oddsidemargin}
%\setlength{\textwidth}{\fullwidth}%%ltjsclassesのみ有効
%\setlength{\fullwidth}{40\zw}
%\setlength{\textwidth}{40\zw}
%\setlength{\marginparsep}{0pt}
%\setlength{\marginparwidth}{0pt}
%\setlength{\footskip}{0pt}
%\setlength{\textheight}{17.5cm}
%%%ベースライン調整
%\ltjsetparameter{yjabaselineshift=0pt,yalbaselineshift=-.75pt}
%\setlength{\baselineskip}{15pt}


\def\tightlist{\itemsep1pt\parskip0pt\parsep0pt}

%リスト環境
\makeatletter
  \parsep   = 0pt
  \labelsep = 1\zw
  \def\@listi{%
     \leftmargin = 0pt \rightmargin = 0pt
     \labelwidth\leftmargin \advance\labelwidth-\labelsep
     \topsep     = 0pt%\baselineskip
     \topsep -0.1\baselineskip \@plus 0\baselineskip \@minus 0.1 \baselineskip
     \partopsep  = 0pt \itemsep       = 0pt
     \itemindent = 0pt \listparindent = 0\zw}
  \let\@listI\@listi
  \@listi
  \def\@listii{%
     \leftmargin = 1\zw \rightmargin = 0pt
     \labelwidth\leftmargin \advance\labelwidth-\labelsep
     \topsep     = 0pt \partopsep     = 0pt \itemsep   = 0pt
     \itemindent = 0pt \listparindent = 1\zw}
  \let\@listiii\@listii
  \let\@listiv\@listii
  \let\@listv\@listii
  \let\@listvi\@listii
\makeatother


  
%\usepackage{fancyhdr}

\usepackage{setspace}
\setstretch{1.1}


%レシピ本文
\usepackage{multicol}

\newenvironment{recette}{\begin{small}\begin{spacing}{1}\begin{multicols}{2}}{\end{multicols}\end{spacing}\end{small}}

%\newenvironment{recette}{\begin{spacing}{1}\begin{multicols}{2}}{\end{multicols}\end{spacing}}


%subsubsectionに連番をつける
%\usepackage{remreset}

\renewcommand{\thechapter}{}
\renewcommand{\thesection}{}
\renewcommand{\thesubsection}{}
\renewcommand{\thesubsubsection}{}
\renewcommand{\theparagraph}{}

%\makeatletter
%\@removefromreset{subsubsection}{subsection}
%\def\thesubsubsection{\arabic{subsubsection}.}
%\newcounter{rnumber}
%\renewcommand{\thernumber}{\refstepcounter{rnumber} }

\renewcommand{\prepartname}{\if@english Part~\else {}\fi}
\renewcommand{\postpartname}{\if@english\else {}\fi}
\renewcommand{\prechaptername}{\if@english Chapter~\else {}\fi}
\renewcommand{\postchaptername}{\if@english\else {}\fi}
\renewcommand{\presectionname}{}%  第
\renewcommand{\postsectionname}{}% 節

\makeatother



% PDF/X-1a
% \usepackage[x-1a]{pdfx}
% \Keywords{pdfTeX\sep PDF/X-1a\sep PDF/A-b}
% \Title{Sample LaTeX input file}
% \Author{LaTeX project team}
% \Org{TeX Users Group}
% \pdfcompresslevel=0
%\usepackage[cmyk]{xcolor}

%biblatex
%\usepackage[notes,strict,backend=biber,autolang=other,%
%                   bibencoding=inputenc,autocite=footnote]{biblatex-chicago}
%\addbibresource{hist-agri.bib}
\let\cite=\autocite

% % % % 
\date{}

%%%脚注番号のページ毎のリセット
%\makeatletter
%  \@addtoreset{footnote}{page}
%\makeatother
\usepackage[perpage,marginal,stable]{footmisc}
\makeatletter
\renewcommand\@makefntext[1]{%
  \advance\leftskip 1.5\zw
  \parindent 1\zw
  \noindent
  \llap{\@thefnmark\hskip0.5\zw}#1}


\renewenvironment{theindex}{% 索引を3段組で出力する環境
    \if@twocolumn
      \onecolumn\@restonecolfalse
    \else
      \clearpage\@restonecoltrue
    \fi
    \columnseprule.4pt \columnsep 2\zw
    \ifx\multicols\@undefined
      \twocolumn[\@makeschapterhead{\indexname}%
      \addcontentsline{toc}{chapter}{\indexname}]%変更点
    \else
      \ifdim\textwidth<\fullwidth
        \setlength{\evensidemargin}{\oddsidemargin}
        \setlength{\textwidth}{\fullwidth}
        \setlength{\linewidth}{\fullwidth}
        \begin{multicols}{3}[\chapter*{\indexname}
	\addcontentsline{toc}{chapter}{\indexname}]%変更点%
      \else
        \begin{multicols}{3}[\chapter*{\indexname}
	\addcontentsline{toc}{chapter}{\indexname}]%変更点%
      \fi
    \fi
    \@mkboth{\indexname}{\indexname}%
    \plainifnotempty % \thispagestyle{plain}
    \parindent\z@
    \parskip\z@ \@plus .3\p@\relax
    \let\item\@idxitem
    \raggedright
    \footnotesize\narrowbaselines
  }{
    \ifx\multicols\@undefined
      \if@restonecol\onecolumn\fi
    \else
      \end{multicols}
    \fi
    \clearpage
  }
  \makeatother



  
  \renewcommand{\ldots}{…}
\usepackage{makeidx}
\makeindex


\usepackage[unicode=true]{hyperref}
%\usepackage{pxjahyper}
\hypersetup{breaklinks=true,%
             bookmarks=true,%
             pdfauthor={},%
             pdftitle={},%
             colorlinks=true,%
             citecolor=blue,%
             urlcolor=cyan,%
             linkcolor=magenta,%
             pdfborder={0 0 0}}


% \hypersetup{
%     pdfborderstyle={/S/U/W 1}, % underline links instead of boxes
%     linkbordercolor=red,       % color of internal links
%     citebordercolor=green,     % color of links to bibliography
%     filebordercolor=magenta,   % color of file links
%     urlbordercolor=cyan        % color of external links
% }

           \urlstyle{same}
%\renewcommand*{\label}[1]{\hypertarget{#1}{}}
%\renewcommand{\hyperlink}[2]{\hyperref[#1]{#2}}

\renewcommand{\ldots}{…}

%\newcommand{\maeaki}{\vspace*{0.5\zw}}
%\newcommand{\maeaki}{\vspace*{1.0\zw}}}
%\newcommand{\maeaki}{\vspace*{1.1\zw}}
%\newcommand{\maeaki}{\vspace*{1.5\zw}}
\newcommand{\maeaki}{\vspace*{1.7\zw}}
                     
           
\begin{document}

\frontmatter
\hypertarget{avant-propos}{%
\chapter{序}\label{avant-propos}}

\fifteenq
\setstretch{1.3}

もう20年も前のことだ。本書の着想を我が尊敬する師、今は亡きユルバン・デュボワ\footnote{Urbain
  Dubois (1818〜1901)。19世紀後半を代表する料理人。}先生に話したのは。先生は\ruby{是非}{ぜひ}とも実現させなさいと強く勧めてくださった。けれども忙しさにかまけてしまい、\ruby{漸}{ようや}く
1898年になって、フィレアス・ジルベール\footnote{Philéas Gilbert
  (1857〜1942)。19世紀末から20世紀初頭に活躍した料理人。料理雑誌「ポトフ」を主宰した。}君と話し合い協力をとりつけることが出来た。ところがまもなく、カールトンホテル開業のために私はロンドンに呼び戻され、その厨房の準備や運営に忙殺されることとなった\footnote{エスコフィエはセザール・リッツの経営するホテルグループにおいて料理に関わる重要な役割を一手に担っていた。1890年〜1897年にかけてロンドンのサヴォイホテルの総料理長を勤めた後、1898年にはパリのオテル・リッツの、1899年にはロンドンのカールトンホテルの開業に携わり、1920
  年までカールトンホテルで総料理長を務めた。}。本書の計画を実現させるために落ち着いた時間を取り戻さねばならなくなってしまった。

1898年から放置したままだった本書に再び着手出来たのは、多くの同僚たる料理人諸君の助力と、友人でもあるフィレアス・ジルベール君とエミール・フェチュ\footnote{Emile
  Fétu 生没年不詳。}君の献身的な協力を得られたからに他ならない。この一大事業を完成させることが出来たのは、ひとえに皆の励ましと、とりわけ辛抱強く、粘り強く仕事を手伝ってくれた二人の共著者\footnote{ジルベールとフェチュを指しているが、初版には、この二人の他にも共著者として4人の名が挙げられている。第二版以降は共著者としてジルベールとフェチュの名しかクレジットされていない。第二版は初版から構成も含め大幅な改訂が行なわれた。その作業を実際に行なったのがジルベールとフェチュだったために、他の共著者のクレジットが抹消されたと考えられる。なお、現行の第四版にはエスコフィエの名しかクレジットされていない。}のおかげだ。

私が作りたいと思ったのは立派な書物というよりはむしろ実用的な本だ。だから、執筆協力者の皆には、作業手順を各自の考えにもとづいて自由にレシピを書いてもらい、私自身は、40年にわたる現場経験に即して、少なくとも原理原則、料理における伝統的基礎を明確に説明するのに専念した\footnote{字句どおりにとれば、各章、各節における「概説」に相当する部分と、「原注」はもっぱらエスコフィエ自身の手になるものであると解釈されよう。ただし、口頭によるコメントの「聞き書き」的なものも含まれていることは原書の文体における「ゆらぎ」から推測することは可能。}。

本書は、かつて私が構想したとおりとは言い難い出来だが、いずれはそうなるべく努めねばなるまい。それでもなお、現状でも料理人諸君にとって大いに役立つものと信じている。だからこそ、本書を誰にでも、とりわけ若い料理人にも買える価格にした\footnote{1903年の初版の売価は、\href{http://gallica.bnf.fr/ark:/12148/bpt6k65768837}{フランス国立図書館蔵}のものの表紙には、フランス国内で12フランと記したシールが貼られている。また、\href{https://archive.org/details/b21525912}{リーズ大学図書館蔵の第二版}にも同様に国内売価12フランのシールが貼られている。1912年の第三版も同じく12フランだった(\href{http://gallica.bnf.fr/ark:/12148/bpt6k96923116}{フランス国立図書館蔵}のものに価格を示すシールはないが、訳者個人蔵のものには12フランと記されたシールが貼られている)。なお、辻静雄は「1903年の初版発売当時は、800ページでたった8フラン、全く破格の値段だった」(「エスコフィエ 偉大なる料理人の生涯」、『辻静雄著作集』、新潮社、1995年、729〜730頁)と記しているが、その数字の典拠は示されていない。現在と当時の通貨価値、物価の違いが分りにくいため、この「破格に安い」という言葉にはやや疑問が残るだろう。1900年当時の書籍広告において『料理の手引き』初版と同様の八折り版800ページの料理書が、フランス装10フラン、厚紙の表紙のものが11フランとあるため、初版の12フランという価格は、むしろ料理書としては一般的だったと考えられる。つまり、豪華本ではなく、普通に利用できる料理書だということを強調しているに過ぎないと解釈すべきところだろう。なお、八折り判というのは書籍の大きさを表す用語で、概ね縦20〜25
  cm、横12〜16
  cm程度。この序文でことさらに「実用性」や入手しやすい価格であることが強調されているのは、何度も言及されているデュボワとベルナールの名著『古典料理』が四折り判(概ね縦45
  cm、横30 cm)の豪華本であったことを意識していたためとも推測されよう。}。そもそも若い料理人諸君にこそこの本を読んで
\ruby{貰}{もら}いたい。今はまだ初心者であったとしても、20年後には組織のトップに立つべき人材なのだから。

私はこの本を豪華な装丁の\footnote{かつてフランスでは、大判の紙の両面に印刷して折ったものを糸で綴じただけの状態(いわゆる「フランス装」)で販売された本を、書店で買い求めた者が別途、業者に製本、装丁させることが一般的に行なわれていた。}、書棚の飾りのごときにはして欲しくない。そうではなく、いつでも、どんな時でも手元に置いて、分からないことを常に明らかにしてくれる\ruby{盟友}{めいゆう}として欲しい。

本書には五千を越える\footnote{初版、第二版は「五千近い」。第三版になってようやくこの表現になった。}レシピが掲載されているが、それでも私は、この教本が完全だとは思っていない。たとえ今この瞬間に完璧であったとしても、明日にはそうではないかも知れぬ。料理は進化し、新しいレシピが日々創案されている。まことにもって不都合だが、版を重ねる毎に新しい料理を採り入れ、古くなってしまったものは改善せねばなるまい。

ユルバン・デュボワ、エミール・ベルナール\footnote{Emile Bernard
  (1827〜1897)。クラシンスキ将軍の料理人を務めた。}両氏の著作\footnote{デュボワとベルナールの共著は他にもあるが、ここでは『古典料理』(1856年)を指している。}に昔から慣れ親しみ、その巨大な影がなおも料理の地平を覆い尽している現在、私としては本書がその後継になって欲しいと思っている。カレーム以後、最高の料理の高みに逹した二人に対し、ここであらためて心から敬意を表させていただきいと思う。

調理現場を取り巻く諸事情により、私は、デュボワ、ベルナール両氏がもたらしたサービス(給仕)面での革新\footnote{\protect\hypertarget{service-russe}{19世紀後半に一般的となった
  「ロシア式サービス」のこと。中世以来、格式の高い宴席では、卓上に大
  皿の料理が一度に何種も並べられ、食べる者がそれぞれ好きなように取り
  分けていた。そして卓上の料理がほぼなくなると、また何種類もの皿が卓
  上に並べられる、というのが数回繰り返された。19世紀中頃から、献立を
  食べる順に1種ずつ、大皿料理の場合は食べ手に見せて回ってから、給仕
  が取り分けて供する方式に変えたものがロシア式サービスである。これと
  対比するかたちで旧来の方式をフランス式サービスと呼ぶようになった。
  ロシア式サービスでは、食卓に大皿を並べない代わりに、花を飾りナフキ
  ンを美しく折るなどの工夫により卓上も洗練されたものとなっていった。
  19世紀パリに駐在していたロシア帝国の外交官クラーキンが提唱したと言
  われている。デュボワとベルナールの『古典料理』序文において詳述され
  ている。}}に対し、こんにちのようなとりわけスピードが重視される目まぐるしい生活リズムに合わせて、大きな変更を加えざるを得なかった。そもそも物理的理由から、料理を載せる飾り台\footnote{\protect\hypertarget{socle}{socle ソークル。パンや米、ジュレな
  どで作った、料理を盛り付けるために銀の盆の上に据える飾り台。カレー
  ムの時代、つまり19世紀前半にはその装飾に凝ることが多かった。食べも
  ので作られてはいるが、料理の一部ではなく、あくまで装飾的要素でしか
  なかった。この飾り台はロシア式サービスの時代になっても豪華絢爛たる
  宴席においては重要なものとして扱われており、デュボワとベ}ルナール『古典料理』でも相応のページ数を割いて説明がなされている。}をやめて、シンプルな盛り付けにする新たなメソッドと新たな道具を考案する必要があったのだ。デュボワ、ベルナール両氏が推奨した壮麗な盛り付けを私自身も行なっていた頃はもちろん、今なお二方の思想にはまったく共感している。冗談でこんなことを言っているのではない。しかし、カレームを信奉する者たちは、装飾の才があるが\ruby{故}{ゆえ}に、時代にもはや\ruby{似}{そぐ}わなくなってしまった作品に対して改良を加えようとはしなかった。時代に合わせて改良することこそ、まさに重要なのに。本書で奨励している盛り付けは、少なくともそれなりの期間、有用であり続けると思う。全ては変化する。姿を変える。それなのに、装飾芸術の役割が変化しないと主張するなどとは\ruby{蒙昧}{も
うまい}ではないか。芸術は流行によって栄えるものだし、流行のように移ろいやすいものだ。

だが、カレームの時代にはこんにちと同じく\ruby{既}{すで}にあり、料理が続く限りなくならないだろうものがある。それが料理のベースとなるフォンやストックだ。そもそも、料理が見た目にシンプルになっても料理そのものの価値は失なわれないが、その逆はどうだろう?
人々の味覚は絶え間なく洗練され続け、それを満足させるために料理そのものも洗練されることになる。こんにちの余剰活動が精神におよぼす悪影響に打ち\ruby{克}{か}つためには、料理そのものがいっそう科学的な、正確なものとなるべきなのだ。

その意味で料理が進歩すればする程、我々料理人たちにとって、19世紀、料理の行く末に大きく影響を与えた三人の料理人の存在は大きなものとなるだろう。カレームとデュボワ、ベルナールはともすれば技術的側面ばかり評価されるが、料理芸術の基礎において何よりも優れているのだ。

既に物故した名だけ挙げるが、確かにグフェ\footnote{Jules Gouffé
  (1807〜1877)。著書多数。主著『料理の本(1867年)は前半が家庭料理、後半が高級料理(オート・キュイジーヌ)の2部構成になっており、レシピもまず材料表を掲げた後に調理手順を説明するという現代の書き方に近く、挿絵も多く分りやすい。この『料理の手引き』とともに19世紀後半のフランス食文化史における名著のひとつ。19世紀前半からのヴィアールやオドが版を内容を増補しながら版を重ねたのに対して、この本は再版の際もほとんど異同がない点もまた特徴のひとつ。}、ファーヴル\footnote{Joséph
  Favre
  (1849〜1903)。スイス生まれの料理人で、パリ、ドイツ、イギリス、ベルギー等において活躍した。著書『料理および食品衛生事典』
  (1884〜1895年)。この『事典』に収録されているレシピの数は5,531であり(番号が振られている)、エスコフィエがレシピ数5千という表現にややこだわりを示しているように思われるのも、ほぼ同時代の出版物であるファーヴルの『事典』を意識していた可能性はある。}、エルーイ\footnote{Edouard
  Hélouis(生没年不詳)。イギリスのアルバート王配(ヴィクトリア女王の夫)(1819〜1861)やイタリアのヴィットーリオ・エマヌエーレ二世(1820〜1878)に仕えたという。著書『王室の晩餐』(1878年)。}、ルキュレ\footnote{『実践的料理』(1859年)の著者C.
  Reculetのこと。}はとても素晴らしい著作を残した。だが、『古典料理』という\ruby{稀代}{きたい}の名著に\ruby{比肩}{ひけん}し得るものはひとつとしてない。

料理人諸君に、新たに本書を使っていただくにあたり、言うべきことがある。いろいろな料理書、雑誌を読み散らかすのもいいが、偉大な先達の不朽の名著はしっかり読み込むように、と。\ruby{諺}{ことわざ}にあるように「知り過ぎることなはい」のだ。学べば学ぶ程、さらに学ぶべきことは増えていく。そうすれば、柔軟な思考が出来るようになり、料理が上達するためのより効果的な方法を知ることも出来るだろう。

本書を\ruby{上梓}{じょうし}するにあたって\ruby{唯}{ただ}ひとつ望むこと、切に願う\ruby{唯一}{ゆいいつ}のことは、上記の点において、本書の対象たる読者諸君が我が\ruby{言}{げん}に耳を傾け、実践するさまを見ることに尽きる。\nopagebreak

\begin{flushright}
A. エスコフィエ \nopagebreak
\end{flushright}

1902年11月1日

\newpage

\hypertarget{introduction-deuxieme-edition}{%
\section[第二版序文]{\texorpdfstring{第二版序文\footnote{この第二版序文は文体が初版序文と異なり、とりわけ前半部分については、いわゆる「悪文」と見なさざるを得ないものとなっている。また、前半と後半でも文体の「ゆらぎ」のようなものが認められる。内容から判断するかぎり、エスコフィエ自身の言葉であることは確かだが、末尾に署名がなく日付のみ記されていることも含めて考えると、ジルベールとフェチュによる「聞き書き」によって作成された可能性も完全には否定できないと思われる。}}{第二版序文}}\label{introduction-deuxieme-edition}}

\normalsize
\setstretch{1.1}
\vspace*{1\zw}

ここに第二版を上梓するに至ったわけだが、二人の共著者による熱意あふれる仕事のおかげで、私の強い期待をさらに越える本書の成功が約束されたも同然だろう。だからこそ、共著者両君および本書の読者諸君に心からの謝辞を申しあげる次第だ。また、ありがたいことに、称賛の言葉を寄せてくださった方々と、貴重な批判をくださった方々にも御礼申しあげる。批判については、それが正当なものと思われる場合については、本書に反映させるべく努めさせていただいた。

かくも多くの人々に本書を受け入れていただけたことへの謝意を表するには、本書における技術的な価値を高め、初版ではロジカルにレシピを分類しようとしたが故に生じた欠点を解消する他ないだろう。それは、調理理論とレシピを損なうことなしに、本書の計画段階において簡単に済まさざるを得ないと思われたテーマについて\ruby{能}{あた}う限り肉薄することでもある。私たちは本文の見直しをするとともに、多くのレシピを追加した。そのほとんどは調理法と盛り付けにおいて、こんにちの顧客のニーズを\ruby{鑑}{かんが}みて着想したものであり、そのニーズが正当かつ実現可能な範囲において、顧客への給仕のペースが日増しに加速していく傾向をも考慮に入れたものだ。こういった傾向は数年来まさしく際立ってきているが\ruby{故}{ゆえ}に、我々としも常に気を配っておかねばならぬ。

「料理芸術」というものは、その表現形態において、社会心理に左右されるものだ。社会から受ける衝撃に逆らわぬことも必要であり、\ruby{抗}{あらが}
えぬことでもある。快適で安楽な生活がいかなる心配事にも乱されることのないような社会であれば、未来が保証され、財をなす機会もいろいろあるような社会であれば、料理芸術はたゆまぬことなく驚異的な進歩を遂げるだろう。料理芸術とは、ひとが得られる悦びのうちでもっとも快適なもののひとつに寄与しているのだから。

反対に、安穏とした生活の出来ぬ、商工業からもたらされる\ruby{数多}{あま
た}の不安で頭がいっぱいになるような社会において、料理芸術は心配事でいっぱいの人々の心のごく限られた部分にしか美味しさを届けられない。ほとんどの場合、諸事という渦巻きに巻き込まれた人々にとって、食事をするという必要な行為はもはや悦びではなく、辛い義務でしかないのだ。

\ruby{斯}{か}くのごとき生活習慣は\ruby{嘆}{なげ}いていい、\ruby{否}{い
な}、嘆くべきことなのだ。食べ手の健康という観点からも、食べたものを胃が受け付けないという結果になるとしたら、それは絶対に生活習慣が悪いのだ。そういう結果を抑える力は私に出来る範囲を越えている。そういう場合に調理科学が出来ることといえば、軽率な人々に\ruby{能}{あた}うかぎり最良の食べものを与えるという対症療法だけなのだ。

顧客は料理を早く出せと言う。それに対して私たち料理人としては、ご満足いただけるようにするか、失望させてしまうことのどちらかしか出来ない。料理を早く出せという顧客の要求を拒む方法があるとするなら、それ以上の方法で顧客にご満足いただけるようにすることしかない。だから、私たちは顧客の気まぐれの前に折れざるを得ないのだ。これまで私たちが慣れ親しんできた仕事のやり方では、これまでの給仕のスタイルでは、顧客の気まぐれに応えることが出来ぬ。意を決して仕事の方法を改革すべきなのだ。だがひとつだけ、変えてはならぬ、手をつけてはならぬ領域がある。料理ひとつひとつのクオリティだ。それは、料理人にとって仕事のベースとなるフォンや事前に仕込んでおいたストック類がもたらすゆたかな風味に他ならぬ。私たちは既に、盛り付けの領域においては改革に着手した。足手まといにしかならぬ多くのものは既に姿を消したか、いままさに消え去らんとしている。料理の飾り台\footnote{socle
  (ソークル)、\protect\hyperlink{socle}{序p.ii訳注4}参照。}、料理の周囲の装飾\footnote{bordure
  ボルデュール。本書においてもガルニチュールの扱いにおいてこの指示はあるが、19世紀のものと比較するとかなりシンプルな内容になっている。}、飾り串\footnote{hâtelet
  アトレ。一方の端に動物などの姿の装飾の施された銀製の串に、トリュフやクルヴェット(海老)などを事前に別の串(ブロシェット)で焼いてからこの飾り串に刺し直し、それを大きな塊肉や丸鶏、大型の魚
  1尾の料理に刺した。19世紀初頭、カレームの時代に全盛となり、その著書『パリ風料理』において詳述されている。19世紀末まではこの装飾がなされることが多かった。また、その飾り串そのものが美麗な装飾品であるためにコレクションの対象になっていた。}などのことだ。この方向性は推し進められると思う。これについては後述しよう。私たちはシンプルであるということを極限まで追究したい。それと同時に、料理の風味や栄養面での価値を増すことも目指している。料理はより軽い、弱った胃にも優しいものにしたいと考えている。私たちはこの点にのみ尽力したい。料理において役をなさない大部分はすっかり剥ぎ取ってしまいたいと考えているのだ。一言でまとめると、料理は芸術であり続けつつも、より科学的なものとなるだろうし、その作り方はいまだ経験則に基づいただけのものばかりであるが、ひとつのメソッド、偶然などに左右されない正確なものになっていくことだろう。

こんにちは料理の過渡期にある。古典料理メソッドの愛好者はいまなお多く、私たちもそれを理解し、その思想に心から共感するところもある。だが、食事というものがセレモニーであり、かつパーティであった時代を懐しんでどうするというのだ?
古典料理がこんにちの美食家に至福の時を与えるために力を発揮出来る場がどこにあるというのだ?
いったいどうすれば、美食と宴の神コモス\footnote{フランス語 Comus
  (コミュス)。ラテン語では同じ綴りでコムスと読む。ギリシア、ローマ神話における、悦びと美食の神。18世紀の料理本作家マランの主著は『コモス神の贈り物』がタイトル。}に捧げ物を供えるという幸せな機会を毎回得られるのだろうか?だから私たちは本書において、個人的な創作よりむしろ伝統的なフランス料理のレシピ集として、こんにちの料理のレパートリーから姿を消してしまったものも残すことに固執した。その名に値する料理人なら、機会さえ与えられたら王侯貴族も近代の大ブルジョワもひとしく満足させるためには、知っておくべきものなのだ。時間のことなんぞ気にもせぬ穏かな美食家の方々にも、時こそ全てと言わんばかりの金融家やビジネスマンたちにも満足していただくために。だから、本書が新しいメソッドに偏ったものだという非難にはあたらない。私はただ単に、料理芸術の進化の歩みをたどり、いまの時代に即しつつ、食べ手すなわち食事会の主催者と招待客の皆様の意向を絶対的なものとして、それに従いたいと願っているだけなのだ。食べ手の意向に対して私たち料理人は
\ruby{頭}{こうべ}を垂れて従うことしか出来ぬのだから。

私たちは、料理の美味しさを損なうことなくより早く料理を提供できるような方法を、料理人各人が自らの嗜好を犠牲にすることなしに探求すべく
\ruby{誘}{いざな}うことこそが、料理人諸君にとって有益と信じている。全体として、私たちのメソッドはまだまだ日々のルーチンワークに依存し過ぎているものだ。顧客の求めに応えるため、私たちは既に仕事のやり方をシンプルなものにせざるを得なかった。だが、残念ながらいまだ\ruby{途}{み
ち}\ruby{半}{なか}ばに過ぎぬと感じている。私たちは自己の信念をしっかり堅持しており、どうしようもない場合にのみ自説を曲げることもある。だから、装飾に満ちた飾り台を廃止した一方で、盛り付けに時間のかかる厄介で複雑なガルニチュールは残してある。こういったガルニチュールを濫用することはガストロノミーの観点から言って、常に間違っているのは事実だが、残しておくべきものと思われる。それを求める顧客あるいは食事会主催者に絶対に従う必要のある場合はとりわけそうだ。ごく稀にとはいえ、料理の美味しさを損なうことなくそれらを実現可能なこともあるからだ。時間と金銭、広くてスタッフの充実した会場、という3つの本質的要素を最大限活用可能な場合のことだが。

通常の厨房業務においては、ガルニチュールをかなりシンプルな、せいぜい3〜
4種の構成要素からなるものに減らさざるを得なくなっている。そのガルニチュールを添える料理がアントレであれルルヴェ\footnote{19世紀前半まで主流であった「フランス式サービス」つまり、一度に多くの料理の皿を食卓に並べるという給仕方式において、ポタージュを入れた大きな深皿が空くと、それを給仕が下げて、豪華な装飾を施した大きな塊肉の料理がポタージュを置いてあった場所に据えられた。これを
  \protect\hypertarget{releve}{relevé}ルルヴェ(交代したもの、の意)と呼んだ。エスコフィエの時代にはフランス式サービスではなくロシア式サービスに移っており、大きな塊肉の料理や大型の魚1尾まるごとを大皿で出し、給仕が切り分けて配膳するようになっていたが、名称はそのまま残った。Entréeアントレ(もとは「入口」の意)は現代において「前菜」の意味で用いられているが、食卓に大皿で並べられた肉料理(場合によっては魚料理も含む)の総称としてこの語が用いられていた。本書はそれを踏襲している。本書においてルルヴェおよびアントレに分類されている料理の多くは現代においてコース料理の「メイン」に相当するものが多く、実際、英語ではコース料理のメインのことを現在でもこの語で表わすことが多い(前菜はappetizerアペタイザーと呼ぶ)。}であれ、牛・羊肉料理であれ、家禽であれ魚料理であれ、そうせざるを得ない。そのようにして構成要素を減らしたガルニチュールは、素早い皿出しが要求される場合には必ず、ソースと同様に別添で供するのがいい。その場合、盛り付けは奇抜というくらいシンプルなものとなるが\ldots{}\ldots{}メインの料理はより冷めない状態で、より早く、よりきれいに供することが可能になる。給仕が料理を取り皿に分けてお客様に出すにせよ、お客様が大皿を自分たちで受け渡して取り分けるにせよ、サービス担当者は安心して仕事が出来るし、そのほうが容易だ。メインの大皿が山盛りになることはないし、その上に盛り付けられたいろいろな素材のガルニチュールも簡単に取ることが出来るからだ。

こんにちの他のシステムだと、料理を載せるための台や装飾のための飾り串を作り、さらに料理の周囲にガルニチュールを配置するのに、看過出来ぬ程の時間を要していた。こういう盛り付けというのは、料理そのものがさして大きくないものであっても、食べ手の人数が少ない場合であっても、大面積の皿を用いる必要があった。だから、お客様が料理を自分たちで受け渡して取り分ける必要がある場合などは、お客様にとっても、サービス担当者にとってもまことに窮屈なものであった。これは、複雑な構成のガルニチュールの持つ大きな欠点のひとつとして無視できないことだ。他の欠点というのは、あらかじめ盛り付けを行なうことによって美味しさが減じてしまうこと、食べ手が少人数の場合には必然的に、料理を見せて周る間に冷めてしまうこと、などがある。こういう愉快とは言えぬことの結果は何とも情けないことになる。つまり、お客様に大皿に盛り付けた料理をお見せするのはほんの一瞬だけ、お客様は多少なりとも豪華で精密に盛り付けられた料理をちらりと見る暇があるかないか、ということだ。昔日のごとき豪華壮麗な料理を供することの可能な場所もこんにちでは少なくなってきたが、それ以外のところでもこういった悪習が頑固なまでに続けられているというのは、それが昔からの習慣だということでしか説明がつかぬ。

給仕のスピードを容易に上げるために、大きな塊肉の料理でない場合には毎回、下の図のごとき四角形の深皿を出来るだけ用いるよう是非ともお勧めしたい。温かい料理でも、冷製の料理でも、この皿は非常に優れたものであるから、その目的において厨房に備えておくべきものとして他の追随を許さないと言える
\footnote{この段落は、初版の序文の後にある「盛り付け方法をシンプルにすることについて」という挿絵付きの節の内容を短かく縮めたために、ややわかりにくいものになっている。ただし、第二版および第三版においては序文の最後に皿の挿絵が添えられている。}。

繰り返しになるが、本書が新しい方法を勧めているからといって、偏見で古典的なものを悪いと断じているのでは決してない。私たちは、料理人諸君に、顧客たちの生活習慣や味の好みを研究し、自らの仕事をそれらに適合させるよう
\ruby{誘}{いざな}いたいと思っているだけなのだ。我々料理人にとって高名な師とも呼ぶべきカレームは、ある日、同業たる料理人のひとりとおしゃべりをしていた際に、その料理人が仕えている主人の洗練さに欠けた食事の習慣や下卑た味覚を苦々しげに語るのを聞かされたという。その食事の習慣と味覚に憤慨して、自分が人生をかけて追究してきた知的な料理の原則を曲げてまで仕え続けるくらいなら、いっそ辞めてしまいたいと思っている、と。カレームはこう答えた。「そんなことをするのは君のほうが間違っているよ。料理において原則なんていくつも存在しないんだ。あるのはひとつだけ、仕えているお方に満足していただけるか、ということだけなんだよ」と。

今度は我々がその答を考える番だ。自分たちの習慣やこだわりを、料理を出す相手に押しつけるなどと言い張るとしたら、まったくもって馬鹿げたことだ。我々料理人は食べ手の味覚に合わせて料理することこそが第一でありもっとも本質的なことなのだと、私たちは確信している。

私たちがかくも安易に顧客の気まぐれにおもねったり、過度なまでに盛り付けをシンプルにするせいで、料理芸術の価値を下げ、単なる仕事のひとつにしてしまっている、と非難する向きもあるだろう。\ldots{}\ldots{}だがそれは間違いだ。シンプルであることは美しさを排するものではない\footnote{この序文における名句のひとつ。ただし、エスコフィエの時代における「シンプル」とポストモダン以降の時代であるこんにちの「シンプル」はもはや具体的な意味がまったく違うことに留意。もちろん、理念として普遍的な価値を持つ名句であることは確かだろう。}。

ここで、本書の初版において盛り付けについて述べた部分を繰り返すことをお許しいただきたい。

「どんなにささやかな作品にも自らの最高の印をつけられる才というのは、その作品をエレガントで歪みのないものに見せられるわけで、技術というものに不可欠だと私は信じている。

だが、職人が美しい盛り付けを行なうことで自らに課すべき目的とは、食材を他に類のない方法で節度をもって用いつつ大胆に配置することによってのみ、実現されるのだ。未来の盛り付けにおいて絶対に守るべきこととして、食べられないものを使わないこと、シンプルな趣味のよささこそが未来の盛り付けに特徴的な原則となるだろうことを、認めるべきなのだ。

そのような仕事を成し遂げるために、能力ある職人にはいくつもの手段がある。トリュフ、マッシュルーム、固茹で卵の白身、野菜、舌肉などの食べられるものだけを用いて、素晴らしい装飾を組み合わせ、無限に展開できるのだ。

王政復古期\footnote{1814年ナポレオンが退位して国外へ亡命、ルイ18世を戴く王政へ回帰した時期。1830年まで続いたが7月革命でブルボン家は断絶し、その後オルレアン朝による七月王政が1848年まで続いた。}に料理人たちによって流行した複雑な盛り付けの時代は終わった。だが、特殊な例になるが、古い方法で盛り付けをしなければならない場合もあり、そういう時は何よりもまず、盛り付けにかかる時間と利用できる手段を見積らなくてはならない。土台の形状を犠牲にしなくても、装飾の繊細さを忘れなくても、風味ゆたかな素材を軽んじたり劣化させてしまっては、価値のないものにしかならないのだ」。

以上の見解はずっと変わっていない。料理は進歩する(社会がそうであるように)。だが常に芸術であり続けるのだ。

例えば、1850年から人々の生活習慣、習俗が変化したことを皆が認めるにやぶさかでないように、料理もまた変化するのだ。デュボワとベルナールの素晴しい業績は当時のニーズに応えたものだ。だが、たとえ二人がその著書と同じく永遠の存在であったとしても、彼らが称揚した形態は、料理の知識として、我々の時代の要求に応えうるものではない。

私たちは二人の名著を尊重し、敬愛し、研究しなくてはならない。それはカレームとともに、料理人の仕事の\ruby{礎}{いしずえ}たるものだ。だが、書いてあることを盲目的に真似るのではなく、私たち自身で新たな道を切り
\ruby{拓}{ひら}き、私たちもまたこの時代の習俗や慣習に合わせた教本を残すべきと考える次第だ。

\begin{flushright}
1907年2月1日
\end{flushright}

\newpage

\hypertarget{introduction-troisieme-edition}{%
\section{第三版序文}\label{introduction-troisieme-edition}}

\vspace*{1\zw}

『料理の手引き』第三版を同業たる料理人諸賢に向けて上梓するにあたり、絶えず本書を好意的に支持してくださったことと、多くの方々から著者一同にお寄せくださった励ましのお言葉に対し、あらためて深く御礼申しあげる次第だ。

第二版序文の内容につけ加えるべきことは何もない。というのも、第二版序文で料理という仕事について申しあげたことは、1907年当時も今も変わっていない事実だからだし、今後も長くそうであり続けるだろう。とはいえ、この第三版は内容を精査し、かなりの部分を改訂してある。かつては予測でしかなかったことを実証し、この『料理の手引き』初版の序文においてエスコフィエ氏\footnote{この表現から、第三版序文がエスコフィエ自身ではなく、フィレアス・ジルベールかエミール・フェチュのいずれか、あるいは二人によって書かれたと判断される。}が以下のように書かれた約束も果せたと思う。「本書には五千近くもの\footnote{初版および第二版では「五千近い」となっており、第三版で「五千以上」と表現が変更された。}レシピが掲載されているが、それでも私は、この教本が完全だとは思っていない。たとえ今この瞬間に完璧であったとしても、明日にはそうではないかも知れぬ。料理は進化し、新しいレシピが日々創案されているのだ。まことにもって不都合なことだが、版を重ねる毎に新しい料理を採り入れ、古くなってしまったものは改良を加えねばなるまい。」

この言葉が、前回の第二版から300ページを増やしたことの説明となっているわけで、この新版でいくつかの変更を我々が必要と考えた理由でもある。

\begin{enumerate}
\def\labelenumi{\arabic{enumi}.}
\item
  判型の変更\ldots{}\ldots{}あえて判型を大きくすることで、より扱いやすいものとしたこと\footnote{初版および第二版はいわゆる「八折り版」約21.5
    cm×13.5 cmであったのに対し、第三版は約24 cm×16
    cm、つまり現代のB5版よりほんの少し小さめの判型。}
\item
  巻末の目次の組みなおし\ldots{}\ldots{}当初は料理の種類別であったが、本書全体の項目をアルファベット順にまとめたこと\footnote{原文ではTable
    des
    Matière「目次」とあるが、これは巻頭の章を示す目次のことではなく、巻末の「索引」のこと。}
\item
  時代遅れになったと思われるレシピを相当数削除し、その代わりとしてこの数年の間に創案され好評を博したレシピを追加したこと
\end{enumerate}

既に大著であって本書にこれらの変更を加えるために、我々は第二版の巻末に付されていた献立のページを削除せざるを得なかった。

献立についても内容を一新し、多くの献立例を追加して、『メニューの本』という独立した書籍として、この第三版と同時に刊行する予定となっている。この『メニューの本』において我々は献立とその説明文はもちろんのこと、大規模な厨房における日々の業務配分を示す表を入れておいた。

このように別冊とすることで、献立の作成という非常に重要な問題を適切に展開し、ゆとりを持って論じることが可能となったわけだ。

この新刊『メニューの本』は料理人諸賢だけではなくメートルドテル、食事施設の責任者に必携のものとなった。さらには必要なものを奇抜なまでに単純化してしまう家庭の主婦にとっても必携となろう。我々は上記の改良点が、これまで多くの好意的見解をお寄せくださった料理関わる皆様方に、好意的に受け容れていただけると信じている。また、料理芸術の栄光のもと未来に続くモニュメントを建てるべく努めた我々のささやかなる尽力が、料理芸術に利をもたらさんことを信じる次第だ。

\begin{flushright}
1912年5月1日
\end{flushright}

\hypertarget{introduction-quatrieme-edition}{%
\section{第四版序文}\label{introduction-quatrieme-edition}}

\vspace*{1\zw}

『料理の手引き』第三版刊行当時(1912年5月)から後、他の職業、産業と同様に料理界もまた大いなる危機に見舞われた\footnote{第一次世界大戦(1914〜1918)による社会的影響を指している。フランスは戦中から戦後にかけて激しいインフレに見舞われた。なお、この第四版から出版社がそれまでのラール・キュリネールからフラマリオン社に変わった。}。こんにちもなお料理は厳しい試練にさらされている。しかしながら、料理界はその試練に耐えてきたし、戦後のこの辛い時期に終止符を打ち、料理界がさらに前進し始めるのもさして遠いことではないと信じている。だが、目下のところ、あらゆる食材の異常なまでの高騰により、料理長諸賢が責務を果すことがひどく難しくなっている。料理長がその責務を果すということの困難さを経験上よく知っているからこそ、今回の版において我々は、多くのレシピ、とりわけガルニチュールについて、その本質的なところを曲げることなしに、よりシンプルなものにすることにこだわった。

さらに、もはやあまり興味を持たれないであろうレシピは全て削除して、その代わりに近年創案されたレシピを収録することとした。

したがって、料理人諸賢および料理に関心を持つ皆様方に向けてこの『料理の手引き』第四版を上梓するにあたり、旧版同様、皆様に温かく受け容れていただけると信じる次第である\footnote{原書の文体から、この序文も第三版序文と同様に、ジルベールとフェチュによって書かれた可能性も考えられる。}。

\begin{flushright}
1921年1月
\end{flushright}

\newpage
\small
\setstretch{1.0}

\hypertarget{remarque-sur-la-simplification-des-procedes-de-dressage}{%
\section{【参考】盛り付けをシンプルにするということ(初版のみ)}\label{remarque-sur-la-simplification-des-procedes-de-dressage}}

本書では、かつては料理の盛り付けによく用いられた飾り串\footnote{hâtelet
  アトレ。}、縁飾り \footnote{bordure ボルデュール。}、クルトン\footnote{菱形やハート形にしたパンを揚げたもの。}、チョップ花\footnote{papillote
  パピヨット。紙製で、骨付き肉の先端を飾るもの。}などを使う指示がほとんど出てこない。著者としては、盛り付け方法を近代化すると同時に、ほぼ完全に上記のものどもを削除しなくしてしまいたいとさえ考えたくらいだ。

我らが先達が考えていたような盛り付けには、長所がたったひとつしかない。皿を荘厳に、魅力的な姿にすることで、料理を味わう前に、食べ手の目を楽しませ、喜んでいただくということだ。

だが、そうした盛り付けの作業は複雑で難しいものであり、かなりの時間を必要とする。比較的少人数の宴席でないかぎりは、こうした盛り付けは事前に用意しておく必要がある。そのようにして作られた料理は、それを置いておく場所のことを考えに入れないとしても、必ずといっていい程、冷めてしまっている。また、料理を載せる台や縁飾り、飾り串に費す時間も考えなくてはならないし、そういった装飾にかかる費用も考えなくてはならない。忘れてはならないことだが、そのように装飾した皿の見た目の調和がとれている時間というのは、その皿をお客様にお見せする間だけなのだ。メートルドテルのスプーンが料理に触れるやいなや、かくも無惨な姿となりお客様の目には不快なものとなってしまう。こういう不都合はなんとしても改善しなければならなかったのだ。

ここで図に示すような四角形の皿を採用したことで、上記のような問題は解決したと考えている。この皿はパリのリッツホテルで初めて用いられ、ロンドンのカールトンホテルにおいて正式に採用されることとなったものだ。この皿を用いることの利点は絶大で、これを用いない盛り付けなどもはや考えられない程だ。この皿は場所をとらず、皿の内側に盛り付けられた料理は冷めることがない。蓋との距離が近いから保温されているわけだ。魚や肉の切り身は上に重ねて盛るのではなく、ガルニチュールとともに並べて盛り付けることが出来る。そうすることで、最初に給仕されるお客様から最後に給仕される方まで、料理は美味しそうな見た目を保つことが出来るのだ。その結果、クルトンやチョップ花、皿の上にしつらえる料理を載せる台や縁飾り、飾り串、昔の給仕で用いられた面倒なクロッシュ\footnote{cloche
  主に金属製で半球形の保温を目的としたディッシュカバー。}は不要なものとなる。

この皿は冷製料理にもまた便利に使うことが出来る。周囲に氷を積み重ねて囲うか、薄い氷のブロックの上に盛り付ければ、飾りには、ごく繊細なジュレだけていい。そのような繊細なジュレを使うのは昔の方法では不可能だった。かくして、邪魔にさえ思える飾り台も、皿の底の飾りも、アトレも必要なくなった。ショフロワは1切れずつ並べて、周囲を琥珀色のとろけるようなジュレで満たしてやればいい。ムースはもはや「つなぎ」をまったく、あるいはほとんど必要としない。こういうことが、冷製料理の芸術的な見た目を、豪華さや美しさという点でいっかな失なうことなく可能となるのだ。

この新式の什器とそれによって実現可能となる料理に習熟することについて料理人諸君にお報せすることは我々の義務であると考える。利点がとても大きいので、あえて申しあげるが、これを使うことが、給仕を素早く、きれいに、経済的に、そして文句ないまでに実践的なものにする唯一の方法である。

\hypertarget{avertissement-premier-edition}{%
\section{【参考】初版はしがき}\label{avertissement-premier-edition}}

本書はある特定の階層の料理人を対象としているものではなく、全ての料理人が対象であるため、本書のレシピは、経済的観点や料理人が実際に利用可能な手段に応じて、改変できるものだということを述べておきたい。

本書に収められたレシピはすべて、グランドメゾンでの仕事における原則にもとづいて組み立てて調整してある。だから、より格下の店舗などでも、必然的に量を減らせば作れるだろうし、適価で提供出来るようにもなるだろう。

ひとつひとつの項目において、いろいろな飲食を提供する形態を網羅するようにレシピを書くことが不可能だったということは理解されよう。料理人自身が自主性をもって本書の内容を補えるし、そうすべきなのだ。ある者たちにとって非常に大切なことが、大多数の者にとってはそこそこの興味しか引かず、一般的に見たら無益で幼稚に思われることだってあるのだ。

だから、本書に収録したレシピは最大の分量でまとめられたものを考えるべきであり、必要に応じて、各人の判断および物理的に出来る範囲に合わせて、量を減らして作るといい。
\normalsize \setstretch{1.0}


\mainmatter

%% Chapitre I. Sauces

\hypertarget{sauces}{%
\chapter{I. ソース Sauces}\label{sauces}}

\hypertarget{les-fonds-de-cuisine}{%
\section{フォン、その他のストック}\label{les-fonds-de-cuisine}}

\frsec{Les Fonds de Cuisine}

\index{fonds@fonds} \index{ふおん@フォン}

\normalsize
\setstretch{1.0}

本書は実際に厨房で働く料理人を対象としたものだが、まず最初に料理のベー
スとして仕込んでストックしておくもの\footnote{本書での fonds の語は fond
  (基礎、土台)、fonds (資産、資
  本)、そして料理用語として一般に用いられているフォン、のトリプルミー
  ニングになっている。そのまま「フォン」と訳したいところだが、日本語
  の場合「出汁」としての意味合いが強いため、本文中では分りやすさを重
  視してやや冗長に「料理のベースとして仕込んでストックしておくもの」
  のように訳している。}について少々述べておきた い\footnote{この部分は経営者に向けて書かれているようにも読めるが、エス
  コフィエの時代以降、料理人がオーナーシェフとして経営に携わるケース
  が激増したことを考えると、その先見の明に驚かざるを得ない。}。我々料理人にとって重要なものだからだ。

ここで述べる料理のベースとして仕込んでストックしておくものは、実際、料
理の土台そのものであり、それなしでは美味しい料理を作ることの出来ない、
まず最初に必要なものだ。だからこそ、料理のベースとして仕込んでおくストッ
クはとても重要であり、いい仕事をしたいと努めている料理人ほどこれらを重
視している。

これらは、料理において常に立ち戻るべき出発点となるものだが、料理人がい
い仕事をしたいと望んでも、才能があっても、それだけでいいものを作ること
は出来ない。料理のベースを作るにも材料が必要なのだ。だから、必要な材料
は良質のものを自由に使えるようにしなければならない。

筆者としては、むやみな贅沢には反対だが、それと同じくらい、食材コストを
抑え過ぎるのも良くないと考えている。そんなことをしていては、伸びる筈の
才能の芽を摘んでしまうばかりか、意識の高い料理人ならモチベーションの維
持すら出来ないだろう。

どんなに優秀な料理人だって、無から何かを作り出すことは不可能だ。期待さ
れる結果に対して、素材の質が劣っていたり量が足りないことがあれば、それ
でも料理人にいい仕事をしろと要求するなど言語道断である。

料理のベースとして仕込んでおくストックに関するの重要ポイントは、必要な
材料は質、量ともに充分に、惜しげもなく使えるようにすることだ。

ある調理現場で可能なことが、別の調理現場では不可能な場合があるのは言う
までもない。料理人の仕事内容は顧客層によっても変わる。到達すべき目標に
よって手段も変わるということだ。

そういう意味で、何事も相対的なものであるとはいえ、こと料理のベースとし
て仕込んでストックすべきものに関しては絶対に外してはならないポイントが
あるわけだ。組織のトップがこの点で出費を惜しんだり、コスト面で過度に目
くじらを立てるようでは、美味しい料理なんて出来るわけがないのだから、現
実に厨房を仕切っている料理長を批判する資格もない。そんなのが根拠のない
言い掛かりなのは明らかだ。素材の質が悪かったり、量が足りないのであれば、
料理長が素晴しい料理を出せないのは言うまでもあるまい。ぶどうの搾りかす
に水を加えて醗酵させた安ワインを立派な瓶に詰めてしまえば高級ワインにな
ると思う程に馬鹿げたことはないのだ。

料理人は、必要なものを何でも使っていいなら、料理のベースとして仕込んで
おくストックにとりわけ力を入れるべきであり、文句のつけようのない出来に
なるよう気を使うべきだ。そこに手間隙かけていればそれだけ厨房全体の仕事
がきちんと進むのだから、注文を受けた料理をきちんと作れるかどうかは、結
局のところ、料理のベースとなる仕込み類にどれだけ手間\ruby{隙}{ひま}をかけるかという
ことなのだ。

\newpage

\hypertarget{principaux-fonds-de-cuisine}{%
\section{主要なフォンとストック}\label{principaux-fonds-de-cuisine}}

\frsec{Principaux Fonds de Cuisine}

料理のベースとして仕込んでおくべきものは主として\ldots{}\ldots{}

\begin{itemize}
\tightlist
\item
  \textbf{コンソメ・サンプルとコンソメ・ドゥーブル}
\item
  \textbf{茶色いフォン、白いフォン、鶏のフォン、ジビエのフォン、魚のフォン}\ldots{}\ldots{}これらはとろみを付けたジュ、基本ソースのベースになる
\item
  \textbf{フュメ、エッセンス}\ldots{}\ldots{}派生ソースに用いる
\item
  \textbf{グラスドヴィアンド、鶏のグラス、ジビエのグラス}
\item
  \textbf{茶色いルー、ブロンドのルー、白いルー}
\item
  \textbf{基本ソース}\ldots{}\ldots{}エスパニョル、ヴルテ、ベシャメル、トマト
\item
  \textbf{肉料理用ジュレ、魚料理用ジュレ}
\end{itemize}

\vspace{1\zw}

以下も日常的に使う料理のベースとして仕込んでおくものとして扱う。

\begin{itemize}
\tightlist
\item
  \textbf{ミルポワ、マティニョン}
\item
  \textbf{クールブイヨン、肉および野菜用のブラン}
\item
  \textbf{マリナード、ソミュール}
\item
  \textbf{肉料理用ファルス、魚料理用ファルス}
\item
  \textbf{ガルニチュールに用いるアパレイユ}、など\ldots{}\ldots{}
\end{itemize}

\vspace{1\zw}

本書は上記を順に説明していく構成にはなっていない。グリル、ロースト、グ
ラタン等の調理技法についても順を追っていくわけではない。料理の種類ごと
に一定の位置、つまりは関連の深い料理の章の冒頭において説明していくこと
になる。

\vspace{1\zw}

そのようなわけで、本書においては以下のようになる\ldots{}\ldots{}

\begin{itemize}
\tightlist
\item
  フォン、フュメ、エッセンス、グラス、マリナード、ジュレの説明\ldots{}\ldots{}
  \textbf{ 第1章 ソース}
\item
  コンソメおよびそのクラリフィエ、ポタージュの浮き実についての説
  明\ldots{}\ldots{}\textbf{第3章 ポタージュ}
\item
  ファルスとガルニチュール用アパレイユの作り方\ldots{}\ldots{}\textbf{第2章
  ガルニチュー ル}
\item
  クールブイヨン、魚料理用ファルス等\ldots{}\ldots{}\textbf{第6章
  魚料理}
\item
  グリル、ブレゼ、 ポワレの調理理論\ldots{}\ldots{}\textbf{第7章 肉料理}
\end{itemize}

\newpage

\hypertarget{section-grandes-sauces-de-base}{%
\section{基本ソース}\label{section-grandes-sauces-de-base}}

\frsec{Grandes Sauces de Base}

\index{そーす@ソース!きほん@基本---}
\index{sauce@sauce!00grandes@*Grandes ---s de Base}

\begin{itemize}
\item
  \textbf{およびそれらを組み合せたり煮詰めるなどの方法で作る派生ソース}
\item
  \textbf{イギリス風ソース(温製および冷製)}
\item
  \textbf{いろいろな冷製ソース}
\item
  \textbf{ブール・コンポゼ(ミックスバター)}
\item
  \textbf{マリナード}
\item
  \textbf{ジュレ}
\end{itemize}

\hypertarget{osbservation-sur-la-sauce}{%
\section{概説}\label{osbservation-sur-la-sauce}}

ソースは料理においてもっとも主要な位置にある。フランス料理が世界に冠た
るものであるのもひとえにソースの存在によるのだ。だから、ソースは出来る
かぎり手間をかけ、細心の注意を払って作るようにしなければならない。

ソースを作るうえでその基礎となるのが何らかの「ジュ」である\footnote{ここではジュといわゆるフォンが同じ意味で使われている。}。すなわ
ち、茶色いソースは「茶色いジュ」(エストゥファード)から作る。ヴルテ
には「澄んだジュ(白いフォン\footnote{日本の調理現場で「白いフォン」を意味する「フォン・ブラン」は主と
  して鶏のフォンを指すことが多いが、本書で扱われている白いフォンのう
  ち標準的なものは仔牛肉、家禽類をベースとしており、鶏のフォンは別途
  説明されている。})を使う。ソースを担当する料理人はまず
第一に、完璧なジュを作るところから始めなければならない。キュシー侯爵
\footnote{1767-1841。19世紀の著名な美食家。
  著書に『食卓の古典』(1843)があ
  る。料理名にキュシーの名を冠したものも多い。}が言うように、ソース担当の料理人は「頭脳明晰な化学者\footnote{原文
  chimiste。現代は分子ガストロノミーが盛んだが、料理を
  作る過程で起きる現象や結果を「化学」で説明しようとする試みは少なく
  ともカレームまで遡ることが出来る。\protect\hyperlink{fonds-brun}{茶色いフォン}のレシピにおいて言
  及されるオスマゾームという想像上の物質もその範疇に含まれるだろう。
  また、化学の前身たる「錬金術」的概念は中世以来いくつかの料理書にお
  いて散見される。}でありかつ天才
的なクリエイターで、卓越した料理という建造物のいわば大黒柱たる存在」な
のだ。

昔のフランス料理\footnote{本書において「昔の料理」と表現される場合は概ね17〜18世紀末と考え
  ていい。}では、素材に串を刺してあぶり焼きするローストを別に
すれば、どんな料理も「ブレゼ」か「エチュヴェ」のようなものばかりだった。
だが、その時代には既に、フォンが料理という大建築の丸天井の\ruby{要}{か
なめ}だったし、材料コストが重視されるこんにちの我々と比べたら想像も出
来ないくらい贅沢に材料を使ってフォンをとっていたのだ。実際、アンヌ・ドー
トリッシュ\footnote{17世紀に絶対王政を確立したルイ14世の母。}がスペインからルイ13世に嫁いだ際に随行してきたスペインの
料理人たちによってフランス料理にルーを用いる方法が伝えられたが\footnote{ルーがスペインからもたらされたというのは逸話、伝承の域を出ない。}、当時は
ほとんど看過された。ジュそれ自体で充分だったからだ。ところが時代が下り、
料理におけるコストの問題が重視されるようになった。ジュはその結果、貧相な
ものになってしまった。その美味しさを補うものとして、ルーを用いて作るソー
ス・エスパニョルが欠くべからざる存在となった。

ソース・エスパニョルはその完成度の高さゆえに成功をおさめたわけだ。だが、
すぐに当初の目的を越えた使い方をされるようになった。19世紀末には本当に
このソースが必要な場合以外にも使われたわけだ。ソース・エスパニョルの濫
用によって、どんな料理も固有の香りのない、全部の風味の混ざりあったのっ
ぺりとした調子のものばかりになってしまった。

ようやく近年になって、料理の風味がどれも同じようなものであることに批判
が集まってきて、その結果として激しい揺り戻しが起きたのだった。グランド
キュイジーヌでは、透き通ったような薄い色合いでしかも風味のしっかりした
仔牛のフォンが見直されつつある。そのようなわけで、ソース・エスパニョル
それ自体の重要性はだんだん減っていくだろうと思われる。

ソース・エスパニョルが基本ソースとして扱われるべき理由は何か? ソース・
エスパニョルそれ自体に固有の色合いや風味というものはなく、これらはどん
なフォンを用いて作るかで決まる。まさにこの点にソース・エスパニョルの長
所が存するのだ。補助材料としてルーを加えるが、ルーにはとろみを付けると
いう意味しかなく、風味にはまったく寄与しない。そもそも、ソースを完璧に
仕上げるためには、とろみ以外のルーに含まれる成分はソースからほぼ完全に
取り除いてしまっても差し支えはない。不純物を丁寧に取り除いたソースには
ルーに含まれていたでんぷん質だけが残っているわけだ。だから、ソースの口
あたりを滑らかなものにするために必要なのがでんぷん質だけなら、純粋なで
んぷんだけを用いる方がずっと簡単で、作業時間も大幅に短縮されるし、その
結果として、ソースを火にかけ過ぎてしまうようなミスも防げる。将来的には、
小麦粉ではなく純粋なでんぷんでルーを作るようになるかも知れない。

料理界の現状を\ruby{鑑}{かんが}みるに、\ul{ソース・エスパニョル}と
\ul{とろみを付けたジュ}をそれぞれ使い分けざるを得ない。これにはさまざ
まな理由があるが、大きな仕立てのブレゼや、羊や仔羊以外を材料にしたラグー
では、肉汁が煮汁に染み出してきて美味しくなるわけだから、トマトを加えた
ソース・エスパニョルを用いるのがいい。なお、ソース・エスパニョルをさら
に丁寧に仕上げるとソース・ドゥミグラスとなる。これはいろいろなソテーに
不可欠なもので、今後も変わることはないだろう。

一方、牛や羊、家禽を使った繊細で軽い仕立ての料理にはとろみを付けたジュ
の方が好まれる。デグラセの際に少量だけ、料理の主素材と同じものからとっ
たジュを用いる。

こんにちのフランス料理においては、肉とソースの調和がとれているべきとい
う、まことに理に適った厳守すべき決まりがある。

だから、ジビエ料理にはジビエのフォンを用いるか、とりたてて際立った個性
を持たないフォンを用いて作ったソースを添える。牛や羊のフォンは用いない。
ジビエのフォンというのは、さほど濃厚なものを作ることは出来ないが、素材
の個性的な風味を表現するには最適だ。こういった事情は魚料理にも当て
\ruby{嵌}{はま}る。ソースそれ自体が際だった風味を持たないものの場合に
は必ず魚のフュメを加えてやるのだ。このようにしてそれぞれの料理に個性的
な風味を実現させることになる。

もちろん、ここまで述べた原則を実現しようにも、コストの問題がしばしば起
こることは承知している。けれども、熱意のある、他者の評価を意識している
料理人なら問題点を熟考して、完璧とは言わぬまでも満足のいく結果を得るこ
とが出来るだろう。\newpage

\normalsize
\setstretch{1.0}

\hypertarget{traitement-des-elements-de-base}{%
\section{ソースのベース作り}\label{traitement-des-elements-de-base}}

\frsec{Traitement des Éléments de Base dans le Travail des Sauces}

\index{そーす@ソース!そーすつくりのべーす@---のベース作り}
\index{sauce@sauce!Traitement des elements de base dans le travail des sauces@Traitement des Éléments de Base dans le Travail des ---s}
\begin{recette}
\hypertarget{fonds-brun}{%
\subsubsection{茶色いフォン(エストゥファード)}\label{fonds-brun}}

\frsub{Fonds brun ou Estouffade}

\index{ふおん@フォン!ちやいろいふおん@茶色い---}
\index{えすとうふあーと@エストゥファード}
\index{fonds@fonds!brun@--- brun}
\index{fonds@fonds!estouffade@estouffade (fonds brun)}
\index{estouffade@estouffade!fonds brun@ --- (fonds brun)}

(仕上がり10 L分)

\begin{itemize}
\item
  主素材\ldots{}\ldots{}牛すね6
  kg、仔牛のすね6kgまたは仔牛の端肉で脂身を含まな いもの6
  kg、骨付きハムのすねの部分1本(前もって下茹でしておくこと)、
  塩漬けしていない豚皮を下茹でしたもの650 g。
\item
  香味素材\ldots{}\ldots{}にんじん650 g、玉ねぎ650
  g、ブーケガルニ(パセリの枝100 g、 タイム10 g、ローリエ5
  g、にんにく1片)。
\item
  作業手順\ldots{}\ldots{}肉を骨から外す。
\end{itemize}

骨は細かく砕き、オーブンに入れて軽く焼き色を付ける。野菜は焼き色が付く
まで炒める。これらを鍋に入れて14 Lの水を注ぎ、ゆっくりと、最低12時間煮
込む。水位が下がらぬように、適宜沸騰した湯を足すこと。

大きめのさいの目に切った牛すね肉を別鍋で焼き色が付くまで炒める。先に煮
込んでいたフォンを少量加えて煮詰める。この作業を2〜3回行ない、フォン
の残りを注ぐ。

鍋を沸騰させて、浮いてくる泡を取り除く。浮き脂も丁寧に取り除く。蓋をし
て弱火で完全に火が通るまで煮込んだら、布で漉してストックしておく。

\hypertarget{nota-fonds-brun}{%
\subparagraph{【原注】}\label{nota-fonds-brun}}

フォンの材料に牛の骨などが含まれている場合には、事前にその骨だけで12〜
15時間かけてとろ火でフォンをとるといい。

フォンの材料を鍋に焦げ付くくらいまで強く焼き色を付ける\footnote{パンセ
  pincer と呼ばれる手法。原義は「抓む」。材料が鍋底に張り付
  いて、トングなどでしっかり「抓ま」ないと取れないくらい強く焼き付け
  ることからそう呼ばれるようになった。古い料理書では推奨するものも多
  かった。}のはよろしく
ない。経験からいって、丁度いい色合いのフォンに仕上げるには、肉に含まれてい
るオスマゾーム\footnote{19世紀頃、赤身肉の美味しさの本質であると考えられていた想像上の物
  質。赤褐色をした窒素化合物の一種で水に溶ける性質があるとされた。な
  お、当時のヨーロッパではグルタミン酸はもとよりイノシン酸が「うま味」
  の要素であるという概念すらなく、「コクがある」corsé とか「肉汁たっ
  ぷり」onctueux (オンクチュー)や succulent
  (スュキロン)などの表現で肉料理やソースの美味しさが表 現された。}の働きだけで充分だ。

\hypertarget{fonds-blanc}{%
\subsubsection{白いフォン}\label{fonds-blanc}}

\frsub{Fonds blanc ordinaire}

\index{ふおん@フォン!しろい@白い---}
\index{fonds@fonds!blanc ordinaire@--- blanc ordinaire}

(仕上がり10 L分)

\begin{itemize}
\item
  主素材\ldots{}\ldots{}仔牛のすね、および端肉10k
  g、鶏の手羽やとさか、足など、ま たは鶏がら4羽分、
\item
  香味素材\ldots{}\ldots{}にんじん800 g、玉ねぎ400 g、ポワロー300
  g、セロリ100 g、ブー ケガルニ(パセリの枝100
  g、タイム1枝、ローリエの葉1枚、クローブ4本)。
\item
  使用する液体と味付け\ldots{}\ldots{}水12 L、塩60 g。
\item
  作業手順\ldots{}\ldots{}肉は骨を外し、紐で縛る。骨は細かく砕く。鍋に肉と骨を入
  れ、水を注ぎ塩を加える。火にかけ、浮いてくるアクを取り除き香味素材を加
  える。
\item
  加熱時間\ldots{}\ldots{}弱火で3時間。
\end{itemize}

\hypertarget{nota-fonds-blanc}{%
\subparagraph{【原注】}\label{nota-fonds-blanc}}

このフォンは火加減を抑えて、出来るだけ澄んだ仕上がりにすること。アクや
浮き脂は丁寧に取り除くこと。

茶色いフォンの場合と同様に、始めに細かく砕いた骨だけを煮てから指定量の
水を注ぎ、弱火で5時間煮る方法もある。

この骨を煮た汁で肉を煮るわけだ。その作業内容は上記茶色いフォンの場合と
同様。この方法は、骨からゼラチン質を完全に抽出出来るという利点がある。
当然のことだが、煮ている間に蒸発して失なわれてしまった分は湯を足してや
り、全体量を12 Lにしてから肉を煮ること。

\hypertarget{fonds-de-volaille}{%
\subsubsection{鶏のフォン(フォンドヴォライユ)}\label{fonds-de-volaille}}

\frsub{Fonds de volaille}

\index{ふおん@フォン!とりのふおん@鶏の---}
\index{fonds@fonds!volaille@--- de volaille}
\index{かきん@家禽!とりのふおん@鶏のフォン}
\index{うおらいゆ@ヴォライユ!ふおんとうおらいゆ@フォンドヴォライユ}

白いフォンと同じ主素材、香味素材、水の量で、さらに鶏のとさかや手羽、ガ
ラを適宜増量し、廃鶏3羽を加えて作る。

\hypertarget{jus-de-veau-brun}{%
\subsubsection{仔牛の茶色いフォン(仔牛の茶色いジュ)}\label{jus-de-veau-brun}}

\frsub{Fonds, ou Jus de veau brun}

\index{ふおん@フォン!こうしのちやいろい@仔牛の茶色い---}
\index{しゆ@ジュ!こうしのちやいろいしゆ@仔牛の茶色い---}
\index{fonds@fonds!fonds de veau brun@--- de veau brun}
\index{jus@jus!jus de veau brun@--- de veau brun}
\index{こうし@仔牛!こうしのちやいろいふおん@---の茶色いフォン(ジュ)}
\index{veau@veau!fonds brun@fonds ou jus de --- brun}

(仕上がり10 L分)

\begin{itemize}
\item
  主素材\ldots{}\ldots{}骨を取り除いた仔牛のすね肉と肩肉(紐で縛っておく)6kg、
  細かく砕いた仔牛の骨5 kg。
\item
  香味素材\ldots{}\ldots{}にんじん600 g、玉ねぎ400 g、パセリの枝100
  g、ローリエの葉 2枚、タイム2枝。
\item
  使用する液体\ldots{}\ldots{}白いフォンまたは水12 L。水を用いる場合は1
  Lあたり3 gの塩を加える。
\item
  作業手順\ldots{}\ldots{}厚手の片手鍋または寸胴鍋の底に輪切りにしたにんじんと玉
  ねぎを敷きつめる。その他の香味素材と、あらかじめオーブンで焼き色を付けておい
  た骨と肉を鍋に加える。
\end{itemize}

蓋をして約10分間、蓋をして弱火にかけた野菜から水分が汗をかくように出る
イメージで蒸し焼き状態にし、素材の味を引き出す\footnote{suer
  (スュエ)シュエ。}。フォンま
たは水少量を加え、煮詰める。この作業をさらに1〜2回行なう。残りのフォ
ンまたは水を注ぎ、蓋をし、沸騰させる。アクを丁寧に取る。微沸騰の状
態で6時間煮る。

布で漉し、ストックしておく。使用目的や必要に応じて、さらに煮詰めてから
ストックしてもいい。

\hypertarget{fonds-de-gibier}{%
\subsubsection{ジビエのフォン}\label{fonds-de-gibier}}

\frsub{Fonds de gibier}

\index{ふおん@フォン!しひえ@ジビエの---}
\index{fonds@fonds!fonds de gibier@--- de gibier}
\index{しひえ@ジビエ!ふおん@---のフォン}
\index{gibier@gibier!fonds@fonds de ---}

(仕上がり5 L分)

\begin{itemize}
\item
  主素材\ldots{}\ldots{}ノロ鹿の頸、胸肉および端肉3
  kg(老いたノロ鹿がいいが、新 鮮なものを使うこと)、野うさぎ\footnote{lièvre
    (リエーヴル)。}の端肉1 kg、老うさぎ2羽、山うずら2羽、 老きじ1羽。
\item
  香味素材\ldots{}\ldots{}にんじん250 g、玉ねぎ250
  g、セージ1枝、ジュニパーベリー \footnote{セイヨウネズの樹の実。}15粒、標準的なブーケガルニ。
\end{itemize}

\begin{itemize}
\item
  使用する液体\ldots{}\ldots{}水6 Lおよび白ワイン1瓶。
\item
  加熱時間\ldots{}\ldots{}3時間。
\item
  作業手順\ldots{}\ldots{}ジビエは事前にオーブンで焼き色を付けておき、野菜と香草を
  敷き詰めた鍋に入れる。野菜類も事前に焼き色を付けておくこと。ジビエを焼
  くのに用いた天板を白ワインでデグラセし、これを鍋に注ぐ。同量の水も加え、
  ほぼ水分がなくなるまで煮詰める。
\end{itemize}

この作業の後で、残りの水全量を注ぎ、沸騰させる。丁寧にアクを引きながら
ごく弱火で煮る\footnote{最後に布で漉す必要があるが、当然のこととして明記されていないの
  で注意。}。

\hypertarget{fumet-de-poisson}{%
\subsubsection[魚のフュメ(フュメドポワソン)]{\texorpdfstring{魚のフュメ(フュメドポワソン)\footnote{本質的には前出の「フォン」と同様のものだが、魚(およびジビエ)
  を素材としたフォンは香りがポイントとなるため、フュメ fumet (香気、
  良い香りの意)の名称のほうが一般的に使われている。}}{魚のフュメ(フュメドポワソン)}}\label{fumet-de-poisson}}

\frsub{Fonds, ou Fumet de poisson}

\index{ふおん@フォン!さかな@魚の---}
\index{ふゆめ@フュメ!さかな@魚の---}
\index{ふゆめ@フュメ!ほわそん@フュメドポワソン}
\index{fumet@fumet!fumet de poisson@--- de poisson}
\index{fonds@fonds!fumet de poisson@fumet de poisson}

(仕上がり10L分)

\begin{itemize}
\item
  主素材\ldots{}\ldots{}舌びらめ、メルラン\footnote{タラの近縁種。}やバルビュ\footnote{ヒラメの近縁種。}のあら10
  kg。
\item
  香味素材\ldots{}\ldots{}薄切りにした玉ねぎ500 g、パセリの根\footnote{パセリには根がにんじん形に肥大する品種もある(persil
    tubéreux 根パセリ。葉は平らでイタリアンパセリのように使う)。}と茎100
  g、マッ シュルームの切りくず250 g、レモンの搾り汁1個分、粒こしょう15
  g(これは フュメを漉す10分前に投入する)。
\item
  使用する液体と調味料\ldots{}\ldots{}水10 L、白ワイン1瓶。液体1
  Lあたり3〜4 gの 塩。
\item
  加熱時間\ldots{}\ldots{}30分。
\item
  作業手順\ldots{}\ldots{}鍋底に香味野菜を敷き詰め、魚のあらを入れる。水と白ワイ
  ンを注ぎ、強火にかける。丁寧にアクを引き、微沸騰の状態を保つようにする。
  30分煮たら目の細かい網で漉す。
\end{itemize}

\hypertarget{nota-fumet-de-poisson}{%
\subparagraph{【原注】}\label{nota-fumet-de-poisson}}

質の悪い白ワインを使うと灰色がかったフュメになってしまう。品質の疑わし
いワインは使わないほうがいい。

このフュメはソースを作る際に加える液体として用いる。魚料理用ソース・エ
スパニョルを作ることを想定する場合には、魚のあらをバターでエチュベして
から水と白ワインを注いで煮るといい。

\hypertarget{fonds-de-poisson-au-vin-rouge}{%
\subsubsection{赤ワインを用いた魚のフォン}\label{fonds-de-poisson-au-vin-rouge}}

\frsub{Fonds de poisson au vin rouge}

\index{ふおん@フォン!あかわいんをもちいたさかなのふおん@赤ワインを用いた魚の---}
\index{fonds@fonds!fonds de poisson au vin rouge@--- de poisson au vin rouge}

このフォンそれ自体を用意することは滅多にない。というのも、例えばマトロッ
トのような料理の魚の煮汁そのものだからだ。

とはいえ、こんにちでは魚のアラをすっかり取り除いた状態で料理を提供する
必要がますます高まってきているので、ここでそのレシピを記しておくべきだ
ろう。このフォンの必要性と有用さはどんどん高まっていくと思われる。

原則として、このフォンの仕込みには、料理として提供するのと同じ種類の魚
のアラを用いて、その香りの特徴を生かす必要がある。だが、どんな種類の魚
を使う場合でも作り方は同じだ。

(仕上がり5 L分)

\begin{itemize}
\item
  主素材\ldots{}\ldots{}料理に用いるのと同じ魚種の頭とアラ2.5 kg。
\item
  香味素材\ldots{}\ldots{}薄切りにして下茹でした玉ねぎ300
  g、パセリの枝100 g、タイ
  ムの小枝1本、小さめのローリエの葉2枚、にんにく5片、マッシュルームの切
  りくず100 g。
\item
  使用する液体と調味料\ldots{}\ldots{}水3.5 L、良質の赤ワイン2 L、塩15
  g。
\item
  加熱時間\ldots{}\ldots{}30分。
\item
  作業手順\ldots{}\ldots{}「魚の白いフォン\footnote{前項のフュメドポワソンのこと。}」と同様にする。
\end{itemize}

\hypertarget{nota-fonds-de-poisson-au-vin-rouge}{%
\subparagraph{【原注】}\label{nota-fonds-de-poisson-au-vin-rouge}}

このフォンは魚の白いフォンよりも濃く煮詰めることが可能。とはい
え、保存のために煮詰めないでいいように、その都度、必要な量だけ仕込むこ
とを勧める。

\hypertarget{essence-de-poisson}{%
\subsubsection{魚のエッセンス}\label{essence-de-poisson}}

\frsub{Essence de poisson}

\index{えつせんす@エッセンス!さかな@魚の---}
\index{essence@essence!poisson@--- de poisson}

\begin{itemize}
\item
  主素材\ldots{}\ldots{}メルラン\footnote{タラの近縁種。}および舌びらめの頭、アラ2
  kg。
\item
  香味素材\ldots{}\ldots{}薄切りにした玉ねぎ125
  g、マッシュルームの切りくず300 g、 パセリの枝50
  g、レモンの搾り汁1個分。
\item
  使用する液体\ldots{}\ldots{}煮詰めていないフュメドポワソン1\undemi{}
  L、良質の白ワイ ン3 dL。
\item
  所要時間\ldots{}\ldots{}45分。
\item
  作業手順\ldots{}\ldots{}鍋にバター100
  gと玉ねぎ、パセリの枝、マッシュルームの切
  りくずを入れ、強火で色づかないようさっと炒める。アラと端肉を加える。蓋をして約15分弱火で蒸
  し煮する\footnote{素材を入れた鍋に蓋をして弱火にかけ、少量の水分で蒸し煮状態にす
    ることを étuver エチュベという。このフランス語をそのまま用いている
    調理現場も少なくない。}。その間、小まめに混ぜてやること。白ワインを注ぎ、半量になるま
  で煮詰める。最後にフュメドポワソンを注ぎ、レモン汁と塩2 gを加える。
\end{itemize}

再び火にかけて、とろ火で15分程煮込んだら、布で漉す。

\hypertarget{nota-essence-de-poisson}{%
\subparagraph{【原注】}\label{nota-essence-de-poisson}}

魚のエッセンスは、舌びらめやチュルボ、チュルボタン、バルビュ\footnote{いずれも鰈、ひらめの近縁種。チュルボタンはチュルボの小さいもの
  を言う。} などのフィレ\footnote{3枚おろし、または5枚おろしにして、頭とアラを取り除いた状態。}をポシェする際に用いる。

さらに、このエッセンスを煮詰めて、上記でポシェした魚のソースに加えて風
味を強くするのに使う。

\hypertarget{essences-diverses}{%
\subsubsection{エッセンスについて}\label{essences-diverses}}

\frsub{Essences diverses}

\index{えつせんす@エッセンス!01えつせんすについて@---について(フォン)}
\index{essence@essence!01 diverses@---s diverses (fonds)}

その名のとおり、エッセンスとはごく少量になるまで煮詰めて非常に強い風味
を持たせたフォンのこと。

エッセンスは普通のフォンと本質的には同じものだが、素材の風味をしっかり
出すために、使用する液体の量はずっと少ない。したがって、仕上げにエッセ
ンスを加える指示がある料理の場合でも、そもそも充分に風味ゆたかなフォン
を用いていれば、エッセンスは必要ないことが分かるだろう。

まず最初に、美味しく風味ゆたかなフォンを用いるほうが、あまり出来のよく
ないフォンで調理し、後からエッセンスで欠点を補うよりもずっと簡単なのだ。
その方がいい結果が得られるし、時間と材料の節約にもなる。

セロリ、マッシュルーム、モリーユ\footnote{morille
  キノコの一種。和名アミガサタケ。}、トリュフなど、とりわけ明確な風
味の素材のエッセンスを、必要に応じて用いるにとどめるのがいい。

また、十中八九、フォンを仕込む際に素材そのものを加えた方が、エッセンス
を仕込むよりもいい結果が得られることは頭に入れておくこと。

そのようなわけで、エッセンスについてこれ以上長々と述べる必要もないと思
われる。ベースとなるフォンがコクと風味がゆたかなものならであるなら、エッ
センスはまったく無用の長物と言える。

\hypertarget{glaces-diverses}{%
\subsubsection{グラスについて}\label{glaces-diverses}}

\frsub{Glaces diverses}

\index{くらす@グラス!01くらすについて@---について}
\index{glace@glace!01 diverses@---s diverses}

グラスドヴィアンド、鶏のグラス(グラスドヴォライユ)、ジビエのグラス、
魚のグラスの用途は多岐にわたる。これらは、上記いずれかの素材でとったフォ
ンをシロップ状になるまで煮詰めたもののことだ。

これらの使い途は、料理の仕上げに表面に塗ってしっとりとした艶を出させる
のに用いる場合もあれば、ソースの味や色合いを濃くするために用いたり、あ
るいは、あまりに出来のよくないフォンで作った料理の場合にはコクを与える
ために使うこともある。また、料理によっては適量のバターやクリームを加え
てグラスそのものをソースとして用いることもある。

グラスとエッセンスの違いだが、エッセンスが料理の風味そのものを強くする
ことだけが目的であるのに対して、グラスは素材の持つコクと風味をごく少量
にまで濃縮したものだ。

だからほとんどの場合、エッセンスよりもグラスを使うほうがいい。

とはいえ昔の料理長たちの中には、グラスの使用を絶対に認めない者もいた。
その理由は、料理を作る度に毎回その料理のためのフォンをとるべきであり、
それだけで料理として充分なものにすべき、ということだった。

確かに時間と費用の点で制限がなければその理屈は正しい。だが、こんにちで
は、そのようなことの出来る調理現場はほとんどない。そもそもグラスは、正
しく適量を用いるのであれば、そのグラスが丁寧に作られたものであるならな、
素晴しい結果が得られる。 だから多くの場合、グラスはまことに有用なもの
と言える。

\hypertarget{glace-de-viande}{%
\subsubsection{グラスドヴィアンド}\label{glace-de-viande}}

\frsub{Glace de viande}

\index{くらす@グラス!くらすとういあんと@---ドヴィアンド}
\index{glace@glace!viande@--- de viande}

茶色いフォン(エストゥファード)を煮詰めて作る。

煮詰めて濃くなっていく途中、何度か布で漉して、より小さな鍋に移しかえて
いく。煮詰めている際に、丁寧にアクを引くことが、澄んだグラスを作るポイ
ント。

煮詰めている際には、フォンの濃縮具合に応じて、火加減を弱めていくこと。
最初は強火でいいが、作業の最後の方は弱火にしてゆっくり煮詰めてやること。

スプーンを入れてみて、引き上げた際に、艶のあるグラスの層でスプーンが覆
われ、しっかり張り付いているくらいが丁度いい。要するに、スプーンがグラ
スでコーティングされた状態になればいいということだ。

\hypertarget{nota-glace-de-viande}{%
\subparagraph{【原注】}\label{nota-glace-de-viande}}

色が薄くて軽い仕上がりのグラスが必要な場合には、茶色いフォンではなく、標
準的な仔牛のフォンを用いる。

\hypertarget{glace-de-volaille}{%
\subsubsection{鶏のグラス(グラスドヴォライユ)}\label{glace-de-volaille}}

\frsub{Glace de volaille}

\index{くらす@グラス!とり@鶏の---(---ドヴォライユ)}
\index{くらす@グラス!うおらいゆ@---ドヴォライユ}
\index{glace@glace!volaille@--- de volaille}

鶏のフォン(フォンドヴォライユ)を用いて、グラスドヴィアンドと同様にし
て作る。

\hypertarget{glace-de-gibier}{%
\subsubsection{ジビエのグラス}\label{glace-de-gibier}}

\frsub{Glace de gibier}

\index{くらす@グラス!しひえ@ジビエの---}
\index{glace@glace!glace de gibier@--- de gibier}
\index{しひえ@ジビエ!くらす@---のグラス}
\index{gibier@gibier!gibier@glace de ---}

ジビエのフォンを煮詰めて作る。ある特定のジビエの風味を生かしたグラスを
作るには、そのジビエだけでとったフォンを用いること。

\hypertarget{glace-de-poisson}{%
\subsubsection{魚のグラス}\label{glace-de-poisson}}

\frsub{Glace de poisson}

\index{くらす@グラス!さかな@魚の---}
\index{glace@glace!poisson@--- de poisson}

このグラスを用いることはあまり多くない。日常的な業務においては「魚のエッ
センス」を用いることが好まれる。そのほうが魚の風味も繊細になる。魚のエッ
センスで魚をポシェした後に煮詰めてソースに加える。
\end{recette}
\hypertarget{roux}{%
\section{ルー}\label{roux}}

\frsec{Roux}

\index{るー@ルー} \index{roux@roux}

ルーはいろいろな派生ソースのベースとなる基本ソースにとろみを付ける役目
を持つ。ルーの仕込みは、一見したところさほど重要に思われぬだろうが、実
際には正反対だ。丁寧に注意深く作業すること。

茶色いルーは加熱に時間がかかるので、大規模な調理現場では前もって仕込ん
でおく。ブロンドのルーと白いルーはその都度用意すればいい。
\begin{recette}
\hypertarget{roux-brun}{%
\subsubsection{茶色いルー}\label{roux-brun}}

\frsub{Roux brun}

\index{るー@ルー!ちやいろ@茶色い---} \index{roux@roux!brun@--- brun}

(仕上がり1 kg分)

\begin{enumerate}
\def\labelenumi{\arabic{enumi}.}
\tightlist
\item
  澄ましバター\ldots{}\ldots{}500 g
\item
  ふるった小麦粉\ldots{}\ldots{}600 g
\end{enumerate}

\hypertarget{cuisson-des-roux}{%
\subsubsection{ルーの火入れについて}\label{cuisson-des-roux}}

\index{るー@ルー!ひいれについて@---の火入れについて}
\index{roux@roux!cuisson@cuisson du ---}

加熱時間は使用する熱源の強さで変わってくる。だから数字で何分とは言えな
い。ただし、火力が強過ぎるよりは弱いくらいの方がいい。というのも、温度
が高すぎると小麦粉の細胞が硬化して中身を閉じ込めてしまい、そうなると後
でフォンなどの液体を加えた際に上手く混ざらず、滑らかなとろみの付いたソー
スにならない。乾燥豆をいきなり熱湯で茹でるのと同じようなことが起きるわ
けだ。低い温度から始めてだんだんと熱くしていけば、小麦粉の細胞壁がゆる
んで細胞中のでんぷんが膨張し、熱によって発酵状態の初期のようになる。こ
のようにして、でんぷんをデキストリンに変化させる\footnote{現代の科学的見地からすると必ずしも正確な記述ではないので注意。}。デキストリ
ンは水溶性の物質で、これが「とろみ」の主な要素なのだ。茶色いルーは淡褐
色の美しい色合いで滑らかな仕上がりにする。だまがあってはいけない。

ルーを作る際には必ず、澄ましバターを使うこと\footnote{初版〜第三版では「澄ましバターまたは充分に澄ましたグレスド
  マルミット」となっている。グレスドマルミットとは、コンソメなどを作
  る際に、浮いてくる油脂を取り除く必要があるが、それを捨てずにまとめ
  てから漉して澄ませたもののこと。基本的に獣脂と考えていい。なお、同
  時代の料理書 --- 例えばペラプラ『近代料理技術』(1935年)--- には、
  ルーを作るのにバターを使う必要はなく、グレスドマルミットで充分、と
  しているものもある。}。 生のバターに
は相当量のカゼインが含まれている。カゼインがあると火を均質に通すことが
出来なくなってしまう。とはいえ、以下を覚えておくといい。ソースとして仕
上げた段階で、ルーで使ったバターは風味という点ではほとんど意味が失なわ
れている。そもそもソースの仕上げに不純物を取り除く\footnote{dépouiller
  デプイエ。ソースや煮込み料理を仕上げる際に、浮
  き上がってくる不純物を徹底的に取り除き、目の細かい布などで漉すこと。
  原義は動物などの皮を剥ぐ、剥くことの意で、野うさぎの皮を剥ぐ、うな
  ぎの皮を剥く、という意味で現代の厨房でも用いられているる。ソースの
  場合は表面に凝固した蛋白質や油脂の膜が出来、それを「剥ぐように」取
  り除くことから、あるいは表面に浮いてくる不純物を徹底的に取り除いて
  きれいなソースに仕上げることを、動物の皮を剥いてきれいな身だけにす
  ることになぞらえて、この用語が用いられるようになったようだ。なお、
  本書においてécumer(エキュメ)が単に浮いてくる泡やアクを取る、とい
  う作業であるのに対して、dépouiller(デプイエ)は「徹底的に不純物を
  取り除いて美しく仕上げる」という意味合いが込められている。現代では
  品種改良や農法の変化によって野菜のアクも少なくなり、小麦粉も精製度
  の高いものを利用出来るなど、食材および調味料の多くで純度の高いもの
  を使用する場合がほとんどであり、このデプイエという作業は20世紀後半
  にはほとんど行なわれなくなり、écumer(エキュメ)という用語だけで済ませ
  ることがほとんど(cf.辻静雄監訳『オリヴェ ソースの本』柴田書店、
  1970年、27〜28頁)。}段階でバター
も完全に取り除かれてしまうわけだ。だからルーに用いるバターは小麦粉に熱
を通すためだけのものと考えていい。

ルーはソース作りの出発点だ。だから次の点も記憶に\ruby{留}{とど}めるこ
と。小麦粉にでんぷんが含まれているからこそソースに「とろみ」が付く。だ
から純粋なでんぷん(特性が小麦のでんぷんと同じでも異なったものでも)で
ルーを作っても、小麦粉の場合と同様の結果が得られるだろう。ただしその場
合は小麦粉でルーを作る場合より注意して作業する必要がある。また、小麦粉
と違って余計な物質が含まれていないために、全体の分量比率を考え直すこと
になる。

\hypertarget{nota-roux}{%
\subparagraph{【原注】}\label{nota-roux}}

本文で述べたように、茶色いルーを作る際には澄ましバターを用いる。他の動
物性油脂はよほど経済的事情が逼迫していない限り使わないこと。材料コスト
が問題になる場合でも、ソースの仕上げに不純物を取り除く際に多少の注意を
払えば、ルーに用いたバターを回収するのはさして難しいことではない\footnote{既に述べたように初版〜第三版まではバターまたはグレスドマル
  ミットという指示であったことに留意する必要はあるだろう。実際のとこ
  ろ、良質のバターを用いてルーを作ったほうが、軽やかな仕上りのソース
  になる傾向があることは言うまでもない。}。それ
を後で他の用途で使えばいいだろう。

\hypertarget{roux-blond}{%
\subsubsection{ブロンドのルー}\label{roux-blond}}

\frsub{Roux blond}

\index{るー@ルー!ふろんと@ブロンドの---}
\index{roux@roux!blond@--- blond}

(仕上がり1 kg分)

材料の比率は茶色いルーと同じ。すなわちバター500 gと、ふるった小麦粉600
g。

火入れは、ルーがほんのりブロンド色になるまで、ごく弱火で行なう。

\hypertarget{roux-blanc}{%
\subsubsection{白いルー}\label{roux-blanc}}

\frsub{Roux blanc}

\index{るー@ルー!しろい@白い---} \index{roux@roux!blanc@--- blanc}

500 gのバターと、ふるった小麦粉600 g。

このルーの火入れは数分、つまり粉っぽさがなくなるまでの時間でいい。
\end{recette}\newpage
\hypertarget{ux57faux672cux30bdux30fcux30b9}{%
\section{基本ソース}\label{ux57faux672cux30bdux30fcux30b9}}

\hypertarget{grandes-sauces-de-base}{%
\subsection{Grandes Sauces de Base}\label{grandes-sauces-de-base}}

\index{そーす@ソース!きほん@基本---}
\index{sauce@sauce!00grandes@*Grandes ---s de Base}
\begin{recette}
\hypertarget{ux30bdux30fcux30b9ux30a8ux30b9ux30d1ux30cbux30e7ux30eb102008}{%
\subsubsection[ソース・エスパニョル]{\texorpdfstring{ソース・エスパニョル\footnote{本節冒頭では、ルーがスペインの料理人によってもたらされ、そ
  の結果としてソース・エスパニョルが作られるようになったと読める記述
  があるが、これはむしろ誤りと考えるべき。エスパニョル espagnol(e)は
  「スペイン(風)の」意だが、スペイン料理起源というわけでもない。ス
  ペインを想起させるトマトを使うから、あるいは、ソースが茶褐色なのが
  ムーア系スペイン人を想起させるから、など定説はない。カレーム『19世
  紀フランス料理』第3巻に収められたソース・エスパニョルの作り方は、
  フォンをとるところから始まり4ページにわたって詳細なものとなってい
  る(pp.8-11)。その中で、肉を入れた鍋に少量のブイヨンを注いで煮詰
  めることを繰り返す。ここまでは18世紀の料理書で一般的な手法であるが、
  その後に大量のブイヨンを注いだ後、いきなり強火にかけるのではなく、
  弱火で加熱していくやり方を「スペイン式の方法」と述べている。カレー
  ムにおいては、これがソースの名称の根拠のひとつになっていると考えて
  いいだろう。もちろん、ソース・エスパニョルという名称のソースはカレー
  ム以前からあり、1806年刊のヴィアール『帝国料理の本』にもカレームの
  レシピより簡単だが、ほぼ同様のものが基本ソースとして収録されている。
  また、それ以前にもソース・エスパニョルに類する名称のソースはあった
  が、たとえば1739年刊ムノン『新料理研究』第2巻にある「スペイン風ソー
  ス」はかなり趣きが異なる(コリアンダーひと把みを加えるのが特徴的)。
  同じ料理名でも時代や料理書の著者によってまったく違う料理になってい
  ることは、食文化史において珍しいことではない。また、とりわけ料理名
  に地名、国名が冠されているものの中には根拠や由来のはっきりしないも
  のも多い。いずれにしても、本書のソース・エスパニョルの源流は19世紀
  初頭のヴィアールあたりからと考えられる。ソース・エスパニョルは19世
  紀を通して普及し、茶色いソースの代表的な存在となった。こんにちでも
  フォンドヴォーをベースとしたソースは、ルーでとろみ付けこそしないが、
  仔牛の骨などから出るコラーゲンによるとろみを利用したもので、仕上が
  りの色合いや、ごく標準的ともいえる風味付けの方法などが引継がれ続け
  ている調理現場も少なくない。もっとも、上述のように本書では「茶色い
  ルー」を使うところに「エスパニョル」であることの理由を見い出そうと
  していると解釈される。}}{ソース・エスパニョル}}\label{ux30bdux30fcux30b9ux30a8ux30b9ux30d1ux30cbux30e7ux30eb102008}}

\hypertarget{sauce-espagnole}{%
\paragraph{Sauce espagnole}\label{sauce-espagnole}}

\index{そーす@ソース!えすはによる@---・エスパニョル}
\index{そーす@ソース!きほん@基本---!えすはによる@---・エスパニョル}
\index{きほんそーす@基本ソース!えすはによる@---・エスパニョル}
\index{えすはによる@エスパニョル!そーす@ソース・---}
\index{すぺいんふう@スペイン風(エスパニョル)!そーすえすはによる@ソース・エスパニョル}
\index{sauce@sauce!00grandes@*Grandes ---s de Base!espagnole@--- Espagnole}
\index{sauce@sauce!espagnole@--- Espagnole}
\index{espagnol@espagnol!sauce@Sauce ---e}

(仕上がり5 L分)

\begin{itemize}
\item
  \textbf{とろみ付けのためのルー}\ldots{}\ldots{}625 g。
\item
  \textbf{茶色いフォン(ソースを仕上げるのに必要な全量)}\ldots{}\ldots{}12
  L。
\item
  \textbf{\protect\hyperlink{}{ミルポワ}\footnote{mirepoix
    (ミルポワ)。ソースやフォンにコクを与える目的で、
    細かいさいの目に切った香味野菜や塩漬け豚ばら肉を合わせたもの。18世
    紀にミルポワ公爵の料理人が考案したという説が有力。同様のものに
    matignon(マティニョン)があるが、ミルポワより大きめのさいの目に切
    るのが一般的とされるが、調理現場によってはあまり区別せずミルポワと
    のみ呼称するケースも多い。}(香味素材)}\ldots{}\ldots{}小さなさいの目に切った塩漬け豚
  ばら肉150 g、2 mm程度のさいの目\footnote{brunoise
    (ブリュノワーズ)。1〜2 mm のさいの目に切ること。 couper en
    mirepoix(クゥペオンミルポワ)ミルポワに切るとも言う。}に切ったにんじん250
  gと玉ねぎ 150
  g、タイム2枝、ローリエの葉2枚。\index{みるぽわ@ミルポワ}\index{mirepoix}
\item
  \textbf{作業手順}
\end{itemize}

\begin{enumerate}
\def\labelenumi{\arabic{enumi}.}
\item
  フォン8 Lを鍋で沸かす。あらかじめ柔らかくしておいたルーを加え、木杓
  子か泡立て器で混ぜながら沸騰させる。

  弱火にして\footnote{原文から直訳すると「鍋を火の脇に置く」だが、現代の調理環境
    では単純に「弱火にする」と解釈していい。}微沸騰の状態を保つ。
\item
  以下のようにしてあらかじめ用意しておいたミルポワを投入する。ソテー
  鍋に塩漬け豚ばら肉を入れて火にかけて脂を溶かす。そこに、細かく刻ん
  だにんじんと玉ねぎ、タイム、ローリエの葉を加える。野菜が軽く色づく
  まで強火で炒める。丁寧に、余分な脂を捨てる。これをソースに加える。
  野菜を炒めたソテー鍋に白ワイン約100 ml\footnote{原文 un verre de vin
    blanc (アンヴェールドヴァンブロン)。
    直訳すると「グラス1杯の白ワイン」だが、本書において un verre de 〜
    は「約1dl=100ml」と覚えておくといいだろう。}を加えてデグラセ\footnote{dégrasser
    鍋に粘液状になって付着している肉汁を酒類あるいは水で溶かし出してソースなどに利用すること。}し、それを半量
  まで煮詰める。これも同様にソースの鍋に加える。こまめに浮いてくる夾
  雑物を徹底的に取り除き\footnote{dépouiller
    デプイエ。前節「ルーの火入れについて」訳注参照。}ながら弱火で約1時間煮込む。
\item
  ソースをシノワ\footnote{小さな穴が多く空けられた円錐形で、取っ手の付いた漉し器の一
    種。金属製のものが主流。}で、ミルポワ野菜を軽く押しながら漉し、別の
  片手鍋に移す。フォン2 Lを注ぎ足す。さらに2時間、微沸騰の状態を保ち
  ならが煮込む。その後、陶製の鍋に移し、ゆっくり混ぜながら冷ます。
\item
  翌日、再び厚手の片手鍋に移してから、フォン2 Lとトマトピュレ1 Lまた
  は同等の生のトマトつまり2 kgを加える。\\
  トマトピュレを用いる場合は、あらかじめオーブンでほとんど茶色になる
  まで焼いておくといい。そうするとトマトピュレの酸味を抜くことが出来
  る。\\
  そうすればソースを澄ませる作業が楽になるし、ソースの色合いも温かそ
  うで美しいものになる。\\
  ソースをヘラか泡立て器で混ぜながら強火で沸騰させる。弱火にして1時間
  微沸騰の状態を保つ。最後に、表面に浮いている不純物を、細心の注意を
  払いながら徹底的に取り除く。布で漉し、完全に冷めるまで、ゆっくり混
  ぜ続けること。
\end{enumerate}

\hypertarget{ux539fux6ce8}{%
\subparagraph{【原注】}\label{ux539fux6ce8}}

ソース・エスパニョルで仕上げに不純物を取り除くのにかかる時間はいちがい
には言えない。これは、ソースに用いるフォンの質次第で変わるからだ。

ソースにするフォンが上質なものであればある程、仕上げに不純物を取り除く
作業は早く済む。そういう場合には、ソース・エスパニョルを5時間で作るこ
とも無理ではない。

\maeaki

\hypertarget{ux9b5aux6599ux7406ux75280102006ux30bdux30fcux30b9ux30a8ux30b9ux30d1ux30cbux30e7ux30eb}{%
\subsubsection[魚料理用ソース・エスパニョル]{\texorpdfstring{魚料理用\footnote{フランス語のソース名にあるmaigre(メーグル)はこの場合、一般的に「魚用、
  魚料理用」と訳すが、厳密には「小斉の際の料理用」となろう。小斉とは、
  カトリックで古くから特定の期間、曜日に肉類を断つ食事をする宗教的食
  習慣。日本の「お精進」とニュアンスは近いが、小斉においては忌避され
  るのは鳥獣肉のみであり、魚介や乳製品はいいとされた。こじつけのよう
  に、水鳥は水のものだから魚介扱いであり、またイルカも魚類として扱わ
  れていた。小斉が行なわれるのは復活祭の前46日間(四旬節、逆に言えば
  カーニバルの最終日マルディグラの翌日から46日)と、週に一度(多くの
  場合は金曜)であった。合計すると小斉が行なわれるのは年間100日近く
  もあり、中世から18世紀の料理人たちは小斉の宴席に供する料理に工夫を
  凝らしていた。この習慣は19世紀になるとだんだん廃れていき、エスコフィ
  エの時代には、料理人に対して小斉のための料理を要求することは少なく
  なっていった。}ソース・エスパニョル}{魚料理用ソース・エスパニョル}}\label{ux9b5aux6599ux7406ux75280102006ux30bdux30fcux30b9ux30a8ux30b9ux30d1ux30cbux30e7ux30eb}}

\hypertarget{sauce-espagnole-maigre}{%
\paragraph{Sauce espagnole maigre}\label{sauce-espagnole-maigre}}

\index{そーす@ソース!えすはによるるさかな@---・エスパニョル (魚料理用)}
\index{そーす@ソース!きほん@基本---!えすはによるさかな@魚料理用---・エスパニョル}
\index{きほんそーす@基本ソース!えすはによるさかな@魚料理用---・エスパニョル}
\index{えすはによる@エスパニョル!そーすさかなよう@ソース・--- (魚料理用)}
\index{すへいんふう@スペイン風(エスパニョル)!そーすえすはによるさかな@ソース・エスパニョル(魚料理用)}
\index{sauce@sauce!00grandes@*Grandes ---s de Base!espagnole maigre@--- Espagnole maigre}
\index{sauce@sauce!espagnole maigre@--- Espagnole maigre}
\index{espagnol@espagnol!sauce maigre@Sauce Espagnole maigre}

(仕上がり5 L分)

\begin{itemize}
\item
  \textbf{バターを用いて\footnote{初版〜第三版にかけては、茶色いルーを作るのに「バターまたは、
    きれいなグレスドマルミット(コンソメなどを作る際に表面に浮いてくる
    脂をすくい取って、不純物を漉し取ったものであり、基本的に獣脂)」を
    用いる、とある。上述のように、カトリックにおける「小斉」の場合、獣
    脂は忌避されたがバターなどの乳製品は許容された。そのため特に「バター
    を用いて作ったルー」という指定がなされ、第四版では茶色いルーに澄ま
    しバターのみを使う旨が強調されたが、ここでは初版以来の記述がそのま
    ま残っているために、やや冗長に思われる表現となっている。}作ったルー}\ldots{}\ldots{}500
  g。
\item
  \textbf{魚のフュメ(フュメドポワソン)(ソースを仕上げるために必要な全量)
  }\ldots{}\ldots{}10 L。
\item
  \textbf{ミルポワ}\ldots{}\ldots{}標準的なソース・エスパニョルと同じ\protect\hyperlink{mirepoix}{ミルポワ}野菜を同
  量と、塩漬け豚ばら肉の代わりにバターを用い、マッシュルームまたはマッシュ
  ルームの切りくず\footnote{champignons de Paris
    (シャンピニョン ドパリ)いわゆるマッ
    シュルームは、料理の一部として提供する際にはトゥルネ tourner といっ
    て\{螺旋\}\{らせん\}状の切れ込みを入れて装飾したものを使う。その際に
    少なくない量の切りくずが発生するのでそれを利用する。なお、tourner
    (トゥルネ)の原義は「回す」であり、包丁を持った側の手は動かさずに、
    材料のほうを回すようにして切れ目を入れたり、アーティチョークや果物
    などの皮を剥くことを意味する。}250 gを加える。
\item
  \textbf{作業手順}\ldots{}\ldots{}標準的なソース・エスパニョルとまったく同様に作る。
\item
  \textbf{加熱時間と不純物を取り除くのに必要な時間}\ldots{}\ldots{}5時間。
\end{itemize}

仕上げに漉してから、標準的なソース・エスパニョルとまったく同様に、完全
に冷めるまでゆっくり混ぜ続けること。

\maeaki

\hypertarget{ux9b5aux6599ux7406ux7528ux30bdux30fcux30b9ux30a8ux30b9ux30d1ux30cbux30e7ux30ebux88dcux8db3}{%
\subparagraph{魚料理用ソース・エスパニョル補足}\label{ux9b5aux6599ux7406ux7528ux30bdux30fcux30b9ux30a8ux30b9ux30d1ux30cbux30e7ux30ebux88dcux8db3}}

このソースを日常的な料理のベースとなる仕込みに含めるかどうかについては
意見が分れるところだ。

普通のソース・エスパニョルは、つまるところ風味の点ではほとんどニュート
ラルなものだから、それに魚のフュメを加えれば、魚料理用ソース・エスパニョ
ルとして充分に通用するだろう。どうしても上で挙げた魚料理用ソース・エス
パニョルが必要になるのは、宗教的に厳格に小斉の決まりを守って料理を作る
場合のみで、さすがにその場合は代用品などない。

\maeaki

\hypertarget{ux30bdux30fcux30b9ux30c9ux30a5ux30dfux30b0ux30e9ux30b9102009}{%
\subsubsection[ソース・ドゥミグラス]{\texorpdfstring{ソース・ドゥミグラス\footnote{日本の洋食などで一般的な「デミグラス」あるいは「ドミグラス」」
  とはかなり異なった仕上りのソースであることに注意。ソース・エスパニョ
  ルの仕上げにあたって、徹底的に不純物を取り除くことを何度も強調して
  いるのは、透き通った茶色がかった色合いの、なめらかなソースを目指す
  からであり、それをさらに徹底させるということは、透明度、なめらかさ
  の面でさらに上を目指すということを意味するからだ。ちなみに、アメリ
  カに本社のあるメーカーの「デミグラスソース」の缶詰はもっぱら日本で販売
  されている製品であり、ヨーロッパおよびアメリカでは同一ブランドに該
  当する商品は存在しないようだ。}}{ソース・ドゥミグラス}}\label{ux30bdux30fcux30b9ux30c9ux30a5ux30dfux30b0ux30e9ux30b9102009}}

\hypertarget{sauce-demi-glace}{%
\paragraph{Sauce demi-glace}\label{sauce-demi-glace}}

\index{そーす@ソース!とうみくらす@---・ドゥミグラス}
\index{そーす@ソース!きほん@基本---!とうみくらす@---・ドゥミグラス}
\index{きほんそーす@基本ソース!とうみくらす@---・ドゥミグラス}
\index{sauce@sauce!00grandes@*Grandes ---s de Base!demi-glace@--- Demi-glace}
\index{sauce@sauce!demi-glace@--- Demi-glace}

一般に「ドゥミグラス」と呼ばれているものは、いったん仕上がったソース・
エスパニョルをさらに、もうこれ以上は無理という位に徹底的に不純物を取り
除いたもののことだ。

最後の仕上げに\protect\hyperlink{glace-de-viande}{グラスドヴィアンド}などを加える。風味付けに何らかの酒類\footnote{本書ではマデイラワイン(ポルトガルの酒精強化ワイン、すなわ
  ちブドウ果汁が酵母により醗酵している途中で蒸留酒を加えて醗酵を止め
  る製法のもので、甘口のものが多い)が用いられることが多い。}
を加えれば、当然ながらソースの性格も変わるので、最終的な使い途に応じて
決めること。

\hypertarget{ux539fux6ce8-1}{%
\subparagraph{【原注】}\label{ux539fux6ce8-1}}

ソースの色合いを決めるワインを仕上げに加える際には、「火から外して」行
なうこと。沸騰しているとワインの香りがとんでしまうからだ。

\maeaki

\hypertarget{ux3068ux308dux307fux3092ux4ed8ux3051ux305fux4ed4ux725bux306eux30b8ux30e5}{%
\subsubsection{とろみを付けた仔牛のジュ}\label{ux3068ux308dux307fux3092ux4ed8ux3051ux305fux4ed4ux725bux306eux30b8ux30e5}}

\hypertarget{jus-de-veau-lie}{%
\paragraph{Jus de veau lié}\label{jus-de-veau-lie}}

\index{そーす@ソース!きほん@基本---!とろみをつけたこうしのしゆ@とろみを付けた仔牛のジュ}
\index{きほんそーす@基本ソース!とろみをつけたこうしのしゆ@とろみを付けた仔牛のジュ}
\index{しゆ@ジュ!こうしのしゆ@仔牛の---(とろみを付けた)}
\index{そーす@ソース!とろみをつけたこうしのしゆ@とろみを付けた仔牛のジュ}
\index{こうし@仔牛!とろみをつけたこうしのしゆ@とろみを付けた---のジュ}
\index{sauce@sauce!00grandes@*Grandes ---s de Base!jus veau lie@--- de veau lié}
\index{jus@jus!jus veau lie@--- de veau lié}
\index{veau@veau!jus lie@jus de --- lié}

(仕上がり1 L分)

\begin{itemize}
\item
  \textbf{仔牛のフォン}\ldots{}\ldots{}仔牛の茶色いフォン4 L。
\item
  \textbf{とろみ付け材料}\ldots{}\ldots{}アロールート\footnote{allow-root
    南米産のクズウコンを原料とした良質のでんぷん。日
    本では入手が難しいこともあり、コーンスターチが用いられることがほとんど。}30
  g。
\item
  \textbf{作業手順}\ldots{}\ldots{}よく澄んだ仔牛のフォン4
  Lを強火にかけ、\unquart{}量つ まり1 Lになるまで煮詰める。
\end{itemize}

大さじ数杯分の冷たいフォンでアロールートを溶く。これを沸騰している鍋に
加える。1分程度だけ火にかけ続けたら、布で漉す。

\hypertarget{ux539fux6ce8-2}{%
\subparagraph{【原注】}\label{ux539fux6ce8-2}}

この、とろみを付けた仔牛のジュは、本書では頻繁に使う指示をしているが、
必ず、しっかりした味で透き通った、きれいな薄茶色に仕上げること。

\maeaki

\hypertarget{ux30f4ux30ebux30c6102013ux6a19ux6e96ux7684ux306aux767dux3044ux30bdux30fcux30b9}{%
\subsubsection[ヴルテ(標準的な白いソース)]{\texorpdfstring{ヴルテ\footnote{velouté
  (ヴルテ)原義は「ビロードのように柔らかな、なめら
  かな」。日本ではベシャメルソースと混同されやすいが、内容がまったく
  異なるソースなので注意。}(標準的な白いソース)}{ヴルテ(標準的な白いソース)}}\label{ux30f4ux30ebux30c6102013ux6a19ux6e96ux7684ux306aux767dux3044ux30bdux30fcux30b9}}

\hypertarget{veloute}{%
\paragraph{Velouté, ou sauce blanche graisse}\label{veloute}}

\index{そーす@ソース!きほん@基本---!うるて@ヴルテ(標準的な)}
\index{きほんそーす@基本ソース!うるて@ヴルテ(標準的な)}
\index{うるて@ヴルテ!ひようひゆんてきなそーすうるて@標準的なソース・---}
\index{そーす@ソース!うるてひようひゆん@ヴルテ(標準的な)}
\index{ふるーて@ブルーテ ⇒ ヴルテ} \index{veloute@velouté}
\index{veloute@velouté!sauce blanche grasse@sauce blanche grasse}
\index{sauce@sauce!00grandes@*Grandes ---s de Base!veloute@Velouté}

(仕上がり5 L分)

\begin{itemize}
\item
  \textbf{とろみ付けの材料}\ldots{}\ldots{}バターを用いて作った\footnote{\protect\hyperlink{sauce-espagnole-maigre}{魚料理用ソース・エスパニョル}、訳
    注参照。}ブロンドのルー 625 g。
\item
  \textbf{よく澄んだ仔牛の白いフォン}\ldots{}\ldots{}5 L。
\item
  \textbf{作業手順}\ldots{}\ldots{}ルーをフォンに溶かし込む。フォンは冷たくても熱くても
  いいが、フォンが熱い場合にはソースが充分なめらかになるよう注意して溶か
  すこと。混ぜながら沸騰させる。微沸騰の状態を保ちながら、浮いてくる不純
  物を完全に取り除いていく\footnote{dépouiller
    (デプイエ)。\protect\hyperlink{sauce-espagnole}{ソース・エスパニョル}、
    訳注参照。}。この作業はとりわけ細心の注意を払っ て行なうこと。
\item
  \textbf{加熱時間と不純物を取り除く作業に必要な時間}\ldots{}\ldots{}1時間半。
\end{itemize}

その後、ヴルテを布で漉す\footnote{ある程度濃度のある液体やピュレを布で漉す場合、昔は「二人が
  かりで行なう必要があり、それぞれが巻いた布の端を左手に持ち、右手に
  持った木杓子を使って圧し搾る」(『ラルース・ガストロノミーク』初版、
  1938年)という方法が一般的だった。}。陶製の鍋に移してゆっくり混ぜながら
完全に冷ます。

\maeaki

\hypertarget{ux9d8fux306eux30f4ux30ebux30c6}{%
\subsubsection{鶏のヴルテ}\label{ux9d8fux306eux30f4ux30ebux30c6}}

\hypertarget{veloute-de-volaille}{%
\paragraph{Velouté de volaille}\label{veloute-de-volaille}}

\index{きほんそーす@基本ソース!とりのうるて@鶏のヴルテ}
\index{そーす@ソース!きほん@基本---!とりのうるて@鶏のヴルテ}
\index{うるて@ヴルテ!とりのうるて@鶏の---(ヴルテドヴォライユ)}
\index{そーす@ソース!うるてとり@ヴルテ(鶏)}
\index{ぶるーて@ブルーテ ⇒ ヴルテ}
\index{うおらいゆ@ヴォライユ!うるてとうおらいゆ@ヴルテドヴォライユ(鶏
のヴルテ)} \index{かきん@家禽!とりのうるて@鶏のヴルテ}
\index{veloute@velouté!volaille@--- de Volaille}
\index{sauce@sauce!veloute volaille@Velout\'e de Volaille}

このヴルテの作り方だが、上述の標準的なヴルテと、材料比率と作業はまった
く同じ。使用する液体として鶏の白いフォン(フォンドヴォライユ)を使う。

\maeaki

\hypertarget{ux9b5aux6599ux7406ux7528ux30f4ux30ebux30c6}{%
\subsubsection{魚料理用ヴルテ}\label{ux9b5aux6599ux7406ux7528ux30f4ux30ebux30c6}}

\hypertarget{veloute-de-poisson}{%
\paragraph{Velouté de poisson}\label{veloute-de-poisson}}

\index{そーす@ソース!きほん@基本---!さかなりようりよううるて@魚料理用ヴルテ}
\index{きほんそーす@基本ソース!さかなりようりよううるて@魚料理用ヴルテ}
\index{うるて@ヴルテ!さかなうるて@魚料理用---} \index{そーす@ソース!う
るてさかな@ヴルテ(魚料理用)} \index{veloute@velouté!poisson@--- de
Poisson} \index{sauce@sauce!veloute poisson@Velouté de Poisson}

ルーと液体の分量は標準的なヴルテとまったく同じだが、仔牛のフォンではな
く魚のフュメを用いて作る。

ただし、魚を素材として用いるストックはどれもそうだが、手早く作業するこ
と。不純物を取り除く作業も20分程度にとどめること。その後、布で漉し、陶
製の鍋に移してゆっくり混ぜながら完全に冷ます。

\maeaki

\hypertarget{ux30bdux30fcux30b9ux30a2ux30ebux30deux30f3ux30c9ux30d1ux30eaux98a8ux30bdux30fcux30b9102024}{%
\subsubsection[ソース・アルマンド(パリ風ソース)]{\texorpdfstring{ソース・アルマンド(パリ風ソース\footnote{原書では「パリ風ソース(元ソース・アルマンド)」となってい
  るが、後述のように、こんにちでもソース・アルマンドの名称のほうが一
  般的であるため、ここではSauce Parisienneの「訳語」としてソース・ア
  ルマンドをあてることとした。})}{ソース・アルマンド(パリ風ソース)}}\label{ux30bdux30fcux30b9ux30a2ux30ebux30deux30f3ux30c9ux30d1ux30eaux98a8ux30bdux30fcux30b9102024}}

\hypertarget{sauce-allemande}{%
\paragraph{Sauce parisienne (ex-Allemande)}\label{sauce-allemande}}

\index{そーす@ソース!きほん@基本---!あるまんと@---・アルマンド}
\index{きほんそーす@基本ソース!あるまんと@---・アルマンド}
\index{そーす@ソース!ぱりふう@パリ風--- ⇒ ---・アルマンド}
\index{はりふう@パリ風!そーす@---ソース ⇒ ---・アルマンド}
\index{といつふう@ドイツ風!そーす@ソース・アルマンド(ドイツ風ソース)}
\index{あるまん@アルマン(ド)!そーす@ソース・アルマンド}
\index{sauce@sauce!00grandes@*Grandes ---s de Base!--- Allemande}
\index{sauce@sauce!parisienne@--- parisienne (ex-Allemande)}
\index{parisien@parisien!sauce@Sauce Parisienne = Sauce Allemande}
\index{allemand@allemand!sauce@Sauce allemande (--- Parisienne)}

(仕上がり1 L分)

標準的なヴルテに卵黄でとろみを付けたソース。

\begin{itemize}
\item
  \textbf{標準的なヴルテ}\ldots{}\ldots{}1 L。
\item
  \textbf{追加素材}\ldots{}\ldots{}卵黄5個、白いフォン(冷たいもの)\undemi{}
  L、粗く砕
  いたこしょう1ひとつまみ、すりおろしたナツメグ少々、マッシュルームの煮
  汁2 dl、レモン汁少々。
\item
  \textbf{作業手順}\ldots{}\ldots{}厚手のソテー鍋にマッシュルームの茹で汁と白いフォン、卵
  黄、粗く砕いたこしょう、ナツメグ、レモン汁を入れる。泡立て器でよく混ぜ、
  そこにヴルテを加える。火にかけて沸騰させ、強火で\deuxtiers{}量になるまで、ヘラ
  で混ぜながら煮詰める。
\end{itemize}

ヘラの表面がソースでコーティングされる状態になるまで煮詰めたら、布で漉
す。

膜が張らないよう、表面にバターのかけらをいくつか載せてやり、湯煎にかけ
ておく。

\begin{itemize}
\tightlist
\item
  \textbf{仕上げ}\ldots{}\ldots{}提供直前に、バター100
  gを加えて仕上げる。
\end{itemize}

\hypertarget{ux539fux6ce8-3}{%
\subparagraph{【原注】}\label{ux539fux6ce8-3}}

ソース・アルマンド(ドイツ風)とも呼ばれるが、本書では「パリ風」の名称
を採用した。そもそも「アルマンド」というの名称に正当性がないからだ。習
慣としてそう呼ばれてきただけであって、明らかに理屈に合わない名称だ
\footnote{エスコフィエは普仏戦争に従軍した経歴があり、ドイツ嫌いとし
  て知られていた。}。1883年に雑誌「料理技術」に某タヴェルネ氏が寄せた記事
には、当時ある優秀な料理人がアルマンドなどという理屈に合わない名称を使
うのはやめたという話が出ている。

こんにち既に「パリ風ソース」の名称を採用している料理長もいる。そう呼ん
だほうが好ましいわけだが、残念なことにまだ一般的にはなっていない
\footnote{エスコフィエの願いもむなしく、現代においてもソース・アルマ
  ンドの名称で定着している。この「全注解」においても以後は「ソース・
  アルマンド」と訳しているので注意されたい。なお、「ドイツ風」とい
  うソース名の由来について、ソースの淡い黄色がドイツ人に多い金髪を
  想起させるからだとカレームは述べている。}。

\maeaki

\hypertarget{ux30bdux30fcux30b9ux30b7ux30e5ux30d7ux30ecux30fcux30e0102023}{%
\subsubsection[ソース・シュプレーム]{\texorpdfstring{ソース・シュプレーム\footnote{suprême
  原義は「至高の」だが、料理においてはしばしば鶏や鴨
  の胸肉、白身魚のフィレなどを意味する。また、このソースのように、と
  くに意味もなくこの名を料理につけられているケースも多い。}}{ソース・シュプレーム}}\label{ux30bdux30fcux30b9ux30b7ux30e5ux30d7ux30ecux30fcux30e0102023}}

\hypertarget{sauce-supreme}{%
\paragraph{Sauce supême}\label{sauce-supreme}}

\index{きほんそーす@基本ソース!しゆふれーむ@---・シュプレーム}
\index{そーす@ソース!きほん@基本---!しゆふれーむ@---・シュプレーム}
\index{そーす@ソース!そーすしゆふれーむ@---・シュプレーム}
\index{しゆふれーむ@シュプレーム!そーす@ソース・---}
\index{sauce@sauce!00grandes@*Grandes ---s de Base!supreme@--- Suprême}
\index{sauce@sauce!supreme@--- Suprême}
\index{supreme@suprême!sauce@Sauce ---}

\protect\hyperlink{veloute-de-volaille}{鶏のヴルテ}に生クリーム\footnote{フランスの生クリームのうち、料理でよく使われるのは、日本の
  生クリームにやや近い「クレーム・フレッシュ・パストゥリゼ」(低温殺
  菌した生クリームで乳脂肪分30〜38%)のほか、「クレーム・フレッシュ・
  エペス」(低温殺菌後に乳酸醗酵させたもので日本で一般的な生クリーム
  より濃度がある)、「クレーム・ドゥーブル」(殺菌後に乳酸醗酵させた
  もので乳脂肪分40%程度でかなり濃度がある)などがある。}を加えてなめら
かに仕上げ\footnote{monter
  モンテ。原義は「上げる、ホイップする」だが、ソースの
  仕上げの際などに、バターや生クリームを加えて、なめらかに仕上げるこ
  とも「モンテ」の語を使用する場合が多い。}たもの。ソース・シュプレームは、正しく作った場合
「白さの\ruby{際}{きわ}だったとても繊細な」仕上がりのものでなくてはい
けない。

(仕上がり1 L分)

\begin{itemize}
\item
  \textbf{鶏のヴルテ}\ldots{}\ldots{}1 L。
\item
  \textbf{追加素材}\ldots{}\ldots{}鶏の白いフォン1
  L、マッシュルームの茹で汁1dl、良質な生 クリーム2 \undemi{} dl。
\item
  \textbf{作業手順}\ldots{}\ldots{}鍋に鶏のフォンとマッシュルームの茹で汁、鶏のヴルテを入
  れて強火にかけ、ヘラで混ぜながら、生クリームを少しずつ加え、煮詰めてい
  く。このヴルテと生クリームを煮詰めたものの分量は、上で示した仕上がり1L
  のソース・シュプレームを作るには、\untiers{}量まで煮詰まっていなくては
  ならない。
\end{itemize}

布で漉し、仕上げに1 dlの生クリームとバター80 gを加えてゆっくり混ぜなが
ら冷ますと、丁度最初のヴルテと同量になる。

\maeaki

\hypertarget{ux30d9ux30b7ux30e3ux30e1ux30ebux30bdux30fcux30b9102020}{%
\subsubsection[ベシャメルソース]{\texorpdfstring{ベシャメルソース\footnote{17世紀にルイ14世のメートルドテルを務めたこともあるルイ・ベ
  シャメイユLouis Béchameil(1630〜1703)の名が冠されているこのソー
  スは、彼自身の創案あるいは彼に仕えていた料理人によるものという説も
  あったが真偽は疑わしい。17世紀頃の成立であることは確かだが、おそら
  くは古くからあったソースを改良したものに過ぎず、また、19世紀前半の
  カレームのレシピはヴルテを煮詰め、卵黄と煮詰めた生クリームでとろみ
  を付けるというものだった。同様に1867年刊グフェ『料理の本』のレシピ
  も、炒めた仔牛肉と野菜に小麦粉を振りかけてからブイヨン注ぎ、これを
  煮詰め、漉してから生クリームを加えるというものだった。}}{ベシャメルソース}}\label{ux30d9ux30b7ux30e3ux30e1ux30ebux30bdux30fcux30b9102020}}

\hypertarget{sauce-bechamel}{%
\paragraph{Sauce Béchamel}\label{sauce-bechamel}}

\index{きほんそーす@基本ソース!へしやめる@ベシャメル---}
\index{そーす@ソース!きほん@基本---!へしやめる@ベシャメル---}
\index{そーす@ソース!へしやめる@ベシャメル---}
\index{へしやめる@ベシャメル!そーす@---ソース}
\index{sauce@sauce!00grandes@*Grandes ---s de Base!bechamel@--- Béchamel}
\index{sauce@sauce!bechamel@--- Béchamel}
\index{bechamel@Béchamel (sauce)}

(仕上がり 5 L分)

\begin{itemize}
\item
  \textbf{白いルー}\ldots{}\ldots{}650 g。
\item
  \textbf{使用する液体}\ldots{}\ldots{}沸かした牛乳5 L。
\item
  \textbf{追加素材}\ldots{}\ldots{}白身で脂肪のない仔牛肉300
  gをさいの目に切り、みじん切
  りにした玉ねぎ(小)2個分とタイム1枝、粗く砕いたこしょう1つまみ、塩25
  g とバターを鍋に入れて蓋をし、色付かないように弱火で蒸し煮したもの。
\item
  \textbf{作業手順}\ldots{}\ldots{}沸かした牛乳でルーを溶く。混ぜながら沸騰させる。ここ
  に、先に蒸し煮しておいた野菜と調味料、仔牛肉を加える。弱火で1時間煮込
  む。布で漉し\footnote{\protect\hyperlink{veloute}{ヴルテ}訳注参照。}、表面にバターのかけらをいくつか載せて膜が張らな
  いようにする。肉類を絶対に使わない\footnote{小斉のこと。\protect\hyperlink{sauce-espagnole-maigre}{魚料理用ソース・エスパニョ
    ル}訳注参照。}で調理する必要がある場合は、
  仔牛肉を省き、香味野菜などは上記のとおりに作ること。
\end{itemize}

このソースは次のようなやり方をすると手早く作ることも出来る。沸かした牛
乳に塩、薄切りにした玉ねぎ、タイム、粗く砕いたこしょう、ナツメグを加え
る。蓋をして弱火で10分煮る。これを漉してルーを入れた鍋の中に入れ、強火
にかけて沸騰させる。その後15〜20分だけ煮込めばいい。

\maeaki

\hypertarget{ux30c8ux30deux30c8ux30bdux30fcux30b9}{%
\subsubsection{トマトソース}\label{ux30c8ux30deux30c8ux30bdux30fcux30b9}}

\hypertarget{sauce-tomate}{%
\paragraph{Sauce tomate}\label{sauce-tomate}}

\index{きほんそーす@基本ソース!とまと@トマト---}
\index{そーす@ソース!きほん@基本---!とまと@トマト---}
\index{そーす@ソース!とまとそーす@トマト---}
\index{とまと@トマト!ソース@---ソース}
\index{sauce@sauce!00grandes@*Grandes ---s de Base!tomate@--- tomate}
\index{sauce@sauce!tomate@--- tomate}
\index{tomate@tomate!sauce@Sauce ---}

(仕上がり5L分)

\begin{itemize}
\item
  \textbf{主素材}\ldots{}\ldots{}トマトピュレ4 L、または生のトマト6 kg。
\item
  \textbf{ミルポワ}\ldots{}\ldots{}さいの目に切って下茹でしておいた塩漬け豚ばら肉140
  g 、1〜2 mm 角のさいの目に刻んだにんじん200 gと玉ねぎ150
  g、ローリエの葉 1枚、タイム1枝、バター100 g。
\item
  \textbf{追加素材}\ldots{}\ldots{}小麦粉150 g、白いフォン2
  L、にんにく2片。
\item
  \textbf{調味料}\ldots{}\ldots{}塩20 g、砂糖30 g、こしょう1つまみ。
\item
  \textbf{作業手順}\ldots{}\ldots{}厚手の片手鍋で、塩漬け豚ばら肉を軽く色付くまで炒める。
  ミルポワの野菜を加え、野菜も色よく炒める。小麦粉を振りかける。ブロンド
  色になるまで炒めてから、トマトピュレまたは潰した生トマトと白いフォン、
  砕いたにんにく、塩、砂糖、こしょうを加える。
\end{itemize}

火にかけて混ぜながら沸騰させる。鍋に蓋をして弱火のオーブンに入れ1時間
半〜2時間加熱する。

目の細かい漉し器または布で漉す。再度、火にかけて数分間沸騰させる。保存
用の器に移し、ソースが空気に触れて表面に膜が張らないよう、バターのかけ
らを載せてやる。

\hypertarget{ux539fux6ce8-4}{%
\subparagraph{【原注】}\label{ux539fux6ce8-4}}

トマトピュレを使い、小麦粉は使わず、その他は上記のとおりに作ってもいい。
漉し器か布で漉してから、充分な濃度になるまでしっかり煮詰めてやること。
\end{recette}\newpage
\hypertarget{ux30d6ux30e9ux30a6ux30f3ux7cfbux306eux6d3eux751fux30bdux30fcux30b9}{%
\section{ブラウン系の派生ソース}\label{ux30d6ux30e9ux30a6ux30f3ux7cfbux306eux6d3eux751fux30bdux30fcux30b9}}

\hypertarget{petites-sauces-brunes-composuxe9es}{%
\subsection{Petites Sauces Brunes
Composées}\label{petites-sauces-brunes-composuxe9es}}
\begin{recette}
\hypertarget{ux30bdux30fcux30b9ux30d3ux30acux30e9ux30fcux30c91}{%
\subsubsection[ソース・ビガラード]{\texorpdfstring{ソース・ビガラード\footnote{ビガラードは本来、南フランスで栽培されるビターオレンジの一種。}}{ソース・ビガラード}}\label{ux30bdux30fcux30b9ux30d3ux30acux30e9ux30fcux30c91}}

\hypertarget{sauce-bigarade}{%
\paragraph{Sauce Bigarade}\label{sauce-bigarade}}

\index{そーす@ソース!びがらーど@---・ビガラード} \index{びがらーど@ビ
ガラード!そーす@ソース・---} \index{sauce@sauce!bigarade@--- Bigarade}
\index{bigarade@bigarade!sauce@Sauce ---}

\hypertarget{ux4ed4ux9d28ux306eux30d6ux30ecux30bc2-ux7528}{%
\subparagraph[仔鴨のブレゼ 用]{\texorpdfstring{仔鴨のブレゼ\footnote{料理の仕立てとしてのブレゼはたんに「蒸し煮」することではない。原
  則的な手順をごく簡単に述べておく。厚めに輪切りにしたにんじんと玉ね
  ぎをバターまたはラードで炒め、ブーケガルニとともに鍋に入れる。表面
  を色よく焼き固めた肉を、脂身の少ない肉の場合には豚背脂のシートで包
  んで素材がぴったり入る大きさ鍋に入れ、\protect\hyperlink{fonds-brun}{茶色いフォン}
  を注ぎ、蓋をしてオーブンに入れ、微沸騰の状態を保つようにして煮込む。
  火が通ったら肉を取り出し、鍋に残った煮汁でソースを作る。詳細につい
  ては\protect\hyperlink{}{第7章 肉料理}参照。}
用}{仔鴨のブレゼ 用}}\label{ux4ed4ux9d28ux306eux30d6ux30ecux30bc2-ux7528}}

仔鴨をブレゼした際の煮汁を漉してから浮き脂を取り除き\footnote{dégraisser
  デグレセ。}、煮詰める。 煮詰まったらさらに目の細かい布で漉し、ソース1
Lあたりオレンジ4個とレモ ン1個の搾り汁でのばす。

\hypertarget{ux4ed4ux9d28ux306eux30ddux30efux30ec3ux7528}{%
\subparagraph[仔鴨のポワレ用]{\texorpdfstring{仔鴨のポワレ\footnote{ポワレについても簡単に述べておく。本書においてポワレは「フライパ
  ンで焼く」という意味で用いられることは決してない(フライパンで魚な
  どを焼くことをポワレと呼ぶようになったのは20世紀後半のこと)。本書
  では「ローストの一種」と定義されており(この点がカレームとはまった
  く異なる)、3〜4mm角に切った香味野菜(マティニョン)を生のまま鍋の
  底に入れ、その上に味付けをした肉を置く。溶かしバターをかけてから、
  蓋をして中火のオーブンに入れて蒸し焼きにする。時折様子を見て溶かし
  バターをかけてやること。肉に火が通ったら鍋から取り出し、\protect\hyperlink{fonds-de-veau-brun}{茶色い仔
  牛のフォン}を注いで弱火にかけて10分程煮込み、
  マティニョンとして用いた野菜から風味を引き出してソースにする。これ
  がレシピにある「ポワレのフォン」となる。}用}{仔鴨のポワレ用}}\label{ux4ed4ux9d28ux306eux30ddux30efux30ec3ux7528}}

仔鴨をポワレのフォンから浮き脂を取り除き、でんぷんで軽くとろみ付け
する。砂糖20gに大さじ\undemi{}杯のヴィネガーを加えて火にかけカラメル状
にしたものを加える。ブレゼ用と同様に、オレンジとレモンの搾り汁でのばす。

仔鴨のブレゼ用、ポワレ用いずれの場合も、細かい千切りにしてよく下茹でし
ておいたオレンジの皮大さじ2とレモンの皮\footnote{柑橘類の表皮を薄く剥いてごく細い千切りにしたり、器具を用いてお
  ろしたものをゼスト zeste と呼ぶ。千切りにしたものは苦味を取り除く
  ために下茹ですることが多い。}大さじ1を加えて仕上げる。

\maeaki

\hypertarget{ux30dcux30ebux30c9ux30fcux98a8ux30bdux30fcux30b9}{%
\subsubsection{ボルドー風ソース}\label{ux30dcux30ebux30c9ux30fcux98a8ux30bdux30fcux30b9}}

\hypertarget{sauce-bordelaise}{%
\paragraph{Sauce Bordelaise}\label{sauce-bordelaise}}

\index{そーす@ソース!ぼるどーふう@ボルドー風---} \index{ぼるどーふう@
ボルドー風!そーす@---ソース} \index{sauce@sauce!bordelaise@---
Bordelaise} \index{bordelais@bordelais!sauce@Sauce Bordelaise}

赤ワイン3 dlにエシャロットのみじん切り大さじ2、粗く砕いたこしょう、タ
イム、ローリエの葉\undemi{}枚を加えて火にかけ、\unquart{}量になるまで
煮詰める。ソース・エスパニョル1 dlを加えて火にかけ、浮いてくる夾雑物を
丁寧に取り除きながら弱火で15分間煮る。目の細かい布で漉す。

溶かしたグラスドヴィアンド大さじ1杯とレモン汁\unquart{}個分、細かいさ
いの目か輪切りにしてポシェしておいた牛骨髄を加えて仕上げる。

\ldots{}\ldots{}牛、羊の赤身肉のグリル用

【原注】こんにちではボルドー風ソースをこのように赤ワインを用いて作るが、
本来的には誤りである。元来は白ワインが用いられていた。これは\protect\hyperlink{sauce-bonnefoy}{ボルドー
風ソース・ボヌフォワ}として後述。

\maeaki

\hypertarget{ux30d6ux30ebux30b4ux30fcux30cbux30e5ux98a8ux30bdux30fcux30b9}{%
\subsubsection{ブルゴーニュ風ソース}\label{ux30d6ux30ebux30b4ux30fcux30cbux30e5ux98a8ux30bdux30fcux30b9}}

\hypertarget{sauce-bourguignonne}{%
\paragraph{Sauce Bourguignonne}\label{sauce-bourguignonne}}

\index{そーす@ソース!ぶるごーにゅふう@ブルゴーニュ風---} \index{ぶるごー
にゅふう@ブルゴーニュ風!そーす@---ソース}
\index{sauce@sauce!bourguignonne@--- Bourguignonne}
\index{bourguignon@bourguignon!sauce@Sauce Bourguignonne}

上質の赤ワイン1\undemi{} L に、エシャロット5個の薄切りとパセリの枝、タ
イム、ローリエの葉\undemi{}枚、マッシュルームの切りくず\footnote{料理に使うマッシュルームは通常、トゥルネ(包丁を持った側の手は動
  かさずに材料を回して切ることからついた用語)すなわち螺旋状に切って
  供するが、その際に少なくない量の切りくずが出るのでこれを使う。}25gを加えて、
半量になるまで煮詰める。布で漉し、ブールマニエ80g(バター45gと小麦粉
35g)を加えてとろみを付ける。提供直前にバター150gを溶かし込み、カイエ
ンヌ\footnote{赤唐辛子の粉末だがカイエンヌは本来、品種名。日本のタカノツメと
  比べると辛さもややマイルドで、風味も異なる。}ごく少量で加えて風味よく仕上げる。

\ldots{}\ldots{}いろいろな卵料理や、家庭料理に好適なソース。

\maeaki

\hypertarget{ux30d6ux30ebux30bfux30fcux30cbux30e5ux98a8ux30bdux30fcux30b9}{%
\subsubsection{ブルターニュ風ソース}\label{ux30d6ux30ebux30bfux30fcux30cbux30e5ux98a8ux30bdux30fcux30b9}}

\hypertarget{sauce-bretonne}{%
\paragraph{Sauce Bretonne}\label{sauce-bretonne}}

\index{そーす@ソース!ぶるたーにゅふうちゃいろ@ブルターニュ風--- (茶色)}
\index{ぶるたーにゅふう@ブルターニュ風!そーすちゃいろ@---ソース (茶色)}
\index{sauce@sauce!bretonne brune@--- Bretonne (brune)}
\index{breton@breton!sauce brune@Sauce Bretonne (brune)}

中位の玉ねぎ2個をみじん切りにして、バターでブロンド色になるまで炒める。
白ワイン2\undemi{}dlを注ぎ、半量になるまで煮詰める。ここにソース・エス
パニョル3\undemi{}およびトマトソース同量を加える。7〜8分間煮立ててから、
刻んだパセリを加えて仕上げる。

【原注】このソースは{[}白いんげん豆のブルターニュ風{]}以外にはほとんど使わ
れない。

\maeaki

\hypertarget{ux30bdux30fcux30b9ux30b9ux30eaux30fcux30ba6}{%
\subsubsection[ソース・スリーズ]{\texorpdfstring{ソース・スリーズ\footnote{スリーズ
  cerises はさくらんぼのこと。このレシピでグロゼイユ(す
  ぐり)のジュレを用いるが、古くはさくらんぼを用いていたことからこの
  名称となった。}}{ソース・スリーズ}}\label{ux30bdux30fcux30b9ux30b9ux30eaux30fcux30ba6}}

\hypertarget{sauce-aux-cerises}{%
\paragraph{Sauce aux cerises}\label{sauce-aux-cerises}}

\index{そーす@ソース!すりーず@---・スリーズ}
\index{sauce@sauce!cerise@--- aux Cerises}

ポルト酒2dlにイギリス風ミックススパイス\footnote{Mixed
  spiceのこと。Pudding spiceとも呼ばれる。シナモン、ナツメ
  グ、オールスパイスの組み合わせが典型的。これにクローブ、生姜、コリ
  アンダーシード、キャラウェイシードなどが加わっていることも多い。}1つまみと、すりおろしたオレ
ンジの皮を大さじ\undemi{}杯加えて\deuxtiers{}量になるまで煮詰める。
\protect\hyperlink{}{グロゼイユのジュレ}
2\undemi{}を加え、仕上げにオレンジ果汁を加える。

\ldots{}\ldots{}大型ジビエの料理用だが、鴨のポワレやブレゼにも用いられる。

\maeaki

\hypertarget{ux30bdux30fcux30b9ux30b7ux30e3ux30f3ux30d4ux30cbux30e7ux30f37}{%
\subsubsection[ソース・シャンピニョン]{\texorpdfstring{ソース・シャンピニョン\footnote{champignons
  キノコ全般を意味する語だが、単独で用いられる場合はい
  わゆるマッシュルームを指す。}}{ソース・シャンピニョン}}\label{ux30bdux30fcux30b9ux30b7ux30e3ux30f3ux30d4ux30cbux30e7ux30f37}}

\hypertarget{sauce-aux-champignons}{%
\paragraph{Sauce aux Champignons}\label{sauce-aux-champignons}}

\index{そーす@ソース!まっしゅるーむちゃいろ@マッシュルーム--- (茶色)}
\index{まっしゅるーむ@マッシュルーム!そーすちゃいろ@---ソース (茶色)}
\index{sauce@sauce!champignons brune@--- aux Champignons (brune)}
\index{champignon@champignon!sauce brune@Sauce aux Champignons
(brune)}

マッシュルームの茹で汁2\undemi{} dl を半量になるまで煮詰める。
\protect\hyperlink{sauce-demi-glace}{ソース・ ドゥミグラス}8
dlを加えて数分間煮立てる。布で漉し、 バター50
gを投入して味を調え、あらかじめ下茹でしておいた小さめのマッシュ
ルームの笠100 gを加えて仕上げる。

\maeaki

\hypertarget{ux30bdux30fcux30b9ux30b7ux30e3ux30ebux30adux30e5ux30c6ux30a3ux30a8ux30fcux30eb8}{%
\subsubsection[ソース・シャルキュティエール]{\texorpdfstring{ソース・シャルキュティエール\footnote{シャルキュトリ(豚肉加工業)風、の意。Charcutrieの語源はchar(肉)
  +cuite(調理された)+rie(業)。ハムやソーセージなどと定番の組合せ
  であるマスタードを使う\protect\hyperlink{sauce-robert}{ソース・ロベール}と、おなじ
  く定番のつけ合わせであるコルニション(小さいうちに収穫してヴィネガー
  漬けにしたきゅうり。専用品種がある)を使うことに由来。}}{ソース・シャルキュティエール}}\label{ux30bdux30fcux30b9ux30b7ux30e3ux30ebux30adux30e5ux30c6ux30a3ux30a8ux30fcux30eb8}}

\hypertarget{sauce-charcutiere}{%
\paragraph{Sauce Charcutière}\label{sauce-charcutiere}}

\index{そーす@ソース!しゃるきゅとりふう@シャルキュトリ風---} \index{しゃ
るきゅとりふう@シャルキュトリ風!そーす@---ソース}
\index{sauce@sauce!charcutière@--- Charcutière}
\index{charcutier@charcutier!sauce@Sauce Charcutière}

提供直前に、\protect\hyperlink{sauce-robert}{ソース・ロベール}1 L
に細さ2mm程度で短かめの千切り\footnote{1〜2mm程度の細さの千切りにした野菜などをジュリエンヌjulienneと呼
  ぶ。}
にしたものを加える(\protect\hyperlink{sauce-robert}{ソース・ロベール}参照)。

\maeaki

\hypertarget{ux30bdux30fcux30b9ux30b7ux30e3ux30b9ux30fcux30eb10}{%
\subsubsection[ソース・シャスール]{\texorpdfstring{ソース・シャスール\footnote{狩人風、の意。古くは猟獣肉をすり潰したものを使った料理を指した
  という説もある。マッシュルームとエシャロット、白ワインを使うのが特
  徴であり、このソースを使った料理にも「シャスール」の名が付けられる。}}{ソース・シャスール}}\label{ux30bdux30fcux30b9ux30b7ux30e3ux30b9ux30fcux30eb10}}

\hypertarget{sauce-chasseur}{%
\paragraph{Sauce Chasseur}\label{sauce-chasseur}}

\index{そーす@ソース!しゃすーる@---・シャスール} \index{しゃすーる@シャ
スール!そーす@ソース・---} \index{sauce@sauce!chasseur@--- Chasseur}
\index{chasseur@chasseur!sauce@Sauce ---}

生のマッシュルームを薄切りにしたもの150gをバターで炒める。エシャロット
\footnote{échalote
  玉ねぎによく似ているが、小ぶりで水分が少なく、香味野菜
  としてよく用いられる。伝統的な品種は種子ではなく種球を植えて栽培す
  る。なお、日本でしばしば「エシャレット」の名称で流通しているものは
  ラッキョウの若どりであり、フランス料理で用いるエシャロットとはまっ
  たく異なる。}のみじん切り大さじ2\undemi{}杯を加えてさらに軽く炒め、白ワイン3
dl
を注ぎ、半量になるまで煮詰める。\protect\hyperlink{sauce-tomate}{ソマトソース}3
dl と\protect\hyperlink{sauce-demi-glace}{ソース・ドゥミグラス}2
dlを加える。数分間沸騰さ せたら、バター150 gと、セルフイユ\footnote{cerfeuil
  日本ではチャービルとも呼ばれるセリ科のハーブ。}とエストラゴン\footnote{estragon
  日本ではタラゴンとも呼ばれるヨモギ科のハーブ。フランス
  料理ではとても好まれる重要なハーブのひとつ。フレンチタラゴンとロシ
  アンタラゴンの2種がある。料理に用いるのはフレンチタラゴンであり、
  この品種は種子ではなく株分けや挿し芽で殖やして栽培される。寒さには
  比較的強いが、日本の梅雨の湿度や夏の暑さには弱い。}をみじん切り
にしたもの大さじ1\undemi{}杯を加えて仕上げる。

\maeaki

\hypertarget{ux30bdux30fcux30b9ux30b7ux30e3ux30b9ux30fcux30ebux30a8ux30b9ux30b3ux30d5ux30a3ux30a8ux6d41}{%
\subsubsection{ソース・シャスール(エスコフィエ流)}\label{ux30bdux30fcux30b9ux30b7ux30e3ux30b9ux30fcux30ebux30a8ux30b9ux30b3ux30d5ux30a3ux30a8ux6d41}}

\hypertarget{sauce-chasseur-procede-escoffier}{%
\paragraph{Sauce Chasseur (Procédé
Escoffier)}\label{sauce-chasseur-procede-escoffier}}

\index{そーす@ソース!しゃすーるえすこふぃえ@---・シャスール(エスコフィ
エ流)} \index{しゃすーる@シャスール!そーすしゃすーるえすこふぃえ@ソー
ス・--- (エスコフィエ流)} \index{sauce@sauce!chasseur escoffier@---
Chasseur (Procédé Escoffier)} \index{chasseur@chasseur!sauce
escoffier@Sauce --- (Procédé Escoffier)}

生のマッシュルームを薄切りにしたもの150gを、バターと植物油で軽く色付く
まで炒める。みじん切りにしたエシャロット大さじ1杯を加え、なるべくすぐ
に余分な油をきる。白ワイン2dl とコニャック約50ml を注ぎ、半量になるま
で煮詰める。\protect\hyperlink{sauce-demi-glace}{ソース・ドゥミグラス}4
dlと{[}トマトソー ス{]}2
dl、\protect\hyperlink{glace-de-viande}{グラスドヴィアンド}大さじ\undemi{}杯を加え
る。

5分間沸騰させたら、仕上げにパセリのみじん切り少々を加える。

\maeaki

\hypertarget{ux8336ux8272ux3044ux30bdux30fcux30b9ux30b7ux30e7ux30d5ux30edux30ef15}{%
\subsubsection[茶色いソース・ショフロワ]{\texorpdfstring{茶色いソース・ショフロワ\footnote{chaudショ「熱い、温かい」とfroidフロワ「冷たい」の合成語で、火
  を通した肉や魚を冷まし、表面にこのソース・ショフロワを覆うように塗
  り付け、さらにジュレを覆いかけた料理。料理の発祥については諸説あり、
  なかでもルイ15世に仕えていた料理長ショフロワChaufroixが考案したと
  いう説を支持してなのか、英語ではこの料理をChaufroixと綴ることも多
  い。Chaud-froidの表記は19世紀後半には文献に見られる。なお、複数形
  はchauds-froidsと綴る。トリュフの薄切りやエストラゴンなどのハーブ
  その他で表面に華麗な装飾を施すことが19世紀には盛んに行なわれていた。
  現代でも装飾に凝った仕立てにするケースは多い。}}{茶色いソース・ショフロワ}}\label{ux8336ux8272ux3044ux30bdux30fcux30b9ux30b7ux30e7ux30d5ux30edux30ef15}}

\hypertarget{sauce-chaud-froid-brune}{%
\paragraph{Sauce Chaud-froid brune}\label{sauce-chaud-froid-brune}}

\index{そーす@ソース!しょふろわちゃいろ@---・ショフロワ(茶色)}
\index{しょふろわ@ショフロワ!そーす(ちゃいろ)@ソース・--- (茶色)}
\index{sauce@sauce!chaud-froid brune@--- Chaud-froid brune}
\index{chaud-froid@chaud-froid!sauce brune@Sauce --- brune}

(仕上がり1L 分)

\protect\hyperlink{sauce-demi-glace}{ソース・ドゥミグラス}\troisquarts{}
Lとトリュフエッ センス1 dl、ジュレ6〜7 dlを用意する。

ソース・ドゥミグラスにトリュフエッセンスを加えて、強火で煮詰めるが、こ
の時に鍋から離れないこと。煮詰めながらジュレを少量ずつ加えていく。最終
的に\deuxtiers{}量程度まで煮詰める。

味見をして、ソースがショフロワに使うのに丁度いい濃さになっているか確認
すること。

マデラ酒またはポルト酒\undemi{}dlを加える。布で漉し、ショフロワの主素
材の表面に塗り付けるのに丁度いい固さになるまで、丁寧にゆっくり混ぜなが
ら冷ます。

\maeaki

\hypertarget{ux8336ux8272ux3044ux30bdux30fcux30b9ux30b7ux30e7ux30d5ux30edux30efux9d28ux7528}{%
\subsubsection{茶色いソース・ショフロワ(鴨用)}\label{ux8336ux8272ux3044ux30bdux30fcux30b9ux30b7ux30e7ux30d5ux30edux30efux9d28ux7528}}

\hypertarget{sauce-chaud-froid-brune-pour-canards}{%
\paragraph{Sauce Chaud-froid brune pour
Canards}\label{sauce-chaud-froid-brune-pour-canards}}

\index{そーす@ソース!しょふろわちゃいろかもよう@茶色い---・ショフロワ
(鴨用)} \index{しょふろわ@ショフロワ!ちゃいろいそーすしょふろわかも
よう@茶色いソース・---(鴨用)} \index{sauce@sauce!chaud-froid brune
pour canards@--- Chaud-froid brune pour Canards}
\index{chaud-froid@chaud-froid!sauce brune pour Canards@Sauce ---
brune pour Canards}

作り方は上記、\protect\hyperlink{sauce-chaud-froid-brune}{茶色いソース・ショフロワ}と同
様だが、トリュフエッセンスではなく、鴨のガラでとったフュメ1\undemi{}
dlを用いること。また、上記のレシピよりややしっかり煮詰めること。

ソースを布で漉したら、オレンジ3個分の搾り汁、とオレンジの皮をごく薄く
剥いて細かい千切りにしたもの\footnote{zeste
  ゼスト。オレンジやレモンの皮の表面を器具を用いてすりおろ
  すか、ナイフでごく薄く表皮を向き、細かい千切りにしたもの。ここでは
  後者を使う指定になっている。}大さじ2杯を加える。オレンジの皮の千切
りはしっかりと下茹でしてよく水気をきっておくこと。

\maeaki

\hypertarget{ux8336ux8272ux3044ux30bdux30fcux30b9ux30b7ux30e7ux30d5ux30edux30efux30b8ux30d3ux30a8ux7528}{%
\subsubsection{茶色いソース・ショフロワ(ジビエ用)}\label{ux8336ux8272ux3044ux30bdux30fcux30b9ux30b7ux30e7ux30d5ux30edux30efux30b8ux30d3ux30a8ux7528}}

\hypertarget{sauce-chaud-froid-brune-pour-gibier}{%
\paragraph{Sauce Chaud-froid brune pour
Gibier}\label{sauce-chaud-froid-brune-pour-gibier}}

\index{そーす@ソース!しょふろわちゃいろじびえよう@茶色い---・ショフロ
ワ(ジビエ用)} \index{しょふろわ@ショフロワ!そーすしょふろわじびえよ
う@茶色いソース・---(ジビエ用)} \index{sauce@sauce!chaud-froid brune
pour Gibier@--- Chaud-froid brune pour Gibier}
\index{chaud-froid@chaud-froid!sauce brune pour Gibier@Sauce --- brune
pour Gibier}

作り方は上記\protect\hyperlink{sauce-chaud-froid-brune}{標準的なソース・ショフロワ}と同
じだが、トリュフエッセンスではなく、ショフロワとして供するジビエのガラ
でとったフュメ\footnote{\protect\hyperlink{fonds-de-gibier}{ジビエのフォン}参照。}2dlを用いること。

\maeaki

\hypertarget{ux30c8ux30deux30c8ux5165ux308aux30bdux30fcux30b9ux30b7ux30e7ux30d5ux30edux30ef}{%
\subsubsection{トマト入りソース・ショフロワ}\label{ux30c8ux30deux30c8ux5165ux308aux30bdux30fcux30b9ux30b7ux30e7ux30d5ux30edux30ef}}

\hypertarget{sauce-chaud-froid-tomatee}{%
\paragraph{Sauce Chaud-froid tomatée}\label{sauce-chaud-froid-tomatee}}

\index{そーす@ソース!しょふろわとまといり@トマト入り---・ショフロワ}
\index{しょふろわ@ショフロワ!そーす(とまといり)@トマト入りソース・---}
\index{sauce@sauce!chaud-froid tomatée@--- Chaud-froid tomatée}
\index{chaud-froid@chaud-froid!sauce tomatée@Sauce --- tomatée}

良質で、既によく煮詰めてあるトマトピュレ1 Lを、さらに煮詰めながら7〜8
dlのジュレを少しずつ加えていく。全体量が1L以下になるまで煮詰めること。

布で漉し、使いやすい固さになるまで、ゆっくり混ぜながら冷ます。

\maeaki

\hypertarget{ux30bdux30fcux30b9ux30b7ux30e5ux30f4ux30ebux30a4ux30e6}{%
\subsubsection{ソース・シュヴルイユ}\label{ux30bdux30fcux30b9ux30b7ux30e5ux30f4ux30ebux30a4ux30e6}}

\hypertarget{sauce-chevreuil}{%
\paragraph{Sauce Chevreuil}\label{sauce-chevreuil}}

\index{しゅうるいゆ@シュヴルイユ!そーす@ソース・---} \index{そーす@ソー
ス!しゅうるいゆ@---・シュヴルイユ} \index{のろしか@ノロ鹿!そーすしゅう
るいゆ@ソース・シュヴルイユ} \index{sauce@sauce!chevreuil@---
Chevreuil} \index{chevreuil@chevreuil!sauce@Sauce ---}

\protect\hyperlink{sauce-poivrade}{標準的なソース・ポワヴラード})と同様に作るが、

\begin{enumerate}
\def\labelenumi{\arabic{enumi}.}
\item
  マリネした牛・羊肉の料理に添える場合\footnote{chevreuil
    シュヴルイユはノロ鹿のことだが、このように事前にマリ
    ネした牛・羊肉を用いた料理にもこのソースを使い「シュヴルイユ(風)」
    と\ruby{謳}{うた}う。1806年刊ヴィアール『帝国料理の本』においてノ
    ロ鹿のフィレは香辛料を加えたワインヴィネガーで48時間マリネしてから
    調理すると書かれている。オド『女性料理人のための本』では、確認出来
    た1834年の第4版から1900年の第78版に至るまで、ノロ鹿の項において
    「一週間もヴィネガーたっぷりの漬け汁でマリネするのはやりすぎだが、
    強い味が好みなら1〜4日間」香辛料と赤ワインあるいはヴィネガーでマリ
    ネするといい、と説明されている。つまり、ノロ鹿とは必ずマリネしてか
    ら調理するものという一種のコンセンサスがあったために、マリネした牛・
    羊肉の料理にも「シュヴルイユ(風)」の名称が謳われるようになったと考
    えられる。}は、ハム入りの\protect\hyperlink{mirepoix}{ミルポ
  ワ}を加える。
\item
  ジビエ料理に添える場合は、そのジビエの端肉を加える。
\end{enumerate}

素材をヘラなどで強く押し付けるようにして漉す\footnote{シノワ(\protect\hyperlink{sauce-espagnole}{ソース・エスパニョル}訳注参照)などを用いる。}。良質の赤ワイン
1\undemi{}dlをスプーン1杯ずつ加えながら煮て、浮き上がってくる不純物を
丁寧に取り除いていく\footnote{dépouiller デプイエ ≒ écumer エキュメ}。

最後に、カイエンヌごく少量と砂糖1つまみを加えて味を\ruby{調}{とと
の}え、布で漉す。

\maeaki

\hypertarget{ux30bdux30fcux30b9ux30b3ux30ebux30d9ux30fcux30eb23}{%
\subsubsection[ソース・コルベール]{\texorpdfstring{ソース・コルベール\footnote{17世紀の政治家、ジャン・バティスト・コルベール(1619〜1683)の
  名を冠したもの。}}{ソース・コルベール}}\label{ux30bdux30fcux30b9ux30b3ux30ebux30d9ux30fcux30eb23}}

\hypertarget{sauce-colbert}{%
\paragraph{Sauce Colbert}\label{sauce-colbert}}

\index{そーす@ソース!こるべーる@---・コルベール} \index{こるべーる@コ
ルベール!そーす@ソース・---} \index{sauce@sauce!colbert@--- Colbert}
\index{colbert@Colbert!sauce@Sauce ---}

\protect\hyperlink{}{メートルドテルバター}に\protect\hyperlink{glace-de-viande}{グラスドヴィアンド}を加え
たもののことだが、正しくは「\protect\hyperlink{}{コルベールバター}」と呼ぶべきものだ
\footnote{具体的なレシピは\protect\hyperlink{}{コルベールバター}参照のこと。}。

また、コルベールバターと\protect\hyperlink{sauce-chateaubriand}{ソース・シャトーブリアン}との違いを明確にさ
せようとして、メートルドテルバターにエストラゴンを加える者もいる。だが、
必ずそうすべきということではない。実際、ブール・コルベールとソース・シャ
トーブリアンは明らかに違うものだからだ。ソース・シャトーブリアンは軽く
仕上げたグラスドヴィアントにバターとパセリのみじん切りを加えたものであ
る。一方、コルベールバターあるいはソース・コルベールと呼ばれているもの
はあくまでもバターが主であって、グラスドヴィアンドは補助的なものに過ぎ
ない。

\maeaki

\hypertarget{ux30bdux30fcux30b9ux30c7ux30a3ux30a2ux30fcux30d6ux30eb25}{%
\subsubsection[ソース・ディアーブル]{\texorpdfstring{ソース・ディアーブル\footnote{悪魔の意。}}{ソース・ディアーブル}}\label{ux30bdux30fcux30b9ux30c7ux30a3ux30a2ux30fcux30d6ux30eb25}}

\hypertarget{sauce-diable}{%
\paragraph{Sauce Diable}\label{sauce-diable}}

\index{そーす@ソース!でぃあーぶる@---・ディアーブル} \index{でぃあーぶ
る@ディアーブル!そーす@ソース・---} \index{sauce@sauce!diable@---
Diable} \index{diable@diable!sauce@Sauce ---}

このソースはごく少量ずつ作るのが一般的だが、ここではそれを守らずに、仕
上り2\undemi{} dlとして説明する

白ワイン3dlにエシャロット3個分のみじん切りを加え、\untiers{}量以下にな
るまで煮詰める。

\protect\hyperlink{sauce-demi-glace}{ソース・ドゥミグラス}2
dlを加えて数分間煮立たせ、
仕上げにカイエンヌの粉末をたっぷり効かせる\footnote{「たっぷり」という表現に惑わされないよう注意。}。

\ldots{}\ldots{}鶏と鳩のグリルに合わせる。

\hypertarget{ux539fux6ce8}{%
\subparagraph{【原注】}\label{ux539fux6ce8}}

白ワインではなくヴィネガーを煮詰め、仕上げにハーブを加えて作る調理現場
もあるが、著者としては上記の作り方がいいと思う。

\maeaki

\hypertarget{ux30bdux30fcux30b9ux30c7ux30a3ux30a2ux30fcux30d6ux30ebux30a8ux30b9ux30b3ux30d5ux30a3ux30a8}{%
\subsubsection{ソース・ディアーブル・エスコフィエ}\label{ux30bdux30fcux30b9ux30c7ux30a3ux30a2ux30fcux30d6ux30ebux30a8ux30b9ux30b3ux30d5ux30a3ux30a8}}

\hypertarget{sauce-diable-escoffier}{%
\paragraph{Sauce Diable Escoffier}\label{sauce-diable-escoffier}}

\index{そーす@ソース!でぃあーぶるえすこふぃえ@---・ディアーブル・エス
コフィエ} \index{でぃあーぶるえすこふぃえ@ディアーブル・エスコフィエ!
そーす@ソース・---・エスコフィエ} \index{sauce@sauce!diable
escoffier@--- Diable Escoffier} \index{diable@diable!sauce
escoffier@Sauce --- Escoffier}

このソースは完成品が市販\footnote{現在は市販されていないと思われる。フランスにおいては未確認だが、
  1980年代までアメリカ合衆国ではナビスコがソース・ロベール・エスコフィ
  エとともに瓶詰めを生産、販売していた。初版ではこれら2つの製品への
  言及がなく、第二版で追加されたことから、1903年〜1907年の間に製品化
  された可能性もある。また、第二版(1907年)と同年の英訳版、第三版
  (1912年)にはソース・スリーズ・エスコフィエの記述が見られるが、こ
  れは第四版で削除されており、生産中止になったと思われる。エスコフィ
  エ・ブランドの既製品ソースはさらに他にもあったようだが詳細は不明。なお、
  エスコフィエは1922年頃、ジュリユス・マジがブイヨンキューブ(日本で
  は「マギーブイヨン」の商品名)を開発する際にも協力した。}されている。同量の柔くしたバターを混ぜ合
わせるだけでいい。

\maeaki

\hypertarget{ux30bdux30fcux30b9ux30c7ux30a3ux30a2ux30fcux30cc28}{%
\subsubsection[ソース・ディアーヌ]{\texorpdfstring{ソース・ディアーヌ\footnote{ローマ神話の女神ディアーナのこと。ギリシア神話のアルテミスに相
  当し、狩猟、貞潔の女神。また月の女神ルーナ(セレーネー)と同一視さ
  れた。ここでは大型ジビエ料理用のソースであるから、狩猟の女神という
  意味合いが強い。}}{ソース・ディアーヌ}}\label{ux30bdux30fcux30b9ux30c7ux30a3ux30a2ux30fcux30cc28}}

\hypertarget{sauce-diane}{%
\paragraph{Sauce Diane}\label{sauce-diane}}

\index{そーす@ソース!てぃあーぬ@---・ディアーヌ} \index{てぃあーぬ@ディ
アーヌ!そーす@ソース・---} \index{sauce@sauce!diane@--- Diane}
\index{diane@Diane!sauce@Sauce ---}

不純物を充分に取り除き、コクと風味ゆたかな\protect\hyperlink{sauce-poivrade}{ソース・ポワヴラー
ド}5 dlを用意する。提供直前に、泡立てた生クリーム4 dl
(生クリーム2dlを泡立てて倍量にする)と、小さな三日月の形にしたトリュ
フのスライスと固茹で卵の白身を加える。

\ldots{}\ldots{}大型ジビエの骨付き背肉および、その中心部を円筒形に切り出したもの
\footnote{noisette ノワゼット。}、フィレ料理用。

\maeaki

\hypertarget{ux30bdux30fcux30b9ux30c7ux30e5ux30afux30bbux30eb29}{%
\subsubsection[ソース・デュクセル]{\texorpdfstring{ソース・デュクセル\footnote{デュクセル・セッシュ(第2章ガルニチュール参照)を用いることか
  らこの名称が用いられている。}}{ソース・デュクセル}}\label{ux30bdux30fcux30b9ux30c7ux30e5ux30afux30bbux30eb29}}

\hypertarget{sauce-duxelles}{%
\paragraph{Sauce Duxelles}\label{sauce-duxelles}}

\index{そーす@ソース!てゅくせる@---・デュクセル} \index{てゅくせる@デュ
クセル!そーす@ソース・---} \index{sauce@sauce!duxelles@--- Duxelles}
\index{duxelles@duxelles!sauce@Sauce ---}

白ワイン2dlとマッシュルームの茹で汁2 dlにエシャロットのみじん切り大さじ2
杯を加えて、\untiers{}量まで煮詰める。\protect\hyperlink{sauce-demi-glace}{ソース・ドゥミグラ
ス}\undemi{} Lとトマトピュレ1\undemi{} dl、\protect\hyperlink{}{デュク
セル・セッシュ}大さじ4杯を加える。5分間煮立たせ、パセリのみじん切り
大さじ\undemi{}を加える。

\ldots{}\ldots{}グラタンの他、いろいろな料理に用いられる。

\hypertarget{ux539fux6ce8-1}{%
\subparagraph{【原注】}\label{ux539fux6ce8-1}}

ソース・デュクセルはイタリア風ソースと混同されることが多いが、ソース・
デュクセルにはハムも、赤く漬けた舌肉も入れないので、まったく別のものだ。

\maeaki

\hypertarget{ux30bdux30fcux30b9ux30a8ux30b9ux30c8ux30e9ux30b4ux30f399}{%
\subsubsection[ソース・エストラゴン]{\texorpdfstring{ソース・エストラゴン\footnote{ヨモギ科のハーブ。\protect\hyperlink{sauce-chasseur}{ソース・シャスール}訳注参照。}}{ソース・エストラゴン}}\label{ux30bdux30fcux30b9ux30a8ux30b9ux30c8ux30e9ux30b4ux30f399}}

\hypertarget{sauce-estragon}{%
\paragraph{Sauce Estragon}\label{sauce-estragon}}

\index{そーす@ソース!えすとらこんちゃいろ@---・エストラゴン(茶色いソー
ス)} \index{えすとらこんちゃいろ@エストラゴン!そーす@ソース・---(茶
色いソース)} \index{sauce@sauce!estragonbrune@--- Estragon (sauce
brune)} \index{estragon@estragon!sauce brune@Sauce --- (brune)}

(仕上り2\undemi{}dl分)

白ワイン2dlを沸かし、エストラゴンの枝20gを投入する。蓋をして10分間、煎
じる\footnote{infuserアンフュゼ。}。2\undemi{}dlの\protect\hyperlink{sauce-demi-glace}{ソース・ドゥミグラス}また
は、\protect\hyperlink{jus-de-veau-lie}{とろみを付けた仔牛のジュ}を加え、約\deuxtiers{}
量になるまで煮詰める。布で漉し、みじん切りにしたエストラゴン小さじ1杯
を加えて仕上げる。

\ldots{}\ldots{}仔牛や仔羊の背肉の中心を円筒形に切り出した料理や家禽料理用。

\maeaki

\hypertarget{ux30bdux30fcux30b9ux30d5ux30a3ux30caux30f3ux30b7ux30a8ux30fcux30eb34}{%
\subsubsection[ソース・フィナンシエール]{\texorpdfstring{ソース・フィナンシエール\footnote{Financier徴税官(財務官)風の意。フランス革命以前の徴税官は、王
  に代わって徴税を行なう大貴族が就く役職であり、膨大な利権によりきわめて
  裕福であったという。このソースと組み合わせる\protect\hyperlink{}{ガルニチュール・フィナン
  シエール}が、雄鶏のとさかと睾丸、仔羊の胸腺肉、トリュフなどの比較的
  入手困難あるいは高級とされる食材で構成されていることが名称の由来と思わ
  れる。ブリヤ=サヴァランは『美味礼讃』(味覚の生理学)において、徴税官
  たちは旬のはしりの食材を真っ先に食べられる、いわば特権階級だと述べてい
  る。なお、カレーム『19世紀フランス料理』においては、ソースとガルニチュー
  ルを分離せず、「ラグー・アラ・フィナンシエール」として採りあげられてい
  るが、全ての素材を別々に加熱調理してソースと合わせるものであり、いわゆ
  る「煮込み」とは呼びがたいものとなっている。フランス料理の影響が比較的
  強かった北イタリアにこの原型に近いと思われるラグー「ピエモンテ風フィナ
  ンツィエラ」がある。鶏のとさか、肉垂、睾丸、鶏レバーおよび仔牛の胸腺肉
  などを煮込んだものだが、レシピを読む限りにおいては比較的庶民的あるいは
  農民的料理に変化したものと思われる (cf.~Anna Gosetti della Salda,
  \emph{Le Ricette Regionali Italiane}, Milano, Solares, 1967,
  p.57.)。ちなみに焼
  き菓子のフィナンシエfinancierも同語源だが、何故その名称になったかは不
  明。}}{ソース・フィナンシエール}}\label{ux30bdux30fcux30b9ux30d5ux30a3ux30caux30f3ux30b7ux30a8ux30fcux30eb34}}

\hypertarget{sauce-financiere}{%
\paragraph{Sauce Financière}\label{sauce-financiere}}

\index{そーす@ソース!ふぃなんしえーる@---・フィナンシエール} \index{ふぃ
なんしえーる@フィナンシエール!そーす@ソース・---} \index{ちょうせいか
んふう@徴税官風!そーすふぃなんしえーる@ソース・---}
\index{sauce@sauce!financiere@--- Financière}
\index{financier@financier!sauce@Sauce Financière}

\protect\hyperlink{sauce-madere}{ソース・マデール}1\unquart{}Lを\troisquarts{}量以下に
なるまで煮詰め、火から外してトリュフエッセンス1 dlを加える。布で漉して
仕上げる。

\ldots{}\ldots{}\protect\hyperlink{}{ガルニチュール・フィナンシエール}用だが、その他の肉料理にも用い
られる。

\maeaki

\hypertarget{ux9999ux8349ux30bdux30fcux30b9}{%
\subsubsection{香草ソース}\label{ux9999ux8349ux30bdux30fcux30b9}}

\hypertarget{sauce-aux-fines-herbes}{%
\paragraph[Sauce aux Fines Herbes]{\texorpdfstring{Sauce aux Fines
Herbes\footnote{料理名としていわゆる「ハーブ」についてかつてはfines
  herbesの表
  現が多かった。とはいえ、こんにちでは特定のハーブ名をソースや料理名
  に添えて言うことが多い。例えばCôtelette de veau au thymコトレット
  ドヴォオタン(仔牛の骨付き背肉、タイム風味)、やFilet de bar poêlé,
  compote de tomate au basilicフィレドバールポワレ コンポットートド
  トマトバジリック(スズキのフィレとトマトのコンポート、バジル風味)
  など。また、栽培レベルで「香草、ハーブ」の総称としては herbes
  aromatiques エルブアロマティック、あるいはたんにaromatiquesアロマ
  ティックが一般的。}}{Sauce aux Fines Herbes}}\label{sauce-aux-fines-herbes}}

\index{そーす@ソース!こうそう@香草---} \index{こうそう@香草!そーす@---
ソース} \index{はーぶ@ハーブ!こうそうそーす@香草ソース}
\index{sauce@sauce!fines herbes@--- aux Fines Herbes} \index{fines
herbes@fines herbes!sauce@Sauce aux ---}

白ワイン3dlを沸かし、パセリの葉、セルフイユ、エストラゴン、シブレット
を各1つまみ強、投入する。約20分間煎じる。布で漉し、\protect\hyperlink{sauce-demi-glace}{ソース・ドゥミグラ
ス}または\protect\hyperlink{jus-de-veau-lie}{とろみを付けた仔牛の ジュ}6
dlを加える。仕上げに、煎じるのに使ったのと同
じ香草を細かく刻んだもの計、大さじ2\undemi{}杯とレモンの搾り汁少々を加
える。

\hypertarget{ux539fux6ce8-2}{%
\subparagraph{【原注】}\label{ux539fux6ce8-2}}

古典料理ではこの「香草ソース」と\protect\hyperlink{sauce-duxelles}{ソース・デュクセル}
が混同されることもあったが、こんにちではまったく違うものとして扱われて
いる。

\maeaki

\hypertarget{ux30b8ux30e5ux30cdux30fcux30f4ux98a8ux30bdux30fcux30b9}{%
\subsubsection{ジュネーヴ風ソース}\label{ux30b8ux30e5ux30cdux30fcux30f4ux98a8ux30bdux30fcux30b9}}

\hypertarget{sauce-genevoise}{%
\paragraph{Sauce Genevoise}\label{sauce-genevoise}}

\index{そーす@ソース!じゅねーうふう@ジュネーヴ風---} \index{じゅねーう
ふう@ジュネーヴ風!そーす@---ソース} \index{sauce@sauce!genevoise@---
Genevoise} \index{genevois@genevois!sauce@Sauce Genevoise}

鍋にバターを熱し、細かく刻んだミルポワを色付かないよう強火でさっと炒め
る。ミルポワの材料は、にんじん100 g、玉ねぎ80 g、タイムとローリエ少々、
パセリの枝20 g。そこにサーモンの頭1kgと粗く砕いたこしょう1つまみを入れ、
蓋をして弱火で15分程蒸し煮する。

鍋に残ったバターを捨て、赤ワイン1Lを注ぐ。半量になるまで煮詰める。そこ
に\protect\hyperlink{sauce-espagnole-maigre}{魚料理用ソース・エスパニョル}\undemi{}
Lを
加える。弱火で1時間煮込む。漉し器を使い、材料を押しつけながら漉す。し
ばらく休ませてから、表面に浮いた油脂を取り除く\footnote{dégraisser
  デグレセ。レードルなどを用いて浮いてきた余計な油脂を取り除く作業。}

さらに赤ワイン\undemi{} Lと、魚のフュメ\undemi{} Lを加える。ソースの表
面に浮いてくる不純物を徹底的に取り除き\footnote{dépouiller デプイエ ≒
  écumer エキュメ。}、丁度いい濃さになるまで煮 詰める。

これを布で漉し、静かに混ぜながら、アンチョヴィのエッセンス大さじ1杯と
バター150 gを加えて仕上げる。

\ldots{}\ldots{}サーモン、鱒料理用。

\hypertarget{ux539fux6ce8-3}{%
\subparagraph{【原注】}\label{ux539fux6ce8-3}}

このソースはもともとカレームが「ジェノヴァ風」\footnote{Sauce à la
  génoise au vin de Bordeaux ボルドー産ワインを用いた
  ジェノヴァ風ソース(『19世紀フランス料理』第3巻、80頁)。本書のこのレシ
  ピと同様に魚料理用ソースだ。ボルドーの赤ワインにみじん切りにして下茹で
  したマッシュルーム、トリュフ、エシャロットを加えてオールスパイスとこしょ
  う少々を入れ、適度に煮詰める。ソース・エスパニョルと赤ワインを加え、湯
  煎にかけておく。提供直前にバター少量を加えて仕上げる、というもの。本書
  においてこのソースを「原型」とするのには疑問が残るところだろう。}と名付けたものだが、その
後ルキュレ、グフェ\footnote{グフェ『料理の本』(1867年)の420ページにあるジュネーヴ風ソー
  スは、薄切りにした玉ねぎ、エシャロット、粗挽きこしょう、にんにく、
  バターを鍋に入れて色付くまで炒め、そこにブルゴーニュ産赤ワインを注
  ぐ。弱火で玉ねぎに火が通るまで煮る。ソース・エスパニョルと仔牛のブ
  ロンドのジュを加えて煮詰め、布で漉す。提供直前にマデラ酒の風味を加
  えて茹でたトリュフのみじん切りとアンチョビバターを加える、というも
  の。赤ワインと玉ねぎ、仕上げにアンチョビを加える点は共通しているが、
  グフェのが肉料理用であるのに対して、本書のこのソースは明らかに魚料
  理用であり、まったく同じソースと呼べるかは疑問の残るところだろう。}が立て続けに「ジュネーヴ風」の名称を用いた。だが、ジュ
ネーヴは赤ワインの産地ではないから理屈としてはおかしい\footnote{料理名に冠された地名は、由来が明確にあるものがある一方で、まっ
  たく意味不明か、あるいはいい加減な思い付きで付けられたのではないか
  とさえ思われるものも少なくない。(à la) russe「ロシア風」や (à la)
  moscovite「モスクワ風」などはロシア料理起源か、あるいは18世紀末〜
  19世紀前半にかけてロシア帝国の宮廷や貴族がこぞってフランスから料理
  人を招聘し、帰国した彼らが創案した料理などはある程度しっかりとした
  由来がわかるものも多い。一方で、(à l')espagnole「スペイン風」(à
  l')italienne「イタリア風」(à la) romaine「ローマ風」(à la grecque)
  「ギリシア風」(à l')allemande「ドイツ風」(à l')hollandaise「オラン
  ダ風」などは由来の不明なケースが非常に多い。\protect\hyperlink{sauce-espagnole}{ソース・エスパニョ
  ル}などはその典型例とも言うべきものだろう。\\
  この原注では由来に非常にこだわっているが、そもそもカレームのレシピ
  は上述のように「ボルドー産ワインを用いたジェノバ風ソース」であるか
  ら、赤ワインの産地かどうかということは実はさしたる問題にはならない。
  重要なのは後半の、赤ワインを用いることがこのソースのポイントという
  こと。}。

間違っているとはいえ、ジュネーヴ風という名称で定着してしまっているので、
本書でもそのままにしている。だが、ジュネーヴ風であれジェノヴァ風であれ、
カレーム、ルキュレ、デュボワ、グフェはいずれもこのソースに赤ワインを用
いるよう指示している。つまり赤ワインを用いることがこのソースのポイント。

\maeaki

\hypertarget{ux30bdux30fcux30b9ux30b4ux30c0ux30fcux30eb37}{%
\subsubsection[ソース・ゴダール]{\texorpdfstring{ソース・ゴダール\footnote{ガルニチュール・ゴダールの構成要素がガルニチュール・フィナンシ
  エールとよく似ている点などから、おそらくは18世紀の徴税官(つまりフィ
  ナンシエ)であり作家としても活動したクロード・ゴダール・ドクール
  Claude Godard d'Aucour(1716〜1795)の名を冠したものと考えられる。}}{ソース・ゴダール}}\label{ux30bdux30fcux30b9ux30b4ux30c0ux30fcux30eb37}}

\hypertarget{sauce-godard}{%
\paragraph[Sauce Godard]{\texorpdfstring{Sauce Godard\footnote{底本とした現行版(第四版)では最後がdではなくtとなっているが、
  初版から第三版にいたるまでdとなっており、現行版は明らかな誤植。}}{Sauce Godard}}\label{sauce-godard}}

\index{そーす@ソース!ごだーる@---・ゴダール} \index{ごだーる@ゴダール!
そーす@ソース・---} \index{sauce@sauce!godart@--- Godart}
\index{godard@Godard!sauce@Sauce ---}

シャンパーニュまたは辛口の白ワイン4 dlにハム入りの細かく刻んだ{[}ミルポ
ワ{]}。{[}ソース・ドゥミグラス{]}1
Lとマッシュルームのエッセンス2dlを加える。
弱火に10分かけ、シノワ\footnote{\protect\hyperlink{sauce-espagnole}{ソース・エスパニョル}訳注参照。}で漉す。

\deuxtiers{}量になるまで煮詰め、布で漉す。

\ldots{}\ldots{}\protect\hyperlink{}{ガルニチュール ゴタール}用。

\maeaki

\hypertarget{ux30bdux30fcux30b9ux30b0ux30e9ux30f3ux30f4ux30ccux30fcux30eb40}{%
\subsubsection[ソース・グランヴヌール]{\texorpdfstring{ソース・グランヴヌール\footnote{王家や貴族に仕える狩猟長のことをグランヴヌールと呼ぶ。}}{ソース・グランヴヌール}}\label{ux30bdux30fcux30b9ux30b0ux30e9ux30f3ux30f4ux30ccux30fcux30eb40}}

\hypertarget{sauce-grand-veneur}{%
\paragraph{Sauce Grand-Veneur}\label{sauce-grand-veneur}}

\index{そーす@ソース!くらんうぬーる@---・グランヴヌール} \index{くらん
うぬーる@グランヴヌール!そーす@ソース・---}
\index{sauce@sauce!grand-veneur@--- Grand-Veneur}
\index{grand-veneur@grand-veneur!sauce@Sauce ---}

\protect\hyperlink{fonds-de-gibier}{大型ジビエのフュメ}で澄んだ色合いに作った\protect\hyperlink{sauce-poivrade}{ソース・
ポワヴラード}に、ソース1Lあたり野うさぎの血1dlをマリ
ネ液1dlで薄めたものを加える。

火をごく弱くして、血が沸騰しないよう気をつけながら数分間煮る。布で漉す。

\maeaki

\hypertarget{ux30bdux30fcux30b9ux30b0ux30e9ux30f3ux30f4ux30ccux30fcux30ebux30a8ux30b9ux30b3ux30d5ux30a3ux30a8ux6d41}{%
\subsubsection{ソース・グランヴヌール(エスコフィエ流)}\label{ux30bdux30fcux30b9ux30b0ux30e9ux30f3ux30f4ux30ccux30fcux30ebux30a8ux30b9ux30b3ux30d5ux30a3ux30a8ux6d41}}

\hypertarget{sauce-grand-veneur-procede-escoffier}{%
\paragraph{Sauce Grand-Veneur (Procédé
Escoffier)}\label{sauce-grand-veneur-procede-escoffier}}

\index{そーす@ソース!くらんうぬーるえすこふぃえ@---・グランヴヌール(エ
スコフィエ)} \index{くらんうぬーるえすこふぃえ@グランヴヌール(エスコフィ
エ)!そーす@ソース・---} \index{sauce@sauce!grand-veneur escoffier@---
Grand-Veneur (Procédé Escoffier)}
\index{grand-veneur@grand-veneur!sauce escoffier@Sauce --- (Procédé
Escoffier)}

軽く仕上げた\protect\hyperlink{sauce-poivrade}{ソース・ポワヴラード}1
Lあたり{[}グロゼイ
ユのジュレ{]}大さじ2杯と生クリーム2\undemi{}dlを加える。

\ldots{}\ldots{}上記2つのソースは鹿、猪などの大きな塊肉の料理に用いる。

\maeaki

\hypertarget{ux30bdux30fcux30b9ux30b0ux30e9ux30bfux30f345}{%
\subsubsection[ソース・グラタン]{\texorpdfstring{ソース・グラタン\footnote{魚のグラタン用ソースだが、グラタンの技術的ポイントについては第
  7章「肉料理」参照。}}{ソース・グラタン}}\label{ux30bdux30fcux30b9ux30b0ux30e9ux30bfux30f345}}

\hypertarget{sauce-gratin}{%
\paragraph{Sauce Gratin}\label{sauce-gratin}}

\index{そーす@ソース!くらたん@---・グラタン} \index{くらたん@グラタン!
そーす@ソース・---} \index{sauce@sauce!gratin@--- Gratin}
\index{gratin@gratin!sauce@Sauce ---}

白ワインと、このソースを合わせる魚のアラなどでとった\protect\hyperlink{fumet-de-poisson}{魚のフュ
メ}各3 dlにエシャロットのみじん切り大さじ1\undemi{}
杯を加え、半量以下になるまで煮詰める。

\protect\hyperlink{}{デュクセル・セッシュ}大さじ3杯と、\protect\hyperlink{sauce-espagnole-maigre}{魚料理用ソース・エスパニョ
ル}または\protect\hyperlink{sauce-demi-glace}{ソース・ドゥミグラ ス}5
dlを加える。5〜6分間煮立たせる。提供直前に、パ
セリのみじん切り大さじ\undemi{}を加えて仕上げる。

\ldots{}\ldots{}舌びらめ、メルラン\footnote{タラの近縁種。}、バルビュ\footnote{鰈の近縁種。この場合のフィレはいわゆる「五枚おろし」にしたもの。}のフィレなどのグラタン用。

\maeaki

\hypertarget{ux30bdux30fcux30b9ux30a2ux30b7ux30a743}{%
\subsubsection[ソース・アシェ]{\texorpdfstring{ソース・アシェ\footnote{細かく刻んだもの、の意。}}{ソース・アシェ}}\label{ux30bdux30fcux30b9ux30a2ux30b7ux30a743}}

\hypertarget{sauce-hachee}{%
\paragraph{Sauce Hachée}\label{sauce-hachee}}

\index{そーす@ソース!あしぇ@---・アシェ} \index{sauce@sauce!hachee@---
Hach\'ee}

玉ねぎの細かいみじん切り100gと、エシャロットの細かいみじん切り大さじ
1\undemi{}杯をバターで色付かないよう炒める。ヴィネガー3 dlを注ぎ、半量
まで煮詰める。{[}ソース・エスパニョル{]}4
dlと{[}トマトソース{]}1\undemi{} dl を加える。5〜6分煮立たせる。

ハムの脂身のない部分を細かく刻んだもの大さじ1\undemi{}杯と小ぶりのケイ
パー大さじ1\undemi{}杯、{[}デュクセル・セッシュ{]}大さじ1\undemi{}杯、パセ
リのみじん切り大さじ\undemi{}杯を加えて仕上げる

\ldots{}\ldots{}このソースは\protect\hyperlink{ux30bdux30fcux30b9ux30d4ux30abux30f3ux30c8}{ソース・ピカント}と等価のものと考えていい。用途も同じ。

\maeaki

\hypertarget{ux9b5aux6599ux7406ux7528ux30bdux30fcux30b9ux30a2ux30b7ux30a7}{%
\subsubsection{魚料理用ソース・アシェ}\label{ux9b5aux6599ux7406ux7528ux30bdux30fcux30b9ux30a2ux30b7ux30a7}}

\hypertarget{sauce-hachee-maigre}{%
\paragraph{Sauce Hachée maigre}\label{sauce-hachee-maigre}}

上記と同様に、玉ねぎとエシャロットを色付かないようバターで炒め、ヴィネ
ガーを注いで煮詰める。

魚の\protect\hyperlink{}{クールブイヨン}5
dlを注ぎ、\protect\hyperlink{roux-brun}{茶色いルー}45 gまたはブー
ルマニエ50 gでとろみを付ける。弱火で8〜10分間煮込む。

提供直前に、細かく刻んだハーブミックス大さじ1杯と{[}デュクセル・セッシュ{]}大
さじ1\undemi{}杯、小粒のケイパー大さじ1\undemi{}杯、アンチョヴィソース
大さじ\undemi{}杯とバター60 g、または80〜100 gのアンチョヴィバターを加
えて仕上げる。

\ldots{}\ldots{}エイのような、あまり高級ではない魚のブイイ\footnote{茹でた肉、魚のこと。}用。

\maeaki

\hypertarget{ux30bdux30fcux30b9ux30e6ux30b5ux30ebux30c951}{%
\subsubsection[ソース・ユサルド]{\texorpdfstring{ソース・ユサルド\footnote{もとはハンガリーで農家20戸につき1人の割合で招集された騎兵
  hussard を指す。この語は16世紀まで遡ることが出来るが、のちに「乱暴
  者」といったニュアンスでも使われるようになった。à la hussarde は
  「乱暴に、粗野に」の意味でも用いられるが、料理においてはレフォール
  を使ったものに名付けられることが多い。}}{ソース・ユサルド}}\label{ux30bdux30fcux30b9ux30e6ux30b5ux30ebux30c951}}

\hypertarget{sauce-hussarde}{%
\paragraph{Sauce Hussarde}\label{sauce-hussarde}}

\index{そーす@ソース!ゆさるど@---・ユサルド} \index{ゆさるど@ユサルド!
そーす@ソース・---} \index{sauce@sauce!hussarde@--- Hussarde}
\index{hussarde@Hussarde!sauce@Sauce ---}

玉ねぎ2個とエシャロット2個を細かくみじん切りにして、バターで色よく炒め
る。白ワイン4
dlを注ぎ、半量になるまで煮詰める。\protect\hyperlink{sauce-demi-glace}{ソース・ドゥミグラ
ス}4
dlとトマトピュレ大さじ2杯、\protect\hyperlink{fonds-blanc-ordinaire}{白いフォ
ン}2 dl、生ハムの脂身のないところ80 g、潰した
にんにく1片、ブーケガルニを加える。弱火で25〜30分煮込む。

ハムを取り出して、ソースをスプーンで押すようにして布で漉す。

火にかけて温め、小さなさいの目\footnote{brunoise ブリュノワーズ。}に刻んだハムと、おろしたレフォール
\footnote{raifort いわゆる西洋わさび。}少々、パセリのみじん切りをたっぷり1つまみ加えて仕上げる。

\ldots{}\ldots{}牛、羊肉のグリルまたは串を刺してローストしてアントレ\footnote{通常、ローストは料理区分としてアントレに含められることはないが、
  牛フィレは牛の部位のなかでも比較的小さいものとして、まるごと1本の
  ローストであっても原則的にはアントレに分類される。このソースを用い
  る「牛フィレ ユサルド」は牛フィレの塊に串を刺してローストし、ポム・
  デュシェスとマッシュルームを合わせる。}として供 する際に用いる。

\maeaki

\hypertarget{ux30a4ux30bfux30eaux30a2ux98a8ux30bdux30fcux30b9}{%
\subsubsection{イタリア風ソース}\label{ux30a4ux30bfux30eaux30a2ux98a8ux30bdux30fcux30b9}}

\hypertarget{sauce-italienne}{%
\paragraph{Sauce Italienne}\label{sauce-italienne}}

\index{そーす@ソース!いたりあふう@イタリア風---} \index{いたりあん@イ
タリアン!いたりあふうそーす@イタリア風ソース} \index{いたりあふう@イタ
リア風!そーす@---ソース} \index{sauce@sauce!Italienne@--- Italienne}
\index{italien@italien!sauce italienne@Sauce Italienne}

トマトの風味の効いた\protect\hyperlink{sauce-demi-glace}{ソース・ドゥミグラ
ス}\troisquarts{} Lに、\protect\hyperlink{}{デュクセル・セッシュ}大さ
じ4杯と、加熱ハムの脂身のないところを小さなさいの目に切ったもの125 gを
加える。5〜6分間煮る。提供直前に、パセリとセルフイユ、エスゴラゴンのみ
じん切り大さじ1杯を加えて仕上げる。

\ldots{}\ldots{}いろいろな肉料理に合わせる。

\hypertarget{ux539fux6ce8-4}{%
\subparagraph{【原注】}\label{ux539fux6ce8-4}}

このソースを魚料理に合わせる場合、ハムは使わずに\protect\hyperlink{fumet-de-poisson}{魚のフュ
メ}を煮詰めて加える。

\maeaki

\hypertarget{ux3068ux308dux307fux3092ux4ed8ux3051ux305fux30b8ux30e5ux30a8ux30b9ux30c8ux30e9ux30b4ux30f3ux98a8ux5473}{%
\subsubsection{とろみを付けたジュ エストラゴン風味}\label{ux3068ux308dux307fux3092ux4ed8ux3051ux305fux30b8ux30e5ux30a8ux30b9ux30c8ux30e9ux30b4ux30f3ux98a8ux5473}}

\hypertarget{jus-lie-a-lestragon}{%
\paragraph{Jus lié à l'Estragon}\label{jus-lie-a-lestragon}}

\index{そーす@ソース!とろみをつけたしゅえすとらごん@とろみを付けたジュ
エストラゴン風味} \index{しゅ@ジュ!えすとらごん@とろみを付けた--- エス
トラゴン風味} \index{sauce@sauce!jus lie a l'estragon@Jus lié à
l'Estragon} \index{estragon@estragon!jus lie a l'estragon@Jus lié à
l'Estragon} \index{jus@jus!estragon@--- lié à l'Estragon}

\protect\hyperlink{fonds-de-veau-brun}{仔牛のフォン}または\protect\hyperlink{fonds-de-volaille}{鶏のフォ
ン}に、エストラゴン50gを加えて香りを煮出し\footnote{imfuser アンフュゼ。}た
もの。

布で漉してから、アロールート\footnote{コーンスターチで代用する。}または、でんぷん30
gでとろみを付ける。

\ldots{}\ldots{}白身肉のノワゼットや家禽のフィレなどに添える。

\maeaki\columnbreak

\hypertarget{ux3068ux308dux307fux3092ux4ed8ux3051ux305fux30b8ux30e5ux30c8ux30deux30c8ux98a8ux5473}{%
\subsubsection{とろみを付けたジュ トマト風味}\label{ux3068ux308dux307fux3092ux4ed8ux3051ux305fux30b8ux30e5ux30c8ux30deux30c8ux98a8ux5473}}

\hypertarget{jus-lie-tomate}{%
\paragraph{Jus lié tomaté}\label{jus-lie-tomate}}

\index{そーす@ソース!じゅとまといり@とろみを付けたジュ トマト入り}
\index{じゅ@ジュ!とまといり@とろみを付けた--- トマト入り}
\index{sauce@sauce!jus lie tomateq@Jus lié tomaté}
\index{tomate@tomate!jus lie tomate@Jus lié tomaté}
\index{jus@jus!tomate@--- lié tomaté}

\protect\hyperlink{fonds-de-veau-brun}{仔牛のフォン}1
Lあたりトマトエッセンス3 dlを加え、 \quatrecinquiemes{}量まで煮詰める。

\ldots{}\ldots{}牛、羊肉料理用。

\maeaki

\hypertarget{ux30eaux30e8ux30f3ux98a8ux30bdux30fcux30b9}{%
\subsubsection{リヨン風ソース}\label{ux30eaux30e8ux30f3ux98a8ux30bdux30fcux30b9}}

\hypertarget{sauce-lyonnaise}{%
\paragraph{Sauce Lyonnaise}\label{sauce-lyonnaise}}

\index{そーす@ソース!りよんふう@リヨン風---} \index{りよんふう@リヨン
風!りよんふうそーす@---ソース} \index{sauce@sauce!lyonnaise@---
Lyonnaise} \index{liyonnais@lyonnais!sauce lyonnaise@Sauce Lyonnaise}

中位の大きさの玉ねぎ3個をみじん切りにし、バターでじっくり、ごく弱火で
ブロンド色になるまで炒める。白ワイン2 dlとヴィネガー2 dlを注ぐ。
\untiers{}量まで煮詰め、\protect\hyperlink{sauce-demi-glace}{ソース・ドゥミグラ
ス}\troisquarts{} Lを加える。5〜6分かけて表面に浮い
てくる不純物を丁寧に取り除き\footnote{dépouiller
  デプイエ。現代ではエキュメと呼ぶ現場が多い。}、布で漉す。

\hypertarget{ux539fux6ce8-5}{%
\subparagraph{【原注】}\label{ux539fux6ce8-5}}

このソースを合わせる料理によっては、ソースを布で漉さずに玉ねぎを残して
もいい。

\maeaki

\hypertarget{ux30bdux30fcux30b9ux30deux30c7ux30fcux30eb}{%
\subsubsection{ソース・マデール}\label{ux30bdux30fcux30b9ux30deux30c7ux30fcux30eb}}

\hypertarget{sauce-madere}{%
\paragraph{Sauce Madère}\label{sauce-madere}}

\index{そーす@ソース!までーる@---・マデール} \index{までいら@マデイラ!
そーすまでーる@ソース・マデール} \index{sauce@sauce!madere@--- Madère}
\index{madere@madère!sauce madere@Sauce Madère}

\protect\hyperlink{sauce-demi-glace}{ソース・ドゥミグラス}を煮詰め\footnote{ソース・ドゥミグラスは既に煮詰めて仕上がった状態のものなので、9
  割程度にまでしか煮詰めないことに注意。}、火から外して、 ソース1
Lあたりマデラ酒1 dlの割合で加え、普通の濃度にする。

\maeaki

\hypertarget{ux30bdux30fcux30b9ux30deux30c8ux30edux30c3ux30c854}{%
\subsubsection[ソース・マトロット]{\texorpdfstring{ソース・マトロット\footnote{水夫風、船員風、の意。トゥーレーヌ地方の郷土料理Matelote
  d'anguilleマトロットダンギーユ(うなぎの赤ワイン煮込み)が有名だが、
  赤ワイン煮込みにとどまらず、マトロットの名称を持つ料理は他にも複数
  存在する。}}{ソース・マトロット}}\label{ux30bdux30fcux30b9ux30deux30c8ux30edux30c3ux30c854}}

\hypertarget{sauce-matelote}{%
\paragraph{Sauce Matelote}\label{sauce-matelote}}

\index{そーす@ソース!まとろつと@---・マトロット} \index{まとろつと@マ
トロット!そーすまとろつと@ソース・---} \index{sauce@sauce!matelote@---
Matelote} \index{matelote@matelote!sauce matelote@Sauce Matelote}

魚をポシェするのに使った\protect\hyperlink{}{赤ワイン入りの魚用クールブイヨン}3
dlにマッ シュルームの切りくず25
gを加え、\untiers{}量になるまで煮詰める。

煮詰めたら\protect\hyperlink{sauce-espagnole-maigre}{魚料理用ソース・エスパニョル}8
dl を加えてひと煮立ちさせる。布で漉し、バター150 gとごく少量のカイエンヌ
の粉末を加えて仕上げる。

\maeaki

\hypertarget{ux30bdux30fcux30b9ux30e2ux30efux30eb}{%
\subsubsection{ソース・モワル}\label{ux30bdux30fcux30b9ux30e2ux30efux30eb}}

\hypertarget{sauce-moelle}{%
\paragraph[Sauce Moelle]{\texorpdfstring{Sauce Moelle\footnote{骨髄のこと。}}{Sauce Moelle}}\label{sauce-moelle}}

\index{そーす@ソース!もわる@---・モワル} \index{こつずい@骨髄!そーすも
わる@ソース・モワル} \index{sauce@sauce!moelle@--- Moelle}
\index{moelle@moelle!sauce moelle@Sauce ---}

ソースの作り方は\protect\hyperlink{sauce-bordelaise}{ボルドー風ソース}とまったく同じだ
が、バターを加えるのは何らかの野菜料理に添える場合のみであり、その場合
のバターの量は通常どおりとするこ。

どんな場合にせよ、仕上げに、小さなさいの目に切ってポシェしておいた骨髄
をソース1 Lあたり150〜180 gおよび刻んで下茹でしたパセリの葉小さじ1杯を
加える。

\maeaki

\hypertarget{ux30e2ux30b9ux30afux30efux98a856ux30bdux30fcux30b9}{%
\subsubsection[モスクワ風ソース]{\texorpdfstring{モスクワ風\footnote{モスクワ風の名称を持つ料理や菓子は多い。
  18世紀後半から19世紀前
  半にかけて、ロシアの宮廷や貴族らの間でフランスの食文化が流行し、多
  くのフランス人料理人が招聘され、彼らはロシア料理のレシピをフランス
  に持ち帰った。クーリビヤックなどが代表的な例だろう。また、19世紀後
  半になると、とりわけフランス料理においてもロシア料理からの影響が多
  く見られるようになる。キャビアとウォトカを食前に愉しむのが流行した
  のもその時代からである。フランスとロシアの食文化は相互に影響関係に
  あったと言えよう。}ソース}{モスクワ風ソース}}\label{ux30e2ux30b9ux30afux30efux98a856ux30bdux30fcux30b9}}

\hypertarget{sauce-moscovite}{%
\paragraph{Sauce Moscovite}\label{sauce-moscovite}}

\index{そーす@ソース!もすくわふう@モスクワ風---} \index{もすくわふう@
モスクワ風!そーす@---ソース} \index{sauce@sauce!moscovite@---
Moscovite} \index{moscovite@moscovite!sauce moscovite@Sauce ---}

\protect\hyperlink{fonds-de-gibier}{大型ジビエのフュメ}で作った\protect\hyperlink{sauce-poivrade}{ソース・ポワヴラー
ド}を\troisquarts{} L用意する。提供直前にマラガ酒1 dl
とジェニパーベリーを煎じた汁7 cl、焼いた松の実かスライスして焼いたアー
モンド40 g、大きさを揃えてぬるま湯でもどしておいたコリント産干しぶどう
\footnote{小粒で黒いギリシア産干しぶどう。}40 gを加えて仕上げる。

\ldots{}\ldots{}大型ジビエ\footnote{venaison
  ヴネゾン。ジビエのうちとりわけ大型のものを指す。実際は
  ノロ鹿や猪を指すことがほとんど。}の塊肉の料理用。

\maeaki

\hypertarget{ux30bdux30fcux30b9ux30daux30eaux30b0ux30fc59}{%
\subsubsection[ソース・ペリグー]{\texorpdfstring{ソース・ペリグー\footnote{トリュフの産地として有名なペリゴール地方の町の名。}}{ソース・ペリグー}}\label{ux30bdux30fcux30b9ux30daux30eaux30b0ux30fc59}}

\hypertarget{sauce-perigueux}{%
\paragraph{Sauce Périgueux}\label{sauce-perigueux}}

\index{そーす@ソース!へりくー@---・ペリグー} \index{へりくー@ペリグー!
そーす@ソース・---} \index{sauce@sauce!perigueux@--- Péerigueux}
\index{perigueux@Périgueux!sauce perigueux@Sauce ---}

やや濃いめに煮詰めた\protect\hyperlink{sauce-demi-glace}{ソース・ドゥミグラ
ス}\troisquarts{} Lに、トリュフエッセンス1 \undemi{}
dlと細かく刻んだトリュフ100 gを加える。

\ldots{}\ldots{}いろいろな肉料理、\protect\hyperlink{}{タンバル}、\protect\hyperlink{}{温製パテ}に合わせる。

\maeaki

\hypertarget{ux30bdux30fcux30b9ux30daux30eaux30b0ux30ebux30c7ux30a3ux30fcux30cc60}{%
\subsubsection[ソース・ペリグルディーヌ]{\texorpdfstring{ソース・ペリグルディーヌ\footnote{ペリゴール地方風の意。}}{ソース・ペリグルディーヌ}}\label{ux30bdux30fcux30b9ux30daux30eaux30b0ux30ebux30c7ux30a3ux30fcux30cc60}}

\hypertarget{sauce-puxe9rigourdine}{%
\paragraph{Sauce Périgourdine}\label{sauce-puxe9rigourdine}}

\index{そーす@ソース!へりくるていーぬ@---・ペリグゥルディーヌ}
\index{へりこーるふう@ペリゴール風!そーす@ソース・ペリグルディーヌ}
\index{sauce@sauce!perigourdine@--- Périgourdine}
\index{perigourdin@périgourdin!sauce perigueux@Sauce Périgourdine}

ソース・ペリグーのバリエーション。トリュフを細かく刻むのではなく、オリー
ブ形か小さな真珠のような形状にナイフで成形\footnote{tourner
  トゥルネ。包丁を持っている側の手は動かさずに材料を回す
  ようにして形を整えること。}したものを加える。トリュ
フを厚めにスライスして加える場合もある。

\maeaki

\hypertarget{ux30bdux30fcux30b9ux30d4ux30abux30f3ux30c8}{%
\subsubsection{ソース・ピカント}\label{ux30bdux30fcux30b9ux30d4ux30abux30f3ux30c8}}

\hypertarget{sauce-piquante}{%
\paragraph[Sauce Piquante]{\texorpdfstring{Sauce Piquante\footnote{piquant
  一般的には唐辛子などが「辛い」の意だが、このソースでは
  唐辛子の類は使われておらず、むしろ酸味の効いたソースと言えよう。}}{Sauce Piquante}}\label{sauce-piquante}}

\index{そーす@ソース!ぴかんと@---・ピカント}
\index{sauce@sauce!piquante@--- Piquante}

白ワイン3 dlと良質のヴィネガー3 dlにエシャロットのみじん切り大さじ2
\undemi{}杯を合わせて半量に煮詰める。

\protect\hyperlink{sauce-espagnole}{ソース・エスパニョル}6
dlを加え、浮いてくる不純物を 取り除きながら\footnote{dépouiller
  デプイエ。エキュメécumerと呼ぶ現場も多い。}10分間煮る。

火から外し、コルニション\footnote{専用品種のきゅうりを小さなうちに収穫して酢漬けにしたもの。同様
  のピクルス用きゅうりとしてガーキンスという品種系統があるがもっぱら
  アメリカのハンバーガーに挟まれるようなサイズで収穫して漬けたもので
  あり、フランス料理では用いない。}、パセリ、セルフイユ、エストラゴンを細か
く刻んだもの大さじ2杯を加えて仕上げる。

\ldots{}\ldots{}豚肉のグリル焼き、ブイイ\footnote{bouilli 茹で肉。}、ローストによく合わせるソース。牛肉
のブイイや牛や羊の\protect\hyperlink{}{エマンセ}にも合わせることが出来る。

\maeaki

\hypertarget{ux30bdux30fcux30b9ux30ddux30efux30f4ux30e9ux30fcux30c9105-ux6a19ux6e96}{%
\subsubsection[ソース・ポワヴラード
(標準)]{\texorpdfstring{ソース・ポワヴラード\footnote{このソースは遅くとも16世紀まで遡ることが出来る。1505年に出版さ
  れた\href{http://gallica.bnf.fr/ark:/12148/bpt6k792720}{『フランス語版プラティ
  ナ』}がpoivradeとい
  うフランス語の初出。この本において「ジビエ用こしょうのソース、ポワ
  ヴラード」Saulce de poyvre ou poyvrade pour saulvagieとしてレシピ
  が見られる。パンをよく焼いてヴィネガーに浸してすり潰す。水でもどし
  た干しぶどうと獣の血を加えて混ぜ、玉ねぎと未熟ぶどう果汁、パンを浸
  した残りのヴィネガーを加えて漉し器か布で漉す。これを鍋に入れ、こしょ
  う、生姜、シナモンを入れて炭火の上で30分程煮込む。獣の肉を獣脂を熱
  したフライパンで焼き、皿に盛る。上からポワヴラードをかけて供する、
  という内容(f.LXII)。またこの本には、魚料理用のポワヴラードも掲載
  されている。ただし、これが現代まで続くソース・ポワヴラードの原型と
  捉えるのは早計に過ぎる。ここで注目すべきは、最終的に肉あるいは魚の
  ような主素材とソースが一体化したものは中世〜ルネサンス期にはポター
  ジュと呼ばれていたのに対し、ここではソースを別のものと捉えている点
  である。ポワヴラードという語そのものは「こしょうを効かせたもの」と
  いう意味に過ぎず、1660年刊ピエール・ド・リュヌPierre de Lune『新フ
  ランス料理』におけるPoivrade de pigeonneaux 若鳩のポワヴラードは、
  背開きにした若鳩を平たくのばし、塩、こしょう をして弱火でグリルす
  る。薔薇の香りもしくはにんにく風味のヴィネガーを添えて供する、とい
  うもの(p.190)。ピエール・ド・リュヌのレシピにおいてソースに相当す
  るものはヴィネガーであり、むしろ味付けでこしょうを効かせているとい
  うことが料理名の根拠となっているに過ぎない。ちなみに、生食可能な小
  さなサイズのアーティチョークも古くからポワヴラードと呼ばれている。}
(標準)}{ソース・ポワヴラード (標準)}}\label{ux30bdux30fcux30b9ux30ddux30efux30f4ux30e9ux30fcux30c9105-ux6a19ux6e96}}

\hypertarget{sauce-poivrade}{%
\paragraph{Sauce Poivrade ordinaire}\label{sauce-poivrade}}

\index{そーす@ソース!ほわうらーど@---・ポワヴラード} \index{ほわうらー
ど@ポワヴラード!そーす@ソース・---} \index{sauce@sauce!poivrade
ordinaire@--- Poivrade ordinaire} \index{poivrade@poivrade!sauce
poivrade ordinaire@Sauce --- ordinaire}

細かいさいの目に切ったにんじん100 gと玉ねぎ80 g、刻んだパセリの茎、タ
イム少々、ローリエの葉少々からなる\protect\hyperlink{mirepoix}{ミルポワ}を油で色付くま
で炒める。

ヴィネガー1 dlとマリナード2 dlを注ぎ、\untiers{}量になるまで煮詰める。
\protect\hyperlink{sauce-espagnole}{ソース・エスパニョル}1
Lを注ぎ、約45分間煮込む。

ソースを漉す10分前に、大粒のこしょう8個を叩きつぶして加える。ソースに
こしょうを入れてからの時間がこれ以上少しでも長いと、こしょうの風味が支
配的になり過ぎることになるので注意。

漉し器で香味素材を軽く押すようにして漉す。\protect\hyperlink{}{マリナード}\footnote{ヴィネガーやワイン、香味素材、塩などを合わせて肉を漬け込む液体。
  マリネ液と呼ぶこともある。}2 dlでソー
スをのばす。火にかけて35分間、所定の量\footnote{明記されていないが、ここでは約1
  L。}になるまで煮詰めながら、表
面に浮いてくる不純物を徹底的に取り除く\footnote{dépouiller
  デプイエ。現代ではécumerエキュメの語を使う現場が多い。}。

さらに布で漉し、バター50 gを加えて仕上げる\footnote{現代では、バターでモンテするmonter
  au beurreという表現を用いる 現場も多い。}。

\maeaki

\hypertarget{ux30bdux30fcux30b9ux30ddux30efux30f4ux30e9ux30fcux30c9ux30b8ux30d3ux30a8ux7528}{%
\subsubsection{ソース・ポワヴラード(ジビエ用)}\label{ux30bdux30fcux30b9ux30ddux30efux30f4ux30e9ux30fcux30c9ux30b8ux30d3ux30a8ux7528}}

\hypertarget{sauce-poivrade-pour-gibier}{%
\paragraph{Sauce Poivrade pour
Gibier}\label{sauce-poivrade-pour-gibier}}

\index{そーす@ソース!ほわうらーとしひえ@---・ポワヴラード(ジビエ用)}
\index{ほわうらーと@ポワヴラード!そーすしひえよう@ソース・---(ジビエ
用)} \index{sauce@sauce!poivrade pour gibier@--- Poivrade pour
Gibier} \index{poivrade@poivrade!sauce poivrade pour gibier@Sauce ---
pour Gibier}

細かいさいの目に切ったにんじん125 gと玉ねぎ125 g、タイムの枝と鳥類では
ないジビエ\footnote{gibier à poil
  逐語訳すると「毛の生えているジビエ」すなわち」鹿、
  猪、野うさぎなどを指す。}の端肉1
kgからなる\protect\hyperlink{mirepoix}{ミルポワ}を油で色よく炒め る。

ミルポワが色付いてきたら、鍋の油を捨てる。ヴィネガー3 dlと白ワイン2 dl
を注ぎ、完全に煮詰める。

ソース・エスパニョル1
Lと\protect\hyperlink{fonds-de-gibier}{ジビエの茶色いフォン}2 L、
\protect\hyperlink{}{マリナード}1 Lを加える。

鍋に蓋をして弱火にかける。可能ならオーブンがいい。3時間半〜4時間加熱す
る。

ソースを漉す8分前に、大粒のこしょう12個を叩きつぶして加える。

漉し器で材料を押すようにして漉す。

これをジビエのフォン\unquart{} Lとマリナード\unquart{} Lでのばし、再び
火にかけて40分間、表面に浮いてくる不純物を丁寧に取り除きながら、1 Lに
なるまで煮詰める。

これを布で漉し、バター75gを加えて仕上げる。

\hypertarget{ux539fux6ce8-6}{%
\subparagraph{【原注】}\label{ux539fux6ce8-6}}

一般的にはジビエ料理のソースにはバターを加えないことになっているが、本
書では軽くバターを加えることを推奨する。そうすると、ソースの色の赤みは
薄まるが、繊細で滑らかな口あたりに仕上がる。

\maeaki

\hypertarget{ux30bdux30fcux30b9ux30ddux30ebux30c8}{%
\subsubsection{ソース・ポルト}\label{ux30bdux30fcux30b9ux30ddux30ebux30c8}}

\hypertarget{sauce-au-porto}{%
\paragraph{Sauce au Porto}\label{sauce-au-porto}}

\index{そーす@ソース!ほると@---・ポルト} \index{ほると@ポルト!そーす@
ソース・---} \index{sauce@sauce!porto@--- au Porto}
\index{porto@Porto!sauce au porto@Sauce au ---}

マデラ酒ではなくポルト酒を用いて、\protect\hyperlink{sauce-madere}{ソース・マデール}と
同様に作る。

\maeaki

\hypertarget{ux30ddux30ebux30c8ux30acux30ebux98a873ux30bdux30fcux30b9}{%
\subsubsection[ポルトガル風ソース]{\texorpdfstring{ポルトガル風\footnote{日本でもフランス語のままソース・ポルチュゲーズと呼ばれることは
  多い。フランス料理においてポルトガル風の名称を付けた料理はトマトをベー
  スとしたものがほとんど。ただし、トマトを使うからといってポルトガル風の
  名が必ず付くということはまったくない。なお、このソースとまったく関係な
  いが、\emph{Lettres Portugaises}
  レットル・ポルチュゲーズ『ぽるとがる文{[}ぶ
  み{]}』という題名の本が17世紀にフランスで出版され人々の感動を誘った。リ
  ルケや佐藤春夫が自国語に翻訳、翻案したものも非常に有名。実在したポルト
  ガルの修道女マリアナ・アルコフォラドがポルトガルに駐屯していたフランス
  軍人シャシー公爵に宛てた5通の恋文をまとめた、事実にもとづく書簡集だと
  長い間信じられていた。しかし、20世紀になってから、ガブリエル・ド・ギユ
  ラーグという男性文筆家によるまったくの創作であることが証明された。とは
  いえ作品の文学的価値はまったく減じることのない名作であり、書簡体小説の
  嚆矢とも言うべきもの。この小説形式は18世紀に隆盛を迎え、ラクロ『危険な
  関係』やルソー『新エロイーズ』、ゲーテ『若きウェルテルの悩み』など多く
  の傑作がものされた。19世紀バルザック『二人の若妻の手記』やドストエフス
  キー『貧しき人々』がこの小説形式の流行の最後を飾る傑作だろう。なお、ト
  マトは16世紀に既にフランスにもたらされており、オリヴィエ・ド・セールは
  1600年に刊行した『農業経営論』において、見た目には美しいが食べても美味
  しくないと断じている。食材として広く普及したのは19世紀以降であり、ヴィ
  アール、カレームなどの料理書でも材料として用いられるようになる。トマト
  ペーストの製品化は19世紀前半に実現されており、グリモ・ド・ラ・レニエー
  ルの「アルマナ」において言及がある。とりわけ19世紀後半はトマトを用いた
  料理が増えたのが特徴であり、オマール・アメリケーヌや舌びらめのデュグレ
  レなどは当時のトマトブームの反映と考えていい。また19世紀後半の小説家フ
  ロベールの遺作『ブヴァールとペキュシェ』においても、主人公二人が当時流
  行であったトマトの栽培に挑戦するが「芽掻き」をしなかったために失敗して
  しまうという場面が描かれている。}ソース}{ポルトガル風ソース}}\label{ux30ddux30ebux30c8ux30acux30ebux98a873ux30bdux30fcux30b9}}

\hypertarget{sauce-portugaise}{%
\paragraph{Sauce Portugaise}\label{sauce-portugaise}}

\index{そーす@ソース!ほるとかるふう@ポルトガル風---} \index{ほるとかる
ふう@ポルトガル風!そーす@---ソース} \index{sauce@sauce!porugaise@---
Portugaise} \index{portugais@portugais!sauce portugaise@Sauce
Portugaise}

(仕上り1 L分)

大きめの玉ねぎ1個を細かくみじん切りにする。鍋に油を熱し、強火で玉ねぎ
を炒める。玉ねぎがブロンド色になったら、皮を剥いて種子を取り除き、粗み
じん切りにしたトマト750 gと、つぶしたにんにく1片、塩、こしょうを加える。
トマトの酸味が強い場合は砂糖少々も加える。鍋に蓋をして、弱火で煮る。
\protect\hyperlink{essences-diverses}{トマトエッセンス}少々と、薄めに作ったトマトソース
を適量\footnote{仕上りの全体量が1
  Lなので、トマトソースを加える量は、グラスドヴィアンド
  を加える前の段階で0.9 L程度になるよう調整する。}、温めて溶かした\protect\hyperlink{glace-de-viande}{グラスドヴィアンド}1
dl、 新鮮なパセリの葉のみじん切り大さじ1杯を加えて仕上げる。

\maeaki

\hypertarget{ux30d7ux30edux30f4ux30a1ux30f3ux30b9ux98a8ux30bdux30fcux30b9}{%
\subsubsection{プロヴァンス風ソース}\label{ux30d7ux30edux30f4ux30a1ux30f3ux30b9ux98a8ux30bdux30fcux30b9}}

\hypertarget{sauce-provencal}{%
\paragraph{Sauce Provençale}\label{sauce-provencal}}

\index{そーす@ソース!ふろうあんすふう@プロヴァンス風---} \index{ふろう
あんすふう@プロヴァンス風!そーす@---ソース}
\index{sauce@sauce!provencale@--- Provençale}
\index{provencal@provençal!sauce provencale@Sauce Provençale}

大ぶりのトマト12個の皮を剥き、つぶして種子は取り除いて、粗く刻む\footnote{concasser
  コンカセ。}。 ソテー鍋に2\undemi{}
dlの油を熱し、そこにトマトを入れる。塩、こしょう、
粉砂糖1つまみで味を調える。しっかりつぶしたにんにく(小)1片と細かく刻
んだパセリ小さじ1杯を加える。

蓋をして弱火で30分間程、煮溶かす。

\hypertarget{ux539fux6ce8-7}{%
\subparagraph{【原注】}\label{ux539fux6ce8-7}}

このソースについてはさまざまな解釈があるが、本書ではブルジョワ料理にお
ける本物の「プロヴァンス風ソース」のレシピ、つまりはトマトを煮溶かした
もの、を収録した。

\maeaki

\hypertarget{ux30bdux30fcux30b9ux30ecux30b8ux30e3ux30f3ux30b975}{%
\subsubsection[ソース・レジャンス]{\texorpdfstring{ソース・レジャンス\footnote{摂政時代、すなわちオルレアン公フィリップが幼少だったルイ15世の
  摂政を務めた時代(1715〜1723年)のこと。オルレアン公は美食家として
  有名で、とりわけシャンパーニュを好んだという。この時代はフランス宮
  廷料理の絶頂期でもあった。}}{ソース・レジャンス}}\label{ux30bdux30fcux30b9ux30ecux30b8ux30e3ux30f3ux30b975}}

\hypertarget{sauce-regence}{%
\paragraph{Sauce Régence}\label{sauce-regence}}

\index{そーす@ソース!れしやんす@---・レジャンス} \index{れしやんす@レ
ジャンス!そーす@ソース・---} \index{sauce@sauce!regence@--- Régence}
\index{regence@Régence!sauce@Sauce ---}

ライン産ワイン3
dlに、細かく刻んであらかじめ日を通しておいた\protect\hyperlink{mirepoix}{ミルポ
ワ}1 dlと生トリュフの切りくず25gを加え、半量になるまで煮詰
める。トリュフのシーズンでない時季はトリュフエッセンスを使う。\protect\hyperlink{sauce-demi-glace}{ソース・
ドゥミグラス}8 dlを加え、数分間弱火にかけて浮いてく
る不純物を丁寧に取り除き\footnote{dépouiller デプイエ ≒ écumer
  エキュメ。}、布で漉す。

\ldots{}\ldots{}牛、羊の大きな塊肉の料理用。

\maeaki

\hypertarget{ux30bdux30fcux30b9ux30edux30d9ux30fcux30eb77}{%
\subsubsection[ソース・ロベール]{\texorpdfstring{ソース・ロベール\footnote{この名称のソースは古くからある。文献で初めて出てくるのは16世紀
  フランソワ・ラブレーの小説『ガルガンチュアとパンタグリュエル』。そ
  の「第四の書」で料理人の名が大量に列挙される章がある。そのうちの多
  くは架空の人名だが、その中のロベールという料理人がこのソースを考案
  したと書いている。ただし、具体的にどのようなソースかまでは描写され
  ておらず「うさぎのロースト、鴨、加工していない豚肉、卵のポシェ、塩
  漬けのメルラン{[}鱈の近縁種{]}、その他まことに多くの料理に欠かせない
  ソース」と書いてあるのみ(第40章)。どんな料理にも合うと書かれてし
  まうとむしろ特徴を捉え難くなってしまう。いずれにせよ、遅くとも16世
  紀には「ソース」として成立していたと考えられる。また、17世紀のシャ
  ルル・ペロー著『物語集』の「眠れる森の美女」においても、このソース
  名が登場する一節がある。このように16世紀以降多くの文学作品をはじめ
  とする文献にこのソース名は見られる。レシピとしては、1651年刊ラ・ヴァ
  レーヌ『フランス料理の本』における「豚腰肉 ソース・ロベール添え」
  がもっとも古いもののひとつだろう。概略は、豚腰肉を、ヴェルジュ{[}未
  熟ぶどう果汁、中世料理においてよく用いられた{]}とヴィネガー、セージ
  を振り掛けながらローストする。下に置いた脂受け皿に焼いた豚肉から流
  れ落ちた脂がたまるので、これを使って玉ねぎをこんがり炒める。炒めた
  玉ねぎの上に豚後ろ身を載せ、豚腰肉をローストする際にかけたのと同じ
  ソースをかける。このソースはソースロベールと呼ばれている(p.51)。ま
  た、干鱈のソース・ロベール添えの場合は、バターとヴェルジュ少々、マ
  スタードで作るが、ケイパーやシブール{[}葱{]}を加えてもいい(p.202)と
  あり、同じ名称のソースとは見做しがたい。18世紀以降のソース・ロベー
  ルは多かれ少なかれいずれもマスタードを加える点が共通しているので、
  名称が先にあり、内容が時代とともにはっきりしたものになっていたのだ
  ろう。}}{ソース・ロベール}}\label{ux30bdux30fcux30b9ux30edux30d9ux30fcux30eb77}}

\hypertarget{sauce-robert}{%
\paragraph{Sauce Robert}\label{sauce-robert}}

\index{そーす@ソース!ろへーる@---・ロベール} \index{ろへーる@ロベール!
そーす@ソース・---} \index{sauce@sauce!robert@--- Robert}
\index{robert@Robert!sauce robert@Sauce ---}

(仕上り5 dl分)

大きめの玉ねぎを細かくみじん切りにし、バターで色付かないよう強火でさっ
と炒める。

白ワイン2
dlを注ぎ、\untiers{}量になるまで煮詰める。\protect\hyperlink{sauce-demi-glace}{ソース・ドゥミグ
ラス}3 dlを加え、弱火で10分間煮る。

シノワ\footnote{主として金属製で円錐形に取っ手の付いた漉し器。清朝の高級役人が
  かぶっていた帽子の形状から「中国の」を意味するchinoisの名称となっ
  たと言われている。}で漉し(これは任意。漉さなくてもいい)、火から外して、粉砂
糖1つまみとマスタード大さじ1杯を加えて仕上げる。

\maeaki

\hypertarget{ux30bdux30fcux30b9ux30edux30d9ux30fcux30ebux30a8ux30b9ux30b3ux30d5ux30a3ux30a879}{%
\subsubsection[ソース・ロベール・エスコフィエ]{\texorpdfstring{ソース・ロベール・エスコフィエ\footnote{\protect\hyperlink{sauce-diable-escoffier}{ソース・ディアーブル・エスコフィエ}訳注参照。}}{ソース・ロベール・エスコフィエ}}\label{ux30bdux30fcux30b9ux30edux30d9ux30fcux30ebux30a8ux30b9ux30b3ux30d5ux30a3ux30a879}}

\hypertarget{sauce-robert-escoffier}{%
\paragraph{Sauce Robert Escoffier}\label{sauce-robert-escoffier}}

\index{そーす@ソース!ろへーるえすこふぃえ@---・ロベール・エスコフィエ}
\index{ろべーる@ロベール!そーすろへーるえすこふぃえ@ソース・---・エス
コフィエ} \index{sauce@sauce!robert escoffier@--- Robert Escoffier}
\index{robert@Robert!sauce robert escoffier@Sauce --- Escoffier}

このソースは完成品が市販されている。

温かい料理にも冷たい料理にもよく合う。温かい料理に合わせる場合は、同量
の\protect\hyperlink{fonds-de-veau-brun}{仔牛の茶色いフォン}と混ぜること。

\ldots{}\ldots{}豚、仔牛、鶏、魚のグリル焼きに特によく合う。

\maeaki

\hypertarget{ux30edux30fcux30deux98a880ux30bdux30fcux30b9}{%
\subsubsection[ローマ風ソース]{\texorpdfstring{ローマ風\footnote{フランス料理における「ローマ風」の名称は「イタリア風」と同様に
  とくに根拠や由来が見出せないものが多い。このソースの場合は松の実を
  使うところから、20世紀前半に活躍したイタリアの作曲家レスピーギのロー
  マ三部作のうちの「ローマの松」を想起させるが、残念ながらこの曲が作
  曲されたのは1924年、つまり本書より後なので関係はない。だが、松の実
  を採るイタリアカサマツは、アッピア街道の並木などで有名なように、イ
  タリアとりわけローマ近辺において多く見られる(だからこそレスピーギ
  が曲の題材にしたわけだが)。その意味においては、松の実を使っている
  ということがこのソース名の根拠と見ることも不可能ではないだろう。し
  かしながら、それを証明する文献、史料があるかは不明。}ソース}{ローマ風ソース}}\label{ux30edux30fcux30deux98a880ux30bdux30fcux30b9}}

\hypertarget{sauce-romaine}{%
\paragraph{Sauce Romaine}\label{sauce-romaine}}

砂糖50 gを火にかけてブロンド色にカラメリゼ\footnote{焦がさないように弱火で混ぜながら熱で砂糖を溶かしていく。}する。これをヴィネガー
1\undemi{}
dlでのばす。砂糖を完全に溶かし込めたら、\protect\hyperlink{sauce-espagnole}{ソース・エスパニョ
ル}6 dlと\protect\hyperlink{fonds-de-gibier}{ジビエのフォン}3 dlを加
える。これを\troisquarts{}量弱まで煮詰める。布で漉し、松の実20 gをロー
ストしたものと、大きさが揃るよう選別したスミヌル干しぶどう\footnote{トルコ産の白い干しぶどう。}20
gお よびコリント干しぶとう\footnote{ギリシア産の黒い小粒の干しぶどう(\protect\hyperlink{sauce-moscovite}{モスクワ風ソー
  ス}参照)。}20 gを温湯でもどしたものを加えて仕上げる。

\hypertarget{ux539fux6ce8-8}{%
\subparagraph{【原注】}\label{ux539fux6ce8-8}}

上記のとおり作る場合、このソースは大型ジビエ料理用だが、ジビエのフォン
ではなく通常の\protect\hyperlink{fonds-brun}{茶色いフォン}を使えば、マリネした牛、羊肉
の料理に合わせることも可能。

\maeaki

\hypertarget{ux30ebux30fcux30a2ux30f3ux98a884ux30bdux30fcux30b9}{%
\subsubsection[ルーアン風ソース]{\texorpdfstring{ルーアン風\footnote{ルーアンは野生のcolvertコルヴェール、いわゆる青首鴨を家禽化した
  ルーアン鴨の産地として有名。}ソース}{ルーアン風ソース}}\label{ux30ebux30fcux30a2ux30f3ux98a884ux30bdux30fcux30b9}}

\hypertarget{sauce-rouennaise}{%
\paragraph{Sauce Rouennaise}\label{sauce-rouennaise}}

\index{そーす@ソース!るーあんふう@ルーアン風---} \index{るーあんふう@
ルーアン風!そーす@---ソース} \index{sauce@sauce!rouannaise@---
Rouannaise} \index{rouannais@rouannais!sauce rouannaise@Sauce
Rouannaise}

(仕上り5 dl分)

\protect\hyperlink{sauce-bordelaise}{ボルドー風ソース}4 dl
を用意する。ただし、良質な赤
ワインを使って作ること。(\protect\hyperlink{sauce-bordelaise}{ボルドー風ソース}参照)。

中位の大きさの鴨のレバー3個を裏漉しする。こうして出来たレバーのピュレ
をソースに加え、沸騰させない程度の温度で火を通す\footnote{pocher
  ポシェする。}。絶対に沸騰させ
ないこと。沸騰させてしまうと途端にレバーのピュレが粒状になってしまう。

布で漉し、塩こしょうを効かせる。

このソースの特質\ldots{}\ldots{}エシャロットを加えた赤ワインを煮詰めたものに鴨の生
レバーのピュレを加えたもの。

\ldots{}\ldots{}ルーアン産鴨のローストには、いわば必須といってもいいソース。

\maeaki

\hypertarget{ux30bdux30fcux30b9ux30b5ux30ebux30df92}{%
\subsubsection[ソース・サルミ]{\texorpdfstring{ソース・サルミ\footnote{語源は「ごった煮」を意味する
  salmigondis とするのが定説のようだ
  が、salmigondisがその意味で用いられるようになったのは19世紀以降と
  考えられ、それ以前はragoûtラグーと同義と見なされていた。ラグーはそ
  の語源的意味が「食欲をそそるもの」であり、17世紀に、それまでポター
  ジュと呼ばれていた煮込み料理についてラグーの名称をつけることが流行
  した。また、salmigondisの古い語形のひとつsalmigondinは16世紀の小説
  家フランソワ・ラブレー『ガルガンチュアとパンタグリュエル』の「第四
  の書」において用いられているが、日本語の「ごった煮」のニュアンスと
  はかなり違う意味で、美味な料理のひとつとして挙げられている。いずれ
  にしても、salmigondin, salmigondisというラグーの別称が、ある時期か
  ら鳥類を材料にしたものに限定されるようになったことは確かで、カレー
  ムの『19世紀フランス料理』ではsalmisの語で、野鳥などのラグーを呼ん
  でいる。例えば「ベカスのサルミ」「ペルドローのサルミ」など。カレー
  ムとエスコフィエを比較すると、しばしばカレームにおいてラグーとして
  ひとまとめにされていた料理とソースの組合せが、『料理の手引き』にお
  いては、例えば\protect\hyperlink{}{ガルニチュール・フィナンシエール}と\protect\hyperlink{sauce-financiere}{ソース・フィ
  ナンシエール}のように、別々の項目に分離されてい るものが多くある。}}{ソース・サルミ}}\label{ux30bdux30fcux30b9ux30b5ux30ebux30df92}}

\hypertarget{sauce-salmis}{%
\paragraph{Sauce Salmis}\label{sauce-salmis}}

\index{そーす@ソース!さるみ@---・サルミ} \index{さるみ@サルミ!そーす@
ソース・---} \index{sauce@sauce!salmis@--- Salmis}
\index{salmis@salmis!sauce salmis@Sauce ---}

ソースというよりはむしろクリ\footnote{coulis \textless{} couler
  クレ「流れる」から派生した語だが、料理用語とし
  ては、やや水分の多いピュレと理解するといい。ここでは二つの解釈が可
  能で、ひとつは\protect\hyperlink{}{ポタージュ・クリ}に近いという意味。もうひとつは
  「昔ながらのソース」の意。後者の場合、エスコフィエが「古典料理」と
  呼ぶ17、18世紀においてソースのことをクリと呼んでいたのを踏まえてい
  ると考えられる。}と呼んだほうがいいこのソースの作り方
はどんな場合も一点を除いて変わることがない。それは、このソースを合わせ
るジビエ(鳥)の種類によって、つまり普通に肉料理として扱えるジビエか、
肉断ち\footnote{小斉のこと。カトリックの習慣として(厳密な教義ではない)四旬節
  (復活祭までの46日間)や毎週金曜などに行なわれる、肉食を断つ行為の
  こと。}の際の食材として扱えるもの\footnote{ある種の水鳥はイルカと同様に魚と同等のものと見做され、小斉の場
  合にも食材として認められていた。具体的にはハシヒロ鴨、オナガ鴨、サ
  ルセル鴨など。もっとも、水鳥を肉断ちの際の食材として扱うというのは
  一種の詭弁ともいえなくないわけで、このソースを作る際に\protect\hyperlink{sauce-espagnole-maigre}{魚料理用ソー
  ス・エスパニョル}をベースとした\protect\hyperlink{sauce-demi-glace}{ソース・
  ドゥミグラス}を使うとは考え難く、本文にあるよう
  にフォンの代用としてマッシュルームの茹で汁を用いるという指示を守るだ
  けで、厳密に小斉の料理として成立するレシピと言えるかは疑問の残ると
  ころだ。}かで、どんな液体を用いるかと いうことだけだ。

細かく刻んだ\protect\hyperlink{mirepoix}{ミルポワ}150
gをバターでじっくり色付くまで炒め
る。そこに、その料理で用いているジビエの手羽と腿の皮、ガラを細かく刻ん
で加える。

白ワイン3
dlを注ぎ、\untiers{}量まで煮詰める。\protect\hyperlink{sauce-demi-glace}{ソース・ドゥミグラ
ス}8 dlを加えて、約45分間弱火で煮込む。漉し器で漉す
が、その際に香味野菜とガラのエキス\footnote{原文quintessenceカンテサンス。本来の意味は錬金術でいう「第五元
  素」。16世紀の作家フランソワ・ラブレーは存命当時、自著を筆名「カン
  テサンス抽出をなし遂げたアルコフリバス師」で出版していた時期がある。
  もっとも、このカンテサンスという語自体は中世以来、料理において「エ
  キス」「美味しさの本質」程度の意味でよく用いられた。}が得られるよう、強く押し絞って
やること。こうして出来たクリを、このソースを合わせる鳥と同種のものでとっ
たフォン4 dlで薄める。

ジビエが肉断ちの食材と見做されるもので、なおかつそれを厳格に守って作ら
なければならない場合は、このときフォンの代わりにマッシュルームの茹で汁を
用いればいい。

約45分〜1時間、弱火にかけて浮いてくる不純物を丁寧に取り除いてやる\footnote{dépouiller
  デプイエ。現代ではécumerエキュメの語を用いる現場が多 い。}。
さらにソースを\deuxtiers{}以下の量になるまで煮詰める。これにマッシュルー
ムの茹で汁とトリュフエッセンスを適量加えて丁度いい濃度になるよう調製する。

布で漉し、軽くバターを加えて仕上げる\footnote{原文は légèrement
  beurrerでありそのまま訳したが、現代の調理現場 ではmonter au beurre
  バターでモンテする、という表現がよく使われる。}。

\hypertarget{ux539fux6ce8-9}{%
\subparagraph{【原注】}\label{ux539fux6ce8-9}}

仕上げの際に、ソース1 Lあたりバター約50 gを加えるが、これは任意。

\maeaki

\hypertarget{ux30bdux30fcux30b9ux30c8ux30ebux30c1ux30e593}{%
\subsubsection[ソース・トルチュ]{\texorpdfstring{ソース・トルチュ\footnote{tortue
  トルチュは海亀のこと。古くは海亀料理用のソースだったが、
  19世紀以降は仔牛の頭肉料理に合わせるのが一般的になった。}}{ソース・トルチュ}}\label{ux30bdux30fcux30b9ux30c8ux30ebux30c1ux30e593}}

\hypertarget{sauce-tortue}{%
\paragraph{Sauce Tortue}\label{sauce-tortue}}

\index{そーす@ソース!とるちゅ@---・トルチュ} \index{とるちゅ@トルチュ!
そーす@ソース・---} \index{sauce@sauce!tortue@--- Tortue}
\index{tortue@tortue!sauce tortue@Sauce ---}

2\undemi{}
Lの\protect\hyperlink{fonds-de-veau-brun}{仔牛のフォン}を鍋で沸かし、セージ3
g、マジョラム1 g、ローズマリー1 g、バジル2 g、タイム1 g、ローリエの葉1
g、パセリの葉1つまみ、マッシュルームの切りくず25 gを投入する。蓋をして
25分間煎じる。こうして煎じた液体を漉す2分前に大粒のこしょう4個を加える。

布で漉し、\protect\hyperlink{sauce-demi-glace}{ソース・ドゥミグラス}7
dlに\protect\hyperlink{sauce-tomate}{トマトソー ス}3
dlを合わせたものに、上記で煎じた液体を、風味が際立
つ程度に適量加える。\troisquarts{}量まで煮詰め、布で漉す。仕上げにマデ
ラ酒1 dlとトリュフエッセンス少々を加え、さらにカイエンヌで風味を引き締
める。

\hypertarget{ux539fux6ce8-10}{%
\subparagraph{【原注】}\label{ux539fux6ce8-10}}

このソースはある程度まとまった量で作る必要がある。カイエンヌを使う指示
があるからだ。それでも、カイエンヌはとても気をつけて量を加減する必要が
ある\footnote{フランス料理において(というよりも一般的なフランス人にとって)
  は、唐辛子の辛さは嫌われる傾向が非常に強い。}。

\maeaki

\hypertarget{ux30bdux30fcux30b9ux30f4ux30cdux30beux30f395}{%
\subsubsection[ソース・ヴネゾン]{\texorpdfstring{ソース・ヴネゾン\footnote{ノロ鹿chevreuilや猪sanglierなどの大型ジビエのこと。なおニホンジ
  カやエゾジカはcerfに分類され、フランス料理の食材としてはあまり高く
  評価されない傾向がある。}}{ソース・ヴネゾン}}\label{ux30bdux30fcux30b9ux30f4ux30cdux30beux30f395}}

\hypertarget{sauce-venaison}{%
\paragraph{Sauce Venaison}\label{sauce-venaison}}

完全に仕上げた「\protect\hyperlink{sauce-poivrade-pour-gibier}{ジビエ用ソース・ポワヴラー
ド}」\troisquarts{} Lに、\protect\hyperlink{}{グロゼイユのジュ
レ}大さじ3杯強を生クリーム1dlで溶いてから加える。

グロゼイユのジュレと生クリームを加えるのは、鍋を火から外して、提供直前
にすること。

\ldots{}\ldots{}大型ジビエ料理用。

\maeaki

\hypertarget{ux8d64ux30efux30a4ux30f3ux30bdux30fcux30b9}{%
\subsubsection{赤ワインソース}\label{ux8d64ux30efux30a4ux30f3ux30bdux30fcux30b9}}

\hypertarget{sauce-au-vin-rouge}{%
\paragraph{Sauce au Vin rouge}\label{sauce-au-vin-rouge}}

\index{そーす@ソース!あかわいん@赤ワイン---} \index{あかわいん@赤ワイ
ン!そーす@---ソース} \index{sauce@sauce!vin rouge@--- au Vin rouge}
\index{vin@vin!sauce au vin rouge@Sauce au Vin rouge}

「赤ワインソース」という場合、煮詰めてからブールマニエでとろみを付ける
ブルゴーニュ風の仕立てか、魚を煮るのに用いた赤ワインを使うことが特徴で
ある「ソース・マトロット」のいずれかから派生したものなのは言うまでもな
い。もっとも、後者の場合はワインの風味は失われてしまっていてソースの水
気と味付けの意味しか持っていないと言える。

両者どちらもまさしく「赤ワインソース」だが、\protect\hyperlink{sauce-bourguignonne}{ブルゴーニュ風ソー
ス}と\protect\hyperlink{sauce-matelote}{ソース・マトロット}はそれ
ぞれ作り方も用途も違うから別々の名称として、この「茶色い派生ソース」の
節で説明した。

筆者としては、本当の「赤ワインソース」は以下のように作るものと考えてい
る。

ごく細かく刻んだ標準的な\protect\hyperlink{mirepoix}{ミルポワ}125
gをバターで炒める。良 質の赤ワイン\undemi{}
Lを注ぐ。半量になるまで煮詰める。つぶしたにんに
く1片、\protect\hyperlink{sauce-espagnole}{ソース・エスパニョル}7\undemi{}
dlを加え、12〜
15分、火ひかけて浮いてくる不純物を丁寧に取り除く\footnote{dépouiller
  デプイエ ≒ écumer エキュメ。}。

布で漉し、バター100 gとアンチョビエッセンス小さじ1杯、カイエンヌ1つま
みを加えて仕上げる。

\ldots{}\ldots{}魚料理用ソース。

\maeaki

\hypertarget{ux30bdux30fcux30b9ux30b6ux30f3ux30acux30e997-a}{%
\subsubsection[ソース・ザンガラ
A]{\texorpdfstring{ソース・ザンガラ\footnote{もとの語形はzingaro
  ザンガロ、またはヂンガロ。ジプシー、ボヘミ
  アンの意。料理ではパプリカ粉末やカイエンヌを用いたものに命名される
  ことが多い。}
A}{ソース・ザンガラ A}}\label{ux30bdux30fcux30b9ux30b6ux30f3ux30acux30e997-a}}

\hypertarget{sauce-zingara-a}{%
\paragraph{Sauce Zingara A}\label{sauce-zingara-a}}

\index{そーす@ソース!さんからa@---・ザンガラ A} \index{さんから@ザンガ
ラ!そーすa@ソース・--- A} \index{しぷしーふう@ジプシー風!そーすa@ソー
ス・ザンガラ A} \index{sauce@sauce!zingaraa@--- Zingara A}
\index{zingara@Zingara!saucea@Sauce --- A}

このソースは古典料理の\protect\hyperlink{}{ガルニチュール・ザンガラ}とはまったく関係が
ない。むしろイギリス料理に由来し、本書でもイギリス風ソースの節において
似たようなものはいくつも採り上げている。

ヴィネガー2\undemi{} dlにエシャロットのみじん切り大さじ1杯を加えて半量
になるまで煮詰める。\protect\hyperlink{jus-de-veau-lie}{茶色いジュ}7
dlを注ぎ、バターで
揚げたパンの身160gを加える。弱火で5〜6分間煮る。パセリのみじん切り大さ
じ1杯とレモン\undemi{}個分の搾り汁を加えて仕上げる。

\maeaki

\hypertarget{ux30bdux30fcux30b9ux30b6ux30f3ux30acux30e9-b}{%
\subsubsection{ソース・ザンガラ
B}\label{ux30bdux30fcux30b9ux30b6ux30f3ux30acux30e9-b}}

\hypertarget{sauce-zingara-b}{%
\paragraph{Sauce Zingara B}\label{sauce-zingara-b}}

白ワイン3 dlとマッシュルームの茹で汁3 dlを合わせて\untiers{}量になるまで
煮詰める。

\protect\hyperlink{sauce-demi-glace}{ソース・ドゥミグラス}4
dlと\protect\hyperlink{sauce-tomate}{トマトソー ス}2\undemi{}
dl、\protect\hyperlink{fonds-blanc}{白いフォン}1 dlを注ぐ。
浮いてくる不純物を徹底的に取り除きながら5〜6分火かける。

仕上げに、カイエンヌ1つまみで風味を引き締め、太さ1〜2 mmの千切りにした
\footnote{julienne ジュリエンヌ。}ハム(脂身のないところ)と赤く漬けた舌肉70
gおよびマッシュルーム 50 g、トリュフ 30 gを加える。

\ldots{}\ldots{}仔牛料理、鶏料理用。
\end{recette}\newpage





%%%Chapitre II. Garnitures







{\printindex}

\end{document}
