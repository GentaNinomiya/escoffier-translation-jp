% !TEX program = LuaLaTeX
%
%%%%PDFのつくりかた%%%
%
%%%%エスコフィエ『料理の手引き』全注解のTeX原稿はLuaLaTeXで書かれています。
%%%%
%%%% 1. Githubのリポジトリをクローンまたはフォークさせてください。これ
%%%% でリポジトリ(つまり原稿全部)があなたの手中にあります。
%
%%%% 2. TeXのプログラムをパソコンにインストールしてください。インストール方法は
%%%% https://texwiki.texjp.org/?TeX%20Live
%%%% をご参照ください。
%
%%%% 3. フォントの設定をいじらない場合はMoboMogaフォントをインストールしてください。
%%%% http://yozvox.web.fc2.com/82A882B782B782DF8374834883938367.html
%
%%%% 4. Macの場合は自分のホームディレクトリに .latexmkrcというファイルを作成します。
%%%% https://texwiki.texjp.org/?Latexmk
%%%% を参考にしてください(コピペで大丈夫だと思います)
%
%%%% 5. 上記サイトにあるようにLuaLaTeXを使いますので、ターミナル(macの
%%%% 場合)を開いて、このファイルのあるディレクトリに移動し
%
%%%%     latexmk -gg -pdflua escoffier-le-guide-culinaire.tex
%
%%%% と打ち込むと、あとは勝手にPDFを生成してくれる(はず)です。質問など
%%%% は、たとえ僕が存命中でも、元気であっても、いっさいおうけできません。
%%%% すべて自己責任でお願いします。
%
%%%% あとは改訳するもよし、版面設定を変更するもよし、好きになさってくだ
%%%% さい。知識と能力さえあれば、さして時間もかからずに、お好みのものが
%%%% 出来るでしょう。ちなみに注釈なしバージョンを作るには、この下の
%% \renewcommand{\footnote}[1]{}%注釈なしバージョンを作るにはこの行頭の%を消す
%%%% というところを書いてあるとおりにすれば簡単にいまのA4版の設定で注釈
%%%% なしのPDFで出来るはずです。
%
%%%% ただし、2018年現在、日本国の著作権法にもとづいて、訳文と注釈につい
%%%% てのすべての権利を訳、注釈者である五島学が保持しています。したがっ
%%%% て、上記のPDF作成作業や改変はあくまでも「個人の私的利用の範囲」に
%%%% 留まるものとしてください。無断配布は禁止となります。商用利用の場合
%%%% には著作権法にもとづいて適正なロイヤリティをお支払いいただきます。
%
%
%
%%% Preamble読み込み
\documentclass[twoside,14Q,a4paper,openany]{ltjsbook}

\usepackage{amsmath}
\usepackage{amssymb}
\usepackage[no-math]{fontspec}
\usepackage{geometry}
\usepackage{luaotfload}
\usepackage{graphicx}

%% 欧文フォント設定
% Libertine/Biolinum
\setmainfont[Ligatures=Historic,Scale=1.0]{Linux Libertine O}
\setsansfont[Ligatures=TeX, Scale=MatchLowercase]{Linux Biolinum O} 
%\usepackage{libertine}
\usepackage{unicode-math}
\setmathfont[Scale=1.2]{libertinusmath-regular.otf}
%\unimathsetup{math-style=ISO,bold-style=ISO}
%\setmathfont{xits-math.otf}
%\setmathfont{xits-math.otf}[range={cal,bfcal},StylisticSet=1]




%% Garamond
%\usepackage{ebgaramond-maths}
%\setmainfont[Ligatures=Historic,Scale=1.1]{EB Garamond}%fontspecによるフォント設定

%\usepackage{qpalatin}%palatino

%\setmainfont[Ligatures=Historic,Scale=MatchLowercase]{Tex Gyre Schola}
%\setmainfont[Ligatures=Historic,Scale=MatchLowercase]{Tex Gyre Pagella}
%\setsansfont[Scale=MatchLowercase]{TeX Gyre Heros}  % \sffamily のフォント
%\setsansfont[Scale=MatchLowercase]{TeX Gyre Adventor}  % \sffamily のフォント

%\setmonofont[Scale=MatchLowercase]{Inconsolata}       % \ttfamily のフォント

%\usepackage[cmintegrals,cmbraces]{newtxmath}%数式フォント

\usepackage{luatexja}
\usepackage{luatexja-fontspec}
%\ltjdefcharrange{8}{"2000-"2013, "2015-"2025, "2027-"203A, "203C-"206F}
%\ltjsetparameter{jacharrange={-2, +8}}
\usepackage{luatexja-ruby}

%%%%和文フォント設定
%\usepackage[CharacterWidth=AlternateProportional,sourcehan,bold,jis,jis2004,expert,deluxe]{luatexja-preset}%Adobe源ノ明朝、ゴチ
%\usepackage[hiragino-pron,jis,bold,jis2004,expert,deluxe]{luatexja-preset}
%\usepackage[YokoFeatures={JFM=prop,PKana=On},ipaex,bold,jis,jis2004,expert,deluxe]{luatexja-preset}
%\usepackage[YokoFeatures={JFM=prop,PKana=On},ipaex,bold,jis,jis2004,expert,deluxe]{luatexja-preset}
%\usepackage[yu-osx,bold]{myluatexja-preset}
%\usepackage[moga-mogo-ex,bold]{myluatexja-preset}

\newopentypefeature{PKana}{On}{pkna} % "PKana" and "On" can be arbitrary string
   \setmainjfont[%
%   YokoFeatures={JFM=prop,PKana=On},
% %   CharacterWidth=AlternateProportional,
% % %       Kerning=On,
         BoldFont={ MoboGoB },%
         ItalicFont={ MoboGoB },%
         BoldItalicFont={ MoboGoExB }%
%         ]{ MogaHMin }
          ]{ IPAExMincho }
     \setsansjfont[%
%        YokoFeatures={JFM=prop,PKana=On},
% %        CharacterWidth=AlternateProportional,
% % %       Kerning=On,
         BoldFont={ MoboGoB },%
         ItalicFont={ MoboGoB },%
         BoldItalicFont={ MoboGoExB }%
         % ]{ MoboGo}
          ]{ IPAExGothic }
%  %%%% 和文仮名プロプーショナルここまで
% %\ltjsetparameter{jacharrange={-2}}%キリル文字%引数に-3を付けるとギリシア文字も可能になるが、%三点リーダーも欧文化されてしまうので注意%


\renewcommand{\bfdefault}{bx}%和文ボールドを有効にする
\renewcommand{\headfont}{\gtfamily\sffamily\bfseries}%和文ボールドを有効にする
%\addfontfeature{Fractions=On}


\defaultfontfeatures[\rmfamily]{Scale=1.2}%効いていない様子
\defaultjfontfeatures{Scale=0.92487}%和文フォントのサイズ調整。デフォルトは 0.962212 倍%ltjsclassesでは不要?
%\defaultjfontfeatures{Scale=0.962212}
%\usepackage{libertineotf}%linux libertine font %ギリシア語含む
%\usepackage[T1]{fontenc}
%\usepackage[full]{textcomp}
%\usepackage[osfI,scaled=1.0]{garamondx}
%\usepackage{tgheros,tgcursor}
%\usepackage[garamondx]{newtxmath}
\usepackage{xfrac}

\usepackage{layout}

%% レイアウト調整(A4Paper,13Q,onside,escoffierltjsbook.cls) 
%%
\setlength{\hoffset}{0\zw}
\setlength{\oddsidemargin}{0\zw}
\setlength{\evensidemargin}{\oddsidemargin}
\setlength{\fullwidth}{45\zw}
\setlength{\textwidth}{45\zw}%%ltjsclassesのみ有効
%\setlength{\fullwidth}{159mm}
%\setlength{\textwidth}{159mm}
\setlength{\marginparsep}{0pt}
\setlength{\marginparwidth}{0pt}
\setlength{\footskip}{0pt}
\setlength{\voffset}{-17mm}
\setlength{\textheight}{265mm}
\setlength{\parskip}{0pt}
%\setlength{\parindent}{0pt}
%%%ベースライン調整
%\ltjsetparameter{yjabaselineshift=0pt,yalbaselineshift=-.75pt}


%\usepackage{fancyhdr}

\usepackage{setspace}
\setstretch{1.0}




%文字サイズ、見出しなどの再定義
\makeatletter
%\renewcommand{\large}{\jsc@setfontsize\large\@xipt{14}}
%\renewcommand{\Large}{\jsc@setfontsize\Large{13}{15}}

\newcommand{\medlarge}{\fontsize{11}{13}\selectfont}
\newcommand{\medsmall}{\fontsize{9.23}{9.5}\selectfont}
\newcommand{\twelveq}{\jsc@setfontsize\twelveq{9.230769}{9.75}\selectfont}
\newcommand{\fourteenq}{\jsc@setfontsize\fourteenq{10.7692}{13}\selectfont}
\newcommand{\fifteenq}{\jsc@setfontsize\fifteenq{11.53846}{14}\selectfont}

\renewcommand{\chapter}{%
  \if@openleft\cleardoublepage\else
  \if@openright\cleardoublepage\else\clearpage\fi\fi
  \plainifnotempty % 元: \thispagestyle{plain}
  \global\@topnum\z@
  \if@english \@afterindentfalse \else \@afterindenttrue \fi
  \secdef
    {\@omit@numberfalse\@chapter}%
    {\@omit@numbertrue\@schapter}}
\def\@chapter[#1]#2{%
  \ifnum \c@secnumdepth >\m@ne
    \if@mainmatter
      \refstepcounter{chapter}%
      \typeout{\@chapapp\thechapter\@chappos}%
      \addcontentsline{toc}{chapter}%
        {\protect\numberline
        % {\if@english\thechapter\else\@chapapp\thechapter\@chappos\fi}%
        {\@chapapp\thechapter\@chappos}%
        #1}%
    \else\addcontentsline{toc}{chapter}{#1}\fi
  \else
    \addcontentsline{toc}{chapter}{#1}%
  \fi
  \chaptermark{#1}%
  \addtocontents{lof}{\protect\addvspace{10\jsc@mpt}}%
  \addtocontents{lot}{\protect\addvspace{10\jsc@mpt}}%
  \if@twocolumn
    \@topnewpage[\@makechapterhead{#2}]%
  \else
    \@makechapterhead{#2}%
    \@afterheading
  \fi}
\def\@makechapterhead#1{%
  \vspace*{0\Cvs}% 欧文は50pt
  {\parindent \z@ \centering \normalfont
    \ifnum \c@secnumdepth >\m@ne
      \if@mainmatter
        \huge\headfont \@chapapp\thechapter\@chappos%変更
        \par\nobreak
        \vskip \Cvs % 欧文は20pt
      \fi
    \fi
    \interlinepenalty\@M
    \huge \headfont #1\par\nobreak
    \vskip 1\Cvs}} % 欧文は40pt%変更

\renewcommand{\section}{%
    \if@slide\clearpage\fi
    \@startsection{section}{1}{\z@}%
    {\Cvs \@plus.5\Cdp \@minus.2\Cdp}% 前アキ
    % {.5\Cvs \@plus.3\Cdp}% 後アキ
    {.5\Cvs}
    {\normalfont\Large\headfont\bfseries\centering}}%変更

\renewcommand{\subsection}{\@startsection{subsection}{2}{\z@}%
    {\Cvs \@plus.5\Cdp \@minus.2\Cdp}% 前アキ
    % {.5\Cvs \@plus.3\Cdp}% 後アキ
    {.5\Cvs}
    {\normalfont\large\headfont\bfseries\centering}} %変更


\renewcommand{\subsubsection}{\@startsection{subsubsection}{3}{\z@}%
    {.25\Cvs \@plus.5\Cdp \@minus.5\Cdp}%変更
    {\if@slide .5\Cvs \@plus.3\Cdp \else \z@ \fi}%
    {\normalfont\medlarge\headfont\leftskip -1\zw}}

\renewcommand{\paragraph}{\@startsection{paragraph}{4}{\z@}%
    {0.5\Cvs \@plus.5\Cdp \@minus.2\Cdp}%
    % {\if@slide .5\Cvs \@plus.3\Cdp \else -1\zw\fi}% 改行せず 1\zw のアキ
    {1sp}%後アキ
    {\normalfont\normalsize\headfont}}
\renewcommand{\subparagraph}{\@startsection{subparagraph}{5}{\z@}%
    {\z@}{\if@slide .5\Cvs \@plus.3\Cdp \else -.5\zw\fi}%
    {\normalfont\normalsize\headfont\hskip-.5\zw\noindent}}  



\newcommand{\frsec}[1]{\vspace*{-1\zw}\begin{center}\normalfont\hspace*{1\zw}\headfont\Large\scshape#1\normalfont\normalsize\end{center}\vspace{0.5\zw}}

\newcommand{\frsecb}[1]{\vspace*{-1\zw}\begin{center}\hspace{1\zw}\normalfont\headfont\large\scshape#1\normalfont\normalsize\end{center}\vspace{0.5\zw}}

%\newcounter{frsub}[subsubsection]
%\newcommand{\frsub}{\@startsection{frsub}{6}{\z@}%
%  {1sp}{1sp}%
%  {\normalfont\normalsize\bfseries\baselineskip-.8ex\leftskip-1\zw}}
%\let\frsubmark\@gobble
%\newcommand*{\l@frsub}{%
%          \@tempdima\jsc@tocl@width \advance\@tempdima 16.183\zw
%          \@dottedtocline{7}{\@tempdima}{6.5\zw}}
%\renewcommand{\thefrsub}{6}
%\let\frsub\paragraph

\newenvironment{frsubenv}{\begin{spacing}{0.2}\setlength{\leftskip}{-1\zw}\bfseries}{\end{spacing}\normalfont\normalsize\setlength{\leftskip}{0pt}}
\newcommand{\frsub}[1]{\begin{frsubenv}#1\end{frsubenv}\par\vspace{1.1\zw}}

%\newcommand{\frsub}[1]{\vskip -.8ex\hskip -1\zw\textbf{#1}\leftskip0pt}
%\newcommand{\frsub}{\@startsection{frsub}{6}{\z@}%
%   {-1\zw}% 改行せず 1\zw のアキ
%   {-1\zw}%後アキ     
%   {\normalfont\normalsize\bfseries\leftskip -1\zw\baselineskip -.5ex}}%normalsizeから変更
%\newcommand*{\l@frsub}{%
%          \@tempdima\jsc@tocl@width \advance\@tempdima 16.183\zw
%          \@dottedtocline{5}{\@tempdima}{6.5\zw}}

\makeatother

%%% 脚注番号のページ毎のリセットと脚注位置の調整
\makeatletter

\usepackage[bottom,perpage,stable]{footmisc}%
%\setlength{\skip\footins}{4mm plus 2mm}
%\usepackage{footnpag}
\renewcommand\@makefntext[1]{%
  \advance\leftskip 1.5\zw
  \parindent 1\zw
  \noindent
  \llap{\@thefnmark\hskip0.5\zw}#1}


\let\footnotes@ve=\footnote
\def\footnote{\inhibitglue\footnotes@ve}
\let\footnotemarks@ve=\footnotemark
%\def\footnotemark{\inhibitglue\footnotemarks@ve}
\renewcommand{\footnotemark}{\footnotemarks@ve}%変更
% %\def\thefootnote{\ifnum\c@footnote>\z@\leavevmode\lower.5ex\hbox{(}\@arabic\c@footnote\hbox{)}\fi}
\renewcommand{\thefootnote}{\ifnum\c@footnote>\z@\leavevmode\hbox{}\@arabic\c@footnote\hbox{)}\fi}
%\makeatletter
% \@addtoreset{footnote}{page}
% \makeatother
%\usepackage{dblfnote}
%\usepackage[bottom,perpage]{footmisc}

\makeatother

%subsubsectionに連番をつける
%\usepackage{remreset}

\renewcommand{\thechapter}{}
\renewcommand{\thesection}{}
\renewcommand{\thesubsection}{}
\renewcommand{\thesubsubsection}{}
\renewcommand{\theparagraph}{}


%\makeatletter
%\@removefromreset{subsubsection}{subsection}
%\def\thesubsubsection{\arabic{subsubsection}.}
%\newcounter{rnumber}
%\renewcommand{\thernumber}{\refstepcounter{rnumber} }

\renewcommand{\prepartname}{\if@english Part~\else {}\fi}
\renewcommand{\postpartname}{\if@english\else {}\fi}
\renewcommand{\prechaptername}{\if@english Chapter~\else {}\fi}
\renewcommand{\postchaptername}{\if@english\else {}\fi}
\renewcommand{\presectionname}{}%  第
\renewcommand{\postsectionname}{}% 節

%リスト環境
\def\tightlist{\itemsep1pt\parskip0pt\parsep0pt}%pandoc対策

\makeatletter
  \parsep   = 0pt
  \labelsep = .5\zw
  \def\@listi{%
     \leftmargin = 0pt \rightmargin = 0pt
     \labelwidth\leftmargin \advance\labelwidth-\labelsep
     \topsep     = 0pt%\baselineskip
     %\topsep -0.1\baselineskip \@plus 0\baselineskip \@minus 0.1 \baselineskip
     \partopsep  = 0pt \itemsep       = 0pt
     \itemindent = -.5\zw \listparindent = 0\zw}
  \let\@listI\@listi
  \@listi
  \def\@listii{%
     \leftmargin = 1.8\zw \rightmargin = 0pt
     \labelwidth\leftmargin \advance\labelwidth-\labelsep
     \topsep     = 0pt \partopsep     = 0pt \itemsep   = 0pt
     \itemindent = 0pt \listparindent = 1\zw}
  \let\@listiii\@listii
  \let\@listiv\@listii
  \let\@listv\@listii
  \let\@listvi\@listii
\makeatother




%レシピ本文
\usepackage{multicol}
\setlength{\columnsep}{3\zw}
% \setlength{\columnwidth}{24\zw}
%本文ヨリ小%\small
%\newenvironment{recette}{\setlength{\parindent}{0pt}\begin{small}\begin{spaceing}{0.8}\begin{multicols}{2}}{\end{multicols}\end{spacing}\end{small}}
%本文やや小%\medsmall
%\newenvironment{recette}{\setlength{\parindent}{0pt}\begin{medsmall}\begin{spacing}{0.75}\begin{multicols}{2}}{\end{multicols}\end{spacing}\end{medsmall}}
%本文ナミ(無指定)
\newenvironment{recette}{\setlength{\parindent}{0pt}\begin{spacing}{0.8}\begin{multicols}{2}}{\end{multicols}\end{spacing}}




%% %%%%%%行取りマクロ
% \makeatletter
% \ifx\Cht\undefined
%  \newdimen\Cht\newdimen\Cdp
%  \setbox0\hbox{\char\jis"2121}\Cht=\ht0\Cdp=\dp0\fi
% \catcode`@=11
% \long\def\linespace#1#2{\par\noindent
%   \dimen@=\baselineskip
%   \multiply\dimen@ #1\advance\dimen@-\baselineskip
%   \advance\dimen@-\Cht\advance\dimen@\Cdp
%   \setbox0\vbox{\noindent #2}%
%   \advance\dimen@\ht0\advance\dimen@-\dp0%
%   \vtop to\z@{\hbox{\vrule width\z@ height\Cht depth\z@
%    \raise-.5\dimen@\hbox{\box0}}\vss}%
%   \dimen@=\baselineskip
%   \multiply\dimen@ #1\advance\dimen@-2\baselineskip
%   \par\nobreak\vskip\dimen@
%   \hbox{\vrule width\z@ height\Cht depth\z@}\vskip\z@}
% \catcode`@=12
% \setlength{\parskip}{0pt}
% \setlength{\topskip}{\Cht}
% \setlength{\textheight}{43\baselineskip}
% \addtolength{\textheight}{1\zh}
% \makeatother
 
%%%%%%%%%%%%失敗%%%%%%%%%%%%
%\let\formule\subsubsection
%\renewcommand{\subsubsection}[1]{\linespace{1}{\formule#1}}
%%%%%%%%%%%%失敗%%%%%%%%%%%%






% PDF/X-1a
% \usepackage[x-1a]{pdfx}
% \Keywords{pdfTeX\sep PDF/X-1a\sep PDF/A-b}
% \Title{Sample LaTeX input file}
% \Author{LaTeX project team}
% \Org{TeX Users Group}
% \pdfcompresslevel=0
%\usepackage[cmyk]{xcolor}

%biblatex
%\usepackage[notes,strict,backend=biber,autolang=other,%
%                   bibencoding=inputenc,autocite=footnote]{biblatex-chicago}
%\addbibresource{hist-agri.bib}
\let\cite=\autocite

% % % % 
\date{}



\makeatletter
\renewenvironment{theindex}{% 索引を3段組で出力する環境
    \if@twocolumn
      \onecolumn\@restonecolfalse
    \else
      \clearpage\@restonecoltrue
    \fi
    \columnseprule.4pt \columnsep 2\zw
    \ifx\multicols\@undefined
      \twocolumn[\@makeschapterhead{\indexname}%
      \addcontentsline{toc}{chapter}{\indexname}]%変更点
    \else
      \ifdim\textwidth<\fullwidth
        \setlength{\evensidemargin}{\oddsidemargin}
        \setlength{\textwidth}{\fullwidth}
        \setlength{\linewidth}{\fullwidth}
        \begin{multicols}{3}[\chapter*{\indexname}
	\addcontentsline{toc}{chapter}{\indexname}]%変更点%
      \else
        \begin{multicols}{3}[\chapter*{\indexname}
	\addcontentsline{toc}{chapter}{\indexname}]%変更点%
      \fi
    \fi
    \@mkboth{\indexname}{\indexname}%
    \plainifnotempty % \thispagestyle{plain}
    \parindent\z@
    \parskip\z@ \@plus .3\p@\relax
    \let\item\@idxitem
    \raggedright
    \footnotesize\narrowbaselines
  }{
    \ifx\multicols\@undefined
      \if@restonecol\onecolumn\fi
    \else
      \end{multicols}
    \fi
    \clearpage
  }
\makeatother



%\renewcommand{\ldots}{…}
\usepackage{makeidx}
\makeindex


\usepackage[unicode=true]{hyperref}
%\usepackage{pxjahyper}
\hypersetup{breaklinks=true,%
             bookmarks=true,%
             pdfauthor={五島 学},%
             pdftitle={エスコフィエ『料理の手引き』全注解},%
             colorlinks=true,%
             citecolor=blue,%
             urlcolor=cyan,%
             linkcolor=magenta,%
             bookmarksdepth=subsubsection,%
             pdfborder={0 0 0}}


% \hypersetup{
%     pdfborderstyle={/S/U/W 1}, % underline links instead of boxes
%     linkbordercolor=red,       % color of internal links
%     citebordercolor=green,     % color of links to bibliography
%     filebordercolor=magenta,   % color of file links
%     urlbordercolor=cyan        % color of external links
% }

\urlstyle{same}
%\renewcommand*{\label}[1]{\hypertarget{#1}{}}
%\renewcommand{\hyperlink}[2]{\hyperref[#1]{#2}}

\renewcommand{\ldots}{\noindent…}
%\usepackage{udline}
% \usepackage{ulem}
\usepackage{umoline}
\setlength{\UnderlineDepth}{2pt}
\let\ul\Underline

\newcommand{\maeaki}{}
%\newcommand{\maeaki}{\vspace{0.125\zw}}
%\newcommand{\maeaki}{\vspace{0.7\zw}}
%\newcommand{\maeaki}{\vspace{2.0\zw}}
%\newcommand{\maeaki}{\vspace{1.1\zw}}
%\newcommand{\maeaki}{\vspace{1.5\zw}}
%\newcommand{\maeaki}{\vspace{1.75\zw}}
%\newcommand{\maeaki}{\vspace{1.0mm}}                     
%\newcommand{\maeaki}{\vspace{2.2\zw}}
%\newcommand{\maeaki}{\vspace{-.25mm}}

%%分数の表記
\usepackage{xfrac}
\let\frac\sfrac
\newcommand{\undemi}{\hspace{.25\zw}$\sfrac{1}{2}$}
\newcommand{\untiers}{\hspace{.25\zw}$\sfrac{1}{3}$}
\newcommand{\deuxtiers}{\hspace{.25\zw}$\sfrac{2}{3}$}
\newcommand{\unquart}{\hspace{.25\zw}$\sfrac{1}{4}$}
\newcommand{\troisquarts}{\hspace{.25\zw}$\sfrac{3}{4}$}
\newcommand{\quatrequatrieme}{\hspace{.25\zw}$\sfrac{4}{4$}}
\newcommand{\uncinquieme}{\hspace{.25\zw}$\sfrac{1}{5}$}
\newcommand{\deuxcinquiemes}{\hspace{.25\zw}$\sfrac{2}{5}$}
\newcommand{\troiscinquiemes}{\hspace{.25\zw}$\sfrac{3}{5}$}
\newcommand{\quatrecinquiemes}{\hspace{.25\zw}$\sfrac{4}{5}$}
\newcommand{\unsixieme}{\hspace{.25\zw}$\sfrac{1}{6}$}
\newcommand{\cinqsixiemes}{\hspace{.25\zw}$\sfrac{5}{6}$}
\newcommand{\quatrequart}{\hspace{.25\zw}$\sfrac{4}{4}$}


%\renewcommand{\footnote}[1]{}%単に注釈なしバージョンを作るにはこの行頭の%を消す
%
%%%%以下、TeXにくわしくない方は触らないことをおすすめします。
% \documentclass[twoside,10Q,octavo,openany]{escoffierltjsbook}
\usepackage{amsmath}%数式
\usepackage{amssymb}
\usepackage[no-math]{fontspec}
\usepackage{geometry}
\usepackage{unicode-math}
\usepackage{xfrac}
\usepackage{luaotfload}
\usepackage{graphicx}

%%欧文フォント設定
\setmainfont[Ligatures=Historic,Scale=1.0]{Linux Libertine O}

%%Garamond
%\usepackage{ebgaramond-maths}
%\setmainfont[Ligatures=TeX,Scale=1.0]{EB Garamond}%fontspecによるフォント設定

%\usepackage{qpalatin}%palatino

%\setmainfont[Ligatures=TeX]{TeX Gyre Pagella}%ギリシャ語を用いる場合はこちら
%\setsansfont[Scale=MatchLowercase]{TeX Gyre Heros}  % \sffamily のフォント
\setsansfont[Ligatures=TeX, Scale=1]{Linux Biolinum O}     % Libertine/Biolinum
%\setmonofont[Scale=MatchLowercase]{Inconsolata}       % \ttfamily のフォント
%\unimathsetup{math-style=ISO,bold-style=ISO}
%\setmathfont{xits-math.otf}
%\setmathfont{xits-math.otf}[range={cal,bfcal},StylisticSet=1]

%\index{\usepackage}\usepackage[cmintegrals,cmbraces]{newtxmath}%数式フォント

\usepackage{luatexja}
\usepackage{luatexja-fontspec}
%\ltjdefcharrange{8}{"2000-"2013, "2015-"2025, "2027-"203A, "203C-"206F}
%\ltjsetparameter{jacharrange={-2, +8}}
\usepackage{luatexja-ruby}

%%%%和文仮名プロポーショナル
\usepackage[CharacterWidth=AlternateProportional,sourcehan,bold,jis,jis2004,expert,deluxe]{luatexja-preset}%Adobe源ノ明朝、ゴチ
%\usepackage[hiragino-pron,jis2004,expert,deluxe]{luatexja-preset}
%\usepackage[ipaex,jis2004,expert,deluxe]{luatexja-preset}
\newopentypefeature{PKana}{On}{pkna} % "PKana" and "On" can be arbitrary string
%  \setmainjfont[
%      YokoFeatures={JFM=prop},
%      CharacterWidth=AlternateProportional,
%      BoldFont={SourceHanSans-Medium},
%      ItalicFont={SourceHanSans-Regular},
%      BoldItalicFont={SourceHanSans-medium}
%  ]{SourceHanSerif-Regular}
%  \setsansjfont[
%      YokoFeatures={JFM=prop},
%      CharacterWidth=AlternateProportional,
%      BoldFont={SourceHanSans-Medium},
%      ItalicFont={SourceHanSans-Normal},
%      BoldItalicFont={SourceHanSans-Medium}
%  ]{SourceHanSans-Normal}
 %%%% 和文仮名プロプーショナルここまで
%\ltjsetparameter{jacharrange={-2}}%キリル文字%引数に-3を付けるとギリシア文字も可能になるが、%三点リーダーも欧文化されてしまうので注意%


\renewcommand{\bfdefault}{bx}%和文ボールドを有効にする
\renewcommand{\headfont}{\gtfamily\sffamily\bfseries}%和文ボールドを有効にする
%\addfontfeature{Fractions=On}


\defaultfontfeatures[\rmfamily]{Scale=1.2}%効いていない様子
\defaultjfontfeatures{Scale=0.92487}%和文フォントのサイズ調整。デフォルトは 0.962212 倍%ltjsclassesでは不要?
%\defaultjfontfeatures{Scale=0.962212}
%\usepackage{libertineotf}%linux libertine font %ギリシア語含む
%\usepackage[T1]{fontenc}
%\usepackage[full]{textcomp}
%\usepackage[osfI,scaled=1.0]{garamondx}
%\usepackage{tgheros,tgcursor}
%\usepackage[garamondx]{newtxmath}
\usepackage{xfrac}

\usepackage{layout}

% レイアウト調整(B5,14Q,onside,escoffierltjsbook.cls)
%\setlength{\voffset}{-1.5cm}
%\setlength{\textwidth}{\fullwidth}
%\setlength{\oddsidemargin}{-3mm}
%\setlength{\evensidemargin}{\oddsidemargin}
%\setlength{\textheight}{23cm}

%レイアウト調整(demy-octavo,escoffierltjsbook.cls)
%
\setlength{\hoffset}{-8mm}
\setlength{\oddsidemargin}{0pt}
\setlength{\evensidemargin}{-5mm}
%\setlength{\textwidth}{\fullwidth}%%ltjsclassesのみ有効
\setlength{\fullwidth}{112.5mm}
\setlength{\textwidth}{112.5mm}
\setlength{\marginparsep}{0pt}
\setlength{\marginparwidth}{0pt}
\setlength{\footskip}{0pt}
\setlength{\voffset}{-10mm}
\setlength{\textheight}{195mm}
\setlength{\parskip}{0pt}
%\setlength{\parindent}{0pt}
%%%ベースライン調整
%\ltjsetparameter{yjabaselineshift=0pt,yalbaselineshift=-.75pt}



\def\tightlist{\itemsep1pt\parskip0pt\parsep0pt}

%リスト環境
\makeatletter
  \parsep   = 0pt
  \labelsep = 1\zw
  \def\@listi{%
     \leftmargin = 0pt \rightmargin = 0pt
     \labelwidth\leftmargin \advance\labelwidth-\labelsep
     \topsep     = 0pt%\baselineskip
     %\topsep -0.1\baselineskip \@plus 0\baselineskip \@minus 0.1 \baselineskip
     \partopsep  = 0pt \itemsep       = 0pt
     \itemindent = 0pt \listparindent = 0\zw}
  \let\@listI\@listi
  \@listi
  \def\@listii{%
     \leftmargin = 2\zw \rightmargin = 0pt
     \labelwidth\leftmargin \advance\labelwidth-\labelsep
     \topsep     = 0pt \partopsep     = 0pt \itemsep   = 0pt
     \itemindent = 0pt \listparindent = 1\zw}
  \let\@listiii\@listii
  \let\@listiv\@listii
  \let\@listv\@listii
  \let\@listvi\@listii
\makeatother


%\usepackage{fancyhdr}

\usepackage{setspace}
\setstretch{1.1}


%% %%%%%%行取りマクロ
% \makeatletter
% \ifx\Cht\undefined
%  \newdimen\Cht\newdimen\Cdp
%  \setbox0\hbox{\char\jis"2121}\Cht=\ht0\Cdp=\dp0\fi
% \catcode`@=11
% \long\def\linespace#1#2{\par\noindent
%   \dimen@=\baselineskip
%   \multiply\dimen@ #1\advance\dimen@-\baselineskip
%   \advance\dimen@-\Cht\advance\dimen@\Cdp
%   \setbox0\vbox{\noindent #2}%
%   \advance\dimen@\ht0\advance\dimen@-\dp0%
%   \vtop to\z@{\hbox{\vrule width\z@ height\Cht depth\z@
%    \raise-.5\dimen@\hbox{\box0}}\vss}%
%   \dimen@=\baselineskip
%   \multiply\dimen@ #1\advance\dimen@-2\baselineskip
%   \par\nobreak\vskip\dimen@
%   \hbox{\vrule width\z@ height\Cht depth\z@}\vskip\z@}
% \catcode`@=12
% \setlength{\parskip}{0pt}
% \setlength{\topskip}{\Cht}
% \setlength{\textheight}{43\baselineskip}
% \addtolength{\textheight}{1\zh}
% \makeatother
 
%%%%%%%%%%%%失敗%%%%%%%%%%%%
%\let\formule\subsubsection
%\renewcommand{\subsubsection}[1]{\linespace{1}{\formule#1}}
%%%%%%%%%%%%失敗%%%%%%%%%%%%



%レシピ本文
\usepackage{multicol}
\setlength{\columnsep}{2.7\zw}
%\setlength{\columnwidth}{24\zw}	
\newenvironment{recette}{\setlength{\parindent}{0pt}\begin{small}\begin{spacing}{1.0}\begin{multicols}{2}}{\end{multicols}\end{spacing}\end{small}}

%\newenvironment{recette}{\setlength{\parindent}{0pt}\begin{normalsize}\begin{spacing}{1.0}\begin{multicols}{2}}{\end{multicols}\end{spacing}\end{normalsize}}

%\newenvironment{recette}{\setlength{\parindent}{0pt}\begin{normalsize}\begin{multicols}{2}}{\end{multicols}\end{normalsize}}



%subsubsectionに連番をつける
%\usepackage{remreset}

\renewcommand{\thechapter}{}
\renewcommand{\thesection}{}
\renewcommand{\thesubsection}{}
\renewcommand{\thesubsubsection}{}
\renewcommand{\theparagraph}{}

%\makeatletter
%\@removefromreset{subsubsection}{subsection}
%\def\thesubsubsection{\arabic{subsubsection}.}
%\newcounter{rnumber}
%\renewcommand{\thernumber}{\refstepcounter{rnumber} }

\renewcommand{\prepartname}{\if@english Part~\else {}\fi}
\renewcommand{\postpartname}{\if@english\else {}\fi}
\renewcommand{\prechaptername}{\if@english Chapter~\else {}\fi}
\renewcommand{\postchaptername}{\if@english\else {}\fi}
\renewcommand{\presectionname}{}%  第
\renewcommand{\postsectionname}{}% 節

\makeatother



% PDF/X-1a
% \usepackage[x-1a]{pdfx}
% \Keywords{pdfTeX\sep PDF/X-1a\sep PDF/A-b}
% \Title{Sample LaTeX input file}
% \Author{LaTeX project team}
% \Org{TeX Users Group}
% \pdfcompresslevel=0
%\usepackage[cmyk]{xcolor}

%biblatex
%\usepackage[notes,strict,backend=biber,autolang=other,%
%                   bibencoding=inputenc,autocite=footnote]{biblatex-chicago}
%\addbibresource{hist-agri.bib}
\let\cite=\autocite

% % % % 
\date{}

%%% 脚注番号のページ毎のリセットと脚注位置の調整
%\makeatletter
%  \@addtoreset{footnote}{page}
%  \makeatother
%\usepackage{dblfnote}
%\usepackage[bottom,perpage]{footmisc}
\usepackage[bottom,perpage,stable]{footmisc}%
%\setlength{\skip\footins}{4mm plus 2mm}
%\usepackage{footnpag}
\makeatletter
\renewcommand\@makefntext[1]{%
  \advance\leftskip 1.5\zw
  \parindent 1\zw
  \noindent
  \llap{\@thefnmark\hskip0.5\zw}#1}


\renewenvironment{theindex}{% 索引を3段組で出力する環境
    \if@twocolumn
      \onecolumn\@restonecolfalse
    \else
      \clearpage\@restonecoltrue
    \fi
    \columnseprule.4pt \columnsep 2\zw
    \ifx\multicols\@undefined
      \twocolumn[\@makeschapterhead{\indexname}%
      \addcontentsline{toc}{chapter}{\indexname}]%変更点
    \else
      \ifdim\textwidth<\fullwidth
        \setlength{\evensidemargin}{\oddsidemargin}
        \setlength{\textwidth}{\fullwidth}
        \setlength{\linewidth}{\fullwidth}
        \begin{multicols}{3}[\chapter*{\indexname}
	\addcontentsline{toc}{chapter}{\indexname}]%変更点%
      \else
        \begin{multicols}{3}[\chapter*{\indexname}
	\addcontentsline{toc}{chapter}{\indexname}]%変更点%
      \fi
    \fi
    \@mkboth{\indexname}{\indexname}%
    \plainifnotempty % \thispagestyle{plain}
    \parindent\z@
    \parskip\z@ \@plus .3\p@\relax
    \let\item\@idxitem
    \raggedright
    \footnotesize\narrowbaselines
  }{
    \ifx\multicols\@undefined
      \if@restonecol\onecolumn\fi
    \else
      \end{multicols}
    \fi
    \clearpage
  }
\makeatother



\renewcommand{\ldots}{…}
\usepackage{makeidx}
\makeindex


\usepackage[unicode=true]{hyperref}
%\usepackage{pxjahyper}
\hypersetup{breaklinks=true,%
             bookmarks=true,%
             pdfauthor={},%
             pdftitle={},%
             colorlinks=true,%
             citecolor=blue,%
             urlcolor=cyan,%
             linkcolor=magenta,%
             pdfborder={0 0 0}}


% \hypersetup{
%     pdfborderstyle={/S/U/W 1}, % underline links instead of boxes
%     linkbordercolor=red,       % color of internal links
%     citebordercolor=green,     % color of links to bibliography
%     filebordercolor=magenta,   % color of file links
%     urlbordercolor=cyan        % color of external links
% }

           \urlstyle{same}
%\renewcommand*{\label}[1]{\hypertarget{#1}{}}
%\renewcommand{\hyperlink}[2]{\hyperref[#1]{#2}}

\renewcommand{\ldots}{\noindent…}

\newcommand{\maeaki}{}
%\newcommand{\maeaki}{\vspace*{0.125\zw}}
%\newcommand{\maeaki}{\vspace*{-0.75\zw}}
%\newcommand{\maeaki}{\vspace*{1.0\zw}}
%\newcommand{\maeaki}{\vspace*{1.1\zw}}
%\newcommand{\maeaki}{\vspace*{1.5\zw}}
%\newcommand{\maeaki}{\vspace*{1.75\zw}}
%\newcommand{\maeaki}{\vspace*{1.0mm}}                     
%\newcommand{\maeaki}{\vspace*{2.2\zw}}
%\newcommand{\maeaki}{\vspace*{-.25mm}}

%%分数の表記

\newcommand{\undemi}{$\sfrac{1}{2}$}
\newcommand{\untiers}{$\sfrac{1}{3}$}
\newcommand{\deuxtiers}{$\sfrac{2}{3}$}
\newcommand{\unquart}{$\sfrac{1}{4}$}
\newcommand{\troisquarts}{$\sfrac{3}{4}$}
\newcommand{\quatrequatrieme}{$\sfrac{4}{4$}}
\newcommand{\uncinquieme}{$\sfrac{1}{5}$}
\newcommand{\deuxcinquiemes}{$\sfrac{2}{5}$}
\newcommand{\troiscinquiemes}{$\sfrac{3}{5}$}
\newcommand{\quatrecinquiemes}{$\sfrac{4}{5}$}
\newcommand{\unsixieme}{$\sfrac{1}{6}$}
\newcommand{\cinqsixiemes}{$\sfrac{5}{6}$}
%初版、第二版原書とほぼ同じ判型。ただし調整が必要
% \documentclass[twoside,14Q,a4paper,openany]{ltjsbook}

\usepackage{amsmath}
\usepackage{amssymb}
\usepackage[no-math]{fontspec}
\usepackage{geometry}
\usepackage{xfrac}
\usepackage{luaotfload}
\usepackage{graphicx}

%% 欧文フォント設定
% Libertine/Biolinum
\setmainfont[Ligatures=Historic,Scale=1.0]{Linux Libertine O}
\setsansfont[Ligatures=TeX, Scale=MatchLowercase]{Linux Biolinum O} 
%\usepackage{libertine}
\usepackage{unicode-math}
\setmathfont[Scale=1.2]{libertinusmath-regular.otf}
%\unimathsetup{math-style=ISO,bold-style=ISO}
%\setmathfont{xits-math.otf}
%\setmathfont{xits-math.otf}[range={cal,bfcal},StylisticSet=1]




%% Garamond
%\usepackage{ebgaramond-maths}
%\setmainfont[Ligatures=Historic,Scale=1.1]{EB Garamond}%fontspecによるフォント設定

%\usepackage{qpalatin}%palatino

%\setmainfont[Ligatures=Historic,Scale=MatchLowercase]{Tex Gyre Schola}
%\setmainfont[Ligatures=Historic,Scale=MatchLowercase]{Tex Gyre Pagella}
%\setsansfont[Scale=MatchLowercase]{TeX Gyre Heros}  % \sffamily のフォント
%\setsansfont[Scale=MatchLowercase]{TeX Gyre Adventor}  % \sffamily のフォント

%\setmonofont[Scale=MatchLowercase]{Inconsolata}       % \ttfamily のフォント

%\usepackage[cmintegrals,cmbraces]{newtxmath}%数式フォント

\usepackage{luatexja}
\usepackage{luatexja-fontspec}
%\ltjdefcharrange{8}{"2000-"2013, "2015-"2025, "2027-"203A, "203C-"206F}
%\ltjsetparameter{jacharrange={-2, +8}}
\usepackage{luatexja-ruby}

%%%%和文フォント設定
%\usepackage[CharacterWidth=AlternateProportional,sourcehan,bold,jis,jis2004,expert,deluxe]{luatexja-preset}%Adobe源ノ明朝、ゴチ
%\usepackage[hiragino-pron,jis,bold,jis2004,expert,deluxe]{luatexja-preset}
%\usepackage[YokoFeatures={JFM=prop,PKana=On},ipaex,bold,jis,jis2004,expert,deluxe]{luatexja-preset}
%\usepackage[YokoFeatures={JFM=prop,PKana=On},ipaex,bold,jis,jis2004,expert,deluxe]{luatexja-preset}
%\usepackage[yu-osx,bold]{myluatexja-preset}
%\usepackage[moga-mogo-ex,bold]{myluatexja-preset}

\newopentypefeature{PKana}{On}{pkna} % "PKana" and "On" can be arbitrary string
   \setmainjfont[%
%   YokoFeatures={JFM=prop,PKana=On},
% %   CharacterWidth=AlternateProportional,
% % %       Kerning=On,
         BoldFont={ MoboGoB },%
         ItalicFont={ MoboGoB },%
         BoldItalicFont={ MoboGoExB }%
%         ]{ MogaHMin }
          ]{ IPAExMincho }
     \setsansjfont[%
%        YokoFeatures={JFM=prop,PKana=On},
% %        CharacterWidth=AlternateProportional,
% % %       Kerning=On,
         BoldFont={ MoboGoB },%
         ItalicFont={ MoboGoB },%
         BoldItalicFont={ MoboGoExB }%
         % ]{ MoboGo}
          ]{ IPAExGothic }
%  %%%% 和文仮名プロプーショナルここまで
% %\ltjsetparameter{jacharrange={-2}}%キリル文字%引数に-3を付けるとギリシア文字も可能になるが、%三点リーダーも欧文化されてしまうので注意%


\renewcommand{\bfdefault}{bx}%和文ボールドを有効にする
\renewcommand{\headfont}{\gtfamily\sffamily\bfseries}%和文ボールドを有効にする
%\addfontfeature{Fractions=On}


\defaultfontfeatures[\rmfamily]{Scale=1.2}%効いていない様子
\defaultjfontfeatures{Scale=0.92487}%和文フォントのサイズ調整。デフォルトは 0.962212 倍%ltjsclassesでは不要?
%\defaultjfontfeatures{Scale=0.962212}
%\usepackage{libertineotf}%linux libertine font %ギリシア語含む
%\usepackage[T1]{fontenc}
%\usepackage[full]{textcomp}
%\usepackage[osfI,scaled=1.0]{garamondx}
%\usepackage{tgheros,tgcursor}
%\usepackage[garamondx]{newtxmath}
\usepackage{xfrac}

\usepackage{layout}

%% レイアウト調整(A4Paper,13Q,onside,escoffierltjsbook.cls) 
%%
\setlength{\hoffset}{0\zw}
\setlength{\oddsidemargin}{0\zw}
\setlength{\evensidemargin}{\oddsidemargin}
\setlength{\fullwidth}{45\zw}
\setlength{\textwidth}{45\zw}%%ltjsclassesのみ有効
%\setlength{\fullwidth}{159mm}
%\setlength{\textwidth}{159mm}
\setlength{\marginparsep}{0pt}
\setlength{\marginparwidth}{0pt}
\setlength{\footskip}{0pt}
\setlength{\voffset}{-17mm}
\setlength{\textheight}{265mm}
\setlength{\parskip}{0pt}
%\setlength{\parindent}{0pt}
%%%ベースライン調整
%\ltjsetparameter{yjabaselineshift=0pt,yalbaselineshift=-.75pt}


%\usepackage{fancyhdr}

\usepackage{setspace}
\setstretch{1.0}




%文字サイズ、見出しなどの再定義
\makeatletter
%\renewcommand{\large}{\jsc@setfontsize\large\@xipt{14}}
%\renewcommand{\Large}{\jsc@setfontsize\Large{13}{15}}

\newcommand{\medlarge}{\fontsize{11}{13}\selectfont}
\newcommand{\medsmall}{\fontsize{9.23}{9.5}\selectfont}
\newcommand{\twelveq}{\jsc@setfontsize\twelveq{9.230769}{9.75}\selectfont}
\newcommand{\fourteenq}{\jsc@setfontsize\fourteenq{10.7692}{13}\selectfont}
\newcommand{\fifteenq}{\jsc@setfontsize\fifteenq{11.53846}{14}\selectfont}

\renewcommand{\chapter}{%
  \if@openleft\cleardoublepage\else
  \if@openright\cleardoublepage\else\clearpage\fi\fi
  \plainifnotempty % 元: \thispagestyle{plain}
  \global\@topnum\z@
  \if@english \@afterindentfalse \else \@afterindenttrue \fi
  \secdef
    {\@omit@numberfalse\@chapter}%
    {\@omit@numbertrue\@schapter}}
\def\@chapter[#1]#2{%
  \ifnum \c@secnumdepth >\m@ne
    \if@mainmatter
      \refstepcounter{chapter}%
      \typeout{\@chapapp\thechapter\@chappos}%
      \addcontentsline{toc}{chapter}%
        {\protect\numberline
        % {\if@english\thechapter\else\@chapapp\thechapter\@chappos\fi}%
        {\@chapapp\thechapter\@chappos}%
        #1}%
    \else\addcontentsline{toc}{chapter}{#1}\fi
  \else
    \addcontentsline{toc}{chapter}{#1}%
  \fi
  \chaptermark{#1}%
  \addtocontents{lof}{\protect\addvspace{10\jsc@mpt}}%
  \addtocontents{lot}{\protect\addvspace{10\jsc@mpt}}%
  \if@twocolumn
    \@topnewpage[\@makechapterhead{#2}]%
  \else
    \@makechapterhead{#2}%
    \@afterheading
  \fi}
\def\@makechapterhead#1{%
  \vspace*{0\Cvs}% 欧文は50pt
  {\parindent \z@ \centering \normalfont
    \ifnum \c@secnumdepth >\m@ne
      \if@mainmatter
        \huge\headfont \@chapapp\thechapter\@chappos%変更
        \par\nobreak
        \vskip \Cvs % 欧文は20pt
      \fi
    \fi
    \interlinepenalty\@M
    \huge \headfont #1\par\nobreak
    \vskip 1\Cvs}} % 欧文は40pt%変更

\renewcommand{\section}{%
    \if@slide\clearpage\fi
    \@startsection{section}{1}{\z@}%
    {\Cvs \@plus.5\Cdp \@minus.2\Cdp}% 前アキ
    % {.5\Cvs \@plus.3\Cdp}% 後アキ
    {.5\Cvs}
    {\normalfont\Large\headfont\bfseries\centering}}%変更

\renewcommand{\subsection}{\@startsection{subsection}{2}{\z@}%
    {\Cvs \@plus.5\Cdp \@minus.2\Cdp}% 前アキ
    % {.5\Cvs \@plus.3\Cdp}% 後アキ
    {.5\Cvs}
    {\normalfont\large\headfont\bfseries\centering}} %変更


\renewcommand{\subsubsection}{\@startsection{subsubsection}{3}{\z@}%
    {.25\Cvs \@plus.5\Cdp \@minus.5\Cdp}%変更
    {\if@slide .5\Cvs \@plus.3\Cdp \else \z@ \fi}%
    {\normalfont\medlarge\headfont\leftskip -1\zw}}

\renewcommand{\paragraph}{\@startsection{paragraph}{4}{\z@}%
    {0.5\Cvs \@plus.5\Cdp \@minus.2\Cdp}%
    % {\if@slide .5\Cvs \@plus.3\Cdp \else -1\zw\fi}% 改行せず 1\zw のアキ
    {1sp}%後アキ
    {\normalfont\normalsize\headfont}}
\renewcommand{\subparagraph}{\@startsection{subparagraph}{5}{\z@}%
    {\z@}{\if@slide .5\Cvs \@plus.3\Cdp \else -.5\zw\fi}%
    {\normalfont\normalsize\headfont\hskip-.5\zw\noindent}}  



\newcommand{\frsec}[1]{\vspace*{-1\zw}\begin{center}\normalfont\hspace*{1\zw}\headfont\Large\scshape#1\normalfont\normalsize\end{center}\vspace{0.5\zw}}

\newcommand{\frsecb}[1]{\vspace*{-1\zw}\begin{center}\hspace{1\zw}\normalfont\headfont\large\scshape#1\normalfont\normalsize\end{center}\vspace{0.5\zw}}

%\newcounter{frsub}[subsubsection]
%\newcommand{\frsub}{\@startsection{frsub}{6}{\z@}%
%  {1sp}{1sp}%
%  {\normalfont\normalsize\bfseries\baselineskip-.8ex\leftskip-1\zw}}
%\let\frsubmark\@gobble
%\newcommand*{\l@frsub}{%
%          \@tempdima\jsc@tocl@width \advance\@tempdima 16.183\zw
%          \@dottedtocline{7}{\@tempdima}{6.5\zw}}
%\renewcommand{\thefrsub}{6}
%\let\frsub\paragraph

\newenvironment{frsubenv}{\begin{spacing}{0.2}\setlength{\leftskip}{-1\zw}\bfseries}{\end{spacing}\normalfont\normalsize\setlength{\leftskip}{0pt}}
\newcommand{\frsub}[1]{\begin{frsubenv}#1\end{frsubenv}\par\vspace{1.1\zw}}

%\newcommand{\frsub}[1]{\vskip -.8ex\hskip -1\zw\textbf{#1}\leftskip0pt}
%\newcommand{\frsub}{\@startsection{frsub}{6}{\z@}%
%   {-1\zw}% 改行せず 1\zw のアキ
%   {-1\zw}%後アキ     
%   {\normalfont\normalsize\bfseries\leftskip -1\zw\baselineskip -.5ex}}%normalsizeから変更
%\newcommand*{\l@frsub}{%
%          \@tempdima\jsc@tocl@width \advance\@tempdima 16.183\zw
%          \@dottedtocline{5}{\@tempdima}{6.5\zw}}

\makeatother

%%% 脚注番号のページ毎のリセットと脚注位置の調整
\makeatletter

\usepackage[bottom,perpage,stable]{footmisc}%
%\setlength{\skip\footins}{4mm plus 2mm}
%\usepackage{footnpag}
\renewcommand\@makefntext[1]{%
  \advance\leftskip 1.5\zw
  \parindent 1\zw
  \noindent
  \llap{\@thefnmark\hskip0.5\zw}#1}


\let\footnotes@ve=\footnote
\def\footnote{\inhibitglue\footnotes@ve}
\let\footnotemarks@ve=\footnotemark
%\def\footnotemark{\inhibitglue\footnotemarks@ve}
\renewcommand{\footnotemark}{\footnotemarks@ve}%変更
% %\def\thefootnote{\ifnum\c@footnote>\z@\leavevmode\lower.5ex\hbox{(}\@arabic\c@footnote\hbox{)}\fi}
\renewcommand{\thefootnote}{\ifnum\c@footnote>\z@\leavevmode\hbox{}\@arabic\c@footnote\hbox{)}\fi}
%\makeatletter
% \@addtoreset{footnote}{page}
% \makeatother
%\usepackage{dblfnote}
%\usepackage[bottom,perpage]{footmisc}
\renewcommand{\footnote}[1]{}

\makeatother

%subsubsectionに連番をつける
%\usepackage{remreset}

\renewcommand{\thechapter}{}
\renewcommand{\thesection}{}
\renewcommand{\thesubsection}{}
\renewcommand{\thesubsubsection}{}
\renewcommand{\theparagraph}{}


%\makeatletter
%\@removefromreset{subsubsection}{subsection}
%\def\thesubsubsection{\arabic{subsubsection}.}
%\newcounter{rnumber}
%\renewcommand{\thernumber}{\refstepcounter{rnumber} }

\renewcommand{\prepartname}{\if@english Part~\else {}\fi}
\renewcommand{\postpartname}{\if@english\else {}\fi}
\renewcommand{\prechaptername}{\if@english Chapter~\else {}\fi}
\renewcommand{\postchaptername}{\if@english\else {}\fi}
\renewcommand{\presectionname}{}%  第
\renewcommand{\postsectionname}{}% 節





%レシピ本文
\usepackage{multicol}
\setlength{\columnsep}{3\zw}
%\setlength{\columnwidth}{24\zw}	
%\newenvironment{recette}{\setlength{\parindent}{0pt}\begin{medsmall}\begin{spacing}{0.8}\begin{multicols}{2}}{\end{multicols}\end{spacing}\end{medsmall}}

\newenvironment{recette}{\setlength{\parindent}{0pt}\begin{normalsize}\begin{spacing}{0.8}\begin{multicols}{2}}{\end{multicols}\end{spacing}\end{normalsize}}

%\newenvironment{recette}{\setlength{\parindent}{0pt}\begin{normalsize}\begin{multicols}{2}}{\end{multicols}\end{normalsize}}


%リスト環境
\def\tightlist{\itemsep1pt\parskip0pt\parsep0pt}%pandoc対策

\makeatletter
  \parsep   = 0pt
  \labelsep = .5\zw
  \def\@listi{%
     \leftmargin = 0pt \rightmargin = 0pt
     \labelwidth\leftmargin \advance\labelwidth-\labelsep
     \topsep     = 0pt%\baselineskip
     %\topsep -0.1\baselineskip \@plus 0\baselineskip \@minus 0.1 \baselineskip
     \partopsep  = 0pt \itemsep       = 0pt
     \itemindent = -.5\zw \listparindent = 0\zw}
  \let\@listI\@listi
  \@listi
  \def\@listii{%
     \leftmargin = 1.8\zw \rightmargin = 0pt
     \labelwidth\leftmargin \advance\labelwidth-\labelsep
     \topsep     = 0pt \partopsep     = 0pt \itemsep   = 0pt
     \itemindent = 0pt \listparindent = 1\zw}
  \let\@listiii\@listii
  \let\@listiv\@listii
  \let\@listv\@listii
  \let\@listvi\@listii
\makeatother
%% %%%%%%行取りマクロ
% \makeatletter
% \ifx\Cht\undefined
%  \newdimen\Cht\newdimen\Cdp
%  \setbox0\hbox{\char\jis"2121}\Cht=\ht0\Cdp=\dp0\fi
% \catcode`@=11
% \long\def\linespace#1#2{\par\noindent
%   \dimen@=\baselineskip
%   \multiply\dimen@ #1\advance\dimen@-\baselineskip
%   \advance\dimen@-\Cht\advance\dimen@\Cdp
%   \setbox0\vbox{\noindent #2}%
%   \advance\dimen@\ht0\advance\dimen@-\dp0%
%   \vtop to\z@{\hbox{\vrule width\z@ height\Cht depth\z@
%    \raise-.5\dimen@\hbox{\box0}}\vss}%
%   \dimen@=\baselineskip
%   \multiply\dimen@ #1\advance\dimen@-2\baselineskip
%   \par\nobreak\vskip\dimen@
%   \hbox{\vrule width\z@ height\Cht depth\z@}\vskip\z@}
% \catcode`@=12
% \setlength{\parskip}{0pt}
% \setlength{\topskip}{\Cht}
% \setlength{\textheight}{43\baselineskip}
% \addtolength{\textheight}{1\zh}
% \makeatother
 
%%%%%%%%%%%%失敗%%%%%%%%%%%%
%\let\formule\subsubsection
%\renewcommand{\subsubsection}[1]{\linespace{1}{\formule#1}}
%%%%%%%%%%%%失敗%%%%%%%%%%%%






% PDF/X-1a
% \usepackage[x-1a]{pdfx}
% \Keywords{pdfTeX\sep PDF/X-1a\sep PDF/A-b}
% \Title{Sample LaTeX input file}
% \Author{LaTeX project team}
% \Org{TeX Users Group}
% \pdfcompresslevel=0
%\usepackage[cmyk]{xcolor}

%biblatex
%\usepackage[notes,strict,backend=biber,autolang=other,%
%                   bibencoding=inputenc,autocite=footnote]{biblatex-chicago}
%\addbibresource{hist-agri.bib}
\let\cite=\autocite

% % % % 
\date{}



\makeatletter
\renewenvironment{theindex}{% 索引を3段組で出力する環境
    \if@twocolumn
      \onecolumn\@restonecolfalse
    \else
      \clearpage\@restonecoltrue
    \fi
    \columnseprule.4pt \columnsep 2\zw
    \ifx\multicols\@undefined
      \twocolumn[\@makeschapterhead{\indexname}%
      \addcontentsline{toc}{chapter}{\indexname}]%変更点
    \else
      \ifdim\textwidth<\fullwidth
        \setlength{\evensidemargin}{\oddsidemargin}
        \setlength{\textwidth}{\fullwidth}
        \setlength{\linewidth}{\fullwidth}
        \begin{multicols}{3}[\chapter*{\indexname}
	\addcontentsline{toc}{chapter}{\indexname}]%変更点%
      \else
        \begin{multicols}{3}[\chapter*{\indexname}
	\addcontentsline{toc}{chapter}{\indexname}]%変更点%
      \fi
    \fi
    \@mkboth{\indexname}{\indexname}%
    \plainifnotempty % \thispagestyle{plain}
    \parindent\z@
    \parskip\z@ \@plus .3\p@\relax
    \let\item\@idxitem
    \raggedright
    \footnotesize\narrowbaselines
  }{
    \ifx\multicols\@undefined
      \if@restonecol\onecolumn\fi
    \else
      \end{multicols}
    \fi
    \clearpage
  }
\makeatother



%\renewcommand{\ldots}{…}
\usepackage{makeidx}
\makeindex


\usepackage[unicode=true]{hyperref}
%\usepackage{pxjahyper}
\hypersetup{breaklinks=true,%
             bookmarks=true,%
             pdfauthor={五島 学},%
             pdftitle={エスコフィエ『料理の手引き』全注解},%
             colorlinks=true,%
             citecolor=blue,%
             urlcolor=cyan,%
             linkcolor=magenta,%
             bookmarksdepth=subsubsection,%
             pdfborder={0 0 0}}


% \hypersetup{
%     pdfborderstyle={/S/U/W 1}, % underline links instead of boxes
%     linkbordercolor=red,       % color of internal links
%     citebordercolor=green,     % color of links to bibliography
%     filebordercolor=magenta,   % color of file links
%     urlbordercolor=cyan        % color of external links
% }

\urlstyle{same}
%\renewcommand*{\label}[1]{\hypertarget{#1}{}}
%\renewcommand{\hyperlink}[2]{\hyperref[#1]{#2}}

\renewcommand{\ldots}{\noindent…}
%\usepackage{udline}
% \usepackage{ulem}
\usepackage{umoline}
\setlength{\UnderlineDepth}{2pt}
\let\ul\Underline

\newcommand{\maeaki}{}
%\newcommand{\maeaki}{\vspace{0.125\zw}}
%\newcommand{\maeaki}{\vspace{0.7\zw}}
%\newcommand{\maeaki}{\vspace{2.0\zw}}
%\newcommand{\maeaki}{\vspace{1.1\zw}}
%\newcommand{\maeaki}{\vspace{1.5\zw}}
%\newcommand{\maeaki}{\vspace{1.75\zw}}
%\newcommand{\maeaki}{\vspace{1.0mm}}                     
%\newcommand{\maeaki}{\vspace{2.2\zw}}
%\newcommand{\maeaki}{\vspace{-.25mm}}

%%分数の表記

\newcommand{\undemi}{\hspace{.25\zw}$\sfrac{1}{2}$}
\newcommand{\untiers}{\hspace{.25\zw}$\sfrac{1}{3}$}
\newcommand{\deuxtiers}{\hspace{.25\zw}$\sfrac{2}{3}$}
\newcommand{\unquart}{\hspace{.25\zw}$\sfrac{1}{4}$}
\newcommand{\troisquarts}{\hspace{.25\zw}$\sfrac{3}{4}$}
\newcommand{\quatrequatrieme}{\hspace{.25\zw}$\sfrac{4}{4$}}
\newcommand{\uncinquieme}{\hspace{.25\zw}$\sfrac{1}{5}$}
\newcommand{\deuxcinquiemes}{\hspace{.25\zw}$\sfrac{2}{5}$}
\newcommand{\troiscinquiemes}{\hspace{.25\zw}$\sfrac{3}{5}$}
\newcommand{\quatrecinquiemes}{\hspace{.25\zw}$\sfrac{4}{5}$}
\newcommand{\unsixieme}{\hspace{.25\zw}$\sfrac{1}{6}$}
\newcommand{\cinqsixiemes}{\hspace{.25\zw}$\sfrac{5}{6}$}
\newcommand{\quatresurquatre}{\hspace{.25\zw}$\sfrac{4}{4}$}
%脚注なしバージョン、ただし調整が必要
% \documentclass[twoside,14Q,a4paper,openany]{ltjsbook}

\usepackage{amsmath}
\usepackage{amssymb}
\usepackage[no-math]{fontspec}
\usepackage{geometry}
\usepackage{unicode-math}
\usepackage{xfrac}
\usepackage{luaotfload}
\usepackage{graphicx}

%%欧文フォント設定
\setmainfont[Ligatures=Historic,Scale=1.0]{Linux Libertine O}

%%Garamond
%\usepackage{ebgaramond-maths}
%\setmainfont[Ligatures=Historic,Scale=1.1]{EB Garamond}%fontspecによるフォント設定

%\usepackage{qpalatin}%palatino

%%%%%\setmainfont[Ligatures=Historic,Scale=MatchLowercase]{Tex Gyre Schola}
%\setmainfont[Ligatures=Historic,Scale=MatchLowercase]{Tex Gyre Pagella}
%\setsansfont[Scale=MatchLowercase]{TeX Gyre Heros}  % \sffamily のフォント
%\setsansfont[Scale=MatchLowercase]{TeX Gyre Adventor}  % \sffamily のフォント
\setsansfont[Ligatures=TeX, Scale=MatchLowercase]{Linux Biolinum O}     % Libertine/Biolinum
%\setmonofont[Scale=MatchLowercase]{Inconsolata}       % \ttfamily のフォント
%\unimathsetup{math-style=ISO,bold-style=ISO}
%\setmathfont{xits-math.otf}
%\setmathfont{xits-math.otf}[range={cal,bfcal},StylisticSet=1]

%\index{\usepackage}\usepackage[cmintegrals,cmbraces]{newtxmath}%数式フォント

\usepackage{luatexja}
\usepackage{luatexja-fontspec}
%\ltjdefcharrange{8}{"2000-"2013, "2015-"2025, "2027-"203A, "203C-"206F}
%\ltjsetparameter{jacharrange={-2, +8}}
\usepackage{luatexja-ruby}

%%%%和文フォント設定
%\usepackage[CharacterWidth=AlternateProportional,sourcehan,bold,jis,jis2004,expert,deluxe]{luatexja-preset}%Adobe源ノ明朝、ゴチ
%\usepackage[hiragino-pron,jis,bold,jis2004,expert,deluxe]{luatexja-preset}
%\usepackage[YokoFeatures={JFM=prop,PKana=On},ipaex,bold,jis,jis2004,expert,deluxe]{luatexja-preset}
%\usepackage[YokoFeatures={JFM=prop,PKana=On},ipaex,bold,jis,jis2004,expert,deluxe]{luatexja-preset}
%\usepackage[yu-osx,bold]{myluatexja-preset}
%\usepackage[moga-mogo-ex,bold]{myluatexja-preset}

\newopentypefeature{PKana}{On}{pkna} % "PKana" and "On" can be arbitrary string
   \setmainjfont[%
%   YokoFeatures={JFM=prop,PKana=On},
% %   CharacterWidth=AlternateProportional,
% % %       Kerning=On,
         BoldFont={ MoboGoB },%
         ItalicFont={ MoboGoB },%
         BoldItalicFont={ MoboGoExB }%
%         ]{ MogaHMin }
          ]{ IPAExMincho }
     \setsansjfont[%
%        YokoFeatures={JFM=prop,PKana=On},
% %        CharacterWidth=AlternateProportional,
% % %       Kerning=On,
         BoldFont={ MoboGoB },%
         ItalicFont={ MoboGoB },%
         BoldItalicFont={ MoboGoExB }%
         % ]{ MoboGo}
          ]{ IPAExGothic }
%  %%%% 和文仮名プロプーショナルここまで
% %\ltjsetparameter{jacharrange={-2}}%キリル文字%引数に-3を付けるとギリシア文字も可能になるが、%三点リーダーも欧文化されてしまうので注意%


\renewcommand{\bfdefault}{bx}%和文ボールドを有効にする
\renewcommand{\headfont}{\gtfamily\sffamily\bfseries}%和文ボールドを有効にする
%\addfontfeature{Fractions=On}


\defaultfontfeatures[\rmfamily]{Scale=1.2}%効いていない様子
\defaultjfontfeatures{Scale=0.92487}%和文フォントのサイズ調整。デフォルトは 0.962212 倍%ltjsclassesでは不要?
%\defaultjfontfeatures{Scale=0.962212}
%\usepackage{libertineotf}%linux libertine font %ギリシア語含む
%\usepackage[T1]{fontenc}
%\usepackage[full]{textcomp}
%\usepackage[osfI,scaled=1.0]{garamondx}
%\usepackage{tgheros,tgcursor}
%\usepackage[garamondx]{newtxmath}
\usepackage{xfrac}

\usepackage{layout}

%% レイアウト調整(A4Paper,13Q,onside,escoffierltjsbook.cls) 
%%
\setlength{\hoffset}{0\zw}
\setlength{\oddsidemargin}{0\zw}
\setlength{\evensidemargin}{\oddsidemargin}
\setlength{\fullwidth}{45\zw}
\setlength{\textwidth}{45\zw}%%ltjsclassesのみ有効
%\setlength{\fullwidth}{159mm}
%\setlength{\textwidth}{159mm}
\setlength{\marginparsep}{0pt}
\setlength{\marginparwidth}{0pt}
\setlength{\footskip}{0pt}
\setlength{\voffset}{-17mm}
\setlength{\textheight}{265mm}
\setlength{\parskip}{0pt}
%\setlength{\parindent}{0pt}
%%%ベースライン調整
%\ltjsetparameter{yjabaselineshift=0pt,yalbaselineshift=-.75pt}


%\usepackage{fancyhdr}

\usepackage{setspace}
\setstretch{1.0}




%文字サイズ、見出しなどの再定義
\makeatletter
%\renewcommand{\large}{\jsc@setfontsize\large\@xipt{14}}
%\renewcommand{\Large}{\jsc@setfontsize\Large{13}{15}}

\newcommand{\medlarge}{\fontsize{11}{13}\selectfont}
\newcommand{\medsmall}{\fontsize{9.23}{9.5}\selectfont}
\newcommand{\twelveq}{\jsc@setfontsize\twelveq{9.230769}{9.75}\selectfont}
\newcommand{\fourteenq}{\jsc@setfontsize\fourteenq{10.7692}{13}\selectfont}
\newcommand{\fifteenq}{\jsc@setfontsize\fifteenq{11.53846}{14}\selectfont}

\renewcommand{\chapter}{%
  \if@openleft\cleardoublepage\else
  \if@openright\cleardoublepage\else\clearpage\fi\fi
  \plainifnotempty % 元: \thispagestyle{plain}
  \global\@topnum\z@
  \if@english \@afterindentfalse \else \@afterindenttrue \fi
  \secdef
    {\@omit@numberfalse\@chapter}%
    {\@omit@numbertrue\@schapter}}
\def\@chapter[#1]#2{%
  \ifnum \c@secnumdepth >\m@ne
    \if@mainmatter
      \refstepcounter{chapter}%
      \typeout{\@chapapp\thechapter\@chappos}%
      \addcontentsline{toc}{chapter}%
        {\protect\numberline
        % {\if@english\thechapter\else\@chapapp\thechapter\@chappos\fi}%
        {\@chapapp\thechapter\@chappos}%
        #1}%
    \else\addcontentsline{toc}{chapter}{#1}\fi
  \else
    \addcontentsline{toc}{chapter}{#1}%
  \fi
  \chaptermark{#1}%
  \addtocontents{lof}{\protect\addvspace{10\jsc@mpt}}%
  \addtocontents{lot}{\protect\addvspace{10\jsc@mpt}}%
  \if@twocolumn
    \@topnewpage[\@makechapterhead{#2}]%
  \else
    \@makechapterhead{#2}%
    \@afterheading
  \fi}
\def\@makechapterhead#1{%
  \vspace*{0\Cvs}% 欧文は50pt
  {\parindent \z@ \centering \normalfont
    \ifnum \c@secnumdepth >\m@ne
      \if@mainmatter
        \huge\headfont \@chapapp\thechapter\@chappos%変更
        \par\nobreak
        \vskip \Cvs % 欧文は20pt
      \fi
    \fi
    \interlinepenalty\@M
    \huge \headfont #1\par\nobreak
    \vskip 1\Cvs}} % 欧文は40pt%変更

\renewcommand{\section}{%
    \if@slide\clearpage\fi
    \@startsection{section}{1}{\z@}%
    {\Cvs \@plus.5\Cdp \@minus.2\Cdp}% 前アキ
    % {.5\Cvs \@plus.3\Cdp}% 後アキ
    {.5\Cvs}
    {\normalfont\Large\headfont\bfseries\centering}}%変更

\renewcommand{\subsection}{\@startsection{subsection}{2}{\z@}%
    {\Cvs \@plus.5\Cdp \@minus.2\Cdp}% 前アキ
    % {.5\Cvs \@plus.3\Cdp}% 後アキ
    {.5\Cvs}
    {\normalfont\large\headfont\bfseries\centering}} %変更


\renewcommand{\subsubsection}{\@startsection{subsubsection}{3}{\z@}%
    {.25\Cvs \@plus.5\Cdp \@minus.5\Cdp}%変更
    {\if@slide .5\Cvs \@plus.3\Cdp \else \z@ \fi}%
    {\normalfont\medlarge\headfont\leftskip -1\zw}}

\renewcommand{\paragraph}{\@startsection{paragraph}{4}{\z@}%
    {0.5\Cvs \@plus.5\Cdp \@minus.2\Cdp}%
    % {\if@slide .5\Cvs \@plus.3\Cdp \else -1\zw\fi}% 改行せず 1\zw のアキ
    {1sp}%後アキ
    {\normalfont\normalsize\headfont}}
\renewcommand{\subparagraph}{\@startsection{subparagraph}{5}{\z@}%
    {\z@}{\if@slide .5\Cvs \@plus.3\Cdp \else -.5\zw\fi}%
    {\normalfont\normalsize\headfont\hskip-.5\zw\noindent}}  



\newcommand{\frsec}[1]{\vspace*{-1\zw}\begin{center}\normalfont\hspace*{1\zw}\headfont\Large\scshape#1\normalfont\normalsize\end{center}\vspace{0.5\zw}}

\newcommand{\frsecb}[1]{\vspace*{-1\zw}\begin{center}\hspace{1\zw}\normalfont\headfont\large\scshape#1\normalfont\normalsize\end{center}\vspace{0.5\zw}}

\newcommand{\frsub}[1]{\vskip -.8ex\hskip -1\zw\textbf{#1}\leftskip0pt}
%\newcommand{\frsub}{\@startsection{frsub}{6}{\z@}%
%   {-1\zw}% 改行せず 1\zw のアキ
%   {-1\zw}%後アキ     
%   {\normalfont\normalsize\bfseries\leftskip -1\zw\baselineskip -.5ex}}%normalsizeから変更
%\newcommand*{\l@frsub}{%
%          \@tempdima\jsc@tocl@width \advance\@tempdima 16.183\zw
%          \@dottedtocline{5}{\@tempdima}{6.5\zw}}

\makeatother

%%% 脚注番号のページ毎のリセットと脚注位置の調整
\makeatletter

\usepackage[bottom,perpage,stable]{footmisc}%
%\setlength{\skip\footins}{4mm plus 2mm}
%\usepackage{footnpag}
\renewcommand\@makefntext[1]{%
  \advance\leftskip 1.5\zw
  \parindent 1\zw
  \noindent
  \llap{\@thefnmark\hskip0.5\zw}#1}


\let\footnotes@ve=\footnote
\def\footnote{\inhibitglue\footnotes@ve}
\let\footnotemarks@ve=\footnotemark
%\def\footnotemark{\inhibitglue\footnotemarks@ve}
\renewcommand{\footnotemark}{\footnotemarks@ve}%変更
% %\def\thefootnote{\ifnum\c@footnote>\z@\leavevmode\lower.5ex\hbox{(}\@arabic\c@footnote\hbox{)}\fi}
\renewcommand{\thefootnote}{\ifnum\c@footnote>\z@\leavevmode\hbox{}\@arabic\c@footnote\hbox{)}\fi}
%\makeatletter
% \@addtoreset{footnote}{page}
% \makeatother
%\usepackage{dblfnote}
%\usepackage[bottom,perpage]{footmisc}


\makeatother

%subsubsectionに連番をつける
%\usepackage{remreset}

\renewcommand{\thechapter}{}
\renewcommand{\thesection}{}
\renewcommand{\thesubsection}{}
\renewcommand{\thesubsubsection}{}
\renewcommand{\theparagraph}{}

%\makeatletter
%\@removefromreset{subsubsection}{subsection}
%\def\thesubsubsection{\arabic{subsubsection}.}
%\newcounter{rnumber}
%\renewcommand{\thernumber}{\refstepcounter{rnumber} }

\renewcommand{\prepartname}{\if@english Part~\else {}\fi}
\renewcommand{\postpartname}{\if@english\else {}\fi}
\renewcommand{\prechaptername}{\if@english Chapter~\else {}\fi}
\renewcommand{\postchaptername}{\if@english\else {}\fi}
\renewcommand{\presectionname}{}%  第
\renewcommand{\postsectionname}{}% 節





%レシピ本文
\usepackage{multicol}
\setlength{\columnsep}{3\zw}
%\setlength{\columnwidth}{24\zw}	
%\newenvironment{recette}{\setlength{\parindent}{0pt}\begin{medsmall}\begin{spacing}{0.8}\begin{multicols}{2}}{\end{multicols}\end{spacing}\end{medsmall}}

%\newenvironment{recette}{\setlength{\parindent}{0pt}\begin{normalsize}\begin{spacing}{0.8}\begin{multicols}{2}}{\end{multicols}\end{spacing}\end{normalsize}}

%\newenvironment{recette}{\setlength{\parindent}{0pt}\begin{normalsize}\begin{multicols}{2}}{\end{multicols}\end{normalsize}}

\newenvironment{recette}{}{}

%リスト環境
\def\tightlist{\itemsep1pt\parskip0pt\parsep0pt}%pandoc対策

\makeatletter
  \parsep   = 0pt
  \labelsep = .5\zw
  \def\@listi{%
     \leftmargin = 0pt \rightmargin = 0pt
     \labelwidth\leftmargin \advance\labelwidth-\labelsep
     \topsep     = 0pt%\baselineskip
     %\topsep -0.1\baselineskip \@plus 0\baselineskip \@minus 0.1 \baselineskip
     \partopsep  = 0pt \itemsep       = 0pt
     \itemindent = -.5\zw \listparindent = 0\zw}
  \let\@listI\@listi
  \@listi
  \def\@listii{%
     \leftmargin = 1.8\zw \rightmargin = 0pt
     \labelwidth\leftmargin \advance\labelwidth-\labelsep
     \topsep     = 0pt \partopsep     = 0pt \itemsep   = 0pt
     \itemindent = 0pt \listparindent = 1\zw}
  \let\@listiii\@listii
  \let\@listiv\@listii
  \let\@listv\@listii
  \let\@listvi\@listii
\makeatother
%% %%%%%%行取りマクロ
% \makeatletter
% \ifx\Cht\undefined
%  \newdimen\Cht\newdimen\Cdp
%  \setbox0\hbox{\char\jis"2121}\Cht=\ht0\Cdp=\dp0\fi
% \catcode`@=11
% \long\def\linespace#1#2{\par\noindent
%   \dimen@=\baselineskip
%   \multiply\dimen@ #1\advance\dimen@-\baselineskip
%   \advance\dimen@-\Cht\advance\dimen@\Cdp
%   \setbox0\vbox{\noindent #2}%
%   \advance\dimen@\ht0\advance\dimen@-\dp0%
%   \vtop to\z@{\hbox{\vrule width\z@ height\Cht depth\z@
%    \raise-.5\dimen@\hbox{\box0}}\vss}%
%   \dimen@=\baselineskip
%   \multiply\dimen@ #1\advance\dimen@-2\baselineskip
%   \par\nobreak\vskip\dimen@
%   \hbox{\vrule width\z@ height\Cht depth\z@}\vskip\z@}
% \catcode`@=12
% \setlength{\parskip}{0pt}
% \setlength{\topskip}{\Cht}
% \setlength{\textheight}{43\baselineskip}
% \addtolength{\textheight}{1\zh}
% \makeatother
 
%%%%%%%%%%%%失敗%%%%%%%%%%%%
%\let\formule\subsubsection
%\renewcommand{\subsubsection}[1]{\linespace{1}{\formule#1}}
%%%%%%%%%%%%失敗%%%%%%%%%%%%






% PDF/X-1a
% \usepackage[x-1a]{pdfx}
% \Keywords{pdfTeX\sep PDF/X-1a\sep PDF/A-b}
% \Title{Sample LaTeX input file}
% \Author{LaTeX project team}
% \Org{TeX Users Group}
% \pdfcompresslevel=0
%\usepackage[cmyk]{xcolor}

%biblatex
%\usepackage[notes,strict,backend=biber,autolang=other,%
%                   bibencoding=inputenc,autocite=footnote]{biblatex-chicago}
%\addbibresource{hist-agri.bib}
\let\cite=\autocite

% % % % 
\date{}



\makeatletter
\renewenvironment{theindex}{% 索引を3段組で出力する環境
    \if@twocolumn
      \onecolumn\@restonecolfalse
    \else
      \clearpage\@restonecoltrue
    \fi
    \columnseprule.4pt \columnsep 2\zw
    \ifx\multicols\@undefined
      \twocolumn[\@makeschapterhead{\indexname}%
      \addcontentsline{toc}{chapter}{\indexname}]%変更点
    \else
      \ifdim\textwidth<\fullwidth
        \setlength{\evensidemargin}{\oddsidemargin}
        \setlength{\textwidth}{\fullwidth}
        \setlength{\linewidth}{\fullwidth}
        \begin{multicols}{3}[\chapter*{\indexname}
	\addcontentsline{toc}{chapter}{\indexname}]%変更点%
      \else
        \begin{multicols}{3}[\chapter*{\indexname}
	\addcontentsline{toc}{chapter}{\indexname}]%変更点%
      \fi
    \fi
    \@mkboth{\indexname}{\indexname}%
    \plainifnotempty % \thispagestyle{plain}
    \parindent\z@
    \parskip\z@ \@plus .3\p@\relax
    \let\item\@idxitem
    \raggedright
    \footnotesize\narrowbaselines
  }{
    \ifx\multicols\@undefined
      \if@restonecol\onecolumn\fi
    \else
      \end{multicols}
    \fi
    \clearpage
  }
\makeatother



%\renewcommand{\ldots}{…}
\usepackage{makeidx}
\makeindex


\usepackage[unicode=true]{hyperref}
%\usepackage{pxjahyper}
\hypersetup{breaklinks=true,%
             bookmarks=true,%
             pdfauthor={五島 学},%
             pdftitle={エスコフィエ『料理の手引き』全注解},%
             colorlinks=true,%
             citecolor=blue,%
             urlcolor=cyan,%
             linkcolor=magenta,%
             bookmarksdepth=subsubsection,%
             pdfborder={0 0 0}}


% \hypersetup{
%     pdfborderstyle={/S/U/W 1}, % underline links instead of boxes
%     linkbordercolor=red,       % color of internal links
%     citebordercolor=green,     % color of links to bibliography
%     filebordercolor=magenta,   % color of file links
%     urlbordercolor=cyan        % color of external links
% }

\urlstyle{same}
%\renewcommand*{\label}[1]{\hypertarget{#1}{}}
%\renewcommand{\hyperlink}[2]{\hyperref[#1]{#2}}

\renewcommand{\ldots}{\noindent…}
%\usepackage{udline}
% \usepackage{ulem}
%\usepackage{umoline}
%\setlength{\UnderlineDepth}{2pt}
\let\ul\underline

\newcommand{\maeaki}{}
%\newcommand{\maeaki}{\vspace{0.125\zw}}
%\newcommand{\maeaki}{\vspace{0.7\zw}}
%\newcommand{\maeaki}{\vspace{2.0\zw}}
%\newcommand{\maeaki}{\vspace{1.1\zw}}
%\newcommand{\maeaki}{\vspace{1.5\zw}}
%\newcommand{\maeaki}{\vspace{1.75\zw}}
%\newcommand{\maeaki}{\vspace{1.0mm}}                     
%\newcommand{\maeaki}{\vspace{2.2\zw}}
%\newcommand{\maeaki}{\vspace{-.25mm}}

%%分数の表記

\newcommand{\undemi}{$\sfrac{1}{2}$}
\newcommand{\untiers}{$\sfrac{1}{3}$}
\newcommand{\deuxtiers}{$\sfrac{2}{3}$}
\newcommand{\unquart}{$\sfrac{1}{4}$}
\newcommand{\troisquarts}{$\sfrac{3}{4}$}
\newcommand{\quatrequatrieme}{$\sfrac{4}{4$}}
\newcommand{\uncinquieme}{$\sfrac{1}{5}$}
\newcommand{\deuxcinquiemes}{$\sfrac{2}{5}$}
\newcommand{\troiscinquiemes}{$\sfrac{3}{5}$}
\newcommand{\quatrecinquiemes}{$\sfrac{4}{5}$}
\newcommand{\unsixieme}{$\sfrac{1}{6}$}
\newcommand{\cinqsixiemes}{$\sfrac{5}{6}$}%1段組みバージョン、ただし調整が必要
%
%
%
%
%%% Important!%%%%%%% 文書開始%%%

%%%info%%%
\title{\Huge{オーギュスト・エスコフィエ}\\\HUGE{『料理の手引き』全注解}}
\author{\huge{五 島 学}}
%%%%
%%% Important! 文書開始%
\begin{document}

%%% 扉 %%%
\maketitle
%%% 序文開始
%\layout%レイアウト数値確認用
\frontmatter

% 企画本参考例
%\hypertarget{ux30a8ux30b9ux30b3ux30d5ux30a3ux30a8ux306eux65b0ux89e3ux91c8-ux53c2ux8003ux4f8b}{%
\chapter{エスコフィエの新解釈 --- 参考例
---}\label{ux30a8ux30b9ux30b3ux30d5ux30a3ux30a8ux306eux65b0ux89e3ux91c8-ux53c2ux8003ux4f8b}}

\hypertarget{les-hors-d-oeuvres}{%
\section{前菜}\label{les-hors-d-oeuvres}}
\begin{recette}
\hypertarget{bouchees}{%
\subsubsection{ブシェ}\label{bouchees}}

\frsub{Bouchées}

通常、ブシェをメニューの「温製オードブル」に位置付ける場合には、標準的なブシェよりも小さいサイズのものにしなくてはいけない。そのうえで、「かわいらしいブシェ」のように明記される。形状はどんな仕上りのものにするかでいろいろに変えてやり、大きなブシェを切った場合とは全然違うものであるとわかるようにすること。

場合によっては、ブシェの蓋の部分は残して蓋にするが、スライスしたまま、あるいは飾り切りをしたトリュフを蓋にすることもあるし、また別の場合には、詰めものの一部を蓋として利用することもある。

ブシェは必ずナフキンの上にのせて供すること。

\hypertarget{bouchee-a-la-reine}{%
\subsubsection{ブシェ・王妃風}\label{bouchee-a-la-reine}}

\frsub{Bouchée à la Reine}

この種のブシェの、クラシックな、本来の詰め物は生クリーム入りの鶏のピュレが用いられていた。だが、こんにちでは鶏胸肉とマッシュルーム、トリュフを1〜2
mm角の細かいみじん切りにして\protect\hyperlink{sauce-allemande}{ソース・アルマンド}であえたもので代用されている。ほとんど全ての調理現場では詰め物に後者を用いるようになってしまった。このブシェの形状は必ず円形で、縁に波形の模様が入ったものであること。

\begin{center}\rule{0.5\linewidth}{\linethickness}\end{center}

\hypertarget{epinards-a-la-viroflay}{%
\subsubsection[ほうれんそう・ヴィロフレー]{\texorpdfstring{ほうれんそう・ヴィロフレー\footnote{パリ郊外南西のヴェルサイユ近くの地名。ほうれんそうの栽培で有名で、ヴィロフレーという名称の伝統品種もある。}}{ほうれんそう・ヴィロフレー}}\label{epinards-a-la-viroflay}}

\frsub{Epinards à la Viroflay}

布の上に下茹でしたほうれんそうの葉(大)を広げる。それぞれの葉の中心に「ほうれんそうのシュブリック」を置く。このシュブリックにはパンの身をバターで揚げた小さなクルトンを混ぜ込んでおくこと。シュブリックをほうれんそうの葉で丸くなるように包む。これをバターを塗ったグラタン皿に並べ、\protect\hyperlink{sauce-mornay}{ソース・モルネー}を覆いかける。上からおろしたチーズを振りかけ、溶かしバターをかけてやり、高温のオーブンでこんがり焼く。

\hypertarget{subric-d-epinards}{%
\subsubsection{ほうれんそうのシュブリック}\label{subric-d-epinards}}

\frsub{Subric d'épinards}

ほうれんそうは上述のとおり\footnote{「ほうれんそうのクリームあえ」参照。ほうれんそうは下茹でして水にはなしてから、水気を絞り、みじん切りにするか裏漉ししてから、ほうれんそう500
  gあたりバター60
  gとともにソテー鍋に入れて強火にかけ、余計な水分をとばす。}にバターを加えて強火にかけて水気をとばす。鍋を火からはずし、ほうれんそう500
gあたり、濃い\protect\hyperlink{sauce-bechamel}{ベシャメルソース}1
dL、クレーム・エペス大さじ2杯、溶きほぐした全卵1
個と卵黄3個、塩、こしょう、ナツメグを加える。フライパンにたっぷりのバターを熱して充分な量の澄ましバターを用意する。

ほうれんそうでつくったアパレイユをスプーンで掬いとり、指で押し出すようにして澄ましバターの中に落としていく。シュブリックの成形をそのまま続けていくが、隣り同士で触れ合わないように注意すること。1分程焼いたら、パレットナイフかフォークで反対側の面にも焼き色を付けてやる。これをメインの料理の皿か野菜料理用の皿に盛り、ソース・クレームを別添で供する。

\hypertarget{nota-subric-d-epinards}{%
\subparagraph{【原注】}\label{nota-subric-d-epinards}}

シュブリックのアパレイユには別の作り方もある。バターを加えてほうれんそうの水気をとばしたら、ほうれんそうと同量の、やや濃い目に作ったクレープ生地を混ぜ込む。
\end{recette}
\begin{center}\rule{0.5\linewidth}{\linethickness}\end{center}
\newpage
\hypertarget{Potages}{%
\section{ポタージュ}\label{Potages}}
\begin{recette}
\hypertarget{consomme-rabelais}{%
\subsubsection[コンソメ・ラブレー]{\texorpdfstring{コンソメ・ラブレー\footnote{フランスのルネサンス期を代表する人文主義者、小説家であり医師でもあったフランソワ・ラブレー(?〜1553)のこと。なおこのレシピは第四版のみで、初版は「ジビエのコンソメにヴヴレ産白ワイン2
  dLを煮詰めて加える(コンソメ4
  Lあたり)。浮き実は小さな棒状にしたトリュフ風味のひばりの小さなクネルと、セロリの千切りをコンソメで軽く煮たもの
  (p.23)」。第二版では「ジビエのコンソメに、1
  Lあたりヴヴレ産白ワイン\(\frac{1}{2}\)
  dLを煮詰めて加える。浮き実\ldots{}\ldots{}トリュフを加えたひばりのファルスを刻み模様の付いた口金で絞り出したクネル。セロリの千切りをコンソメで軽く煮たもの(p.170)」となっているが、第三版にこの名称のレシピは掲載されていない。なお、ラブレーはシノン郊外の生まれであるため、トゥーレーヌ産のワイン(とりわけシノンの赤)が引き合いに出されることが多い。}}{コンソメ・ラブレー}}\label{consomme-rabelais}}

\begin{itemize}
\item
  鶏のコンソメにペルドローのフュメを加える。
\item
  浮き実\ldots{}\ldots{}\protect\hyperlink{farce-c}{生クリーム入りペルドローのファルス}をコーヒースプーンで成形し、提供直前に沸騰しない程度の温度で火を通した\footnote{pocher
    (ポシェ)。}クネル。マデイラ酒風味で火を通したトリュフの細い千切り\footnote{fine
    julienne (フィーヌジュリエンヌ)。}。
\item
  別添\ldots{}\ldots{}パルメザン風味の小さなプロフィットロール。
\end{itemize}

\hypertarget{puree-conde}{%
\subsubsection[ピュレ・コンデ]{\texorpdfstring{ピュレ・コンデ\footnote{ブルボン王家の支流にあたる
  Prince de Condée
  (プランスドコンデ)コンデ大公のこと。赤いんげん豆のポタージュにコンデの名称を冠したのは文献上はおそらくヴィアール『帝国料理の本』(1806年)が初出。
  (Potage) A la Condé
  となっている。また、18世紀以前の料理書において赤いんげん豆のポタージュはほとんど見つからない。よく知られているように、いんげん豆はアメリカ大陸原産で16世紀くらいにはフランスに伝えられていたはずだが、広まるのに時間がかかったようだ。さて、ヴィアールのレシピの概要は、1リトロン(≒0.8
  L)の赤いんげん豆をブイヨンで煮る。にんじん2本、玉ねぎ2個、ポタージュの浮き脂少々、クローブ2本を加える。豆が煮えたら裏漉しして滑らかなピュレにする。これをバターで揚げたパンのクルートの上に注いで供する(p.8)。この本にはレンズ豆のピュレのポタージュも続けて掲載されているが、そこにコンティの名はなく、たんに「レンズ豆のピュレのポタージュ」と称されているのみ。作り方上述のコンデとほぼ同じ。ヴィアールでは『料理の手引き』に近い非常にシンプルなレシピだが、カレーム『19世紀フランス料理』第1巻(1833年)の「赤いんげん豆のピュレのポタージュ・コンデ風」は、1
  \(\frac{1}{2}\)
  Lの赤いんげん豆の殻を剥いて洗う。これを大鍋に入れて、ペルドリ1羽、バイヨンヌの生ハム1切れ、にんじん2本、玉ねぎ2個、ブイヨン適量を加える。火にかけて煮ながらアクを取る。ペルドリに火が通ったらすぐに、ハムや他の根菜とともに取り出す。豆は煮汁ごと布で漉す。このピュレをごく標準的な鶏のコンソメに流し入れ、粗く砕いたこしょう
  1つまみ加えて弱火で煮込む。フルノーの端に鍋を置いて弱火で2時間程、アクを取りながら煮込む。その後スープ入れに移し、バターで揚げたクルトンを入れておいた各自のスープ皿に供する(p.144)。この本では赤いんげん豆のポタージュには「コンデ風」の名称が付けられているが、その次のレシピは「白いんげん豆のピュレのポタージュ」というだけの単純な名称になっている。ブルジョワ料理の本として19世紀から20世紀初頭まで版を重ねたオド『女性料理人のための本』第15版(1834年)では早くも「ポタージュ・コンデ風」として簡単にだが赤いんげん豆のピュレのポタージュのレシピが掲載されている。その一方で、レンズ豆を用いたポタージュについては1909年の第97版に至るまでレンズ豆のピュレのポタージュは掲載されているが「コンティ」の名は冠されていない。}}{ピュレ・コンデ}}\label{puree-conde}}

\frsub{Purée Condé}

赤いんげん豆は塩18 gを加えた冷水1 \(\frac{1}{2}\)
Lに入れて火にかける。沸騰したら、しっかりアクを取り\footnote{écumer
  (エキュメ)浮いてくる泡を取り除く、が原義。}、赤ワイン2
\(\frac{1}{2}\)
dLを沸かしてから加える。ブーケガルニ、クローブを刺した玉ねぎ1個、四つ割りに切ったにんじん1本を加えて弱火にして煮込む。いんげん豆がよく煮えたら、煮汁から出して、ブーケガルニと玉ねぎ、にんじんは取り除く。いんげん豆を丁寧にすり潰す。煮汁でのばしてから布で漉し、提供直前にバターを加える。

\hypertarget{puree-conti}{%
\subsubsection[ピュレ・コンティ]{\texorpdfstring{ピュレ・コンティ\footnote{上記コンデ大公家のさらに傍流。王家の分家の分家という扱いになるが、
  Prince de Conti
  (プランスドコンティ)の称号を持つ。ポタージュにコンティの名が冠されたのは、上記コンデの名よりずっと早く、ムノン『宮廷の晩餐』(1755年)第1巻に「ポタージュ・コンティ風」とある。ただしこれはレンズ豆を材料にしたポタージュではなく、スライスした玉ねぎを炒めて煮込み、スープ入れの底にバターで揚げたパン(クルート)を敷いてその上に盛り、刻んだアンチョビを玉ねぎに散らすというもの
  (pp.91-92)。ボヴィリエの『調理技術』(1814年)第1巻では「レンズ豆のピュレのポタージュ・王妃風」と出ている。作り方はえんどう豆のピュレのポタージュと同様にするが、赤レンズ豆を用いて「王妃風」を謳う場合は、上手に煮込んできれいな赤色に仕上げるべし、とある(p.22)。「レンズ豆のポタージュ・コンティ風」の名称が出てくるのはカレーム『19世紀フランス料理』第1巻。1
  \(\frac{1}{2}\) Lの赤レンズ豆 (lentilles à la
  reine)の殻を剥いて洗う。下茹でしたハム、ペルドリ1羽、にんじん2
  本、蕪1個、玉ねぎ2個、ポワロー2本を束ねたものとセロリの根元1株を加え、適量のブイヨンを注いで煮る。アクを取り、3時間弱火で煮込む。根菜、ペルドリ、ハムを取り出してから、レンズ豆を布で漉す。このピュレを普段のとおり作ったコンソメに加える。沸騰したらフルノーの端に鍋を寄せて、浮いてくるアク油脂を取り除きながら澄ませていく。提供直前に、スープ入れに移し、バターで揚げた小さなクルトンを散らす(p.142)。デュボワ、ベルナール『古典料理』(1856年)にはピュレ・コンデもピュレ・コンティも掲載されていないが、グフェ『料理の本』(1867年)では「赤いんげん豆のポタージュ・ピュレ・コンデ」(p.369)と「レンズ豆のピュレ・コンティ」(p.371)がともに掲載されている。このように、ポタージュにおけるコンデとコンティはまったく別々に命名されたものと考えられるため、ブルボン王家の傍流とそのさらに傍流を揶揄したようなものではないと思われる。また、レンズ豆のピュレ自体の歴史は非常に古く、1660年ピエール・ド・リュヌ『新料理の本』においてPotage
  de nantilles
  としてレンズ豆を煮込んで潰したもののレシピが掲載されている(p.315)。
  nantilles
  という表現は誤植ではなく、17、18世紀の料理書においてしばしば見られる表現で、もちろんレンズ豆を意味する。裕福な、の意である形容詞nantiをレンズ豆lentillesをかけた造語であり、レンズ豆の形状が硬貨に似ているところから連想されたものと思われる。また、レンズ豆は地中海世界で農業が始まった頃からの古い作物であり、聖書にも出てくる。詳しくは\protect\hyperlink{garniture-conti}{ガルニチュール・コンティ}訳注参照。}}{ピュレ・コンティ}}\label{puree-conti}}

レンズ豆は欠けたものや割れたものを取り除いて大きさを揃え、
\(\frac{3}{4}\)
Lを軽い\protect\hyperlink{consomme-blanc-simple}{コンソメ}1
Lにさいの目に切って下茹でした塩漬け豚バラ肉を加えて煮込む。乾燥豆を煮る際の標準的な香味野菜を加える。レンズ豆を取り出して水気をきり、香味野菜は取り除く。レンズ豆をすり潰して、茹で汁でピュレをのばし、布で漉す。

\protect\hyperlink{consomme-ordinaire}{コンソメ}2 \(\frac{1}{2}\)
dLを加えて丁度いい濃度にし、提供直前にバターを加え、セルフイユ1つまみで仕上げる。
\end{recette}
\begin{center}\rule{0.5\linewidth}{\linethickness}\end{center}

\hypertarget{les-poissons}{%
\section{魚料理}\label{les-poissons}}
\begin{recette}
\hypertarget{sole-duglere}{%
\subsubsection[舌びらめ・デュグレレ]{\texorpdfstring{舌びらめ・デュグレレ\footnote{アドルフ・デュグレレ
  Adolphe Dugléré
  (1805〜1884)。カレームのもとで学び、カフェ・アングレやトロワ・フレール・プロヴオンソーで料理長を務めた。\protect\hyperlink{pommes-de-terre-anna}{ポム・アンナ}、\protect\hyperlink{potage-germiny}{ポタージュ・ジェルミニ}、この舌びらめ・デュグレレなどの料理を考案したことで知られる。とりわけこの料理は19世紀中葉に食材として大流行していたトマトを用いている点で、時代性をよく表わしている。また、小説家アレクサンドル・デュマ(1802〜1870)の『料理事典』(1882年版と1883年版があるが、いずれも死後出版。前者は「選集」。他の著作からの無断引用が多く、食文化史の史料としてはあまり重要視されていない)の編纂に助力したとも言われている。}}{舌びらめ・デュグレレ}}\label{sole-duglere}}

\frsub{Sole Dugléré}

\index{sole@sole!duglere@--- Dugléré}
\index{duglere@Dugléré!sole@sole ---}
\index{したひらめ@舌びらめ!てゆくれれ@--- ・デュグレレ}
\index{てゆくれれ@デュグレレ!したひらめ@舌びらめ・---}
\index{そーる@ソール ⇒ 舌びらめ!てゆくれれ@---・デュグレレ}

基本的に、この調理をする魚はトロンソン\footnote{tronçon
  筒切り、の意で、うなぎなどは文字通りにやや長めのぶつ切りにすることを言うが、チュルボなどのような平らな魚の場合には、まず縦2つに切ってから、骨の方向に添うようにいくつかに切り分ける。}に切っておくべきなのだが、舌びらめの場合は例外的に丸ごと1尾で調理してかまわない\footnote{このレシピは舌びらめをフィレではなく丸ごと1尾で調理する節に含まれていることに注意。}。

舌びらめはバターを塗った平鍋に入れる。玉ねぎ \(\frac{1}{2}\)
個とエシャロット2個はみじん切りにし、トマト2個は皮を剥いて潰してからざく切りにして加える。パセリのみじん切り少々と塩、こしょう、白ワイン大さじ数杯を加える。弱火で沸騰させないよう火を通し\footnote{pocher
  (ポシェ)。【参考】\textbf{ごく少量のクールブイヨンを用いたポシェ}\ldots{}\ldots{}この火入れの方法は主としてチュルボタン、バルビュ、舌びらめ丸ごとでもフィレでも用いらる。バターを塗った天板あるいはソテー鍋に魚丸ごとあるいはそのフィレを置き、軽く塩をして、所要量の魚のフュメかマッシュルームの煮汁を注ぐ。フュメとマッシュルームの煮汁を合わせたものを用いる場合もある。蓋をして、中温のオーヴンに入れる。魚丸ごとの場合は時折煮汁をかけてやる(原書
  pp.279-280)。}、皿に盛り付ける。

舌びらめの煮汁を煮詰める。これに\protect\hyperlink{veloute-de-poisson}{魚料理用ヴルテ}大さじ2〜3杯を加えてとろみを付ける。仕上げにバター30
gとレモン果汁少々を加え、舌びらめに覆いかける。

\hypertarget{coulibiac-de-saumon-a}{%
\subsubsection{サーモンのクリビヤック A}\label{coulibiac-de-saumon-a}}

\frsub{Coulibiac de Saumon A}

\index{saumon@saumon!coulibiac a@Coulibiac de --- A}
\index{coulibiac@coulibiac!saumon a@--- de Saumon A}
\index{さけ@鮭 ⇒ サーモン}
\index{さーもん@サーモン!くりひやつくa!サーモンのクリビヤックA}
\index{くりひやつく@クリビヤック!さーもんa!サーモンの--- A}

(材料)

\begin{itemize}
\item
  砂糖を加えずにやや固めに作った標準的なブリオシュ生地約1
  kg(\protect\hyperlink{pate-a-brioche}{ブリオシュ生地}参照)。
\item
  サーモン650 gは線維と垂直に1
  cm程度の厚さにスライスし、バターで色付かないよう表面を焼き固め\footnote{raidir
    (レディール)素材の表面を色付けないように強火でさっと焼いて表面を焼き固めること。語義としては「焼く」限定されるものではなく、熱湯などの液体を用いる場合もある。}て冷ましておく。
\item
  マッシュルーム 75
  gと玉ねぎ(中)はみじん切りにし、バターで炒めて冷ましておく。パセリのみじん切り大さじ1杯強を加えておく。
\item
  \protect\hyperlink{kache-de-semoule-pour-coulibiac}{セモリナ粉のカーシャ}200
  gまたはコンソメで茹でた米200
  g(「\protect\hyperlink{garnitures}{ガルニチュール}」\protect\hyperlink{kache-de-semoule-pour-coulibiac}{カシャ}参照)。
\item
  固茹で卵2個のみじん切り。卵白、卵黄は分けなくていい。
\item
  茹でたヴェジガ(後述参照)500 g(乾燥状態のヴェジガ90
  gが必要)。乾燥ヴェジガは最低5時間冷水に漬けてもどし、\protect\hyperlink{consomme-blanc-simple}{白いコンソメ}か湯で3時間半茹でてから、粗くみじん切りにしておく。
\end{itemize}

(作業手順)

ブリオシュ生地を長さ32〜35
cm、幅18〜20cmの長方形に\ruby{伸}{の}す。中央に「パンタン\footnote{一般的には板などで出来た色とりどりの操り人形のことだが、料理においては、豚肉のファルスを詰めた正方形または楕円型の小さなパイ包み焼きのこと。ファルスにはトリュフを混ぜ込むこともある。ただし、ここではサーモンを1cm厚程度の薄切りにしているため、前者のイメージのほうが正しく伝わると思われる。}」のように具を詰めていく。カーシャまたは米とサーモン、みじん切りにしたヴェジガ、卵、マッシュルームと玉ねぎの層を順に重ねていくわけだ。最後はカーシャの層になるようにする。

生地の端を軽く濡らして、生地の両端が詰め物の中心に来るようにしてつなぎ合わせる。こうして成形したクリビヤックを裏返して、継ぎ目が下になるように天板にのせる。

暖い場所に置いて、25分間生地を醗酵させる。

最後に、溶かしバターを刷毛でクリビヤックに塗り、細かいパン粉を上から振りかける。加熱中に蒸気が抜けるように上面に切れ目を入れて穴を空けてやる。中温のオーブンでとりわけ炉床の温度の強い状態で焼く。

焼成時間\ldots{}\ldots{}45分間。

クーリビヤックをオーブンから出したら、溶かしバターをスプーン数杯、中に流し込んでやること。

\hypertarget{note-sur-vesiga}{%
\subparagraph{【ヴェジガについて】}\label{note-sur-vesiga}}

ヴェジガとはすなわちチョウザメの脊髄のことで、ロシア料理のいくつかの品でしか用いられないものだ。これは市場で入手可能で、リボン状のゼラチンのような見た目で、質感は魚膠のような感じだ。いろいろな方法で水で戻して火を通して試した結果、

\begin{enumerate}
\def\labelenumi{\arabic{enumi}.}
\item
  ヴェジガを冷水に漬けて普通にもどすのにかかる時間は5時間程度。
\item
  その程度の時間水でもどすと、だいたい5倍の量になる。さらに長い時間漬けておけばもっと嵩も重さも増すが、実際のところ5時間で充分。
\item
  乾燥ヴェジガ10 gは水で戻すと52〜55 gということになる。
\item
  乾燥ヴェジガを水でもどしてから茹でるのに必要な液体の量は、ヴェジガ
  260〜270gにつき 3 L必要。加熱はごく弱火で、蓋をしてすること。
\item
  ヴェジガの小さな切れ端を茹でる場合はせいぜい3時間半〜4時間半でいい。
\end{enumerate}
\end{recette}
\begin{center}\rule{0.5\linewidth}{\linethickness}\end{center}

\hypertarget{ux8089ux6599ux7406}{%
\section{肉料理}\label{ux8089ux6599ux7406}}
\begin{recette}
\hypertarget{boeuf-a-la-mode}{%
\subsubsection[ブフアラモード]{\texorpdfstring{ブフアラモード\footnote{à
  la mode
  (アラモード)元来は「流行の、おしゃれな」の意だが、この料理名については日本語の「プリンアラモード」と同様に、本来の意味が失なわれて、料理名として定着していると考えるのがいいだろう。
  Boeuf à la bourgeoise
  (ブフアラブルジョワーズ)ブルジョワ風とも呼ばれる。後者の料理名から考えると、産業革命の進展につれてブルジョワ階級が台頭してきた時代、すなわち18世紀末〜19世紀初頭の「流行」と見ることも出来なくはないが、その後も料理内容にほぼ変化がないままこの名称で作られ続けているので、上述のように料理名本来の意味は失なわれていると見るべき。牛イチボ肉の塊と小玉ねぎを用いるこの料理の原型とも言えるべきものは18世紀ムノン『ブルジョワ料理』に見出せるが,料理名に「ア・ラ・モード」の表現はない。一方、同じく18世紀マラン『食の贈り物』には「ブフ・ア・ラ・モード」の料理名が見られる。カレーム『19世紀フランス料理』の「
  ブフ・ア・ラ・モード ブルジョワ風」は『ル・ギード・キュリネール』のものと非常に近い内容であり、遅くともカレームの時点で料理としてほぼ完成していると考えられる。}}{ブフアラモード}}\label{boeuf-a-la-mode}}

\frsub{Boeuf à la mode}

\index{boeuf@boeuf!mode@--- à la mode}
\index{piece de boeuf@pièce de boeuf!mode@Boeuf à la mode}
\index{mode@mode (à la)!boeuf@Boeuf à la mode}
\index{うしかたまりにく@牛塊肉!もーと@ブフアラモード}
\index{ふふ@ブフ!もーと@ブルアラモード}
\index{もーと@モード!ふふあらもーと@ブフアラモード}
\index{あらもーと@アラモード!ふふ@ブフ---}

作業しやすいよう、2.5〜3kgを越えない程度のイチボ肉を用いる。この重量で約20人分となる。

豚背脂350
gをコニャックで20分間マリネし、こしょう、香辛料で味つけし、直前に刻んだパセリをまぶす。これをラルデ針でイチボ肉に刺し込む。

塩、こしょう、ナツメグ少量を肉にすり込む。これを赤ワイン \(\frac{1}{2}\)
本とコニャック1 dLで5〜6時間マリネする。

通常の方法でブレゼするが、煮汁にマリネ液を加える。さらに仔牛の足を小さいものなら3本、中位のものなら2本、骨を外して下茹でし、紐で縛って、鍋に入れる。

\(\frac{3}{4}\)
程度火が通ったら、ひとまわり小さな鍋に肉を移す。小さなさいの目か長方形に切った仔牛の足と、バターで色よく炒めた小玉ねぎ400
g、オリーヴ形に整形し固めに茹でたにんじん600
gを肉の周囲に入れる。煮汁をシノワで漉してから浮き脂を取り除く。これを肉の入った鍋に注ぎ、弱火で火入れを仕上げる。

塊肉を皿に盛り、周囲につけあわせの野菜と仔牛の足を種類ごとにまとめてブーケのように飾る。ほどよく煮詰めた煮汁をかける。

\hypertarget{haricot-de-mouton}{%
\subsubsection[羊のアリコ]{\texorpdfstring{羊のアリコ\footnote{haricot
  (アリコ)は現代フランス語ではもっぱら、いんげん豆、さやいんげんを意味するが、中世フランス語においてはある種の「煮込み料理」を意味した。14世紀末に成立されたとされる手稿本『ル・メナジエ・ド・パリ』における「羊のアリコ」のレシピには当然ながらいんげん豆は使われていない。そもそもいんげん豆はアメリカ大陸原産なので、フランスに伝播して広まるのは16世紀以降のこと。にもかかわらず、かつて豆の代表であったえんどう豆が現代ではもっぱら若どりのプチポワでの利用が中心となった一方で、いんげん豆は若どりのさやいんげんも乾燥豆、さらに半乾燥のものも非常に好まれる食材となっている。}}{羊のアリコ}}\label{haricot-de-mouton}}

\frsub{Haricot de Mouton}

\index{haricot@haricot!mouton@--- de mouton}
\index{mouton@mouton!haricot@Haricot de ---}
\index{ありこ@アリコ!ひつし@羊の---}
\index{ひつし@羊!ありこ@---のアリコ}

豚ばら肉の塩漬け250
gは大きめのさいの目に切って下茹でし、小玉ねぎ20個とともにラードでこんがり炒める\footnote{faire
  revenir (フェールルヴニール)。}。これらを取り出して、同じ鍋で羊の胸肉、首肉、肩肉をラグー用に切ったもの2
kgを色よく焼く\footnote{risoller
  (リソレ)油脂を熱した鍋などで肉の表面にこんがり焼き色を付けること。≒
  faire revenir}。

肉の表面ががこんがり焼けたら、鍋の脂の半分は取り出す。潰したにんにく3
片と小麦粉40 gを加えてさらに加熱する。

水1
Lを注ぎ入れ、塩こしょうで調味し、ブーケガルニを加える。混ぜながら沸騰させた後、弱火で30分程煮込む。

肉を別の鍋に移して、先に炒めた塩漬け豚ばら肉と小玉ねぎを加える。半ば火を通した状態の白いんげん豆1
Lを加える。先の煮汁を全体にかけてやり、弱火のオーブンに入れて火入れを仕上げる。

小さな陶製の器に入れて供する。

\hypertarget{poularde-albufera}{%
\subsubsection{肥鶏 アルビュフェラ}\label{poularde-albufera}}

\frsub{Poularde Albuféra}

\index{poularde@poularde!albufera@Albuféra}
\index{albufera@Albuféra!poularde@Poularde ---}
\index{ひとり@肥鶏!あるひゆふえら@---・アルビュフェラ}
\index{ふーらると@プーラルド ⇒ 肥鶏!あるひゆふえら@---・アルビュフェラ}
\index{あるひゆふえら@アルビュフェラ@ひとり!肥鶏・---}

フォワグラと大きめのさいの目に切ったトリュフを米と合わせ、肥鶏に詰め物する。肥鶏を\protect\hyperlink{les-poches}{ポシェ}する。

皿に盛り、ソース・アルビュフェラを塗る。

周囲に次のものを盛り込む。くり抜きスプーンで丸く抜いたトリュフ、同様に丸く整形した鶏のクネル、小さめのマッシュルーム、雄鶏のロニョン。これらの\protect\hyperlink{garniture-albufera}{ガルニチュール}は\protect\hyperlink{sauce-albufera}{ソース・アルビュフェラ}であえておく。

\protect\hyperlink{saumure-liquide-pour-langues}{赤く漬けた舌肉}を鶏のとさか形に切って皿の縁を飾る。
\end{recette}
\begin{center}\rule{0.5\linewidth}{\linethickness}\end{center}

\hypertarget{ux30c7ux30b6ux30fcux30c8}{%
\section{デザート}\label{ux30c7ux30b6ux30fcux30c8}}
\begin{recette}
\hypertarget{cerises-jubilee}{%
\subsubsection[さくらんぼのジュビレ]{\texorpdfstring{さくらんぼのジュビレ\footnote{戴冠式、の意。さくらんぼに
  Napoléon という品種があるので、それを使って Jubilée de Napoléon
  と洒落た名称にすることも可能だろう。}}{さくらんぼのジュビレ}}\label{cerises-jubilee}}

\frsub{Cerises Jubilée}

大きさの揃った立派ななさくらんぼの種を抜く。シロップでやや低めの温度で火を通し\footnote{pocher
  (ポシェ)}、銀製の深皿に盛る。シロップを煮詰め、少量の冷水で溶いたアロールート\footnote{南米産のクズウコンから採れる良質のでんぷん。一般的にはコーンスターチで代用する。}を加えてとろみを付ける。比率はシロップ3
dLに対してアロールートがスプーン\(\frac{1}{2}\)杯。もしくは\protect\hyperlink{gelee-de-groseilles-a}{グロゼイユのジュ}を用いる。

さくらんぼにとろみを付けたシロップをかけ、デザートスプーン1杯の温めたキルシュ酒を注ぎ、提供直前に火を点ける。

\hypertarget{timbale-d-arenberg}{%
\subsubsection[タンバル・アーレンベルク]{\texorpdfstring{タンバル\footnote{もとは「小太鼓」を意味する語で、円筒形の仕立てによく命名される。あくまでも形状を指す言葉であって、料理やパティスリの種類を意味しているわけではないことに注意。また、野菜料理を盛る深皿もタンバルと呼ばれ、混同しやすいので注意。}・アーレンベルク\footnote{Arenberg
  とも綴り、現在のドイツ東部の地名。または18世紀末までアーレンベルク公国を治めていたアーレンベルク家のこと。}}{タンバル・アーレンベルク}}\label{timbale-d-arenberg}}

\frsub{Timbale d'Aremberg}

バターを塗ったシャルロット型\footnote{型の口(上部)がやや広くなった円筒形の型。側面に刻み模様や波形模様の付いたものもある。}に、やや固めに作ったブリオシュ生地を敷き詰める。

四つ割りにしてバニラ風味のシロップで少し固めに煮た洋梨とアプリコットのマーマレードの層を交互に敷き詰めていく。

同じブリオシュ生地でタンバルに蓋をする。周囲を軽く湿らせてからしっかり生地を貼り付かせる。中央に、蒸気抜きの小さな穴を空けておく。中温のオーブンで約40分間焼く。

オーブンから出したら、型から外して皿に盛り、マラスキーノ酒\footnote{marasquin
  (マラスカン)。マラスカという品種のさくらんぼで作ったリキュール。}風味のアプリコットソースをかけて供する。

\hypertarget{sauce-a-l-abricot}{%
\subsubsection{アプリコットソース}\label{sauce-a-l-abricot}}

\frsub{Sauce à l'Abricot}

よく熟したアプリコットを目の細かい網で裏漉しする。またはアプリコットのマーマレードを使う。28°Béのシロップでアプリコットのピュレをのばす。沸騰させて浮いてくる泡を丁寧に取り除く。スプーンをコーティングする程度の漉さになったら火から外し、アーモンドミルクかマデイラ酒、マラスキーノ酒で香り付けする。(pp.793-794)
\end{recette}

% 原稿ファイル読み込み
\hypertarget{avant-propos}{%
\chapter{序}\label{avant-propos}}

\fifteenq
\setstretch{1.3}

もう20年も前のことだ。本書の着想を我が尊敬する師、今は亡きユルバン・デュボワ\footnote{Urbain
  Dubois (1818〜1901)。19世紀後半を代表する料理人。}先生に話したのは。先生は\ruby{是非}{ぜひ}とも実現させなさいと強く勧めてくださった。けれども忙しさにかまけてしまい、\ruby{漸}{ようや}く
1898年になって、フィレアス・ジルベール\footnote{Philéas Gilbert
  (1857〜1942)。19世紀末から20世紀初頭に活躍した料理人。料理雑誌「ポトフ」を主宰した。}君と話し合い協力をとりつけることが出来た。ところがまもなく、カールトンホテル開業のために私はロンドンに呼び戻され、その厨房の準備や運営に忙殺されることとなった\footnote{エスコフィエはセザール・リッツの経営するホテルグループにおいて料理に関わる重要な役割を一手に担っていた。1890年〜1897年にかけてロンドンのサヴォイホテルの総料理長を勤めた後、1898年にはパリのオテル・リッツの、1899年にはロンドンのカールトンホテルの開業に携わり、1920
  年までカールトンホテルで総料理長を務めた。}。本書の計画を実現させるために落ち着いた時間を取り戻さねばならなくなってしまった。

1898年から放置したままだった本書に再び着手出来たのは、多くの同僚たる料理人諸君の助力と、友人でもあるフィレアス・ジルベール君とエミール・フェチュ\footnote{Emile
  Fétu 生没年不詳。}君の献身的な協力を得られたからに他ならない。この一大事業を完成させることが出来たのは、ひとえに皆の励ましと、とりわけ辛抱強く、粘り強く仕事を手伝ってくれた二人の共著者\footnote{ジルベールとフェチュを指しているが、初版には、この二人の他にも共著者として4人の名が挙げられている。第二版以降は共著者としてジルベールとフェチュの名しかクレジットされていない。第二版は初版から構成も含め大幅な改訂が行なわれた。その作業を実際に行なったのがジルベールとフェチュだったために、他の共著者のクレジットが抹消されたと考えられる。なお、現行の第四版にはエスコフィエの名しかクレジットされていない。}のおかげだ。

私が作りたいと思ったのは立派な書物というよりはむしろ実用的な本だ。だから、執筆協力者の皆には、作業手順を各自の考えにもとづいて自由にレシピを書いてもらい、私自身は、40年にわたる現場経験に即して、少なくとも原理原則、料理における伝統的基礎を明確に説明するのに専念した\footnote{字句どおりにとれば、各章、各節における「概説」に相当する部分と、「原注」はもっぱらエスコフィエ自身の手になるものであると解釈されよう。ただし、口頭によるコメントの「聞き書き」的なものも含まれていることは原書の文体における「ゆらぎ」から推測することは可能。}。

本書は、かつて私が構想したとおりとは言い難い出来だが、いずれはそうなるべく努めねばなるまい。それでもなお、現状でも料理人諸君にとって大いに役立つものと信じている。だからこそ、本書を誰にでも、とりわけ若い料理人にも買える価格にした\footnote{1903年の初版の売価は、\href{http://gallica.bnf.fr/ark:/12148/bpt6k65768837}{フランス国立図書館蔵}のものの表紙には、フランス国内で12フランと記したシールが貼られている。また、\href{https://archive.org/details/b21525912}{リーズ大学図書館蔵の第二版}にも同様に国内売価12フランのシールが貼られている。1912年の第三版も同じく12フランだった(\href{http://gallica.bnf.fr/ark:/12148/bpt6k96923116}{フランス国立図書館蔵}のものに価格を示すシールはないが、訳者個人蔵のものには12フランと記されたシールが貼られている)。なお、辻静雄は「1903年の初版発売当時は、800ページでたった8フラン、全く破格の値段だった」(「エスコフィエ 偉大なる料理人の生涯」、『辻静雄著作集』、新潮社、1995年、729〜730頁)と記しているが、その数字の典拠は示されていない。現在と当時の通貨価値、物価の違いが分りにくいため、この「破格に安い」という言葉にはやや疑問が残るだろう。1900年当時の書籍広告において『料理の手引き』初版と同様の八折り版800ページの料理書が、フランス装10フラン、厚紙の表紙のものが11フランとあるため、初版の12フランという価格は、むしろ料理書としては一般的だったと考えられる。つまり、豪華本ではなく、普通に利用できる料理書だということを強調しているに過ぎないと解釈すべきところだろう。なお、八折り判というのは書籍の大きさを表す用語で、概ね縦20〜25
  cm、横12〜16
  cm程度。この序文でことさらに「実用性」や入手しやすい価格であることが強調されているのは、何度も言及されているデュボワとベルナールの名著『古典料理』が四折り判(概ね縦45
  cm、横30 cm)の豪華本であったことを意識していたためとも推測されよう。}。そもそも若い料理人諸君にこそこの本を読んで
\ruby{貰}{もら}いたい。今はまだ初心者であったとしても、20年後には組織のトップに立つべき人材なのだから。

私はこの本を豪華な装丁の\footnote{かつてフランスでは、大判の紙の両面に印刷して折ったものを糸で綴じただけの状態(いわゆる「フランス装」)で販売された本を、書店で買い求めた者が別途、業者に製本、装丁させることが一般的に行なわれていた。}、書棚の飾りのごときにはして欲しくない。そうではなく、いつでも、どんな時でも手元に置いて、分からないことを常に明らかにしてくれる\ruby{盟友}{めいゆう}として欲しい。

本書には五千を越える\footnote{初版、第二版は「五千近い」。第三版になってようやくこの表現になった。}レシピが掲載されているが、それでも私は、この教本が完全だとは思っていない。たとえ今この瞬間に完璧であったとしても、明日にはそうではないかも知れぬ。料理は進化し、新しいレシピが日々創案されている。まことにもって不都合だが、版を重ねる毎に新しい料理を採り入れ、古くなってしまったものは改善せねばなるまい。

ユルバン・デュボワ、エミール・ベルナール\footnote{Emile Bernard
  (1827〜1897)。クラシンスキ将軍の料理人を務めた。}両氏の著作\footnote{デュボワとベルナールの共著は他にもあるが、ここでは『古典料理』(1856年)を指している。}に昔から慣れ親しみ、その巨大な影がなおも料理の地平を覆い尽している現在、私としては本書がその後継になって欲しいと思っている。カレーム以後、最高の料理の高みに逹した二人に対し、ここであらためて心から敬意を表させていただきいと思う。

調理現場を取り巻く諸事情により、私は、デュボワ、ベルナール両氏がもたらしたサービス(給仕)面での革新\footnote{\protect\hypertarget{service-russe}{19世紀後半に一般的となった
  「ロシア式サービス」のこと。中世以来、格式の高い宴席では、卓上に大
  皿の料理が一度に何種も並べられ、食べる者がそれぞれ好きなように取り
  分けていた。そして卓上の料理がほぼなくなると、また何種類もの皿が卓
  上に並べられる、というのが数回繰り返された。19世紀中頃から、献立を
  食べる順に1種ずつ、大皿料理の場合は食べ手に見せて回ってから、給仕
  が取り分けて供する方式に変えたものがロシア式サービスである。これと
  対比するかたちで旧来の方式をフランス式サービスと呼ぶようになった。
  ロシア式サービスでは、食卓に大皿を並べない代わりに、花を飾りナフキ
  ンを美しく折るなどの工夫により卓上も洗練されたものとなっていった。
  19世紀パリに駐在していたロシア帝国の外交官クラーキンが提唱したと言
  われている。デュボワとベルナールの『古典料理』序文において詳述され
  ている。}}に対し、こんにちのようなとりわけスピードが重視される目まぐるしい生活リズムに合わせて、大きな変更を加えざるを得なかった。そもそも物理的理由から、料理を載せる飾り台\footnote{\protect\hypertarget{socle}{socle ソークル。パンや米、ジュレな
  どで作った、料理を盛り付けるために銀の盆の上に据える飾り台。カレー
  ムの時代、つまり19世紀前半にはその装飾に凝ることが多かった。食べも
  ので作られてはいるが、料理の一部ではなく、あくまで装飾的要素でしか
  なかった。この飾り台はロシア式サービスの時代になっても豪華絢爛たる
  宴席においては重要なものとして扱われており、デュボワとベ}ルナール『古典料理』でも相応のページ数を割いて説明がなされている。}をやめて、シンプルな盛り付けにする新たなメソッドと新たな道具を考案する必要があったのだ。デュボワ、ベルナール両氏が推奨した壮麗な盛り付けを私自身も行なっていた頃はもちろん、今なお二方の思想にはまったく共感している。冗談でこんなことを言っているのではない。しかし、カレームを信奉する者たちは、装飾の才があるが\ruby{故}{ゆえ}に、時代にもはや\ruby{似}{そぐ}わなくなってしまった作品に対して改良を加えようとはしなかった。時代に合わせて改良することこそ、まさに重要なのに。本書で奨励している盛り付けは、少なくともそれなりの期間、有用であり続けると思う。全ては変化する。姿を変える。それなのに、装飾芸術の役割が変化しないと主張するなどとは\ruby{蒙昧}{も
うまい}ではないか。芸術は流行によって栄えるものだし、流行のように移ろいやすいものだ。

だが、カレームの時代にはこんにちと同じく\ruby{既}{すで}にあり、料理が続く限りなくならないだろうものがある。それが料理のベースとなるフォンやストックだ。そもそも、料理が見た目にシンプルになっても料理そのものの価値は失なわれないが、その逆はどうだろう?
人々の味覚は絶え間なく洗練され続け、それを満足させるために料理そのものも洗練されることになる。こんにちの余剰活動が精神におよぼす悪影響に打ち\ruby{克}{か}つためには、料理そのものがいっそう科学的な、正確なものとなるべきなのだ。

その意味で料理が進歩すればする程、我々料理人たちにとって、19世紀、料理の行く末に大きく影響を与えた三人の料理人の存在は大きなものとなるだろう。カレームとデュボワ、ベルナールはともすれば技術的側面ばかり評価されるが、料理芸術の基礎において何よりも優れているのだ。

既に物故した名だけ挙げるが、確かにグフェ\footnote{Jules Gouffé
  (1807〜1877)。著書多数。主著『料理の本(1867年)は前半が家庭料理、後半が高級料理(オート・キュイジーヌ)の2部構成になっており、レシピもまず材料表を掲げた後に調理手順を説明するという現代の書き方に近く、挿絵も多く分りやすい。この『料理の手引き』とともに19世紀後半のフランス食文化史における名著のひとつ。19世紀前半からのヴィアールやオドが版を内容を増補しながら版を重ねたのに対して、この本は再版の際もほとんど異同がない点もまた特徴のひとつ。}、ファーヴル\footnote{Joséph
  Favre
  (1849〜1903)。スイス生まれの料理人で、パリ、ドイツ、イギリス、ベルギー等において活躍した。著書『料理および食品衛生事典』
  (1884〜1895年)。この『事典』に収録されているレシピの数は5,531であり(番号が振られている)、エスコフィエがレシピ数5千という表現にややこだわりを示しているように思われるのも、ほぼ同時代の出版物であるファーヴルの『事典』を意識していた可能性はある。}、エルーイ\footnote{Edouard
  Hélouis(生没年不詳)。イギリスのアルバート王配(ヴィクトリア女王の夫)(1819〜1861)やイタリアのヴィットーリオ・エマヌエーレ二世(1820〜1878)に仕えたという。著書『王室の晩餐』(1878年)。}、ルキュレ\footnote{『実践的料理』(1859年)の著者C.
  Reculetのこと。}はとても素晴らしい著作を残した。だが、『古典料理』という\ruby{稀代}{きたい}の名著に\ruby{比肩}{ひけん}し得るものはひとつとしてない。

料理人諸君に、新たに本書を使っていただくにあたり、言うべきことがある。いろいろな料理書、雑誌を読み散らかすのもいいが、偉大な先達の不朽の名著はしっかり読み込むように、と。\ruby{諺}{ことわざ}にあるように「知り過ぎることなはい」のだ。学べば学ぶ程、さらに学ぶべきことは増えていく。そうすれば、柔軟な思考が出来るようになり、料理が上達するためのより効果的な方法を知ることも出来るだろう。

本書を\ruby{上梓}{じょうし}するにあたって\ruby{唯}{ただ}ひとつ望むこと、切に願う\ruby{唯一}{ゆいいつ}のことは、上記の点において、本書の対象たる読者諸君が我が\ruby{言}{げん}に耳を傾け、実践するさまを見ることに尽きる。\nopagebreak

\begin{flushright}
A. エスコフィエ \nopagebreak
\end{flushright}

1902年11月1日

\newpage

\hypertarget{introduction-deuxieme-edition}{%
\section[第二版序文]{\texorpdfstring{第二版序文\footnote{この第二版序文は文体が初版序文と異なり、とりわけ前半部分については、いわゆる「悪文」と見なさざるを得ないものとなっている。また、前半と後半でも文体の「ゆらぎ」のようなものが認められる。内容から判断するかぎり、エスコフィエ自身の言葉であることは確かだが、末尾に署名がなく日付のみ記されていることも含めて考えると、ジルベールとフェチュによる「聞き書き」によって作成された可能性も完全には否定できないと思われる。}}{第二版序文}}\label{introduction-deuxieme-edition}}

\normalsize
\setstretch{1.1}
\vspace*{1\zw}

ここに第二版を上梓するに至ったわけだが、二人の共著者による熱意あふれる仕事のおかげで、私の強い期待をさらに越える本書の成功が約束されたも同然だろう。だからこそ、共著者両君および本書の読者諸君に心からの謝辞を申しあげる次第だ。また、ありがたいことに、称賛の言葉を寄せてくださった方々と、貴重な批判をくださった方々にも御礼申しあげる。批判については、それが正当なものと思われる場合については、本書に反映させるべく努めさせていただいた。

かくも多くの人々に本書を受け入れていただけたことへの謝意を表するには、本書における技術的な価値を高め、初版ではロジカルにレシピを分類しようとしたが故に生じた欠点を解消する他ないだろう。それは、調理理論とレシピを損なうことなしに、本書の計画段階において簡単に済まさざるを得ないと思われたテーマについて\ruby{能}{あた}う限り肉薄することでもある。私たちは本文の見直しをするとともに、多くのレシピを追加した。そのほとんどは調理法と盛り付けにおいて、こんにちの顧客のニーズを\ruby{鑑}{かんが}みて着想したものであり、そのニーズが正当かつ実現可能な範囲において、顧客への給仕のペースが日増しに加速していく傾向をも考慮に入れたものだ。こういった傾向は数年来まさしく際立ってきているが\ruby{故}{ゆえ}に、我々としも常に気を配っておかねばならぬ。

「料理芸術」というものは、その表現形態において、社会心理に左右されるものだ。社会から受ける衝撃に逆らわぬことも必要であり、\ruby{抗}{あらが}
えぬことでもある。快適で安楽な生活がいかなる心配事にも乱されることのないような社会であれば、未来が保証され、財をなす機会もいろいろあるような社会であれば、料理芸術はたゆまぬことなく驚異的な進歩を遂げるだろう。料理芸術とは、ひとが得られる悦びのうちでもっとも快適なもののひとつに寄与しているのだから。

反対に、安穏とした生活の出来ぬ、商工業からもたらされる\ruby{数多}{あま
た}の不安で頭がいっぱいになるような社会において、料理芸術は心配事でいっぱいの人々の心のごく限られた部分にしか美味しさを届けられない。ほとんどの場合、諸事という渦巻きに巻き込まれた人々にとって、食事をするという必要な行為はもはや悦びではなく、辛い義務でしかないのだ。

\ruby{斯}{か}くのごとき生活習慣は\ruby{嘆}{なげ}いていい、\ruby{否}{い
な}、嘆くべきことなのだ。食べ手の健康という観点からも、食べたものを胃が受け付けないという結果になるとしたら、それは絶対に生活習慣が悪いのだ。そういう結果を抑える力は私に出来る範囲を越えている。そういう場合に調理科学が出来ることといえば、軽率な人々に\ruby{能}{あた}うかぎり最良の食べものを与えるという対症療法だけなのだ。

顧客は料理を早く出せと言う。それに対して私たち料理人としては、ご満足いただけるようにするか、失望させてしまうことのどちらかしか出来ない。料理を早く出せという顧客の要求を拒む方法があるとするなら、それ以上の方法で顧客にご満足いただけるようにすることしかない。だから、私たちは顧客の気まぐれの前に折れざるを得ないのだ。これまで私たちが慣れ親しんできた仕事のやり方では、これまでの給仕のスタイルでは、顧客の気まぐれに応えることが出来ぬ。意を決して仕事の方法を改革すべきなのだ。だがひとつだけ、変えてはならぬ、手をつけてはならぬ領域がある。料理ひとつひとつのクオリティだ。それは、料理人にとって仕事のベースとなるフォンや事前に仕込んでおいたストック類がもたらすゆたかな風味に他ならぬ。私たちは既に、盛り付けの領域においては改革に着手した。足手まといにしかならぬ多くのものは既に姿を消したか、いままさに消え去らんとしている。料理の飾り台\footnote{socle
  (ソークル)、\protect\hyperlink{socle}{序p.ii訳注4}参照。}、料理の周囲の装飾\footnote{bordure
  ボルデュール。本書においてもガルニチュールの扱いにおいてこの指示はあるが、19世紀のものと比較するとかなりシンプルな内容になっている。}、飾り串\footnote{hâtelet
  アトレ。一方の端に動物などの姿の装飾の施された銀製の串に、トリュフやクルヴェット(海老)などを事前に別の串(ブロシェット)で焼いてからこの飾り串に刺し直し、それを大きな塊肉や丸鶏、大型の魚
  1尾の料理に刺した。19世紀初頭、カレームの時代に全盛となり、その著書『パリ風料理』において詳述されている。19世紀末まではこの装飾がなされることが多かった。また、その飾り串そのものが美麗な装飾品であるためにコレクションの対象になっていた。}などのことだ。この方向性は推し進められると思う。これについては後述しよう。私たちはシンプルであるということを極限まで追究したい。それと同時に、料理の風味や栄養面での価値を増すことも目指している。料理はより軽い、弱った胃にも優しいものにしたいと考えている。私たちはこの点にのみ尽力したい。料理において役をなさない大部分はすっかり剥ぎ取ってしまいたいと考えているのだ。一言でまとめると、料理は芸術であり続けつつも、より科学的なものとなるだろうし、その作り方はいまだ経験則に基づいただけのものばかりであるが、ひとつのメソッド、偶然などに左右されない正確なものになっていくことだろう。

こんにちは料理の過渡期にある。古典料理メソッドの愛好者はいまなお多く、私たちもそれを理解し、その思想に心から共感するところもある。だが、食事というものがセレモニーであり、かつパーティであった時代を懐しんでどうするというのだ?
古典料理がこんにちの美食家に至福の時を与えるために力を発揮出来る場がどこにあるというのだ?
いったいどうすれば、美食と宴の神コモス\footnote{フランス語 Comus
  (コミュス)。ラテン語では同じ綴りでコムスと読む。ギリシア、ローマ神話における、悦びと美食の神。18世紀の料理本作家マランの主著は『コモス神の贈り物』がタイトル。}に捧げ物を供えるという幸せな機会を毎回得られるのだろうか?だから私たちは本書において、個人的な創作よりむしろ伝統的なフランス料理のレシピ集として、こんにちの料理のレパートリーから姿を消してしまったものも残すことに固執した。その名に値する料理人なら、機会さえ与えられたら王侯貴族も近代の大ブルジョワもひとしく満足させるためには、知っておくべきものなのだ。時間のことなんぞ気にもせぬ穏かな美食家の方々にも、時こそ全てと言わんばかりの金融家やビジネスマンたちにも満足していただくために。だから、本書が新しいメソッドに偏ったものだという非難にはあたらない。私はただ単に、料理芸術の進化の歩みをたどり、いまの時代に即しつつ、食べ手すなわち食事会の主催者と招待客の皆様の意向を絶対的なものとして、それに従いたいと願っているだけなのだ。食べ手の意向に対して私たち料理人は
\ruby{頭}{こうべ}を垂れて従うことしか出来ぬのだから。

私たちは、料理の美味しさを損なうことなくより早く料理を提供できるような方法を、料理人各人が自らの嗜好を犠牲にすることなしに探求すべく
\ruby{誘}{いざな}うことこそが、料理人諸君にとって有益と信じている。全体として、私たちのメソッドはまだまだ日々のルーチンワークに依存し過ぎているものだ。顧客の求めに応えるため、私たちは既に仕事のやり方をシンプルなものにせざるを得なかった。だが、残念ながらいまだ\ruby{途}{み
ち}\ruby{半}{なか}ばに過ぎぬと感じている。私たちは自己の信念をしっかり堅持しており、どうしようもない場合にのみ自説を曲げることもある。だから、装飾に満ちた飾り台を廃止した一方で、盛り付けに時間のかかる厄介で複雑なガルニチュールは残してある。こういったガルニチュールを濫用することはガストロノミーの観点から言って、常に間違っているのは事実だが、残しておくべきものと思われる。それを求める顧客あるいは食事会主催者に絶対に従う必要のある場合はとりわけそうだ。ごく稀にとはいえ、料理の美味しさを損なうことなくそれらを実現可能なこともあるからだ。時間と金銭、広くてスタッフの充実した会場、という3つの本質的要素を最大限活用可能な場合のことだが。

通常の厨房業務においては、ガルニチュールをかなりシンプルな、せいぜい3〜
4種の構成要素からなるものに減らさざるを得なくなっている。そのガルニチュールを添える料理がアントレであれルルヴェ\footnote{19世紀前半まで主流であった「フランス式サービス」つまり、一度に多くの料理の皿を食卓に並べるという給仕方式において、ポタージュを入れた大きな深皿が空くと、それを給仕が下げて、豪華な装飾を施した大きな塊肉の料理がポタージュを置いてあった場所に据えられた。これを
  \protect\hypertarget{releve}{relevé}ルルヴェ(交代したもの、の意)と呼んだ。エスコフィエの時代にはフランス式サービスではなくロシア式サービスに移っており、大きな塊肉の料理や大型の魚1尾まるごとを大皿で出し、給仕が切り分けて配膳するようになっていたが、名称はそのまま残った。Entréeアントレ(もとは「入口」の意)は現代において「前菜」の意味で用いられているが、食卓に大皿で並べられた肉料理(場合によっては魚料理も含む)の総称としてこの語が用いられていた。本書はそれを踏襲している。本書においてルルヴェおよびアントレに分類されている料理の多くは現代においてコース料理の「メイン」に相当するものが多く、実際、英語ではコース料理のメインのことを現在でもこの語で表わすことが多い(前菜はappetizerアペタイザーと呼ぶ)。}であれ、牛・羊肉料理であれ、家禽であれ魚料理であれ、そうせざるを得ない。そのようにして構成要素を減らしたガルニチュールは、素早い皿出しが要求される場合には必ず、ソースと同様に別添で供するのがいい。その場合、盛り付けは奇抜というくらいシンプルなものとなるが\ldots{}\ldots{}メインの料理はより冷めない状態で、より早く、よりきれいに供することが可能になる。給仕が料理を取り皿に分けてお客様に出すにせよ、お客様が大皿を自分たちで受け渡して取り分けるにせよ、サービス担当者は安心して仕事が出来るし、そのほうが容易だ。メインの大皿が山盛りになることはないし、その上に盛り付けられたいろいろな素材のガルニチュールも簡単に取ることが出来るからだ。

こんにちの他のシステムだと、料理を載せるための台や装飾のための飾り串を作り、さらに料理の周囲にガルニチュールを配置するのに、看過出来ぬ程の時間を要していた。こういう盛り付けというのは、料理そのものがさして大きくないものであっても、食べ手の人数が少ない場合であっても、大面積の皿を用いる必要があった。だから、お客様が料理を自分たちで受け渡して取り分ける必要がある場合などは、お客様にとっても、サービス担当者にとってもまことに窮屈なものであった。これは、複雑な構成のガルニチュールの持つ大きな欠点のひとつとして無視できないことだ。他の欠点というのは、あらかじめ盛り付けを行なうことによって美味しさが減じてしまうこと、食べ手が少人数の場合には必然的に、料理を見せて周る間に冷めてしまうこと、などがある。こういう愉快とは言えぬことの結果は何とも情けないことになる。つまり、お客様に大皿に盛り付けた料理をお見せするのはほんの一瞬だけ、お客様は多少なりとも豪華で精密に盛り付けられた料理をちらりと見る暇があるかないか、ということだ。昔日のごとき豪華壮麗な料理を供することの可能な場所もこんにちでは少なくなってきたが、それ以外のところでもこういった悪習が頑固なまでに続けられているというのは、それが昔からの習慣だということでしか説明がつかぬ。

給仕のスピードを容易に上げるために、大きな塊肉の料理でない場合には毎回、下の図のごとき四角形の深皿を出来るだけ用いるよう是非ともお勧めしたい。温かい料理でも、冷製の料理でも、この皿は非常に優れたものであるから、その目的において厨房に備えておくべきものとして他の追随を許さないと言える
\footnote{この段落は、初版の序文の後にある「盛り付け方法をシンプルにすることについて」という挿絵付きの節の内容を短かく縮めたために、ややわかりにくいものになっている。ただし、第二版および第三版においては序文の最後に皿の挿絵が添えられている。}。

繰り返しになるが、本書が新しい方法を勧めているからといって、偏見で古典的なものを悪いと断じているのでは決してない。私たちは、料理人諸君に、顧客たちの生活習慣や味の好みを研究し、自らの仕事をそれらに適合させるよう
\ruby{誘}{いざな}いたいと思っているだけなのだ。我々料理人にとって高名な師とも呼ぶべきカレームは、ある日、同業たる料理人のひとりとおしゃべりをしていた際に、その料理人が仕えている主人の洗練さに欠けた食事の習慣や下卑た味覚を苦々しげに語るのを聞かされたという。その食事の習慣と味覚に憤慨して、自分が人生をかけて追究してきた知的な料理の原則を曲げてまで仕え続けるくらいなら、いっそ辞めてしまいたいと思っている、と。カレームはこう答えた。「そんなことをするのは君のほうが間違っているよ。料理において原則なんていくつも存在しないんだ。あるのはひとつだけ、仕えているお方に満足していただけるか、ということだけなんだよ」と。

今度は我々がその答を考える番だ。自分たちの習慣やこだわりを、料理を出す相手に押しつけるなどと言い張るとしたら、まったくもって馬鹿げたことだ。我々料理人は食べ手の味覚に合わせて料理することこそが第一でありもっとも本質的なことなのだと、私たちは確信している。

私たちがかくも安易に顧客の気まぐれにおもねったり、過度なまでに盛り付けをシンプルにするせいで、料理芸術の価値を下げ、単なる仕事のひとつにしてしまっている、と非難する向きもあるだろう。\ldots{}\ldots{}だがそれは間違いだ。シンプルであることは美しさを排するものではない\footnote{この序文における名句のひとつ。ただし、エスコフィエの時代における「シンプル」とポストモダン以降の時代であるこんにちの「シンプル」はもはや具体的な意味がまったく違うことに留意。もちろん、理念として普遍的な価値を持つ名句であることは確かだろう。}。

ここで、本書の初版において盛り付けについて述べた部分を繰り返すことをお許しいただきたい。

「どんなにささやかな作品にも自らの最高の印をつけられる才というのは、その作品をエレガントで歪みのないものに見せられるわけで、技術というものに不可欠だと私は信じている。

だが、職人が美しい盛り付けを行なうことで自らに課すべき目的とは、食材を他に類のない方法で節度をもって用いつつ大胆に配置することによってのみ、実現されるのだ。未来の盛り付けにおいて絶対に守るべきこととして、食べられないものを使わないこと、シンプルな趣味のよささこそが未来の盛り付けに特徴的な原則となるだろうことを、認めるべきなのだ。

そのような仕事を成し遂げるために、能力ある職人にはいくつもの手段がある。トリュフ、マッシュルーム、固茹で卵の白身、野菜、舌肉などの食べられるものだけを用いて、素晴らしい装飾を組み合わせ、無限に展開できるのだ。

王政復古期\footnote{1814年ナポレオンが退位して国外へ亡命、ルイ18世を戴く王政へ回帰した時期。1830年まで続いたが7月革命でブルボン家は断絶し、その後オルレアン朝による七月王政が1848年まで続いた。}に料理人たちによって流行した複雑な盛り付けの時代は終わった。だが、特殊な例になるが、古い方法で盛り付けをしなければならない場合もあり、そういう時は何よりもまず、盛り付けにかかる時間と利用できる手段を見積らなくてはならない。土台の形状を犠牲にしなくても、装飾の繊細さを忘れなくても、風味ゆたかな素材を軽んじたり劣化させてしまっては、価値のないものにしかならないのだ」。

以上の見解はずっと変わっていない。料理は進歩する(社会がそうであるように)。だが常に芸術であり続けるのだ。

例えば、1850年から人々の生活習慣、習俗が変化したことを皆が認めるにやぶさかでないように、料理もまた変化するのだ。デュボワとベルナールの素晴しい業績は当時のニーズに応えたものだ。だが、たとえ二人がその著書と同じく永遠の存在であったとしても、彼らが称揚した形態は、料理の知識として、我々の時代の要求に応えうるものではない。

私たちは二人の名著を尊重し、敬愛し、研究しなくてはならない。それはカレームとともに、料理人の仕事の\ruby{礎}{いしずえ}たるものだ。だが、書いてあることを盲目的に真似るのではなく、私たち自身で新たな道を切り
\ruby{拓}{ひら}き、私たちもまたこの時代の習俗や慣習に合わせた教本を残すべきと考える次第だ。

\begin{flushright}
1907年2月1日
\end{flushright}

\newpage

\hypertarget{introduction-troisieme-edition}{%
\section{第三版序文}\label{introduction-troisieme-edition}}

\vspace*{1\zw}

『料理の手引き』第三版を同業たる料理人諸賢に向けて上梓するにあたり、絶えず本書を好意的に支持してくださったことと、多くの方々から著者一同にお寄せくださった励ましのお言葉に対し、あらためて深く御礼申しあげる次第だ。

第二版序文の内容につけ加えるべきことは何もない。というのも、第二版序文で料理という仕事について申しあげたことは、1907年当時も今も変わっていない事実だからだし、今後も長くそうであり続けるだろう。とはいえ、この第三版は内容を精査し、かなりの部分を改訂してある。かつては予測でしかなかったことを実証し、この『料理の手引き』初版の序文においてエスコフィエ氏\footnote{この表現から、第三版序文がエスコフィエ自身ではなく、フィレアス・ジルベールかエミール・フェチュのいずれか、あるいは二人によって書かれたと判断される。}が以下のように書かれた約束も果せたと思う。「本書には五千近くもの\footnote{初版および第二版では「五千近い」となっており、第三版で「五千以上」と表現が変更された。}レシピが掲載されているが、それでも私は、この教本が完全だとは思っていない。たとえ今この瞬間に完璧であったとしても、明日にはそうではないかも知れぬ。料理は進化し、新しいレシピが日々創案されているのだ。まことにもって不都合なことだが、版を重ねる毎に新しい料理を採り入れ、古くなってしまったものは改良を加えねばなるまい。」

この言葉が、前回の第二版から300ページを増やしたことの説明となっているわけで、この新版でいくつかの変更を我々が必要と考えた理由でもある。

\begin{enumerate}
\def\labelenumi{\arabic{enumi}.}
\item
  判型の変更\ldots{}\ldots{}あえて判型を大きくすることで、より扱いやすいものとしたこと\footnote{初版および第二版はいわゆる「八折り版」約21.5
    cm×13.5 cmであったのに対し、第三版は約24 cm×16
    cm、つまり現代のB5版よりほんの少し小さめの判型。}
\item
  巻末の目次の組みなおし\ldots{}\ldots{}当初は料理の種類別であったが、本書全体の項目をアルファベット順にまとめたこと\footnote{原文ではTable
    des
    Matière「目次」とあるが、これは巻頭の章を示す目次のことではなく、巻末の「索引」のこと。}
\item
  時代遅れになったと思われるレシピを相当数削除し、その代わりとしてこの数年の間に創案され好評を博したレシピを追加したこと
\end{enumerate}

既に大著であって本書にこれらの変更を加えるために、我々は第二版の巻末に付されていた献立のページを削除せざるを得なかった。

献立についても内容を一新し、多くの献立例を追加して、『メニューの本』という独立した書籍として、この第三版と同時に刊行する予定となっている。この『メニューの本』において我々は献立とその説明文はもちろんのこと、大規模な厨房における日々の業務配分を示す表を入れておいた。

このように別冊とすることで、献立の作成という非常に重要な問題を適切に展開し、ゆとりを持って論じることが可能となったわけだ。

この新刊『メニューの本』は料理人諸賢だけではなくメートルドテル、食事施設の責任者に必携のものとなった。さらには必要なものを奇抜なまでに単純化してしまう家庭の主婦にとっても必携となろう。我々は上記の改良点が、これまで多くの好意的見解をお寄せくださった料理関わる皆様方に、好意的に受け容れていただけると信じている。また、料理芸術の栄光のもと未来に続くモニュメントを建てるべく努めた我々のささやかなる尽力が、料理芸術に利をもたらさんことを信じる次第だ。

\begin{flushright}
1912年5月1日
\end{flushright}

\hypertarget{introduction-quatrieme-edition}{%
\section{第四版序文}\label{introduction-quatrieme-edition}}

\vspace*{1\zw}

『料理の手引き』第三版刊行当時(1912年5月)から後、他の職業、産業と同様に料理界もまた大いなる危機に見舞われた\footnote{第一次世界大戦(1914〜1918)による社会的影響を指している。フランスは戦中から戦後にかけて激しいインフレに見舞われた。なお、この第四版から出版社がそれまでのラール・キュリネールからフラマリオン社に変わった。}。こんにちもなお料理は厳しい試練にさらされている。しかしながら、料理界はその試練に耐えてきたし、戦後のこの辛い時期に終止符を打ち、料理界がさらに前進し始めるのもさして遠いことではないと信じている。だが、目下のところ、あらゆる食材の異常なまでの高騰により、料理長諸賢が責務を果すことがひどく難しくなっている。料理長がその責務を果すということの困難さを経験上よく知っているからこそ、今回の版において我々は、多くのレシピ、とりわけガルニチュールについて、その本質的なところを曲げることなしに、よりシンプルなものにすることにこだわった。

さらに、もはやあまり興味を持たれないであろうレシピは全て削除して、その代わりに近年創案されたレシピを収録することとした。

したがって、料理人諸賢および料理に関心を持つ皆様方に向けてこの『料理の手引き』第四版を上梓するにあたり、旧版同様、皆様に温かく受け容れていただけると信じる次第である\footnote{原書の文体から、この序文も第三版序文と同様に、ジルベールとフェチュによって書かれた可能性も考えられる。}。

\begin{flushright}
1921年1月
\end{flushright}

\newpage
\small
\setstretch{1.0}

\hypertarget{remarque-sur-la-simplification-des-procedes-de-dressage}{%
\section{【参考】盛り付けをシンプルにするということ(初版のみ)}\label{remarque-sur-la-simplification-des-procedes-de-dressage}}

本書では、かつては料理の盛り付けによく用いられた飾り串\footnote{hâtelet
  アトレ。}、縁飾り \footnote{bordure ボルデュール。}、クルトン\footnote{菱形やハート形にしたパンを揚げたもの。}、チョップ花\footnote{papillote
  パピヨット。紙製で、骨付き肉の先端を飾るもの。}などを使う指示がほとんど出てこない。著者としては、盛り付け方法を近代化すると同時に、ほぼ完全に上記のものどもを削除しなくしてしまいたいとさえ考えたくらいだ。

我らが先達が考えていたような盛り付けには、長所がたったひとつしかない。皿を荘厳に、魅力的な姿にすることで、料理を味わう前に、食べ手の目を楽しませ、喜んでいただくということだ。

だが、そうした盛り付けの作業は複雑で難しいものであり、かなりの時間を必要とする。比較的少人数の宴席でないかぎりは、こうした盛り付けは事前に用意しておく必要がある。そのようにして作られた料理は、それを置いておく場所のことを考えに入れないとしても、必ずといっていい程、冷めてしまっている。また、料理を載せる台や縁飾り、飾り串に費す時間も考えなくてはならないし、そういった装飾にかかる費用も考えなくてはならない。忘れてはならないことだが、そのように装飾した皿の見た目の調和がとれている時間というのは、その皿をお客様にお見せする間だけなのだ。メートルドテルのスプーンが料理に触れるやいなや、かくも無惨な姿となりお客様の目には不快なものとなってしまう。こういう不都合はなんとしても改善しなければならなかったのだ。

ここで図に示すような四角形の皿を採用したことで、上記のような問題は解決したと考えている。この皿はパリのリッツホテルで初めて用いられ、ロンドンのカールトンホテルにおいて正式に採用されることとなったものだ。この皿を用いることの利点は絶大で、これを用いない盛り付けなどもはや考えられない程だ。この皿は場所をとらず、皿の内側に盛り付けられた料理は冷めることがない。蓋との距離が近いから保温されているわけだ。魚や肉の切り身は上に重ねて盛るのではなく、ガルニチュールとともに並べて盛り付けることが出来る。そうすることで、最初に給仕されるお客様から最後に給仕される方まで、料理は美味しそうな見た目を保つことが出来るのだ。その結果、クルトンやチョップ花、皿の上にしつらえる料理を載せる台や縁飾り、飾り串、昔の給仕で用いられた面倒なクロッシュ\footnote{cloche
  主に金属製で半球形の保温を目的としたディッシュカバー。}は不要なものとなる。

この皿は冷製料理にもまた便利に使うことが出来る。周囲に氷を積み重ねて囲うか、薄い氷のブロックの上に盛り付ければ、飾りには、ごく繊細なジュレだけていい。そのような繊細なジュレを使うのは昔の方法では不可能だった。かくして、邪魔にさえ思える飾り台も、皿の底の飾りも、アトレも必要なくなった。ショフロワは1切れずつ並べて、周囲を琥珀色のとろけるようなジュレで満たしてやればいい。ムースはもはや「つなぎ」をまったく、あるいはほとんど必要としない。こういうことが、冷製料理の芸術的な見た目を、豪華さや美しさという点でいっかな失なうことなく可能となるのだ。

この新式の什器とそれによって実現可能となる料理に習熟することについて料理人諸君にお報せすることは我々の義務であると考える。利点がとても大きいので、あえて申しあげるが、これを使うことが、給仕を素早く、きれいに、経済的に、そして文句ないまでに実践的なものにする唯一の方法である。

\hypertarget{avertissement-premier-edition}{%
\section{【参考】初版はしがき}\label{avertissement-premier-edition}}

本書はある特定の階層の料理人を対象としているものではなく、全ての料理人が対象であるため、本書のレシピは、経済的観点や料理人が実際に利用可能な手段に応じて、改変できるものだということを述べておきたい。

本書に収められたレシピはすべて、グランドメゾンでの仕事における原則にもとづいて組み立てて調整してある。だから、より格下の店舗などでも、必然的に量を減らせば作れるだろうし、適価で提供出来るようにもなるだろう。

ひとつひとつの項目において、いろいろな飲食を提供する形態を網羅するようにレシピを書くことが不可能だったということは理解されよう。料理人自身が自主性をもって本書の内容を補えるし、そうすべきなのだ。ある者たちにとって非常に大切なことが、大多数の者にとってはそこそこの興味しか引かず、一般的に見たら無益で幼稚に思われることだってあるのだ。

だから、本書に収録したレシピは最大の分量でまとめられたものを考えるべきであり、必要に応じて、各人の判断および物理的に出来る範囲に合わせて、量を減らして作るといい。
\normalsize \setstretch{1.0}


%%% 本文開始
\mainmatter%%%%%本文開始
% \layout%レイアウト数値確認用

%%%原稿ファイル読み込みz
%

%%% Chapitre I. Saucesa
%%%%I. Sauces 
\hypertarget{sauces}{%
\chapter{I. ソース Sauces}\label{sauces}}

\hypertarget{les-fonds-de-cuisine}{%
\section{フォン、その他のストック}\label{les-fonds-de-cuisine}}

\frsec{Les Fonds de Cuisine}

\index{fonds@fonds} \index{ふおん@フォン}

\normalsize
\setstretch{1.0}

本書は実際に厨房で働く料理人を対象としたものだが、まず最初に料理のベー
スとして仕込んでストックしておくもの\footnote{本書での fonds の語は fond
  (基礎、土台)、fonds (資産、資
  本)、そして料理用語として一般に用いられているフォン、のトリプルミー
  ニングになっている。そのまま「フォン」と訳したいところだが、日本語
  の場合「出汁」としての意味合いが強いため、本文中では分りやすさを重
  視してやや冗長に「料理のベースとして仕込んでストックしておくもの」
  のように訳している。}について少々述べておきた い\footnote{この部分は経営者に向けて書かれているようにも読めるが、エス
  コフィエの時代以降、料理人がオーナーシェフとして経営に携わるケース
  が激増したことを考えると、その先見の明に驚かざるを得ない。}。我々料理人にとって重要なものだからだ。

ここで述べる料理のベースとして仕込んでストックしておくものは、実際、料
理の土台そのものであり、それなしでは美味しい料理を作ることの出来ない、
まず最初に必要なものだ。だからこそ、料理のベースとして仕込んでおくストッ
クはとても重要であり、いい仕事をしたいと努めている料理人ほどこれらを重
視している。

これらは、料理において常に立ち戻るべき出発点となるものだが、料理人がい
い仕事をしたいと望んでも、才能があっても、それだけでいいものを作ること
は出来ない。料理のベースを作るにも材料が必要なのだ。だから、必要な材料
は良質のものを自由に使えるようにしなければならない。

筆者としては、むやみな贅沢には反対だが、それと同じくらい、食材コストを
抑え過ぎるのも良くないと考えている。そんなことをしていては、伸びる筈の
才能の芽を摘んでしまうばかりか、意識の高い料理人ならモチベーションの維
持すら出来ないだろう。

どんなに優秀な料理人だって、無から何かを作り出すことは不可能だ。期待さ
れる結果に対して、素材の質が劣っていたり量が足りないことがあれば、それ
でも料理人にいい仕事をしろと要求するなど言語道断である。

料理のベースとして仕込んでおくストックに関するの重要ポイントは、必要な
材料は質、量ともに充分に、惜しげもなく使えるようにすることだ。

ある調理現場で可能なことが、別の調理現場では不可能な場合があるのは言う
までもない。料理人の仕事内容は顧客層によっても変わる。到達すべき目標に
よって手段も変わるということだ。

そういう意味で、何事も相対的なものであるとはいえ、こと料理のベースとし
て仕込んでストックすべきものに関しては絶対に外してはならないポイントが
あるわけだ。組織のトップがこの点で出費を惜しんだり、コスト面で過度に目
くじらを立てるようでは、美味しい料理なんて出来るわけがないのだから、現
実に厨房を仕切っている料理長を批判する資格もない。そんなのが根拠のない
言い掛かりなのは明らかだ。素材の質が悪かったり、量が足りないのであれば、
料理長が素晴しい料理を出せないのは言うまでもあるまい。ぶどうの搾りかす
に水を加えて醗酵させた安ワインを立派な瓶に詰めてしまえば高級ワインにな
ると思う程に馬鹿げたことはないのだ。

料理人は、必要なものを何でも使っていいなら、料理のベースとして仕込んで
おくストックにとりわけ力を入れるべきであり、文句のつけようのない出来に
なるよう気を使うべきだ。そこに手間隙かけていればそれだけ厨房全体の仕事
がきちんと進むのだから、注文を受けた料理をきちんと作れるかどうかは、結
局のところ、料理のベースとなる仕込み類にどれだけ手間\ruby{隙}{ひま}をかけるかという
ことなのだ。

\newpage

\hypertarget{principaux-fonds-de-cuisine}{%
\section{主要なフォンとストック}\label{principaux-fonds-de-cuisine}}

\frsec{Principaux Fonds de Cuisine}

料理のベースとして仕込んでおくべきものは主として\ldots{}\ldots{}

\begin{itemize}
\tightlist
\item
  \textbf{コンソメ・サンプルとコンソメ・ドゥーブル}
\item
  \textbf{茶色いフォン、白いフォン、鶏のフォン、ジビエのフォン、魚のフォン}\ldots{}\ldots{}これらはとろみを付けたジュ、基本ソースのベースになる
\item
  \textbf{フュメ、エッセンス}\ldots{}\ldots{}派生ソースに用いる
\item
  \textbf{グラスドヴィアンド、鶏のグラス、ジビエのグラス}
\item
  \textbf{茶色いルー、ブロンドのルー、白いルー}
\item
  \textbf{基本ソース}\ldots{}\ldots{}エスパニョル、ヴルテ、ベシャメル、トマト
\item
  \textbf{肉料理用ジュレ、魚料理用ジュレ}
\end{itemize}

\vspace{1\zw}

以下も日常的に使う料理のベースとして仕込んでおくものとして扱う。

\begin{itemize}
\tightlist
\item
  \textbf{ミルポワ、マティニョン}
\item
  \textbf{クールブイヨン、肉および野菜用のブラン}
\item
  \textbf{マリナード、ソミュール}
\item
  \textbf{肉料理用ファルス、魚料理用ファルス}
\item
  \textbf{ガルニチュールに用いるアパレイユ}、など\ldots{}\ldots{}
\end{itemize}

\vspace{1\zw}

本書は上記を順に説明していく構成にはなっていない。グリル、ロースト、グ
ラタン等の調理技法についても順を追っていくわけではない。料理の種類ごと
に一定の位置、つまりは関連の深い料理の章の冒頭において説明していくこと
になる。

\vspace{1\zw}

そのようなわけで、本書においては以下のようになる\ldots{}\ldots{}

\begin{itemize}
\tightlist
\item
  フォン、フュメ、エッセンス、グラス、マリナード、ジュレの説明\ldots{}\ldots{}
  \textbf{ 第1章 ソース}
\item
  コンソメおよびそのクラリフィエ、ポタージュの浮き実についての説
  明\ldots{}\ldots{}\textbf{第3章 ポタージュ}
\item
  ファルスとガルニチュール用アパレイユの作り方\ldots{}\ldots{}\textbf{第2章
  ガルニチュー ル}
\item
  クールブイヨン、魚料理用ファルス等\ldots{}\ldots{}\textbf{第6章
  魚料理}
\item
  グリル、ブレゼ、 ポワレの調理理論\ldots{}\ldots{}\textbf{第7章 肉料理}
\end{itemize}

\newpage

\hypertarget{section-grandes-sauces-de-base}{%
\section{基本ソース}\label{section-grandes-sauces-de-base}}

\frsec{Grandes Sauces de Base}

\index{そーす@ソース!きほん@基本---}
\index{sauce@sauce!00grandes@*Grandes ---s de Base}

\begin{itemize}
\item
  \textbf{およびそれらを組み合せたり煮詰めるなどの方法で作る派生ソース}
\item
  \textbf{イギリス風ソース(温製および冷製)}
\item
  \textbf{いろいろな冷製ソース}
\item
  \textbf{ブール・コンポゼ(ミックスバター)}
\item
  \textbf{マリナード}
\item
  \textbf{ジュレ}
\end{itemize}

\hypertarget{osbservation-sur-la-sauce}{%
\section{概説}\label{osbservation-sur-la-sauce}}

ソースは料理においてもっとも主要な位置にある。フランス料理が世界に冠た
るものであるのもひとえにソースの存在によるのだ。だから、ソースは出来る
かぎり手間をかけ、細心の注意を払って作るようにしなければならない。

ソースを作るうえでその基礎となるのが何らかの「ジュ」である\footnote{ここではジュといわゆるフォンが同じ意味で使われている。}。すなわ
ち、茶色いソースは「茶色いジュ」(エストゥファード)から作る。ヴルテ
には「澄んだジュ(白いフォン\footnote{日本の調理現場で「白いフォン」を意味する「フォン・ブラン」は主と
  して鶏のフォンを指すことが多いが、本書で扱われている白いフォンのう
  ち標準的なものは仔牛肉、家禽類をベースとしており、鶏のフォンは別途
  説明されている。})を使う。ソースを担当する料理人はまず
第一に、完璧なジュを作るところから始めなければならない。キュシー侯爵
\footnote{1767-1841。19世紀の著名な美食家。
  著書に『食卓の古典』(1843)があ
  る。料理名にキュシーの名を冠したものも多い。}が言うように、ソース担当の料理人は「頭脳明晰な化学者\footnote{原文
  chimiste。現代は分子ガストロノミーが盛んだが、料理を
  作る過程で起きる現象や結果を「化学」で説明しようとする試みは少なく
  ともカレームまで遡ることが出来る。\protect\hyperlink{fonds-brun}{茶色いフォン}のレシピにおいて言
  及されるオスマゾームという想像上の物質もその範疇に含まれるだろう。
  また、化学の前身たる「錬金術」的概念は中世以来いくつかの料理書にお
  いて散見される。}でありかつ天才
的なクリエイターで、卓越した料理という建造物のいわば大黒柱たる存在」な
のだ。

昔のフランス料理\footnote{本書において「昔の料理」と表現される場合は概ね17〜18世紀末と考え
  ていい。}では、素材に串を刺してあぶり焼きするローストを別に
すれば、どんな料理も「ブレゼ」か「エチュヴェ」のようなものばかりだった。
だが、その時代には既に、フォンが料理という大建築の丸天井の\ruby{要}{か
なめ}だったし、材料コストが重視されるこんにちの我々と比べたら想像も出
来ないくらい贅沢に材料を使ってフォンをとっていたのだ。実際、アンヌ・ドー
トリッシュ\footnote{17世紀に絶対王政を確立したルイ14世の母。}がスペインからルイ13世に嫁いだ際に随行してきたスペインの
料理人たちによってフランス料理にルーを用いる方法が伝えられたが\footnote{ルーがスペインからもたらされたというのは逸話、伝承の域を出ない。}、当時は
ほとんど看過された。ジュそれ自体で充分だったからだ。ところが時代が下り、
料理におけるコストの問題が重視されるようになった。ジュはその結果、貧相な
ものになってしまった。その美味しさを補うものとして、ルーを用いて作るソー
ス・エスパニョルが欠くべからざる存在となった。

ソース・エスパニョルはその完成度の高さゆえに成功をおさめたわけだ。だが、
すぐに当初の目的を越えた使い方をされるようになった。19世紀末には本当に
このソースが必要な場合以外にも使われたわけだ。ソース・エスパニョルの濫
用によって、どんな料理も固有の香りのない、全部の風味の混ざりあったのっ
ぺりとした調子のものばかりになってしまった。

ようやく近年になって、料理の風味がどれも同じようなものであることに批判
が集まってきて、その結果として激しい揺り戻しが起きたのだった。グランド
キュイジーヌでは、透き通ったような薄い色合いでしかも風味のしっかりした
仔牛のフォンが見直されつつある。そのようなわけで、ソース・エスパニョル
それ自体の重要性はだんだん減っていくだろうと思われる。

ソース・エスパニョルが基本ソースとして扱われるべき理由は何か? ソース・
エスパニョルそれ自体に固有の色合いや風味というものはなく、これらはどん
なフォンを用いて作るかで決まる。まさにこの点にソース・エスパニョルの長
所が存するのだ。補助材料としてルーを加えるが、ルーにはとろみを付けると
いう意味しかなく、風味にはまったく寄与しない。そもそも、ソースを完璧に
仕上げるためには、とろみ以外のルーに含まれる成分はソースからほぼ完全に
取り除いてしまっても差し支えはない。不純物を丁寧に取り除いたソースには
ルーに含まれていたでんぷん質だけが残っているわけだ。だから、ソースの口
あたりを滑らかなものにするために必要なのがでんぷん質だけなら、純粋なで
んぷんだけを用いる方がずっと簡単で、作業時間も大幅に短縮されるし、その
結果として、ソースを火にかけ過ぎてしまうようなミスも防げる。将来的には、
小麦粉ではなく純粋なでんぷんでルーを作るようになるかも知れない。

料理界の現状を\ruby{鑑}{かんが}みるに、\ul{ソース・エスパニョル}と
\ul{とろみを付けたジュ}をそれぞれ使い分けざるを得ない。これにはさまざ
まな理由があるが、大きな仕立てのブレゼや、羊や仔羊以外を材料にしたラグー
では、肉汁が煮汁に染み出してきて美味しくなるわけだから、トマトを加えた
ソース・エスパニョルを用いるのがいい。なお、ソース・エスパニョルをさら
に丁寧に仕上げるとソース・ドゥミグラスとなる。これはいろいろなソテーに
不可欠なもので、今後も変わることはないだろう。

一方、牛や羊、家禽を使った繊細で軽い仕立ての料理にはとろみを付けたジュ
の方が好まれる。デグラセの際に少量だけ、料理の主素材と同じものからとっ
たジュを用いる。

こんにちのフランス料理においては、肉とソースの調和がとれているべきとい
う、まことに理に適った厳守すべき決まりがある。

だから、ジビエ料理にはジビエのフォンを用いるか、とりたてて際立った個性
を持たないフォンを用いて作ったソースを添える。牛や羊のフォンは用いない。
ジビエのフォンというのは、さほど濃厚なものを作ることは出来ないが、素材
の個性的な風味を表現するには最適だ。こういった事情は魚料理にも当て
\ruby{嵌}{はま}る。ソースそれ自体が際だった風味を持たないものの場合に
は必ず魚のフュメを加えてやるのだ。このようにしてそれぞれの料理に個性的
な風味を実現させることになる。

もちろん、ここまで述べた原則を実現しようにも、コストの問題がしばしば起
こることは承知している。けれども、熱意のある、他者の評価を意識している
料理人なら問題点を熟考して、完璧とは言わぬまでも満足のいく結果を得るこ
とが出来るだろう。\newpage

\normalsize
\setstretch{1.0}

\hypertarget{traitement-des-elements-de-base}{%
\section{ソースのベース作り}\label{traitement-des-elements-de-base}}

\frsec{Traitement des Éléments de Base dans le Travail des Sauces}

\index{そーす@ソース!そーすつくりのべーす@---のベース作り}
\index{sauce@sauce!Traitement des elements de base dans le travail des sauces@Traitement des Éléments de Base dans le Travail des ---s}
\begin{recette}
\hypertarget{fonds-brun}{%
\subsubsection{茶色いフォン(エストゥファード)}\label{fonds-brun}}

\frsub{Fonds brun ou Estouffade}

\index{ふおん@フォン!ちやいろいふおん@茶色い---}
\index{えすとうふあーと@エストゥファード}
\index{fonds@fonds!brun@--- brun}
\index{fonds@fonds!estouffade@estouffade (fonds brun)}
\index{estouffade@estouffade!fonds brun@ --- (fonds brun)}

(仕上がり10 L分)

\begin{itemize}
\item
  主素材\ldots{}\ldots{}牛すね6
  kg、仔牛のすね6kgまたは仔牛の端肉で脂身を含まな いもの6
  kg、骨付きハムのすねの部分1本(前もって下茹でしておくこと)、
  塩漬けしていない豚皮を下茹でしたもの650 g。
\item
  香味素材\ldots{}\ldots{}にんじん650 g、玉ねぎ650
  g、ブーケガルニ(パセリの枝100 g、 タイム10 g、ローリエ5
  g、にんにく1片)。
\item
  作業手順\ldots{}\ldots{}肉を骨から外す。
\end{itemize}

骨は細かく砕き、オーブンに入れて軽く焼き色を付ける。野菜は焼き色が付く
まで炒める。これらを鍋に入れて14 Lの水を注ぎ、ゆっくりと、最低12時間煮
込む。水位が下がらぬように、適宜沸騰した湯を足すこと。

大きめのさいの目に切った牛すね肉を別鍋で焼き色が付くまで炒める。先に煮
込んでいたフォンを少量加えて煮詰める。この作業を2〜3回行ない、フォン
の残りを注ぐ。

鍋を沸騰させて、浮いてくる泡を取り除く。浮き脂も丁寧に取り除く。蓋をし
て弱火で完全に火が通るまで煮込んだら、布で漉してストックしておく。

\hypertarget{nota-fonds-brun}{%
\subparagraph{【原注】}\label{nota-fonds-brun}}

フォンの材料に牛の骨などが含まれている場合には、事前にその骨だけで12〜
15時間かけてとろ火でフォンをとるといい。

フォンの材料を鍋に焦げ付くくらいまで強く焼き色を付ける\footnote{パンセ
  pincer と呼ばれる手法。原義は「抓む」。材料が鍋底に張り付
  いて、トングなどでしっかり「抓ま」ないと取れないくらい強く焼き付け
  ることからそう呼ばれるようになった。古い料理書では推奨するものも多
  かった。}のはよろしく
ない。経験からいって、丁度いい色合いのフォンに仕上げるには、肉に含まれてい
るオスマゾーム\footnote{19世紀頃、赤身肉の美味しさの本質であると考えられていた想像上の物
  質。赤褐色をした窒素化合物の一種で水に溶ける性質があるとされた。な
  お、当時のヨーロッパではグルタミン酸はもとよりイノシン酸が「うま味」
  の要素であるという概念すらなく、「コクがある」corsé とか「肉汁たっ
  ぷり」onctueux (オンクチュー)や succulent
  (スュキロン)などの表現で肉料理やソースの美味しさが表 現された。}の働きだけで充分だ。

\hypertarget{fonds-blanc}{%
\subsubsection{白いフォン}\label{fonds-blanc}}

\frsub{Fonds blanc ordinaire}

\index{ふおん@フォン!しろい@白い---}
\index{fonds@fonds!blanc ordinaire@--- blanc ordinaire}

(仕上がり10 L分)

\begin{itemize}
\item
  主素材\ldots{}\ldots{}仔牛のすね、および端肉10k
  g、鶏の手羽やとさか、足など、ま たは鶏がら4羽分、
\item
  香味素材\ldots{}\ldots{}にんじん800 g、玉ねぎ400 g、ポワロー300
  g、セロリ100 g、ブー ケガルニ(パセリの枝100
  g、タイム1枝、ローリエの葉1枚、クローブ4本)。
\item
  使用する液体と味付け\ldots{}\ldots{}水12 L、塩60 g。
\item
  作業手順\ldots{}\ldots{}肉は骨を外し、紐で縛る。骨は細かく砕く。鍋に肉と骨を入
  れ、水を注ぎ塩を加える。火にかけ、浮いてくるアクを取り除き香味素材を加
  える。
\item
  加熱時間\ldots{}\ldots{}弱火で3時間。
\end{itemize}

\hypertarget{nota-fonds-blanc}{%
\subparagraph{【原注】}\label{nota-fonds-blanc}}

このフォンは火加減を抑えて、出来るだけ澄んだ仕上がりにすること。アクや
浮き脂は丁寧に取り除くこと。

茶色いフォンの場合と同様に、始めに細かく砕いた骨だけを煮てから指定量の
水を注ぎ、弱火で5時間煮る方法もある。

この骨を煮た汁で肉を煮るわけだ。その作業内容は上記茶色いフォンの場合と
同様。この方法は、骨からゼラチン質を完全に抽出出来るという利点がある。
当然のことだが、煮ている間に蒸発して失なわれてしまった分は湯を足してや
り、全体量を12 Lにしてから肉を煮ること。

\hypertarget{fonds-de-volaille}{%
\subsubsection{鶏のフォン(フォンドヴォライユ)}\label{fonds-de-volaille}}

\frsub{Fonds de volaille}

\index{ふおん@フォン!とりのふおん@鶏の---}
\index{fonds@fonds!volaille@--- de volaille}
\index{かきん@家禽!とりのふおん@鶏のフォン}
\index{うおらいゆ@ヴォライユ!ふおんとうおらいゆ@フォンドヴォライユ}

白いフォンと同じ主素材、香味素材、水の量で、さらに鶏のとさかや手羽、ガ
ラを適宜増量し、廃鶏3羽を加えて作る。

\hypertarget{jus-de-veau-brun}{%
\subsubsection{仔牛の茶色いフォン(仔牛の茶色いジュ)}\label{jus-de-veau-brun}}

\frsub{Fonds, ou Jus de veau brun}

\index{ふおん@フォン!こうしのちやいろい@仔牛の茶色い---}
\index{しゆ@ジュ!こうしのちやいろいしゆ@仔牛の茶色い---}
\index{fonds@fonds!fonds de veau brun@--- de veau brun}
\index{jus@jus!jus de veau brun@--- de veau brun}
\index{こうし@仔牛!こうしのちやいろいふおん@---の茶色いフォン(ジュ)}
\index{veau@veau!fonds brun@fonds ou jus de --- brun}

(仕上がり10 L分)

\begin{itemize}
\item
  主素材\ldots{}\ldots{}骨を取り除いた仔牛のすね肉と肩肉(紐で縛っておく)6kg、
  細かく砕いた仔牛の骨5 kg。
\item
  香味素材\ldots{}\ldots{}にんじん600 g、玉ねぎ400 g、パセリの枝100
  g、ローリエの葉 2枚、タイム2枝。
\item
  使用する液体\ldots{}\ldots{}白いフォンまたは水12 L。水を用いる場合は1
  Lあたり3 gの塩を加える。
\item
  作業手順\ldots{}\ldots{}厚手の片手鍋または寸胴鍋の底に輪切りにしたにんじんと玉
  ねぎを敷きつめる。その他の香味素材と、あらかじめオーブンで焼き色を付けておい
  た骨と肉を鍋に加える。
\end{itemize}

蓋をして約10分間、蓋をして弱火にかけた野菜から水分が汗をかくように出る
イメージで蒸し焼き状態にし、素材の味を引き出す\footnote{suer
  (スュエ)シュエ。}。フォンま
たは水少量を加え、煮詰める。この作業をさらに1〜2回行なう。残りのフォ
ンまたは水を注ぎ、蓋をし、沸騰させる。アクを丁寧に取る。微沸騰の状
態で6時間煮る。

布で漉し、ストックしておく。使用目的や必要に応じて、さらに煮詰めてから
ストックしてもいい。

\hypertarget{fonds-de-gibier}{%
\subsubsection{ジビエのフォン}\label{fonds-de-gibier}}

\frsub{Fonds de gibier}

\index{ふおん@フォン!しひえ@ジビエの---}
\index{fonds@fonds!fonds de gibier@--- de gibier}
\index{しひえ@ジビエ!ふおん@---のフォン}
\index{gibier@gibier!fonds@fonds de ---}

(仕上がり5 L分)

\begin{itemize}
\item
  主素材\ldots{}\ldots{}ノロ鹿の頸、胸肉および端肉3
  kg(老いたノロ鹿がいいが、新 鮮なものを使うこと)、野うさぎ\footnote{lièvre
    (リエーヴル)。}の端肉1 kg、老うさぎ2羽、山うずら2羽、 老きじ1羽。
\item
  香味素材\ldots{}\ldots{}にんじん250 g、玉ねぎ250
  g、セージ1枝、ジュニパーベリー \footnote{セイヨウネズの樹の実。}15粒、標準的なブーケガルニ。
\end{itemize}

\begin{itemize}
\item
  使用する液体\ldots{}\ldots{}水6 Lおよび白ワイン1瓶。
\item
  加熱時間\ldots{}\ldots{}3時間。
\item
  作業手順\ldots{}\ldots{}ジビエは事前にオーブンで焼き色を付けておき、野菜と香草を
  敷き詰めた鍋に入れる。野菜類も事前に焼き色を付けておくこと。ジビエを焼
  くのに用いた天板を白ワインでデグラセし、これを鍋に注ぐ。同量の水も加え、
  ほぼ水分がなくなるまで煮詰める。
\end{itemize}

この作業の後で、残りの水全量を注ぎ、沸騰させる。丁寧にアクを引きながら
ごく弱火で煮る\footnote{最後に布で漉す必要があるが、当然のこととして明記されていないの
  で注意。}。

\hypertarget{fumet-de-poisson}{%
\subsubsection[魚のフュメ(フュメドポワソン)]{\texorpdfstring{魚のフュメ(フュメドポワソン)\footnote{本質的には前出の「フォン」と同様のものだが、魚(およびジビエ)
  を素材としたフォンは香りがポイントとなるため、フュメ fumet (香気、
  良い香りの意)の名称のほうが一般的に使われている。}}{魚のフュメ(フュメドポワソン)}}\label{fumet-de-poisson}}

\frsub{Fonds, ou Fumet de poisson}

\index{ふおん@フォン!さかな@魚の---}
\index{ふゆめ@フュメ!さかな@魚の---}
\index{ふゆめ@フュメ!ほわそん@フュメドポワソン}
\index{fumet@fumet!fumet de poisson@--- de poisson}
\index{fonds@fonds!fumet de poisson@fumet de poisson}

(仕上がり10L分)

\begin{itemize}
\item
  主素材\ldots{}\ldots{}舌びらめ、メルラン\footnote{タラの近縁種。}やバルビュ\footnote{ヒラメの近縁種。}のあら10
  kg。
\item
  香味素材\ldots{}\ldots{}薄切りにした玉ねぎ500 g、パセリの根\footnote{パセリには根がにんじん形に肥大する品種もある(persil
    tubéreux 根パセリ。葉は平らでイタリアンパセリのように使う)。}と茎100
  g、マッ シュルームの切りくず250 g、レモンの搾り汁1個分、粒こしょう15
  g(これは フュメを漉す10分前に投入する)。
\item
  使用する液体と調味料\ldots{}\ldots{}水10 L、白ワイン1瓶。液体1
  Lあたり3〜4 gの 塩。
\item
  加熱時間\ldots{}\ldots{}30分。
\item
  作業手順\ldots{}\ldots{}鍋底に香味野菜を敷き詰め、魚のあらを入れる。水と白ワイ
  ンを注ぎ、強火にかける。丁寧にアクを引き、微沸騰の状態を保つようにする。
  30分煮たら目の細かい網で漉す。
\end{itemize}

\hypertarget{nota-fumet-de-poisson}{%
\subparagraph{【原注】}\label{nota-fumet-de-poisson}}

質の悪い白ワインを使うと灰色がかったフュメになってしまう。品質の疑わし
いワインは使わないほうがいい。

このフュメはソースを作る際に加える液体として用いる。魚料理用ソース・エ
スパニョルを作ることを想定する場合には、魚のあらをバターでエチュベして
から水と白ワインを注いで煮るといい。

\hypertarget{fonds-de-poisson-au-vin-rouge}{%
\subsubsection{赤ワインを用いた魚のフォン}\label{fonds-de-poisson-au-vin-rouge}}

\frsub{Fonds de poisson au vin rouge}

\index{ふおん@フォン!あかわいんをもちいたさかなのふおん@赤ワインを用いた魚の---}
\index{fonds@fonds!fonds de poisson au vin rouge@--- de poisson au vin rouge}

このフォンそれ自体を用意することは滅多にない。というのも、例えばマトロッ
トのような料理の魚の煮汁そのものだからだ。

とはいえ、こんにちでは魚のアラをすっかり取り除いた状態で料理を提供する
必要がますます高まってきているので、ここでそのレシピを記しておくべきだ
ろう。このフォンの必要性と有用さはどんどん高まっていくと思われる。

原則として、このフォンの仕込みには、料理として提供するのと同じ種類の魚
のアラを用いて、その香りの特徴を生かす必要がある。だが、どんな種類の魚
を使う場合でも作り方は同じだ。

(仕上がり5 L分)

\begin{itemize}
\item
  主素材\ldots{}\ldots{}料理に用いるのと同じ魚種の頭とアラ2.5 kg。
\item
  香味素材\ldots{}\ldots{}薄切りにして下茹でした玉ねぎ300
  g、パセリの枝100 g、タイ
  ムの小枝1本、小さめのローリエの葉2枚、にんにく5片、マッシュルームの切
  りくず100 g。
\item
  使用する液体と調味料\ldots{}\ldots{}水3.5 L、良質の赤ワイン2 L、塩15
  g。
\item
  加熱時間\ldots{}\ldots{}30分。
\item
  作業手順\ldots{}\ldots{}「魚の白いフォン\footnote{前項のフュメドポワソンのこと。}」と同様にする。
\end{itemize}

\hypertarget{nota-fonds-de-poisson-au-vin-rouge}{%
\subparagraph{【原注】}\label{nota-fonds-de-poisson-au-vin-rouge}}

このフォンは魚の白いフォンよりも濃く煮詰めることが可能。とはい
え、保存のために煮詰めないでいいように、その都度、必要な量だけ仕込むこ
とを勧める。

\hypertarget{essence-de-poisson}{%
\subsubsection{魚のエッセンス}\label{essence-de-poisson}}

\frsub{Essence de poisson}

\index{えつせんす@エッセンス!さかな@魚の---}
\index{essence@essence!poisson@--- de poisson}

\begin{itemize}
\item
  主素材\ldots{}\ldots{}メルラン\footnote{タラの近縁種。}および舌びらめの頭、アラ2
  kg。
\item
  香味素材\ldots{}\ldots{}薄切りにした玉ねぎ125
  g、マッシュルームの切りくず300 g、 パセリの枝50
  g、レモンの搾り汁1個分。
\item
  使用する液体\ldots{}\ldots{}煮詰めていないフュメドポワソン1\undemi{}
  L、良質の白ワイ ン3 dL。
\item
  所要時間\ldots{}\ldots{}45分。
\item
  作業手順\ldots{}\ldots{}鍋にバター100
  gと玉ねぎ、パセリの枝、マッシュルームの切
  りくずを入れ、強火で色づかないようさっと炒める。アラと端肉を加える。蓋をして約15分弱火で蒸
  し煮する\footnote{素材を入れた鍋に蓋をして弱火にかけ、少量の水分で蒸し煮状態にす
    ることを étuver エチュベという。このフランス語をそのまま用いている
    調理現場も少なくない。}。その間、小まめに混ぜてやること。白ワインを注ぎ、半量になるま
  で煮詰める。最後にフュメドポワソンを注ぎ、レモン汁と塩2 gを加える。
\end{itemize}

再び火にかけて、とろ火で15分程煮込んだら、布で漉す。

\hypertarget{nota-essence-de-poisson}{%
\subparagraph{【原注】}\label{nota-essence-de-poisson}}

魚のエッセンスは、舌びらめやチュルボ、チュルボタン、バルビュ\footnote{いずれも鰈、ひらめの近縁種。チュルボタンはチュルボの小さいもの
  を言う。} などのフィレ\footnote{3枚おろし、または5枚おろしにして、頭とアラを取り除いた状態。}をポシェする際に用いる。

さらに、このエッセンスを煮詰めて、上記でポシェした魚のソースに加えて風
味を強くするのに使う。

\hypertarget{essences-diverses}{%
\subsubsection{エッセンスについて}\label{essences-diverses}}

\frsub{Essences diverses}

\index{えつせんす@エッセンス!01えつせんすについて@---について(フォン)}
\index{essence@essence!01 diverses@---s diverses (fonds)}

その名のとおり、エッセンスとはごく少量になるまで煮詰めて非常に強い風味
を持たせたフォンのこと。

エッセンスは普通のフォンと本質的には同じものだが、素材の風味をしっかり
出すために、使用する液体の量はずっと少ない。したがって、仕上げにエッセ
ンスを加える指示がある料理の場合でも、そもそも充分に風味ゆたかなフォン
を用いていれば、エッセンスは必要ないことが分かるだろう。

まず最初に、美味しく風味ゆたかなフォンを用いるほうが、あまり出来のよく
ないフォンで調理し、後からエッセンスで欠点を補うよりもずっと簡単なのだ。
その方がいい結果が得られるし、時間と材料の節約にもなる。

セロリ、マッシュルーム、モリーユ\footnote{morille
  キノコの一種。和名アミガサタケ。}、トリュフなど、とりわけ明確な風
味の素材のエッセンスを、必要に応じて用いるにとどめるのがいい。

また、十中八九、フォンを仕込む際に素材そのものを加えた方が、エッセンス
を仕込むよりもいい結果が得られることは頭に入れておくこと。

そのようなわけで、エッセンスについてこれ以上長々と述べる必要もないと思
われる。ベースとなるフォンがコクと風味がゆたかなものならであるなら、エッ
センスはまったく無用の長物と言える。

\hypertarget{glaces-diverses}{%
\subsubsection{グラスについて}\label{glaces-diverses}}

\frsub{Glaces diverses}

\index{くらす@グラス!01くらすについて@---について}
\index{glace@glace!01 diverses@---s diverses}

グラスドヴィアンド、鶏のグラス(グラスドヴォライユ)、ジビエのグラス、
魚のグラスの用途は多岐にわたる。これらは、上記いずれかの素材でとったフォ
ンをシロップ状になるまで煮詰めたもののことだ。

これらの使い途は、料理の仕上げに表面に塗ってしっとりとした艶を出させる
のに用いる場合もあれば、ソースの味や色合いを濃くするために用いたり、あ
るいは、あまりに出来のよくないフォンで作った料理の場合にはコクを与える
ために使うこともある。また、料理によっては適量のバターやクリームを加え
てグラスそのものをソースとして用いることもある。

グラスとエッセンスの違いだが、エッセンスが料理の風味そのものを強くする
ことだけが目的であるのに対して、グラスは素材の持つコクと風味をごく少量
にまで濃縮したものだ。

だからほとんどの場合、エッセンスよりもグラスを使うほうがいい。

とはいえ昔の料理長たちの中には、グラスの使用を絶対に認めない者もいた。
その理由は、料理を作る度に毎回その料理のためのフォンをとるべきであり、
それだけで料理として充分なものにすべき、ということだった。

確かに時間と費用の点で制限がなければその理屈は正しい。だが、こんにちで
は、そのようなことの出来る調理現場はほとんどない。そもそもグラスは、正
しく適量を用いるのであれば、そのグラスが丁寧に作られたものであるならな、
素晴しい結果が得られる。 だから多くの場合、グラスはまことに有用なもの
と言える。

\hypertarget{glace-de-viande}{%
\subsubsection{グラスドヴィアンド}\label{glace-de-viande}}

\frsub{Glace de viande}

\index{くらす@グラス!くらすとういあんと@---ドヴィアンド}
\index{glace@glace!viande@--- de viande}

茶色いフォン(エストゥファード)を煮詰めて作る。

煮詰めて濃くなっていく途中、何度か布で漉して、より小さな鍋に移しかえて
いく。煮詰めている際に、丁寧にアクを引くことが、澄んだグラスを作るポイ
ント。

煮詰めている際には、フォンの濃縮具合に応じて、火加減を弱めていくこと。
最初は強火でいいが、作業の最後の方は弱火にしてゆっくり煮詰めてやること。

スプーンを入れてみて、引き上げた際に、艶のあるグラスの層でスプーンが覆
われ、しっかり張り付いているくらいが丁度いい。要するに、スプーンがグラ
スでコーティングされた状態になればいいということだ。

\hypertarget{nota-glace-de-viande}{%
\subparagraph{【原注】}\label{nota-glace-de-viande}}

色が薄くて軽い仕上がりのグラスが必要な場合には、茶色いフォンではなく、標
準的な仔牛のフォンを用いる。

\hypertarget{glace-de-volaille}{%
\subsubsection{鶏のグラス(グラスドヴォライユ)}\label{glace-de-volaille}}

\frsub{Glace de volaille}

\index{くらす@グラス!とり@鶏の---(---ドヴォライユ)}
\index{くらす@グラス!うおらいゆ@---ドヴォライユ}
\index{glace@glace!volaille@--- de volaille}

鶏のフォン(フォンドヴォライユ)を用いて、グラスドヴィアンドと同様にし
て作る。

\hypertarget{glace-de-gibier}{%
\subsubsection{ジビエのグラス}\label{glace-de-gibier}}

\frsub{Glace de gibier}

\index{くらす@グラス!しひえ@ジビエの---}
\index{glace@glace!glace de gibier@--- de gibier}
\index{しひえ@ジビエ!くらす@---のグラス}
\index{gibier@gibier!gibier@glace de ---}

ジビエのフォンを煮詰めて作る。ある特定のジビエの風味を生かしたグラスを
作るには、そのジビエだけでとったフォンを用いること。

\hypertarget{glace-de-poisson}{%
\subsubsection{魚のグラス}\label{glace-de-poisson}}

\frsub{Glace de poisson}

\index{くらす@グラス!さかな@魚の---}
\index{glace@glace!poisson@--- de poisson}

このグラスを用いることはあまり多くない。日常的な業務においては「魚のエッ
センス」を用いることが好まれる。そのほうが魚の風味も繊細になる。魚のエッ
センスで魚をポシェした後に煮詰めてソースに加える。
\end{recette}
\hypertarget{roux}{%
\section{ルー}\label{roux}}

\frsec{Roux}

\index{るー@ルー} \index{roux@roux}

ルーはいろいろな派生ソースのベースとなる基本ソースにとろみを付ける役目
を持つ。ルーの仕込みは、一見したところさほど重要に思われぬだろうが、実
際には正反対だ。丁寧に注意深く作業すること。

茶色いルーは加熱に時間がかかるので、大規模な調理現場では前もって仕込ん
でおく。ブロンドのルーと白いルーはその都度用意すればいい。
\begin{recette}
\hypertarget{roux-brun}{%
\subsubsection{茶色いルー}\label{roux-brun}}

\frsub{Roux brun}

\index{るー@ルー!ちやいろ@茶色い---} \index{roux@roux!brun@--- brun}

(仕上がり1 kg分)

\begin{enumerate}
\def\labelenumi{\arabic{enumi}.}
\tightlist
\item
  澄ましバター\ldots{}\ldots{}500 g
\item
  ふるった小麦粉\ldots{}\ldots{}600 g
\end{enumerate}

\hypertarget{cuisson-des-roux}{%
\subsubsection{ルーの火入れについて}\label{cuisson-des-roux}}

\index{るー@ルー!ひいれについて@---の火入れについて}
\index{roux@roux!cuisson@cuisson du ---}

加熱時間は使用する熱源の強さで変わってくる。だから数字で何分とは言えな
い。ただし、火力が強過ぎるよりは弱いくらいの方がいい。というのも、温度
が高すぎると小麦粉の細胞が硬化して中身を閉じ込めてしまい、そうなると後
でフォンなどの液体を加えた際に上手く混ざらず、滑らかなとろみの付いたソー
スにならない。乾燥豆をいきなり熱湯で茹でるのと同じようなことが起きるわ
けだ。低い温度から始めてだんだんと熱くしていけば、小麦粉の細胞壁がゆる
んで細胞中のでんぷんが膨張し、熱によって発酵状態の初期のようになる。こ
のようにして、でんぷんをデキストリンに変化させる\footnote{現代の科学的見地からすると必ずしも正確な記述ではないので注意。}。デキストリ
ンは水溶性の物質で、これが「とろみ」の主な要素なのだ。茶色いルーは淡褐
色の美しい色合いで滑らかな仕上がりにする。だまがあってはいけない。

ルーを作る際には必ず、澄ましバターを使うこと\footnote{初版〜第三版では「澄ましバターまたは充分に澄ましたグレスド
  マルミット」となっている。グレスドマルミットとは、コンソメなどを作
  る際に、浮いてくる油脂を取り除く必要があるが、それを捨てずにまとめ
  てから漉して澄ませたもののこと。基本的に獣脂と考えていい。なお、同
  時代の料理書 --- 例えばペラプラ『近代料理技術』(1935年)--- には、
  ルーを作るのにバターを使う必要はなく、グレスドマルミットで充分、と
  しているものもある。}。 生のバターに
は相当量のカゼインが含まれている。カゼインがあると火を均質に通すことが
出来なくなってしまう。とはいえ、以下を覚えておくといい。ソースとして仕
上げた段階で、ルーで使ったバターは風味という点ではほとんど意味が失なわ
れている。そもそもソースの仕上げに不純物を取り除く\footnote{dépouiller
  デプイエ。ソースや煮込み料理を仕上げる際に、浮
  き上がってくる不純物を徹底的に取り除き、目の細かい布などで漉すこと。
  原義は動物などの皮を剥ぐ、剥くことの意で、野うさぎの皮を剥ぐ、うな
  ぎの皮を剥く、という意味で現代の厨房でも用いられているる。ソースの
  場合は表面に凝固した蛋白質や油脂の膜が出来、それを「剥ぐように」取
  り除くことから、あるいは表面に浮いてくる不純物を徹底的に取り除いて
  きれいなソースに仕上げることを、動物の皮を剥いてきれいな身だけにす
  ることになぞらえて、この用語が用いられるようになったようだ。なお、
  本書においてécumer(エキュメ)が単に浮いてくる泡やアクを取る、とい
  う作業であるのに対して、dépouiller(デプイエ)は「徹底的に不純物を
  取り除いて美しく仕上げる」という意味合いが込められている。現代では
  品種改良や農法の変化によって野菜のアクも少なくなり、小麦粉も精製度
  の高いものを利用出来るなど、食材および調味料の多くで純度の高いもの
  を使用する場合がほとんどであり、このデプイエという作業は20世紀後半
  にはほとんど行なわれなくなり、écumer(エキュメ)という用語だけで済ませ
  ることがほとんど(cf.辻静雄監訳『オリヴェ ソースの本』柴田書店、
  1970年、27〜28頁)。}段階でバター
も完全に取り除かれてしまうわけだ。だからルーに用いるバターは小麦粉に熱
を通すためだけのものと考えていい。

ルーはソース作りの出発点だ。だから次の点も記憶に\ruby{留}{とど}めるこ
と。小麦粉にでんぷんが含まれているからこそソースに「とろみ」が付く。だ
から純粋なでんぷん(特性が小麦のでんぷんと同じでも異なったものでも)で
ルーを作っても、小麦粉の場合と同様の結果が得られるだろう。ただしその場
合は小麦粉でルーを作る場合より注意して作業する必要がある。また、小麦粉
と違って余計な物質が含まれていないために、全体の分量比率を考え直すこと
になる。

\hypertarget{nota-roux}{%
\subparagraph{【原注】}\label{nota-roux}}

本文で述べたように、茶色いルーを作る際には澄ましバターを用いる。他の動
物性油脂はよほど経済的事情が逼迫していない限り使わないこと。材料コスト
が問題になる場合でも、ソースの仕上げに不純物を取り除く際に多少の注意を
払えば、ルーに用いたバターを回収するのはさして難しいことではない\footnote{既に述べたように初版〜第三版まではバターまたはグレスドマル
  ミットという指示であったことに留意する必要はあるだろう。実際のとこ
  ろ、良質のバターを用いてルーを作ったほうが、軽やかな仕上りのソース
  になる傾向があることは言うまでもない。}。それ
を後で他の用途で使えばいいだろう。

\hypertarget{roux-blond}{%
\subsubsection{ブロンドのルー}\label{roux-blond}}

\frsub{Roux blond}

\index{るー@ルー!ふろんと@ブロンドの---}
\index{roux@roux!blond@--- blond}

(仕上がり1 kg分)

材料の比率は茶色いルーと同じ。すなわちバター500 gと、ふるった小麦粉600
g。

火入れは、ルーがほんのりブロンド色になるまで、ごく弱火で行なう。

\hypertarget{roux-blanc}{%
\subsubsection{白いルー}\label{roux-blanc}}

\frsub{Roux blanc}

\index{るー@ルー!しろい@白い---} \index{roux@roux!blanc@--- blanc}

500 gのバターと、ふるった小麦粉600 g。

このルーの火入れは数分、つまり粉っぽさがなくなるまでの時間でいい。
\end{recette}\newpage
\hypertarget{ux57faux672cux30bdux30fcux30b9}{%
\section{基本ソース}\label{ux57faux672cux30bdux30fcux30b9}}

\hypertarget{grandes-sauces-de-base}{%
\subsection{Grandes Sauces de Base}\label{grandes-sauces-de-base}}

\index{そーす@ソース!きほん@基本---}
\index{sauce@sauce!00grandes@*Grandes ---s de Base}
\begin{recette}
\hypertarget{ux30bdux30fcux30b9ux30a8ux30b9ux30d1ux30cbux30e7ux30eb102008}{%
\subsubsection[ソース・エスパニョル]{\texorpdfstring{ソース・エスパニョル\footnote{本節冒頭では、ルーがスペインの料理人によってもたらされ、そ
  の結果としてソース・エスパニョルが作られるようになったと読める記述
  があるが、これはむしろ誤りと考えるべき。エスパニョル espagnol(e)は
  「スペイン(風)の」意だが、スペイン料理起源というわけでもない。ス
  ペインを想起させるトマトを使うから、あるいは、ソースが茶褐色なのが
  ムーア系スペイン人を想起させるから、など定説はない。カレーム『19世
  紀フランス料理』第3巻に収められたソース・エスパニョルの作り方は、
  フォンをとるところから始まり4ページにわたって詳細なものとなってい
  る(pp.8-11)。その中で、肉を入れた鍋に少量のブイヨンを注いで煮詰
  めることを繰り返す。ここまでは18世紀の料理書で一般的な手法であるが、
  その後に大量のブイヨンを注いだ後、いきなり強火にかけるのではなく、
  弱火で加熱していくやり方を「スペイン式の方法」と述べている。カレー
  ムにおいては、これがソースの名称の根拠のひとつになっていると考えて
  いいだろう。もちろん、ソース・エスパニョルという名称のソースはカレー
  ム以前からあり、1806年刊のヴィアール『帝国料理の本』にもカレームの
  レシピより簡単だが、ほぼ同様のものが基本ソースとして収録されている。
  また、それ以前にもソース・エスパニョルに類する名称のソースはあった
  が、たとえば1739年刊ムノン『新料理研究』第2巻にある「スペイン風ソー
  ス」はかなり趣きが異なる(コリアンダーひと把みを加えるのが特徴的)。
  同じ料理名でも時代や料理書の著者によってまったく違う料理になってい
  ることは、食文化史において珍しいことではない。また、とりわけ料理名
  に地名、国名が冠されているものの中には根拠や由来のはっきりしないも
  のも多い。いずれにしても、本書のソース・エスパニョルの源流は19世紀
  初頭のヴィアールあたりからと考えられる。ソース・エスパニョルは19世
  紀を通して普及し、茶色いソースの代表的な存在となった。こんにちでも
  フォンドヴォーをベースとしたソースは、ルーでとろみ付けこそしないが、
  仔牛の骨などから出るコラーゲンによるとろみを利用したもので、仕上が
  りの色合いや、ごく標準的ともいえる風味付けの方法などが引継がれ続け
  ている調理現場も少なくない。もっとも、上述のように本書では「茶色い
  ルー」を使うところに「エスパニョル」であることの理由を見い出そうと
  していると解釈される。}}{ソース・エスパニョル}}\label{ux30bdux30fcux30b9ux30a8ux30b9ux30d1ux30cbux30e7ux30eb102008}}

\hypertarget{sauce-espagnole}{%
\paragraph{Sauce espagnole}\label{sauce-espagnole}}

\index{そーす@ソース!えすはによる@---・エスパニョル}
\index{そーす@ソース!きほん@基本---!えすはによる@---・エスパニョル}
\index{きほんそーす@基本ソース!えすはによる@---・エスパニョル}
\index{えすはによる@エスパニョル!そーす@ソース・---}
\index{すぺいんふう@スペイン風(エスパニョル)!そーすえすはによる@ソース・エスパニョル}
\index{sauce@sauce!00grandes@*Grandes ---s de Base!espagnole@--- Espagnole}
\index{sauce@sauce!espagnole@--- Espagnole}
\index{espagnol@espagnol!sauce@Sauce ---e}

(仕上がり5 L分)

\begin{itemize}
\item
  \textbf{とろみ付けのためのルー}\ldots{}\ldots{}625 g。
\item
  \textbf{茶色いフォン(ソースを仕上げるのに必要な全量)}\ldots{}\ldots{}12
  L。
\item
  \textbf{\protect\hyperlink{}{ミルポワ}\footnote{mirepoix
    (ミルポワ)。ソースやフォンにコクを与える目的で、
    細かいさいの目に切った香味野菜や塩漬け豚ばら肉を合わせたもの。18世
    紀にミルポワ公爵の料理人が考案したという説が有力。同様のものに
    matignon(マティニョン)があるが、ミルポワより大きめのさいの目に切
    るのが一般的とされるが、調理現場によってはあまり区別せずミルポワと
    のみ呼称するケースも多い。}(香味素材)}\ldots{}\ldots{}小さなさいの目に切った塩漬け豚
  ばら肉150 g、2 mm程度のさいの目\footnote{brunoise
    (ブリュノワーズ)。1〜2 mm のさいの目に切ること。 couper en
    mirepoix(クゥペオンミルポワ)ミルポワに切るとも言う。}に切ったにんじん250
  gと玉ねぎ 150
  g、タイム2枝、ローリエの葉2枚。\index{みるぽわ@ミルポワ}\index{mirepoix}
\item
  \textbf{作業手順}
\end{itemize}

\begin{enumerate}
\def\labelenumi{\arabic{enumi}.}
\item
  フォン8 Lを鍋で沸かす。あらかじめ柔らかくしておいたルーを加え、木杓
  子か泡立て器で混ぜながら沸騰させる。

  弱火にして\footnote{原文から直訳すると「鍋を火の脇に置く」だが、現代の調理環境
    では単純に「弱火にする」と解釈していい。}微沸騰の状態を保つ。
\item
  以下のようにしてあらかじめ用意しておいたミルポワを投入する。ソテー
  鍋に塩漬け豚ばら肉を入れて火にかけて脂を溶かす。そこに、細かく刻ん
  だにんじんと玉ねぎ、タイム、ローリエの葉を加える。野菜が軽く色づく
  まで強火で炒める。丁寧に、余分な脂を捨てる。これをソースに加える。
  野菜を炒めたソテー鍋に白ワイン約100 ml\footnote{原文 un verre de vin
    blanc (アンヴェールドヴァンブロン)。
    直訳すると「グラス1杯の白ワイン」だが、本書において un verre de 〜
    は「約1dl=100ml」と覚えておくといいだろう。}を加えてデグラセ\footnote{dégrasser
    鍋に粘液状になって付着している肉汁を酒類あるいは水で溶かし出してソースなどに利用すること。}し、それを半量
  まで煮詰める。これも同様にソースの鍋に加える。こまめに浮いてくる夾
  雑物を徹底的に取り除き\footnote{dépouiller
    デプイエ。前節「ルーの火入れについて」訳注参照。}ながら弱火で約1時間煮込む。
\item
  ソースをシノワ\footnote{小さな穴が多く空けられた円錐形で、取っ手の付いた漉し器の一
    種。金属製のものが主流。}で、ミルポワ野菜を軽く押しながら漉し、別の
  片手鍋に移す。フォン2 Lを注ぎ足す。さらに2時間、微沸騰の状態を保ち
  ならが煮込む。その後、陶製の鍋に移し、ゆっくり混ぜながら冷ます。
\item
  翌日、再び厚手の片手鍋に移してから、フォン2 Lとトマトピュレ1 Lまた
  は同等の生のトマトつまり2 kgを加える。\\
  トマトピュレを用いる場合は、あらかじめオーブンでほとんど茶色になる
  まで焼いておくといい。そうするとトマトピュレの酸味を抜くことが出来
  る。\\
  そうすればソースを澄ませる作業が楽になるし、ソースの色合いも温かそ
  うで美しいものになる。\\
  ソースをヘラか泡立て器で混ぜながら強火で沸騰させる。弱火にして1時間
  微沸騰の状態を保つ。最後に、表面に浮いている不純物を、細心の注意を
  払いながら徹底的に取り除く。布で漉し、完全に冷めるまで、ゆっくり混
  ぜ続けること。
\end{enumerate}

\hypertarget{ux539fux6ce8}{%
\subparagraph{【原注】}\label{ux539fux6ce8}}

ソース・エスパニョルで仕上げに不純物を取り除くのにかかる時間はいちがい
には言えない。これは、ソースに用いるフォンの質次第で変わるからだ。

ソースにするフォンが上質なものであればある程、仕上げに不純物を取り除く
作業は早く済む。そういう場合には、ソース・エスパニョルを5時間で作るこ
とも無理ではない。

\maeaki

\hypertarget{ux9b5aux6599ux7406ux75280102006ux30bdux30fcux30b9ux30a8ux30b9ux30d1ux30cbux30e7ux30eb}{%
\subsubsection[魚料理用ソース・エスパニョル]{\texorpdfstring{魚料理用\footnote{フランス語のソース名にあるmaigre(メーグル)はこの場合、一般的に「魚用、
  魚料理用」と訳すが、厳密には「小斉の際の料理用」となろう。小斉とは、
  カトリックで古くから特定の期間、曜日に肉類を断つ食事をする宗教的食
  習慣。日本の「お精進」とニュアンスは近いが、小斉においては忌避され
  るのは鳥獣肉のみであり、魚介や乳製品はいいとされた。こじつけのよう
  に、水鳥は水のものだから魚介扱いであり、またイルカも魚類として扱わ
  れていた。小斉が行なわれるのは復活祭の前46日間(四旬節、逆に言えば
  カーニバルの最終日マルディグラの翌日から46日)と、週に一度(多くの
  場合は金曜)であった。合計すると小斉が行なわれるのは年間100日近く
  もあり、中世から18世紀の料理人たちは小斉の宴席に供する料理に工夫を
  凝らしていた。この習慣は19世紀になるとだんだん廃れていき、エスコフィ
  エの時代には、料理人に対して小斉のための料理を要求することは少なく
  なっていった。}ソース・エスパニョル}{魚料理用ソース・エスパニョル}}\label{ux9b5aux6599ux7406ux75280102006ux30bdux30fcux30b9ux30a8ux30b9ux30d1ux30cbux30e7ux30eb}}

\hypertarget{sauce-espagnole-maigre}{%
\paragraph{Sauce espagnole maigre}\label{sauce-espagnole-maigre}}

\index{そーす@ソース!えすはによるるさかな@---・エスパニョル (魚料理用)}
\index{そーす@ソース!きほん@基本---!えすはによるさかな@魚料理用---・エスパニョル}
\index{きほんそーす@基本ソース!えすはによるさかな@魚料理用---・エスパニョル}
\index{えすはによる@エスパニョル!そーすさかなよう@ソース・--- (魚料理用)}
\index{すへいんふう@スペイン風(エスパニョル)!そーすえすはによるさかな@ソース・エスパニョル(魚料理用)}
\index{sauce@sauce!00grandes@*Grandes ---s de Base!espagnole maigre@--- Espagnole maigre}
\index{sauce@sauce!espagnole maigre@--- Espagnole maigre}
\index{espagnol@espagnol!sauce maigre@Sauce Espagnole maigre}

(仕上がり5 L分)

\begin{itemize}
\item
  \textbf{バターを用いて\footnote{初版〜第三版にかけては、茶色いルーを作るのに「バターまたは、
    きれいなグレスドマルミット(コンソメなどを作る際に表面に浮いてくる
    脂をすくい取って、不純物を漉し取ったものであり、基本的に獣脂)」を
    用いる、とある。上述のように、カトリックにおける「小斉」の場合、獣
    脂は忌避されたがバターなどの乳製品は許容された。そのため特に「バター
    を用いて作ったルー」という指定がなされ、第四版では茶色いルーに澄ま
    しバターのみを使う旨が強調されたが、ここでは初版以来の記述がそのま
    ま残っているために、やや冗長に思われる表現となっている。}作ったルー}\ldots{}\ldots{}500
  g。
\item
  \textbf{魚のフュメ(フュメドポワソン)(ソースを仕上げるために必要な全量)
  }\ldots{}\ldots{}10 L。
\item
  \textbf{ミルポワ}\ldots{}\ldots{}標準的なソース・エスパニョルと同じ\protect\hyperlink{mirepoix}{ミルポワ}野菜を同
  量と、塩漬け豚ばら肉の代わりにバターを用い、マッシュルームまたはマッシュ
  ルームの切りくず\footnote{champignons de Paris
    (シャンピニョン ドパリ)いわゆるマッ
    シュルームは、料理の一部として提供する際にはトゥルネ tourner といっ
    て\{螺旋\}\{らせん\}状の切れ込みを入れて装飾したものを使う。その際に
    少なくない量の切りくずが発生するのでそれを利用する。なお、tourner
    (トゥルネ)の原義は「回す」であり、包丁を持った側の手は動かさずに、
    材料のほうを回すようにして切れ目を入れたり、アーティチョークや果物
    などの皮を剥くことを意味する。}250 gを加える。
\item
  \textbf{作業手順}\ldots{}\ldots{}標準的なソース・エスパニョルとまったく同様に作る。
\item
  \textbf{加熱時間と不純物を取り除くのに必要な時間}\ldots{}\ldots{}5時間。
\end{itemize}

仕上げに漉してから、標準的なソース・エスパニョルとまったく同様に、完全
に冷めるまでゆっくり混ぜ続けること。

\maeaki

\hypertarget{ux9b5aux6599ux7406ux7528ux30bdux30fcux30b9ux30a8ux30b9ux30d1ux30cbux30e7ux30ebux88dcux8db3}{%
\subparagraph{魚料理用ソース・エスパニョル補足}\label{ux9b5aux6599ux7406ux7528ux30bdux30fcux30b9ux30a8ux30b9ux30d1ux30cbux30e7ux30ebux88dcux8db3}}

このソースを日常的な料理のベースとなる仕込みに含めるかどうかについては
意見が分れるところだ。

普通のソース・エスパニョルは、つまるところ風味の点ではほとんどニュート
ラルなものだから、それに魚のフュメを加えれば、魚料理用ソース・エスパニョ
ルとして充分に通用するだろう。どうしても上で挙げた魚料理用ソース・エス
パニョルが必要になるのは、宗教的に厳格に小斉の決まりを守って料理を作る
場合のみで、さすがにその場合は代用品などない。

\maeaki

\hypertarget{ux30bdux30fcux30b9ux30c9ux30a5ux30dfux30b0ux30e9ux30b9102009}{%
\subsubsection[ソース・ドゥミグラス]{\texorpdfstring{ソース・ドゥミグラス\footnote{日本の洋食などで一般的な「デミグラス」あるいは「ドミグラス」」
  とはかなり異なった仕上りのソースであることに注意。ソース・エスパニョ
  ルの仕上げにあたって、徹底的に不純物を取り除くことを何度も強調して
  いるのは、透き通った茶色がかった色合いの、なめらかなソースを目指す
  からであり、それをさらに徹底させるということは、透明度、なめらかさ
  の面でさらに上を目指すということを意味するからだ。ちなみに、アメリ
  カに本社のあるメーカーの「デミグラスソース」の缶詰はもっぱら日本で販売
  されている製品であり、ヨーロッパおよびアメリカでは同一ブランドに該
  当する商品は存在しないようだ。}}{ソース・ドゥミグラス}}\label{ux30bdux30fcux30b9ux30c9ux30a5ux30dfux30b0ux30e9ux30b9102009}}

\hypertarget{sauce-demi-glace}{%
\paragraph{Sauce demi-glace}\label{sauce-demi-glace}}

\index{そーす@ソース!とうみくらす@---・ドゥミグラス}
\index{そーす@ソース!きほん@基本---!とうみくらす@---・ドゥミグラス}
\index{きほんそーす@基本ソース!とうみくらす@---・ドゥミグラス}
\index{sauce@sauce!00grandes@*Grandes ---s de Base!demi-glace@--- Demi-glace}
\index{sauce@sauce!demi-glace@--- Demi-glace}

一般に「ドゥミグラス」と呼ばれているものは、いったん仕上がったソース・
エスパニョルをさらに、もうこれ以上は無理という位に徹底的に不純物を取り
除いたもののことだ。

最後の仕上げに\protect\hyperlink{glace-de-viande}{グラスドヴィアンド}などを加える。風味付けに何らかの酒類\footnote{本書ではマデイラワイン(ポルトガルの酒精強化ワイン、すなわ
  ちブドウ果汁が酵母により醗酵している途中で蒸留酒を加えて醗酵を止め
  る製法のもので、甘口のものが多い)が用いられることが多い。}
を加えれば、当然ながらソースの性格も変わるので、最終的な使い途に応じて
決めること。

\hypertarget{ux539fux6ce8-1}{%
\subparagraph{【原注】}\label{ux539fux6ce8-1}}

ソースの色合いを決めるワインを仕上げに加える際には、「火から外して」行
なうこと。沸騰しているとワインの香りがとんでしまうからだ。

\maeaki

\hypertarget{ux3068ux308dux307fux3092ux4ed8ux3051ux305fux4ed4ux725bux306eux30b8ux30e5}{%
\subsubsection{とろみを付けた仔牛のジュ}\label{ux3068ux308dux307fux3092ux4ed8ux3051ux305fux4ed4ux725bux306eux30b8ux30e5}}

\hypertarget{jus-de-veau-lie}{%
\paragraph{Jus de veau lié}\label{jus-de-veau-lie}}

\index{そーす@ソース!きほん@基本---!とろみをつけたこうしのしゆ@とろみを付けた仔牛のジュ}
\index{きほんそーす@基本ソース!とろみをつけたこうしのしゆ@とろみを付けた仔牛のジュ}
\index{しゆ@ジュ!こうしのしゆ@仔牛の---(とろみを付けた)}
\index{そーす@ソース!とろみをつけたこうしのしゆ@とろみを付けた仔牛のジュ}
\index{こうし@仔牛!とろみをつけたこうしのしゆ@とろみを付けた---のジュ}
\index{sauce@sauce!00grandes@*Grandes ---s de Base!jus veau lie@--- de veau lié}
\index{jus@jus!jus veau lie@--- de veau lié}
\index{veau@veau!jus lie@jus de --- lié}

(仕上がり1 L分)

\begin{itemize}
\item
  \textbf{仔牛のフォン}\ldots{}\ldots{}仔牛の茶色いフォン4 L。
\item
  \textbf{とろみ付け材料}\ldots{}\ldots{}アロールート\footnote{allow-root
    南米産のクズウコンを原料とした良質のでんぷん。日
    本では入手が難しいこともあり、コーンスターチが用いられることがほとんど。}30
  g。
\item
  \textbf{作業手順}\ldots{}\ldots{}よく澄んだ仔牛のフォン4
  Lを強火にかけ、\unquart{}量つ まり1 Lになるまで煮詰める。
\end{itemize}

大さじ数杯分の冷たいフォンでアロールートを溶く。これを沸騰している鍋に
加える。1分程度だけ火にかけ続けたら、布で漉す。

\hypertarget{ux539fux6ce8-2}{%
\subparagraph{【原注】}\label{ux539fux6ce8-2}}

この、とろみを付けた仔牛のジュは、本書では頻繁に使う指示をしているが、
必ず、しっかりした味で透き通った、きれいな薄茶色に仕上げること。

\maeaki

\hypertarget{ux30f4ux30ebux30c6102013ux6a19ux6e96ux7684ux306aux767dux3044ux30bdux30fcux30b9}{%
\subsubsection[ヴルテ(標準的な白いソース)]{\texorpdfstring{ヴルテ\footnote{velouté
  (ヴルテ)原義は「ビロードのように柔らかな、なめら
  かな」。日本ではベシャメルソースと混同されやすいが、内容がまったく
  異なるソースなので注意。}(標準的な白いソース)}{ヴルテ(標準的な白いソース)}}\label{ux30f4ux30ebux30c6102013ux6a19ux6e96ux7684ux306aux767dux3044ux30bdux30fcux30b9}}

\hypertarget{veloute}{%
\paragraph{Velouté, ou sauce blanche graisse}\label{veloute}}

\index{そーす@ソース!きほん@基本---!うるて@ヴルテ(標準的な)}
\index{きほんそーす@基本ソース!うるて@ヴルテ(標準的な)}
\index{うるて@ヴルテ!ひようひゆんてきなそーすうるて@標準的なソース・---}
\index{そーす@ソース!うるてひようひゆん@ヴルテ(標準的な)}
\index{ふるーて@ブルーテ ⇒ ヴルテ} \index{veloute@velouté}
\index{veloute@velouté!sauce blanche grasse@sauce blanche grasse}
\index{sauce@sauce!00grandes@*Grandes ---s de Base!veloute@Velouté}

(仕上がり5 L分)

\begin{itemize}
\item
  \textbf{とろみ付けの材料}\ldots{}\ldots{}バターを用いて作った\footnote{\protect\hyperlink{sauce-espagnole-maigre}{魚料理用ソース・エスパニョル}、訳
    注参照。}ブロンドのルー 625 g。
\item
  \textbf{よく澄んだ仔牛の白いフォン}\ldots{}\ldots{}5 L。
\item
  \textbf{作業手順}\ldots{}\ldots{}ルーをフォンに溶かし込む。フォンは冷たくても熱くても
  いいが、フォンが熱い場合にはソースが充分なめらかになるよう注意して溶か
  すこと。混ぜながら沸騰させる。微沸騰の状態を保ちながら、浮いてくる不純
  物を完全に取り除いていく\footnote{dépouiller
    (デプイエ)。\protect\hyperlink{sauce-espagnole}{ソース・エスパニョル}、
    訳注参照。}。この作業はとりわけ細心の注意を払っ て行なうこと。
\item
  \textbf{加熱時間と不純物を取り除く作業に必要な時間}\ldots{}\ldots{}1時間半。
\end{itemize}

その後、ヴルテを布で漉す\footnote{ある程度濃度のある液体やピュレを布で漉す場合、昔は「二人が
  かりで行なう必要があり、それぞれが巻いた布の端を左手に持ち、右手に
  持った木杓子を使って圧し搾る」(『ラルース・ガストロノミーク』初版、
  1938年)という方法が一般的だった。}。陶製の鍋に移してゆっくり混ぜながら
完全に冷ます。

\maeaki

\hypertarget{ux9d8fux306eux30f4ux30ebux30c6}{%
\subsubsection{鶏のヴルテ}\label{ux9d8fux306eux30f4ux30ebux30c6}}

\hypertarget{veloute-de-volaille}{%
\paragraph{Velouté de volaille}\label{veloute-de-volaille}}

\index{きほんそーす@基本ソース!とりのうるて@鶏のヴルテ}
\index{そーす@ソース!きほん@基本---!とりのうるて@鶏のヴルテ}
\index{うるて@ヴルテ!とりのうるて@鶏の---(ヴルテドヴォライユ)}
\index{そーす@ソース!うるてとり@ヴルテ(鶏)}
\index{ぶるーて@ブルーテ ⇒ ヴルテ}
\index{うおらいゆ@ヴォライユ!うるてとうおらいゆ@ヴルテドヴォライユ(鶏
のヴルテ)} \index{かきん@家禽!とりのうるて@鶏のヴルテ}
\index{veloute@velouté!volaille@--- de Volaille}
\index{sauce@sauce!veloute volaille@Velout\'e de Volaille}

このヴルテの作り方だが、上述の標準的なヴルテと、材料比率と作業はまった
く同じ。使用する液体として鶏の白いフォン(フォンドヴォライユ)を使う。

\maeaki

\hypertarget{ux9b5aux6599ux7406ux7528ux30f4ux30ebux30c6}{%
\subsubsection{魚料理用ヴルテ}\label{ux9b5aux6599ux7406ux7528ux30f4ux30ebux30c6}}

\hypertarget{veloute-de-poisson}{%
\paragraph{Velouté de poisson}\label{veloute-de-poisson}}

\index{そーす@ソース!きほん@基本---!さかなりようりよううるて@魚料理用ヴルテ}
\index{きほんそーす@基本ソース!さかなりようりよううるて@魚料理用ヴルテ}
\index{うるて@ヴルテ!さかなうるて@魚料理用---} \index{そーす@ソース!う
るてさかな@ヴルテ(魚料理用)} \index{veloute@velouté!poisson@--- de
Poisson} \index{sauce@sauce!veloute poisson@Velouté de Poisson}

ルーと液体の分量は標準的なヴルテとまったく同じだが、仔牛のフォンではな
く魚のフュメを用いて作る。

ただし、魚を素材として用いるストックはどれもそうだが、手早く作業するこ
と。不純物を取り除く作業も20分程度にとどめること。その後、布で漉し、陶
製の鍋に移してゆっくり混ぜながら完全に冷ます。

\maeaki

\hypertarget{ux30bdux30fcux30b9ux30a2ux30ebux30deux30f3ux30c9ux30d1ux30eaux98a8ux30bdux30fcux30b9102024}{%
\subsubsection[ソース・アルマンド(パリ風ソース)]{\texorpdfstring{ソース・アルマンド(パリ風ソース\footnote{原書では「パリ風ソース(元ソース・アルマンド)」となってい
  るが、後述のように、こんにちでもソース・アルマンドの名称のほうが一
  般的であるため、ここではSauce Parisienneの「訳語」としてソース・ア
  ルマンドをあてることとした。})}{ソース・アルマンド(パリ風ソース)}}\label{ux30bdux30fcux30b9ux30a2ux30ebux30deux30f3ux30c9ux30d1ux30eaux98a8ux30bdux30fcux30b9102024}}

\hypertarget{sauce-allemande}{%
\paragraph{Sauce parisienne (ex-Allemande)}\label{sauce-allemande}}

\index{そーす@ソース!きほん@基本---!あるまんと@---・アルマンド}
\index{きほんそーす@基本ソース!あるまんと@---・アルマンド}
\index{そーす@ソース!ぱりふう@パリ風--- ⇒ ---・アルマンド}
\index{はりふう@パリ風!そーす@---ソース ⇒ ---・アルマンド}
\index{といつふう@ドイツ風!そーす@ソース・アルマンド(ドイツ風ソース)}
\index{あるまん@アルマン(ド)!そーす@ソース・アルマンド}
\index{sauce@sauce!00grandes@*Grandes ---s de Base!--- Allemande}
\index{sauce@sauce!parisienne@--- parisienne (ex-Allemande)}
\index{parisien@parisien!sauce@Sauce Parisienne = Sauce Allemande}
\index{allemand@allemand!sauce@Sauce allemande (--- Parisienne)}

(仕上がり1 L分)

標準的なヴルテに卵黄でとろみを付けたソース。

\begin{itemize}
\item
  \textbf{標準的なヴルテ}\ldots{}\ldots{}1 L。
\item
  \textbf{追加素材}\ldots{}\ldots{}卵黄5個、白いフォン(冷たいもの)\undemi{}
  L、粗く砕
  いたこしょう1ひとつまみ、すりおろしたナツメグ少々、マッシュルームの煮
  汁2 dl、レモン汁少々。
\item
  \textbf{作業手順}\ldots{}\ldots{}厚手のソテー鍋にマッシュルームの茹で汁と白いフォン、卵
  黄、粗く砕いたこしょう、ナツメグ、レモン汁を入れる。泡立て器でよく混ぜ、
  そこにヴルテを加える。火にかけて沸騰させ、強火で\deuxtiers{}量になるまで、ヘラ
  で混ぜながら煮詰める。
\end{itemize}

ヘラの表面がソースでコーティングされる状態になるまで煮詰めたら、布で漉
す。

膜が張らないよう、表面にバターのかけらをいくつか載せてやり、湯煎にかけ
ておく。

\begin{itemize}
\tightlist
\item
  \textbf{仕上げ}\ldots{}\ldots{}提供直前に、バター100
  gを加えて仕上げる。
\end{itemize}

\hypertarget{ux539fux6ce8-3}{%
\subparagraph{【原注】}\label{ux539fux6ce8-3}}

ソース・アルマンド(ドイツ風)とも呼ばれるが、本書では「パリ風」の名称
を採用した。そもそも「アルマンド」というの名称に正当性がないからだ。習
慣としてそう呼ばれてきただけであって、明らかに理屈に合わない名称だ
\footnote{エスコフィエは普仏戦争に従軍した経歴があり、ドイツ嫌いとし
  て知られていた。}。1883年に雑誌「料理技術」に某タヴェルネ氏が寄せた記事
には、当時ある優秀な料理人がアルマンドなどという理屈に合わない名称を使
うのはやめたという話が出ている。

こんにち既に「パリ風ソース」の名称を採用している料理長もいる。そう呼ん
だほうが好ましいわけだが、残念なことにまだ一般的にはなっていない
\footnote{エスコフィエの願いもむなしく、現代においてもソース・アルマ
  ンドの名称で定着している。この「全注解」においても以後は「ソース・
  アルマンド」と訳しているので注意されたい。なお、「ドイツ風」とい
  うソース名の由来について、ソースの淡い黄色がドイツ人に多い金髪を
  想起させるからだとカレームは述べている。}。

\maeaki

\hypertarget{ux30bdux30fcux30b9ux30b7ux30e5ux30d7ux30ecux30fcux30e0102023}{%
\subsubsection[ソース・シュプレーム]{\texorpdfstring{ソース・シュプレーム\footnote{suprême
  原義は「至高の」だが、料理においてはしばしば鶏や鴨
  の胸肉、白身魚のフィレなどを意味する。また、このソースのように、と
  くに意味もなくこの名を料理につけられているケースも多い。}}{ソース・シュプレーム}}\label{ux30bdux30fcux30b9ux30b7ux30e5ux30d7ux30ecux30fcux30e0102023}}

\hypertarget{sauce-supreme}{%
\paragraph{Sauce supême}\label{sauce-supreme}}

\index{きほんそーす@基本ソース!しゆふれーむ@---・シュプレーム}
\index{そーす@ソース!きほん@基本---!しゆふれーむ@---・シュプレーム}
\index{そーす@ソース!そーすしゆふれーむ@---・シュプレーム}
\index{しゆふれーむ@シュプレーム!そーす@ソース・---}
\index{sauce@sauce!00grandes@*Grandes ---s de Base!supreme@--- Suprême}
\index{sauce@sauce!supreme@--- Suprême}
\index{supreme@suprême!sauce@Sauce ---}

\protect\hyperlink{veloute-de-volaille}{鶏のヴルテ}に生クリーム\footnote{フランスの生クリームのうち、料理でよく使われるのは、日本の
  生クリームにやや近い「クレーム・フレッシュ・パストゥリゼ」(低温殺
  菌した生クリームで乳脂肪分30〜38%)のほか、「クレーム・フレッシュ・
  エペス」(低温殺菌後に乳酸醗酵させたもので日本で一般的な生クリーム
  より濃度がある)、「クレーム・ドゥーブル」(殺菌後に乳酸醗酵させた
  もので乳脂肪分40%程度でかなり濃度がある)などがある。}を加えてなめら
かに仕上げ\footnote{monter
  モンテ。原義は「上げる、ホイップする」だが、ソースの
  仕上げの際などに、バターや生クリームを加えて、なめらかに仕上げるこ
  とも「モンテ」の語を使用する場合が多い。}たもの。ソース・シュプレームは、正しく作った場合
「白さの\ruby{際}{きわ}だったとても繊細な」仕上がりのものでなくてはい
けない。

(仕上がり1 L分)

\begin{itemize}
\item
  \textbf{鶏のヴルテ}\ldots{}\ldots{}1 L。
\item
  \textbf{追加素材}\ldots{}\ldots{}鶏の白いフォン1
  L、マッシュルームの茹で汁1dl、良質な生 クリーム2 \undemi{} dl。
\item
  \textbf{作業手順}\ldots{}\ldots{}鍋に鶏のフォンとマッシュルームの茹で汁、鶏のヴルテを入
  れて強火にかけ、ヘラで混ぜながら、生クリームを少しずつ加え、煮詰めてい
  く。このヴルテと生クリームを煮詰めたものの分量は、上で示した仕上がり1L
  のソース・シュプレームを作るには、\untiers{}量まで煮詰まっていなくては
  ならない。
\end{itemize}

布で漉し、仕上げに1 dlの生クリームとバター80 gを加えてゆっくり混ぜなが
ら冷ますと、丁度最初のヴルテと同量になる。

\maeaki

\hypertarget{ux30d9ux30b7ux30e3ux30e1ux30ebux30bdux30fcux30b9102020}{%
\subsubsection[ベシャメルソース]{\texorpdfstring{ベシャメルソース\footnote{17世紀にルイ14世のメートルドテルを務めたこともあるルイ・ベ
  シャメイユLouis Béchameil(1630〜1703)の名が冠されているこのソー
  スは、彼自身の創案あるいは彼に仕えていた料理人によるものという説も
  あったが真偽は疑わしい。17世紀頃の成立であることは確かだが、おそら
  くは古くからあったソースを改良したものに過ぎず、また、19世紀前半の
  カレームのレシピはヴルテを煮詰め、卵黄と煮詰めた生クリームでとろみ
  を付けるというものだった。同様に1867年刊グフェ『料理の本』のレシピ
  も、炒めた仔牛肉と野菜に小麦粉を振りかけてからブイヨン注ぎ、これを
  煮詰め、漉してから生クリームを加えるというものだった。}}{ベシャメルソース}}\label{ux30d9ux30b7ux30e3ux30e1ux30ebux30bdux30fcux30b9102020}}

\hypertarget{sauce-bechamel}{%
\paragraph{Sauce Béchamel}\label{sauce-bechamel}}

\index{きほんそーす@基本ソース!へしやめる@ベシャメル---}
\index{そーす@ソース!きほん@基本---!へしやめる@ベシャメル---}
\index{そーす@ソース!へしやめる@ベシャメル---}
\index{へしやめる@ベシャメル!そーす@---ソース}
\index{sauce@sauce!00grandes@*Grandes ---s de Base!bechamel@--- Béchamel}
\index{sauce@sauce!bechamel@--- Béchamel}
\index{bechamel@Béchamel (sauce)}

(仕上がり 5 L分)

\begin{itemize}
\item
  \textbf{白いルー}\ldots{}\ldots{}650 g。
\item
  \textbf{使用する液体}\ldots{}\ldots{}沸かした牛乳5 L。
\item
  \textbf{追加素材}\ldots{}\ldots{}白身で脂肪のない仔牛肉300
  gをさいの目に切り、みじん切
  りにした玉ねぎ(小)2個分とタイム1枝、粗く砕いたこしょう1つまみ、塩25
  g とバターを鍋に入れて蓋をし、色付かないように弱火で蒸し煮したもの。
\item
  \textbf{作業手順}\ldots{}\ldots{}沸かした牛乳でルーを溶く。混ぜながら沸騰させる。ここ
  に、先に蒸し煮しておいた野菜と調味料、仔牛肉を加える。弱火で1時間煮込
  む。布で漉し\footnote{\protect\hyperlink{veloute}{ヴルテ}訳注参照。}、表面にバターのかけらをいくつか載せて膜が張らな
  いようにする。肉類を絶対に使わない\footnote{小斉のこと。\protect\hyperlink{sauce-espagnole-maigre}{魚料理用ソース・エスパニョ
    ル}訳注参照。}で調理する必要がある場合は、
  仔牛肉を省き、香味野菜などは上記のとおりに作ること。
\end{itemize}

このソースは次のようなやり方をすると手早く作ることも出来る。沸かした牛
乳に塩、薄切りにした玉ねぎ、タイム、粗く砕いたこしょう、ナツメグを加え
る。蓋をして弱火で10分煮る。これを漉してルーを入れた鍋の中に入れ、強火
にかけて沸騰させる。その後15〜20分だけ煮込めばいい。

\maeaki

\hypertarget{ux30c8ux30deux30c8ux30bdux30fcux30b9}{%
\subsubsection{トマトソース}\label{ux30c8ux30deux30c8ux30bdux30fcux30b9}}

\hypertarget{sauce-tomate}{%
\paragraph{Sauce tomate}\label{sauce-tomate}}

\index{きほんそーす@基本ソース!とまと@トマト---}
\index{そーす@ソース!きほん@基本---!とまと@トマト---}
\index{そーす@ソース!とまとそーす@トマト---}
\index{とまと@トマト!ソース@---ソース}
\index{sauce@sauce!00grandes@*Grandes ---s de Base!tomate@--- tomate}
\index{sauce@sauce!tomate@--- tomate}
\index{tomate@tomate!sauce@Sauce ---}

(仕上がり5L分)

\begin{itemize}
\item
  \textbf{主素材}\ldots{}\ldots{}トマトピュレ4 L、または生のトマト6 kg。
\item
  \textbf{ミルポワ}\ldots{}\ldots{}さいの目に切って下茹でしておいた塩漬け豚ばら肉140
  g 、1〜2 mm 角のさいの目に刻んだにんじん200 gと玉ねぎ150
  g、ローリエの葉 1枚、タイム1枝、バター100 g。
\item
  \textbf{追加素材}\ldots{}\ldots{}小麦粉150 g、白いフォン2
  L、にんにく2片。
\item
  \textbf{調味料}\ldots{}\ldots{}塩20 g、砂糖30 g、こしょう1つまみ。
\item
  \textbf{作業手順}\ldots{}\ldots{}厚手の片手鍋で、塩漬け豚ばら肉を軽く色付くまで炒める。
  ミルポワの野菜を加え、野菜も色よく炒める。小麦粉を振りかける。ブロンド
  色になるまで炒めてから、トマトピュレまたは潰した生トマトと白いフォン、
  砕いたにんにく、塩、砂糖、こしょうを加える。
\end{itemize}

火にかけて混ぜながら沸騰させる。鍋に蓋をして弱火のオーブンに入れ1時間
半〜2時間加熱する。

目の細かい漉し器または布で漉す。再度、火にかけて数分間沸騰させる。保存
用の器に移し、ソースが空気に触れて表面に膜が張らないよう、バターのかけ
らを載せてやる。

\hypertarget{ux539fux6ce8-4}{%
\subparagraph{【原注】}\label{ux539fux6ce8-4}}

トマトピュレを使い、小麦粉は使わず、その他は上記のとおりに作ってもいい。
漉し器か布で漉してから、充分な濃度になるまでしっかり煮詰めてやること。
\end{recette}\newpage
\hypertarget{ux30d6ux30e9ux30a6ux30f3ux7cfbux306eux6d3eux751fux30bdux30fcux30b9}{%
\section{ブラウン系の派生ソース}\label{ux30d6ux30e9ux30a6ux30f3ux7cfbux306eux6d3eux751fux30bdux30fcux30b9}}

\hypertarget{petites-sauces-brunes-composuxe9es}{%
\subsection{Petites Sauces Brunes
Composées}\label{petites-sauces-brunes-composuxe9es}}
\begin{recette}
\hypertarget{ux30bdux30fcux30b9ux30d3ux30acux30e9ux30fcux30c91}{%
\subsubsection[ソース・ビガラード]{\texorpdfstring{ソース・ビガラード\footnote{ビガラードは本来、南フランスで栽培されるビターオレンジの一種。}}{ソース・ビガラード}}\label{ux30bdux30fcux30b9ux30d3ux30acux30e9ux30fcux30c91}}

\hypertarget{sauce-bigarade}{%
\paragraph{Sauce Bigarade}\label{sauce-bigarade}}

\index{そーす@ソース!びがらーど@---・ビガラード} \index{びがらーど@ビ
ガラード!そーす@ソース・---} \index{sauce@sauce!bigarade@--- Bigarade}
\index{bigarade@bigarade!sauce@Sauce ---}

\hypertarget{ux4ed4ux9d28ux306eux30d6ux30ecux30bc2-ux7528}{%
\subparagraph[仔鴨のブレゼ 用]{\texorpdfstring{仔鴨のブレゼ\footnote{料理の仕立てとしてのブレゼはたんに「蒸し煮」することではない。原
  則的な手順をごく簡単に述べておく。厚めに輪切りにしたにんじんと玉ね
  ぎをバターまたはラードで炒め、ブーケガルニとともに鍋に入れる。表面
  を色よく焼き固めた肉を、脂身の少ない肉の場合には豚背脂のシートで包
  んで素材がぴったり入る大きさ鍋に入れ、\protect\hyperlink{fonds-brun}{茶色いフォン}
  を注ぎ、蓋をしてオーブンに入れ、微沸騰の状態を保つようにして煮込む。
  火が通ったら肉を取り出し、鍋に残った煮汁でソースを作る。詳細につい
  ては\protect\hyperlink{}{第7章 肉料理}参照。}
用}{仔鴨のブレゼ 用}}\label{ux4ed4ux9d28ux306eux30d6ux30ecux30bc2-ux7528}}

仔鴨をブレゼした際の煮汁を漉してから浮き脂を取り除き\footnote{dégraisser
  デグレセ。}、煮詰める。 煮詰まったらさらに目の細かい布で漉し、ソース1
Lあたりオレンジ4個とレモ ン1個の搾り汁でのばす。

\hypertarget{ux4ed4ux9d28ux306eux30ddux30efux30ec3ux7528}{%
\subparagraph[仔鴨のポワレ用]{\texorpdfstring{仔鴨のポワレ\footnote{ポワレについても簡単に述べておく。本書においてポワレは「フライパ
  ンで焼く」という意味で用いられることは決してない(フライパンで魚な
  どを焼くことをポワレと呼ぶようになったのは20世紀後半のこと)。本書
  では「ローストの一種」と定義されており(この点がカレームとはまった
  く異なる)、3〜4mm角に切った香味野菜(マティニョン)を生のまま鍋の
  底に入れ、その上に味付けをした肉を置く。溶かしバターをかけてから、
  蓋をして中火のオーブンに入れて蒸し焼きにする。時折様子を見て溶かし
  バターをかけてやること。肉に火が通ったら鍋から取り出し、\protect\hyperlink{fonds-de-veau-brun}{茶色い仔
  牛のフォン}を注いで弱火にかけて10分程煮込み、
  マティニョンとして用いた野菜から風味を引き出してソースにする。これ
  がレシピにある「ポワレのフォン」となる。}用}{仔鴨のポワレ用}}\label{ux4ed4ux9d28ux306eux30ddux30efux30ec3ux7528}}

仔鴨をポワレのフォンから浮き脂を取り除き、でんぷんで軽くとろみ付け
する。砂糖20gに大さじ\undemi{}杯のヴィネガーを加えて火にかけカラメル状
にしたものを加える。ブレゼ用と同様に、オレンジとレモンの搾り汁でのばす。

仔鴨のブレゼ用、ポワレ用いずれの場合も、細かい千切りにしてよく下茹でし
ておいたオレンジの皮大さじ2とレモンの皮\footnote{柑橘類の表皮を薄く剥いてごく細い千切りにしたり、器具を用いてお
  ろしたものをゼスト zeste と呼ぶ。千切りにしたものは苦味を取り除く
  ために下茹ですることが多い。}大さじ1を加えて仕上げる。

\maeaki

\hypertarget{ux30dcux30ebux30c9ux30fcux98a8ux30bdux30fcux30b9}{%
\subsubsection{ボルドー風ソース}\label{ux30dcux30ebux30c9ux30fcux98a8ux30bdux30fcux30b9}}

\hypertarget{sauce-bordelaise}{%
\paragraph{Sauce Bordelaise}\label{sauce-bordelaise}}

\index{そーす@ソース!ぼるどーふう@ボルドー風---} \index{ぼるどーふう@
ボルドー風!そーす@---ソース} \index{sauce@sauce!bordelaise@---
Bordelaise} \index{bordelais@bordelais!sauce@Sauce Bordelaise}

赤ワイン3 dlにエシャロットのみじん切り大さじ2、粗く砕いたこしょう、タ
イム、ローリエの葉\undemi{}枚を加えて火にかけ、\unquart{}量になるまで
煮詰める。ソース・エスパニョル1 dlを加えて火にかけ、浮いてくる夾雑物を
丁寧に取り除きながら弱火で15分間煮る。目の細かい布で漉す。

溶かしたグラスドヴィアンド大さじ1杯とレモン汁\unquart{}個分、細かいさ
いの目か輪切りにしてポシェしておいた牛骨髄を加えて仕上げる。

\ldots{}\ldots{}牛、羊の赤身肉のグリル用

【原注】こんにちではボルドー風ソースをこのように赤ワインを用いて作るが、
本来的には誤りである。元来は白ワインが用いられていた。これは\protect\hyperlink{sauce-bonnefoy}{ボルドー
風ソース・ボヌフォワ}として後述。

\maeaki

\hypertarget{ux30d6ux30ebux30b4ux30fcux30cbux30e5ux98a8ux30bdux30fcux30b9}{%
\subsubsection{ブルゴーニュ風ソース}\label{ux30d6ux30ebux30b4ux30fcux30cbux30e5ux98a8ux30bdux30fcux30b9}}

\hypertarget{sauce-bourguignonne}{%
\paragraph{Sauce Bourguignonne}\label{sauce-bourguignonne}}

\index{そーす@ソース!ぶるごーにゅふう@ブルゴーニュ風---} \index{ぶるごー
にゅふう@ブルゴーニュ風!そーす@---ソース}
\index{sauce@sauce!bourguignonne@--- Bourguignonne}
\index{bourguignon@bourguignon!sauce@Sauce Bourguignonne}

上質の赤ワイン1\undemi{} L に、エシャロット5個の薄切りとパセリの枝、タ
イム、ローリエの葉\undemi{}枚、マッシュルームの切りくず\footnote{料理に使うマッシュルームは通常、トゥルネ(包丁を持った側の手は動
  かさずに材料を回して切ることからついた用語)すなわち螺旋状に切って
  供するが、その際に少なくない量の切りくずが出るのでこれを使う。}25gを加えて、
半量になるまで煮詰める。布で漉し、ブールマニエ80g(バター45gと小麦粉
35g)を加えてとろみを付ける。提供直前にバター150gを溶かし込み、カイエ
ンヌ\footnote{赤唐辛子の粉末だがカイエンヌは本来、品種名。日本のタカノツメと
  比べると辛さもややマイルドで、風味も異なる。}ごく少量で加えて風味よく仕上げる。

\ldots{}\ldots{}いろいろな卵料理や、家庭料理に好適なソース。

\maeaki

\hypertarget{ux30d6ux30ebux30bfux30fcux30cbux30e5ux98a8ux30bdux30fcux30b9}{%
\subsubsection{ブルターニュ風ソース}\label{ux30d6ux30ebux30bfux30fcux30cbux30e5ux98a8ux30bdux30fcux30b9}}

\hypertarget{sauce-bretonne}{%
\paragraph{Sauce Bretonne}\label{sauce-bretonne}}

\index{そーす@ソース!ぶるたーにゅふうちゃいろ@ブルターニュ風--- (茶色)}
\index{ぶるたーにゅふう@ブルターニュ風!そーすちゃいろ@---ソース (茶色)}
\index{sauce@sauce!bretonne brune@--- Bretonne (brune)}
\index{breton@breton!sauce brune@Sauce Bretonne (brune)}

中位の玉ねぎ2個をみじん切りにして、バターでブロンド色になるまで炒める。
白ワイン2\undemi{}dlを注ぎ、半量になるまで煮詰める。ここにソース・エス
パニョル3\undemi{}およびトマトソース同量を加える。7〜8分間煮立ててから、
刻んだパセリを加えて仕上げる。

【原注】このソースは{[}白いんげん豆のブルターニュ風{]}以外にはほとんど使わ
れない。

\maeaki

\hypertarget{ux30bdux30fcux30b9ux30b9ux30eaux30fcux30ba6}{%
\subsubsection[ソース・スリーズ]{\texorpdfstring{ソース・スリーズ\footnote{スリーズ
  cerises はさくらんぼのこと。このレシピでグロゼイユ(す
  ぐり)のジュレを用いるが、古くはさくらんぼを用いていたことからこの
  名称となった。}}{ソース・スリーズ}}\label{ux30bdux30fcux30b9ux30b9ux30eaux30fcux30ba6}}

\hypertarget{sauce-aux-cerises}{%
\paragraph{Sauce aux cerises}\label{sauce-aux-cerises}}

\index{そーす@ソース!すりーず@---・スリーズ}
\index{sauce@sauce!cerise@--- aux Cerises}

ポルト酒2dlにイギリス風ミックススパイス\footnote{Mixed
  spiceのこと。Pudding spiceとも呼ばれる。シナモン、ナツメ
  グ、オールスパイスの組み合わせが典型的。これにクローブ、生姜、コリ
  アンダーシード、キャラウェイシードなどが加わっていることも多い。}1つまみと、すりおろしたオレ
ンジの皮を大さじ\undemi{}杯加えて\deuxtiers{}量になるまで煮詰める。
\protect\hyperlink{}{グロゼイユのジュレ}
2\undemi{}を加え、仕上げにオレンジ果汁を加える。

\ldots{}\ldots{}大型ジビエの料理用だが、鴨のポワレやブレゼにも用いられる。

\maeaki

\hypertarget{ux30bdux30fcux30b9ux30b7ux30e3ux30f3ux30d4ux30cbux30e7ux30f37}{%
\subsubsection[ソース・シャンピニョン]{\texorpdfstring{ソース・シャンピニョン\footnote{champignons
  キノコ全般を意味する語だが、単独で用いられる場合はい
  わゆるマッシュルームを指す。}}{ソース・シャンピニョン}}\label{ux30bdux30fcux30b9ux30b7ux30e3ux30f3ux30d4ux30cbux30e7ux30f37}}

\hypertarget{sauce-aux-champignons}{%
\paragraph{Sauce aux Champignons}\label{sauce-aux-champignons}}

\index{そーす@ソース!まっしゅるーむちゃいろ@マッシュルーム--- (茶色)}
\index{まっしゅるーむ@マッシュルーム!そーすちゃいろ@---ソース (茶色)}
\index{sauce@sauce!champignons brune@--- aux Champignons (brune)}
\index{champignon@champignon!sauce brune@Sauce aux Champignons
(brune)}

マッシュルームの茹で汁2\undemi{} dl を半量になるまで煮詰める。
\protect\hyperlink{sauce-demi-glace}{ソース・ ドゥミグラス}8
dlを加えて数分間煮立てる。布で漉し、 バター50
gを投入して味を調え、あらかじめ下茹でしておいた小さめのマッシュ
ルームの笠100 gを加えて仕上げる。

\maeaki

\hypertarget{ux30bdux30fcux30b9ux30b7ux30e3ux30ebux30adux30e5ux30c6ux30a3ux30a8ux30fcux30eb8}{%
\subsubsection[ソース・シャルキュティエール]{\texorpdfstring{ソース・シャルキュティエール\footnote{シャルキュトリ(豚肉加工業)風、の意。Charcutrieの語源はchar(肉)
  +cuite(調理された)+rie(業)。ハムやソーセージなどと定番の組合せ
  であるマスタードを使う\protect\hyperlink{sauce-robert}{ソース・ロベール}と、おなじ
  く定番のつけ合わせであるコルニション(小さいうちに収穫してヴィネガー
  漬けにしたきゅうり。専用品種がある)を使うことに由来。}}{ソース・シャルキュティエール}}\label{ux30bdux30fcux30b9ux30b7ux30e3ux30ebux30adux30e5ux30c6ux30a3ux30a8ux30fcux30eb8}}

\hypertarget{sauce-charcutiere}{%
\paragraph{Sauce Charcutière}\label{sauce-charcutiere}}

\index{そーす@ソース!しゃるきゅとりふう@シャルキュトリ風---} \index{しゃ
るきゅとりふう@シャルキュトリ風!そーす@---ソース}
\index{sauce@sauce!charcutière@--- Charcutière}
\index{charcutier@charcutier!sauce@Sauce Charcutière}

提供直前に、\protect\hyperlink{sauce-robert}{ソース・ロベール}1 L
に細さ2mm程度で短かめの千切り\footnote{1〜2mm程度の細さの千切りにした野菜などをジュリエンヌjulienneと呼
  ぶ。}
にしたものを加える(\protect\hyperlink{sauce-robert}{ソース・ロベール}参照)。

\maeaki

\hypertarget{ux30bdux30fcux30b9ux30b7ux30e3ux30b9ux30fcux30eb10}{%
\subsubsection[ソース・シャスール]{\texorpdfstring{ソース・シャスール\footnote{狩人風、の意。古くは猟獣肉をすり潰したものを使った料理を指した
  という説もある。マッシュルームとエシャロット、白ワインを使うのが特
  徴であり、このソースを使った料理にも「シャスール」の名が付けられる。}}{ソース・シャスール}}\label{ux30bdux30fcux30b9ux30b7ux30e3ux30b9ux30fcux30eb10}}

\hypertarget{sauce-chasseur}{%
\paragraph{Sauce Chasseur}\label{sauce-chasseur}}

\index{そーす@ソース!しゃすーる@---・シャスール} \index{しゃすーる@シャ
スール!そーす@ソース・---} \index{sauce@sauce!chasseur@--- Chasseur}
\index{chasseur@chasseur!sauce@Sauce ---}

生のマッシュルームを薄切りにしたもの150gをバターで炒める。エシャロット
\footnote{échalote
  玉ねぎによく似ているが、小ぶりで水分が少なく、香味野菜
  としてよく用いられる。伝統的な品種は種子ではなく種球を植えて栽培す
  る。なお、日本でしばしば「エシャレット」の名称で流通しているものは
  ラッキョウの若どりであり、フランス料理で用いるエシャロットとはまっ
  たく異なる。}のみじん切り大さじ2\undemi{}杯を加えてさらに軽く炒め、白ワイン3
dl
を注ぎ、半量になるまで煮詰める。\protect\hyperlink{sauce-tomate}{ソマトソース}3
dl と\protect\hyperlink{sauce-demi-glace}{ソース・ドゥミグラス}2
dlを加える。数分間沸騰さ せたら、バター150 gと、セルフイユ\footnote{cerfeuil
  日本ではチャービルとも呼ばれるセリ科のハーブ。}とエストラゴン\footnote{estragon
  日本ではタラゴンとも呼ばれるヨモギ科のハーブ。フランス
  料理ではとても好まれる重要なハーブのひとつ。フレンチタラゴンとロシ
  アンタラゴンの2種がある。料理に用いるのはフレンチタラゴンであり、
  この品種は種子ではなく株分けや挿し芽で殖やして栽培される。寒さには
  比較的強いが、日本の梅雨の湿度や夏の暑さには弱い。}をみじん切り
にしたもの大さじ1\undemi{}杯を加えて仕上げる。

\maeaki

\hypertarget{ux30bdux30fcux30b9ux30b7ux30e3ux30b9ux30fcux30ebux30a8ux30b9ux30b3ux30d5ux30a3ux30a8ux6d41}{%
\subsubsection{ソース・シャスール(エスコフィエ流)}\label{ux30bdux30fcux30b9ux30b7ux30e3ux30b9ux30fcux30ebux30a8ux30b9ux30b3ux30d5ux30a3ux30a8ux6d41}}

\hypertarget{sauce-chasseur-procede-escoffier}{%
\paragraph{Sauce Chasseur (Procédé
Escoffier)}\label{sauce-chasseur-procede-escoffier}}

\index{そーす@ソース!しゃすーるえすこふぃえ@---・シャスール(エスコフィ
エ流)} \index{しゃすーる@シャスール!そーすしゃすーるえすこふぃえ@ソー
ス・--- (エスコフィエ流)} \index{sauce@sauce!chasseur escoffier@---
Chasseur (Procédé Escoffier)} \index{chasseur@chasseur!sauce
escoffier@Sauce --- (Procédé Escoffier)}

生のマッシュルームを薄切りにしたもの150gを、バターと植物油で軽く色付く
まで炒める。みじん切りにしたエシャロット大さじ1杯を加え、なるべくすぐ
に余分な油をきる。白ワイン2dl とコニャック約50ml を注ぎ、半量になるま
で煮詰める。\protect\hyperlink{sauce-demi-glace}{ソース・ドゥミグラス}4
dlと{[}トマトソー ス{]}2
dl、\protect\hyperlink{glace-de-viande}{グラスドヴィアンド}大さじ\undemi{}杯を加え
る。

5分間沸騰させたら、仕上げにパセリのみじん切り少々を加える。

\maeaki

\hypertarget{ux8336ux8272ux3044ux30bdux30fcux30b9ux30b7ux30e7ux30d5ux30edux30ef15}{%
\subsubsection[茶色いソース・ショフロワ]{\texorpdfstring{茶色いソース・ショフロワ\footnote{chaudショ「熱い、温かい」とfroidフロワ「冷たい」の合成語で、火
  を通した肉や魚を冷まし、表面にこのソース・ショフロワを覆うように塗
  り付け、さらにジュレを覆いかけた料理。料理の発祥については諸説あり、
  なかでもルイ15世に仕えていた料理長ショフロワChaufroixが考案したと
  いう説を支持してなのか、英語ではこの料理をChaufroixと綴ることも多
  い。Chaud-froidの表記は19世紀後半には文献に見られる。なお、複数形
  はchauds-froidsと綴る。トリュフの薄切りやエストラゴンなどのハーブ
  その他で表面に華麗な装飾を施すことが19世紀には盛んに行なわれていた。
  現代でも装飾に凝った仕立てにするケースは多い。}}{茶色いソース・ショフロワ}}\label{ux8336ux8272ux3044ux30bdux30fcux30b9ux30b7ux30e7ux30d5ux30edux30ef15}}

\hypertarget{sauce-chaud-froid-brune}{%
\paragraph{Sauce Chaud-froid brune}\label{sauce-chaud-froid-brune}}

\index{そーす@ソース!しょふろわちゃいろ@---・ショフロワ(茶色)}
\index{しょふろわ@ショフロワ!そーす(ちゃいろ)@ソース・--- (茶色)}
\index{sauce@sauce!chaud-froid brune@--- Chaud-froid brune}
\index{chaud-froid@chaud-froid!sauce brune@Sauce --- brune}

(仕上がり1L 分)

\protect\hyperlink{sauce-demi-glace}{ソース・ドゥミグラス}\troisquarts{}
Lとトリュフエッ センス1 dl、ジュレ6〜7 dlを用意する。

ソース・ドゥミグラスにトリュフエッセンスを加えて、強火で煮詰めるが、こ
の時に鍋から離れないこと。煮詰めながらジュレを少量ずつ加えていく。最終
的に\deuxtiers{}量程度まで煮詰める。

味見をして、ソースがショフロワに使うのに丁度いい濃さになっているか確認
すること。

マデラ酒またはポルト酒\undemi{}dlを加える。布で漉し、ショフロワの主素
材の表面に塗り付けるのに丁度いい固さになるまで、丁寧にゆっくり混ぜなが
ら冷ます。

\maeaki

\hypertarget{ux8336ux8272ux3044ux30bdux30fcux30b9ux30b7ux30e7ux30d5ux30edux30efux9d28ux7528}{%
\subsubsection{茶色いソース・ショフロワ(鴨用)}\label{ux8336ux8272ux3044ux30bdux30fcux30b9ux30b7ux30e7ux30d5ux30edux30efux9d28ux7528}}

\hypertarget{sauce-chaud-froid-brune-pour-canards}{%
\paragraph{Sauce Chaud-froid brune pour
Canards}\label{sauce-chaud-froid-brune-pour-canards}}

\index{そーす@ソース!しょふろわちゃいろかもよう@茶色い---・ショフロワ
(鴨用)} \index{しょふろわ@ショフロワ!ちゃいろいそーすしょふろわかも
よう@茶色いソース・---(鴨用)} \index{sauce@sauce!chaud-froid brune
pour canards@--- Chaud-froid brune pour Canards}
\index{chaud-froid@chaud-froid!sauce brune pour Canards@Sauce ---
brune pour Canards}

作り方は上記、\protect\hyperlink{sauce-chaud-froid-brune}{茶色いソース・ショフロワ}と同
様だが、トリュフエッセンスではなく、鴨のガラでとったフュメ1\undemi{}
dlを用いること。また、上記のレシピよりややしっかり煮詰めること。

ソースを布で漉したら、オレンジ3個分の搾り汁、とオレンジの皮をごく薄く
剥いて細かい千切りにしたもの\footnote{zeste
  ゼスト。オレンジやレモンの皮の表面を器具を用いてすりおろ
  すか、ナイフでごく薄く表皮を向き、細かい千切りにしたもの。ここでは
  後者を使う指定になっている。}大さじ2杯を加える。オレンジの皮の千切
りはしっかりと下茹でしてよく水気をきっておくこと。

\maeaki

\hypertarget{ux8336ux8272ux3044ux30bdux30fcux30b9ux30b7ux30e7ux30d5ux30edux30efux30b8ux30d3ux30a8ux7528}{%
\subsubsection{茶色いソース・ショフロワ(ジビエ用)}\label{ux8336ux8272ux3044ux30bdux30fcux30b9ux30b7ux30e7ux30d5ux30edux30efux30b8ux30d3ux30a8ux7528}}

\hypertarget{sauce-chaud-froid-brune-pour-gibier}{%
\paragraph{Sauce Chaud-froid brune pour
Gibier}\label{sauce-chaud-froid-brune-pour-gibier}}

\index{そーす@ソース!しょふろわちゃいろじびえよう@茶色い---・ショフロ
ワ(ジビエ用)} \index{しょふろわ@ショフロワ!そーすしょふろわじびえよ
う@茶色いソース・---(ジビエ用)} \index{sauce@sauce!chaud-froid brune
pour Gibier@--- Chaud-froid brune pour Gibier}
\index{chaud-froid@chaud-froid!sauce brune pour Gibier@Sauce --- brune
pour Gibier}

作り方は上記\protect\hyperlink{sauce-chaud-froid-brune}{標準的なソース・ショフロワ}と同
じだが、トリュフエッセンスではなく、ショフロワとして供するジビエのガラ
でとったフュメ\footnote{\protect\hyperlink{fonds-de-gibier}{ジビエのフォン}参照。}2dlを用いること。

\maeaki

\hypertarget{ux30c8ux30deux30c8ux5165ux308aux30bdux30fcux30b9ux30b7ux30e7ux30d5ux30edux30ef}{%
\subsubsection{トマト入りソース・ショフロワ}\label{ux30c8ux30deux30c8ux5165ux308aux30bdux30fcux30b9ux30b7ux30e7ux30d5ux30edux30ef}}

\hypertarget{sauce-chaud-froid-tomatee}{%
\paragraph{Sauce Chaud-froid tomatée}\label{sauce-chaud-froid-tomatee}}

\index{そーす@ソース!しょふろわとまといり@トマト入り---・ショフロワ}
\index{しょふろわ@ショフロワ!そーす(とまといり)@トマト入りソース・---}
\index{sauce@sauce!chaud-froid tomatée@--- Chaud-froid tomatée}
\index{chaud-froid@chaud-froid!sauce tomatée@Sauce --- tomatée}

良質で、既によく煮詰めてあるトマトピュレ1 Lを、さらに煮詰めながら7〜8
dlのジュレを少しずつ加えていく。全体量が1L以下になるまで煮詰めること。

布で漉し、使いやすい固さになるまで、ゆっくり混ぜながら冷ます。

\maeaki

\hypertarget{ux30bdux30fcux30b9ux30b7ux30e5ux30f4ux30ebux30a4ux30e6}{%
\subsubsection{ソース・シュヴルイユ}\label{ux30bdux30fcux30b9ux30b7ux30e5ux30f4ux30ebux30a4ux30e6}}

\hypertarget{sauce-chevreuil}{%
\paragraph{Sauce Chevreuil}\label{sauce-chevreuil}}

\index{しゅうるいゆ@シュヴルイユ!そーす@ソース・---} \index{そーす@ソー
ス!しゅうるいゆ@---・シュヴルイユ} \index{のろしか@ノロ鹿!そーすしゅう
るいゆ@ソース・シュヴルイユ} \index{sauce@sauce!chevreuil@---
Chevreuil} \index{chevreuil@chevreuil!sauce@Sauce ---}

\protect\hyperlink{sauce-poivrade}{標準的なソース・ポワヴラード})と同様に作るが、

\begin{enumerate}
\def\labelenumi{\arabic{enumi}.}
\item
  マリネした牛・羊肉の料理に添える場合\footnote{chevreuil
    シュヴルイユはノロ鹿のことだが、このように事前にマリ
    ネした牛・羊肉を用いた料理にもこのソースを使い「シュヴルイユ(風)」
    と\ruby{謳}{うた}う。1806年刊ヴィアール『帝国料理の本』においてノ
    ロ鹿のフィレは香辛料を加えたワインヴィネガーで48時間マリネしてから
    調理すると書かれている。オド『女性料理人のための本』では、確認出来
    た1834年の第4版から1900年の第78版に至るまで、ノロ鹿の項において
    「一週間もヴィネガーたっぷりの漬け汁でマリネするのはやりすぎだが、
    強い味が好みなら1〜4日間」香辛料と赤ワインあるいはヴィネガーでマリ
    ネするといい、と説明されている。つまり、ノロ鹿とは必ずマリネしてか
    ら調理するものという一種のコンセンサスがあったために、マリネした牛・
    羊肉の料理にも「シュヴルイユ(風)」の名称が謳われるようになったと考
    えられる。}は、ハム入りの\protect\hyperlink{mirepoix}{ミルポ
  ワ}を加える。
\item
  ジビエ料理に添える場合は、そのジビエの端肉を加える。
\end{enumerate}

素材をヘラなどで強く押し付けるようにして漉す\footnote{シノワ(\protect\hyperlink{sauce-espagnole}{ソース・エスパニョル}訳注参照)などを用いる。}。良質の赤ワイン
1\undemi{}dlをスプーン1杯ずつ加えながら煮て、浮き上がってくる不純物を
丁寧に取り除いていく\footnote{dépouiller デプイエ ≒ écumer エキュメ}。

最後に、カイエンヌごく少量と砂糖1つまみを加えて味を\ruby{調}{とと
の}え、布で漉す。

\maeaki

\hypertarget{ux30bdux30fcux30b9ux30b3ux30ebux30d9ux30fcux30eb23}{%
\subsubsection[ソース・コルベール]{\texorpdfstring{ソース・コルベール\footnote{17世紀の政治家、ジャン・バティスト・コルベール(1619〜1683)の
  名を冠したもの。}}{ソース・コルベール}}\label{ux30bdux30fcux30b9ux30b3ux30ebux30d9ux30fcux30eb23}}

\hypertarget{sauce-colbert}{%
\paragraph{Sauce Colbert}\label{sauce-colbert}}

\index{そーす@ソース!こるべーる@---・コルベール} \index{こるべーる@コ
ルベール!そーす@ソース・---} \index{sauce@sauce!colbert@--- Colbert}
\index{colbert@Colbert!sauce@Sauce ---}

\protect\hyperlink{}{メートルドテルバター}に\protect\hyperlink{glace-de-viande}{グラスドヴィアンド}を加え
たもののことだが、正しくは「\protect\hyperlink{}{コルベールバター}」と呼ぶべきものだ
\footnote{具体的なレシピは\protect\hyperlink{}{コルベールバター}参照のこと。}。

また、コルベールバターと\protect\hyperlink{sauce-chateaubriand}{ソース・シャトーブリアン}との違いを明確にさ
せようとして、メートルドテルバターにエストラゴンを加える者もいる。だが、
必ずそうすべきということではない。実際、ブール・コルベールとソース・シャ
トーブリアンは明らかに違うものだからだ。ソース・シャトーブリアンは軽く
仕上げたグラスドヴィアントにバターとパセリのみじん切りを加えたものであ
る。一方、コルベールバターあるいはソース・コルベールと呼ばれているもの
はあくまでもバターが主であって、グラスドヴィアンドは補助的なものに過ぎ
ない。

\maeaki

\hypertarget{ux30bdux30fcux30b9ux30c7ux30a3ux30a2ux30fcux30d6ux30eb25}{%
\subsubsection[ソース・ディアーブル]{\texorpdfstring{ソース・ディアーブル\footnote{悪魔の意。}}{ソース・ディアーブル}}\label{ux30bdux30fcux30b9ux30c7ux30a3ux30a2ux30fcux30d6ux30eb25}}

\hypertarget{sauce-diable}{%
\paragraph{Sauce Diable}\label{sauce-diable}}

\index{そーす@ソース!でぃあーぶる@---・ディアーブル} \index{でぃあーぶ
る@ディアーブル!そーす@ソース・---} \index{sauce@sauce!diable@---
Diable} \index{diable@diable!sauce@Sauce ---}

このソースはごく少量ずつ作るのが一般的だが、ここではそれを守らずに、仕
上り2\undemi{} dlとして説明する

白ワイン3dlにエシャロット3個分のみじん切りを加え、\untiers{}量以下にな
るまで煮詰める。

\protect\hyperlink{sauce-demi-glace}{ソース・ドゥミグラス}2
dlを加えて数分間煮立たせ、
仕上げにカイエンヌの粉末をたっぷり効かせる\footnote{「たっぷり」という表現に惑わされないよう注意。}。

\ldots{}\ldots{}鶏と鳩のグリルに合わせる。

\hypertarget{ux539fux6ce8}{%
\subparagraph{【原注】}\label{ux539fux6ce8}}

白ワインではなくヴィネガーを煮詰め、仕上げにハーブを加えて作る調理現場
もあるが、著者としては上記の作り方がいいと思う。

\maeaki

\hypertarget{ux30bdux30fcux30b9ux30c7ux30a3ux30a2ux30fcux30d6ux30ebux30a8ux30b9ux30b3ux30d5ux30a3ux30a8}{%
\subsubsection{ソース・ディアーブル・エスコフィエ}\label{ux30bdux30fcux30b9ux30c7ux30a3ux30a2ux30fcux30d6ux30ebux30a8ux30b9ux30b3ux30d5ux30a3ux30a8}}

\hypertarget{sauce-diable-escoffier}{%
\paragraph{Sauce Diable Escoffier}\label{sauce-diable-escoffier}}

\index{そーす@ソース!でぃあーぶるえすこふぃえ@---・ディアーブル・エス
コフィエ} \index{でぃあーぶるえすこふぃえ@ディアーブル・エスコフィエ!
そーす@ソース・---・エスコフィエ} \index{sauce@sauce!diable
escoffier@--- Diable Escoffier} \index{diable@diable!sauce
escoffier@Sauce --- Escoffier}

このソースは完成品が市販\footnote{現在は市販されていないと思われる。フランスにおいては未確認だが、
  1980年代までアメリカ合衆国ではナビスコがソース・ロベール・エスコフィ
  エとともに瓶詰めを生産、販売していた。初版ではこれら2つの製品への
  言及がなく、第二版で追加されたことから、1903年〜1907年の間に製品化
  された可能性もある。また、第二版(1907年)と同年の英訳版、第三版
  (1912年)にはソース・スリーズ・エスコフィエの記述が見られるが、こ
  れは第四版で削除されており、生産中止になったと思われる。エスコフィ
  エ・ブランドの既製品ソースはさらに他にもあったようだが詳細は不明。なお、
  エスコフィエは1922年頃、ジュリユス・マジがブイヨンキューブ(日本で
  は「マギーブイヨン」の商品名)を開発する際にも協力した。}されている。同量の柔くしたバターを混ぜ合
わせるだけでいい。

\maeaki

\hypertarget{ux30bdux30fcux30b9ux30c7ux30a3ux30a2ux30fcux30cc28}{%
\subsubsection[ソース・ディアーヌ]{\texorpdfstring{ソース・ディアーヌ\footnote{ローマ神話の女神ディアーナのこと。ギリシア神話のアルテミスに相
  当し、狩猟、貞潔の女神。また月の女神ルーナ(セレーネー)と同一視さ
  れた。ここでは大型ジビエ料理用のソースであるから、狩猟の女神という
  意味合いが強い。}}{ソース・ディアーヌ}}\label{ux30bdux30fcux30b9ux30c7ux30a3ux30a2ux30fcux30cc28}}

\hypertarget{sauce-diane}{%
\paragraph{Sauce Diane}\label{sauce-diane}}

\index{そーす@ソース!てぃあーぬ@---・ディアーヌ} \index{てぃあーぬ@ディ
アーヌ!そーす@ソース・---} \index{sauce@sauce!diane@--- Diane}
\index{diane@Diane!sauce@Sauce ---}

不純物を充分に取り除き、コクと風味ゆたかな\protect\hyperlink{sauce-poivrade}{ソース・ポワヴラー
ド}5 dlを用意する。提供直前に、泡立てた生クリーム4 dl
(生クリーム2dlを泡立てて倍量にする)と、小さな三日月の形にしたトリュ
フのスライスと固茹で卵の白身を加える。

\ldots{}\ldots{}大型ジビエの骨付き背肉および、その中心部を円筒形に切り出したもの
\footnote{noisette ノワゼット。}、フィレ料理用。

\maeaki

\hypertarget{ux30bdux30fcux30b9ux30c7ux30e5ux30afux30bbux30eb29}{%
\subsubsection[ソース・デュクセル]{\texorpdfstring{ソース・デュクセル\footnote{デュクセル・セッシュ(第2章ガルニチュール参照)を用いることか
  らこの名称が用いられている。}}{ソース・デュクセル}}\label{ux30bdux30fcux30b9ux30c7ux30e5ux30afux30bbux30eb29}}

\hypertarget{sauce-duxelles}{%
\paragraph{Sauce Duxelles}\label{sauce-duxelles}}

\index{そーす@ソース!てゅくせる@---・デュクセル} \index{てゅくせる@デュ
クセル!そーす@ソース・---} \index{sauce@sauce!duxelles@--- Duxelles}
\index{duxelles@duxelles!sauce@Sauce ---}

白ワイン2dlとマッシュルームの茹で汁2 dlにエシャロットのみじん切り大さじ2
杯を加えて、\untiers{}量まで煮詰める。\protect\hyperlink{sauce-demi-glace}{ソース・ドゥミグラ
ス}\undemi{} Lとトマトピュレ1\undemi{} dl、\protect\hyperlink{}{デュク
セル・セッシュ}大さじ4杯を加える。5分間煮立たせ、パセリのみじん切り
大さじ\undemi{}を加える。

\ldots{}\ldots{}グラタンの他、いろいろな料理に用いられる。

\hypertarget{ux539fux6ce8-1}{%
\subparagraph{【原注】}\label{ux539fux6ce8-1}}

ソース・デュクセルはイタリア風ソースと混同されることが多いが、ソース・
デュクセルにはハムも、赤く漬けた舌肉も入れないので、まったく別のものだ。

\maeaki

\hypertarget{ux30bdux30fcux30b9ux30a8ux30b9ux30c8ux30e9ux30b4ux30f399}{%
\subsubsection[ソース・エストラゴン]{\texorpdfstring{ソース・エストラゴン\footnote{ヨモギ科のハーブ。\protect\hyperlink{sauce-chasseur}{ソース・シャスール}訳注参照。}}{ソース・エストラゴン}}\label{ux30bdux30fcux30b9ux30a8ux30b9ux30c8ux30e9ux30b4ux30f399}}

\hypertarget{sauce-estragon}{%
\paragraph{Sauce Estragon}\label{sauce-estragon}}

\index{そーす@ソース!えすとらこんちゃいろ@---・エストラゴン(茶色いソー
ス)} \index{えすとらこんちゃいろ@エストラゴン!そーす@ソース・---(茶
色いソース)} \index{sauce@sauce!estragonbrune@--- Estragon (sauce
brune)} \index{estragon@estragon!sauce brune@Sauce --- (brune)}

(仕上り2\undemi{}dl分)

白ワイン2dlを沸かし、エストラゴンの枝20gを投入する。蓋をして10分間、煎
じる\footnote{infuserアンフュゼ。}。2\undemi{}dlの\protect\hyperlink{sauce-demi-glace}{ソース・ドゥミグラス}また
は、\protect\hyperlink{jus-de-veau-lie}{とろみを付けた仔牛のジュ}を加え、約\deuxtiers{}
量になるまで煮詰める。布で漉し、みじん切りにしたエストラゴン小さじ1杯
を加えて仕上げる。

\ldots{}\ldots{}仔牛や仔羊の背肉の中心を円筒形に切り出した料理や家禽料理用。

\maeaki

\hypertarget{ux30bdux30fcux30b9ux30d5ux30a3ux30caux30f3ux30b7ux30a8ux30fcux30eb34}{%
\subsubsection[ソース・フィナンシエール]{\texorpdfstring{ソース・フィナンシエール\footnote{Financier徴税官(財務官)風の意。フランス革命以前の徴税官は、王
  に代わって徴税を行なう大貴族が就く役職であり、膨大な利権によりきわめて
  裕福であったという。このソースと組み合わせる\protect\hyperlink{}{ガルニチュール・フィナン
  シエール}が、雄鶏のとさかと睾丸、仔羊の胸腺肉、トリュフなどの比較的
  入手困難あるいは高級とされる食材で構成されていることが名称の由来と思わ
  れる。ブリヤ=サヴァランは『美味礼讃』(味覚の生理学)において、徴税官
  たちは旬のはしりの食材を真っ先に食べられる、いわば特権階級だと述べてい
  る。なお、カレーム『19世紀フランス料理』においては、ソースとガルニチュー
  ルを分離せず、「ラグー・アラ・フィナンシエール」として採りあげられてい
  るが、全ての素材を別々に加熱調理してソースと合わせるものであり、いわゆ
  る「煮込み」とは呼びがたいものとなっている。フランス料理の影響が比較的
  強かった北イタリアにこの原型に近いと思われるラグー「ピエモンテ風フィナ
  ンツィエラ」がある。鶏のとさか、肉垂、睾丸、鶏レバーおよび仔牛の胸腺肉
  などを煮込んだものだが、レシピを読む限りにおいては比較的庶民的あるいは
  農民的料理に変化したものと思われる (cf.~Anna Gosetti della Salda,
  \emph{Le Ricette Regionali Italiane}, Milano, Solares, 1967,
  p.57.)。ちなみに焼
  き菓子のフィナンシエfinancierも同語源だが、何故その名称になったかは不
  明。}}{ソース・フィナンシエール}}\label{ux30bdux30fcux30b9ux30d5ux30a3ux30caux30f3ux30b7ux30a8ux30fcux30eb34}}

\hypertarget{sauce-financiere}{%
\paragraph{Sauce Financière}\label{sauce-financiere}}

\index{そーす@ソース!ふぃなんしえーる@---・フィナンシエール} \index{ふぃ
なんしえーる@フィナンシエール!そーす@ソース・---} \index{ちょうせいか
んふう@徴税官風!そーすふぃなんしえーる@ソース・---}
\index{sauce@sauce!financiere@--- Financière}
\index{financier@financier!sauce@Sauce Financière}

\protect\hyperlink{sauce-madere}{ソース・マデール}1\unquart{}Lを\troisquarts{}量以下に
なるまで煮詰め、火から外してトリュフエッセンス1 dlを加える。布で漉して
仕上げる。

\ldots{}\ldots{}\protect\hyperlink{}{ガルニチュール・フィナンシエール}用だが、その他の肉料理にも用い
られる。

\maeaki

\hypertarget{ux9999ux8349ux30bdux30fcux30b9}{%
\subsubsection{香草ソース}\label{ux9999ux8349ux30bdux30fcux30b9}}

\hypertarget{sauce-aux-fines-herbes}{%
\paragraph[Sauce aux Fines Herbes]{\texorpdfstring{Sauce aux Fines
Herbes\footnote{料理名としていわゆる「ハーブ」についてかつてはfines
  herbesの表
  現が多かった。とはいえ、こんにちでは特定のハーブ名をソースや料理名
  に添えて言うことが多い。例えばCôtelette de veau au thymコトレット
  ドヴォオタン(仔牛の骨付き背肉、タイム風味)、やFilet de bar poêlé,
  compote de tomate au basilicフィレドバールポワレ コンポットートド
  トマトバジリック(スズキのフィレとトマトのコンポート、バジル風味)
  など。また、栽培レベルで「香草、ハーブ」の総称としては herbes
  aromatiques エルブアロマティック、あるいはたんにaromatiquesアロマ
  ティックが一般的。}}{Sauce aux Fines Herbes}}\label{sauce-aux-fines-herbes}}

\index{そーす@ソース!こうそう@香草---} \index{こうそう@香草!そーす@---
ソース} \index{はーぶ@ハーブ!こうそうそーす@香草ソース}
\index{sauce@sauce!fines herbes@--- aux Fines Herbes} \index{fines
herbes@fines herbes!sauce@Sauce aux ---}

白ワイン3dlを沸かし、パセリの葉、セルフイユ、エストラゴン、シブレット
を各1つまみ強、投入する。約20分間煎じる。布で漉し、\protect\hyperlink{sauce-demi-glace}{ソース・ドゥミグラ
ス}または\protect\hyperlink{jus-de-veau-lie}{とろみを付けた仔牛の ジュ}6
dlを加える。仕上げに、煎じるのに使ったのと同
じ香草を細かく刻んだもの計、大さじ2\undemi{}杯とレモンの搾り汁少々を加
える。

\hypertarget{ux539fux6ce8-2}{%
\subparagraph{【原注】}\label{ux539fux6ce8-2}}

古典料理ではこの「香草ソース」と\protect\hyperlink{sauce-duxelles}{ソース・デュクセル}
が混同されることもあったが、こんにちではまったく違うものとして扱われて
いる。

\maeaki

\hypertarget{ux30b8ux30e5ux30cdux30fcux30f4ux98a8ux30bdux30fcux30b9}{%
\subsubsection{ジュネーヴ風ソース}\label{ux30b8ux30e5ux30cdux30fcux30f4ux98a8ux30bdux30fcux30b9}}

\hypertarget{sauce-genevoise}{%
\paragraph{Sauce Genevoise}\label{sauce-genevoise}}

\index{そーす@ソース!じゅねーうふう@ジュネーヴ風---} \index{じゅねーう
ふう@ジュネーヴ風!そーす@---ソース} \index{sauce@sauce!genevoise@---
Genevoise} \index{genevois@genevois!sauce@Sauce Genevoise}

鍋にバターを熱し、細かく刻んだミルポワを色付かないよう強火でさっと炒め
る。ミルポワの材料は、にんじん100 g、玉ねぎ80 g、タイムとローリエ少々、
パセリの枝20 g。そこにサーモンの頭1kgと粗く砕いたこしょう1つまみを入れ、
蓋をして弱火で15分程蒸し煮する。

鍋に残ったバターを捨て、赤ワイン1Lを注ぐ。半量になるまで煮詰める。そこ
に\protect\hyperlink{sauce-espagnole-maigre}{魚料理用ソース・エスパニョル}\undemi{}
Lを
加える。弱火で1時間煮込む。漉し器を使い、材料を押しつけながら漉す。し
ばらく休ませてから、表面に浮いた油脂を取り除く\footnote{dégraisser
  デグレセ。レードルなどを用いて浮いてきた余計な油脂を取り除く作業。}

さらに赤ワイン\undemi{} Lと、魚のフュメ\undemi{} Lを加える。ソースの表
面に浮いてくる不純物を徹底的に取り除き\footnote{dépouiller デプイエ ≒
  écumer エキュメ。}、丁度いい濃さになるまで煮 詰める。

これを布で漉し、静かに混ぜながら、アンチョヴィのエッセンス大さじ1杯と
バター150 gを加えて仕上げる。

\ldots{}\ldots{}サーモン、鱒料理用。

\hypertarget{ux539fux6ce8-3}{%
\subparagraph{【原注】}\label{ux539fux6ce8-3}}

このソースはもともとカレームが「ジェノヴァ風」\footnote{Sauce à la
  génoise au vin de Bordeaux ボルドー産ワインを用いた
  ジェノヴァ風ソース(『19世紀フランス料理』第3巻、80頁)。本書のこのレシ
  ピと同様に魚料理用ソースだ。ボルドーの赤ワインにみじん切りにして下茹で
  したマッシュルーム、トリュフ、エシャロットを加えてオールスパイスとこしょ
  う少々を入れ、適度に煮詰める。ソース・エスパニョルと赤ワインを加え、湯
  煎にかけておく。提供直前にバター少量を加えて仕上げる、というもの。本書
  においてこのソースを「原型」とするのには疑問が残るところだろう。}と名付けたものだが、その
後ルキュレ、グフェ\footnote{グフェ『料理の本』(1867年)の420ページにあるジュネーヴ風ソー
  スは、薄切りにした玉ねぎ、エシャロット、粗挽きこしょう、にんにく、
  バターを鍋に入れて色付くまで炒め、そこにブルゴーニュ産赤ワインを注
  ぐ。弱火で玉ねぎに火が通るまで煮る。ソース・エスパニョルと仔牛のブ
  ロンドのジュを加えて煮詰め、布で漉す。提供直前にマデラ酒の風味を加
  えて茹でたトリュフのみじん切りとアンチョビバターを加える、というも
  の。赤ワインと玉ねぎ、仕上げにアンチョビを加える点は共通しているが、
  グフェのが肉料理用であるのに対して、本書のこのソースは明らかに魚料
  理用であり、まったく同じソースと呼べるかは疑問の残るところだろう。}が立て続けに「ジュネーヴ風」の名称を用いた。だが、ジュ
ネーヴは赤ワインの産地ではないから理屈としてはおかしい\footnote{料理名に冠された地名は、由来が明確にあるものがある一方で、まっ
  たく意味不明か、あるいはいい加減な思い付きで付けられたのではないか
  とさえ思われるものも少なくない。(à la) russe「ロシア風」や (à la)
  moscovite「モスクワ風」などはロシア料理起源か、あるいは18世紀末〜
  19世紀前半にかけてロシア帝国の宮廷や貴族がこぞってフランスから料理
  人を招聘し、帰国した彼らが創案した料理などはある程度しっかりとした
  由来がわかるものも多い。一方で、(à l')espagnole「スペイン風」(à
  l')italienne「イタリア風」(à la) romaine「ローマ風」(à la grecque)
  「ギリシア風」(à l')allemande「ドイツ風」(à l')hollandaise「オラン
  ダ風」などは由来の不明なケースが非常に多い。\protect\hyperlink{sauce-espagnole}{ソース・エスパニョ
  ル}などはその典型例とも言うべきものだろう。\\
  この原注では由来に非常にこだわっているが、そもそもカレームのレシピ
  は上述のように「ボルドー産ワインを用いたジェノバ風ソース」であるか
  ら、赤ワインの産地かどうかということは実はさしたる問題にはならない。
  重要なのは後半の、赤ワインを用いることがこのソースのポイントという
  こと。}。

間違っているとはいえ、ジュネーヴ風という名称で定着してしまっているので、
本書でもそのままにしている。だが、ジュネーヴ風であれジェノヴァ風であれ、
カレーム、ルキュレ、デュボワ、グフェはいずれもこのソースに赤ワインを用
いるよう指示している。つまり赤ワインを用いることがこのソースのポイント。

\maeaki

\hypertarget{ux30bdux30fcux30b9ux30b4ux30c0ux30fcux30eb37}{%
\subsubsection[ソース・ゴダール]{\texorpdfstring{ソース・ゴダール\footnote{ガルニチュール・ゴダールの構成要素がガルニチュール・フィナンシ
  エールとよく似ている点などから、おそらくは18世紀の徴税官(つまりフィ
  ナンシエ)であり作家としても活動したクロード・ゴダール・ドクール
  Claude Godard d'Aucour(1716〜1795)の名を冠したものと考えられる。}}{ソース・ゴダール}}\label{ux30bdux30fcux30b9ux30b4ux30c0ux30fcux30eb37}}

\hypertarget{sauce-godard}{%
\paragraph[Sauce Godard]{\texorpdfstring{Sauce Godard\footnote{底本とした現行版(第四版)では最後がdではなくtとなっているが、
  初版から第三版にいたるまでdとなっており、現行版は明らかな誤植。}}{Sauce Godard}}\label{sauce-godard}}

\index{そーす@ソース!ごだーる@---・ゴダール} \index{ごだーる@ゴダール!
そーす@ソース・---} \index{sauce@sauce!godart@--- Godart}
\index{godard@Godard!sauce@Sauce ---}

シャンパーニュまたは辛口の白ワイン4 dlにハム入りの細かく刻んだ{[}ミルポ
ワ{]}。{[}ソース・ドゥミグラス{]}1
Lとマッシュルームのエッセンス2dlを加える。
弱火に10分かけ、シノワ\footnote{\protect\hyperlink{sauce-espagnole}{ソース・エスパニョル}訳注参照。}で漉す。

\deuxtiers{}量になるまで煮詰め、布で漉す。

\ldots{}\ldots{}\protect\hyperlink{}{ガルニチュール ゴタール}用。

\maeaki

\hypertarget{ux30bdux30fcux30b9ux30b0ux30e9ux30f3ux30f4ux30ccux30fcux30eb40}{%
\subsubsection[ソース・グランヴヌール]{\texorpdfstring{ソース・グランヴヌール\footnote{王家や貴族に仕える狩猟長のことをグランヴヌールと呼ぶ。}}{ソース・グランヴヌール}}\label{ux30bdux30fcux30b9ux30b0ux30e9ux30f3ux30f4ux30ccux30fcux30eb40}}

\hypertarget{sauce-grand-veneur}{%
\paragraph{Sauce Grand-Veneur}\label{sauce-grand-veneur}}

\index{そーす@ソース!くらんうぬーる@---・グランヴヌール} \index{くらん
うぬーる@グランヴヌール!そーす@ソース・---}
\index{sauce@sauce!grand-veneur@--- Grand-Veneur}
\index{grand-veneur@grand-veneur!sauce@Sauce ---}

\protect\hyperlink{fonds-de-gibier}{大型ジビエのフュメ}で澄んだ色合いに作った\protect\hyperlink{sauce-poivrade}{ソース・
ポワヴラード}に、ソース1Lあたり野うさぎの血1dlをマリ
ネ液1dlで薄めたものを加える。

火をごく弱くして、血が沸騰しないよう気をつけながら数分間煮る。布で漉す。

\maeaki

\hypertarget{ux30bdux30fcux30b9ux30b0ux30e9ux30f3ux30f4ux30ccux30fcux30ebux30a8ux30b9ux30b3ux30d5ux30a3ux30a8ux6d41}{%
\subsubsection{ソース・グランヴヌール(エスコフィエ流)}\label{ux30bdux30fcux30b9ux30b0ux30e9ux30f3ux30f4ux30ccux30fcux30ebux30a8ux30b9ux30b3ux30d5ux30a3ux30a8ux6d41}}

\hypertarget{sauce-grand-veneur-procede-escoffier}{%
\paragraph{Sauce Grand-Veneur (Procédé
Escoffier)}\label{sauce-grand-veneur-procede-escoffier}}

\index{そーす@ソース!くらんうぬーるえすこふぃえ@---・グランヴヌール(エ
スコフィエ)} \index{くらんうぬーるえすこふぃえ@グランヴヌール(エスコフィ
エ)!そーす@ソース・---} \index{sauce@sauce!grand-veneur escoffier@---
Grand-Veneur (Procédé Escoffier)}
\index{grand-veneur@grand-veneur!sauce escoffier@Sauce --- (Procédé
Escoffier)}

軽く仕上げた\protect\hyperlink{sauce-poivrade}{ソース・ポワヴラード}1
Lあたり{[}グロゼイ
ユのジュレ{]}大さじ2杯と生クリーム2\undemi{}dlを加える。

\ldots{}\ldots{}上記2つのソースは鹿、猪などの大きな塊肉の料理に用いる。

\maeaki

\hypertarget{ux30bdux30fcux30b9ux30b0ux30e9ux30bfux30f345}{%
\subsubsection[ソース・グラタン]{\texorpdfstring{ソース・グラタン\footnote{魚のグラタン用ソースだが、グラタンの技術的ポイントについては第
  7章「肉料理」参照。}}{ソース・グラタン}}\label{ux30bdux30fcux30b9ux30b0ux30e9ux30bfux30f345}}

\hypertarget{sauce-gratin}{%
\paragraph{Sauce Gratin}\label{sauce-gratin}}

\index{そーす@ソース!くらたん@---・グラタン} \index{くらたん@グラタン!
そーす@ソース・---} \index{sauce@sauce!gratin@--- Gratin}
\index{gratin@gratin!sauce@Sauce ---}

白ワインと、このソースを合わせる魚のアラなどでとった\protect\hyperlink{fumet-de-poisson}{魚のフュ
メ}各3 dlにエシャロットのみじん切り大さじ1\undemi{}
杯を加え、半量以下になるまで煮詰める。

\protect\hyperlink{}{デュクセル・セッシュ}大さじ3杯と、\protect\hyperlink{sauce-espagnole-maigre}{魚料理用ソース・エスパニョ
ル}または\protect\hyperlink{sauce-demi-glace}{ソース・ドゥミグラ ス}5
dlを加える。5〜6分間煮立たせる。提供直前に、パ
セリのみじん切り大さじ\undemi{}を加えて仕上げる。

\ldots{}\ldots{}舌びらめ、メルラン\footnote{タラの近縁種。}、バルビュ\footnote{鰈の近縁種。この場合のフィレはいわゆる「五枚おろし」にしたもの。}のフィレなどのグラタン用。

\maeaki

\hypertarget{ux30bdux30fcux30b9ux30a2ux30b7ux30a743}{%
\subsubsection[ソース・アシェ]{\texorpdfstring{ソース・アシェ\footnote{細かく刻んだもの、の意。}}{ソース・アシェ}}\label{ux30bdux30fcux30b9ux30a2ux30b7ux30a743}}

\hypertarget{sauce-hachee}{%
\paragraph{Sauce Hachée}\label{sauce-hachee}}

\index{そーす@ソース!あしぇ@---・アシェ} \index{sauce@sauce!hachee@---
Hach\'ee}

玉ねぎの細かいみじん切り100gと、エシャロットの細かいみじん切り大さじ
1\undemi{}杯をバターで色付かないよう炒める。ヴィネガー3 dlを注ぎ、半量
まで煮詰める。{[}ソース・エスパニョル{]}4
dlと{[}トマトソース{]}1\undemi{} dl を加える。5〜6分煮立たせる。

ハムの脂身のない部分を細かく刻んだもの大さじ1\undemi{}杯と小ぶりのケイ
パー大さじ1\undemi{}杯、{[}デュクセル・セッシュ{]}大さじ1\undemi{}杯、パセ
リのみじん切り大さじ\undemi{}杯を加えて仕上げる

\ldots{}\ldots{}このソースは\protect\hyperlink{ux30bdux30fcux30b9ux30d4ux30abux30f3ux30c8}{ソース・ピカント}と等価のものと考えていい。用途も同じ。

\maeaki

\hypertarget{ux9b5aux6599ux7406ux7528ux30bdux30fcux30b9ux30a2ux30b7ux30a7}{%
\subsubsection{魚料理用ソース・アシェ}\label{ux9b5aux6599ux7406ux7528ux30bdux30fcux30b9ux30a2ux30b7ux30a7}}

\hypertarget{sauce-hachee-maigre}{%
\paragraph{Sauce Hachée maigre}\label{sauce-hachee-maigre}}

上記と同様に、玉ねぎとエシャロットを色付かないようバターで炒め、ヴィネ
ガーを注いで煮詰める。

魚の\protect\hyperlink{}{クールブイヨン}5
dlを注ぎ、\protect\hyperlink{roux-brun}{茶色いルー}45 gまたはブー
ルマニエ50 gでとろみを付ける。弱火で8〜10分間煮込む。

提供直前に、細かく刻んだハーブミックス大さじ1杯と{[}デュクセル・セッシュ{]}大
さじ1\undemi{}杯、小粒のケイパー大さじ1\undemi{}杯、アンチョヴィソース
大さじ\undemi{}杯とバター60 g、または80〜100 gのアンチョヴィバターを加
えて仕上げる。

\ldots{}\ldots{}エイのような、あまり高級ではない魚のブイイ\footnote{茹でた肉、魚のこと。}用。

\maeaki

\hypertarget{ux30bdux30fcux30b9ux30e6ux30b5ux30ebux30c951}{%
\subsubsection[ソース・ユサルド]{\texorpdfstring{ソース・ユサルド\footnote{もとはハンガリーで農家20戸につき1人の割合で招集された騎兵
  hussard を指す。この語は16世紀まで遡ることが出来るが、のちに「乱暴
  者」といったニュアンスでも使われるようになった。à la hussarde は
  「乱暴に、粗野に」の意味でも用いられるが、料理においてはレフォール
  を使ったものに名付けられることが多い。}}{ソース・ユサルド}}\label{ux30bdux30fcux30b9ux30e6ux30b5ux30ebux30c951}}

\hypertarget{sauce-hussarde}{%
\paragraph{Sauce Hussarde}\label{sauce-hussarde}}

\index{そーす@ソース!ゆさるど@---・ユサルド} \index{ゆさるど@ユサルド!
そーす@ソース・---} \index{sauce@sauce!hussarde@--- Hussarde}
\index{hussarde@Hussarde!sauce@Sauce ---}

玉ねぎ2個とエシャロット2個を細かくみじん切りにして、バターで色よく炒め
る。白ワイン4
dlを注ぎ、半量になるまで煮詰める。\protect\hyperlink{sauce-demi-glace}{ソース・ドゥミグラ
ス}4
dlとトマトピュレ大さじ2杯、\protect\hyperlink{fonds-blanc-ordinaire}{白いフォ
ン}2 dl、生ハムの脂身のないところ80 g、潰した
にんにく1片、ブーケガルニを加える。弱火で25〜30分煮込む。

ハムを取り出して、ソースをスプーンで押すようにして布で漉す。

火にかけて温め、小さなさいの目\footnote{brunoise ブリュノワーズ。}に刻んだハムと、おろしたレフォール
\footnote{raifort いわゆる西洋わさび。}少々、パセリのみじん切りをたっぷり1つまみ加えて仕上げる。

\ldots{}\ldots{}牛、羊肉のグリルまたは串を刺してローストしてアントレ\footnote{通常、ローストは料理区分としてアントレに含められることはないが、
  牛フィレは牛の部位のなかでも比較的小さいものとして、まるごと1本の
  ローストであっても原則的にはアントレに分類される。このソースを用い
  る「牛フィレ ユサルド」は牛フィレの塊に串を刺してローストし、ポム・
  デュシェスとマッシュルームを合わせる。}として供 する際に用いる。

\maeaki

\hypertarget{ux30a4ux30bfux30eaux30a2ux98a8ux30bdux30fcux30b9}{%
\subsubsection{イタリア風ソース}\label{ux30a4ux30bfux30eaux30a2ux98a8ux30bdux30fcux30b9}}

\hypertarget{sauce-italienne}{%
\paragraph{Sauce Italienne}\label{sauce-italienne}}

\index{そーす@ソース!いたりあふう@イタリア風---} \index{いたりあん@イ
タリアン!いたりあふうそーす@イタリア風ソース} \index{いたりあふう@イタ
リア風!そーす@---ソース} \index{sauce@sauce!Italienne@--- Italienne}
\index{italien@italien!sauce italienne@Sauce Italienne}

トマトの風味の効いた\protect\hyperlink{sauce-demi-glace}{ソース・ドゥミグラ
ス}\troisquarts{} Lに、\protect\hyperlink{}{デュクセル・セッシュ}大さ
じ4杯と、加熱ハムの脂身のないところを小さなさいの目に切ったもの125 gを
加える。5〜6分間煮る。提供直前に、パセリとセルフイユ、エスゴラゴンのみ
じん切り大さじ1杯を加えて仕上げる。

\ldots{}\ldots{}いろいろな肉料理に合わせる。

\hypertarget{ux539fux6ce8-4}{%
\subparagraph{【原注】}\label{ux539fux6ce8-4}}

このソースを魚料理に合わせる場合、ハムは使わずに\protect\hyperlink{fumet-de-poisson}{魚のフュ
メ}を煮詰めて加える。

\maeaki

\hypertarget{ux3068ux308dux307fux3092ux4ed8ux3051ux305fux30b8ux30e5ux30a8ux30b9ux30c8ux30e9ux30b4ux30f3ux98a8ux5473}{%
\subsubsection{とろみを付けたジュ エストラゴン風味}\label{ux3068ux308dux307fux3092ux4ed8ux3051ux305fux30b8ux30e5ux30a8ux30b9ux30c8ux30e9ux30b4ux30f3ux98a8ux5473}}

\hypertarget{jus-lie-a-lestragon}{%
\paragraph{Jus lié à l'Estragon}\label{jus-lie-a-lestragon}}

\index{そーす@ソース!とろみをつけたしゅえすとらごん@とろみを付けたジュ
エストラゴン風味} \index{しゅ@ジュ!えすとらごん@とろみを付けた--- エス
トラゴン風味} \index{sauce@sauce!jus lie a l'estragon@Jus lié à
l'Estragon} \index{estragon@estragon!jus lie a l'estragon@Jus lié à
l'Estragon} \index{jus@jus!estragon@--- lié à l'Estragon}

\protect\hyperlink{fonds-de-veau-brun}{仔牛のフォン}または\protect\hyperlink{fonds-de-volaille}{鶏のフォ
ン}に、エストラゴン50gを加えて香りを煮出し\footnote{imfuser アンフュゼ。}た
もの。

布で漉してから、アロールート\footnote{コーンスターチで代用する。}または、でんぷん30
gでとろみを付ける。

\ldots{}\ldots{}白身肉のノワゼットや家禽のフィレなどに添える。

\maeaki\columnbreak

\hypertarget{ux3068ux308dux307fux3092ux4ed8ux3051ux305fux30b8ux30e5ux30c8ux30deux30c8ux98a8ux5473}{%
\subsubsection{とろみを付けたジュ トマト風味}\label{ux3068ux308dux307fux3092ux4ed8ux3051ux305fux30b8ux30e5ux30c8ux30deux30c8ux98a8ux5473}}

\hypertarget{jus-lie-tomate}{%
\paragraph{Jus lié tomaté}\label{jus-lie-tomate}}

\index{そーす@ソース!じゅとまといり@とろみを付けたジュ トマト入り}
\index{じゅ@ジュ!とまといり@とろみを付けた--- トマト入り}
\index{sauce@sauce!jus lie tomateq@Jus lié tomaté}
\index{tomate@tomate!jus lie tomate@Jus lié tomaté}
\index{jus@jus!tomate@--- lié tomaté}

\protect\hyperlink{fonds-de-veau-brun}{仔牛のフォン}1
Lあたりトマトエッセンス3 dlを加え、 \quatrecinquiemes{}量まで煮詰める。

\ldots{}\ldots{}牛、羊肉料理用。

\maeaki

\hypertarget{ux30eaux30e8ux30f3ux98a8ux30bdux30fcux30b9}{%
\subsubsection{リヨン風ソース}\label{ux30eaux30e8ux30f3ux98a8ux30bdux30fcux30b9}}

\hypertarget{sauce-lyonnaise}{%
\paragraph{Sauce Lyonnaise}\label{sauce-lyonnaise}}

\index{そーす@ソース!りよんふう@リヨン風---} \index{りよんふう@リヨン
風!りよんふうそーす@---ソース} \index{sauce@sauce!lyonnaise@---
Lyonnaise} \index{liyonnais@lyonnais!sauce lyonnaise@Sauce Lyonnaise}

中位の大きさの玉ねぎ3個をみじん切りにし、バターでじっくり、ごく弱火で
ブロンド色になるまで炒める。白ワイン2 dlとヴィネガー2 dlを注ぐ。
\untiers{}量まで煮詰め、\protect\hyperlink{sauce-demi-glace}{ソース・ドゥミグラ
ス}\troisquarts{} Lを加える。5〜6分かけて表面に浮い
てくる不純物を丁寧に取り除き\footnote{dépouiller
  デプイエ。現代ではエキュメと呼ぶ現場が多い。}、布で漉す。

\hypertarget{ux539fux6ce8-5}{%
\subparagraph{【原注】}\label{ux539fux6ce8-5}}

このソースを合わせる料理によっては、ソースを布で漉さずに玉ねぎを残して
もいい。

\maeaki

\hypertarget{ux30bdux30fcux30b9ux30deux30c7ux30fcux30eb}{%
\subsubsection{ソース・マデール}\label{ux30bdux30fcux30b9ux30deux30c7ux30fcux30eb}}

\hypertarget{sauce-madere}{%
\paragraph{Sauce Madère}\label{sauce-madere}}

\index{そーす@ソース!までーる@---・マデール} \index{までいら@マデイラ!
そーすまでーる@ソース・マデール} \index{sauce@sauce!madere@--- Madère}
\index{madere@madère!sauce madere@Sauce Madère}

\protect\hyperlink{sauce-demi-glace}{ソース・ドゥミグラス}を煮詰め\footnote{ソース・ドゥミグラスは既に煮詰めて仕上がった状態のものなので、9
  割程度にまでしか煮詰めないことに注意。}、火から外して、 ソース1
Lあたりマデラ酒1 dlの割合で加え、普通の濃度にする。

\maeaki

\hypertarget{ux30bdux30fcux30b9ux30deux30c8ux30edux30c3ux30c854}{%
\subsubsection[ソース・マトロット]{\texorpdfstring{ソース・マトロット\footnote{水夫風、船員風、の意。トゥーレーヌ地方の郷土料理Matelote
  d'anguilleマトロットダンギーユ(うなぎの赤ワイン煮込み)が有名だが、
  赤ワイン煮込みにとどまらず、マトロットの名称を持つ料理は他にも複数
  存在する。}}{ソース・マトロット}}\label{ux30bdux30fcux30b9ux30deux30c8ux30edux30c3ux30c854}}

\hypertarget{sauce-matelote}{%
\paragraph{Sauce Matelote}\label{sauce-matelote}}

\index{そーす@ソース!まとろつと@---・マトロット} \index{まとろつと@マ
トロット!そーすまとろつと@ソース・---} \index{sauce@sauce!matelote@---
Matelote} \index{matelote@matelote!sauce matelote@Sauce Matelote}

魚をポシェするのに使った\protect\hyperlink{}{赤ワイン入りの魚用クールブイヨン}3
dlにマッ シュルームの切りくず25
gを加え、\untiers{}量になるまで煮詰める。

煮詰めたら\protect\hyperlink{sauce-espagnole-maigre}{魚料理用ソース・エスパニョル}8
dl を加えてひと煮立ちさせる。布で漉し、バター150 gとごく少量のカイエンヌ
の粉末を加えて仕上げる。

\maeaki

\hypertarget{ux30bdux30fcux30b9ux30e2ux30efux30eb}{%
\subsubsection{ソース・モワル}\label{ux30bdux30fcux30b9ux30e2ux30efux30eb}}

\hypertarget{sauce-moelle}{%
\paragraph[Sauce Moelle]{\texorpdfstring{Sauce Moelle\footnote{骨髄のこと。}}{Sauce Moelle}}\label{sauce-moelle}}

\index{そーす@ソース!もわる@---・モワル} \index{こつずい@骨髄!そーすも
わる@ソース・モワル} \index{sauce@sauce!moelle@--- Moelle}
\index{moelle@moelle!sauce moelle@Sauce ---}

ソースの作り方は\protect\hyperlink{sauce-bordelaise}{ボルドー風ソース}とまったく同じだ
が、バターを加えるのは何らかの野菜料理に添える場合のみであり、その場合
のバターの量は通常どおりとするこ。

どんな場合にせよ、仕上げに、小さなさいの目に切ってポシェしておいた骨髄
をソース1 Lあたり150〜180 gおよび刻んで下茹でしたパセリの葉小さじ1杯を
加える。

\maeaki

\hypertarget{ux30e2ux30b9ux30afux30efux98a856ux30bdux30fcux30b9}{%
\subsubsection[モスクワ風ソース]{\texorpdfstring{モスクワ風\footnote{モスクワ風の名称を持つ料理や菓子は多い。
  18世紀後半から19世紀前
  半にかけて、ロシアの宮廷や貴族らの間でフランスの食文化が流行し、多
  くのフランス人料理人が招聘され、彼らはロシア料理のレシピをフランス
  に持ち帰った。クーリビヤックなどが代表的な例だろう。また、19世紀後
  半になると、とりわけフランス料理においてもロシア料理からの影響が多
  く見られるようになる。キャビアとウォトカを食前に愉しむのが流行した
  のもその時代からである。フランスとロシアの食文化は相互に影響関係に
  あったと言えよう。}ソース}{モスクワ風ソース}}\label{ux30e2ux30b9ux30afux30efux98a856ux30bdux30fcux30b9}}

\hypertarget{sauce-moscovite}{%
\paragraph{Sauce Moscovite}\label{sauce-moscovite}}

\index{そーす@ソース!もすくわふう@モスクワ風---} \index{もすくわふう@
モスクワ風!そーす@---ソース} \index{sauce@sauce!moscovite@---
Moscovite} \index{moscovite@moscovite!sauce moscovite@Sauce ---}

\protect\hyperlink{fonds-de-gibier}{大型ジビエのフュメ}で作った\protect\hyperlink{sauce-poivrade}{ソース・ポワヴラー
ド}を\troisquarts{} L用意する。提供直前にマラガ酒1 dl
とジェニパーベリーを煎じた汁7 cl、焼いた松の実かスライスして焼いたアー
モンド40 g、大きさを揃えてぬるま湯でもどしておいたコリント産干しぶどう
\footnote{小粒で黒いギリシア産干しぶどう。}40 gを加えて仕上げる。

\ldots{}\ldots{}大型ジビエ\footnote{venaison
  ヴネゾン。ジビエのうちとりわけ大型のものを指す。実際は
  ノロ鹿や猪を指すことがほとんど。}の塊肉の料理用。

\maeaki

\hypertarget{ux30bdux30fcux30b9ux30daux30eaux30b0ux30fc59}{%
\subsubsection[ソース・ペリグー]{\texorpdfstring{ソース・ペリグー\footnote{トリュフの産地として有名なペリゴール地方の町の名。}}{ソース・ペリグー}}\label{ux30bdux30fcux30b9ux30daux30eaux30b0ux30fc59}}

\hypertarget{sauce-perigueux}{%
\paragraph{Sauce Périgueux}\label{sauce-perigueux}}

\index{そーす@ソース!へりくー@---・ペリグー} \index{へりくー@ペリグー!
そーす@ソース・---} \index{sauce@sauce!perigueux@--- Péerigueux}
\index{perigueux@Périgueux!sauce perigueux@Sauce ---}

やや濃いめに煮詰めた\protect\hyperlink{sauce-demi-glace}{ソース・ドゥミグラ
ス}\troisquarts{} Lに、トリュフエッセンス1 \undemi{}
dlと細かく刻んだトリュフ100 gを加える。

\ldots{}\ldots{}いろいろな肉料理、\protect\hyperlink{}{タンバル}、\protect\hyperlink{}{温製パテ}に合わせる。

\maeaki

\hypertarget{ux30bdux30fcux30b9ux30daux30eaux30b0ux30ebux30c7ux30a3ux30fcux30cc60}{%
\subsubsection[ソース・ペリグルディーヌ]{\texorpdfstring{ソース・ペリグルディーヌ\footnote{ペリゴール地方風の意。}}{ソース・ペリグルディーヌ}}\label{ux30bdux30fcux30b9ux30daux30eaux30b0ux30ebux30c7ux30a3ux30fcux30cc60}}

\hypertarget{sauce-puxe9rigourdine}{%
\paragraph{Sauce Périgourdine}\label{sauce-puxe9rigourdine}}

\index{そーす@ソース!へりくるていーぬ@---・ペリグゥルディーヌ}
\index{へりこーるふう@ペリゴール風!そーす@ソース・ペリグルディーヌ}
\index{sauce@sauce!perigourdine@--- Périgourdine}
\index{perigourdin@périgourdin!sauce perigueux@Sauce Périgourdine}

ソース・ペリグーのバリエーション。トリュフを細かく刻むのではなく、オリー
ブ形か小さな真珠のような形状にナイフで成形\footnote{tourner
  トゥルネ。包丁を持っている側の手は動かさずに材料を回す
  ようにして形を整えること。}したものを加える。トリュ
フを厚めにスライスして加える場合もある。

\maeaki

\hypertarget{ux30bdux30fcux30b9ux30d4ux30abux30f3ux30c8}{%
\subsubsection{ソース・ピカント}\label{ux30bdux30fcux30b9ux30d4ux30abux30f3ux30c8}}

\hypertarget{sauce-piquante}{%
\paragraph[Sauce Piquante]{\texorpdfstring{Sauce Piquante\footnote{piquant
  一般的には唐辛子などが「辛い」の意だが、このソースでは
  唐辛子の類は使われておらず、むしろ酸味の効いたソースと言えよう。}}{Sauce Piquante}}\label{sauce-piquante}}

\index{そーす@ソース!ぴかんと@---・ピカント}
\index{sauce@sauce!piquante@--- Piquante}

白ワイン3 dlと良質のヴィネガー3 dlにエシャロットのみじん切り大さじ2
\undemi{}杯を合わせて半量に煮詰める。

\protect\hyperlink{sauce-espagnole}{ソース・エスパニョル}6
dlを加え、浮いてくる不純物を 取り除きながら\footnote{dépouiller
  デプイエ。エキュメécumerと呼ぶ現場も多い。}10分間煮る。

火から外し、コルニション\footnote{専用品種のきゅうりを小さなうちに収穫して酢漬けにしたもの。同様
  のピクルス用きゅうりとしてガーキンスという品種系統があるがもっぱら
  アメリカのハンバーガーに挟まれるようなサイズで収穫して漬けたもので
  あり、フランス料理では用いない。}、パセリ、セルフイユ、エストラゴンを細か
く刻んだもの大さじ2杯を加えて仕上げる。

\ldots{}\ldots{}豚肉のグリル焼き、ブイイ\footnote{bouilli 茹で肉。}、ローストによく合わせるソース。牛肉
のブイイや牛や羊の\protect\hyperlink{}{エマンセ}にも合わせることが出来る。

\maeaki

\hypertarget{ux30bdux30fcux30b9ux30ddux30efux30f4ux30e9ux30fcux30c9105-ux6a19ux6e96}{%
\subsubsection[ソース・ポワヴラード
(標準)]{\texorpdfstring{ソース・ポワヴラード\footnote{このソースは遅くとも16世紀まで遡ることが出来る。1505年に出版さ
  れた\href{http://gallica.bnf.fr/ark:/12148/bpt6k792720}{『フランス語版プラティ
  ナ』}がpoivradeとい
  うフランス語の初出。この本において「ジビエ用こしょうのソース、ポワ
  ヴラード」Saulce de poyvre ou poyvrade pour saulvagieとしてレシピ
  が見られる。パンをよく焼いてヴィネガーに浸してすり潰す。水でもどし
  た干しぶどうと獣の血を加えて混ぜ、玉ねぎと未熟ぶどう果汁、パンを浸
  した残りのヴィネガーを加えて漉し器か布で漉す。これを鍋に入れ、こしょ
  う、生姜、シナモンを入れて炭火の上で30分程煮込む。獣の肉を獣脂を熱
  したフライパンで焼き、皿に盛る。上からポワヴラードをかけて供する、
  という内容(f.LXII)。またこの本には、魚料理用のポワヴラードも掲載
  されている。ただし、これが現代まで続くソース・ポワヴラードの原型と
  捉えるのは早計に過ぎる。ここで注目すべきは、最終的に肉あるいは魚の
  ような主素材とソースが一体化したものは中世〜ルネサンス期にはポター
  ジュと呼ばれていたのに対し、ここではソースを別のものと捉えている点
  である。ポワヴラードという語そのものは「こしょうを効かせたもの」と
  いう意味に過ぎず、1660年刊ピエール・ド・リュヌPierre de Lune『新フ
  ランス料理』におけるPoivrade de pigeonneaux 若鳩のポワヴラードは、
  背開きにした若鳩を平たくのばし、塩、こしょう をして弱火でグリルす
  る。薔薇の香りもしくはにんにく風味のヴィネガーを添えて供する、とい
  うもの(p.190)。ピエール・ド・リュヌのレシピにおいてソースに相当す
  るものはヴィネガーであり、むしろ味付けでこしょうを効かせているとい
  うことが料理名の根拠となっているに過ぎない。ちなみに、生食可能な小
  さなサイズのアーティチョークも古くからポワヴラードと呼ばれている。}
(標準)}{ソース・ポワヴラード (標準)}}\label{ux30bdux30fcux30b9ux30ddux30efux30f4ux30e9ux30fcux30c9105-ux6a19ux6e96}}

\hypertarget{sauce-poivrade}{%
\paragraph{Sauce Poivrade ordinaire}\label{sauce-poivrade}}

\index{そーす@ソース!ほわうらーど@---・ポワヴラード} \index{ほわうらー
ど@ポワヴラード!そーす@ソース・---} \index{sauce@sauce!poivrade
ordinaire@--- Poivrade ordinaire} \index{poivrade@poivrade!sauce
poivrade ordinaire@Sauce --- ordinaire}

細かいさいの目に切ったにんじん100 gと玉ねぎ80 g、刻んだパセリの茎、タ
イム少々、ローリエの葉少々からなる\protect\hyperlink{mirepoix}{ミルポワ}を油で色付くま
で炒める。

ヴィネガー1 dlとマリナード2 dlを注ぎ、\untiers{}量になるまで煮詰める。
\protect\hyperlink{sauce-espagnole}{ソース・エスパニョル}1
Lを注ぎ、約45分間煮込む。

ソースを漉す10分前に、大粒のこしょう8個を叩きつぶして加える。ソースに
こしょうを入れてからの時間がこれ以上少しでも長いと、こしょうの風味が支
配的になり過ぎることになるので注意。

漉し器で香味素材を軽く押すようにして漉す。\protect\hyperlink{}{マリナード}\footnote{ヴィネガーやワイン、香味素材、塩などを合わせて肉を漬け込む液体。
  マリネ液と呼ぶこともある。}2 dlでソー
スをのばす。火にかけて35分間、所定の量\footnote{明記されていないが、ここでは約1
  L。}になるまで煮詰めながら、表
面に浮いてくる不純物を徹底的に取り除く\footnote{dépouiller
  デプイエ。現代ではécumerエキュメの語を使う現場が多い。}。

さらに布で漉し、バター50 gを加えて仕上げる\footnote{現代では、バターでモンテするmonter
  au beurreという表現を用いる 現場も多い。}。

\maeaki

\hypertarget{ux30bdux30fcux30b9ux30ddux30efux30f4ux30e9ux30fcux30c9ux30b8ux30d3ux30a8ux7528}{%
\subsubsection{ソース・ポワヴラード(ジビエ用)}\label{ux30bdux30fcux30b9ux30ddux30efux30f4ux30e9ux30fcux30c9ux30b8ux30d3ux30a8ux7528}}

\hypertarget{sauce-poivrade-pour-gibier}{%
\paragraph{Sauce Poivrade pour
Gibier}\label{sauce-poivrade-pour-gibier}}

\index{そーす@ソース!ほわうらーとしひえ@---・ポワヴラード(ジビエ用)}
\index{ほわうらーと@ポワヴラード!そーすしひえよう@ソース・---(ジビエ
用)} \index{sauce@sauce!poivrade pour gibier@--- Poivrade pour
Gibier} \index{poivrade@poivrade!sauce poivrade pour gibier@Sauce ---
pour Gibier}

細かいさいの目に切ったにんじん125 gと玉ねぎ125 g、タイムの枝と鳥類では
ないジビエ\footnote{gibier à poil
  逐語訳すると「毛の生えているジビエ」すなわち」鹿、
  猪、野うさぎなどを指す。}の端肉1
kgからなる\protect\hyperlink{mirepoix}{ミルポワ}を油で色よく炒め る。

ミルポワが色付いてきたら、鍋の油を捨てる。ヴィネガー3 dlと白ワイン2 dl
を注ぎ、完全に煮詰める。

ソース・エスパニョル1
Lと\protect\hyperlink{fonds-de-gibier}{ジビエの茶色いフォン}2 L、
\protect\hyperlink{}{マリナード}1 Lを加える。

鍋に蓋をして弱火にかける。可能ならオーブンがいい。3時間半〜4時間加熱す
る。

ソースを漉す8分前に、大粒のこしょう12個を叩きつぶして加える。

漉し器で材料を押すようにして漉す。

これをジビエのフォン\unquart{} Lとマリナード\unquart{} Lでのばし、再び
火にかけて40分間、表面に浮いてくる不純物を丁寧に取り除きながら、1 Lに
なるまで煮詰める。

これを布で漉し、バター75gを加えて仕上げる。

\hypertarget{ux539fux6ce8-6}{%
\subparagraph{【原注】}\label{ux539fux6ce8-6}}

一般的にはジビエ料理のソースにはバターを加えないことになっているが、本
書では軽くバターを加えることを推奨する。そうすると、ソースの色の赤みは
薄まるが、繊細で滑らかな口あたりに仕上がる。

\maeaki

\hypertarget{ux30bdux30fcux30b9ux30ddux30ebux30c8}{%
\subsubsection{ソース・ポルト}\label{ux30bdux30fcux30b9ux30ddux30ebux30c8}}

\hypertarget{sauce-au-porto}{%
\paragraph{Sauce au Porto}\label{sauce-au-porto}}

\index{そーす@ソース!ほると@---・ポルト} \index{ほると@ポルト!そーす@
ソース・---} \index{sauce@sauce!porto@--- au Porto}
\index{porto@Porto!sauce au porto@Sauce au ---}

マデラ酒ではなくポルト酒を用いて、\protect\hyperlink{sauce-madere}{ソース・マデール}と
同様に作る。

\maeaki

\hypertarget{ux30ddux30ebux30c8ux30acux30ebux98a873ux30bdux30fcux30b9}{%
\subsubsection[ポルトガル風ソース]{\texorpdfstring{ポルトガル風\footnote{日本でもフランス語のままソース・ポルチュゲーズと呼ばれることは
  多い。フランス料理においてポルトガル風の名称を付けた料理はトマトをベー
  スとしたものがほとんど。ただし、トマトを使うからといってポルトガル風の
  名が必ず付くということはまったくない。なお、このソースとまったく関係な
  いが、\emph{Lettres Portugaises}
  レットル・ポルチュゲーズ『ぽるとがる文{[}ぶ
  み{]}』という題名の本が17世紀にフランスで出版され人々の感動を誘った。リ
  ルケや佐藤春夫が自国語に翻訳、翻案したものも非常に有名。実在したポルト
  ガルの修道女マリアナ・アルコフォラドがポルトガルに駐屯していたフランス
  軍人シャシー公爵に宛てた5通の恋文をまとめた、事実にもとづく書簡集だと
  長い間信じられていた。しかし、20世紀になってから、ガブリエル・ド・ギユ
  ラーグという男性文筆家によるまったくの創作であることが証明された。とは
  いえ作品の文学的価値はまったく減じることのない名作であり、書簡体小説の
  嚆矢とも言うべきもの。この小説形式は18世紀に隆盛を迎え、ラクロ『危険な
  関係』やルソー『新エロイーズ』、ゲーテ『若きウェルテルの悩み』など多く
  の傑作がものされた。19世紀バルザック『二人の若妻の手記』やドストエフス
  キー『貧しき人々』がこの小説形式の流行の最後を飾る傑作だろう。なお、ト
  マトは16世紀に既にフランスにもたらされており、オリヴィエ・ド・セールは
  1600年に刊行した『農業経営論』において、見た目には美しいが食べても美味
  しくないと断じている。食材として広く普及したのは19世紀以降であり、ヴィ
  アール、カレームなどの料理書でも材料として用いられるようになる。トマト
  ペーストの製品化は19世紀前半に実現されており、グリモ・ド・ラ・レニエー
  ルの「アルマナ」において言及がある。とりわけ19世紀後半はトマトを用いた
  料理が増えたのが特徴であり、オマール・アメリケーヌや舌びらめのデュグレ
  レなどは当時のトマトブームの反映と考えていい。また19世紀後半の小説家フ
  ロベールの遺作『ブヴァールとペキュシェ』においても、主人公二人が当時流
  行であったトマトの栽培に挑戦するが「芽掻き」をしなかったために失敗して
  しまうという場面が描かれている。}ソース}{ポルトガル風ソース}}\label{ux30ddux30ebux30c8ux30acux30ebux98a873ux30bdux30fcux30b9}}

\hypertarget{sauce-portugaise}{%
\paragraph{Sauce Portugaise}\label{sauce-portugaise}}

\index{そーす@ソース!ほるとかるふう@ポルトガル風---} \index{ほるとかる
ふう@ポルトガル風!そーす@---ソース} \index{sauce@sauce!porugaise@---
Portugaise} \index{portugais@portugais!sauce portugaise@Sauce
Portugaise}

(仕上り1 L分)

大きめの玉ねぎ1個を細かくみじん切りにする。鍋に油を熱し、強火で玉ねぎ
を炒める。玉ねぎがブロンド色になったら、皮を剥いて種子を取り除き、粗み
じん切りにしたトマト750 gと、つぶしたにんにく1片、塩、こしょうを加える。
トマトの酸味が強い場合は砂糖少々も加える。鍋に蓋をして、弱火で煮る。
\protect\hyperlink{essences-diverses}{トマトエッセンス}少々と、薄めに作ったトマトソース
を適量\footnote{仕上りの全体量が1
  Lなので、トマトソースを加える量は、グラスドヴィアンド
  を加える前の段階で0.9 L程度になるよう調整する。}、温めて溶かした\protect\hyperlink{glace-de-viande}{グラスドヴィアンド}1
dl、 新鮮なパセリの葉のみじん切り大さじ1杯を加えて仕上げる。

\maeaki

\hypertarget{ux30d7ux30edux30f4ux30a1ux30f3ux30b9ux98a8ux30bdux30fcux30b9}{%
\subsubsection{プロヴァンス風ソース}\label{ux30d7ux30edux30f4ux30a1ux30f3ux30b9ux98a8ux30bdux30fcux30b9}}

\hypertarget{sauce-provencal}{%
\paragraph{Sauce Provençale}\label{sauce-provencal}}

\index{そーす@ソース!ふろうあんすふう@プロヴァンス風---} \index{ふろう
あんすふう@プロヴァンス風!そーす@---ソース}
\index{sauce@sauce!provencale@--- Provençale}
\index{provencal@provençal!sauce provencale@Sauce Provençale}

大ぶりのトマト12個の皮を剥き、つぶして種子は取り除いて、粗く刻む\footnote{concasser
  コンカセ。}。 ソテー鍋に2\undemi{}
dlの油を熱し、そこにトマトを入れる。塩、こしょう、
粉砂糖1つまみで味を調える。しっかりつぶしたにんにく(小)1片と細かく刻
んだパセリ小さじ1杯を加える。

蓋をして弱火で30分間程、煮溶かす。

\hypertarget{ux539fux6ce8-7}{%
\subparagraph{【原注】}\label{ux539fux6ce8-7}}

このソースについてはさまざまな解釈があるが、本書ではブルジョワ料理にお
ける本物の「プロヴァンス風ソース」のレシピ、つまりはトマトを煮溶かした
もの、を収録した。

\maeaki

\hypertarget{ux30bdux30fcux30b9ux30ecux30b8ux30e3ux30f3ux30b975}{%
\subsubsection[ソース・レジャンス]{\texorpdfstring{ソース・レジャンス\footnote{摂政時代、すなわちオルレアン公フィリップが幼少だったルイ15世の
  摂政を務めた時代(1715〜1723年)のこと。オルレアン公は美食家として
  有名で、とりわけシャンパーニュを好んだという。この時代はフランス宮
  廷料理の絶頂期でもあった。}}{ソース・レジャンス}}\label{ux30bdux30fcux30b9ux30ecux30b8ux30e3ux30f3ux30b975}}

\hypertarget{sauce-regence}{%
\paragraph{Sauce Régence}\label{sauce-regence}}

\index{そーす@ソース!れしやんす@---・レジャンス} \index{れしやんす@レ
ジャンス!そーす@ソース・---} \index{sauce@sauce!regence@--- Régence}
\index{regence@Régence!sauce@Sauce ---}

ライン産ワイン3
dlに、細かく刻んであらかじめ日を通しておいた\protect\hyperlink{mirepoix}{ミルポ
ワ}1 dlと生トリュフの切りくず25gを加え、半量になるまで煮詰
める。トリュフのシーズンでない時季はトリュフエッセンスを使う。\protect\hyperlink{sauce-demi-glace}{ソース・
ドゥミグラス}8 dlを加え、数分間弱火にかけて浮いてく
る不純物を丁寧に取り除き\footnote{dépouiller デプイエ ≒ écumer
  エキュメ。}、布で漉す。

\ldots{}\ldots{}牛、羊の大きな塊肉の料理用。

\maeaki

\hypertarget{ux30bdux30fcux30b9ux30edux30d9ux30fcux30eb77}{%
\subsubsection[ソース・ロベール]{\texorpdfstring{ソース・ロベール\footnote{この名称のソースは古くからある。文献で初めて出てくるのは16世紀
  フランソワ・ラブレーの小説『ガルガンチュアとパンタグリュエル』。そ
  の「第四の書」で料理人の名が大量に列挙される章がある。そのうちの多
  くは架空の人名だが、その中のロベールという料理人がこのソースを考案
  したと書いている。ただし、具体的にどのようなソースかまでは描写され
  ておらず「うさぎのロースト、鴨、加工していない豚肉、卵のポシェ、塩
  漬けのメルラン{[}鱈の近縁種{]}、その他まことに多くの料理に欠かせない
  ソース」と書いてあるのみ(第40章)。どんな料理にも合うと書かれてし
  まうとむしろ特徴を捉え難くなってしまう。いずれにせよ、遅くとも16世
  紀には「ソース」として成立していたと考えられる。また、17世紀のシャ
  ルル・ペロー著『物語集』の「眠れる森の美女」においても、このソース
  名が登場する一節がある。このように16世紀以降多くの文学作品をはじめ
  とする文献にこのソース名は見られる。レシピとしては、1651年刊ラ・ヴァ
  レーヌ『フランス料理の本』における「豚腰肉 ソース・ロベール添え」
  がもっとも古いもののひとつだろう。概略は、豚腰肉を、ヴェルジュ{[}未
  熟ぶどう果汁、中世料理においてよく用いられた{]}とヴィネガー、セージ
  を振り掛けながらローストする。下に置いた脂受け皿に焼いた豚肉から流
  れ落ちた脂がたまるので、これを使って玉ねぎをこんがり炒める。炒めた
  玉ねぎの上に豚後ろ身を載せ、豚腰肉をローストする際にかけたのと同じ
  ソースをかける。このソースはソースロベールと呼ばれている(p.51)。ま
  た、干鱈のソース・ロベール添えの場合は、バターとヴェルジュ少々、マ
  スタードで作るが、ケイパーやシブール{[}葱{]}を加えてもいい(p.202)と
  あり、同じ名称のソースとは見做しがたい。18世紀以降のソース・ロベー
  ルは多かれ少なかれいずれもマスタードを加える点が共通しているので、
  名称が先にあり、内容が時代とともにはっきりしたものになっていたのだ
  ろう。}}{ソース・ロベール}}\label{ux30bdux30fcux30b9ux30edux30d9ux30fcux30eb77}}

\hypertarget{sauce-robert}{%
\paragraph{Sauce Robert}\label{sauce-robert}}

\index{そーす@ソース!ろへーる@---・ロベール} \index{ろへーる@ロベール!
そーす@ソース・---} \index{sauce@sauce!robert@--- Robert}
\index{robert@Robert!sauce robert@Sauce ---}

(仕上り5 dl分)

大きめの玉ねぎを細かくみじん切りにし、バターで色付かないよう強火でさっ
と炒める。

白ワイン2
dlを注ぎ、\untiers{}量になるまで煮詰める。\protect\hyperlink{sauce-demi-glace}{ソース・ドゥミグ
ラス}3 dlを加え、弱火で10分間煮る。

シノワ\footnote{主として金属製で円錐形に取っ手の付いた漉し器。清朝の高級役人が
  かぶっていた帽子の形状から「中国の」を意味するchinoisの名称となっ
  たと言われている。}で漉し(これは任意。漉さなくてもいい)、火から外して、粉砂
糖1つまみとマスタード大さじ1杯を加えて仕上げる。

\maeaki

\hypertarget{ux30bdux30fcux30b9ux30edux30d9ux30fcux30ebux30a8ux30b9ux30b3ux30d5ux30a3ux30a879}{%
\subsubsection[ソース・ロベール・エスコフィエ]{\texorpdfstring{ソース・ロベール・エスコフィエ\footnote{\protect\hyperlink{sauce-diable-escoffier}{ソース・ディアーブル・エスコフィエ}訳注参照。}}{ソース・ロベール・エスコフィエ}}\label{ux30bdux30fcux30b9ux30edux30d9ux30fcux30ebux30a8ux30b9ux30b3ux30d5ux30a3ux30a879}}

\hypertarget{sauce-robert-escoffier}{%
\paragraph{Sauce Robert Escoffier}\label{sauce-robert-escoffier}}

\index{そーす@ソース!ろへーるえすこふぃえ@---・ロベール・エスコフィエ}
\index{ろべーる@ロベール!そーすろへーるえすこふぃえ@ソース・---・エス
コフィエ} \index{sauce@sauce!robert escoffier@--- Robert Escoffier}
\index{robert@Robert!sauce robert escoffier@Sauce --- Escoffier}

このソースは完成品が市販されている。

温かい料理にも冷たい料理にもよく合う。温かい料理に合わせる場合は、同量
の\protect\hyperlink{fonds-de-veau-brun}{仔牛の茶色いフォン}と混ぜること。

\ldots{}\ldots{}豚、仔牛、鶏、魚のグリル焼きに特によく合う。

\maeaki

\hypertarget{ux30edux30fcux30deux98a880ux30bdux30fcux30b9}{%
\subsubsection[ローマ風ソース]{\texorpdfstring{ローマ風\footnote{フランス料理における「ローマ風」の名称は「イタリア風」と同様に
  とくに根拠や由来が見出せないものが多い。このソースの場合は松の実を
  使うところから、20世紀前半に活躍したイタリアの作曲家レスピーギのロー
  マ三部作のうちの「ローマの松」を想起させるが、残念ながらこの曲が作
  曲されたのは1924年、つまり本書より後なので関係はない。だが、松の実
  を採るイタリアカサマツは、アッピア街道の並木などで有名なように、イ
  タリアとりわけローマ近辺において多く見られる(だからこそレスピーギ
  が曲の題材にしたわけだが)。その意味においては、松の実を使っている
  ということがこのソース名の根拠と見ることも不可能ではないだろう。し
  かしながら、それを証明する文献、史料があるかは不明。}ソース}{ローマ風ソース}}\label{ux30edux30fcux30deux98a880ux30bdux30fcux30b9}}

\hypertarget{sauce-romaine}{%
\paragraph{Sauce Romaine}\label{sauce-romaine}}

砂糖50 gを火にかけてブロンド色にカラメリゼ\footnote{焦がさないように弱火で混ぜながら熱で砂糖を溶かしていく。}する。これをヴィネガー
1\undemi{}
dlでのばす。砂糖を完全に溶かし込めたら、\protect\hyperlink{sauce-espagnole}{ソース・エスパニョ
ル}6 dlと\protect\hyperlink{fonds-de-gibier}{ジビエのフォン}3 dlを加
える。これを\troisquarts{}量弱まで煮詰める。布で漉し、松の実20 gをロー
ストしたものと、大きさが揃るよう選別したスミヌル干しぶどう\footnote{トルコ産の白い干しぶどう。}20
gお よびコリント干しぶとう\footnote{ギリシア産の黒い小粒の干しぶどう(\protect\hyperlink{sauce-moscovite}{モスクワ風ソー
  ス}参照)。}20 gを温湯でもどしたものを加えて仕上げる。

\hypertarget{ux539fux6ce8-8}{%
\subparagraph{【原注】}\label{ux539fux6ce8-8}}

上記のとおり作る場合、このソースは大型ジビエ料理用だが、ジビエのフォン
ではなく通常の\protect\hyperlink{fonds-brun}{茶色いフォン}を使えば、マリネした牛、羊肉
の料理に合わせることも可能。

\maeaki

\hypertarget{ux30ebux30fcux30a2ux30f3ux98a884ux30bdux30fcux30b9}{%
\subsubsection[ルーアン風ソース]{\texorpdfstring{ルーアン風\footnote{ルーアンは野生のcolvertコルヴェール、いわゆる青首鴨を家禽化した
  ルーアン鴨の産地として有名。}ソース}{ルーアン風ソース}}\label{ux30ebux30fcux30a2ux30f3ux98a884ux30bdux30fcux30b9}}

\hypertarget{sauce-rouennaise}{%
\paragraph{Sauce Rouennaise}\label{sauce-rouennaise}}

\index{そーす@ソース!るーあんふう@ルーアン風---} \index{るーあんふう@
ルーアン風!そーす@---ソース} \index{sauce@sauce!rouannaise@---
Rouannaise} \index{rouannais@rouannais!sauce rouannaise@Sauce
Rouannaise}

(仕上り5 dl分)

\protect\hyperlink{sauce-bordelaise}{ボルドー風ソース}4 dl
を用意する。ただし、良質な赤
ワインを使って作ること。(\protect\hyperlink{sauce-bordelaise}{ボルドー風ソース}参照)。

中位の大きさの鴨のレバー3個を裏漉しする。こうして出来たレバーのピュレ
をソースに加え、沸騰させない程度の温度で火を通す\footnote{pocher
  ポシェする。}。絶対に沸騰させ
ないこと。沸騰させてしまうと途端にレバーのピュレが粒状になってしまう。

布で漉し、塩こしょうを効かせる。

このソースの特質\ldots{}\ldots{}エシャロットを加えた赤ワインを煮詰めたものに鴨の生
レバーのピュレを加えたもの。

\ldots{}\ldots{}ルーアン産鴨のローストには、いわば必須といってもいいソース。

\maeaki

\hypertarget{ux30bdux30fcux30b9ux30b5ux30ebux30df92}{%
\subsubsection[ソース・サルミ]{\texorpdfstring{ソース・サルミ\footnote{語源は「ごった煮」を意味する
  salmigondis とするのが定説のようだ
  が、salmigondisがその意味で用いられるようになったのは19世紀以降と
  考えられ、それ以前はragoûtラグーと同義と見なされていた。ラグーはそ
  の語源的意味が「食欲をそそるもの」であり、17世紀に、それまでポター
  ジュと呼ばれていた煮込み料理についてラグーの名称をつけることが流行
  した。また、salmigondisの古い語形のひとつsalmigondinは16世紀の小説
  家フランソワ・ラブレー『ガルガンチュアとパンタグリュエル』の「第四
  の書」において用いられているが、日本語の「ごった煮」のニュアンスと
  はかなり違う意味で、美味な料理のひとつとして挙げられている。いずれ
  にしても、salmigondin, salmigondisというラグーの別称が、ある時期か
  ら鳥類を材料にしたものに限定されるようになったことは確かで、カレー
  ムの『19世紀フランス料理』ではsalmisの語で、野鳥などのラグーを呼ん
  でいる。例えば「ベカスのサルミ」「ペルドローのサルミ」など。カレー
  ムとエスコフィエを比較すると、しばしばカレームにおいてラグーとして
  ひとまとめにされていた料理とソースの組合せが、『料理の手引き』にお
  いては、例えば\protect\hyperlink{}{ガルニチュール・フィナンシエール}と\protect\hyperlink{sauce-financiere}{ソース・フィ
  ナンシエール}のように、別々の項目に分離されてい るものが多くある。}}{ソース・サルミ}}\label{ux30bdux30fcux30b9ux30b5ux30ebux30df92}}

\hypertarget{sauce-salmis}{%
\paragraph{Sauce Salmis}\label{sauce-salmis}}

\index{そーす@ソース!さるみ@---・サルミ} \index{さるみ@サルミ!そーす@
ソース・---} \index{sauce@sauce!salmis@--- Salmis}
\index{salmis@salmis!sauce salmis@Sauce ---}

ソースというよりはむしろクリ\footnote{coulis \textless{} couler
  クレ「流れる」から派生した語だが、料理用語とし
  ては、やや水分の多いピュレと理解するといい。ここでは二つの解釈が可
  能で、ひとつは\protect\hyperlink{}{ポタージュ・クリ}に近いという意味。もうひとつは
  「昔ながらのソース」の意。後者の場合、エスコフィエが「古典料理」と
  呼ぶ17、18世紀においてソースのことをクリと呼んでいたのを踏まえてい
  ると考えられる。}と呼んだほうがいいこのソースの作り方
はどんな場合も一点を除いて変わることがない。それは、このソースを合わせ
るジビエ(鳥)の種類によって、つまり普通に肉料理として扱えるジビエか、
肉断ち\footnote{小斉のこと。カトリックの習慣として(厳密な教義ではない)四旬節
  (復活祭までの46日間)や毎週金曜などに行なわれる、肉食を断つ行為の
  こと。}の際の食材として扱えるもの\footnote{ある種の水鳥はイルカと同様に魚と同等のものと見做され、小斉の場
  合にも食材として認められていた。具体的にはハシヒロ鴨、オナガ鴨、サ
  ルセル鴨など。もっとも、水鳥を肉断ちの際の食材として扱うというのは
  一種の詭弁ともいえなくないわけで、このソースを作る際に\protect\hyperlink{sauce-espagnole-maigre}{魚料理用ソー
  ス・エスパニョル}をベースとした\protect\hyperlink{sauce-demi-glace}{ソース・
  ドゥミグラス}を使うとは考え難く、本文にあるよう
  にフォンの代用としてマッシュルームの茹で汁を用いるという指示を守るだ
  けで、厳密に小斉の料理として成立するレシピと言えるかは疑問の残ると
  ころだ。}かで、どんな液体を用いるかと いうことだけだ。

細かく刻んだ\protect\hyperlink{mirepoix}{ミルポワ}150
gをバターでじっくり色付くまで炒め
る。そこに、その料理で用いているジビエの手羽と腿の皮、ガラを細かく刻ん
で加える。

白ワイン3
dlを注ぎ、\untiers{}量まで煮詰める。\protect\hyperlink{sauce-demi-glace}{ソース・ドゥミグラ
ス}8 dlを加えて、約45分間弱火で煮込む。漉し器で漉す
が、その際に香味野菜とガラのエキス\footnote{原文quintessenceカンテサンス。本来の意味は錬金術でいう「第五元
  素」。16世紀の作家フランソワ・ラブレーは存命当時、自著を筆名「カン
  テサンス抽出をなし遂げたアルコフリバス師」で出版していた時期がある。
  もっとも、このカンテサンスという語自体は中世以来、料理において「エ
  キス」「美味しさの本質」程度の意味でよく用いられた。}が得られるよう、強く押し絞って
やること。こうして出来たクリを、このソースを合わせる鳥と同種のものでとっ
たフォン4 dlで薄める。

ジビエが肉断ちの食材と見做されるもので、なおかつそれを厳格に守って作ら
なければならない場合は、このときフォンの代わりにマッシュルームの茹で汁を
用いればいい。

約45分〜1時間、弱火にかけて浮いてくる不純物を丁寧に取り除いてやる\footnote{dépouiller
  デプイエ。現代ではécumerエキュメの語を用いる現場が多 い。}。
さらにソースを\deuxtiers{}以下の量になるまで煮詰める。これにマッシュルー
ムの茹で汁とトリュフエッセンスを適量加えて丁度いい濃度になるよう調製する。

布で漉し、軽くバターを加えて仕上げる\footnote{原文は légèrement
  beurrerでありそのまま訳したが、現代の調理現場 ではmonter au beurre
  バターでモンテする、という表現がよく使われる。}。

\hypertarget{ux539fux6ce8-9}{%
\subparagraph{【原注】}\label{ux539fux6ce8-9}}

仕上げの際に、ソース1 Lあたりバター約50 gを加えるが、これは任意。

\maeaki

\hypertarget{ux30bdux30fcux30b9ux30c8ux30ebux30c1ux30e593}{%
\subsubsection[ソース・トルチュ]{\texorpdfstring{ソース・トルチュ\footnote{tortue
  トルチュは海亀のこと。古くは海亀料理用のソースだったが、
  19世紀以降は仔牛の頭肉料理に合わせるのが一般的になった。}}{ソース・トルチュ}}\label{ux30bdux30fcux30b9ux30c8ux30ebux30c1ux30e593}}

\hypertarget{sauce-tortue}{%
\paragraph{Sauce Tortue}\label{sauce-tortue}}

\index{そーす@ソース!とるちゅ@---・トルチュ} \index{とるちゅ@トルチュ!
そーす@ソース・---} \index{sauce@sauce!tortue@--- Tortue}
\index{tortue@tortue!sauce tortue@Sauce ---}

2\undemi{}
Lの\protect\hyperlink{fonds-de-veau-brun}{仔牛のフォン}を鍋で沸かし、セージ3
g、マジョラム1 g、ローズマリー1 g、バジル2 g、タイム1 g、ローリエの葉1
g、パセリの葉1つまみ、マッシュルームの切りくず25 gを投入する。蓋をして
25分間煎じる。こうして煎じた液体を漉す2分前に大粒のこしょう4個を加える。

布で漉し、\protect\hyperlink{sauce-demi-glace}{ソース・ドゥミグラス}7
dlに\protect\hyperlink{sauce-tomate}{トマトソー ス}3
dlを合わせたものに、上記で煎じた液体を、風味が際立
つ程度に適量加える。\troisquarts{}量まで煮詰め、布で漉す。仕上げにマデ
ラ酒1 dlとトリュフエッセンス少々を加え、さらにカイエンヌで風味を引き締
める。

\hypertarget{ux539fux6ce8-10}{%
\subparagraph{【原注】}\label{ux539fux6ce8-10}}

このソースはある程度まとまった量で作る必要がある。カイエンヌを使う指示
があるからだ。それでも、カイエンヌはとても気をつけて量を加減する必要が
ある\footnote{フランス料理において(というよりも一般的なフランス人にとって)
  は、唐辛子の辛さは嫌われる傾向が非常に強い。}。

\maeaki

\hypertarget{ux30bdux30fcux30b9ux30f4ux30cdux30beux30f395}{%
\subsubsection[ソース・ヴネゾン]{\texorpdfstring{ソース・ヴネゾン\footnote{ノロ鹿chevreuilや猪sanglierなどの大型ジビエのこと。なおニホンジ
  カやエゾジカはcerfに分類され、フランス料理の食材としてはあまり高く
  評価されない傾向がある。}}{ソース・ヴネゾン}}\label{ux30bdux30fcux30b9ux30f4ux30cdux30beux30f395}}

\hypertarget{sauce-venaison}{%
\paragraph{Sauce Venaison}\label{sauce-venaison}}

完全に仕上げた「\protect\hyperlink{sauce-poivrade-pour-gibier}{ジビエ用ソース・ポワヴラー
ド}」\troisquarts{} Lに、\protect\hyperlink{}{グロゼイユのジュ
レ}大さじ3杯強を生クリーム1dlで溶いてから加える。

グロゼイユのジュレと生クリームを加えるのは、鍋を火から外して、提供直前
にすること。

\ldots{}\ldots{}大型ジビエ料理用。

\maeaki

\hypertarget{ux8d64ux30efux30a4ux30f3ux30bdux30fcux30b9}{%
\subsubsection{赤ワインソース}\label{ux8d64ux30efux30a4ux30f3ux30bdux30fcux30b9}}

\hypertarget{sauce-au-vin-rouge}{%
\paragraph{Sauce au Vin rouge}\label{sauce-au-vin-rouge}}

\index{そーす@ソース!あかわいん@赤ワイン---} \index{あかわいん@赤ワイ
ン!そーす@---ソース} \index{sauce@sauce!vin rouge@--- au Vin rouge}
\index{vin@vin!sauce au vin rouge@Sauce au Vin rouge}

「赤ワインソース」という場合、煮詰めてからブールマニエでとろみを付ける
ブルゴーニュ風の仕立てか、魚を煮るのに用いた赤ワインを使うことが特徴で
ある「ソース・マトロット」のいずれかから派生したものなのは言うまでもな
い。もっとも、後者の場合はワインの風味は失われてしまっていてソースの水
気と味付けの意味しか持っていないと言える。

両者どちらもまさしく「赤ワインソース」だが、\protect\hyperlink{sauce-bourguignonne}{ブルゴーニュ風ソー
ス}と\protect\hyperlink{sauce-matelote}{ソース・マトロット}はそれ
ぞれ作り方も用途も違うから別々の名称として、この「茶色い派生ソース」の
節で説明した。

筆者としては、本当の「赤ワインソース」は以下のように作るものと考えてい
る。

ごく細かく刻んだ標準的な\protect\hyperlink{mirepoix}{ミルポワ}125
gをバターで炒める。良 質の赤ワイン\undemi{}
Lを注ぐ。半量になるまで煮詰める。つぶしたにんに
く1片、\protect\hyperlink{sauce-espagnole}{ソース・エスパニョル}7\undemi{}
dlを加え、12〜
15分、火ひかけて浮いてくる不純物を丁寧に取り除く\footnote{dépouiller
  デプイエ ≒ écumer エキュメ。}。

布で漉し、バター100 gとアンチョビエッセンス小さじ1杯、カイエンヌ1つま
みを加えて仕上げる。

\ldots{}\ldots{}魚料理用ソース。

\maeaki

\hypertarget{ux30bdux30fcux30b9ux30b6ux30f3ux30acux30e997-a}{%
\subsubsection[ソース・ザンガラ
A]{\texorpdfstring{ソース・ザンガラ\footnote{もとの語形はzingaro
  ザンガロ、またはヂンガロ。ジプシー、ボヘミ
  アンの意。料理ではパプリカ粉末やカイエンヌを用いたものに命名される
  ことが多い。}
A}{ソース・ザンガラ A}}\label{ux30bdux30fcux30b9ux30b6ux30f3ux30acux30e997-a}}

\hypertarget{sauce-zingara-a}{%
\paragraph{Sauce Zingara A}\label{sauce-zingara-a}}

\index{そーす@ソース!さんからa@---・ザンガラ A} \index{さんから@ザンガ
ラ!そーすa@ソース・--- A} \index{しぷしーふう@ジプシー風!そーすa@ソー
ス・ザンガラ A} \index{sauce@sauce!zingaraa@--- Zingara A}
\index{zingara@Zingara!saucea@Sauce --- A}

このソースは古典料理の\protect\hyperlink{}{ガルニチュール・ザンガラ}とはまったく関係が
ない。むしろイギリス料理に由来し、本書でもイギリス風ソースの節において
似たようなものはいくつも採り上げている。

ヴィネガー2\undemi{} dlにエシャロットのみじん切り大さじ1杯を加えて半量
になるまで煮詰める。\protect\hyperlink{jus-de-veau-lie}{茶色いジュ}7
dlを注ぎ、バターで
揚げたパンの身160gを加える。弱火で5〜6分間煮る。パセリのみじん切り大さ
じ1杯とレモン\undemi{}個分の搾り汁を加えて仕上げる。

\maeaki

\hypertarget{ux30bdux30fcux30b9ux30b6ux30f3ux30acux30e9-b}{%
\subsubsection{ソース・ザンガラ
B}\label{ux30bdux30fcux30b9ux30b6ux30f3ux30acux30e9-b}}

\hypertarget{sauce-zingara-b}{%
\paragraph{Sauce Zingara B}\label{sauce-zingara-b}}

白ワイン3 dlとマッシュルームの茹で汁3 dlを合わせて\untiers{}量になるまで
煮詰める。

\protect\hyperlink{sauce-demi-glace}{ソース・ドゥミグラス}4
dlと\protect\hyperlink{sauce-tomate}{トマトソー ス}2\undemi{}
dl、\protect\hyperlink{fonds-blanc}{白いフォン}1 dlを注ぐ。
浮いてくる不純物を徹底的に取り除きながら5〜6分火かける。

仕上げに、カイエンヌ1つまみで風味を引き締め、太さ1〜2 mmの千切りにした
\footnote{julienne ジュリエンヌ。}ハム(脂身のないところ)と赤く漬けた舌肉70
gおよびマッシュルーム 50 g、トリュフ 30 gを加える。

\ldots{}\ldots{}仔牛料理、鶏料理用。
\end{recette}\newpage
\hypertarget{ux30dbux30efux30a4ux30c8ux7cfbux306eux6d3eux751fux30bdux30fcux30b9}{%
\section{ホワイト系の派生ソース}\label{ux30dbux30efux30a4ux30c8ux7cfbux306eux6d3eux751fux30bdux30fcux30b9}}

\hypertarget{petites-sauces-blanches-composuxe9es-et-de-ruxe9ductions}{%
\subsection{Petites Sauces Blanches, Composées et de
Réductions}\label{petites-sauces-blanches-composuxe9es-et-de-ruxe9ductions}}
\begin{recette}
\hypertarget{ux30bdux30fcux30b9ux30a2ux30ebux30d3ux30e5ux30d5ux30a7ux30e91}{%
\subsubsection[ソース・アルビュフェラ]{\texorpdfstring{ソース・アルビュフェラ\footnote{ナポレオン軍の元帥、ルイ・ガブリエル・スーシェ
  Louis-Gabriel Suchet, duc d'Albufera
  (1770〜1826)のこと。スペイン戦役の際にそれ
  までの軍功を称えられ、ナポレオンが1812年にアルビュフェラ公爵位を新
  設して授けた。帝政期の英雄のひとりであり、アルビュフェラおよびスー
  シェの名を冠した料理がいくつかある。1814年に帝政が崩壊した後も軍務、
  政務に携わり、最終的にフランス貴族院議員の地位を得た。アルビュフェ
  ラ公爵位については、1815年7月24日の勅令においてに正式に抹消されて
  いる。このソースの特徴は赤ピーマン(パプリカ)を加熱してなめらかに
  すり潰し、バターに練り込んだものを使う点にあるが、どのような経緯で
  このソースに赤ピーマンを用いるようになったのかは不明。ただし、この
  ソースを合わせる「肥鶏 アルビュフェラ」は詰め物(ファルス)に米を
  用いるが、アルビュフェラは湖の周辺の湿地帯で米の生産がおこなわれて
  いるという点では一応の関連性が認められよう。なお、アルビュフェラは
  バレンシアの湖とそこに形成された潟であり、現在はバレンシア州のアル
  ブフェーラ自然公園となっている。}}{ソース・アルビュフェラ}}\label{ux30bdux30fcux30b9ux30a2ux30ebux30d3ux30e5ux30d5ux30a7ux30e91}}

\hypertarget{sauce-albufera}{%
\paragraph{Sauce Albuféra}\label{sauce-albufera}}

\index{そーす@ソース!あるひゆふえら@---・アルビュフェラ}
\index{あるひゆふえら@アルビュフェラ!そーす@ソース・---}
\index{sauce@sauce!albufera@--- Albuféra}
\index{albufera@Albuféra!sauce@Sauce ---}

\protect\hyperlink{sauce-supreme}{ソース・シュプレーム}1
Lあたりに、溶かしたブロンド色
の\protect\hyperlink{glace-de-viande}{グラスドヴィアンド}2
dlと、標準的な分量比率で作っ た\href{}{赤ピーマンバター}50 gを加える。

\maeaki

\hypertarget{ux30bdux30fcux30b9ux30a2ux30e1ux30eaux30b1ux30fcux30cc3}{%
\subsubsection[ソース・アメリケーヌ]{\texorpdfstring{ソース・アメリケーヌ\footnote{アメリケーヌという名称の由来は諸説あるが、19世紀フランスの料理人
  ピエール・フレス Pierre Fraysse がアメリカで働いた後にパリで1853年
  に開いたレストラン「シェ・ピーターズ」でこの料理名で提供したという
  のが定説。ただし、1853年以前にレストラン「ボヌフォワ」に「ラングドッ
  ク産オマール ソース・アメリケーヌ添え」というメニューあり、フレス
  はその料理に改変を加えたか、名前だけをシンプルに「アメリケーヌ」と
  した程度という説もある。かつては、オマールの主産地のひとつブルター
  ニュ地方を意味する古い形容詞 armoricain(e) アルモリカン、アルモリ
  ケーヌの音が変化した料理名だと主張されることもあったが、19世紀には
  南仏産が中心であったトマトを用いる点で矛盾が生じてしまう。いずれに
  しても、この料理名がフレスの店シェ・ピーターズを基点として広く知ら
  れるようになったことは事実。}}{ソース・アメリケーヌ}}\label{ux30bdux30fcux30b9ux30a2ux30e1ux30eaux30b1ux30fcux30cc3}}

\hypertarget{sauce-americaine}{%
\paragraph{Sauce Américaine}\label{sauce-americaine}}

\index{そーす@ソース!あめりけーぬ@---・アメリケーヌ}
\index{あめりふう@アメリカ風!そーす@ソース・アメリケーヌ}
\index{sauce@sauce!americaine@--- Américaine}
\index{americain@américain!sauce americaine@Sauce Américaine}

このソースは\protect\hyperlink{homard-a-l-americaine}{オマール・アメリケーヌ}という料理
そのものと言っていい(「魚料理」の章、甲殻類、\protect\hyperlink{homard-a-l-americaine}{オマール・アメリケー
ヌ}参照)。

このソースは通常、オマール\footnote{アカザエビ科の甲殻類。加熱すると殻が真紅になることから、「海の
  枢機卿」(カトリックの枢機卿は赤い衣服を着るのが通常だった)とも呼
  ばれる。日本語では英語由来のロブスターと言うことも多い。ヨーロッパ
  オマールは一般的には300〜500 g程度のものが多いが、高級料理では800 g〜1
  kgのものが好んで用いられる。また、アメリカのオマールと異なり、
  活けの状態では甲殻が青みがかった黒褐色のものがしばしば存在し、 homard
  bleuオマールブルーといって珍重される。ちなみに日本の伊勢エ
  ビはフランス語のLangousteラングーストに近いもので、大きさ、色など
  にあまり違いは認められない。}の身をガルニチュールとした魚料理に添えられる。
オマールの身をやや斜めになるよう厚さ1 cm程度の輪切りにし\footnote{escalopper
  エスカロペ。エスカロップに切る。ここで使用するオマー
  ルは900g〜1kg程度のものを想定していることに注意。}、魚料理の
ガルニチュールとして供するわけだ。

\maeaki

\hypertarget{ux30a2ux30f3ux30c1ux30e7ux30d3ux30bdux30fcux30b9}{%
\subsubsection{アンチョビソース}\label{ux30a2ux30f3ux30c1ux30e7ux30d3ux30bdux30fcux30b9}}

\hypertarget{sauce-anchois}{%
\paragraph{Sauce Anchois}\label{sauce-anchois}}

\index{そーす@ソース!あんちょうい@アンチョビ---}
\index{あんちょひ@アンチョビ!そーす@---ソース}
\index{sauce@sauce!anchois@--- Anchois}
\index{anchois@anchois!sauce anchois@Sauce ---}

\href{}{ノルマンディー風ソース}8
dlを、バターを加える前の段階まで作る。\href{}{ア ンチョビバター}125
gを混ぜ込む。アンチョビのフィレ50 gを洗い、よく水
気を絞ってから小さなさいの目に切ったのを加えて仕上げる。

\ldots{}\ldots{}魚料理用。

\maeaki

\hypertarget{ux30bdux30fcux30b9ux30aaux30fcux30edux30fcux30eb4}{%
\subsubsection[ソース・オーロール]{\texorpdfstring{ソース・オーロール\footnote{夜明けの光、曙光のこと。オーロラの意味もあるため、日本では「オー
  ロラソース」と呼ばれることもあるが、マヨネーズとトマトケチャップを
  同量で混ぜ合わせたものもそう呼ばれることが多いので注意。}}{ソース・オーロール}}\label{ux30bdux30fcux30b9ux30aaux30fcux30edux30fcux30eb4}}

\hypertarget{sauce-aurore}{%
\paragraph{Sauce Aurore}\label{sauce-aurore}}

\index{そーす@ソース!おーろーる@---・オーロール}
\index{おーろーる@オーロール!そーす@ソース・---}
\index{sauce@sauce!aurore@--- Aurore}
\index{aurore@aurore!sauce@Sauce ---}

\protect\hyperlink{veloute}{ヴルテ}に真っ赤なトマトピュレを加えたもの。分量は、ヴルテが\troisquarts{}に対し、トマトピュレ\unquart{}とする。仕上げに、ソース1
Lあたり100 gのバターを加える。

\ldots{}\ldots{}卵料理、仔牛、仔羊肉の料理、鶏料理用。

\maeaki

\hypertarget{ux9b5aux6599ux7406ux7528ux30bdux30fcux30b9ux30aaux30fcux30edux30fcux30eb}{%
\subsubsection{魚料理用ソース・オーロール}\label{ux9b5aux6599ux7406ux7528ux30bdux30fcux30b9ux30aaux30fcux30edux30fcux30eb}}

\hypertarget{sauce-aurore-maigre}{%
\paragraph{Sauce Aurore maigre}\label{sauce-aurore-maigre}}

\index{そーす@ソース!おーろーるさかなよう@魚料理用---・オーロール}
\index{おーろーる@オーロール!そーすさかな@魚料理用ソース・---}
\index{sauce@sauce!aurore maigre@--- Aurore maigre}
\index{aurore@aurore!sauce maigre@Sauce --- maigre}

\protect\hyperlink{veloute-de-poisson}{魚料理用ヴルテ}に、上記と同じ割合でトマトピュレ
を加える。ソース1 Lあたりバター125 gを加えて仕上げる。

\ldots{}\ldots{}魚料理用

\maeaki

\hypertarget{ux30d0ux30a4ux30a8ux30ebux30f3ux98a8ux30bdux30fcux30b9}{%
\subsubsection{バイエルン風ソース}\label{ux30d0ux30a4ux30a8ux30ebux30f3ux98a8ux30bdux30fcux30b9}}

\hypertarget{sauce-bavaroise}{%
\paragraph{Sauce Bavaroise}\label{sauce-bavaroise}}

\index{そーす@ソース!はいえるんふう@バイエルン風---}
\index{はいえるんふう@バイエルン風!そーす@---ソース}
\index{sauce@sauce!bavarois@--- Bavaroise}
\index{bavarois@bavarois!sauce bavaroise@Sauce Bavaroise}

ヴィネガー5 dlにタイムとローリエの葉少々とパセリの枝4本、大粒のこしょ
う7〜8個と、おろした\footnote{原文 râpé \textless{} râpe
  ラープと呼ばれる器具を用いておろすが、日本のお
  ろし金と目の大きさが違うので注意。多くの場合、マンドリーヌ mandrine
  と呼ばれる野菜用スライサーにこの機能が付属している。}レフォール\footnote{raifort
  西洋わさび、ホースラディッシュ。}大さじ2杯を加え、半量になるまで
煮詰める。

この煮詰めた汁に卵黄6個を加え\footnote{卵黄を加える前に一度漉しておいたほうがいいだろう。}、\protect\hyperlink{sauce-hollandaise}{オランデーズソー
ス}を作る要領で、バター400 gと大さじ1\undemi{}杯の
水を少しずつ加えながら、ソースがしっかり乳化するまで混ぜていく。布で漉
す。

\protect\hyperlink{beurre-d-ecrevisse}{エクルヴィスバター}100
gと泡立てた生クリーム大さ じ2杯、さいの目に切ったエクルヴィス\footnote{ざりがにのこと。通常はヨーロッパザリガニécrevisse
  à pattes
  rougesエクルヴィスアパットルージュを指す。高級食材としてとても好ま
  れている。現在は代用としてécrevisse de Californieエクルヴィスドカ
  リフォルニ(ウチダザリガニ)が用いられることもある。日本在来のニホ
  ンザリガニや、外来種だが多く生息しているアメリカザリガニは通常、フ
  ランス料理には用いられない。いずれもジストマ(寄生虫)のリスクがあ
  るため、生食は厳禁。}の尾の身を加えて仕上げる。

\ldots{}\ldots{}魚料理用のこのソースは、ムースのような仕上りにすること。

\maeaki

\hypertarget{ux30bdux30fcux30b9ux30d9ux30a2ux30ebux30cdux30fcux30ba8}{%
\subsubsection[ソース・ベアルネーズ]{\texorpdfstring{ソース・ベアルネーズ\footnote{ベアルヌは旧地方名で、フランス南西部、現在のピレネー・アトラン
  ティック県のことを指すが、このソースはその地方とまったく関係がない。
  19世紀パリ郊外のレストラン「パヴィヨン・アンリIV」が店名に掲げてい
  るアンリ四世がベアルヌのポー生まれであることにちなんで命名したソー
  ス名というのが定説。}}{ソース・ベアルネーズ}}\label{ux30bdux30fcux30b9ux30d9ux30a2ux30ebux30cdux30fcux30ba8}}

\hypertarget{sauce-bearnaise}{%
\paragraph{Sauce Béarnaise}\label{sauce-bearnaise}}

\index{そーす@ソース!へあるねーす@---・ベアルネーズ}
\index{へあるぬふう@ベアルヌ風!そーす@ソース・---}
\index{へあるねーす@ベアルネーズ!そーす@ソース・---}
\index{sauce@sauce!bearnaise@--- Béarnaise}
\index{bearnais@béarnais!sauce bearnaise@Sauce Béarnaise}

白ワイン2 dlとエストラゴンヴィネガー2 dlに、エシャロットのみじん切り大
さじ4杯、枝のままの粗く刻んだエストラゴン20 g、セルフイユ10 g、粗挽き
こしょう5 g、塩1つまみを加えて、\untiers{}量になるまで煮詰める。

煮詰まったら、数分間放置して温度を下げる。ここに卵黄6個を加え、弱火に
かけて、生のバター(あるいはあらかじめ溶かしておいてもいい)500 gを加
えて軽くホイップしながらなめらかになるよう混ぜる。

卵黄に徐々に火が通っていくことでソースにとろみが付くので、絶対に弱火で
作業をすること\footnote{卵黄をソースのとろみ付けに用いること自体は中世から行なわれていた。
  開放式の炉の上に鍋を鉤で吊っている場合は鍋を火から外す必要があった
  が、その後の閉鎖式かまどや、オーブンの機能も備えた fourneau フルノー
  (日本の調理現場ではストーブあるいはピアノと呼ばれることも多い)の
  場合、熱の弱い部分に鍋を置けばいいことになる。また、このソースのよ
  うにバターが中心となる場合は水よりも高温になりやすいので本文にある
  ように注意が必要だが、ブランケットのような水が中心のものに卵黄を加
  えてとろみを付ける場合は、生クリームなどでよく溶きほぐした卵黄(こ
  の時点でしっかり乳化させておくのがポイント)を、鍋全体をしっかり混
  ぜながら加える場合は比較的高温でも問題なくきれいにとろみが付く。}。

バターを混ぜ込んだら、布で漉して味を調える。カイエンヌごく少量を加えて
風味を引き締める。仕上げに、刻んだエストラゴン大さじ杯とセルフイユ大さ
じ\undemi{}杯を加える。

\ldots{}\ldots{}牛、羊肉のグリル用。

\hypertarget{ux539fux6ce8}{%
\subparagraph{【原注】}\label{ux539fux6ce8}}

このソースを熱々で提供しようとは考えないこと。このソースは要するにバター
で作ったマヨネーズなのだ。ほの温い程度で充分であり、もし熱くし過ぎてし
まうと、ソースが分離してしまう。

そうなってしまったら、冷水少々を加えて泡立て器でホイップして元のあるべ
き状態に戻してやること。

\maeaki

\hypertarget{ux30c8ux30deux30c8ux5165ux308aux30bdux30fcux30b9ux30d9ux30a2ux30ebux30cdux30fcux30ba-ux30bdux30fcux30b9ux30b7ux30e7ux30edux30f310}{%
\subsubsection[トマト入りソース・ベアルネーズ /
ソース・ショロン]{\texorpdfstring{トマト入りソース・ベアルネーズ /
ソース・ショロン\footnote{19世紀後半、パリで有名レストラン「ヴォワザン」の料理長を務めた
  アレクサンドル・ショロン Alexandre Choron (1837〜1924)。自ら考案
  し、命名したという。}}{トマト入りソース・ベアルネーズ / ソース・ショロン}}\label{ux30c8ux30deux30c8ux5165ux308aux30bdux30fcux30b9ux30d9ux30a2ux30ebux30cdux30fcux30ba-ux30bdux30fcux30b9ux30b7ux30e7ux30edux30f310}}

\hypertarget{sauce-bearnaise-tomatee}{%
\paragraph{Sauce Béarnaise tomatée, dite Sauce
Choron}\label{sauce-bearnaise-tomatee}}

\index{そーす@ソース!へあるねーすとまといり@トマト入り---・ベアルネーズ}
\index{へあるぬふう@ベアルヌ風!そーすとまといり@トマト入りソース・ベアルネーズ}
\index{そーす@ソース!しょろん@---・ショロン}
\index{しょろん@ショロン!そーす@ソース・---}
\index{sauce@sauce!bearnaise tomatee@--- Béarnaise tomatée}
\index{bearnais@b\'earnais!sauce bearnaise tomatee@Sauce Béarnaise tomatée}
\index{sauce@sauce!choron@--- Choron}
\index{choron@Choron!sauce@Sauce ---}

ソース・ベアルネーズを上記のとおりに作るが、最後にセルフイユとエストラ
ゴンのみじん切りは加えない。充分固めに作っておき、ソースの\unquart{}量
の、充分に煮詰めたトマトピュレを加える。ソースの濃度が丁度いい具合にな
るよう注意すること。

\ldots{}\ldots{}\href{}{トゥルヌド・ショロン}、および他のさまざまな料理に添える。

\maeaki

\hypertarget{ux30b0ux30e9ux30b9ux30c9ux30f4ux30a3ux30a2ux30f3ux30c9ux5165ux308aux30bdux30fcux30b9ux30d9ux30a2ux30ebux30cdux30fcux30ba-ux30bdux30fcux30b9ux30d5ux30a9ux30a4ux30e811-ux30bdux30fcux30b9ux30f4ux30a1ux30edux30ef12}{%
\subsubsection[グラスドヴィアンド入りソース・ベアルネーズ /
ソース・フォイヨ /
ソース・ヴァロワ]{\texorpdfstring{グラスドヴィアンド入りソース・ベアルネーズ
/ ソース・フォイヨ\footnote{19世紀〜20世紀初頭にパリにあったレストランおよびそのオーナーシェ
  フの名。このソースを使った「仔牛の背肉・フォイヨ」がスペシャリテだっ
  たという。} / ソース・ヴァロワ\footnote{ヴァロワ王家およびヴァロワ公爵であったルイ・フィリップ(7月王政
  期のフランス国王。在位1830〜1848)にちなんだ名称。前出のフォイヨは
  レストランを開く以前、ルイ・フィリップに仕えていた。}}{グラスドヴィアンド入りソース・ベアルネーズ / ソース・フォイヨ / ソース・ヴァロワ}}\label{ux30b0ux30e9ux30b9ux30c9ux30f4ux30a3ux30a2ux30f3ux30c9ux5165ux308aux30bdux30fcux30b9ux30d9ux30a2ux30ebux30cdux30fcux30ba-ux30bdux30fcux30b9ux30d5ux30a9ux30a4ux30e811-ux30bdux30fcux30b9ux30f4ux30a1ux30edux30ef12}}

\hypertarget{sauce-bearnaise-a-la-glace-de-viande}{%
\paragraph{Sauce Béarnaise à la glace de viande, dite Foyot, ou
Valois}\label{sauce-bearnaise-a-la-glace-de-viande}}

\index{へあるぬふう@ベアルヌ風!そーすぐらすとういあんといり@グラスドヴィアンド入りソース・ベアルネーズ}
\index{そーす@ソース!へあるねーすくらすどういあんといり@---・ベアルネーズ(グラス・ド・ヴィアンド入り)}
\index{そーす@ソース!ふおいよ@---・フォイヨ}
\index{ふおいよ@フォイヨ!そーす@ソース・---}
\index{そーす@ソース!うあろわ@---・ヴァロワ}
\index{うあろわ@ヴァロワ!そーす@ソース・---}
\index{sauce@sauce!bearnaise a la glace de viande@--- Béarnaise à la glace de viande}
\index{bearnais@b\'earnais!sauce bearnaise a la glace de viande@Sauce Béarnaise à la glace de viande}
\index{sauce@sauce!foyot@--- Foyot} \index{foyot@Foyot!sauce@Sauce ---}
\index{sauce@sauce!valois@--- Valois}
\index{valois@Valois!sauce@Sauce ---}

標準的な\protect\hyperlink{sauce-bearnaise}{ソース・ベアルネーズ}を上記の分量で、固めに作る。溶かした\protect\hyperlink{glace-de-viande}{グラスドヴィアンド}を少しずつ加えて仕上げる。

\ldots{}\ldots{}牛、羊肉のグリル用。

\maeaki

\hypertarget{ux30bdux30fcux30b9ux30d9ux30ebux30b7ux30fc13}{%
\subsubsection[ソース・ベルシー]{\texorpdfstring{ソース・ベルシー\footnote{パリ東部、セーヌ川左岸にある地名。かつては荷揚げ港があり、19世
  紀には小さなレストランが多く店を構えていたという。}}{ソース・ベルシー}}\label{ux30bdux30fcux30b9ux30d9ux30ebux30b7ux30fc13}}

\hypertarget{sauce-bercy}{%
\paragraph{Sauce Bercy}\label{sauce-bercy}}

\index{そーす@ソース!へるしー@---・ベルシー}
\index{へるしー@ベルシー!そーす@ソース・---}
\index{sauce@sauce!bercy@--- Bercy} \index{bercy@Bercy!sauce@Sauce ---}

細かくみじん切りにしたエシャロット大さじ2杯をバターでさっと色付かない
よう炒める。白ワイン2\undemi{}
dlと\protect\hyperlink{fumet-de-poisson}{魚のフュメ}か、
このソースを合わせる魚の茹で汁2\undemi{} dlを注ぐ。

\deuxtiers{}量弱まで煮詰めたら、\protect\hyperlink{veloute-de-poisson}{ヴル
テ}\troisquarts{} Lを加える。ひと煮立ちさせてから、
鍋を火から外し、バター100 gとパセリのみじん切り大さじ1杯を加えて仕上げ
る。

\maeaki

\hypertarget{ux30bdux30fcux30b9ux30aaux30d6ux30fcux30eb-ux30bdux30fcux30b9ux30d0ux30bfux30ebux30c914}{%
\subsubsection[ソース・オ・ブール /
ソース・バタルド]{\texorpdfstring{ソース・オ・ブール /
ソース・バタルド\footnote{バタルドは「雑種の、中間の」の意。卵黄とバターだけでとろみを付
  ける\protect\hypertarget{sauce-hollandaise}{}{ソース・オランデーズ}と似てはいるが小麦粉
  も使うことからこの名が付いたと言われている。なお、パンのバタール
  bâtard も同じ語だが、細いバゲットと太いドゥーリーヴルの「中間」
  の太さとだからというのが通説。}}{ソース・オ・ブール / ソース・バタルド}}\label{ux30bdux30fcux30b9ux30aaux30d6ux30fcux30eb-ux30bdux30fcux30b9ux30d0ux30bfux30ebux30c914}}

\hypertarget{sauce-au-beurre}{%
\paragraph{Sauce au Beurre, dite Sauce Bâtarde}\label{sauce-au-beurre}}

\index{そーす@ソース!ふーる@---・オ・ブール}
\index{はたー@バター!そーす@ソース・オ・ブール}
\index{そーす@ソース!はたると@---・バタルド}
\index{はたると@バタルド!そーす@ソース・---}
\index{sauce@sauce!beurre@--- au Beurre}
\index{beurre@beurre!sauce@Sauce au Beurre}
\index{sauce@sauce!batarde@--- Bâtarde}
\index{batard@bâtard!sauce@Sauce Bâtarde}

小麦粉45 gと溶かしバター45gをよく混ぜ合わせ粘土状にする。そこに、7 gの
塩を加えた熱湯7\undemi{} dlを一気に注ぎ、泡立て器で勢いよく混ぜ合わせ
る。とろみ付け用の卵黄5個を生クリーム大さじ1\undemi{}杯でゆるめたもの
と、レモン汁少々を加える。

布で漉し、鍋を火から外して、良質なバター300gを加えて仕上げる。

\ldots{}\ldots{}アスパラガスや、さまざまな魚のブイイ\footnote{茹でたもの、の意。料理では、シンプルに茹でた肉、魚のこと。}

\hypertarget{ux539fux6ce8-1}{%
\subparagraph{【原注】}\label{ux539fux6ce8-1}}

このソースはとろみを付けた後、湯煎にかけておき、提供直前にバターを加え
るようにするといい。\footnote{本書には、日本でもかつて有名だった、エシャロットのみじん切りを
  加えたヴィネガーを煮詰めてバターを溶かし込んだ魚料理用ソース「ソー
  ス・ブールブラン」Sauce (au) Beurre blanc は収録されていない。この
  ソース・ブールブランはナント地方やアンジュー地方で淡水魚アローズや
  ブロシェに合わせる伝統的なソース。1890年頃にナント地方の女性料理人
  クレマンス・ルフーヴルが、ソース・ベアルネーズを作るつもりが誤って
  卵を加えるのを忘れてしまった結果として出来たものだとも言われている。}

\maeaki

\hypertarget{ux30bdux30fcux30b9ux30dcux30ccux30d5ux30a9ux30ef-ux767dux30efux30a4ux30f3ux3067ux4f5cux308bux30dcux30ebux30c9ux30fcux98a8ux30bdux30fcux30b9}{%
\subsubsection{ソース・ボヌフォワ /
白ワインで作るボルドー風ソース}\label{ux30bdux30fcux30b9ux30dcux30ccux30d5ux30a9ux30ef-ux767dux30efux30a4ux30f3ux3067ux4f5cux308bux30dcux30ebux30c9ux30fcux98a8ux30bdux30fcux30b9}}

\hypertarget{sauce-bonnefoy}{%
\paragraph{Sauce Bonnefoy, ou Sauce Bordelaise au vin
blanc}\label{sauce-bonnefoy}}

\index{ほぬふおわ@ボヌフォワ!そーす@ソース・---}
\index{そーす@ソース!おぬふおわ@---・ボヌフォワ}
\index{そーす@ソース!ほるどーふうしろわいん@ボルドー風--- (白)}
\index{ほるどーふう@ボルドー風!そーす@---ソース(白)}
\index{sauce@sauce!bonnefoy@--- Bonnefoy}
\index{bonnefoy@Bonnefoy!sauce@Sauce ---}
\index{sauce@sauce!bordelaise vin blanc@--- Bordelaise au vin blanc}
\index{bordelais@bordelais!sauce vin blanc@Sauce Bordelaise au vin blanc}

ブラウン系の派生ソースの節で採り上げた、赤ワインを用いて作る\protect\hyperlink{sauce-bordelaise}{ボルドー
風ソース}とまったく同じ作り方だが、赤ワインではなく、
グラーヴかソテルヌの白ワインを用いる。また\protect\hyperlink{sauce-espagnole}{ソース・エスパニョ
ル}ではなく\protect\hyperlink{veloute}{標準的なヴルテ}を使うこと。

このソースは仕上げに、みじん切りにしたエストラゴンを加える。

\ldots{}\ldots{}魚のグリル、白身肉のグリル用。

\maeaki

\hypertarget{ux30d6ux30ebux30bfux30fcux30cbux30e5ux98a8ux30bdux30fcux30b9}{%
\subsubsection{ブルターニュ風ソース}\label{ux30d6ux30ebux30bfux30fcux30cbux30e5ux98a8ux30bdux30fcux30b9}}

\hypertarget{sauce-bretonne-blanche}{%
\paragraph{Sauce Bretonne}\label{sauce-bretonne-blanche}}

\index{そーす@ソース!ぶるたーにゅふうしろ@ブルターニュ風---(ホワイト系)}
\index{ぶるたーにゅふう@ブルターニュ風!そーすしろ@---ソース(ホワイト系)}
\index{sauce@sauce!bretonne blanche@--- Bretonne (blanche)}
\index{breton@breton!sauce blanche@Sauce Bretonne (blanche)}

長さ3〜5 cm位の、ごく細い千切り\footnote{julienne ジュリエンヌ。}にしたポワローの白い部分30
gとセロリの白い部分30 g、玉ねぎ30 g、マッシュルーム30
gをバターで完全に火が通るまで鍋に蓋をして弱火で蒸し煮する\footnote{étuver
  エチュヴェ。本来は油脂とごく少量の水分を加えて弱火で蒸し煮することだが、野菜については、バターだけを使う場合も多い。étouffer
  エトゥフェとほぼ同じ意味で用いられることも多い。}。

\protect\hyperlink{veloute-de-poisson}{魚のヴルテ}\troisquarts{}
Lを加え、しばらく弱火にかけて浮いてくる不純物を丁寧に取り除く\footnote{dépouiller
  デプイエ ≒ écumer エキュメ。}。生クリーム大さじ3杯とバター50gを加えて仕上げる。

\maeaki

\hypertarget{ux30bdux30fcux30b9ux30abux30ceux30c6ux30a3ux30a8ux30fcux30eb20}{%
\subsubsection[ソース・カノティエール]{\texorpdfstring{ソース・カノティエール\footnote{小舟の漕ぎ手、の意。}}{ソース・カノティエール}}\label{ux30bdux30fcux30b9ux30abux30ceux30c6ux30a3ux30a8ux30fcux30eb20}}

\hypertarget{sauce-canotiere}{%
\paragraph{Sauce Canotière}\label{sauce-canotiere}}

\index{そーす@ソース!かのてぃえーる@---・カノティエール}
\index{かのてぃえーる@カノティエール!そーす@ソース・---}
\index{sauce@sauce!canotiere@--- Canotière}
\index{canotiere@Canotière!sauce@Sauce ---}

淡水魚を煮るのに用いた、\href{}{白ワイン入りクールブイヨン}を\untiers{}量に
煮詰める。クールブイヨンにはしっかり香り付けしてあり塩はごく少量しか入っ
ていないこと。

1 Lあたり80 gのブールマニエを加えてとろみを付ける。軽く煮立たせたら、
鍋を火から外してバター150 gとカイエンヌごく少量を加えて仕上げる。

\ldots{}\ldots{}淡水魚のクールブイヨン煮用。

\hypertarget{ux539fux6ce8-2}{%
\subparagraph{【原注】}\label{ux539fux6ce8-2}}

バターでグラセした小玉ねぎと小ぶりのマッシュルームを加えると、「\protect\hyperlink{sauce-matelote-blanche}{白いソー
ス・マトロット}」の代用となる。

\maeaki

\hypertarget{ux30b1ux30a4ux30d1ux30fcux5165ux308aux30bdux30fcux30b9}{%
\subsubsection{ケイパー入りソース}\label{ux30b1ux30a4ux30d1ux30fcux5165ux308aux30bdux30fcux30b9}}

\hypertarget{sauce-aux-capres}{%
\paragraph{Sauce aux Câpres}\label{sauce-aux-capres}}

\index{そーす@ソース!けいぱー@ケイパー---}
\index{けいぱー@ケイパー!そーす@---ソース}
\index{sauce@sauce!capres@--- aux Câpres}
\index{capre@câpre!sauce capres@Sauce aux Câpres}

上記の\protect\hyperlink{sauce-au-beurre}{ソース・オ・ブール}に、ソース1
Lあたり大さじ4 杯のケイパーを提供直前に加える。

\ldots{}\ldots{}いろいろな種類の魚を煮た料理に用いる。

\maeaki

\hypertarget{ux30bdux30fcux30b9ux30abux30ebux30c7ux30a3ux30caux30eb21}{%
\subsubsection[ソース・カルディナル]{\texorpdfstring{ソース・カルディナル\footnote{カトリックの枢機卿(カルディナル)の衣が伝統的に赤いものである
  ことと、オマールが「海の枢機卿」と呼ばれることに由来。}}{ソース・カルディナル}}\label{ux30bdux30fcux30b9ux30abux30ebux30c7ux30a3ux30caux30eb21}}

\hypertarget{sauce-cardinal}{%
\paragraph{Sauce Cardinal}\label{sauce-cardinal}}

\index{そーす@ソース!かるでぃなる@---・カルディナル}
\index{かるでぃなる@カルディナル!そーす@ソース・---}
\index{sauce@sauce!cardinal@--- Cardinal}
\index{cardinal@cardinal!sauce@Sauce ---}

\protect\hyperlink{sauce-bechamel}{ベシャメルソース}\troisquarts{}
Lに、(1)\protect\hyperlink{fumet-de-poisson}{魚のフュ
メ}とトリュフエッセンスを同量ずつ合わせて
\troisquarts{}量まで煮詰めたものを1\undemi{} dl加える。(2)生クリーム
1\undemi{} dlを加える。

鍋を火から外し、真っ赤に作った\protect\hyperlink{beurre-de-homard}{オマールバター}を加え、カ
イエンヌごく少量で風味を引き締める。

\ldots{}\ldots{}魚料理用。

\maeaki

\hypertarget{ux30deux30c3ux30b7ux30e5ux30ebux30fcux30e0ux5165ux308aux30bdux30fcux30b9}{%
\subsubsection{マッシュルーム入りソース}\label{ux30deux30c3ux30b7ux30e5ux30ebux30fcux30e0ux5165ux308aux30bdux30fcux30b9}}

\hypertarget{sauce-aux-champignons-blanche}{%
\paragraph{Sauce aux Champignons}\label{sauce-aux-champignons-blanche}}

\index{そーす@ソース!まっしゅるーむしろ@マッシュルーム---(ホワイト系)}
\index{まっしゅるーむ@マッシュルーム!そーすしろ@---ソース(ホワイト系)}
\index{sauce@sauce!champignonsblanche@--- aux Champignons (blanches)}
\index{champignon@champignon!sauce blanche@Sauce aux Champignons (blanche)}

マッシュルームを茹でた汁3
dlを\untiers{}量まで煮詰める。\protect\hyperlink{sauce-allemande}{ソース・アルマン
ド}\troisquarts{} Lを加え、数分間沸騰させる。あ
らかじめ\ruby{螺旋}{らせん}状に刻みを入れて整形\footnote{tourner
  トゥルネ。原義は「回す」。包丁を動かさずに材料の方を回
  すようにして切る、刻み目を入れることがこの用語の由来。マッシュルー
  ムの場合はその際に大量の切りくずが発生するので、それをソースなどの
  風味付けに利用することも多い。}してから茹でておいた真っ
白で小さなマッシュルーム100 gを加えて仕上げる。

\ldots{}\ldots{}鶏料理用。魚料理に添えることもある。魚料理に合わせる場合は、ソース・
アルマンドではなく\protect\hyperlink{veloute-de-poisson}{魚料理用ヴルテ}を用いること。

\maeaki

\hypertarget{ux30bdux30fcux30b9ux30b7ux30e3ux30f3ux30c6ux30a3ux30a422}{%
\subsubsection[ソース・シャンティイ]{\texorpdfstring{ソース・シャンティイ\footnote{料理においては生クリームをホイップしたクレーム・シャンティイが
  有名だが、元来は、パリ北方に位置する町の名。17世紀、コンデ公ルイ2
  世(大コンデとも呼ばれる)の城館があり、ヴァテル Vatel
  (Watel)(1635〜1671)がメートルドテルとして仕えていた。その館でル
  イ14世をはじめとする約千名もの賓客を招いて開かれた数日にわたる宴会
  の際に、食材の魚が少ししか届かないと誤解したヴァテルは責任をとるた
  めに自殺したと言われている。なお、魚はその後すぐに大量に館に届けら
  れたという。ヴァテルという人物についての記録は少ないが、この逸話は
  非常に有名で、2000年にジェラール・ドパルデュー主演で映画化された。
  料理や宴席での見世物、厨房の様子、16世紀以来珍重された飴細工などの
  歴史考証がとてもしっかりしており、一見に値する。}}{ソース・シャンティイ}}\label{ux30bdux30fcux30b9ux30b7ux30e3ux30f3ux30c6ux30a3ux30a422}}

\hypertarget{sauce-chantilly}{%
\paragraph{Sauce Chantilly}\label{sauce-chantilly}}

\index{そーす@ソース!しやんていい@---・シャンティイ}
\index{しやんていい@シャンティイ!そーす@ソース・---}
\index{sauce@sauce!chantilly@--- Chantilly}
\index{Chantilly@Chantilly!sauce@Sauce ---}

まれに「ソース・シャンティイ」の名で呼ばれることもあるが、これは後述の
「\protect\hyperlink{sauce-mousseline}{ソース・ムスリーヌ}」に他ならない。

\maeaki

\hypertarget{ux30bdux30fcux30b9ux30b7ux30e3ux30c8ux30fcux30d6ux30eaux30e4ux30f323}{%
\subsubsection[ソース・シャトーブリヤン]{\texorpdfstring{ソース・シャトーブリヤン\footnote{料理において通常、シャトーブリヤンは牛フィレの中心部分を3cm程度
  の厚さに切ったものを指す。この名称の由来には主に2説あり、ひとつは
  フランスロマン主義文学の父と言われる小説家フランソワ・ルネ・シャトー
  ブリヤン François René Chateaubriand (1768〜1848)の名を冠したと
  いうもの。ちなみにフランスロマン主義文学の母と呼ばれているのはス
  タール夫人Anne Louise Germaine de Staël(1766〜1817)。料理における
  シャトーブリヤンという名の由来のもうひとつの説は、ブルターニュ地
  方で畜産物の集積地であったシャトーブリヤン Châteaubriant という地
  名に由来するというもの。なお、本書の初版および第四版では
  Chateaubriandの綴り、第二版はChâteaubriantであり、第三版は
  Châteaubrian\textbf{d}という奇妙な綴りとなっている。}}{ソース・シャトーブリヤン}}\label{ux30bdux30fcux30b9ux30b7ux30e3ux30c8ux30fcux30d6ux30eaux30e4ux30f323}}

\hypertarget{sauce-chateaubriand}{%
\paragraph{Sauce Chateaubriand}\label{sauce-chateaubriand}}

\index{そーす@ソース!しゃとーふりやん@---・シャトーブリヤン}
\index{しゃとーふりやん@シャトーブリヤン! そーす@ソース・---}
\index{sauce@sauce!chateaubriand@--- Chateaubriand}
\index{chateaubriand@Chateaubriand!sauce@Sauce ---}

(仕上り5 dl分)

白ワイン4 dlに、みじん切りにしたエシャロット4個分とタイム少々、ローリ
エの葉少々、マッシュルームの切りくず40 gを加え、\untiers{}量になるまで
煮詰める。

\protect\hyperlink{jus-de-veau-brun}{仔牛のジュ}\footnote{本書では「仔牛の茶色いジュ」のレシピは掲載されているが、仔牛の
  「白い」ジュについての言及はない。ここでは通常の仔牛の茶色いジュを
  用いればいい。また、\protect\hyperlink{sauce-colbert}{ソース・コルベール}の項(第二
  版で加えられた)で、\href{}{ブール・コルベール}とこのソースを比較するに
  あたり、このソースを「軽く仕上げたグラスドヴィアンドにバターとパセ
  リのみじん切りを加えたもの」と述べている(\protect\hyperlink{sauce-colbert}{ソース・コルベー
  ル}本文参照)。このため、なぜこのソース・シャトー
  ブリヤンが「ブラウン系の派生ソース」の節ではなく「ホワイト系の派生
  ソース」に分類されているのか疑問が残るところ。}4
dlを加え、半量になるまで煮詰める。
布で漉し、鍋を火から外して、メートルドテルバター250 gと細かく刻んだエ
ストラゴン小さじ\undemi{}杯を加えて仕上げる。

\ldots{}\ldots{}牛、羊の赤身肉のグリル用。

\maeaki

\hypertarget{ux767dux3044ux30bdux30fcux30b9ux30b7ux30e7ux30d5ux30edux30efux6a19ux6e96}{%
\subsubsection{白いソース・ショフロワ(標準)}\label{ux767dux3044ux30bdux30fcux30b9ux30b7ux30e7ux30d5ux30edux30efux6a19ux6e96}}

\hypertarget{sauce-chaud-froid-blanche-ordinaire}{%
\paragraph{Sauce Chaud-froid blanche
ordinaire}\label{sauce-chaud-froid-blanche-ordinaire}}

\index{そーす@ソース!しよふろわしろ@白い---・ショフロワ(標準)}
\index{しよふろわ@ショフロワ!そーすしろ@白いソース---(標準)}
\index{sauce@sauce!chaud-froid blanche ordinaire@--- Chaud-froid blanche ordinaire}
\index{chaud-froid@chaud-froid!sauce blanche ordinaire@Sauce --- blanche ordinaire}

(仕上り1
L分)\ldots{}\ldots{}\protect\hyperlink{veloute}{標準的なヴルテ}\troisquarts{}
L、\protect\hyperlink{gelee-de-volaille}{鶏でとっ た白いジュレ}6〜7
dl、生クリーム\footnote{フランスの生クリームについては\protect\hyperlink{sauce-supreme}{ソース・シュプレー
  ム}訳注参照。}3 dl。

厚手のソテー鍋にヴルテを入れる。強火にかけ、ヘラで混ぜながらジュレと用
意した生クリーム\untiers{}量を少しずつ加えていく。

所定の分量にするには、\deuxtiers{}量くらいまで煮詰めることになる。

味見をして、固さを確認する。これを布で漉す\footnote{粘度の高いソースなどを布で漉す方法については、\protect\hyperlink{veloute}{ヴルテ}訳
  注参照。}。生クリームの残りを少
しずつ加え、ゆっくり混ぜながら、ショフロワに仕立てる食材を覆うのにいい
固さになるまで冷ましてやる。

\maeaki

\hypertarget{ux30d6ux30edux30f3ux30c9ux306eux30bdux30fcux30b9ux30b7ux30e7ux30d5ux30edux30ef}{%
\subsubsection{ブロンドのソース・ショフロワ}\label{ux30d6ux30edux30f3ux30c9ux306eux30bdux30fcux30b9ux30b7ux30e7ux30d5ux30edux30ef}}

\hypertarget{sauce-chaud-froid-blonde}{%
\paragraph{Sauce Chaud-froid blonde}\label{sauce-chaud-froid-blonde}}

\index{そーす@ソース!しよふろわふろんと@ブロンドの---・ショーフロワ}
\index{しよふろわ@ショーフロワ!そーす(きつねいろ)@ブロンドのソース---}
\index{sauce@sauce!chaud-froid blonde@--- Chaud-froid blonde}
\index{chaud-froid@chaud-froid!sauce blonde@Sauce --- blonde}

上記と同様に作るが、ヴルテではなく\protect\hyperlink{sauce-allemande}{ソース・アルマン
ド}を用いる。また、生クリームの量は半分に減らすこと。

\maeaki

\hypertarget{ux30bdux30fcux30b9ux30b7ux30e7ux30d5ux30edux30efux30aaux30fcux30edux30fcux30eb28}{%
\subsubsection[ソース・ショフロワ・オーロール]{\texorpdfstring{ソース・ショフロワ・オーロール\footnote{夜明け、曙光の意。}}{ソース・ショフロワ・オーロール}}\label{ux30bdux30fcux30b9ux30b7ux30e7ux30d5ux30edux30efux30aaux30fcux30edux30fcux30eb28}}

\hypertarget{sauce-chaud-froid-aurore}{%
\paragraph{Sauce Chaud-froid Aurore}\label{sauce-chaud-froid-aurore}}

\index{そーす@ソース!しよふろわおーろーる@---・ショーフロワ・オーロール}
\index{しよふろわ@ショーフロワ!そーすおーろーる@ソース・---・オーロール}
\index{おーろーる@オーロール!そーすしよふろわおーろーる@ソース・ショーフロワ・---}
\index{sauce@sauce!chaud-froid aurore@--- Chaud-froid Aurore}
\index{chaud-froid@chaud-froid!sauce aurore@Sauce --- Aurore}
\index{aurore@aurore!sauce chaud-froid aurore@Sauce Chaud-froid ---}

標準的な\protect\hyperlink{sauce-chaud-froid-blanche-ordinaire}{白いソース・ショフロワ}
を上記のとおり作る。そこに、真っ赤なトマトピュレを布で漉したもの
1\undemi{} dlとパプリカ粉末0.25 gを少量のコンソメで煎じた\footnote{infuser
  アンフュゼ。煮出す、煎じる、の意。}ものを加 える。

\ldots{}\ldots{}鶏のショフロワ用。

\hypertarget{ux539fux6ce8-3}{%
\subparagraph{【原注】}\label{ux539fux6ce8-3}}

あまり鮮かな色にしたくない場合は、パプリカを煎じた汁は数滴だけ加えるに
とどめるといい。

\maeaki

\hypertarget{ux30bdux30fcux30b9ux30b7ux30e7ux30d5ux30edux30efux30f4ux30a7ux30fcux30ebux30d7ux30ec}{%
\subsubsection{ソース・ショフロワ・ヴェールプレ}\label{ux30bdux30fcux30b9ux30b7ux30e7ux30d5ux30edux30efux30f4ux30a7ux30fcux30ebux30d7ux30ec}}

\hypertarget{sauce-choud-froid-vert-pre}{%
\paragraph[Sauce Chaud-froid au Vert-pré]{\texorpdfstring{Sauce
Chaud-froid au Vert-pré\footnote{緑の野原、草原、の意。}}{Sauce Chaud-froid au Vert-pré}}\label{sauce-choud-froid-vert-pre}}

\index{そーす@ソース!しよふろわうえーるふれ@---・ショーフロワ・ヴェールプレ}
\index{しよふろわ@ショーフロワ!そーすうえーるふれ@ソース・---・ヴェールプレ}
\index{うえーるふれ@ヴェールプレ!そーすしよふろわうえーるふれ@ソース・ショーフロワ・---}
\index{sauce@sauce!chaud-froid vert-pre@--- Chaud-froid au Vert-pré}
\index{chaud-froid@chaud-froid!sauce vert-pre@Sauce --- au Vert-pré}
\index{vert-pre@vert-pré!sauce chaud-froid vert-pre@Sauce Chaud-froid au ---}

鍋に白ワイン2 dlを沸かし、セルフイユとエストラゴン、刻んだシブレット、
刻んだパセリの葉を各1つまみずつ投入する。蓋をして火から外し、10分間煎
じてから布で漉す。

最初に示したとおりの分量で\protect\hyperlink{sauce-chaud-froid-blanche-ordinaire}{標準的なソース・ショフロ
ワ}を作り、煮詰めながら、上記の
香草を煎じた液体を少しずつ混ぜ込む。この段階で1 Lになるまで煮詰めてお
くこと。

\protect\hyperlink{}{ほうれんそうから採った緑の色素}をソースに加え、\textbf{ほんのり薄い緑色}にする。

この色素を加える際にはよく注意して、上で示したとおりの色合いになるよう少しずつ投入すること。

このソースは各種の鶏\footnote{日本語では鶏と一言で済ませるが、フランス語では
  poussin プサン (ひよこ、ひな鶏)、poulette
  プレット(若い雌鶏)、poulet プレ(若 鶏)、poule
  プール(雌鶏)、poulet de grain プレドグラン(50〜70日
  の若鶏)、poulet reine プレレーヌ(若鶏と肥鶏の中間のサイズでソテー
  やローストにする)、poulet quatre quarts プレカトルカール(45日程
  で食用にする)、poularde プラルド(肥鶏、1.8kg以上のものが多く、
  AOCを取得している産地もある)、chapon シャポン(去勢鶏、最大で6kg
  程になるというが、肉質は雌鶏に近く、高級品とされている)、coq コッ
  ク(雄鶏)などに細かく分類されている。}のショフロワ、とりわけ「\protect\hyperlink{}{ショフロワ・プランタニエ}」に用いる。

\maeaki

\hypertarget{ux9b5aux6599ux7406ux7528ux30bdux30fcux30b9ux30b7ux30e7ux30d5ux30edux30ef}{%
\subsubsection{魚料理用ソース・ショフロワ}\label{ux9b5aux6599ux7406ux7528ux30bdux30fcux30b9ux30b7ux30e7ux30d5ux30edux30ef}}

\hypertarget{sauce-chaud-froid-maigre}{%
\paragraph{Sauce Chaud-froid maigre}\label{sauce-chaud-froid-maigre}}

作り方の手順と分量は\protect\hyperlink{sauce-chaud-froid-blanche-ordinaire}{標準的なソース・ショフロ
ワ}とまったく同じだが、以下
の点を変更する。(1)通常の\protect\hyperlink{veloute}{ヴルテ}ではなく\protect\hyperlink{veloute-de-poisson}{魚料理用ヴル
テ}を用いる。(2)\protect\hyperlink{}{鶏のジュレ}ではなく\protect\hyperlink{}{白
い魚のジュレ}を用いること。

\hypertarget{ux539fux6ce8-4}{%
\subparagraph{【原注】}\label{ux539fux6ce8-4}}

一般的に、このソースは魚のフィレやエスカロップ、甲殻類に\protect\hyperlink{mayonnaise-collee}{コーティング
用マヨネーズ}の代わりとして用いることをお勧めす
る。コーティング用マヨネーズにはいろいろ不都合な点があり、そのうちの最
大のものは、ゼラチンが溶けるにつれて油が浸み出してきてしまうことだ。こ
ういう不都合はこの魚料理用ソース・ショフロワを使う場合には出てこない。
このソースは風味も明確ですっきりしているからコーティング用マヨネーズよ
りも好ましいだろう。

\maeaki

\hypertarget{ux30bdux30fcux30b9ux30b7ux30f4ux30ea33}{%
\subsubsection[ソース・シヴリ]{\texorpdfstring{ソース・シヴリ\footnote{19世紀フランスの作家フレデリック・スリエ
  Frédéric Soulié (1800〜
  1847)の劇『ディアーヌ・ド・シヴリ』\emph{Diane de Chivry}
  (1838年)ある
  いは1897年に新聞「フィガロ」に掲載されたエルネスト・カペンデュの小
  説あ『ビビタパン』の登場人物名Chivryにちなんだか、あるいはまったく別の人物の
  名を冠したものかは不明。}}{ソース・シヴリ}}\label{ux30bdux30fcux30b9ux30b7ux30f4ux30ea33}}

\hypertarget{sacue-chivry}{%
\paragraph{Sauce Chivry}\label{sacue-chivry}}

白ワイン1\undemi{} dlに以下を各1つまみずつ投入する\footnote{明記されていないが、この時点で白ワインは沸かしておく。}\ldots{}\ldots{}セルフイユ、
パセリ、エストラゴン、シブレット、時季が合えばサラダバーネット\footnote{pumprenelle
  パンプルネル、和名ワレモコウ。}の
若い葉。蓋をして鍋を火から外し、10分間煎じる\footnote{infuser
  アンフュゼ。}。布で絞るようにして 漉す。

こうしてハーブ類を煎じた液体を、あらかじめ沸かしておいた\protect\hyperlink{veloute}{ヴル
テ}\troisquarts{}
Lに加える。火から外し、\protect\hyperlink{beurre-a-la-chivry}{ブール・シヴ
リ}100
を加えて仕上げる(\protect\hyperlink{beurres-composes}{合わせバターの
節}参照)。

\ldots{}\ldots{}ポシェ\footnote{pocher
  原則的には、沸騰しない程度の温度で加熱調理すること。この
  場合は、下処理した鶏一羽まるごとをぎりぎり入るくらいの大きさの鍋に
  入れて水あるいはクールブイヨンを用いてゆっくり火を通す調理を意味し
  ている(温度管理が難しい場合はオーブンを用いることもある)。}あるいは茹でた鶏の料理用。

\hypertarget{ux539fux6ce8-5}{%
\subparagraph{【原注】}\label{ux539fux6ce8-5}}

サラダバーネットは生育するにつれて苦味が強くなるの、必ず若いものを使うこと。

\maeaki

\hypertarget{ux30bdux30fcux30b9ux30b7ux30e7ux30edux30f3}{%
\subsubsection{ソース・ショロン}\label{ux30bdux30fcux30b9ux30b7ux30e7ux30edux30f3}}

\hypertarget{sauce-choron}{%
\paragraph{Sauce Choron}\label{sauce-choron}}

\protect\hyperlink{sauce-bearnaise-tomatee}{トマト入りソース・ベアルネーズ}参照。

\maeaki

\hypertarget{ux30bdux30fcux30b9ux30afux30ecux30fcux30e0}{%
\subsubsection{ソース・クレーム}\label{ux30bdux30fcux30b9ux30afux30ecux30fcux30e0}}

\hypertarget{sauce-creme}{%
\paragraph{Sauce à la Crème}\label{sauce-creme}}

\index{そーす@ソース!くれーむ@---・クレーム}
\index{くりーむ@クリーム!そーす@ソース・クレーム}
\index{sauce@sauce!creme@--- à la Crème}
\index{creme@crème!sauce@Sauce à la ---}

\protect\hyperlink{sauce-bechamel}{ベシャメルソース}1 Lに生クリーム2
dlを加えて、ヘラで
混ぜながら強火で、全体量の\troisquarts{}になるまで煮詰める。

布で漉す\footnote{粘度や濃度の高いソースを漉す方法については\protect\hyperlink{veloute}{ヴルテ}訳注参照。}。フレッシュなクレーム・ドゥーブル\footnote{乳酸醗酵させた濃度の高い生クリーム。詳しくは\protect\hyperlink{sauce-supreme}{ソース・シュプレーム}訳注参照。}2\undemi{}
dlとレモン果汁半個分を少しずつ加えて仕上げる。

\ldots{}\ldots{}茹でた魚、野菜料理、鶏、卵料理用。

\maeaki

\hypertarget{ux30bdux30fcux30b9ux30afux30ebux30f4ux30a7ux30c3ux30c840}{%
\subsubsection[ソース・クルヴェット]{\texorpdfstring{ソース・クルヴェット\footnote{小海老のこと。フランスでよく料理に用いられるのは生の状態で甲殻
  が灰色がかった小さめのcrevettes grisesクルヴェット・グリーズと、や
  や大きめでピンク色のcrevettes rosesクルヴェット・ローズ。美味しい。
  ちなみに日本でよく食べられているブラックタイガーはフランス語にする
  とcrevette géante tigréeと言う。}}{ソース・クルヴェット}}\label{ux30bdux30fcux30b9ux30afux30ebux30f4ux30a7ux30c3ux30c840}}

\hypertarget{sauce-aux-crevettes}{%
\paragraph{Sauce aux Crevettes}\label{sauce-aux-crevettes}}

\index{そーす@ソース!くるうえつと@---・クルヴェット}
\index{くるうえつと@クルヴェット!そーす@ソース・---}
\index{sauce@sauce!crevette@--- aux Crevettes}
\index{crevette@crevette!sauce@Sauce aux Crevettes}

\protect\hyperlink{veloute-de-poisson}{魚料理用ヴルテ}または\protect\hyperlink{sauce-bechamel}{ベシャメルソー
ス}1 Lに、生クリーム1\undemi{}
dlと\protect\hyperlink{fumet-de-poisson}{魚のフュ メ}1\undemi{}
dlを加える。

火にかけて9
dlになるまで煮詰める。鍋を火から外し、\protect\hyperlink{}{ブール・ルー
ジュ}25 g(ソース全体に淡いピンクの色合いを付けるのが目的)を足した
\protect\hyperlink{}{クルヴェットバター}100gを加える。殻を剥いたクルヴェットの尾の身大
さじ3杯を加え、カイエンヌ1つまみで風味を引き締めて仕上げる。

\ldots{}\ldots{}魚料理およびある種の卵料理用。

\maeaki

\hypertarget{ux30abux30ecux30fcux30bdux30fcux30b9}{%
\subsubsection{カレーソース}\label{ux30abux30ecux30fcux30bdux30fcux30b9}}

\hypertarget{sauce-currie}{%
\paragraph{Sauce Currie}\label{sauce-currie}}

\index{そーす@ソース!かれー@カレー---}
\index{かれー@カレー!そーす@---ソース}
\index{sauce@sauce!currie@---  Currie}
\index{currie@currie!sauce@Sauce ---}

以下の材料をバターで軽く色付くまで炒める\ldots{}\ldots{}玉ねぎ250
g、セロリ100 g、 パセリの根\footnote{パセリには根パセリpersil
  tubéreuxといって根が肥大する品種系統も
  ある。平葉で、葉の香りはフランスで一般的なモスカールドタイプ(葉の
  縮れるタイプ)とやや異なる。イタリアンパセリのように用いることが可 能。}30
g、これらはすべてやや厚めにスライスする。タイム1枝と
ローリエの葉少々、メース少々を加える。小麦粉50gとカレー粉\footnote{カレーは植民地インドの料理としてイギリスに伝わり、18世紀にはC\&B
  社によって混合スパイスであるカレー粉が開発された。フランスはあまり
  インドやその他のカレーの食文化と接することもなかったために、こんに
  ちでも「珍しい料理」の範疇にとどまっている。とはいえ、19世紀にイン
  ドからアンティル諸島のうちの英領地域に連れて来られたインド人たちが
  カレーを伝え、それが広まってフランス領アンティーユにおいてコロンボ
  colomboというカレーのバリエーションが成立した。コロンボはこんにち
  のフランスでも(インドのカレーとは別のものとして)比較的よく知られ
  たものとなっている(少なくともcurry, currieという語よりは一般的認
  知度が高いと言えるだろう)。}小さじ1
杯弱を振り入れる。小麦粉が色付かない程度に炒めて火を通したら、\protect\hyperlink{}{白いコ
ンソメ} \troisquarts{} Lを注ぐ。沸騰したら、弱火にして約45分煮る。
軽く押し絞るように布で漉す。ソースを温めて、浮いてきた油脂は取り除き
\footnote{dégraisser デグレセ。}、湯煎にかけておく。

\ldots{}\ldots{}魚料理、甲殻類、鶏、さまざまな卵料理に合わせる。

\hypertarget{ux539fux6ce8-6}{%
\subparagraph{【原注】}\label{ux539fux6ce8-6}}

ココナツミルクをソースに加えることもある。その場合、白いコンソメの
\unquart{}量をココナツミルクに代えること。

\maeaki

\hypertarget{ux30a4ux30f3ux30c9ux98a8ux30abux30ecux30fcux30bdux30fcux30b9}{%
\subsubsection{インド風カレーソース}\label{ux30a4ux30f3ux30c9ux98a8ux30abux30ecux30fcux30bdux30fcux30b9}}

\hypertarget{sauce-currie-indienne}{%
\paragraph{Sauce Currie à l'Indienne}\label{sauce-currie-indienne}}

\index{そーす@ソース!いんどかれー@インドカレー---}
\index{かれー@カレー!そーすいんど@インド---ソース}
\index{sauce@sauce!currie indienne@---  Currie à l'Indienne}
\index{currie@currie!sauce indienne@Sauce --- à l'Indienne}

みじん切り\footnote{原文ciseler
  シズレ。鋭利な刃物でみじん切りにすること、スライス
  すること。原義は「ハサミで切る」。なお、日本語でみじん切りに相当す
  る用語にはhacherアシェもある(hache斧から派生した語)。後者は野菜
  の他、肉類を細かく刻む際にも用いられる。ミートチョッパーをフランス
  語ではhachoirアショワールと呼ぶ。}にした玉ねぎ1個と、パセリ、タイム、ローリエ、メース、シ
ナモン各少々のブーケガルニを、バターとともに弱火にかけて色付かないよう蒸
し煮する。

カレー粉3 gを振り入れ、ココナツミルク\undemi{} Lを注ぐ。ヴルテ \undemi{}
Lを加える(ソースを肉料理に合わせるか、魚料理に合わせるかで、
ヴルテも標準的なものを使うか、魚料理用を使うか決めること)。弱火で15分
程煮る。布で漉し、生クリーム1 dlとレモン果汁少々を加えて仕上げる。

\hypertarget{ux539fux6ce8-7}{%
\subparagraph{【原注】}\label{ux539fux6ce8-7}}

ここで示した量のココナツミルクは、生のココヤシの実700 gをおろして、
4\undemi{} dlの温めた牛乳で溶いて作る。それを布で強く絞って漉してから
使うこと。

ココナツミルクがない場合には、同量のアーモンドミルクを用いてもいい。

インドの料理人によるこのソースの作り方はさまざまで、基本だけが同じというものだ。

だが、本来のレシピがあったところで、使い物にはならないだろう。インドの
カレーは我が国の大多数にとっては我慢ならぬものだろうから。ここで記した
作り方は、ヨーロッパ人の味覚を勘案したものなので、本来のものよりいい筈
だ。

\maeaki

\hypertarget{ux30bdux30fcux30b9ux30c7ux30a3ux30d7ux30edux30deux30c3ux30c844}{%
\subsubsection[ソース・ディプロマット]{\texorpdfstring{ソース・ディプロマット\footnote{外交官風、の意。繊細で豪華な仕立ての料理に付けられる名称。}}{ソース・ディプロマット}}\label{ux30bdux30fcux30b9ux30c7ux30a3ux30d7ux30edux30deux30c3ux30c844}}

\hypertarget{sauce-diplomate}{%
\paragraph{Sauce Diplomate}\label{sauce-diplomate}}

\ruby{既}{すで}に仕上げでおいた\protect\hyperlink{sauce-normande}{ノルマンディ風ソース}1
Lに、\protect\hyperlink{}{オマールバター}75 gを加える。

さいの目に切ったオマールの尾の身大さじ2杯と同様にさいの目に切ったトリュ
フ大さじ1杯を加えて仕上げる。

\ldots{}\ldots{}大きな魚一尾まるごとの\footnote{relevé
  ルルヴェ。17世紀〜19世紀前半ににスタイルとして完成したフ
  ランス式サービスでは、最初に、大きな食卓(しばしば長い楕円形)の両
  側の目立つ場所にポタージュが置かれ、その周囲にアントレ(煮込みやソ
  テーなど今日では「メイン」にもなるもの)およびオルドゥーヴル(「作
  品でないもの」の意で、比較的簡単で小さな皿)が所狭しと並べられた。
  客はまずポタージュから食べはじめるのが基本であり、そのポタージュの
  大きな器が空くと、それは下げられて、ポタージュのあった場所に、豪華
  な装飾を施した飾り台(socleソークル)に載せられ、皿の周囲を飾るよ
  うにガルニチュールが配され(bordureボルデュール)、主役である大き
  な塊肉や魚まるごと1尾の料理にはしばしば飾り串(hâteletアトレ)が刺
  してある、きわめて壮麗な大皿料理が置かれた。ポタージュを取り上げた
  後に「より一層高くそびえ立つ(releverルルヴェした)もの、という意
  味でこの語が用いられるようになった。19世紀後半のロシア式サービスに
  おいても、まずポタージュが配られ、その後にオルドゥーヴル、アントレ
  と続き、ルルヴェを供するという習慣はしばらくの間残っていた。このた
  め、初版、第二版に付属している献立表、および第三版以降独立して出版
  された『メニューの本』にはルルヴェの語はしばしば見られる。1970年代
  ごろから宴席での大皿料理を給仕が取り分けるということが減り、厨房で
  銘々の皿に盛り付けをすることが一般化したために、こんにちでは滅多に
  このスタイルの料理は作られる機会がない。}料理用。

\maeaki

\hypertarget{ux30b9ux30b3ux30c3ux30c8ux30e9ux30f3ux30c9ux98a8ux30bdux30fcux30b9}{%
\subsubsection{スコットランド風ソース}\label{ux30b9ux30b3ux30c3ux30c8ux30e9ux30f3ux30c9ux98a8ux30bdux30fcux30b9}}

\hypertarget{sauce-ecossaise}{%
\paragraph{Sauce Ecossaise}\label{sauce-ecossaise}}

\index{そーす@ソース!すこつとらんとふう@スコットランド風---}
\index{すこつとらんとふう@スコットランド風!そーす@---ソース}
\index{sauce@sauce!ecossaise@--- Ecossaise}
\index{ecossais@écossais!sauce@Sauce Ecossaise}

上記の分量どおりに作った\protect\hyperlink{sauce-creme}{ソース・クレーム}9
dlに以下を加
えて作る。1〜2mmの細さに千切りにしたにんじん、セロリ、さやいんげんをバ
ターを加えて鍋に蓋をして弱火で蒸し煮し\footnote{étuver エチュヴェ。}、\protect\hyperlink{}{白いコンソメ}に完全に
浸したものを1dl。

\noindent\ldots{}\ldots{}卵料理、鶏料理に添える。

\maeaki

\hypertarget{ux30bdux30fcux30b9ux30a8ux30b9ux30c8ux30e9ux30b4ux30f350}{%
\subsubsection[ソース・エストラゴン]{\texorpdfstring{ソース・エストラゴン\footnote{ヨモギ科のハーブ。詳しくは茶色い派生ソースの\protect\hyperlink{sauce-chasseur}{ソース・シャスー
  ル}訳注参照。}}{ソース・エストラゴン}}\label{ux30bdux30fcux30b9ux30a8ux30b9ux30c8ux30e9ux30b4ux30f350}}

\hypertarget{sauce-estragon-blanche}{%
\paragraph{Sauce Estragon}\label{sauce-estragon-blanche}}

\index{そーす@ソース!えすとらこんしろ@---・エストラゴン(ホワイト系)}
\index{えすとらこん@エストラゴン!そーす@ソース・--- (ホワイト系)}
\index{sauce@sauce!estragon blanche@--- Estragon (blanche)}
\index{estragon@estragon!sauce blanche@Sauce --- (blanche)}

エストラゴンの枝30 gを粗く刻み\footnote{concasser コンカセ。}、強火で下茹でする\footnote{blanchir
  ブランシール。}。水気をしっ
かりときり、エストラゴンをスプーンですり潰し、あらかじめ用意しておいた
\protect\hyperlink{veloute}{ヴルテ}を大さじ4杯加える。これを布で漉す。こうして作ったエ
ストラゴンのピュレを\protect\hyperlink{veloute-de-volaille}{鶏のヴルテ}または\protect\hyperlink{veloute-de-poisson}{魚料理用
ヴルテ}1 Lに混ぜ込む。どちらのヴルテを使うから、
合わせる料理によって決めること。味を調え、みじん切りにしたエストラゴン
大さじ\undemi{}杯を加えて仕上げる。

\ldots{}\ldots{}卵料理、鶏肉料理、魚料理に合わせる。

\maeaki

\hypertarget{ux9999ux8349ux30bdux30fcux30b9}{%
\subsubsection{香草ソース}\label{ux9999ux8349ux30bdux30fcux30b9}}

\hypertarget{sauce-aux-fines-herbes-blanche}{%
\paragraph{Sauce aux Fines
Herbes}\label{sauce-aux-fines-herbes-blanche}}

\index{そーす@ソース!こうそうしろ@香草---(ホワイト系)}
\index{こうそう@香草!そーすしろ@---ソース(ホワイト系)}
\index{はーぶ@ハーブ!こうそうそーすしろ@香草ソース(ホワイト系)}
\index{sauce@sauce!fines herbes blanche@--- aux Fines Herbes (blanche)}
\index{fines herbes@fines herbes!sauce blanche@Sauce aux --- (blanche)}

(仕上り5 dl分)

あらかじめ2種のうちどちらかの方法(\protect\hyperlink{sauce-vin-blanc}{白ワインソース}
参照)で作っておいた\protect\hyperlink{sauce-vin-blanc}{白ワインソース}\undemi{}
Lに、 \protect\hyperlink{}{エシャロットバター}40
gと、パセリ、セルフイユ、エストラゴンのみじ
ん切りを大さじ1\undemi{}杯加える。

\ldots{}\ldots{}魚料理用。

\maeaki

\hypertarget{ux30bdux30fcux30b9ux30d5ux30a9ux30a4ux30e8}{%
\subsubsection{ソース・フォイヨ}\label{ux30bdux30fcux30b9ux30d5ux30a9ux30a4ux30e8}}

\hypertarget{sauce-foyot}{%
\paragraph{Sauce Foyot}\label{sauce-foyot}}

\protect\hyperlink{sauce-bearnaise-a-la-glace-de-viande}{グラスドヴィアンド入りソース・ベアルネーズ}参照。

\maeaki

\hypertarget{ux30bdux30fcux30b9ux30b0ux30edux30bcux30a4ux30e651}{%
\subsubsection[ソース・グロゼイユ]{\texorpdfstring{ソース・グロゼイユ\footnote{日本語で「すぐりの実」のことだが、こんにちでは「黒すぐり」の方
  が一般的かも知れない。黒すぐりはフランス語では cassis カシスと呼ば
  れる。一般的なグロゼイユにはフサスグリと呼ばれる groseille rouge
  グロゼイユ・ルージュ(赤すぐり)とgroseille blancheグロゼイユ・ブ
  ランシュ(白すぐり)の2種があり、どちらもブドウのように房なりする。
  上記とは別に、このソースで用いられるgroseille à maquereauグロゼイ
  ヤマクロー(maquereauは鯖の意。日本では英語経由のグーズベリーまた
  はグースベリーの名称でも呼ばれることが多い。単に西洋すぐりとも呼ぶ)
  という比較的大粒で薄く縞模様の入る種類もある。これは通常は緑色だが、ま
  れに紫色になる変種もあるという。いずれもフランスでは料理や菓子作り
  によく用いられる。}}{ソース・グロゼイユ}}\label{ux30bdux30fcux30b9ux30b0ux30edux30bcux30a4ux30e651}}

\hypertarget{sauce-groseilles}{%
\paragraph{Sauce Groseilles}\label{sauce-groseilles}}

\index{そーす@ソース!くろせいゆ@---・グロゼイユ}
\index{くろせいゆ@グロゼイユ!そーす@ソース・---}
\index{sauce@sauce!groseilles@--- Groseilles}
\index{groseille@groseille!sauce@Sauce Groseilles}

緑色の濃いグーズベリー500 gを銅の片手鍋で下茹でする。

5分間煮立てたら、水気をきって、粉砂糖大さじ3杯と白ワイン大さじ2〜3杯を
加えて、完全に火をとおす。布で漉す。

こうして出来たピュレに、\protect\hyperlink{sauce-au-beurre}{ソース・オ・ブール}5
dlを加 え、よく混ぜる。

\ldots{}\ldots{}このソースはグリルあるいはイギリス風\footnote{à
  l'anglaise
  アラングレーズ。通常は塩適量を加えた湯でボイルすることを指す。}に茹でた鯖によく合う。と
はいえ、他の魚料理にも合わせてもいい。

\hypertarget{ux539fux6ce8-8}{%
\subparagraph{【原注】}\label{ux539fux6ce8-8}}

このソースは緑色の房なりのグロゼイユ\footnote{一般的なフサスグリであれば白系統の「未熟果」を用いるということと解釈される。}でも作ることが可能。

\maeaki

\hypertarget{ux30aaux30e9ux30f3ux30c7ux30fcux30baux30bdux30fcux30b954}{%
\subsubsection[オランデーズソース]{\texorpdfstring{オランデーズソース\footnote{ニューヨーク発祥の朝食メニューとして知られるエッグ・ベネディク
  ト\emph{Egg
  Benedict}に必ず用いられることで有名なうえ、一般的には「バター
  で作るマヨネーズ」のイメージが強いかも知れない。実際のところは、ラ・
  ヴァレーヌ『フランス料理の本』(1651年)において「アスパラガスの白
  いソース添え」Asperges à la sauce blancheというレシピにおいて、こ
  のオランデーズソースの原型ともいうべきものが示されている。アスパラ
  ガスは固めに塩茹でする。「新鮮なバター、卵黄、塩、ナツメグ、ヴィネ
  ガー少々をよくかき混ぜる。ソースが滑らかになったら、アスパラガスに
  添えて供する(p.238)」。簡潔な記述だが、これがオランデーズソースの
  原型であることは間違いないだろう。おそらくはラ・ヴァレーヌ以前から
  存在していた可能性も否定できない。なお植物油を用いたマヨネーズが文
  献上で確認されるのが18世紀以降で、19世紀初頭から爆発的に流行し、広
  まったもの。また、マヨネーズについては、現代ヨーロッパにおいても卵
  黄ではなく全卵を用いて作るほうが多数を占めている点が異なることに注
  意。なお、オランデーズとは「オランダ風」の意だが、なぜこの名称となっ
  たのかについては不明な点が多い。また、2007年版の『ラルース・ガスト
  ロノミック』では、オランデーズソースを作る際には温度に注意すること
  と、よくメッキされた銅鍋かステンレス製の鍋を用いる必要があり、アル
  ミ製の鍋だと緑色に変色する可能性があることに注意を促している (p.455)。}}{オランデーズソース}}\label{ux30aaux30e9ux30f3ux30c7ux30fcux30baux30bdux30fcux30b954}}

\hypertarget{sauce-hollandaise}{%
\paragraph{Sauce Hollandaise}\label{sauce-hollandaise}}

\index{そーす@ソース!おらんてーす@オランデーズ---}
\index{おらんてーす@オランデーズ!そーす@---ソース}
\index{おらんたふう@オランダ風!そーす@オランデーズソース}
\index{sauce@sauce!hollandaise@--- Hollandaise}
\index{hollandais@hollandais!sauce@Sauce Hollandaise}

大さじ4杯の水とヴィネガー大さじ2杯に、粗挽きこしょう1つまみと肌理の細
かい塩1つまみを加えて、\untiers{}量まで煮詰める。この鍋を熱源のそばか、
湯煎にかける。

大さじ5杯の水と卵黄5個を加える。生のまま、あるいは溶かしたバター500 g
を加えながらしっかりホイップする。ホイップしている途中で、水を大さじ3〜
4杯、少量ずつ足してやる。水を足すのは、軽やかな仕上りにするため。

レモンの搾り汁少々と必要なら塩を足して味を調え、布で漉す。

湯煎にかけておくが、ソースが分離しないように、温度は微温くしておく。

\ldots{}\ldots{}魚料理、野菜料理用。

\hypertarget{ux539fux6ce8-9}{%
\subparagraph{【原注】}\label{ux539fux6ce8-9}}

ヴィネガーを煮詰めて使うのは、いつも最高品質のものが使えるとはかぎらな
いからで、水は\untiers{}量まで減らしたほうがいい。ただし、煮詰める作業
を完全に省いてしまわないこと。

\maeaki

\hypertarget{ux30bdux30fcux30b9ux30aaux30deux30fcux30eb}{%
\subsubsection{ソース・オマール}\label{ux30bdux30fcux30b9ux30aaux30deux30fcux30eb}}

\hypertarget{sauce-homard}{%
\paragraph{Sauce Homard}\label{sauce-homard}}

\index{そーす@ソース!おまーる@---・オマール}
\index{おまーる@オマール!そーす@ソース・---}
\index{sauce@sauce!homard@--- Homard}
\index{homard@homard!sauce@Sauce ---}

\protect\hyperlink{veloute-de-poisson}{魚料理用ヴルテ}\troisquarts{}
Lに、生クリーム1 \undemi{} dlと\protect\hyperlink{}{オマールバター}80
g、\protect\hyperlink{}{赤いバター}40 gを加えて仕上 げる。

\ldots{}\ldots{}魚料理用。

\hypertarget{ux539fux6ce8-10}{%
\subparagraph{【原注】}\label{ux539fux6ce8-10}}

このソースを魚1尾まるごとの料理に添える場合には、さいの目に切ったオマー
ルの尾の身を大さじ3杯加える。

\maeaki

\hypertarget{ux30cfux30f3ux30acux30eaux30fcux98a856ux30bdux30fcux30b9}{%
\subsubsection[ハンガリー風ソース]{\texorpdfstring{ハンガリー風\footnote{原書でも用いられている語paprikaパプリカはハンガリー語。唐辛子、
  ピーマンの仲間であり、16世紀以降17世紀にヨーロッパ全土に広まり、そ
  の土地ごとの風土に合わせて品種が多様化した。パプリカはとりわけ辛味
  成分をほとんど含んでいないのが特徴。ただし、ハンガリーの食文化にお
  いて大きな役割を果すようになったのは19世紀以降になってからと言われ
  ている。}ソース}{ハンガリー風ソース}}\label{ux30cfux30f3ux30acux30eaux30fcux98a856ux30bdux30fcux30b9}}

\hypertarget{sauce-hongroise}{%
\paragraph{Sauce Hongroise}\label{sauce-hongroise}}

\index{そーす@ソース!はんかりーふう@ハンガリー風---}
\index{はんかりーふう@ハンガリー風!そーす@---ソース}
\index{sauce@sauce!hongroise@---  Hongroise}
\index{hongrois@hongrois!sauce@Sauce Hongroise}

大きめの玉ねぎ1個のみじん切りをバターで色付かないよう強火で炒める。塩1
つまみとパプリカ粉末1 gで味付けする。

このソースを添える料理に合わせて\protect\hyperlink{veloute}{標準的なヴルテ}あるいは\protect\hyperlink{veloute-de-poisson}{魚
料理用ヴルテ} 1 Lを加え、数分間軽く煮立てる。

布で漉し、バター100 gを加えて仕上げる。

このソースは淡いピンク色に仕上げるべきであり、その色を出しているのがパ
プリカ粉末だけによるものだということに注意。

\ldots{}\ldots{}仔羊や仔牛のノワゼット\footnote{noisette
  ロースの中心部分を円筒形に切り出して調理したもの。}にとりわけよく合う。卵料理、鶏料理、魚
料理にも。

\maeaki

\hypertarget{ux7261ux8823ux5165ux308aux30bdux30fcux30b9}{%
\subsubsection{牡蠣入りソース}\label{ux7261ux8823ux5165ux308aux30bdux30fcux30b9}}

\hypertarget{sauce-aux-huitres}{%
\paragraph{Sauce aux Huîtres}\label{sauce-aux-huitres}}

\index{そーす@ソース!かきいり@牡蠣入り---}
\index{かき@牡蠣!そーす@牡蠣入りソース}
\index{sauce@sauce!huitres@---  aux Huîtres}
\index{huitre@huître!sauce@Sauce aux Huîtres}

後述の\protect\hyperlink{sauce-normande}{ノルマンディ風ソース}に、ポシェ\footnote{pocher
  \textless{} poche ポシュ(ポケット)、からの派生語。ポーチドエッグ
  を作る際に、ポケット状になるところからこの用語が定着した。沸騰しな
  い程度の温度で加熱調理すること。}して周囲をきれ
いにした牡蠣の身を加えたもの。

\maeaki

\hypertarget{ux30a4ux30f3ux30c9ux98a8ux30bdux30fcux30b9}{%
\subsubsection{インド風ソース}\label{ux30a4ux30f3ux30c9ux98a8ux30bdux30fcux30b9}}

\hypertarget{sauce-indienne}{%
\paragraph{Sauce Indienne}\label{sauce-indienne}}

\index{そーす@ソース!いんとふう@インド風---}
\index{いんとふう@インド風!そーす@---ソース}
\index{sauce@sauce!indienne@---  Indienne}
\index{indien@indien!sauce@Sauce Indienne}

\protect\hyperlink{sauce-currie-indienne}{インド風カレーソース}参照。

\maeaki

\hypertarget{ux30bdux30fcux30b9ux30a4ux30f4ux30a9ux30efux30fcux30eb}{%
\subsubsection{ソース・イヴォワール}\label{ux30bdux30fcux30b9ux30a4ux30f4ux30a9ux30efux30fcux30eb}}

\hypertarget{sauce-ivoire}{%
\paragraph[Sauce Ivoire]{\texorpdfstring{Sauce Ivoire\footnote{象牙、の意。}}{Sauce Ivoire}}\label{sauce-ivoire}}

\index{そーす@ソース!いうおわーる@---・イヴォワール}
\index{いうおわーる@イヴォワール!そーす@ソース・---}
\index{sauce@sauce!ivoire@--- Ivoire}
\index{ivoire@ivoire!sauce@Sauce ---}

\protect\hyperlink{sauce-supreme}{ソース・シュプレーム}1
Lに、ブロンド色の\protect\hyperlink{glace-de-viande}{グラスドヴィ
アンド}大さじ3杯を加え、象牙のようなくすんだ色合いに する。

\ldots{}\ldots{}ポシェした鶏に添える。

\maeaki

\hypertarget{ux30bdux30fcux30b9ux30b8ux30e7ux30efux30f3ux30f4ux30a3ux30eb61}{%
\subsubsection[ソース・ジョワンヴィル]{\texorpdfstring{ソース・ジョワンヴィル\footnote{19世紀、7月王政期の国王ルイ・フィリップの第3子、フランソワ・ド
  ルレアン・ジョワンヴィル海軍大将(1818〜1900)のこと。エクルヴィス
  とクルヴェットを用いた料理に彼の名が冠されたものがいくつかある。}}{ソース・ジョワンヴィル}}\label{ux30bdux30fcux30b9ux30b8ux30e7ux30efux30f3ux30f4ux30a3ux30eb61}}

\hypertarget{sauce-joinville}{%
\paragraph{Sauce Joinville}\label{sauce-joinville}}

\index{しよわんういる@ジョワンヴィル!そーす@ソース・---}
\index{そーす@ソース!しよわんういる@---・ジョワンヴィル}
\index{sauce@sauce!joinville@--- Joinville}
\index{joinville@Joinville!sauce@Sauce ---}

\protect\hyperlink{sauce-normande}{ノルマンディ風ソース}1
Lを、仕上げる直前の段階まで作 る\footnote{すなわち、布で漉すところまで。}。\protect\hyperlink{}{エクルヴィスバター}60
gと\protect\hyperlink{}{クルヴェットバター}60 gを加 えて仕上げる。

このソースを添える魚料理にガルニチュールが既にある場合は、これ以上は何も加えない。

ガルニチュールを伴なわない大きな魚のブイイ\footnote{魚の場合は、クールブイヨンを用いてやや低めの温度で煮たもの。}に添える場合には、細さ1〜
2mmの千切りにした真黒なトリュフを大さじ2杯加えること。

\hypertarget{ux539fux6ce8-11}{%
\subparagraph{【原注】}\label{ux539fux6ce8-11}}

同様のソースはいろいろあるが、最後の仕上げにエクルヴィスバターとクル
ヴェットバターを組み合わせて加える点がソース・ジョワンビルが他のものと
違うポイント。

\hypertarget{ux30bdux30fcux30b9ux30e9ux30aeux30d4ux30a8ux30fcux30eb62}{%
\subsubsection[ソース・ラギピエール]{\texorpdfstring{ソース・ラギピエール\footnote{18世紀末〜19世紀初頭にかけて活躍したフランスを代表する料理人の
  名(?〜1812)。はじめコンデ公に仕え、革命時にコンデ公の亡命にも随
  行したが、後にフランスに帰国し、ナポレオン\ruby{麾下}{きか}に入っ
  た。ナポレオン自身は食に無頓着であったが、直接的にはミュラ元帥のも
  とで料理長として活躍した。タレーランに仕えていたアントナン・カレー
  ムは2年程の期間であったが、ラギピエールとともに宴席の仕事に携わり、
  生涯を通して師と仰ぐ程に尊敬してやまなかった。当然だが料理において
  カレームはラギピエールから大きく影響を受け、そのことを後年、数冊の
  自著で明記している。ラギピエール自身はミュラ元帥に従ってロシア戦線
  に赴き、その撤退の途中、極寒の地で凍死した。カレームは1828年刊『パ
  リ風の料理』の冒頭2ページを「ラギピエールの想い出に」と題し、とて
  も力強い文体でその死を悼んだ。}}{ソース・ラギピエール}}\label{ux30bdux30fcux30b9ux30e9ux30aeux30d4ux30a8ux30fcux30eb62}}

\hypertarget{sauce-laguipiere}{%
\paragraph{Sauce Laguipière}\label{sauce-laguipiere}}

上述のとおりに作った\protect\hyperlink{sauce-au-beurre}{ソース・オ・ブール}
1 Lに、レモ
ン1個の搾り汁と\protect\hyperlink{glace-de-poisson}{魚のグラス}またはそれと同等に煮詰め
た\protect\hyperlink{fumet-de-poisson}{魚のフュメ}大さじ4杯を加える。

このソースは魚のブイイに添える。

\hypertarget{ux539fux6ce8-12}{%
\subparagraph{【原注】}\label{ux539fux6ce8-12}}

カレームが考案したこのソースのレシピに、本書で加えた変更点はただ1箇所
のみ、\protect\hyperlink{glace-de-volaille}{鶏のグラス}ではなく魚のグラスに代えたことだ
けだ。さらに言うと、このソースはカレームによって「ソース・オ・ブール 
ラギピエール風」と名付けられたものだ\footnote{カレームの未完の大著『19世紀フランス料理』第3巻に、このソースの
  レシピが掲載されている。少し長くなるが引用すると「ラグー用片手鍋に、\textbf{
  魚料理用グランドソース}の章で示したソース・オ・ブールをレードル1杯入
  れる。ここに上等のコンソメ大さじ1杯か鶏のグラス少々を加える。塩1つまみ、
  ナツメグ少々、良質のヴィネガーまたはレモン果汁適量を加える。数秒間煮立
  たせ、上等なバターをたっぷり加えてから供する。(中略)ソースに火を通して
  からバターを加えるというこの方法によって、なめらかな口あたりで、油っぽ
  くならない仕上りになる。だからこそ私はこのソース・オ・ブールをグランド
  ソースに分類しなかったのだし、バターを加える派生ソースにおいてこれは重
  要なことだからだ。それは魚料理用ソースについても同様のことだ
  (pp.117-118)」。このレシピにおいて、カレームの表現には矛盾がある。「魚
  用グランドソースの章で示した」とあるのに「グランドソースに分類しなかっ
  た」となっていることだ。実際、ソース・オ・ブールそれ自体はこの「ラギピ
  エール風」の直前にある。さて、このソースが「ラギピエール風」であること
  の理由だが、同じ巻の「魚料理用ソース・エスパニョル」の説明の冒頭におい
  て、ラギピエールから聞いた話として、四旬節の期間(小斉=肉断ちをする慣
  習がカトリックに根強くあった)に、魚料理用のソースにコンソメや仔牛のブ
  ロンドのジュを混ぜている修道士料理人がいたの、と述べている。それなら美
  味しくて当然だろう、とカレームが問うと、ラギピエールは「そうやって作っ
  た料理は、通常の肉を食べていい時の料理とは違うものであり、かといって肉
  断ちの料理でもない、まさに中間のものだ。その判定は天のみぞ知るところだ
  ろう。結局のところ、修道士たちは元気に暮していたのだから、それは正しかっ
  たのだよ」と煙に巻いたという。カトリックの習慣としての小斉=肉断ちのた
  めの魚料理用ソースに、肉由来である鶏のグラスもしくはコンソメを加えると
  いうところが、ラギピエール風と名付けた\ruby{所以}{ゆえん}であり、まさ
  にこれこそがソース・ラギピエールの重要なポイントと考えられる。『料理の
  手引き』においてこのレシピを担当した執筆者はこのエピソードを読んでいな
  かったのだろうか?
  あるいは何らかの誤解ゆえに改変をしたのか、ラギピエー
  ル風の\ruby{所以}{ゆえん}である鶏のグラス、コンソメを用いるべきところ
  を、魚のグラスに代えてしまい、このソース名の由来を換骨奪胎してしまう結
  果となっている。本書の初版において、原注がその文体から、エスコフィエの
  手になるものか、あるいは聞き書きしたコメントであることはほぼ明らかなの
  で、なぜエスコフィエがこの点を見逃したか、あるいは許容したのかは非常に
  興味深い。ところで、カレームが、バターを仕上げの際に加えるということ、
  いわゆるブールモンテmonter au beurreによってソースの口あたりをなめらか
  なものにし、色艶をよくするということをことさらに言及しているの点もまた、
  注目に値すべきだろう。}。

\maeaki

\hypertarget{ux30eaux30f4ux30a9ux30cbux30a264ux98a8ux30bdux30fcux30b9}{%
\subsubsection[リヴォニア風ソース]{\texorpdfstring{リヴォニア\footnote{現在のラトビア東北部からエストニア南部にかけての古い地域名、い
  わゆるバルト三国の一地域と捉えていい。本書執筆時にはロシア帝国の一
  部となっていた。なお、料理名に冠される地名のうちの少からずのものに
  明確な由来のないのと同様に、このソースについても名称の由来は不明。}風ソース}{リヴォニア風ソース}}\label{ux30eaux30f4ux30a9ux30cbux30a264ux98a8ux30bdux30fcux30b9}}

\hypertarget{sauce-livonienne}{%
\paragraph{Sauce Livonienne}\label{sauce-livonienne}}

\index{そーす@ソース!りうおにあふう@リヴォニア風---}
\index{りうおにあふう@リヴォニア風!そーす@---ソース}
\index{sauce@sauce!livonienne@---  Livonienne}
\index{livonien@livonien!sauce@Sauce Livonienne}

バターを加えて仕上げた\footnote{monter au beurre バターでモンテする。}\protect\hyperlink{veloute-de-poisson}{魚のフュメで作ったヴル
テ}1 Lに、1〜2mmの細さで長さ3〜4cmの千切り\footnote{julienne
  ジュリエンヌ}に
したにんじん、セロリ、マッシュリューム、玉ねぎをあらかじめバターを加え
て弱火で蒸し煮\footnote{étuver au beurre バターでエチュヴェする。}したおいたもの100
gを加える。最後に、1〜2mmの細さの
トリュフの千切りと粗く刻んだパセリを加える。\ldots{}\ldots{}味を調えること。

\ldots{}\ldots{}このソースは、トラウト、サーモン、舌びらめ、チュルボタン\footnote{turbotin
  \textless{} turbo チュルボ。鰈の近縁種。}、バルビュ\footnote{barbue
  鰈の近縁種。}のような魚によく合う。

\maeaki

\hypertarget{ux30deux30ebux30bfux98a870ux30bdux30fcux30b9}{%
\subsubsection[マルタ風ソース]{\texorpdfstring{マルタ風\footnote{シチリアの南方に位置するマルタ島を中心とした国、マルタはオレン
  ジをはじめとした柑橘類の産地であり、とりわけ19世紀にはマルタ産のブ
  ラッドオレンジが人気であった。一例としてバルザックの小説『二人の若
  妻の手記』において、つわりに苦しむ妻のために夫がマルセイユの街で
  「マルタ産、ポルトガル産、コルシカ産のオレンジを買い求めた」
  (p.312)と書かれている。}ソース}{マルタ風ソース}}\label{ux30deux30ebux30bfux98a870ux30bdux30fcux30b9}}

\hypertarget{sauce-maltaise}{%
\paragraph{Sauce Maltaise}\label{sauce-maltaise}}

\index{そーす@ソース!まるたふう@マルタ風---}
\index{まるたふう@マルタ風!そーす@---ソース}
\index{maltais@maltais!sauce@Sauce Maltaise}
\index{sauce@sauce!maltaise@--- Maltaise}

前述のとおりに、\protect\hyperlink{sauce-hollandaise}{ソース・オランデーズ}を作り、提供
直前に、\textbf{ブラッドオレンジ}2個の搾り汁を加える。ブラッドオレンジを用
いないとこのソースは成立しないので注意。オレンジの皮の表面をおろしたもの\footnote{zeste
  ゼスト。} 1つまみを加えて仕上げる。

\ldots{}\ldots{}アスパラガスに添える。

\maeaki

\hypertarget{ux30bdux30fcux30b9ux30deux30eaux30cbux30a8ux30fcux30eb72}{%
\subsubsection[ソース・マリニエール]{\texorpdfstring{ソース・マリニエール\footnote{marinier
  / marinière \textless{} mare ラテン語「海」から派生した語。貝や
  魚を白ワインで煮た料理にも付けられる名称。}}{ソース・マリニエール}}\label{ux30bdux30fcux30b9ux30deux30eaux30cbux30a8ux30fcux30eb72}}

\hypertarget{sauce-mariniere}{%
\paragraph{Sauce Marinière}\label{sauce-mariniere}}

\index{まりにえーる@マリニエール!そーす@ソース・---}
\index{そーす@ソース!まりにえーる@---・マリニエール}
\index{sauce@sauce!mariniere@--- Marini\`ere}
\index{mariniere@marini\`ere!sauce@Sauce ---}

\protect\hyperlink{sauce-bercy}{ソース・ベルシー}を本書で示したとおりの分量で用意する。
これにムール貝の茹で汁を詰めたもの大さじ3〜4杯を加え、卵黄6個でとろみ
を付ける\footnote{卵黄でとろみ付けをする場合、よく混ぜてさえいれば、必ずしも弱火
  でなくても問題ない。ただし、沸騰状態だと滑かに仕上がらないリスクが
  残るので、ある程度は弱火にした方がいいだろう。}。

\ldots{}\ldots{}ムール貝の料理専用。

\maeaki

\hypertarget{ux767dux3044ux30bdux30fcux30b9ux30deux30c8ux30edux30c3ux30c874}{%
\subsubsection[白いソース・マトロット]{\texorpdfstring{白いソース・マトロット\footnote{水夫風、船員風、の意。}}{白いソース・マトロット}}\label{ux767dux3044ux30bdux30fcux30b9ux30deux30c8ux30edux30c3ux30c874}}

\hypertarget{sauce-matelote-blanche}{%
\paragraph{Sauce Matelote blanche}\label{sauce-matelote-blanche}}

\index{そーす@ソース!まとろつとしろ@---・マトロット(白)}
\index{まとろつと@マトロット!そーす@ソース・---(白)}
\index{sauce@sauce!matelote blonche@--- Matelote blanche}
\index{matelote@matelote!sauce@Sauce --- blanche}

白ワインで作った魚のクールブイヨン3 dlにフレッシュなマッシュルームの切
りくず\footnote{料理、ガルニチュールとして供するマッシュルームは、トゥルネといっ
  て\ruby{螺旋}{らせん}状に切り込みを入れて装飾するのが一般的。その
  下ごしらえの際に大量のマッシュルームの切りくずが出るので、それを利
  用する。}25 gを加えて\untiers{}量まで煮詰める。

\protect\hyperlink{veloute-de-poisson}{魚料理用ヴルテ}8
dlを加える。数分間煮立たせる。 布で漉し、バター150 gを加える。

カイエンヌ\footnote{cayenne
  唐辛子の1品種。日本で一般的なカエンペッパーよりは辛さが
  マイルドで風味も異なる。}ごく少量で風味を引き締める。

ガルニチュールとして、下茹でしてからバターで色艶よく炒めた\footnote{glacer
  au beurre グラセオブール。バターでグラセする、と表現する
  調理現場も多い。glace グラス(鏡)が語源であるため、本来は「光沢を
  出させる、照りをつける」の意だが、食材や料理によってその手法はさま
  ざま。にんじんや小玉ねぎの場合にはあらかじめ下茹でしておく必要があ
  る。}小玉ねぎ20個
と、あらかじめ茹でておいた小さな白いマッシュルーム\footnote{これを用意している段階で、上述のトゥルネを行なう。常識的なこと
  として明記されていないことに注意。この作業の結果、ソースを作る際に
  魚の茹で汁(クールブイヨン)に加えるマッシュルームの切りくずが発生す
  ることになる。}20個を加える。

\maeaki

\hypertarget{ux30bdux30fcux30b9ux30e2ux30ebux30cdux30fc79}{%
\subsubsection[ソース・モルネー]{\texorpdfstring{ソース・モルネー\footnote{19世紀中頃にパリのレストラン、デュランの料理長ジョゼフ・ヴォワ
  ロンが創案したと言われている。モルネーは人名だが、具体的に誰を指し
  ているかについては諸説ある。}}{ソース・モルネー}}\label{ux30bdux30fcux30b9ux30e2ux30ebux30cdux30fc79}}

\hypertarget{sauce-mornay}{%
\paragraph{Sauce Mornay}\label{sauce-mornay}}

\index{もるねー@モルネー!そーす@---ソース}
\index{そーす@ソース!もるねー@---・モルネー}
\index{sauce@sauce!mornay@--- Mornay}
\index{mornay@Mornay!sauce@Sauce ---}

\protect\hyperlink{sauce-bechamel}{ベシャメルソース}1
Lに、このソースを合わせる魚の茹で汁 2
dlを加え、\deuxtiers{}量程に煮詰める\footnote{初版ではこの煮詰める作業はなく「固めに作ったベシャメルソース1
  L に対し、魚の茹で汁2 dlを加える」となっている。}。おろした\footnote{râper
  ラペ \textless{} râpe ラプという器具を用いておろすこと。パルメザン
  (パルミジャーノ)は硬質チーズなので一般的な半筒形のチーズおろし器
  でいいが、グリュイエールは比較的軟質なので、より目の粗い器具(例え
  ばマンドリーヌに付属している機能のうち、にんじんをおろす際に使う部
  分など)を用いるといい。}グリュイエー ルチーズ50 gとパルメザンチーズ50
gを加える。少しの間、火にかけたままに
してよく混ぜ、チーズを完全に溶かし込む。バター100
gを加えて仕上げる\footnote{monter au beurre
  モンテオブール。バターでモンテする、と表現する ことも多い。}。

\hypertarget{ux539fux6ce8-13}{%
\subparagraph{【原注】}\label{ux539fux6ce8-13}}

魚以外の料理に合わせる場合\footnote{例えば茹でた野菜などにかけて、サラマンダー(強力な上火だけのオー
  ブンの一種)に入れて軽く焦げ目を付け、グラタンにするようなケースも
  多い。}も作り方はまったく同じだが、魚の茹で汁は加えない。

\maeaki

\hypertarget{ux30bdux30fcux30b9ux30e0ux30b9ux30eaux30fcux30cc84-ux30bdux30fcux30b9ux30b7ux30e3ux30f3ux30c6ux30a3ux30a485}{%
\subsubsection[ソース・ムスリーヌ /
ソース・シャンティイ]{\texorpdfstring{ソース・ムスリーヌ\footnote{mousseline
  \textless{} mousse ムース。-ine は「小さい」を意味する接尾辞。
  その前にLの文字が入るのは、mousseの語源がメソポタミアの都市Mossoul
  (モスリン布の生産地だった)であることによる。} /
ソース・シャンティイ\footnote{シャンティイの由来などについては\protect\hyperlink{sauce-chantilly}{ソース・シャンティイ}参照。}}{ソース・ムスリーヌ / ソース・シャンティイ}}\label{ux30bdux30fcux30b9ux30e0ux30b9ux30eaux30fcux30cc84-ux30bdux30fcux30b9ux30b7ux30e3ux30f3ux30c6ux30a3ux30a485}}

\hypertarget{sauce-mousseline}{%
\paragraph{Sauce Mousseline, dite Sauce
Chantilly}\label{sauce-mousseline}}

前述のとおりの分量と作り方で\protect\hyperlink{sauce-hollandaise}{ソース・オランデーズ}を用意する(\protect\hyperlink{sauce-hollandaise}{ソース・オランデーズ}参照)。

提供直前に、固く泡立てた生クリーム大さじ4杯\footnote{大さじ1杯=15ccという考えにとらわれないよう注意。この計量単位は
  日本で戦後普及したものに過ぎず、本書においては文字通りに「大きなス
  プーンで4杯」という大雑把な単位として考える必要がある。このソース
  の場合は「固く泡立てた生クリームを適量」と読み替えてもいいだろう。
  名称どおりに滑らかでふんわりとした口あたりに仕上げるのがポイント。}をソースに混ぜ込む。

\ldots{}\ldots{}このソースは、魚のブイイ\footnote{bouilli 茹でた、の意。}や、アスパラガス、カルドン\footnote{cardon
  アーティチョークの近縁種で、アーティチョークが開花前の蕾
  を食用とするのに対し、カルドンは軟白させた茎葉を食用とする。フラン
  スではトゥーレーヌ地方産が有名。草丈1.5m位まで成長させた株を紐で束
  ねて軟白する。厳冬期は株元から刈り取って小屋などで保管するのが伝統
  的な手法。イタリア北部ピエモンテでは株を倒してその上に土を被せて軟
  白するというユニークな方法で栽培するcardo gobboカルドゴッボもよく
  知られている。}、セロリ\footnote{セロリには緑の濃い品種系統と、やや緑が薄く、中心部が自然に軟白
  されたようになる系統がある。野菜料理として用いられるのは主として後
  者の芯に近い、自然に軟白された部分。coeur de céleri クールドセルリ
  と呼ぶ。前者については、もっぱら香味野菜としてフォンやポタージュ、
  煮込み料理などに用いられる。このタイプは風味に癖があるため、生食に
  はあまり適していない。}に添える。

\maeaki

\hypertarget{ux30bdux30fcux30b9ux30e0ux30b9ux30fcux30ba90}{%
\subsubsection[ソース・ムスーズ]{\texorpdfstring{ソース・ムスーズ\footnote{細かく泡立った、の意。なお、シャンパーニュのようなvin
  mousseux ヴァン・ムスー(発泡ワイン)のムスーは同じ語の男性形.}}{ソース・ムスーズ}}\label{ux30bdux30fcux30b9ux30e0ux30b9ux30fcux30ba90}}

\hypertarget{sauce-mousseuse}{%
\paragraph{Sauce Mousseuse}\label{sauce-mousseuse}}

沸騰した湯の中に、小さめのソテー鍋を入れて熱し、水気をよく拭き取る。こ
のソテー鍋に、あらかじめ充分に柔らかくしておいたバター500 gを入れる。
塩8 gを加え、泡立て器でしっかり混ぜながら、レモン\unquart{}個分の搾り
汁と冷水4 dlを少しずつ加える。

最後に、固く泡立てた生クリーム大さじ4杯を混ぜ込む。

このレシピは、ソースに分類してはいるが、むしろ合わせバターというべきも
のだ。魚のブイイに合わせる。

茹でた魚から伝わる熱だけでバターは充分に溶けるので、見た目も風味も溶か
しバターをソースにするよりずっといいものだ。

\maeaki

\hypertarget{ux30bdux30fcux30b9ux30e0ux30bfux30ebux30c991}{%
\subsubsection[ソース・ムタルド]{\texorpdfstring{ソース・ムタルド\footnote{マスタードのこと。マスタードソースと呼んでもいいが、アメリカ風
  の印象を与えるかも知れない。}}{ソース・ムタルド}}\label{ux30bdux30fcux30b9ux30e0ux30bfux30ebux30c991}}

\hypertarget{sauce-moutarde}{%
\paragraph{Sauce Moutarde}\label{sauce-moutarde}}

\index{そーす@ソース!むたると@---・ムタルド}
\index{むたると@ムタルド(マスタード)!そーす@ソース・ムタルド}
\index{ますたーと@マスタード(ムタルド)!そーす@ソース・ムタルド}
\index{sauce@sauce!moutarde@--- Moutarde}
\index{moutarde@moutarde!sauce@Sauce ---}

普通、このソースは提供直前に作ること。

必要の分量の\protect\hyperlink{sauce-au-beurre}{ソース・オ・ブール}を用意する。鍋を火か
ら外し、ソース2\undemi{} dlあたり大さじ1杯のマスタードを加える。

このソースを仕上げて、提供するまで時間を空けなくてはならない場合は、湯
煎にかけておく。沸騰させないよう注意すること。

\maeaki

\hypertarget{ux30bdux30fcux30b9ux30caux30f3ux30c1ux30e5ux30a292}{%
\subsubsection[ソース・ナンチュア]{\texorpdfstring{ソース・ナンチュア\footnote{ローヌ・アルプ地方にあるナンチュア湖でエクルヴィスが穫れること
  に由来したソース名。エクルヴィスについて詳しくは\protect\hyperlink{sauce-bavaroise}{バイエルン風ソー
  ス}訳注参照。}}{ソース・ナンチュア}}\label{ux30bdux30fcux30b9ux30caux30f3ux30c1ux30e5ux30a292}}

\hypertarget{sauce-nantua}{%
\paragraph{Sauce Nantua}\label{sauce-nantua}}

\protect\hyperlink{sauce-bechamel}{ベシャメルソース}1 Lに生クリーム2
dlを加え、 \deuxtiers{}量まで煮詰める。

布で漉し、生クリームをさらに1\undemi{} dl加えて、通常の濃度に戻す。

良質な\protect\hyperlink{beurre-d-ecrevisse}{エクルヴィスバター}125
gと、小さめのエクル ヴィスの尾の身\footnote{しっかり下茹でして殻を剥いたものを用いること。}20を加えて仕上げる。

\maeaki

\hypertarget{ux6d3bux3051ux30aaux30deux30fcux30ebux3067ux4f5cux308bux30bdux30fcux30b9ux30cbux30e5ux30fcux30d0ux30fcux30b095}{%
\subsubsection[活けオマールで作るソース・ニューバーグ]{\texorpdfstring{活けオマールで作るソース・ニューバーグ\footnote{ここでは英語由来のソース名のため英語風にカタカナ書きしたが、フ
  ランスでは「ニュブール」のように発音されることも多い。}}{活けオマールで作るソース・ニューバーグ}}\label{ux6d3bux3051ux30aaux30deux30fcux30ebux3067ux4f5cux308bux30bdux30fcux30b9ux30cbux30e5ux30fcux30d0ux30fcux30b095}}

\hypertarget{sauce-new-burg-avec-le-homard-cru}{%
\paragraph{Sauce New-burg avec le homard
cru}\label{sauce-new-burg-avec-le-homard-cru}}

\index{そーす@ソース!にゆーはーくいけおまーる@活けオマールを使う---・ニューバーグ}
\index{にゆーはーく@ニューバーグ!そーす@活けオマールを使うソース・---}
\index{sauce@sauce!new-burg homard cru@--- New-burg avec le homard cru}
\index{new-burg@New-burg!sauce homard cru@Sauce --- avec le homard cru}

800〜900 gのオマールを切り分ける。

胴の中のクリーム状の部分をスプーンで取り出し、これをよくすり潰して30 g
のバターを合わせ、別に取り置いておく。

バター40 gと植物油大さじ4杯を鍋に入れて熱し、切り分けたオマールの身を
色付くまで焼く。塩とカイエンヌで調味する。殻が真っ赤になったら、鍋の油
を完全に捨て、コニャック大さじ2杯と、マルサラ酒もしくはマデラの古酒2
dlを注いで火を付けてアルコール分を燃やす\footnote{flamber フランベする。}。注いだ酒が\untiers{}量
になるまで煮詰めたら、生クリーム2
dlと\protect\hyperlink{fumet-de-poisson}{魚のフュメ}2
dlを注ぐ。弱火で25分間煮る。

オマールの身をざるにあげて水気をきる。殻から身を取り出して、さいの目に
切る。

取り置いておいたオマールのクリーム状の部分をソースに混ぜ込み、完全に火
が通るように軽く煮立たせてやる。さいの目に切ったオマールの身を加えて混
ぜる。味見をして、必要なら塩を加えて修正する。

\hypertarget{ux539fux6ce8-14}{%
\subparagraph{【原注】}\label{ux539fux6ce8-14}}

さいの目に切ったオマールの身をソースに混ぜ込むのは絶対必要というわけで
はない。薄くやや斜めにスライスして、このソースを合わせる魚料理に添えて
もいい。

\maeaki

\hypertarget{ux8339ux3067ux305fux30aaux30deux30fcux30ebux3067ux4f5cux308bux30bdux30fcux30b9ux30cbux30e5ux30fcux30d0ux30fcux30b0100}{%
\subsubsection[茹でたオマールで作るソース・ニューバーグ]{\texorpdfstring{茹でたオマールで作るソース・ニューバーグ\footnote{このソースの元となった料理「オマール・ニューバーグ」は、19世
  紀後半にニューヨークのレストラン、デルモニコーズで常連客のアイデア
  をもとにフランス出身の料理長シャルル・ラノフェール(チャールズ・レ
  ンフォーファー)が完成させたと言われており、そのレシピがラノフェー
  ルの著書\href{https://archive.org/details/epicureancomplet00ranhrich}{『ジ・エピキュリア
  ン』}(英
  語)に掲載されている(p.411)。現在もデルモニコーズのスペシャリテと
  して知られている。ただし、ラノフェールのレシピは先にオマールを茹で
  るという、本項のレシピに近いものであり、前項の活けオマールを使うレシ
  ピはエスコフィエもしくは他の料理人によって改変させたものと考えられ
  る。なお、このレシピと次項のソース・ニューバーグは第二版で追加され
  たものであり、その後は原注も含めて異同がない。}}{茹でたオマールで作るソース・ニューバーグ}}\label{ux8339ux3067ux305fux30aaux30deux30fcux30ebux3067ux4f5cux308bux30bdux30fcux30b9ux30cbux30e5ux30fcux30d0ux30fcux30b0100}}

\hypertarget{sauce-new-burg-avec-le-homard-cuit}{%
\paragraph{Sauce New-burg avec le homard
cuit}\label{sauce-new-burg-avec-le-homard-cuit}}

オマールを\protect\hyperlink{}{標準的なクールブイヨン}で茹でる。尾の身を殻から外し、や
や斜めに厚さ1cm程度の筒切りにする\footnote{détailler en escalopes =
  escalopper エスカロップ(厚さ1〜2cm程度の薄切り)に切る。}。ソテー鍋の内側にたっぷりとバター
を塗り、そこに切ったオマールを並べるように入れる。塩とカイエンヌでしっ
かりと味を付け、表皮が赤く発色するように両面を焼く。上等なマデラ酒をひ
たひたの高さまで注ぎ、ほぼ完全になくなるまで煮詰める。

提供直前に、オマールのスライスの上に、生クリーム2 dlと卵黄3個を溶いた
ものを注ぎ、火から外して、ゆっくり混ぜながら\footnote{vanner
  ヴァネする。}しっかりととろみを付 ける。

\hypertarget{ux539fux6ce8-15}{%
\subparagraph{【原注】}\label{ux539fux6ce8-15}}

\protect\hyperlink{sauce-americaine}{ソース・アメリケーヌ}と同様に、これら2種のソースも
元来はオマールを供するための料理だった。ソースとオマールが、要するにひ
とつの料理を構成していたわけだ。

ところが、そのような料理は午餐(ランチ)でしか提供することが出来ない。
多くの人々は胃が弱く、夕食では消化しきれないのだ\footnote{レシピにおいて指示されているオマールが大きなものであることに注
  意。}。

そうした問題解決のために、我々はこれを、舌びらめのフィレやムスリーヌに
添えるオマールのソースとして使うことにしたのだ。オマールの身はガルニ
チュールとして添えるにとどめることにした。結果は好評であった。

カレー粉やパプリカ粉末を調味料として用いれば、このソースのとてもいいバ
リエーションが作れる。とりわけ舌びらめや脂身の少ない白身魚によく合
う。\ldots{}\ldots{}その場合、魚に少量の\protect\hyperlink{riz-a-l-indienne}{インド風ライス}を添えるといい。

\maeaki

\hypertarget{ux30bdux30fcux30b9ux30ceux30efux30bcux30c3ux30c8102}{%
\subsubsection[ソース・ノワゼット]{\texorpdfstring{ソース・ノワゼット\footnote{ヘーゼルナッツ、はしばみの実。}}{ソース・ノワゼット}}\label{ux30bdux30fcux30b9ux30ceux30efux30bcux30c3ux30c8102}}

\hypertarget{sauce-noisette}{%
\paragraph{Sauce Noisette}\label{sauce-noisette}}

\index{そーす@ソース!へーせるなっつ@---・ノワゼット}
\index{へーせるなつつ@ヘーゼルナッツ!そーす@ソース・ノワゼット}
\index{のわせつと@ノワゼット!へーぜるなっつそーす@ヘーゼルナッツソース}
\index{sauce@sauce!noisette@--- Noisette}
\index{noisette@noisette!sauce@Sauce ---}

\protect\hyperlink{sauce-hollandaise}{ソース・オランデーズ}を本書のレシピのとおりに作る。
提供直前に仕上げとして、上等なバターで作った\protect\hyperlink{}{ブール・ド・ノワゼット}75
g を加える。

\ldots{}\ldots{}ポシェ\footnote{pocher
  沸騰しない程度の温度で茹でること。魚の場合は\protect\hyperlink{}{クールブイ
  ヨン}を用いてやや低めの温度で火を通すこと。}したサーモン、トラウトにとてもよく合う。

\maeaki

\hypertarget{ux30ceux30ebux30deux30f3ux30c7ux30a3ux30fcux98a8ux30bdux30fcux30b9}{%
\subsubsection{ノルマンディー風ソース}\label{ux30ceux30ebux30deux30f3ux30c7ux30a3ux30fcux98a8ux30bdux30fcux30b9}}

\hypertarget{sauce-normande}{%
\paragraph{Sauce Normande}\label{sauce-normande}}

\index{そーす@ソース!のるまんてふう@ノルマンディ風---}
\index{のるまんていふう@ノルマンディ風!そーす@---ソース}
\index{sauce@sauce!normande@--- Normande}
\index{normande@normande!sauce@Sauce ---}

\protect\hyperlink{veloute-de-poisson}{魚料理用ヴルテ}\troisquarts{}
Lに\footnote{原書にはリットルの表記がないが、本書における標準的な仕上り量が
  1 Lであることと、文脈から訳者が補った。}、マッシュ ルームの茹で汁1
dlとムール貝の茹で汁1 dl、舌びらめのフュメ\footnote{舌びらめの料理に合わせるソースであるために、舌びらめのアラなど
  が必然的に出るのを無駄にせず使うということだが、現代のレストランの
  厨房などではかえって無理が生じることになる。このレシピの通りに作る
  場合には何らかのオペレーション上の工夫が必要だろう。} 2 dlを加える。
レモン果汁少々と、とろみ付け用に卵黄5個を生クリーム2dlで溶いたものを加
える。強火で\deuxtiers{}量つまり約8 dlまで煮詰める。

布で漉し、クレーム・ドゥーブル\footnote{乳酸醗酵した濃い生クリーム。\protect\hyperlink{sauce-supreme}{ソース・シュプレー
  ム}訳注参照。}1 dlとバター125 gを加える。

\ldots{}\ldots{}このソースは\protect\hyperlink{sole-normande}{舌びらめのノルマンディ風}専用。とはい
え、使い方によっては無限の可能性がある。

\hypertarget{ux539fux6ce8-16}{%
\subparagraph{【原注】}\label{ux539fux6ce8-16}}

基本的に本書では、どんなレシピにおいても、牡蠣の茹で汁は使わないことにし
ている。牡蠣の茹で汁は塩味がするだけで風味がない。だから、可能であればムー
ル貝の茹で汁を大さじ何杯か加えるほうがずっといい\footnote{このレシピは初版からの異同が大きい。初版では「魚料理用ヴルテ1
  Lあたり卵黄6個でとろみを付け、牡蠣の茹で汁2 dlと魚のエッセンス、生ク
  リーム2 dlを加えながら煮詰める。仕上げにバター100gとクレーム・ドゥー
  ブル1 dlを加える」となっており、用途には触れられていない。第二版、
  第三版ではやや細かなレシピとなり用途も「舌びらめのノルマンディ風」
  と指定されて現行版に近いものになるが、牡蠣の茹で汁を使うことは初版と
  同じ。つまり、第四版で牡蠣の茹で汁からムール貝の茹で汁を使うことに変更
  し、この原注が付けられた。このソースにおける改変は、前出のソース・
  ラギピエールのケースとやや似ているところもある。牡蠣を用いることか
  ら、牡蠣の産地であるノルマンディ風という名称となったソースであるの
  に、そこから牡蠣を排除するという、いわば換骨奪胎がなされているから
  だ。とはいえ、このことが、第四版の改訂にエスコフィエ自身が携わった
  という証拠のひとつともなり得る可能性はある。初版刊行時56才、1921年
  刊の第四版の改訂にあたった頃には70才を過ぎていたことになり、味覚や
  嗅覚における感受性に変化があった可能性も考えられる。第三版までは牡
  蠣の茹で汁を指定したいたのに、第一次大戦後、食料事情の変化があったと
  はいえ、きわめて風味の強いムール貝の茹で汁を使うことを第四版で唐突に
  推奨しているということからは、まったくの第三者による改竄か、改訂者
  本人の身体的、感覚的もしくは思想的な変化がうかがわれる。その意味で
  も、やはりエスコフィエ自身が改訂作業に真摯に取り組んだ結果として、
  このレシピの変遷を捉えるべきだろう。}。

\maeaki

\hypertarget{ux30aaux30eaux30a8ux30f3ux30c8ux98a8ux30bdux30fcux30b9108}{%
\subsubsection[オリエント風ソース]{\texorpdfstring{オリエント風ソース\footnote{フランス語の
  orient オリヨン(東方)は、具体的にいうと北アフリ
  カの一部、アラビア半島、西アジアくらいまでを指すのが一般的。その意
  味では、カレー粉を加えたことで「オリエント風」と称するのは、当時の
  フランス人にとって、理解できなくもないだろうが実感は伴わなかった可
  能性がある。フランス人にとっての「オリエント」である北アフリカやト
  ルコといった地域の食文化は19世紀に既にかなりフランスに伝わっていた
  からだ。つまりは、ロンドンのカールトンホテルとパリのオテルリッツの
  それぞれで、もし仮にこのソースを添えた料理の名をメニューで見たとき、
  食べ手すなわち客が受ける印象はかなり異なる可能性が高い。もちろん、
  これらのホテルがインターナショナルな社交の場として機能していことを
  考慮に入れても、同じ料理名がイメージさせる内容には確実にずれが生じ
  ると考えるのが妥当だろう。こういった文化的なイメージのずれは、エス
  コフィエ本人が料理長としてのキャリアの大半をイギリスで過ごしたこと
  とも関係があると思われる。つまり、フランス人にとっての「オリエント」
  とインドという植民地を持つイギリス人の「オリエント」は同じ言葉であっ
  ても、想起される具体的な内容が違うということである。ちなみに、イン
  ドより東の日本などはextrème orientエクストレーモリヨン(極東)と呼
  ばれ、1900年のパリ万博において川上音二郎一座の公演が好評を博すなど、
  19世紀末から20世紀初頭にかけてjaponismeジャポニスムが文化的流行と
  なた。ゴッホやセザンヌ、マネの絵画における浮世絵の影響は有名。少な
  くとも当時のフランスにおいては、極東の日本と「オリエント」のイメー
  ジの混同はなかったと言えるが、一般的には日本と中国と違いは曖昧な認
  識のままだった。}}{オリエント風ソース}}\label{ux30aaux30eaux30a8ux30f3ux30c8ux98a8ux30bdux30fcux30b9108}}

\hypertarget{sauce-orientale}{%
\paragraph{Sauce Orientale}\label{sauce-orientale}}

\index{そーす@ソース!おりえんとふう@オリエント風---}
\index{おりえんとふう@オリエント風!そーす@---ソース}
\index{とうほうふう@東方風!おりえんたるそーす@オリエント風ソース}
\index{sauce@sauce!orientale@--- Orientale}
\index{oriental@oriental!sauce@Sauce Orientale}

\protect\hyperlink{sauce-americaine}{ソース・アメリケーヌ}\undemi{}
Lを用意し、カレー粉
で風味付けをして\deuxtiers{}量まで煮詰める。鍋を火から外し、生クリーム
1\undemi{} dlを混ぜ込む。

\ldots{}\ldots{}このソースの用途は\protect\hyperlink{sauce-americaine}{ソース・アメリケーヌ}と同じ。

\maeaki

\hypertarget{ux30ddux30fcux98a8ux30bdux30fcux30b9}{%
\subsubsection{ポー風ソース}\label{ux30ddux30fcux98a8ux30bdux30fcux30b9}}

\hypertarget{sauce-paloise110}{%
\paragraph[Sauce paloise]{\texorpdfstring{Sauce paloise\footnote{ポーは15世紀以来、ベアルヌ地方の中心都市。}}{Sauce paloise}}\label{sauce-paloise110}}

\index{そーす@ソース!ほーふう@ポー風---}
\index{ほーふう@ポー風!そーす@---ソース}
\index{sauce@sauce!paloise@--- Paloise}
\index{palois@palois!sauce@Sauce Paloise}
\index{pau@Pau!sauce paloise@Sauce Paloise}

\protect\hyperlink{sauce-bearnaise}{ソース・ベアルネーズ}を本書に書いてあるとおりの方法
と分量で用意する(\protect\hyperlink{sauce-bearnaise}{ソース・ベアルネーズ}参照)が、以
下の点を変える。

\begin{enumerate}
\def\labelenumi{\arabic{enumi}.}
\item
  香りの中心となるエストラゴンを同量のミント\footnote{フランス料理よりはむしろイギリス料理でよく使われるミントを用い
    たこのソースをポー風と呼ぶのは、かつてこの地がイギリス貴族たちに保
    養地として好まれたことにちなんでいるという説もある。}に変更し、白ワインとヴィネガーを煮詰める際に加える。
\item
  さらに、仕上げの際に、細かく刻んだエストラゴンも使わない。細かく刻んだミントを使う。
\end{enumerate}

\ldots{}\ldots{}このソースの用途はソース・ベアルネーズとまったく同じ。

\maeaki

\hypertarget{ux30bdux30fcux30b9ux30d7ux30ecux30c3ux30c8109}{%
\subsubsection[ソース・プレット]{\texorpdfstring{ソース・プレット\footnote{ひな鶏、の意。かつて鶏のフリカセがこのソースと同様の作り方であっ
  たためこの名称になったという説もある。ちなみに、「鶏のフリカセ」と
  して文献上もっとも古いもののひとつ、ラ・ヴァレーヌ『フランス料理の
  本』(1651年)のレシピでは、掃除をして切り分けた鶏をブイヨンで完全
  に火が通るまで煮た後に、鶏の水気をきってフライパンに油脂を熱してこ
  んがりと焼き(レシピには明記されていないがブイヨンあるいは他の液体
  を注ぎ)、パセリやシブール(葱の一種)を加えて調味し、溶いた卵黄で
  とろみを付ける、というものだった(p.47)。確かに、ソース・アルマンド
  も卵黄をとろみ付けに使うのが特徴である点から、その類推でこのソース
  名になった可能性はあるだろう。だが、本書においては仔牛のブランケッ
  トも仔牛のフリカセも最後のとろみ付けに卵黄を用いているので、それを
  「ひな鶏」というところに限定するのはいささか疑問が残る。なお、
  fricasserフリカセという語は17世紀頃まで、「油脂を熱したフライパン
  などでこんがり焼く」の意味で用いられていた。それがこんにちのような
  「煮込み」に変化したのは、17世紀におけるragoûtラグーの流行に負うと
  ころは大きいだろう。ラグーとは、もとは、食欲をそそるもの、の意であ
  り、それまでポタージュと総称されていた煮込み料理全般およびソースと
  主素材が一体化したものの一部について、17世紀に付けられるようになっ
  た、一種の流行語であった。}}{ソース・プレット}}\label{ux30bdux30fcux30b9ux30d7ux30ecux30c3ux30c8109}}

\hypertarget{sauce-poulette}{%
\paragraph{Sauce Poulette}\label{sauce-poulette}}

\index{そーす@ソース!ふれっと@---・プレット}
\index{ふれっと@プレット!そーす@ソース・---}
\index{sauce@sauce!poulette@--- Poulette}
\index{poulette@poulette!sauce@Sauce ---}

マッシュルームの茹で汁2
dlを\untiers{}量まで煮詰める。ここに\protect\hyperlink{sauce-allemande}{ソース・
アルマンド}1 Lを加え、数分間沸騰させる。鍋を火から外
し、レモン果汁少々とバター60g、パセリのみじん切り大さじ1杯を加えて仕上
げる。

\ldots{}\ldots{}このソースは野菜料理に合わせるが、羊の足の料理にもよく合う。

\maeaki

\hypertarget{ux30bdux30fcux30b9ux30e9ux30f4ux30a3ux30b4ux30c3ux30c8114}{%
\subsubsection[ソース・ラヴィゴット]{\texorpdfstring{ソース・ラヴィゴット\footnote{ravigote
  \textless{} ravigoter 身体を丈夫にする、元気にさせる、の派生語。
  香草を主体として酸味を効かせたソース(および煮込み料理)は中世以来
  あったが、18世紀以降ravigoteという呼び名が一般的となり、19世紀以降
  はこの表現がしばしば使われるようになった。ソース・ラヴィゴットは冷
  製と温製の2種があるが、日本では冷製の方がよく知られている。なお、
  ソース・ラヴィゴットのレシピとして最初期のもののひとつ、1755年刊ム
  ノン『宮廷の晩餐』第1巻に掲載されているソース・ラヴィゴットの作り
  方は、薄切りにしたにんにく、セルフイユ、サラダバーネット、エストラ
  ゴン、クレソンアレノワ(オルレアン芹)、シブレットを洗ってから圧し
  潰し、コップ1杯のコンソメ(=この当時のコンソメはグラスドヴィアンド
  に近いものであることに注意)に入れて沸騰させないよう1時間以上かけ
  て煎じる。漉し器で押すようにして漉し、ブールマニエ、塩、こしょうで
  味付けをして火にかけ、レモンの搾り汁で仕上げる、というもの(p.135)。}}{ソース・ラヴィゴット}}\label{ux30bdux30fcux30b9ux30e9ux30f4ux30a3ux30b4ux30c3ux30c8114}}

\hypertarget{sauce-ravigote}{%
\paragraph{Sauce Ravigote}\label{sauce-ravigote}}

\index{そーす@ソース!らういこつと@---・ラヴィゴット}
\index{らういこつと@ラヴィゴット!そーす@ソース・---}
\index{sauce@sauce!ravigote@--- Ravigote}
\index{ravigote@ravigote!sauce@Sauce ---}

白ワイン1\undemi{} dlとヴィネガー1\undemi{} dlを半量になるまで煮詰める。
\protect\hyperlink{veloute}{標準的なヴルテ}8
dlを加え、数分間煮立たせる。鍋を火から外し、
\protect\hyperlink{}{エシャロットバター}90〜100
gと、セルフイユ\footnote{cerfeuil チャービル。}とエストラゴン\footnote{estragon
  フレンチタラゴン。詳しくは\protect\hyperlink{sauce-chasseur}{ソース・シャスー
  ル}訳注参照。}、シブレッ ト\footnote{ciboulette
  日本ではチャイブとも呼ばれる。アサツキと訳されるこ
  ともあるが、風味がまったく異なるので代用は不可。春に紫色の小さくて
  きれいな花をたくさん咲かせるので、エディブルフラワーとしてもよく用
  いられる。}を細かく刻んだものを同量ずつ合わせたもの計大さじ1\undemi{}杯を加えて
仕上げる。

\ldots{}\ldots{}茹でた鶏に合わせる。白い内臓\footnote{家畜の副生物すなわち正肉以外の部分のうち、内臓をabatsアバと呼
  ぶ。そのうちの、心臓、レバー、舌などはabats rougeアバルージュ(赤
  い内臓)、耳、尾、胃、腸、足、頭、仔牛および仔羊の胸腺肉(ris de
  veauリドヴォー、ris d'agneauリダニョー)や腸間膜(fraiseフレーズ)
  などはabats blancアバブロン(白い内臓、白い副生物)を呼ばれている。
  こうした副生物の料理は古くから好まれ、16世紀フランソワ・ラブレー
  『ガルガンチュアとパンタグリュエル』においてもしばしば登場する。と
  りわけ「ガルガンチュア」の冒頭では、出産間近なお妃が臓物料理を食べ
  過ぎるなどというエピソードが印象深い。なお、鶏の副生物(とさか、内
  臓、脚など)はabattisアバティと呼ばれるので混同しないよう注意。}料理にも合わせることがある。

\maeaki

\hypertarget{ux9b5aux6599ux7406ux304aux3088ux3073ux9b5aux3067ux69cbux6210ux3057ux305fux30acux30ebux30cbux30c1ux30e5ux30fcux30ebux7528ux30bdux30fcux30b9ux30ecux30b8ux30e3ux30f3ux30b9119}{%
\subsubsection[魚料理および魚で構成したガルニチュール用ソース・レジャンス]{\texorpdfstring{魚料理および魚で構成したガルニチュール用ソース・レジャンス\footnote{ソース・レジャンスという名称については「ブラウン系の派生ソース」の
  \protect\hyperlink{sauce-regence}{ソース・レジャンス}訳注参照。}}{魚料理および魚で構成したガルニチュール用ソース・レジャンス}}\label{ux9b5aux6599ux7406ux304aux3088ux3073ux9b5aux3067ux69cbux6210ux3057ux305fux30acux30ebux30cbux30c1ux30e5ux30fcux30ebux7528ux30bdux30fcux30b9ux30ecux30b8ux30e3ux30f3ux30b9119}}

\hypertarget{sauce-ruxe9gence-pour-poissons-et-garnitures-de-poissons}{%
\paragraph{Sauce Régence pour Poissons, et garnitures de
Poissons}\label{sauce-ruxe9gence-pour-poissons-et-garnitures-de-poissons}}

\index{そーす@ソース!れしやんすさかなよう@魚料理用---・レジャンス}
\index{れじゃんす@レジャンス!そーすさかなよう@魚料理用ソース・---}
\index{sauce@sauce!regence poissons@--- Régence pour Poissons}
\index{regence@Régence!sauce poissons@Sauce --- pour Poissons}

ライン産ワイン2 dlと\protect\hyperlink{fumet-de-poisson}{魚のフォン}2
dlに新鮮なマッシュ
ルームの切りくず20gと生トリュフの切りくず20gを加えて半量になるまで煮詰
める。

煮詰まったら布で漉し、仕上げた状態の\protect\hyperlink{sauce-normande}{ノルマンディ風ソース}8
dlを加える。

トリュフエッセンス大さじ1杯を加えて仕上げる。

\maeaki

\hypertarget{ux9d8fux3067ux69cbux6210ux3055ux308cux305fux30acux30ebux30cbux30c1ux30e5ux30fcux30ebux7528120ux30bdux30fcux30b9ux30ecux30b8ux30e3ux30f3ux30b9}{%
\subsubsection[鶏で構成されたガルニチュール用ソース・レジャンス]{\texorpdfstring{鶏で構成されたガルニチュール用\footnote{わかりやすい例としては、後述の\protect\hyperlink{garniture-regence}{ガルニチュール・レジャンス
  B}参照。}ソース・レジャンス}{鶏で構成されたガルニチュール用ソース・レジャンス}}\label{ux9d8fux3067ux69cbux6210ux3055ux308cux305fux30acux30ebux30cbux30c1ux30e5ux30fcux30ebux7528120ux30bdux30fcux30b9ux30ecux30b8ux30e3ux30f3ux30b9}}

\hypertarget{sauce-ruxe9gence-pour-garnitures-de-volaille}{%
\paragraph{Sauce Régence pour garnitures de
Volaille}\label{sauce-ruxe9gence-pour-garnitures-de-volaille}}

\index{そーす@ソース!れしやんすとりののかるにてゆーるよう@鶏で構成されたガルニチュール用---・レジャンス}
\index{れしやんす@レジャンス!そーすとりのがるにてゅーるよう@鶏で構成されたガルニチュール用ソース・---}
\index{sauce@sauce!regence garnitures de volaille@--- Régence pour garnitures de Volaille}
\index{regence@Régence!sauce garnitures de volaille@Sauce --- pour garnitures de Volaille}

ライン産白ワイン2 dlとマッシュルームの茹で汁2 dlにトリュフの切りくず
40gを加え、半量になるまで煮詰める。

\protect\hyperlink{sauce-allemande}{ソース・アルマンド}8
dlを加え、布で漉す。トリュフエッ センス大さじ1杯を加えて仕上げる。

\hypertarget{ux30bdux30fcux30b9ux30eaux30c3ux30b7ux30e5}{%
\subsubsection{ソース・リッシュ}\label{ux30bdux30fcux30b9ux30eaux30c3ux30b7ux30e5}}

\hypertarget{sauce-riche121}{%
\paragraph[Sauce Riche]{\texorpdfstring{Sauce Riche\footnote{リッチな、裕福な、の意。ソース・ディプロマットがそもそも豪華な
  料理に合わせるものであり、さらにトリュフを足すことでより一層「リッ
  チ」なものにした、ということ。}}{Sauce Riche}}\label{sauce-riche121}}

\index{そーす@ソース!りつしゆ@---・リッシュ}
\index{りつしゆ@リッシュ!そーす@ソース・---}
\index{sauce@sauce!riche@--- Riche} \index{riche@Riche!sauce@Sauce ---}

\protect\hyperlink{sauce-diplomate}{ソース・ディプロマット}を本書で示したとおりの分量と
作り方で用意する。

トリュフエッセンス1
dlと、さいの目に切った真黒なトリュフ80gを加えて仕上げる。

\maeaki

\hypertarget{ux30bdux30fcux30b9ux30ebux30fcux30d9ux30f3ux30b9}{%
\subsubsection{ソース・ルーベンス}\label{ux30bdux30fcux30b9ux30ebux30fcux30d9ux30f3ux30b9}}

\hypertarget{sauce-rubens122}{%
\paragraph[Sauce Rubens]{\texorpdfstring{Sauce Rubens\footnote{フランドル派の画家、Peter
  Paul Rubens ピーテル・パウル・ルーベ
  ンス(1577〜1640)のこと。フランス語では古くから Pierre Paul Rubens
  ピエール・ポール・リュベンスの表記が慣例となっているが、現
  代フランス語では原語のままの綴り、発音を尊重する潮流にある。}}{Sauce Rubens}}\label{sauce-rubens122}}

\index{そーす@ソース!るーへんす@---・ルーベンス}
\index{るーへんす@ルーベンス!そーす@ソース・---}
\index{rubens@Rubens!sauce@Sauce ---}
\index{sauce@sauce!rubens@--- Rubens}

1〜2 mm角の小さなさいの目\footnote{brunoise ブリュノワーズ}に切った\protect\hyperlink{}{標準的なミルポワ}100
gをバ ターで色付くまで炒める。白ワイン2
dlと\protect\hyperlink{fumetux5cux2520deux5cux2520poisson}{魚のフュメ}3
dlを注ぎ、25分間火にかけておく。

目の細かいシノワ\footnote{円錐形で取っ手の付いた漉し器。}で漉す。数分間静かに休ませてから、浮いてきた油脂
を丁寧に取り除く\footnote{dégraisser デグレセ。}。\undemi{}
dlになるまで煮詰め、マデラ酒大さじ1 杯を加える。

ここに卵黄2個を加えてとろみを付け、普通のバター100
gと\protect\hyperlink{beurre-rouge}{ブール・ルー ジュ}30
g、アンチョビエッセンス少々を加えて仕上げる。

\ldots{}\ldots{}魚のブイイすなわちポシェした魚にこのソースはとてもよく合う。

\maeaki

\hypertarget{ux30b5ux30f3ux30deux30ed126ux98a8ux30bdux30fcux30b9}{%
\subsubsection[サンマロ風ソース]{\texorpdfstring{サンマロ\footnote{ブルターニュ地方の港町。観光地として有名であり、バカンスシー
  ズンには多くの人が訪れる。}風ソース}{サンマロ風ソース}}\label{ux30b5ux30f3ux30deux30ed126ux98a8ux30bdux30fcux30b9}}

\hypertarget{sauce-saint-malo}{%
\paragraph{Sauce Saint-Malo}\label{sauce-saint-malo}}

\index{そーす@ソース!さんまろふう@サンマロ風---}
\index{さんまろふう@サンマロ風!そーす@---ソース}
\index{sauce@sauce!saint-malo@--- Saint-Malo}
\index{saint-malo@Saint-Malo!sauce@Sauce ---}

(仕上り5 dl分)

本書で示したとおりに作った\protect\hyperlink{sauce-vin-blanc}{白ワインソース}\undemi{}
Lに細かく刻んで白ワインで茹でたエシャロット大さじ1杯、もしくは、可能な
ら、\protect\hyperlink{beurre-d-echalote}{エシャロットバター}50
gと、マスタード大さじ \undemi{}杯、アンチョビエッセンス少々を加える。

\ldots{}\ldots{}海水魚のグリルに合わせる。

\maeaki

\hypertarget{ux30bdux30fcux30b9ux30b9ux30dfux30bfux30fcux30cc127}{%
\subsubsection[ソース・スミターヌ]{\texorpdfstring{ソース・スミターヌ\footnote{サワークリームを意味するロシア語
  \ltjsetparameter{jacharrange={-2}} Сметана
  \ltjsetparameter{jacharrange={+2}}スメタナが由来。
  ロシア料理とフランス料理との相互影響関係にいては、\protect\hyperlink{service-russe}{序
  p.II 訳注
  3}および\protect\hyperlink{sauce-moscovite}{モスクワ風ソース}訳注参
  照。}}{ソース・スミターヌ}}\label{ux30bdux30fcux30b9ux30b9ux30dfux30bfux30fcux30cc127}}

\hypertarget{sauce-smitane}{%
\paragraph{Sauce Smitane}\label{sauce-smitane}}

\index{すみたーぬ@スミターヌ!そーす@ソース・---}
\index{そーす@ソース!すみたーぬ@---・スミターヌ}
\index{さわーくりーむ@サワークリーム!そーすすみたーぬ@ソース・スミターヌ}
\index{sauce@sauce!smitane@--- Smitane}
\index{smitane@smitane!sauce@Sauce ---}
\index{creme aigre@crèeme aigre!sauce smitane@Sauce Smitane}

中位の大きさの玉ねぎを細かくみじん切りにし、バターで色付くまで炒める。
白ワイン2 dlを注ぎ、完全に煮詰める。サワークリーム\undemi{} Lを加える。
5分間沸騰させたら、布で漉す。サワークリームの風味を生かすために、必要
に応じてレモンの搾り汁少々を加える。

\ldots{}\ldots{}ジビエのソテーやカスロール仕立て\footnote{原文は gibiers
  sautés, ou cuits à la casserole となっており、
  ジビエのソテーまたはカスロール(片手鍋)で火を通したもの、というの
  が逐語訳だが、ここでは en casserole に解釈して訳した。雉、ペルドロー
  (山うずらの若鳥)、野生のうずらなどのen casserole が本書にも多数収
  録されているためである。カスロール仕立てen casseroleとは、油脂を熱
  したカスロールで肉を焼いた後に取り出し、フォンなどを加えてソースを
  作り、肉を鍋に戻し入れて鍋ごと供する仕立てのこと。なお、casserole
  のうちフランスに古くからあるタイプのものは比較的浅い鍋で、ソースパ
  ンとも呼ばれる。深いものはcasserole russe カスロールリュス(ロ
  シア式片手鍋)と言う。}用。

\maeaki

\hypertarget{ux30bdux30fcux30b9ux30bdux30ebux30d5ux30a7ux30eaux30ce}{%
\subsubsection{ソース・ソルフェリノ}\label{ux30bdux30fcux30b9ux30bdux30ebux30d5ux30a7ux30eaux30ce}}

\hypertarget{sauce-solfuxe9rino}{%
\paragraph{Sauce Solférino}\label{sauce-solfuxe9rino}}

\index{そーす@ソース!そるふぇりの@---・ソルフェリノ}
\index{そるふぇりの@ソルフェリノ!そーす@ソース・---}
\index{sauce@sauce!solferino@--- Solf\'erino}
\index{solferino@Solf\'erino!sauce@Sauce ---}

よく熟したトマト15個をしっかり搾って、その果汁を器に入れる。これを布で
漉し、濃いシロップ状になるまで煮詰める。

溶かした\protect\hyperlink{glace-de-viande}{グラスドヴィヤンド}大さじ3杯とカイエンヌ1つ
まみ、レモン\undemi{}個分の搾り汁を加える。

火から外して、エストラゴン風味の\protect\hyperlink{beurre-maitre-d-hotel}{メートルドテルバ
ター}100 gと\protect\hyperlink{beurre-d-echalote}{エシャロットバ
ター}100 gを加える。

\ldots{}\ldots{}このソースはどんな肉のグリルにもよく合う。

\hypertarget{ux539fux6ce8-17}{%
\subparagraph{【原注】}\label{ux539fux6ce8-17}}

言い伝えによると、フランス軍がたびたび進軍して戦ったロンバルディア平野
で、たくさんの料理が創作された。このソースもそのひとつであり、カプリア
ナ村においてフランスとサルデーニャの連合軍司令官の昼食に供されたという。
その村の近くであの苛烈きわまるソルフェリノの戦い\footnote{1859年に起きたフランス=サルデーニャ連合軍とオーストリア帝国軍
  の戦闘。戦場視察したナポレオン三世はその光景のあまりの悲惨さにイタ
  リア独立戦争への介入から手を引くことを決意したともいう。}が繰り広げられた
のだ。

伝えられているレシピはおそらくは調理担当軍人によるものだろうが、充分に
日常的に使えるものだった。このソースは、Sauce Saint-Cloud ソース・サン
クルー\footnote{サンクルーはパリ近郊の地名。普仏戦争時(1870〜1871)にパリ包囲
  戦の舞台となり、休戦協定の結ばれた2日後に大火に見舞われた。いずれ
  にせよ戦争の悲惨さを蔭に持つソース名ということになるが、エスコフィ
  エ自身が普仏戦争において従軍したために、その名称をこのソースに付け
  ることは許し難かったのだろう。}と呼ばれることもあるが、それは誤りだ。作り方も材料もソース・サン
クルーの名を付けるにはまったく値しない程の誤りだ。

\maeaki

\hypertarget{ux30bdux30fcux30b9ux30b9ux30d3ux30fcux30ba-ux7389ux306dux304eux306eux30afux30eaux30b9ux30d3ux30fcux30ba137}{%
\subsubsection[ソース・スビーズ /
玉ねぎのクリ・スビーズ]{\texorpdfstring{ソース・スビーズ /
玉ねぎのクリ・スビーズ\footnote{18世紀の代表的料理人のひとりFrançois
  Marinフランソワ・マラン
  (生没年不詳)が仕えたシャルル・ド・ロアン・スビーズ元帥のこと。
  マランは4巻からなる『コモス神の贈り物、あるいは食卓の悦楽』(1739
  年刊)を著した。}}{ソース・スビーズ / 玉ねぎのクリ・スビーズ}}\label{ux30bdux30fcux30b9ux30b9ux30d3ux30fcux30ba-ux7389ux306dux304eux306eux30afux30eaux30b9ux30d3ux30fcux30ba137}}

\hypertarget{sauce-soubise-ou-coulis-doignons-soubise}{%
\paragraph{Sauce Soubise, ou Coulis d'oignons
Soubise}\label{sauce-soubise-ou-coulis-doignons-soubise}}

\index{そーす@ソース!すひーす@---・スビーズ}
\index{すひーす@スビーズ!そーす@ソース・---}
\index{くり@クリ!たまねぎのくりすひーず@玉ねぎのクリ・スビーズ}
\index{sauce@sauce!soubise@--- Soubise}
\index{soubise@Soubise!sauce@Sauce ---}
\index{coulis@coulis!oignons soubise@Coulis d'oignons Soubise}

このソースの作り方には以下の2つがある。

\begin{enumerate}
\def\labelenumi{\arabic{enumi}.}
\tightlist
\item
  玉ねぎ500 gを薄切りにする\footnote{émincer エマンセ。}。これをしっかり下茹でしておく。
\end{enumerate}

玉ねぎはしっかりと水気をきって、バターを加えて鍋に蓋をして弱火で色付か
ないよう注意して蒸し煮する\footnote{étuver エチュヴェ。}。ここに濃厚に作った\protect\hyperlink{sauce-bechamel}{ベシャメルソー
ス}\undemi{} Lを加える。塩1つまみと白こしょう少々、粉
砂糖1つまみ強を加える。

オーブンに入れてじっくり火入れする。布で漉し、鍋に移したソースを熱する。
バター80 gと生クリーム1 dlを加えて仕上げる。

\begin{enumerate}
\def\labelenumi{\arabic{enumi}.}
\setcounter{enumi}{1}
\item
  上記と同様に薄切りにした玉ねぎを下茹でし、水気をきる。豚背脂の薄い
  シート\footnote{barde de lard
    豚背脂を薄くスライスしたもの。ベーコンと誤解され
    がちなので注意。エスコフィエ以前の時代のフランス料理ではきわめて多
    用されるとても重要なものなのでぜひとも覚えておきたい。自作する際に
    は、豚背脂の塊を冷凍た後、適度な固さに戻してからスライスすると作業が
    容易になる。}を敷き詰めた丁度いい大きさの深手の片鍋\footnote{casserole
    russe
    \protect\hyperlink{sauce-smitane}{ソース・スミターヌ}訳注参照。}に、下茹で
  して水気をきった玉ねぎをすぐに入れ、カロライナ米\footnote{長粒種。リゾットなどに適している。}120
  gと\protect\hyperlink{consomme-blanc}{白い コンソメ}7
  dl、塩、こしょう、砂糖は上記と同様に加 え、さらにバター25gも加える。

  強火にかけて沸騰したら、オーブンに入れてゆっくり加熱する。

  鉢に米と玉ねぎを移し入れてすり潰す。これを布で漉し、温める。上記と
  同様にバターを生クリームを加えて仕上げる。
\end{enumerate}

\hypertarget{ux539fux6ce8-18}{%
\subparagraph{【原注】}\label{ux539fux6ce8-18}}

スビーズはソースというよりはむしろクリ\footnote{クリcoulisについては、\protect\hyperlink{sauce-salmis}{ソース・サルミ}訳注参照。}であって、真っ白な仕上りにすべきだ。

ベシャメルを用いた作り方のほうが米を用いるよりもいいだろう。というのも、
より滑らかな口あたりのクリになるからだ。その一方、米を使うとよりしっか
りした仕上りになる。

どちらの方法で作るかは、このスビーズを合わせる料理の種類によって決める
べきだ。

\maeaki

\hypertarget{ux30c8ux30deux30c8ux5165ux308aux30bdux30fcux30b9ux30b9ux30d3ux30fcux30ba}{%
\subsubsection{トマト入りソース・スビーズ}\label{ux30c8ux30deux30c8ux5165ux308aux30bdux30fcux30b9ux30b9ux30d3ux30fcux30ba}}

\hypertarget{sauce-soubise-tomatuxe9e}{%
\paragraph{Sauce Soubise tomatée}\label{sauce-soubise-tomatuxe9e}}

\index{そーす@ソース!すひーすとまといり@トマト入り---・スゥビーズ}
\index{すっひーす@スビーズ!そーすとまといり@トマト入りソース・---}
\index{sauce@sauce!soubise tomatee@--- Soubise tomatée}
\index{soubise@Soubise!sauce tomatee@Sauce --- tomatée}

上記のいずれかの方法で作ったソース・スビーズに\untiers{}量の、滑らかで
真っ赤なトマトピュレを加える。

\maeaki

\hypertarget{ux30bdux30fcux30b9ux30b9ux30fcux30b7ux30a7}{%
\subsubsection{ソース・スーシェ}\label{ux30bdux30fcux30b9ux30b9ux30fcux30b7ux30a7}}

\hypertarget{sauce-souchet139}{%
\paragraph[Sauce Souchet]{\texorpdfstring{Sauce Souchet\footnote{ナポレオン軍の元帥を務めたルイ・スーシェ・アルビュフェラ公爵の
  こと。 正しくはSuchetだが、料理名としては Souchet とも綴られる。
  \protect\hyperlink{sauce-albufera}{ソース・アルビュフェラ}訳注参照。}}{Sauce Souchet}}\label{sauce-souchet139}}

\index{そーす@ソース!すーしえ@---・スーシェ}
\index{すーしえ@スーシェ!そーす@ソース・---}
\index{sauce@sauce!souchet@--- Souchet}
\index{souchet@Souchet!sauce@Sauce ---}

オランダおよびフランドル地方の\protect\hyperlink{}{ワーテルゾイ}から派生したソース。

いくらか変化したかたちでイギリス料理に取り入れられ、近代料理の原則に合
うようにさらに手を加えたもの。

細さ1〜2 mm角、長さ3〜4 cmの千切り\footnote{julienne ジュリエンヌ。}にした、にんじん、根パセリ、セ
ロリ計150 gを用意する。

これを鍋に入れてバターを加え、蓋をして蒸し煮する\footnote{étuver au
  beurre エチュヴェオブール}。\protect\hyperlink{fumet-de-poisson}{魚のフォ
ン}\troisquarts{} Lと白ワイン2 dlを注ぐ。弱火で煮て、
このクールブイヨン\footnote{court-bouillon
  原義は「量の少ないブイヨン」。実際、魚などを茹
  でる(ポシェする)際には、ぎりぎりの大きさの鍋を用いて茹で汁の量は
  出来るだけ少なく済むようにする。誤解しやすい用語なので注意。}を漉す。千切りにした野菜は別に取り置いておく。

このクールブイヨンで、切り分けた魚を煮る。

魚に火が通ったら、魚の身を取り出して、クールブイヨンはシノワ\footnote{円錐形に取っ手の付いた漉し器。}漉す。
これを約\unquart{}量すなわち2\undemi{}
dlになるまで煮詰める。\protect\hyperlink{sauce-vin-blanc}{白ワイン
ソース}を加えて適当なとろみが付くようにする。あるい
は単純にブールマニエでとろみを付け、軽くバターを加えてもいい。

ソースの中に取り置いていた千切りの野菜を戻し入れる。魚の切り身を覆うよ
うにソースをかけて供する。

\hypertarget{ux30c1ux30edux30ebux98a8145ux30bdux30fcux30b9}{%
\subsubsection[チロル風ソース]{\texorpdfstring{チロル風\footnote{そもそも\protect\hyperlink{sauce-choron}{ソース・ショロン}をバターではなく植物
  油を用いて作るものであるから、オーストリアのチロル地方とはまったく
  関係がない。1848年のイタリア、チロルでのオーストリアに対する反乱を
  記念した命名だという説もあるが、真偽は不明。ただし、本書の初版から
  ほぼ異同のない内容で収録されているため、それなりに古くから存在して
  いるソースと思われる。}ソース}{チロル風ソース}}\label{ux30c1ux30edux30ebux98a8145ux30bdux30fcux30b9}}

\hypertarget{sauce-tyrolienne}{%
\paragraph{Sauce Tyrolienne}\label{sauce-tyrolienne}}

\index{そーす@ソース!ちろるふう@チロル風---}
\index{ちろるふう@チロル風!そーす@---ソース}
\index{sauce@sauce!tyrolienne@--- Tyrolienne}
\index{tyrolien@tyrolien!sauce@Sauce Tyrolienne}

\protect\hyperlink{sauce-bearnaise}{ソース・ベアルネーズ}を作る場合とまったく同じ要領で、
白ワインとヴィネガー、香草類を煮詰める(\protect\hyperlink{sauce-bearnaise}{ソース・ベアルネー
ズ})参照。布で漉してきつく絞る。

これに、よく煮詰めた真っ赤なトマトピュレ大さじ2杯と卵黄6個を加える。鍋
をごく弱火にかけながら、\protect\hyperlink{sauce-mayonnaise}{マヨネーズ}を作る要領で植
物油5
dlを加えてしっかりと乳化させる。最後に味を調え、カイエンヌ\footnote{赤唐辛子の一品種だが、日本のカエンペッパーより辛さもマイルドで
  風味が違うことに注意。}ごく少 量で風味を引き締める。

\ldots{}\ldots{}このソースは牛肉、羊肉のグリルや魚のグリル焼きに合う。

\maeaki

\hypertarget{ux30c1ux30edux30ebux98a8ux30bdux30fcux30b9ux30afux30e9ux30b7ux30c3ux30af146}{%
\subsubsection[チロル風ソース クラシック]{\texorpdfstring{チロル風ソース クラシック\footnote{このレシピは第四版のみ。ここでの
  à l'ancienne は「昔ながらの」
  という意味ではない。ベースとなっているソース・ポワヴラードが古くか
  らあるソースであることからこの名称を第四版で付けたと考えられる。な
  お、本書において à l'ancienne 「昔風」「昔ながらの」という名称が付
  くレシピはその多くが17〜18世紀の古典期に起源を持つか、そのイメージ
  を表現しているものであり、ここでは後者と捉えて、あえて「昔風」では
  なく」「クラシック」と訳した。}}{チロル風ソース クラシック}}\label{ux30c1ux30edux30ebux98a8ux30bdux30fcux30b9ux30afux30e9ux30b7ux30c3ux30af146}}

\hypertarget{sauce-tyrolienne-uxe0-lancienne}{%
\paragraph{Sauce Tyrolienne à
l'ancienne}\label{sauce-tyrolienne-uxe0-lancienne}}

\index{そーす@ソース!ちろるふうくらしつく@チロル風--- クラシック}
\index{ちろるふう@チロル風!そーす@---ソース クラシック}
\index{sauce@sauce!tyrolienne ancienne@--- Tyrolienne à l'ancienne}
\index{tyrolien@Tyrolien!sauce ancienne@Sauce Tyrolienne à l'ancienne}

大きめの玉ねぎ2個をごく薄くスライス\footnote{émincer エマンセ。}してバターで炒める。トマト3個
を押し潰して皮を剥き、種を取り除いてから加える。\protect\hyperlink{sauce-poivrade}{ソース・ポワヴラー
ド}5 dlを加える。7〜8分間煮て仕上げる。

\maeaki

\hypertarget{ux30bdux30fcux30b9ux30f4ux30a1ux30edux30ef}{%
\subsubsection{ソース・ヴァロワ}\label{ux30bdux30fcux30b9ux30f4ux30a1ux30edux30ef}}

\hypertarget{sauce-valois}{%
\paragraph{Sauce Valois}\label{sauce-valois}}

\index{うあろわ@ヴァロワ!そーす@ソース・---}
\index{そーす@ソース!うあろわ@---・ヴァロワ}
\index{sauce@sauce!valois@--- Valois}
\index{valois@Valois!sauce@Sauce ---}

\protect\hyperlink{sauce-bearnaise-a-la-glace-de-viande}{グラスドヴィヤンド入りソース・ベアルネー
ズ}のこと(\protect\hyperlink{sauce-bearnaise}{ソース・ベアルネー
ズ}参照)。

\hypertarget{ux539fux6ce8-19}{%
\subparagraph{【原注】}\label{ux539fux6ce8-19}}

「ソース・ヴァロワ」はグフェが1863年頃に創案したらしい。少なくともその
頃に作られるようになったものであろう。近年では「ソース・フォイヨ」の名
称のほうが一般的だが、いかにもあり得そうな異論反論を受けないためにもこ
こでその起源を記しておくのがいいと思われた。

\maeaki

\hypertarget{ux30f4ux30a7ux30cdux30c4ux30a3ux30a2ux98a8148ux30bdux30fcux30b9}{%
\subsubsection[ヴェネツィア風ソース]{\texorpdfstring{ヴェネツィア風\footnote{ヴェネツィア料理ではさまざまな香草を用いるものがあることから、
  その影響を受けた、あるいは類似したものにこの名称が付けられることが
  多い。なお、ヴェネツィアの近く、漁港で有名なキオッジャ近郊は農業が
  とても盛んで、地場品種の野菜も多い。横に切ると白とピンクの年輪状の
  模様が表れるビーツ・キオッジャ(イタリア語では barbabietola di
  Chioggia バルバビエトラ・ディ・キオッジャ)が代表的だが、カボチャ
  やラディッキオ(radicchio tartivo di Treviso ラディッキオ・タルディー
  ヴォ・ディ・トレヴィーゾが有名だが、radicchio di Chioggiaラディッ
  キオ・ディ・キオッジャはいわゆるトレヴィスに非常に近い)にもキオッ
  ジャの名が付く品種がある。}ソース}{ヴェネツィア風ソース}}\label{ux30f4ux30a7ux30cdux30c4ux30a3ux30a2ux98a8148ux30bdux30fcux30b9}}

\hypertarget{sauce-vuxe9nitienne}{%
\paragraph{Sauce Vénitienne}\label{sauce-vuxe9nitienne}}

\index{そーす@ソース!うえねついあふう@ヴェネツィア風---}
\index{うえねついあふう@ヴェネツィア風!そーす@---ソース}
\index{sauce@sauce!venitienne@--- Vénitienne}
\index{venitien@vénitien!sauce@Sauce Vénitienne}

エストラゴンヴィネガー4 dlに、エシャロットのみじん切り大さじ2杯とセル
フイユ25 gを加え、\untiers{}量まで煮詰める。煮詰めたら布で漉し、軽く絞っ
てやる。ここに\protect\hyperlink{sauce-vin-blanc}{白ワインソース}\troisquarts{}
Lを加え る。\protect\hyperlink{beurre-vert}{ブール・ヴェール}125
gと、セルフイユとエストラゴン のみじん切り大さじ1杯を加えて仕上げる。

\ldots{}\ldots{}さまざまな魚料理に添える。

\maeaki

\hypertarget{ux30bdux30fcux30b9ux30f4ux30a7ux30edux30f3149}{%
\subsubsection[ソース・ヴェロン]{\texorpdfstring{ソース・ヴェロン\footnote{Luis
  Véron (1798〜1867)。医師であり、文学愛好家、美食家として
  も有名だった。文芸誌「ルヴュ・ド・パリ」を主宰した後、新聞「ル・コ
  ンスティチュショネル」の社主となり、ウージェーヌ・シューの新聞連載
  小説『彷徨えるユダヤ人』を掲載、大ヒットに導いた。、自宅は文壇サロ
  ンのようだったという。主著『パリのとあるブルジョワの回想録』(1853〜
  1955年刊)。}}{ソース・ヴェロン}}\label{ux30bdux30fcux30b9ux30f4ux30a7ux30edux30f3149}}

\hypertarget{sauce-veron}{%
\paragraph{Sauce Véron}\label{sauce-veron}}

\index{そーす@ソース!うえろん@---・ヴェロン}
\index{うえろん@ヴェロン!そーす@ソース・---}
\index{veron@Véron!sauce@Sauce ---} \index{sauce@sauce!veron@--- Véron}

仕上げた状態の\protect\hyperlink{sauce-normande}{標準的なノルマンディ風ソー
ス}\troisquarts{} Lに、\protect\hyperlink{sauce-tyrolienne}{チロル風ソー
ス}\unquart{} Lを加える。よく混ぜ合わせ、溶かしたブ
ロンド色の\protect\hyperlink{glace-de-viande}{グラスドヴィアンド}大さじ2杯とアンチョビ
エッセンス大さじ1杯を加えて仕上げる。

\ldots{}\ldots{}魚料理用。

\maeaki

\hypertarget{ux6751ux4ebaux98a8ux30bdux30fcux30b9}{%
\subsubsection{村人風ソース}\label{ux6751ux4ebaux98a8ux30bdux30fcux30b9}}

\hypertarget{sauce-villageoise}{%
\paragraph[Sauce Villageoise]{\texorpdfstring{Sauce
Villageoise\footnote{文字通り「村人風」の意だが、このソースの他にもこの名称を冠した
  料理はあるが、どれもとりたてて素朴というわけではなく、由来は不明。}}{Sauce Villageoise}}\label{sauce-villageoise}}

\index{そーす@ソース!むらひとふう@村人風---}
\index{むらひとふう@村人風!そーす@---ソース}
\index{sauce@sauce!villageoise@--- Villageoise}
\index{villageois@villageois!sauce@Sauce Villageoise}

\protect\hyperlink{veloute}{標準的なヴルテ}\troisquarts{}
Lに、ブロンド色の\protect\hyperlink{jus-de-veau-brun}{仔牛の
ジュ}\footnote{本書には「仔牛の茶色いジュ」のレシピはあるが、ブロンド色のものについては記述がない。}1
dlとマッシュルームの茹で汁1 dlを加える。
\deuxtiers{}量くらいまで煮詰め、布で漉す。

\protect\hyperlink{sauce-soubise}{ベシャメルで作ったソース・スビーズ}\footnote{2つある作り方のうちの1の方。}2
dlと、とろ
み付けの卵黄4個を加える。沸騰させないよう気をつけて温め、火から外して
バター100 gを加えて仕上げる。

\ldots{}\ldots{}仔牛、仔羊などの白身肉に合わせる。

\maeaki

\hypertarget{ux30bdux30fcux30b9ux30f4ux30a3ux30ebux30edux30ef153}{%
\subsubsection[ソース・ヴィルロワ]{\texorpdfstring{ソース・ヴィルロワ\footnote{ルイ15世の養育係を務めたヴィルロワ元帥
  François de Villeroi の名を冠したものとされる。}}{ソース・ヴィルロワ}}\label{ux30bdux30fcux30b9ux30f4ux30a3ux30ebux30edux30ef153}}

\hypertarget{sauce-villeroy}{%
\paragraph{Sauce Villeroy}\label{sauce-villeroy}}

\index{そーす@ソース!ういるろわ@---・ヴィルロワ}
\index{ういるろわ@ヴィルロワ!そーす@ソース・---}
\index{sauce@sauce!villeroy@--- Villeroy}
\index{villeroy@Villeroy!sauce@Sauce ---}

\protect\hyperlink{sauce-allemande}{ソース・アルマンド}1
Lに、トリュフエッセンス大さじ4
杯と\protect\hyperlink{essences-diverses}{ハムのエッセンス}大さじ4杯を加える。

ヘラで混ぜながら強火にかけ、主素材となるものをソースに漬けて取り出したとき際に、全
体をソースが覆うようになるような漉さまで煮詰めていく。

\hypertarget{ux539fux6ce8-20}{%
\subparagraph{【原注】}\label{ux539fux6ce8-20}}

このソースの唯一の使い途は、素材をこのソースで包み込んでから、イギリス
式パン粉衣を付けて揚げるものだ。この方法で調理したものは常に「ヴィルロ
ワ風」の名称となる。このソースは、古典料理において「隠れたソース」と呼
ばれていたもののうちの典型例と言える。

\maeaki

\hypertarget{ux30b9ux30d3ux30fcux30baux5165ux308aux30bdux30fcux30b9ux30f4ux30a3ux30ebux30edux30ef}{%
\subsubsection{スビーズ入りソース・ヴィルロワ}\label{ux30b9ux30d3ux30fcux30baux5165ux308aux30bdux30fcux30b9ux30f4ux30a3ux30ebux30edux30ef}}

\hypertarget{sauce-villeroy-soubisee}{%
\paragraph{Sauce Villeroy Soubisée}\label{sauce-villeroy-soubisee}}

\index{そーす@ソース!ういるろわすう@スビーズ入り---・ヴィルロワ}
\index{ういるろわ@ヴィルロワ!そーすすひーす@スビーズ入りソース・---}
\index{sauce@sauce!villeroy soubisee@--- Villeroy Soubisée}
\index{villeroy@Villeroy!sauce soubisee@Sauce --- Soubisée}

\protect\hyperlink{sauce-allemande}{ソース・アルマンド}に\untiers{}量の\protect\hyperlink{sauce-soubise}{スビーズのピュ
レ}\footnote{ソース・スビーズは濃度があるのでピュレと呼んだと考えていいだろ
  う。クリ coulis は「やや水分の多いピュレ」と同義だからだ。}を加え、上記と同様に煮詰めて作る。

このソースを付ける素材や仕立てに合わせて、ソース1 Lあたり80〜100 gのト
リュフのみじん切りを加えることもある。

\maeaki

\hypertarget{ux30c8ux30deux30c8ux5165ux308aux30bdux30fcux30b9ux30f4ux30a3ux30ebux30edux30ef}{%
\subsubsection{トマト入りソース・ヴィルロワ}\label{ux30c8ux30deux30c8ux5165ux308aux30bdux30fcux30b9ux30f4ux30a3ux30ebux30edux30ef}}

\hypertarget{sauce-villeroy-tomatee}{%
\paragraph{Sauce Villeroy tomatée}\label{sauce-villeroy-tomatee}}

\protect\hyperlink{sauce-villeroy}{標準的なソース・ヴィルロワ}とまったく作り方は同じだ
が、\protect\hyperlink{sauce-allemande}{ソース・アルマンド}の\untiers{}量の上等で真っ赤
なトマトピュレを加えて作る。

\hypertarget{ux767dux30efux30a4ux30f3ux30bdux30fcux30b9}{%
\subsubsection{白ワインソース}\label{ux767dux30efux30a4ux30f3ux30bdux30fcux30b9}}

\hypertarget{sauce-vin-blanc}{%
\paragraph{Sauce vin blanc}\label{sauce-vin-blanc}}

\index{そーす@ソース!しろわいん@白ワイン---}
\index{しろわいん@白ワイン!そーす@---ソース}
\index{わいん@ワイン!しろわいん@白ワイン!そーす@---ソース}
\index{sauce@sauce!vin blanc@--- vin blanc}
\index{vin@vin!sauce vin blanc@Sauce vin blanc}

このソースには以下の3種類の作り方がある。

\begin{enumerate}
\def\labelenumi{\arabic{enumi}.}
\item
  \protect\hyperlink{veloute-de-poisson}{魚料理用ヴルテ}1
  Lに、ソースを合わせる魚
  でとった\protect\hyperlink{fumet-de-poisson}{フュメ}2
  dlと、卵黄4個を加える。\deuxtiers{}量 まで煮詰め、バター150
  gを加える。\\
  この「白ワインソース」は、仕上げにオーブンに入れて照りをつける魚料
  理に合わせる。
\item
  良質の\protect\hyperlink{fumet-de-poisson}{魚のフュメ}1
  を半分にまで煮詰める。卵黄5
  個を加え、\protect\hyperlink{sauce-hollandaise}{オランデーズソース}を作る際の要領で、
  バター500 gを加えてよく乳化させる。
\item
  卵黄5個を片手鍋\footnote{casserole カスロール。}に入れて溶きほぐし、軽く温めてやる。バター500
  gを加えて乳化させていく途中で、上等な\protect\hyperlink{fumet-de-poisson}{魚のフュメ}1
  dlを少しずつ加えていく\footnote{いずれの作り方にも白ワインが出てこないのは、それぞれで使われて
    いる\protect\hyperlink{fumet-de-poisson}{魚のフュメ}において\ruby{既}{すで}に白ワイ
    ンを用いているから。}。
\end{enumerate}
\end{recette}\newpage
\hypertarget{ux30a4ux30aeux30eaux30b9ux98a8ux30bdux30fcux30b9ux6e29ux88fd24}{%
\section[イギリス風ソース(温製)]{\texorpdfstring{イギリス風ソース(温製)\footnote{この節では初版で31、第二版は33、第三版と第四版で30のレシピが掲
  載されている。1907年刊の英語版\emph{A Guide to Modern Cookery}でこの節
  に相当する``Hot English Sauces''には10のレシピしか掲載されていない。
  この大きな数の差をどう解釈するかは意見の分かれるところだろうが、対
  象読者がフランス人であるかイギリス人であるかという違いを意識し、ニー
  ズに応えるかたちをとったと考えるのが妥当だろう。ただし、あくまでも
  エスコフィエあるいは共同執筆者の解釈を経た「イギリス風」のソースが
  ほとんどであることは、例えば「\protect\hyperlink{roe-buck-sauce}{ローバックソース}」
  において\protect\hyperlink{sauce-espagnole}{ソース・エスパニョル}を用いていること、
  つまりはエスコフィエが構築したソースの体系に組み込まれ得るものであ
  ることから判断がつく。}}{イギリス風ソース(温製)}}\label{ux30a4ux30aeux30eaux30b9ux98a8ux30bdux30fcux30b9ux6e29ux88fd24}}

\hypertarget{sauces-anglaises-chaudes}{%
\subsection{Sauces Anglaises Chaudes}\label{sauces-anglaises-chaudes}}

\index{そーす@ソース!いきりすふうおんせい@イギリス風---(温製)}
\index{いきりすふう@イギリス風!そーすおんせい@---ソース(温製)}
\index{sauce@sauce!anglaises chaudes@---s anglaises chaudes}
\index{anglais@anglais(e)!sauces chaudes@sauces ---es chaudes}
\begin{recette}
\hypertarget{ux30afux30e9ux30f3ux30d9ux30eaux30fc1ux30bdux30fcux30b9}{%
\subsubsection[クランベリーソース]{\texorpdfstring{クランベリー\footnote{英語のcranberryはツルコケモモ(学名Vaccinium
  oxycoccos)であり、 フランス語airelles rougesはコケモモ(学名Vaccinium
  vitis-idaea
  L.)で、非常によく似た近縁種であり、しばしば混同される。本書でもと
  くに区別されていない。}ソース}{クランベリーソース}}\label{ux30afux30e9ux30f3ux30d9ux30eaux30fc1ux30bdux30fcux30b9}}

\hypertarget{cranberries-sauce}{%
\paragraph{\texorpdfstring{Sauce aux Airelles
(\emph{Cranberries-Sauce})}{Sauce aux Airelles (Cranberries-Sauce)}}\label{cranberries-sauce}}

\index{いきりすふう@イギリス風!そーすおんせい@---ソース(温製)!くらんへりーそーす@クランベリーソース}
\index{そーす@ソース!いきりすふうおんせい@イギリス風---(温製)!くらんへりーそーす@クランベリー---}
\index{くらんへりー@クランベリー!そーす@---ソース}

\index{sauce@sauce!anglaises chaudes@---s anglaises chaudes!airelles@--- aux Airelles (Cranberries-Sauce)}
\index{airelle@airelle!sauce airelles@Sauce aux Airelles (Cranberries-Sauce)}
\index{anglais@anglais(e)!sauces chaudes@sauces ---es chaudes!airelles@Sauce aux Airelles (Cranberries-Sauce)}
\index{cranberry!Cranberries-Sauce}

クランベリー500 gを1
Lの湯で、鍋に蓋をして茹でる。果肉に火が通ったら、湯をきって、目の細かい網で裏漉しする。

こうして出来たピュレに茹で汁を適量加えてやや濃度のあるソースの状態にする。好みに応じて砂糖を加える。

このソースは市販品があり\footnote{\protect\hyperlink{sauce-robert-escoffier}{ソース・ロベール・エスコフィエ}などの
  ようなエスコフィエブランドの商品というわけではないと思われる。}、水少々を加えて温めるだけで使える。

\ldots{}\ldots{}七面鳥のロースト用。

\maeaki

\hypertarget{ux30a2ux30ebux30d0ux30fcux30c8ux30bdux30fcux30b9}{%
\subsubsection{アルバートソース}\label{ux30a2ux30ebux30d0ux30fcux30c8ux30bdux30fcux30b9}}

\hypertarget{albert-sauce}{%
\paragraph[Sauce Albert (\emph{Albert-Sauce})]{\texorpdfstring{Sauce
Albert\footnote{ザクセン=コーブルク=ゴータ公アルバート王配(ヴィクトリア女王の
  夫)(1819〜1861)のこと。女王エリザベス二世の高祖父。本書序文p.ii
  において触れられている料理人エルーイがアルバート王配に仕えていたこ
  とがある。なお、本書に掲載されていないが、Sole Albert 「舌びらめ 
  アルベール」という料理がある。しかしながら、これはパリのレストラン、
  マキシムズMaxim'sでメートルドテルを務めたアルベール・ブラゼール Albert
  Blazerの名を冠したもので1930年代に創案されたもの。このソー
  スとはまったく関係がないことに注意。}
(\emph{Albert-Sauce})}{Sauce Albert (Albert-Sauce)}}\label{albert-sauce}}

\index{いきりすふう@イギリス風!そーすおんせい@---ソース(温製)!あるはーと@アルバートソース}
\index{そーす@ソース!いきりすふうおんせい@イギリス風---(温製)!あるはーと@アルバート---}
\index{あるはーと@アルバート!そーす@---ソース}
\index{sauce@sauce!anglaises chaudes@---s anglaises chaudes!albert@--- Albert (Albert-Sauce)}
\index{albert@Albert!sauce@Sauce --- (Albert-Sauce)}
\index{anglais@anglais(e)!sauces chaudes@sauces ---es chaudes!albert@Sauce Albert (Albert-Sauce)}

すりおろしたレフォール\footnote{raifort ホースラディッシュ、西洋わさび。}150
gに\protect\hyperlink{}{白いコンソメ}2 dlを注ぎ、弱火で20分間煮る。

\protect\hyperlink{butter-sauce}{イギリス式バターソース}3
dlと生クリーム2\undemi{} dl、パンの白い身の部分40
gを加える。強火にかけて煮詰め、木ヘラで圧し絞るようにしながら布で漉す\footnote{二人で作業すると容易。\protect\hyperlink{veloute}{ヴルテ}訳注参照。}。卵黄2個を加えてとろみを付
け\footnote{このソースの特徴として、イギリスのローストビーフに欠かせないもの
  とされるレフォール(ホースラディッシュ)を用いていることの他に、と
  ろみ付けにパンと卵黄を使っている点にも注目すべきだろう。とろみ付け
  の要素としてはきわめて中世料理風と言ってもいい。ただし、中世の料理
  では、パンはこんがりと焼いてからヴィネガーなどでふやかしてよくすり
  潰し、さらに布で漉してとろみ付けに用いるのが一般的だった。パンの白
  い身の部分をそのまま使えるということは、それだけ小麦の精白度合いが
  高いということでもある。}、塩1つまみとこしょう少々で味を調える。

仕上げに、マスタード小さじ1杯をヴィネガー大さじ1杯で溶いてから加える。

\ldots{}\ldots{}牛肉、主としてフィレ肉のブレゼに添える。

\maeaki

\hypertarget{ux30a2ux30edux30deux30c6ux30a3ux30c3ux30afux30bdux30fcux30b9}{%
\subsubsection{アロマティックソース}\label{ux30a2ux30edux30deux30c6ux30a3ux30c3ux30afux30bdux30fcux30b9}}

\hypertarget{aromatic-sauce}{%
\paragraph{\texorpdfstring{Sauce aux Aromates
(\emph{Aromatic-Sauce})}{Sauce aux Aromates (Aromatic-Sauce)}}\label{aromatic-sauce}}

\index{そーす@ソース!いきりすふうおんせい@イギリス風---(温製)!あろまていつく@アロマティック---}
\index{いきりすふう@イギリス風!そーすおんせい@---ソース(温製)!あろまていつく@アロマティックソース}
\index{こうそう@香草!あろまていつく@アロマティックソース}

\index{sauce@sauce!anglaises chaudes@---s anglaises chaudes!aromates@--- aux Aromates (Aromatic-Sauce}
\index{aromate@aromate!sauce@Sauce aux Aromates}
\index{anglais@anglais(e)!sauces chaudes@sauces ---es chaudes!aromates@Sauce aux Aromates (Aromatic-Sauce)}

\protect\hyperlink{}{コンソメ}\undemi{} Lに、タイム1枝、バジル4
g、サリエット\footnote{シソ科の香草。サマーセイヴォリー。和名キダチハッカ。}1
g、マジョラム1 g、セージ1 g、シブレット\footnote{ciboulette
  チャイヴ。アサツキと訳されることもあるが、日本のアサツキとは風味が違うので注意。}1を刻んだもの1つまみ、エシャロット\footnote{玉ねぎによく似ているが小さくて水分量の少ない香味野菜。英語由来のシャロットと呼ばれることも。日本の青果マーケットに見られる「エシャレット」はらっきょうの若どりであってまったく別のもの。}2個のみじん切り、ナツメグ少々、大粒のこしょう4個を入れて、10分
間煎じる\footnote{infuser アンフュゼ。}。

シノワ\footnote{円錐形で取っ手の付いた漉し器。}で漉し、バターで作った\footnote{本書第四版ではルーは必ずバターを用いる指示がなされているが、初版から第三版までは、バターもしくはグレスドマルミット(コンソメなどを作る際に浮いてきた油脂をすくい取って漉したもの)を使うという指示だっため、「バターで作った」という記述がこのように残っているレシピが散見される。}ブロンドのルー50
gを入れてとろみを付ける。数分間沸かしてから、レモン\undemi{}個分の搾り汁と、みじん切りにして下茹でしておいたセルフイユ\footnote{cerfeuil
  チャービル。}とエストラゴン\footnote{estragon フレンチタラゴン。}計大さじ1杯を加えて仕上げる\footnote{このソースで用いられている香草類の種類の多さは特筆に値するだろう。ブラウン系の派生ソースにある\protect\hyperlink{sauce-aux-fines-herbes}{香草ソース}およびホワイト系派生ソースの\protect\hyperlink{sauce-aux-fines-herbes-blanche}{香草ソース}と比較されたい。}。

\ldots{}\ldots{}大きな魚まるごと1尾のポシェあるいは牛、羊肉の大掛かりな仕立て(ルルヴェ\footnote{relevé
  \protect\hyperlink{sauce-diplomate}{ソース・ディプロマット}訳注参照。})に添える。

\maeaki

\hypertarget{ux30d0ux30bfux30fcux30bdux30fcux30b9}{%
\subsubsection{バターソース}\label{ux30d0ux30bfux30fcux30bdux30fcux30b9}}

\hypertarget{butter-sauce}{%
\paragraph{\texorpdfstring{Sauce au Beurre à l'anglaise (\emph{Butter
Sauce})}{Sauce au Beurre à l'anglaise (Butter Sauce)}}\label{butter-sauce}}

\index{いきりすふう@イギリス風!そーすおんせい@---ソース(温製)!はたーそーす@バターソース}
\index{そーす@ソース!いきりすふうおんせい@イギリス風---(温製)!はたーそーす@バター--}
\index{はたー@バター!そーすいきりすふう@---ソース(イギリス風)}
\index{sauce@sauce!anglaises chaudes@---s anglaises chaudes!beurre@--- au Beurre à l'anglaise (Butter Sauce)}
\index{beurre@beurre!sauce anglaise@Sauce au Beurre à l'anglaise (Butter Sauce)}
\index{anglais@anglais(e)!sauces chaudes@sauces ---es chaudes!beurre@Sauce au Beurre à l'anglaise (Butter Sauce)}

フランスの\protect\hyperlink{sauce-au-beurre}{ソース・オ・ブール}と同様に作るが、より濃度の高い仕上りにする点が違う。分量は、バター60
g、小麦粉60 g、1 Lあたり塩7 gを加えて沸かした湯\troisquarts{}
L。レモンの搾り汁5〜6滴、バター 200 g。とろみ付け用の卵黄は用いない。

\maeaki

\hypertarget{ux30b1ux30a4ux30d1ux30fcux30bdux30fcux30b9}{%
\subsubsection{ケイパーソース}\label{ux30b1ux30a4ux30d1ux30fcux30bdux30fcux30b9}}

\hypertarget{capers-sauce}{%
\paragraph{\texorpdfstring{Sauce aux Câpres
(\emph{Capers-Sauce})}{Sauce aux Câpres (Capers-Sauce)}}\label{capers-sauce}}

\index{いきりすふう@イギリス風!そーすおんせい@---ソース(温製)!けいはー@ケイパーソース}
\index{そーす@ソース!いきりすふうおんせい@イギリス風---(温製)!けいはーそーす@ケイパー---}
\index{けいはー@ケイパー!そーすいきりすふう@---ソース(イギリス風)}
\index{sauce@sauce!anglaise chaude@--- anglaie chaude!capres@--- aux Câpres (Capers-Sauce)}
\index{capre@câpre!sauce capres anglaise@Sauce aux Câpres (Capers-Sauce)}
\index{anglais@anglais(e)!sauces chaudes@sauces ---es chaudes!capres@Sauce aux Câpres (Capers-Sauce)}

上記の\protect\hyperlink{butter-sauce}{バターソース}1
Lあたり大さじ4杯のケイパーを加えたもの。

\ldots{}\ldots{}茹でた魚に添える。また、イギリス風\footnote{à l'anglaise
  アラングレーズ。茹でる(下茹でも含む)場合には、塩を加えた湯で茹でることを指す。なお、パン粉衣
  pané à l'anglaise
  という場合には、現代の日本でもなじみのある、小麦粉、溶きほぐした卵、パン粉の順で衣を付けて揚げることを言う。調理法全体を通しての規則性はなく、あくまでも「イギリス風に由来する」または「イギリス風」を意味するものなので注意。}に茹でた仔羊腿肉には欠かせない。

\maeaki

\hypertarget{ux30bbux30edux30eaux30bdux30fcux30b9}{%
\subsubsection{セロリソース}\label{ux30bbux30edux30eaux30bdux30fcux30b9}}

\hypertarget{celery-sauce}{%
\paragraph{\texorpdfstring{Sauce au Céleri
(\emph{Celery-Sauce})}{Sauce au Céleri (Celery-Sauce)}}\label{celery-sauce}}

\index{いきりすふう@イギリス風!そーすおんせい@---ソース(温製)!せろり@セロリソース}
\index{そーす@ソース!いきりすふうおんせい@イギリス風---(温製)!せろり@セロリ---}
\index{せろり@セロリ!そーすいきりすふう@---ソース(イギリス風)}
\index{sauce@sauce!anglaises chaudes@---s anglaises chaudes!celeri@--- au Céleri (Celery-Sauce)}
\index{celeri@céleri!sauce anglaise@Sauce au Céleri (Celery-Sauce)}
\index{anglais@anglais(e)!sauces chaudes@sauces ---es chaudes!celeri@Sauce au Céleri (Celery-Sauce)}

セロリ6株を掃除して、芯のところだけを使う\footnote{緑色が薄いタイプのセロリは中心部が自然に軟白され、柔らかいので、フランス料理でも非常に好まれる。}。これをソテー鍋に並べ、\protect\hyperlink{}{白いコンソメ}をセロリがかぶるまで注ぐ。ブーケガルニとクローブを刺した玉ねぎ1個を入れ、弱火で加熱する。

セロリの水気をきり、鉢に入れてすり潰す。これを布で漉す。こうして出来たセロリのピュレと同量の\protect\hyperlink{cream-sauce}{クリームソース}を加える。セロリの茹で汁を煮詰めたものを大さじ2〜3杯加える。

沸騰しない程度に温め、すぐに提供しない場合は湯煎にかけておく。

\ldots{}\ldots{}茹でた鶏または鶏のブレゼに添える。

\maeaki

\hypertarget{ux30edux30fcux30d0ux30c3ux30afux30bdux30fcux30b9}{%
\subsubsection{ローバックソース}\label{ux30edux30fcux30d0ux30c3ux30afux30bdux30fcux30b9}}

\hypertarget{roe-buck-sauce}{%
\paragraph[Sauce Chevreuil ()]{\texorpdfstring{Sauce Chevreuil
(\emph{Roe-buck\footnote{英語でノロ鹿のこと。}
Sauce})}{Sauce Chevreuil (Roe-buck Sauce)}}\label{roe-buck-sauce}}

\index{いきりすふう@イギリス風!そーすおんせい@---ソース(温製)!ろーはつく@ローバックソース}
\index{そーす@ソース!いきりすふうおんせい@イギリス風---(温製)!ろーはっく@ローバック---}
\index{ろーはつく@ローバック!そーすいきりすふう@---ソース(イギリス風)}
\index{のろしか@ノロ鹿 ⇒ シュヴルイユ!そーす@ソース!ろーはつくそーす@ローバックソース(イギリス風)}
\index{しゆうるいゆ@シュヴルイユ!ろーはつくそーす@ローバックソース(イギリス風)}
\index{sauce@sauce!anglaises chaudes@---s anglaises chaudes!chevreuil@--- Chevreuil (Roe-buck Sauce)}
\index{chevreuil@chevreuil!sauce chevreuil anglaise@Sauce Chevreuil (Roe-buck Sauce)}
\index{anglais@anglais(e)!sauces chaudes@sauces ---es chaudes!chevreuil@Sauce Chevreuil (Roe-buck Sauce)}

中位の大きさの玉ねぎを1cm角くらいの粗みじん切\footnote{paysanne
  ペイザンヌに切る、と言う。主として野菜について言うが、1 cm角で厚さ1〜2
  mm程度。}りにし、生ハム80gも同様に刻む。これをバターで軽く色付くまで炒める。ブーケガルニを入れ、ヴィネガー1\undemi{}
dlを注ぎ、ほとんど完全に煮詰める。

\protect\hyperlink{sauce-espagnole}{ソース・エスパニョル}3
dlを注ぎ、15分程弱火にかけて、浮いてくる不純物を取り除く\footnote{dépouiller
  デプイエ ≒ écumer エキュメ。}。

15分経ったら、ブーケガルニを取り出し、ポルト酒コップ1杯\footnote{約1
  dl。}と\protect\hyperlink{}{グロゼイユのジュレ}大さじ1杯強を加えて仕上げる。

\ldots{}\ldots{}大型ジビエ肉\footnote{この場合は当然、ノロ鹿の料理だが、フランス料理でノロ鹿は時間をかけてマリネしてから調理し、そのマリナード(漬け汁)もソースに用いるのと比べると非常にシンプルなソースになっている点が興味深い。}の料理に添える。

\maeaki

\hypertarget{ux30afux30eaux30fcux30e0ux30bdux30fcux30b9}{%
\subsubsection{クリームソース}\label{ux30afux30eaux30fcux30e0ux30bdux30fcux30b9}}

\hypertarget{cream-sauce}{%
\paragraph{\texorpdfstring{Sauce Crème à l'anglaise
(\emph{Cream-Sauce})}{Sauce Crème à l'anglaise (Cream-Sauce)}}\label{cream-sauce}}

\index{いきりすふう@イギリス風!そーすおんせい@---ソース(温製)!くりーむ@クリームソース}
\index{そーす@ソース!いきりすふうおんせい@イギリス風---(温製)!くりーむ@クリーム---}
\index{くりーむ@クリーム!そーすいきりすふう@---ソース(イギリス風)}
\index{sauce@sauce!anglaises chaudes@---s anglaises chaudes!creme@--- Crème à l'anglaise (Cream-Sauce)}
\index{creme@crème!sauce creme anglaise@Sauce Crème à l'anglaise (Cream-Sauce)}
\index{anglais@anglais(e)!sauces chaudes@sauces ---es chaudes!creme@Sauce Crème à l'anglaise (Cream-Sauce)}

バター100 gと小麦粉60
gで\protect\hyperlink{roux-blanc}{白いルー}を作る。

\protect\hyperlink{}{白いコンソメ}7
dlでルーをのばし、マッシュルームのエッセンス1 dlと生クリーム2
dlを加える。

火にかけて沸騰させる。小玉ねぎ1個とパセリ1束を加え、弱火で15分程煮込む。提供直前に小玉ねぎとパセリは取り出す。

\ldots{}\ldots{}仔牛の骨付き背肉の塊\footnote{carré
  カレ。もとは「四角形」の意。料理では、肋骨ごとに切り分けていない仔牛および仔羊の骨付き背肉の塊を指す。}のローストに合わせる。

\maeaki

\hypertarget{ux30b7ux30e5ux30eaux30f3ux30d7ux30bdux30fcux30b9}{%
\subsubsection{シュリンプソース}\label{ux30b7ux30e5ux30eaux30f3ux30d7ux30bdux30fcux30b9}}

\hypertarget{shrimps-sauce}{%
\paragraph{\texorpdfstring{Sauce Crevettes à l'anglaise
(\emph{Shrimps-Sauce})}{Sauce Crevettes à l'anglaise (Shrimps-Sauce)}}\label{shrimps-sauce}}

\index{いきりすふう@イギリス風!そーすおんせい@---ソース(温製)!しゆりんふ@シュリンプソース}
\index{そーす@ソース!いきりすふうおんせい@イギリス風---(温製)!しゆりんふ@シュリンプ---}
\index{くるうえつと@クルヴェット!そーすいきりすふう@シュリンプソース(イギリス風)}
\index{sauce@sauce!anglaises chaudes@---s anglaises chaudes!crevettes@--- Crevettes à l'anglaise (Shrimps-Sauce)}
\index{crevette@crevette!sauce crevette anglaise@Sauce Crevettes à l'anglaise (Shrimps-Sauce)}
\index{anglais@anglais(e)!sauces chaudes@sauces ---es chaudes!crevettes@Sauce Crevettes à l'anglaise (Shrimps-Sauce)}

カイエンヌ少量を加えて風味を引き締めた\protect\hyperlink{butter-sauce}{イギリス風バターソース}1
Lに、アンチョビエッセンス小さじ1杯と殻を剥いた小海老\footnote{フランス語は
  crevette(s)
  クルヴェット。\protect\hyperlink{sauce-aux-crevettes}{ソース・クルヴェット}訳注参照。}の尾の身125
gを加える。

\ldots{}\ldots{}魚料理用。

\maeaki

\hypertarget{ux30c7ux30d3ux30ebux30bdux30fcux30b9}{%
\subsubsection{デビルソース}\label{ux30c7ux30d3ux30ebux30bdux30fcux30b9}}

\hypertarget{devilled-sauce}{%
\paragraph{\texorpdfstring{Sauce Diable (\emph{Devilled
Sauce})}{Sauce Diable (Devilled Sauce)}}\label{devilled-sauce}}

\index{いきりすふう@イギリス風!そーすおんせい@---ソース(温製)!てひるそーす@デビルソース}
\index{そーす@ソース!いきりすふうおんせい@イギリス風---(温製)!てひる@デビル---}
\index{あくま@悪魔 ⇒ ディアーブル!そーす@ソース!てひる@デビルソース(イギリス風)}
\index{ていあーふる@ディアーブル!てひるそーす@デビルソース(イギリス風)}
\index{sauce@sauce!anglaises chaudes@---s anglaises chaudes!diable@--- Diable (Devilled Sauce)}
\index{diable@diable!sauce diable anglaise@Sauce Diable (Devilled Sauce)}
\index{anglais@anglais(e)!sauces chaudes@sauces ---es chaudes!Sauce Diable (Devilled Sauce)}

1\undemi{}
dlのヴィネガーにエシャロットのみじん切り大さじ1杯強を加えて、半量になるまで煮詰める。\protect\hyperlink{sauce-espagnole}{ソース・エスパニョル}2\undemi{}
dlとトマトピュレ大さじ2杯を加え、5分間程煮る。

仕上げに、ダービーソース\footnote{原文Derby-sauce、1940年代にアメリカで市販されていたのは確認され
  ているが、ここで言及されているのとまったく同じかは不明。なお、初版
  および第二版でこの部分は「ハーヴェイソースとウスターシャーソース各
  大さじ1杯」、第三版では「ハーヴェイソースとエスコフィエソース各大
  さじ1」となっている。「ダービーソース」が当初「エスコフィエソース」
  として商品化された後に何らかの事情により名称変更がなされたという可
  能性も否定できないが、第二版および英語版において\protect\hyperlink{sauce-diable-escoffier}{ソース・ディアー
  ブル・エスコフィエ}および\protect\hyperlink{sauce-robert-escoffier}{ソース・ロベー
  ル・エスコフィエ}、さらに第二版と同年刊の
  英語版のみに掲載されているSauce aux Cerises Escoffierソース・オ・
  スリーズ・エスコフィエのように既にエスコフィエブランドの既製品ソー
  スがあるために、矛盾が生じてしまう。第三版の記述が\protect\hyperlink{sauce-diable-escoffier}{ソース・ディアー
  ブル・エスコフィエ}を意味していると解釈す
  れば矛盾は生じないだろう。ハーヴェイソースについては\protect\hyperlink{brown-gravy}{ブラウングレ
  イヴィー}訳注参照.}大さじ1杯とカイエンヌ1つまみ強を加え、シノワ\footnote{円錐形で取っ手の付いた漉し器。}か布で漉す。

\maeaki

\hypertarget{ux30b9ux30b3ux30c3ux30c1ux30a8ux30c3ux30b0ux30bdux30fcux30b9}{%
\subsubsection{スコッチエッグソース}\label{ux30b9ux30b3ux30c3ux30c1ux30a8ux30c3ux30b0ux30bdux30fcux30b9}}

\hypertarget{scotch-eggs-sauce}{%
\paragraph{\texorpdfstring{Sauce Ecossaise (\emph{Scotch eggs
Sauce})}{Sauce Ecossaise (Scotch eggs Sauce)}}\label{scotch-eggs-sauce}}

\index{いきりすふう@イギリス風!そーすおんせい@---ソース(温製)!すこつちえつくそーす@スコッチエッグソース}
\index{そーす@ソース!いきりすふうおんせい@イギリス風---(温製)!すこつとらんといきりすふう@スコッチエッグ---}
\index{すこつとらんと@スコットランド!すこつちえつくそーす@スコッチエッグソース(イギリス風)}
\index{sauce@sauce!anglaises chaudes@---s anglaises chaudes!ecossaise@--- Ecossaise (Scotch eggs Sauce)}
\index{scotland@Scotland!sauce ecossaise@Sauce Ecossaise (Scotch eggs Sauce)}
\index{anglais@anglais(e)!sauces chaudes@sauces ---es chaudes!Sauce Ecossaise (Scotch eggs Sauce)}

バター60 gと小麦粉30 g、沸かした牛乳4
dlで\protect\hyperlink{sauce-bechamel}{ベシャメルソース}を用意する。味付けは通常どおりにすること。ソースが沸騰したらすぐに、固茹で卵の白身4個を薄切りにした\footnote{émincer
  エマンセ、薄切りにすること。}ものを加える。

提供直前に、茹で卵の卵黄を目の粗い漉し器で漉したものを混ぜ込む。

\ldots{}\ldots{}\ruby{鱈}{たら}には欠かせないソース。

\maeaki

\hypertarget{ux30d5ux30a7ux30f3ux30cdux30eb30ux30bdux30fcux30b9}{%
\subsubsection[フェンネルソース]{\texorpdfstring{フェンネル\footnote{日本語でフェンネルと呼ばれるものは、(a)主に香草として葉を利用するタイプfenouil
  sauvage(フヌイユソヴァージュ)と、(b)白く肥大した株元を食用とするフローレンス・フェンネルfenouil
  de florence(フヌイユ・ド・フロロンス)またはfenouil
  bulbeux(フヌイユビュルブー)と呼ばれる2種がある。本書ではどちらを用いるのか明記されていないことが多いが、一般に、葉を利用するタイプは香りが非常に強く、フローレンスフェンネルの葉も食用可能だが、香りは比較的おとなしい。}ソース}{フェンネルソース}}\label{ux30d5ux30a7ux30f3ux30cdux30eb30ux30bdux30fcux30b9}}

\hypertarget{fennel-sauce}{%
\paragraph{\texorpdfstring{Sauce au Fenouil (\emph{Fennel
Sauce})}{Sauce au Fenouil (Fennel Sauce)}}\label{fennel-sauce}}

\index{いきりすふう@イギリス風!そーすおんせい@---ソース(温製)!ふえんねる@フェンネルソース}
\index{そーす@ソース!いきりすふうおんせい@イギリス風---(温製)!ふえんねる@フェンネル---}
\index{ふえんねる@フェンネル!そーすいきりすふう@フェンネルソース(イギリス風)}
\index{sauce@sauce!anglaises chaudes@---s anglaises chaudes!fenouil@--- au Fenouil (Fennel Sauce)}
\index{fenouil@fenouil!sauce anglaise@Sauce au Fenouil (Fennel Sauce)}
\index{anglais@anglais(e)!sauces chaudes@sauces ---es chaudes!fenouil@Sauce au Fenouil (Fennel Sauce)}

普通に作った\protect\hyperlink{butter-sauce}{バターソース}2\undemi{}
dlあたり、細かく刻んで下茹でしたフェンネル大さじ1杯を加える。

\ldots{}\ldots{}このソースは主として、グリルあるいは茹でた鯖に合わせる。

\maeaki

\hypertarget{ux30b0ux30fcux30baux30d9ux30eaux30fcux30bdux30fcux30b9}{%
\subsubsection{グーズベリーソース}\label{ux30b0ux30fcux30baux30d9ux30eaux30fcux30bdux30fcux30b9}}

\hypertarget{gooseberry-sauce}{%
\paragraph{\texorpdfstring{Sauce aux Groseilles (\emph{Gooseberry
Sauce})}{Sauce aux Groseilles (Gooseberry Sauce)}}\label{gooseberry-sauce}}

\index{いきりすふう@イギリス風!そーすおんせい@---ソース(温製)!くーすへりーそーす@グーズベリーソース}
\index{そーす@ソース!いきりすふうおんせい@イギリス風---(温製)!くーすへりーいきりすふう@グーズベリー---}
\index{くーすへりー@グーズベリー!そーすいきりすふう@グーズベリーソース(イギリス風)}
\index{すくり@すぐり!そーす@ソース!くーすへりーそーすいきりすふう@グーズベリーソース(イギリス風)}
\index{くろせいゆ@グロゼイユ!そーす@ソース!くーすへりーそーすいきりすふう@グーズベリーソース(イギリス風)}
\index{sauce@sauce!anglaises chaudes@---s anglaises chaudes!groseilles@--- aux Groseilles (Gooseberry Sauce)}
\index{groseille@groseille!sauce anglaise@Sauce aux Groseilles (Gooseberry Sauce)}
\index{anglais@anglais(e)!sauces chaudes@sauces ---es chaudes!groseilles@Sauce aux Groseilles (Gooseberry Sauce)}

グーズベリー1 Lの皮を剥いて洗い、砂糖125 gと水1
dlを加えて火にかける。目の細かい漉し器で裏漉しする。

\ldots{}\ldots{}このピュレはグリルした鯖に合わせる。

\maeaki

\hypertarget{ux30edux30d6ux30b9ux30bfux30fcux30bdux30fcux30b9}{%
\subsubsection{ロブスターソース}\label{ux30edux30d6ux30b9ux30bfux30fcux30bdux30fcux30b9}}

\hypertarget{lobster-sauce}{%
\paragraph{\texorpdfstring{Sauce Homard à l'anglaise (\emph{Lobster
Sauce})}{Sauce Homard à l'anglaise (Lobster Sauce)}}\label{lobster-sauce}}

\index{いきりすふう@イギリス風!そーすおんせい@---ソース(温製)!ろふすたーそーす@ロブスターソース}
\index{そーす@ソース!いきりすふうおんせい@イギリス風---(温製)!ろふすたーいきりすふう@ロブスター---}
\index{ろふすたー@ロブスター!そーすいきりすふう@ロブスターソース(イギリス風)}
\index{おまーる@オマール!そーす@ソース!ろふすたーそーすいきりすふう@ロブスターソース(イギリス風)}
\index{sauce@sauce!anglaises chaudes@---s anglaises chaudes!homard@--- Homard à l'anglaise (Lobster Sauce)}
\index{homard@homard!sauce anglaise@Sauce Homard à l'anglaise (Lobster Sauce)}
\index{anglais@anglais(e)!sauces chaudes@sauces ---es chaudes!homard@Sauce Homard à l'anglaise (Lobster Sauce)}

カイエンヌを加えて風味を引き締めた\protect\hyperlink{sauce-bechamel}{ベシャメルソース}1
Lに、アンチョビエッセンス大さじ1杯と、さいの目に切ったオマールの尾の身100
gを加える\footnote{ホワイト系派生ソースの節にある\protect\hyperlink{sauce-homard}{ソース・オマール}
  を比較すると、このソースのシンプルさが際立って見えるが、ベシャメル
  を基本ソースにしている点で、やはり「ソースの体系」に組込まれたもの
  であり、純粋にイギリス料理由来というわけでもないと思われる。なお、
  このレシピは初版からほぼ異同がなく、1907年の英語版には含まれていな
  い。}。

\ldots{}\ldots{}魚料理用。

\maeaki

\hypertarget{ux7261ux8823ux5165ux308aux30bdux30fcux30b9}{%
\subsubsection{牡蠣入りソース}\label{ux7261ux8823ux5165ux308aux30bdux30fcux30b9}}

\hypertarget{oyster-sauce}{%
\paragraph{\texorpdfstring{Sauce aux Huîtres (\emph{Oyster
Sauce})}{Sauce aux Huîtres (Oyster Sauce)}}\label{oyster-sauce}}

\index{いきりすふう@イギリス風!そーすおんせい@---ソース(温製)!かきいりそーす@牡蠣入りソース}
\index{そーす@ソース!いきりすふうおんせい@イギリス風---(温製)!かきいりいきりすふう@牡蠣入り---}
\index{かき@牡蠣!そーすいきりすふう@牡蠣入りソース(イギリス風)}
\index{sauce@sauce!anglaises chaudes@---s anglaises chaudes!huitres@--- aux huitres (Oyster Sauce)}
\index{huitre@huître!sauce anglaise@Sauce aux Huîtres (Oyster Sauce)}
\index{anglais@anglais(e)!sauces chaudes@sauces ---es chaudes!huitre@Sauce aux Huîtres (Oyster Sauce)}

バター20 gと小麦粉15 gでブロンドのルーを作る。

このルーを、牛乳1 dlと生クリーム1 dlで溶く。塩1つまみを加えて調味し、
火にかけて沸騰させたら弱火にして10分間煮る。

布で漉し、カイエンヌを加えて風味を引き締める。沸騰しない程度の温度で火
を通して周囲をきれいに掃除した牡蠣の身12個を1 cm程度の厚さに切って、ソー
スに加える。

\ldots{}\ldots{}もっぱら茹でた魚\footnote{初版および第二版では「もっぱら茹でた生鱈に合わせる」とある。こ
  のレシピも1907年の英語版には掲載されていない。}に添える。

\maeaki

\hypertarget{ux7261ux8823ux5165ux308aux30d6ux30e9ux30a6ux30f3ux30bdux30fcux30b9}{%
\subsubsection{牡蠣入りブラウンソース}\label{ux7261ux8823ux5165ux308aux30d6ux30e9ux30a6ux30f3ux30bdux30fcux30b9}}

\hypertarget{brown-oyster-sauce}{%
\paragraph{\texorpdfstring{Sauce brune aux Huîtres (\emph{Brown Oyster
Sauce})}{Sauce brune aux Huîtres (Brown Oyster Sauce)}}\label{brown-oyster-sauce}}

\index{いきりすふう@イギリス風!そーすおんせい@---ソース(温製)!かきいりふらうんそーす@牡蠣入りブラウンソース}
\index{そーす@ソース!いきりすふうおんせい@イギリス風---(温製)!かきいりふらうんいきりすふう@牡蠣入りブラウン---}
\index{かき@牡蠣!そーすふらうんいきりすふう@牡蠣入りブラウンソース(イギリス風)}
\index{sauce@sauce!anglaises chaudes@---s anglaises chaudes!brune huitres@--- brune aux huitres (Brown Oyster Sauce)}
\index{huitre@huître!sauce brune anglaise@Sauce brune aux Huîtres (Brown Oyster Sauce)}
\index{anglais@anglais(e)!sauces chaudes@sauces ---es chaudes!brune huitre@Sauce brune aux Huîtres (Brown Oyster Sauce)}

上記の牡蠣入りソースと作り方はまったく同じだが、牛乳と生クリームではな
く、\protect\hyperlink{fonds-brun}{茶色いフォン}2 dlを使うこと。

\ldots{}\ldots{}このソースは、グリル焼きした肉や、肉のプディング\footnote{本書にはイギリス風の肉料理としてのプディングのレシピも掲載され
  ている。\protect\hyperlink{beefteak-pudding}{ビーフステークのプディング}、\protect\hyperlink{beefsteak-and-kidney-pudding}{ビーフ
  ステークとキドニーのプディング}、
  \protect\hyperlink{beefsteak-and-oysters-pudding}{ビーフステークと牡蠣のプディング}。
  なお、本書でのbeefsteakビーフステークとは肉の切り方のことを意味し
  ており、グリル焼きあるいはソテーしたもののことではない。ここでは厚 さ1
  cm程度にスライスした牛肉のことを指している。}、生鱈のグリル焼きに合わせる。

\maeaki

\hypertarget{ux30d6ux30e9ux30a6ux30f3ux30b0ux30ecux30a4ux30f4ux30a3ux30fc}{%
\subsubsection{ブラウングレイヴィー}\label{ux30d6ux30e9ux30a6ux30f3ux30b0ux30ecux30a4ux30f4ux30a3ux30fc}}

\hypertarget{brown-gravy}{%
\paragraph{\texorpdfstring{Jus coloré (\emph{Brown
Gravy})}{Jus coloré (Brown Gravy)}}\label{brown-gravy}}

\index{いきりすふう@イギリス風!そーすおんせい@---ソース(温製)!ふらうんくれいういー@ブラウングレイビヴィー}
\index{そーす@ソース!いきりすふうおんせい@イギリス風---(温製)!ふらうんくれいういー@ブラウングレイヴィー(イギリス風)}
\index{くれいういー@グレイヴィー!そーすふらうんいきりすふう@ブラウングレイヴィー(イギリス風ソース)}
\index{sauce@sauce!anglaises chaudes@---s anglaises chaudes!jus colore@Jus coloré (Brown Gravy)}
\index{gravy@gravy!jus colore anglaise@Jus coloré (Brown Gravy)}
\index{anglais@anglais(e)!sauces chaudes@sauces ---es chaudes!jus colore@Jus coloré (Brown Gravy)}

\protect\hyperlink{butter-sauce}{イギリス風バターソース}4
dlに、ローストの肉汁2 dlとケチャップ\footnote{ここではマッシュルームケチャップのこと。マッシュルームの薄切り
  を塩、こしょう、香辛料で5〜6日漬け込み、その絞り汁を沸かして香辛料
  とトマトを加えて味を調え、漉してから保存する (『ラルース・ガストロ
  ノミック』初版)。なお、ketchupは語源が、中国福建省アモイの方言で、
  香辛料を加えて醗酵させた魚醤の一種を意味するのkôe-chiapまたは
  kê-chiap(鮭汁)だとされている。これがマレー語に伝播し、kecap(発
  音はケーチャプ)と変化し、17世紀頃、現在のシンガポールおよびマレー
  シアを植民地支配していたイギリス人の知るところとなった。イギリスに
  も古くから魚醤の類はあり、そのバリエーションのひとつとして、マッシュ
  ルームとエシャロットを添加した魚醤をketchupと呼ぶようになった。や
  がて魚醤文化の衰退とともに、ケチャップと呼ばれるものはマッシュルー
  ムが主原料となり、いわゆるマッシュルームケチャップが18世紀頃に成立
  したとされる。これは、塩漬けにして醗酵させたマッシュルームの搾り汁
  にメース、ナツメグ、こしょうなどの香辛料を加えて煮詰め、漉したもの。
  これにトマトを添加するようになった時期は判然としないが、おそらくは
  19世紀初頭だったと思われる。フランスの料理書では1814年刊ボヴィリエ
  『調理技術』第1巻に作り方が詳述されているが(p.72)、トマトは用いな
  いマッシュルームケチャップのバリエーション。トマトを主原料としたケ
  チャップは、アメリカのハインツHeinzが1876年にハインツ・トマトケチャッ
  プを製品化して以降、徐々に広まっていった。このため、英語圏で成立、
  普及したトマトケチャップがフランスにおいて知られるようになるのは、
  少なくとも上記『ラルース・ガストロノミック』初版(1938年)よりも後
  のことであり、おそらくは第二次大戦後だろうと思われる。なお、いわゆ
  るマッシュルーム(和名バフンタケ)の人工栽培は、17世紀に流行した食
  材のひとつmelonムロン(日本語ではメロンだが、甘さの際だった品種は
  少なく、むしろ香りが特徴)を少しでも早く収穫できるようにと、地温を
  上げるために畑の表面に厩肥(醗酵熱で積み方によっては60度以上にまで
  上がる)を敷き詰めたところ、大量のマッシュルームが発生することがわ
  かり、まもなく人工栽培が行なわれるようになった。17〜18世紀の料理書
  にマッシュルームが頻出するのはそれが当時「新しい」流行の食材となっ
  たためである。}大さじ\undemi{}杯、ハーヴェイソース\footnote{Herwey
  Sauce 19世紀〜20世紀前半にかけて既製品が流通していた。現
  在は商品としては存在していないと思われる。原料はアンチョビ、ヴィネ
  ガー、マッシュルームケチャップ、にんにく、大豆由来原料(詳細不明、
  おそらくは大豆レシチンすなわち大豆油かと思われる)、カイエンヌ、コ
  チニール色素などであったという。}大さじ\undemi{}杯を加える。

\ldots{}\ldots{}もっぱら仔牛のローストに添える。

\hypertarget{ux30a8ux30c3ux30b0ux30bdux30fcux30b9}{%
\subsubsection{エッグソース}\label{ux30a8ux30c3ux30b0ux30bdux30fcux30b9}}

\hypertarget{eggs-sauce}{%
\paragraph{\texorpdfstring{Sauce aux OEufs à l'anglaise (\emph{Eggs
Sauce})}{Sauce aux OEufs à l'anglaise (Eggs Sauce)}}\label{eggs-sauce}}

\index{いきりすふう@イギリス風!そーすおんせい@---ソース(温製)!えつくそーす@エッグソース}
\index{そーす@ソース!いきりすふうおんせい@イギリス風---(温製)!えつくそーす@エッグソース}
\index{たまこ@卵!そーすうーいきりすふう@エッグソース(イギリス風)}
\index{sauce@sauce!anglaises chaudes@---s anglaises chaudes!oeufs anglaise@--- aux OEufs à l'anglaise (Eggs Sauce)}
\index{oeuf@oeuf!sauce anglaise@Sauce aux OEufs à l'anglaise (Eggs Sauce)}
\index{anglais@anglais(e)!sauces chaudes@sauces ---es chaudes!oeufs anglaise@Sauce aux OEufs à l'anglaise (Eggs Sauce)}

小麦粉60 gとバター30
gで\protect\hyperlink{roux-blanc}{白いルー}を作る。あらかじめ沸かしておいた牛乳\undemi{}
Lで溶く。塩、白こしょう、ナツメグ少々で味を調える。火にかけて沸騰したら弱火にして5〜6分間煮る。

固茹で卵2個を白身、黄身ともに、さいの目に刻んでソースに加える。

\ldots{}\ldots{}ハドック\footnote{Haddock
  鱈の一種。フランス語では同じ綴りでアドックまたは églefin,
  aiglefinエーグルファンと呼ばれる。イギリスでは主に塩漬け
  を燻製にしたものを指す。}やモリュ\footnote{morue
  モリュ。干し鱈、塩鱈のこと。生のものはcabillaudカビヨと呼ばれる。}の料理に合わせるのが一般的。

\maeaki

\hypertarget{ux30a8ux30c3ux30b0ux30a2ux30f3ux30c9ux30d0ux30bfux30fcux30bdux30fcux30b9}{%
\subsubsection{エッグアンドバターソース}\label{ux30a8ux30c3ux30b0ux30a2ux30f3ux30c9ux30d0ux30bfux30fcux30bdux30fcux30b9}}

\hypertarget{eggs-and-butter-sauce}{%
\paragraph{\texorpdfstring{Sauce aux OEufs au beurre à l'anglaise
(\emph{Eggs and Butter
Sauce})}{Sauce aux OEufs au beurre à l'anglaise (Eggs and Butter Sauce)}}\label{eggs-and-butter-sauce}}

\index{いきりすふう@イギリス風!そーすおんせい@---ソース(温製)!えつくあんとはたーそーす@エッグアンドバターソース}
\index{そーす@ソース!いきりすふうおんせい@イギリス風---(温製)!えつくあんとはたーそーす@エッグアンドバターソース}
\index{たまこ@卵!そーすうーるいきりすふう@エッグアンドバターソース(イギリス風)}
\index{はたー@バター!えつくあんとはたーそーす@エッグアンドバターソース(イギリス風)}
\index{sauce@sauce!anglaises chaudes@---s anglaises chaudes!oeufs beurre fondu@--- aux OEufs au Beurre fondu (Eggs and butter Sauce)}
\index{oeuf@oeuf!sauce oeufs beurre fondu@Sauce aux OEufs au beurre fondu (Eggs and butter Sauce)}
\index{anglais@anglais(e)!sauces chaudes@sauces ---es chaudes!oeufs beurre fondu@Sauce aux OEufs au beurre fondue (Eggs and butter Sauce)}

バター250
gを溶かし、塩適量、こしょう少々、レモン\undemi{}個分の搾り汁、固茹で卵3個を熱いうちに大きめのさいの目に刻んだもの、みじん切りにして下茹でしたパセリ小さじ1杯を加える。

\ldots{}\ldots{}茹でた魚の大きな仕立ての料理\footnote{relevé
  ルルヴェ。\protect\hyperlink{releve}{第二版序文訳注2}、および\protect\hyperlink{sauce-diplomate}{ソース・ディプロマット}訳注参照。}に添える。

\maeaki

\hypertarget{ux30aaux30cbux30aaux30f3ux30bdux30fcux30b9}{%
\subsubsection{オニオンソース}\label{ux30aaux30cbux30aaux30f3ux30bdux30fcux30b9}}

\hypertarget{onions-sauce}{%
\paragraph{\texorpdfstring{Sauce aux Oignons (\emph{Onions
Sauce})}{Sauce aux Oignons (Onions Sauce)}}\label{onions-sauce}}

\index{いきりすふう@イギリス風!そーすおんせい@---ソース(温製)!おにおんそーす@オニオンソース}
\index{そーす@ソース!いきりすふうおんせい@イギリス風---(温製)!おにおんそーす@オニオンソース(イギリス風)}
\index{たまねき@玉ねぎ!そーすおにおんいきりすふう@オニオンソース(イギリス風)}
\index{sauce@sauce!anglaises chaudes@---s anglaises chaudes!oignons@--- aux Oignons (Onions Sauce)}
\index{oignon@oignon!sauce anglaise@Sauce aux Oignons (Onions Sauce)}
\index{anglais@anglais(e)!sauces chaudes@sauces ---es chaudes!oignons@Sauce aux Oignons (Onions Sauce)}

玉ねぎ200 gを薄切りにする\footnote{émincer エマンセ。}。牛乳6
dlに塩、こしょう、ナツメグを加えて玉ねぎを茹でる。

火が通ったらすぐに、玉ねぎの水気をしっかりきって、みじん切りにする。

バター40 gと小麦粉40
gで\protect\hyperlink{roux-blanc}{白いルー}を作る。これを玉ねぎを茹でた牛乳でのばす。火にかけて沸騰させ、みじん切りにした玉ねぎを加える。ソースはとても濃い状態になっていること。そのまま7〜8分煮る。

\ldots{}\ldots{}このソースは何にでも合わせられる。うさぎ、鶏、牛などの胃や腸の料理\footnote{tripes
  トリップ。主として反芻動物(すなわち牛)の胃腸の食材とし
  ての総称。日本ではTripes à la mode de Caenトリップ・アラモード・ド・
  カン(カン風トリップ煮込み)が有名だが、他にも牛、羊、豚の副生物を
  主役とした料理は非常に多い。}、茹でたマトン、ジビエのブレゼなど\ldots{}\ldots{}このソースは必ず合わせる肉の上にかけてやること\footnote{本書におけるソースは特に指示がない場合はソース入れ(saucière
  ソ シエール)で料理本体と別添して供すると考えておくといい。}。

\maeaki

\hypertarget{ux30d6ux30ecux30c3ux30c9ux30bdux30fcux30b9}{%
\subsubsection{ブレッドソース}\label{ux30d6ux30ecux30c3ux30c9ux30bdux30fcux30b9}}

\hypertarget{bread-sauce}{%
\paragraph{\texorpdfstring{Sauce au Pain (\emph{Bread
Sauce})}{Sauce au Pain (Bread Sauce)}}\label{bread-sauce}}

\index{いきりすふう@イギリス風!そーすおんせい@---ソース(温製)!ふれつとそーす@ブレッドソース}
\index{そーす@ソース!いきりすふうおんせい@イギリス風---(温製)!ふれつとそーす@ブレッドソース}
\index{はん@パン!そーすふれつといきりすふう@ブレッドソース(イギリス風)}
\index{sauce@sauce!anglaises chaudes@---s anglaises chaudes!pain@--- au Pain (Bread Sauce)}
\index{pain@pain!sauce anglaise@Sauce au Pain (Bread Sauce)}
\index{anglais@anglais(e)!sauces chaudes@sauces ---es chaudes!pain@Sauce au Pain (Bread Sauce)}

牛乳\undemi{} Lを沸かし、フレッシュなパンの白い身80
gを投入する。塩1つまみ強、クローブ1本を刺した小玉ねぎ1個、バター30
gを加える。

弱火で15分程煮る。玉ねぎを取り出し、泡立て器でソースが滑かになるまでよく混ぜる。生クリーム約1
dlを加えて仕上げる。

\ldots{}\ldots{}鶏やジビエ(鳥類)のローストに合わせる。

\hypertarget{ux539fux6ce8}{%
\subparagraph{【原注】}\label{ux539fux6ce8}}

このブレッドソースを鶏のローストに添える場合は、ローストの肉汁もソース
入れで添えること。ジビエの場合はさらに、よく乾かしたパンを揚げた「ブレッ
ドクランプス」をソース入れに入れて添えること。また、フライドポテトの皿
も添えること。

\maeaki

\hypertarget{ux30d5ux30e9ux30a4ux30c9ux30d6ux30ecux30c3ux30c9ux30bdux30fcux30b9}{%
\subsubsection{フライドブレッドソース}\label{ux30d5ux30e9ux30a4ux30c9ux30d6ux30ecux30c3ux30c9ux30bdux30fcux30b9}}

\hypertarget{fried-bread-sauce}{%
\paragraph{\texorpdfstring{Sauce au Pain frit (\emph{Fried bread
Sauce})}{Sauce au Pain frit (Fried bread Sauce)}}\label{fried-bread-sauce}}

\index{いきりすふう@イギリス風!そーすおんせい@---ソース(温製)!ふらいとふれつとそーす@フライドブレッドソース}
\index{そーす@ソース!いきりすふうおんせい@イギリス風---(温製)!ふらいとふれつとそーす@フライドブレッドソース}
\index{はん@パン!そーすふらいとふれつといきりすふう@フライドブレッドソース(イギリス風)}
\index{sauce@sauce!anglaises chaudes@---s anglaises chaudes!pain frit@--- au Pain frit (Fried bread Sauce)}
\index{pain@pain!sauce anglaise pain frit@Sauce au Pain frit (Fried bread Sauce)}
\index{anglais@anglais(e)!sauces chaudes@sauces ---es chaudes!pain frit@Sauce au Pain frit (Fried bread Sauce)}

\protect\hyperlink{}{コンソメ}2
dlに、小さなさいの目に切った脂身のないハム80
gとエシャロット2個のみじん切りを加える。弱火で10分間煮る\footnote{mijoter
  ミジョテ。弱火で煮込むこと。}。

その間に、バター50 gを熱してパンの身50
gを揚げておく。提供直前に、揚げたパンをコンソメに入れる。パセリのみじん切り1つまみとレモンの搾り汁少々で仕上げる。

\ldots{}\ldots{} このレシピは小鳥\footnote{つぐみ(grive
  グリーヴ)など小さな鳥類のローストは、下処理し
  た後に胸肉の部分を豚背脂のシートで一羽ずつ包み、数羽をまとめて串刺しにしてロー
  ストするのが一般的だった。}のロースト用。

\maeaki

\hypertarget{ux30d1ux30bbux30eaux30bdux30fcux30b9}{%
\subsubsection{パセリソース}\label{ux30d1ux30bbux30eaux30bdux30fcux30b9}}

\hypertarget{perseley-sauce}{%
\paragraph{\texorpdfstring{Sauce Persil (\emph{Persley
Sauce})}{Sauce Persil (Persley Sauce)}}\label{perseley-sauce}}

\index{いきりすふう@イギリス風!そーすおんせい@---ソース(温製)!はせりそーす@パセリソース}
\index{そーす@ソース!いきりすふうおんせい@イギリス風---(温製)!はせりそーす@パセリソース}
\index{はせり@パセリ!そーすはせりいきりすふう@パセリソース(イギリス風)}
\index{sauce@sauce!anglaises chaudes@---s anglaises chaudes!persil@--- Persil (Perseley Sauce)}
\index{persil@persil!sauce anglaise@Sauce Persil (Perseley Sauce)}
\index{anglais@anglais(e)!sauces chaudes@sauces ---es chaudes!persil@Sauce Persil (Perseley Sauce)}

\protect\hyperlink{bread-sauce}{イギリス風バターソース}\undemi{}
Lに、パセリの香りを煮出した湯\footnote{infusion アンフュジオン
  \textless{} infuser アンフュゼ(煎じる、香りなど
  を煮出す)。なお、いわゆるハーブティはthéテよりもむしろ、infusion
  と呼ばれるのが一般的。}1
dlを加える。みじん切りして下茹でした\footnote{blanchir
  ブランシール。下茹ですること。モスカールド(葉の縮れる
  タイプ)のパセリは葉が厚く固くなりやすいためにこの作業の指示が書か
  れているのだろう。新鮮で柔らかいパセリの葉であれば、細かく刻んでそ
  のまま用いた方がいい結果を得られる。}パセリの葉大さじ1杯強を加えて仕上げる。

\ldots{}\ldots{}仔牛の頭肉、仔牛の足、脳などに合わせる。

\maeaki

\hypertarget{ux9b5aux6599ux7406ux7528ux30d1ux30bbux30eaux30bdux30fcux30b9}{%
\subsubsection{魚料理用パセリソース}\label{ux9b5aux6599ux7406ux7528ux30d1ux30bbux30eaux30bdux30fcux30b9}}

\hypertarget{sauce-persil-pour-poissons}{%
\paragraph{Sauce Persil pour
Poissons}\label{sauce-persil-pour-poissons}}

\index{いきりすふう@イギリス風!そーすおんせい@---ソース(温製)!はせりそーすさかなりようりよう@パセリソース(魚料理用)}
\index{そーす@ソース!いきりすふうおんせい@イギリス風---(温製)!さかなりようりようはせりそーす@魚料理用パセリソース}
\index{はせり@パセリ!そーすはせりいきりすふうさかなりようりよう@パセリソース(魚料理用、イギリス風)}
\index{sauce@sauce!anglaises chaudes@---s anglaises chaudes!persil poissons@--- Persil pour Poissons}
\index{persil@persil!sauce anglaise poissons@Sauce Persil pour Poissons}
\index{anglais@anglais(e)!sauces chaudes@sauces ---es chaudes!persil poissons@Sauce Persil pour Poissons}

\protect\hyperlink{roux-blanc}{白いルー}60
gを、このソースを合わせる魚に火を通すのに使ったクールブイヨン\undemi{}
Lでのばす。クールブイヨンはパセリの香りをしっかり効かせたものであること。そうでない場合は、パセリの香りを煮出した湯を加えてこのソースの特徴をきちんと出してやること。

5〜6分間煮て、細かく刻んで下茹でしたパセリの葉大さじ1杯とレモン果汁少々で仕上げる。

\maeaki

\hypertarget{ux30a2ux30c3ux30d7ux30ebux30bdux30fcux30b9}{%
\subsubsection{アップルソース}\label{ux30a2ux30c3ux30d7ux30ebux30bdux30fcux30b9}}

\index{いきりすふう@イギリス風!そーすおんせい@---ソース(温製)!あつふるそーす@アップルソース}
\index{そーす@ソース!いきりすふうおんせい@イギリス風---(温製)!あつふるそーす@アップルソース}
\index{りんこ@リンゴ!あつふるそーす@アップルソース(イギリス風)}
\index{sauce@sauce!anglaises chaudes@---s anglaises chaudes!pommes@ aux Pommes (Apple Sauce)}
\index{pomme@pomme!sauce anglaise@Sauce aux Pommes (Apple Sauce)}
\index{anglais@anglais(e)!sauces chaudes@sauces ---es chaudes!pommes@Sauce aux Pommes (Apple Sauce)}

普通にリンゴのマーマレードを作る。砂糖ごく少なめにし、シナモンの粉末を
ほんの少量加えること。\ldots{}\ldots{}これを提供直前に泡立て器で滑らかになるまでよ
く混ぜる。

\ldots{}\ldots{}このマーマレードは微温い温度で供する。鴨、がちょう、豚のローストなど、何にでも合う。

\hypertarget{ux539fux6ce8-1}{%
\subparagraph{【原注】}\label{ux539fux6ce8-1}}

ある種のローストにこのマーマレードを添えるというのは、とくにイギリスに
限ったものではない。ドイツ、ベルギー、オランダでも同様に行なわれている
ことだ。

これらの国では、ジビエのローストにはリンゴかコケモモのマーマレード、あ
るいは果物のコンポート(冷製、温製どちらも)のいずれかを必ず添えるもの
だ\footnote{果物のコンポートへの言及は第三版から。また、1907年の英語版\emph{A
  Guide to Modern Cookery}には原注そのものがない。英語版のレシピは
  「中位の大きさのリンゴ2ポンド(約900g)を四つ割りにして皮を剥き、
  芯を取り除いて刻む。これをシチュー鍋に入れ、大さじ1杯の砂糖とシナ
  モン少々、水を大さじ2〜3杯加える。蓋をして弱火にかけて煮る。提供直
  前に泡立て器で滑らかにする。このソースは微温い温度で、鴨、がちょう、
  うさぎのローストなどに添える」(p.45)となっている。}。

\maeaki

\hypertarget{ux30ddux30fcux30c8ux30efux30a4ux30f3ux30bdux30fcux30b9}{%
\subsubsection{ポートワインソース}\label{ux30ddux30fcux30c8ux30efux30a4ux30f3ux30bdux30fcux30b9}}

\hypertarget{porto-wine-sauce}{%
\paragraph{\texorpdfstring{Sauce au Porto (\emph{Porto Wine
Sauce})}{Sauce au Porto (Porto Wine Sauce)}}\label{porto-wine-sauce}}

\index{いきりすふう@イギリス風!そーすおんせい@---ソース(温製)!ほーとわいんそーす@ポートワインソース}
\index{そーす@ソース!いきりすふうおんせい@イギリス風---(温製)!ほーとわいんそーす@ポートワインソース}
\index{ほるとしゆ@ポルト酒!ほーとわいんそーす@ポートワインソース(イギリス風)}
\index{sauce@sauce!anglaises chaudes@---s anglaises chaudes!porto@--- au Porto (Port Wine Sauce)}
\index{porto@porto!sauce anglaise@Sauce au Porto (Porto Wine Sauce)}
\index{anglais@anglais(e)!sauces chaudes@sauces ---es chaudes!porto@Sauce au Porto (Porto Wine Sauce)}

ポルト酒1\undemi{}
dlにエシャロットのみじん切り大さじ1杯とタイム1枝を加えて半量になるまで煮詰める。オレンジ2個とレモン\undemi{}個の搾り汁を加える。オレンジの外皮の硬い部分を器具でおろしたもの\footnote{zeste
  ゼスト。オレンジやレモンの外皮の硬い部分(ごく表面の部分だけ)を薄く剥いて千切りにしたり、この場合のようにrâpeラップという器具でおろして風味付けに用いる。}小さじ1杯と塩1つまみ、カイエンヌごく少量を加える。

これを布で漉し、美味しい\protect\hyperlink{jus-de-veau-lie}{とろみを付けた仔牛のジュ}5
dlを加える。

\ldots{}\ldots{}野生の鴨、その他のジビエ全般に合わせる。

\hypertarget{ux539fux6ce8-2}{%
\subparagraph{【原注】}\label{ux539fux6ce8-2}}

このイギリス料理のソースは、フランスの多くの飲食店で使われている。

\maeaki

\hypertarget{ux30dbux30fcux30b9ux30e9ux30c7ux30a3ux30c3ux30b7ux30e5ux30bdux30fcux30b9}{%
\subsubsection{ホースラディッシュソース}\label{ux30dbux30fcux30b9ux30e9ux30c7ux30a3ux30c3ux30b7ux30e5ux30bdux30fcux30b9}}

\hypertarget{horse-radish-sauce}{%
\paragraph{\texorpdfstring{Sauce Raifort chaude (\emph{Horse radish
Sauce})}{Sauce Raifort chaude (Horse radish Sauce)}}\label{horse-radish-sauce}}

\index{いきりすふう@イギリス風!そーすおんせい@---ソース(温製)!ほーすらていつしゆそーす@ホースラディッシュソース}
\index{そーす@ソース!いきりすふうおんせい@イギリス風---(温製)!ほーすらていつしゆそーす@ホースラディッシュソース}
\index{れふおーる@レフォール!ほーすらていつしゆそーす@ホースラディッシュソース(イギリス風)}
\index{sauce@sauce!anglaises chaudes@---s anglaises chaudes!raifort chaude@--- au Raifort Chaude (Horse radish Sauce)}
\index{raifort@raifort!sauce anglaise chaude@Sauce au Raifort chaude (Horse radish Sauce)}
\index{anglais@anglais(e)!sauces chaudes@sauces ---es chaudes!raifort chaude@Sauce au Raifort chaude (Horse radish Sauce)}

\protect\hyperlink{albert-sauce}{アルバートソース}の別名。

\maeaki

\hypertarget{ux30eaux30d5ux30a9ux30fcux30e0ux30bdux30fcux30b948}{%
\subsubsection[リフォームソース]{\texorpdfstring{リフォームソース\footnote{19世紀ロンドンの会員制クラブ、リフォームでフランス人料理長アレクシス・ソワイエが考案したソース。このような場合、Reformを固有名詞扱いとして英語のままとするのが現代のフランス語における考え方だが、20世紀初頭にはまだ、固有名詞さえもフランス語的に言い換えることがごく普通であった。}}{リフォームソース}}\label{ux30eaux30d5ux30a9ux30fcux30e0ux30bdux30fcux30b948}}

\hypertarget{reform-sauce}{%
\paragraph{\texorpdfstring{Sauce Réforme (\emph{Reform
Sauce})}{Sauce Réforme (Reform Sauce)}}\label{reform-sauce}}

\index{いきりすふう@イギリス風!そーすおんせい@---ソース(温製)!りふおーむそーす@リフォームソース}
\index{そーす@ソース!いきりすふうおんせい@イギリス風---(温製)!りふおーむそーす@リフォームソース}
\index{りふおーむ@リフォーム!りふおーむそーす@リフォームソース(イギリス風)}
\index{sauce@sauce!anglaises chaudes@---s anglaises chaudes!reforme@--- Réforme (Reform Sauce)}
\index{reform@reform!sauce anglaise@Sauce Réforme (Reform Sauce)}
\index{anglais@anglais(e)!sauces chaudes@sauces ---es chaudes!reform@Sauce Réforme (Reform Sauce)}

\protect\hyperlink{sauce-poivrade}{ソース・ポワヴラード}と\protect\hyperlink{sauce-demi-glace}{ソース・ドゥミグラス}を合わせ、ガルニチュールとして1〜2
mmの細さで短かめの千切り\footnote{julienne courte ジュリエンヌクルト。}にした中位のサイズのコルニション2個、固茹で卵の白身、中位の大きさのマッシュルーム2個、トリュフ20
gおよび赤く漬けた牛舌肉\footnote{langue écarlate ラングエカルラット。}を加える。

\ldots{}\ldots{}このソースは「リフォーム風」羊のコトレット\footnote{côtelette
  コトレット。羊、仔牛、仔羊の肋骨付きでカットした背肉のこと。牛の場合はcôteコットと呼ばれるが、côteという語そのものは元来「肋骨」の意。côtelette
  の -ette
  は「縮小辞」といって、より小さいものという意味を付加している。つまり、牛のcôteよりも小さいからcôteletteとなる。なおこの語が日本語の「カツレツ」の語源だといわれている。}用。

\maeaki

\hypertarget{ux30bbux30fcux30b8ux3068ux7389ux306dux304eux306eux30bdux30fcux30b9}{%
\subsubsection{セージと玉ねぎのソース}\label{ux30bbux30fcux30b8ux3068ux7389ux306dux304eux306eux30bdux30fcux30b9}}

\hypertarget{sage-and-onions-sauce}{%
\paragraph{\texorpdfstring{Sauce Sauge et Oignons (\emph{Sage and onions
Sauce})}{Sauce Sauge et Oignons (Sage and onions Sauce)}}\label{sage-and-onions-sauce}}

\index{いきりすふう@イギリス風!そーすおんせい@---ソース(温製)!せーしとたまねきのそーす@セージと玉ねぎのソース}
\index{そーす@ソース!いきりすふうおんせい@イギリス風---(温製)!せーしとたまねきのそーす@セージと玉ねぎのソース}
\index{たまねき@玉ねぎ!せーしとたまねきのそーす@セージと玉ねぎのソース(イギリス風)}
\index{せーし@セージ!せーしとたまねきのそーす@セージと玉ねぎのソース(イギリス風)}
\index{sauce@sauce!anglaises chaudes@---s anglaises chaudes!sauge oignons@--- Sauge et Oignons (Sage and onions Sauce)}
\index{oignon@oignon!sauce sauge anglaise@Sauce Sauge et Oignons (Sage and onions Sauce)}
\index{sauge@sauge!sauce oignons anglaise@Sauce Sauge et Oignons (Sage and onions Sauce)}
\index{anglais@anglais(e)!sauces chaudes@sauces ---es chaudes!sauge oignons@Sauce Sauge et Oignons (Sage and onions Sauce)}

大きめの玉ねぎ2個をオーブンで焼く。冷めたら皮を剥き、みじん切りにする\footnote{hacher
  アシェ。}。パンの身150
gを牛乳に浸してから圧しつぶして水分を抜く。これを玉ねぎにを混ぜ込む。

セージのみじん切り大さじ2杯と塩、こしょうで調味する。

\ldots{}\ldots{}これは鴨の詰め物にする。

\hypertarget{ux539fux6ce8-3}{%
\subparagraph{【原注】}\label{ux539fux6ce8-3}}

鴨をローストした際のジュを大さじ5〜6杯この詰め物に加えてソース入れで供する。

パンの身と同量の牛の脂身を茹でてみじん切りにしたものを加えることも多い。

\maeaki

\hypertarget{ux30e8ux30fcux30afux30b7ux30e3ux30fcux30bdux30fcux30b953}{%
\subsubsection[ヨークシャーソース]{\texorpdfstring{ヨークシャーソース\footnote{このレシピは初版からほぼ異同がなく(初版では「仔鴨とハムに合わせる」だったのが第二版で現在とまったく同じになることのみ)、原注もない。1907年版の英語版にも掲載されていないが、1903年アメリカ、シカゴで刊行された『\href{https://archive.org/details/stewardshandbook00whitiala}{スチュワードハンドブック}』のソースの項目のなかに、「ヨークシャーソース\ldots{}\ldots{}ハム用のオレンジソース。エスパニョル、カラント(=グロゼイユ)ゼリー、ポートワイン、オレンジジュース、茹でて千切りにしたオレンジの外皮」(p.434)とあり、エスコフィエの『料理の手引き』初版当時には既にアメリカで知られているソースであったことがわかる。ただしイギリスのヨークシャー州とどのような関係あるいはソース名の由来があるのかは不明。}}{ヨークシャーソース}}\label{ux30e8ux30fcux30afux30b7ux30e3ux30fcux30bdux30fcux30b953}}

\hypertarget{sauce-yorkshire}{%
\paragraph{Sauce Yorkshire}\label{sauce-yorkshire}}

\index{いきりすふう@イギリス風!そーすおんせい@---ソース(温製)!よーくしやーそーす@ヨークシャーソース}
\index{そーす@ソース!いきりすふうおんせい@イギリス風---(温製)!よーくしやーそーす@ヨークシャーソース}
\index{よーくしやー@ヨークシャー!よーくしやーそーす@ヨークシャーソース(イギリス風)}
\index{sauce@sauce!anglaises chaudes@---s anglaises chaudes!yorkshire@--- Yorkshire}
\index{yorkshire@Yorkshire!sauce anglaise@Sauce Yorkshire}
\index{anglais@anglais(e)!sauces chaudes@sauces ---es chaudes!yorkshire@Sauce Yorkshire}

オレンジの外皮の硬い表面だけを薄く削って細かい千切りにしたもの大さじ1杯強を、ポルト酒2
dlでしっかり茹でる。

オレンジの皮の千切りを取り出して水気をきる。ポルト酒の入った鍋に、\protect\hyperlink{sauce-espagnole}{ソース・エスパニョル}大さじ1杯強と、\protect\hyperlink{}{グロゼイユのジュレ}も大さじ1杯強を加える。粉末のシナモン少々と、カイエンヌ少々を加える。

わずかの時間、煮詰める。布で漉し、オレンジ1個の搾り汁と千切りにした皮を加えて仕上げる。

\ldots{}\ldots{}仔鴨のローストやブレゼ、およびハムのブレゼに添える。
\end{recette}\newpage
\hypertarget{ux51b7ux88fdux30bdux30fcux30b9}{%
\section{冷製ソース}\label{ux51b7ux88fdux30bdux30fcux30b9}}

\hypertarget{sauces-froides}{%
\subsection{Sauces Froides}\label{sauces-froides}}

\index{sauce@sauce!sauces froides@sauces froides}
\index{そーす@ソース!れいせいそーす@冷製ソース}
\begin{recette}
\hypertarget{ux30a2ux30a4ux30e8ux30ea2-ux30d7ux30edux30f4ux30a1ux30f3ux30b9ux30d0ux30bfux30fc}{%
\subsubsection[アイヨリ /
プロヴァンスバター]{\texorpdfstring{アイヨリ\footnote{ailloliとも綴るが、
  ail(にんにく)+
  oil(油)の合成語。19世前半紀には既にアカデミーフランセージの辞書に収録されており、広く知られていたようだ。ブイヤベースに添えるルイユとよく似ているが、ルイユがカイエンヌを加えるのに対して、こちらはにんにくと油、塩、レモン汁と少々の水だけで作る。用途も、茹でた塩鱈やじゃがいも、茹で卵、アーティチョーク、さやいんげん、などに合わせることが多い。}
/
プロヴァンスバター}{アイヨリ / プロヴァンスバター}}\label{ux30a2ux30a4ux30e8ux30ea2-ux30d7ux30edux30f4ux30a1ux30f3ux30b9ux30d0ux30bfux30fc}}

\hypertarget{sauce-aioli}{%
\paragraph{Sauce Aïoli, ou Beurre de Provence}\label{sauce-aioli}}

\index{そーす@ソース!れいせい@冷製---!あいより@アイヨリ}
\index{そーす@ソース!れいせい@冷製---!ふろふあんすはたー@プロヴァンスバター}
\index{あいより@アイヨリ}
\index{ふろふあんす@プロヴァンス!ふろふあんすはたー@プロヴァンスバター}
\index{はたー@バター!ふろふあんすはたー@プロヴァンスバター}
\index{sauce@sauce!sauce froide@sauce froide!aioli@--- Aïoli}
\index{sauce@sauce!sauce froide@sauce froide!beurre de provence@Beurre de Provence}
\index{aioli@Aïoli!sauce@Sauce ---}
\index{provence@Provence!Beurre de Provence (Aïoli)}
\index{beurre@beurre!beurre de provence@Beurre de Provence (Aïoli)}

にんにく4片(30 g)を鉢\footnote{この種の作業には、大理石製のものが伝統的によく用いられる。。}に入れて細かくすり潰す。ここに生の卵黄1個、塩1つまみを加える。混ぜながら、2\undemi{}
dlの油\footnote{原書ではとくに言及されていないが、プロヴァンス地方ではオリーブオイルを用いることが一般的。}を初めは1滴ずつ加えていき、ソースがまとまりはじめたら糸を垂らすようにして加える。この作業は鉢に入れたままで、棒をはげしく動かして行なう。

攪拌する作業の途中、レモン1個分の搾り汁と冷水大さじ\undemi{}杯を少しずつ加えて、ソースが固くなり過ぎないようにしてやること。

\hypertarget{ux539fux6ce8}{%
\subparagraph{【原注】}\label{ux539fux6ce8}}

このアイヨリソースが分離してしまいそうな時は、卵黄をさらに1個足して、
マヨネーズと場合と同様に修正すること。

\maeaki

\hypertarget{ux30a2ux30f3ux30c0ux30ebux30b7ux30a25ux98a8ux30bdux30fcux30b9}{%
\subsubsection[アンダルシア風ソース]{\texorpdfstring{アンダルシア\footnote{いうまでもなくスペインのアンダルシア地方のことだが、トマトやオリーブオイル、チョリソなどこの地方を「想起」させる食材が使われている料理などがこの名称になっている傾向がある。ところが、トマトにしろオリーブオイルにしろアンダルシア地方特有というわけではなく、アンダルシアが産地として有名なチョリソくらいしか、料理名の根拠となり得るものはない。逆に言えば、アンダルシア地方の食文化との関係は、そこに用いられている食材以外にはないものと考えてもいい。料理名に付けられた地方名がとりたてて根拠や由来のないものであることを示す一例。}風ソース}{アンダルシア風ソース}}\label{ux30a2ux30f3ux30c0ux30ebux30b7ux30a25ux98a8ux30bdux30fcux30b9}}

\hypertarget{sauce-andalouse}{%
\paragraph{Sauce Andalouse}\label{sauce-andalouse}}

\index{そーす@ソース!れいせい@冷製---!あんたるしあふう@アンダルシア風---}
\index{あんたるしあ@アンダルシア!そーす@---風ソース}
\index{そーす@ソース!あんたるしあふう@アンダルシア風---}
\index{sauce@sauce!sauce froide@sauce froide!Andalouse@--- Andalouse}
\index{sauce@sauce!andalouse@--- Andalouse}
\index{andalous@Andalous(e)!sauce@Sauce Andalouse}

ごく固く仕上げた\protect\hyperlink{mayonnaise}{ソース・マヨネーズ}\troisquarts{}
Lに、上等な赤いトマトピュレ2\undemi{}dlを加える。小さなさいの目に切ったポワヴロン\footnote{Poivron
  いわゆる日本で青果として輸入されているパプリカ(肉厚の辛くないピーマン)とほぼ同じものだが、香辛料として用いられる粉末のパプリカと混同を避けるため、あえてフランス語をそのままカタカナに訳した。}75
gを仕上げに加える。

\maeaki

\hypertarget{ux30bdux30fcux30b9ux30dcux30d8ux30dfux30a2ux306eux5a18}{%
\subsubsection{ソース・ボヘミアの娘}\label{ux30bdux30fcux30b9ux30dcux30d8ux30dfux30a2ux306eux5a18}}

\hypertarget{sauce-bohemienne}{%
\paragraph[Sauce Bohémienne]{\texorpdfstring{Sauce Bohémienne\footnote{アイルランド出身の作曲家マイケル・ウィリアム・バルフェMichael
  William Balfe (1808〜1870)のオペラ\emph{The Bohemien
  Girl}『ボヘミアの少女』のフランス語版タイトル\href{https://archive.org/details/labohmiennegrand00balf}{\emph{La
  Bohémienne}}『ラボエミエーヌ』にちなんだものと言われている。この作品はロンドンで1843年初演、1862年に四幕形式のフランス語版がパリのオペラ=コミック劇場で上演され、大ヒットしたという。この名を冠した料理はいくつかあるが、いずれもチェコのボヘミア地方とは何の関連性も認められないため、オペラの人気作品にあやかった料理名と考えるのが妥当だろう。}}{Sauce Bohémienne}}\label{sauce-bohemienne}}

\index{そーす@ソース!れいせい@冷製---!ほへみあのむすめ@---ボヘミアの娘}
\index{ほへみあ@ボヘミア!そーす@ソース・---の娘}
\index{そーす@ソース!ほへみあ@---・ボヘミアの娘}
\index{sauce@sauce!sauce froide@sauce froide!bohemienne@--- Bohémienne}
\index{sauce@sauce!bohemienne@--- Bohémienne}
\index{bohemien@bohémien(ne)!sauce@Sauce Bohémienne}

陶製の容器に、濃厚でよく冷やした\protect\hyperlink{sauce-bechamel}{ベシャメルソース}1\undemi{}
dlと卵黄4個、塩10 g、こしょう少々、ヴィネガー数滴を入れる。

泡立て器で全体をよく混ぜ、標準的なマヨネーズを作るのとまったく同じ要領で、油1
Lとエストラゴンヴィネガー大さじ2杯程を加える。

\ldots{}\ldots{}仕上げに、マスタード大さじ1杯を加える。

\maeaki

\hypertarget{ux30bdux30fcux30b9ux30b7ux30e3ux30f3ux30c6ux30a3ux30a47}{%
\subsubsection[ソース・シャンティイ]{\texorpdfstring{ソース・シャンティイ\footnote{パリ近郊の地名。詳しくはホワイト系派生ソースの\protect\hyperlink{sauce-chantilly}{ソース・シャンティイ}訳注参照。}}{ソース・シャンティイ}}\label{ux30bdux30fcux30b9ux30b7ux30e3ux30f3ux30c6ux30a3ux30a47}}

\hypertarget{sauce-chantilly-froide}{%
\paragraph{Sauce Chantilly}\label{sauce-chantilly-froide}}

\index{そーす@ソース!れいせい@冷製---!しやんていい@---・シャンティイ}
\index{しやんていい@シャンティイ!そーす@ソース・---(冷製)}
\index{そーす@ソース!しやんていい@---・シャンティイ}
\index{sauce@sauce!sauce froide@sauce froide!chantilly@--- Chantilly}
\index{sauce@sauce!chantilly@--- Chantilly (froide)}
\index{chantilly@Chantilly!sauce@Sauce --- (froide)}

酸味付けにレモンを用いて、固く仕上げた\protect\hyperlink{mayonnaise}{ソース・マヨネーズ}\troisquarts{}
Lを用意しておく。提供直前に、ごく固く泡立てた生クリーム大さじ4杯\footnote{大さじ1杯=15ccという概念にとらわれないよう注意。原文は、大きなスプーンで泡立てた生クリームをざっくりと4回加えるイメージで書かれている。本書における通常のソースの仕上り量が約1
  Lであることを考慮すると、最低でも100ml以上は加えることになるだろう。}を加える。その後、味を\ruby{調}{ととの}える。

\ldots{}\ldots{}もっぱら、アスパラガスの冷製、温製に添える。

\hypertarget{ux539fux6ce8-1}{%
\subparagraph{【原注】}\label{ux539fux6ce8-1}}

生クリームを加えるのは、このソースを使うまさにその時にすること。前もっ
て加えておくと、ソースが分離してしまう恐れがあるので注意。

\maeaki

\hypertarget{ux30b8ux30a7ux30ceux30f4ux30a1ux98a812ux30bdux30fcux30b9}{%
\subsubsection[ジェノヴァ風ソース]{\texorpdfstring{ジェノヴァ風\footnote{あまり明確な由来はないが、ジェノヴァが地中海に面した港町であり、このソースが魚料理用であるという点で一応の説明はつくだろう。}ソース}{ジェノヴァ風ソース}}\label{ux30b8ux30a7ux30ceux30f4ux30a1ux98a812ux30bdux30fcux30b9}}

\hypertarget{sauce-genoise-froids}{%
\paragraph{Sauce Génoise}\label{sauce-genoise-froids}}

\index{そーす@ソース!れいせい@冷製---!しえのうあふう@ジェノヴァ風---}
\index{しえのうあふう@ジェノヴァ風!そーす@ソース・---(冷製)}
\index{そーす@ソース!しえのうあふう@ジェノヴァ風---}
\index{sauce@sauce!sauce froide@sauce froide!genoise@--- Génoise}
\index{sauce@sauce!genoise@--- Génoise (froide)}
\index{genois@Génois(e)!sauce@Sauce ---e (froide)}

殻と皮を剥いたばかりのピスタチオ40 gと、松の実25
g、松の実がない場合はスイートアーモンド20
gを鉢に入れてよくすり潰し、冷めた\protect\hyperlink{sauce-bechamel}{ベシャメルソース}小さじ1杯程度を加えて練ってペースト状にする。これを目の細かい網で裏漉しする。陶製の容器に卵黄6個、塩1つまみ、こしょう少々を入れる。泡立て器でよく混ぜる。油1
Lと中位の大きさのレモン2個の搾り汁を少しずつ加えてよく混ぜて乳化させていく\footnote{明記されていないが、ソースをしっかりと乳化させるためには\protect\hyperlink{mayonnaise}{マヨネーズ}と同様に作業すること。}。仕上げにハーブのピュレ大さじ3杯を加える。これは、パセリの葉とセルフイユ、エストラゴン、時季が合えばサラダバーネットを同量ずつ用意し、強火で2分間下茹でしてから湯をきり、冷水にさらしてから水気を強く絞り、裏漉しして作っておく。

\ldots{}\ldots{}冷製の魚料理全般に合わせられる。

\maeaki

\hypertarget{ux30bdux30fcux30b9ux30b0ux30eaux30d3ux30c3ux30b7ux30e5}{%
\subsubsection{ソース・グリビッシュ}\label{ux30bdux30fcux30b9ux30b0ux30eaux30d3ux30c3ux30b7ux30e5}}

\hypertarget{sauce-gribiche13}{%
\paragraph[Sauce Gribiche]{\texorpdfstring{Sauce Gribiche\footnote{由来不明の語。ノルマンディ方言で「子どもを怖がらせるおばさん」
  の意味で用いられるということが分かっているのみ。19世紀後半以降に創
  案もしくは一般化したソースと思われる。本書初版には当然のように既に
  収録されており、その後の大きな異同もない。ただ、本書初版以前に出版
  された料理書においてこのソースのレシピはまだ見つかっていない。ファー
  ヴルは1905年刊『料理および食品衛生事典』第二版で「ある種のレムラー
  ドにレストランで付けられた名称」と定義し、掲載しているレシピは本書
  初版のものと大差ないが、「ウスターシャソース少々も加える」となって
  いるところが目を引く。また、1913年初版のプルーストの長編小説『失な
  われた時を求めて』の「スワン家の方へ」冒頭において「彼(=スワン)を
  招いていない夕食会のために、ソース・グリビッシュやパイナップルのサ
  ラダのレシピが必要になるや、ためらいもなく探しに行かせたりするのだっ
  た」(p.18)。もしこの語り手の記述が正確であるなら、19世紀末には広く
  知られたものであったと考えるべきだが、小説の場合は必ずしも歴史的事
  実と符号するわけではないので注意が必要。}}{Sauce Gribiche}}\label{sauce-gribiche13}}

\index{そーす@ソース!れいせい@冷製---!くりひつしゆ@---・グリビッシュ}
\index{くりひつしゆ@グリビッシュ!そーす@ソース・---(冷製)}
\index{そーす@ソース!くりひつしゆ@---・グリビッシュ}
\index{sauce@sauce!sauce froide@sauce froide!gribiche@--- Gribiche}
\index{sauce@sauce!gribiche@--- Gribiche (froide)}
\index{gribiche@!gribiche!sauce@Sauce --- (froide)}

茹であがったばかりの固茹で卵の黄身6個を陶製のボウルに入れ、マスタード
小さじ1杯、塩1つまみ強、こしょう適量を加えてよく練り、滑らかなペースト
状にする。植物油\undemi{} Lとヴィネガー大さじ1\undemi{}杯を加えながら
よく混ぜて乳化させる。仕上げに、コルニションとケイパーのみじん切り計 100
gと、パセリとセルフイユ、エストラゴンのみじん切りのミックスを大さ
じ1杯、短かめの千切りにした固茹で卵の白身3個分を加える。

\ldots{}\ldots{}冷製の魚料理に添えるのが一般的。

\maeaki

\hypertarget{ux30ecux30d5ux30a9ux30fcux30ebux98a8ux5473ux306eux30bdux30fcux30b9ux30b0ux30edux30bcux30a4ux30e6}{%
\subsubsection{レフォール風味のソース・グロゼイユ}\label{ux30ecux30d5ux30a9ux30fcux30ebux98a8ux5473ux306eux30bdux30fcux30b9ux30b0ux30edux30bcux30a4ux30e6}}

\hypertarget{sauce-groseilles-au-raifort}{%
\paragraph{Sauce Groseilles au
Raifort}\label{sauce-groseilles-au-raifort}}

\index{そーす@ソース!れいせい@冷製---!れふおーるふうみくろせいゆ@レフォール風味の---・グロゼイユ}
\index{くろせいゆ@グロゼイユ!そーすれふおーる@レフォール風味のソース・---(冷製)}
\index{れふおーる@レフォール!そーすくろせいゆ@---風味のソース・グロゼイユ}
\index{そーす@ソース!れふおーるふうみくろせいゆ@レフォール風味の---・グロゼイユ}
\index{sauce@sauce!sauce froide@sauce froide!grroseilles raifort@--- Grroseilles au Rifort}
\index{sauce@sauce!groseille@--- Groseilles au Raifort (froide)}
\index{raiforg@raifort!sauce@Sauce Groseilles au --- (froide)}
\index{groseille@!groseille!sauce@Sauce --- au Raifort (froide)}

ポルト酒1
dlにナツメグ、シナモン、塩、こしょう各1つまみを加え、を\deuxtiers{}量まで煮詰める。溶かした\protect\hyperlink{}{グロゼイユのジュレ}4
dlと細かくすりおろしたレフォール大さじ2杯を加える。

(さまざまな用途に使える)

\maeaki

\hypertarget{ux30a4ux30bfux30eaux30a2ux98a8ux30bdux30fcux30b9}{%
\subsubsection{イタリア風ソース}\label{ux30a4ux30bfux30eaux30a2ux98a8ux30bdux30fcux30b9}}

\hypertarget{sauce-italienne-froide}{%
\paragraph{Sauce Italienne}\label{sauce-italienne-froide}}

\index{そーす@ソース!れいせい@冷製---!いたりあふう@イタリア風---}
\index{いたりあふう@イタリア風!そーす@---ソース(冷製)}
\index{そーす@ソース!いたりあふうれいせい@イタリア風---(冷製)}
\index{sauce@sauce!sauce froide@sauce froide!italienne@--- Italienne}
\index{sauce@sauce!italienne@--- Italienne (froide)}
\index{italien@italien(ne)!sauce froide@Sauce ---ne (froide)}

仔牛の脳半分を、香草を効かせたクールブイヨンで火を通し、目の細かい網で
裏漉しする。同量の牛あるいは羊の脳でもいい。

裏漉ししたピュレを陶製の器に入れ、泡立て器で滑らかになるまで混ぜる。卵黄5個と塩10
g、こしょう1つまみ強、油1
Lとレモン果汁1個分でマヨネーズを作り、そこの脳のピュレを加える。パセリのみじん切り大さじ1杯強を加えて仕上げる。

\ldots{}\ldots{}このソースなどんな冷製の肉料理にも合う。

\hypertarget{ux30deux30e8ux30cdux30fcux30ba}{%
\subsubsection{マヨネーズ}\label{ux30deux30e8ux30cdux30fcux30ba}}

\hypertarget{mayonnaise}{%
\paragraph[Sauce Mayonnaise]{\texorpdfstring{Sauce Mayonnaise\footnote{このソース名の語源には諸説あり、未だ定説と呼べるものはない。
  Mayonnaise という綴りそのものは1806年のヴィアール『帝国料理の本』が初
  出で、Saumon à la Mayonnaise, Filet de Sole en Mayonnaise, Poulet en
  Mayonnaise の3つのレシピが掲載されている。そのうちのひとつ、サーモンの
  マヨネーズは、筒切りにしたサーモンを茹でて冷まし、ジュレを混ぜたマヨネー
  ズをかける、という内容であり、ソースについてはマヨネーズの項を参照となっ
  ているが、どういうわけかこの本にマヨネーズそのもののレシピはない。また、
  「鶏のマヨネーズ仕立て」におけるソースはどう見てもこんにち我々が理解し
  ているマヨネーズとまったく違い、鶏のゼラチン質を冷し固める要素として利
  用したものだ。同じヴィアールの改訂版ともいうべき『王国料理の本』(1822
  年)にはマヨネーズのレシピが掲載されている。興味深いことに「このソース
  にはいろいろな作り方がある。生の卵黄を使うもの、ジュレを使うもの、仔牛
  のグラスを使うものや仔牛の脳を使うもの」として、もっとも一般的な方法と
  して生の卵黄を使う方法が示されている。生の卵黄に攪拌しながら少しずつ油
  を加えていき、固くなってきたらヴィネガー少々を加えてコシをきる、という
  方法であり、こんにち我々の知るマヨネーズに非常に近いものとなっている。
  綴りについては、カレームはmagner(マニェ)捏ねる、という意味の動詞から
  派生したものだとして、magnonnaiseもしくはmagnionnaiseと綴るべきだと
  『パリ風料理の本』で力説している。グリモ・ド・ラ・レニェールは中世フラ
  ンス語で卵黄を意味するmoyeuの派生語としてmoyeunnaiseという綴りを使って
  いる。そのほかフランス大西洋岸の地名バイヨンヌの形容詞bayonnais(バヨ
  ネ)が語源だという説もある。綴りの起源についてある程度有力視されている
  のは、1756年にリシュリュー公爵が当時イギリスに占領されていたミノルカ島
  のマオン港 Mahon を奪取したことにちなんで、mahonnaise と名づけられたと
  いうもの。もっとも、卵黄とヴィネガーを植物油で乳化させたソースという点
  では、beurre de Provenceが1758年刊マラン『コモス神の贈り物』にPigeons,
  au beurre de Provence鳩のプロヴァンスバター添え、というレシピが掲載さ
  れている(t.2,
  pp.290-230)。これは本書『料理の手引き』における\protect\hyperlink{aioli}{アイヨリ
  / プロヴァンスバター}の作り方にやや近く、茹でたにんにくを鉢に
  入れてよくすり潰し、塩、こしょう、ケイパー、アンチョビを加えてさらにす
  り潰し、そこに油を加えて攪拌して濃度を出させる、つまり乳化させる、とい
  うもの。また、植物油ではなくバターを用いるものとして、\protect\hyperlink{sauce-hollandaise}{オランデーズソー
  ス}の原型ともいうべきレシピが1651年のラ・ヴァレー
  ヌ『フランス料理の本』に、Asperges アスパラガスのホワイトソース添え
  (p.238)として掲載されていることや、卵黄をポタージュやラグーのとろみ付
  けに使うことが古くから行なわれていたことなどを総合すると、良質のオリー
  ブオイルやひまわり油を利用しやすい環境にある南フランスの方がどちらかと
  いえば、卵黄と植物油の乳化作用を利用したソースの発達、普及しやすい環境
  にあったと想像される。なお、この『料理の手引き』では卵黄のみを用いたレ
  シピとなっているが、全卵を用いるマヨネーズが現代のフランスではむしろ一
  般的と言えるだろう。全卵を用いた場合、当然ながら黄色というより白に近い
  色合いに仕上がる。}}{Sauce Mayonnaise}}\label{mayonnaise}}

\index{そーす@ソース!れいせい@冷製---!まよねーす@マヨネーズ}
\index{まよねーす@マヨネーズ}
\index{そーす@ソース!まよねーす@マヨネーズ}
\index{sauce@sauce!sauce froide@sauce froide!mayonnaise@--- Mayonnaise}
\index{sauce@sauce!italienne@--- Mayonnaise}
\index{mayonnaise@!mayonnaise!sauce@Sauce Mayonnaise}

冷製ソースのほとんどはマヨネーズの派生ソースだから、\protect\hyperlink{sauce-espagnole}{ソース・エスパニョ
ル}や\protect\hyperlink{veloute}{ヴルテ}と同様に基本ソースと見なされる。
マヨネーズの作り方はきわめてシンプルだが、以下に述べるポイントはしっか
り頭に入れておく必要がある。

\hypertarget{ux6750ux6599ux3068ux5206ux91cf}{%
\subparagraph{材料と分量}\label{ux6750ux6599ux3068ux5206ux91cf}}

\ldots{}\ldots{}卵黄6個、「からざ」は取り除いておくこと。油1 L。塩 10
g、白こしょう1
g、ヴィネガー大さじ1\undemi{}杯または、より白い仕上りを目指す場合にはヴィネガーと同等量のレモン果汁。

\begin{enumerate}
\def\labelenumi{\arabic{enumi}.}
\item
  塩、こしょう、ヴィネガーまたはレモン果汁ほんの少々を加えて、泡立て器で卵黄を溶く。
\item
  油を最初は1滴ずつ加えていき、滑らかにまとまっり始めたら、糸を垂らすようにして油を加えていく。
\item
  何回かに分けて、ヴィネガーもしくはレモン果汁を少量ずつ加え、粘りをなくしてやる。
\item
  最後に熱い湯を大さじ3杯加える。これは乳化をしっかりさせて、作り置きしておく必要がある場合でもソースが分離しないようにするため。
\end{enumerate}

\hypertarget{ux539fux6ce8-2}{%
\subparagraph{【原注】}\label{ux539fux6ce8-2}}

\noindent 1.
卵黄だけの段階で塩こしょうをするとソースが分離してしまうのではないかというのは思い込みに過ぎず、実際に調理現場で作業している者はそう考えていない。むしろ、塩を卵黄の水分に溶かし込んでおいた方が、卵黄がまとまりやすくなることは科学的に証明されている\footnote{当時の知見であることに注意。ただ、塩が植物油に溶けにくいことは自明であるから、この方法そのものは正しいと言える。}。

\begin{enumerate}
\def\labelenumi{\arabic{enumi}.}
\setcounter{enumi}{1}
\item
  マヨネーズを作る際に、氷の上に容器を置いて作業するも間違いだ。事実はまったく逆で、冷気が伝わることがもっとも分離させてしまいやすい原因だ。寒い季節には、油はやや微温めか、せめて厨房の室温くらいにするべきだ\footnote{オリーブオイルのように、飽和温度が高い種類の油ではよく見られる現象。ひまわり油でさえも寒さで濁るので、この指摘は正しい。}。
\item
  マヨネーズが分離してしまう原因としては\ldots{}\ldots{}

  \begin{enumerate}
  \def\labelenumii{\arabic{enumii}.}
  \item
    最初に油を入れ過ぎてしまうこと。
  \item
    冷え過ぎた油を使うこと。
  \item
    卵黄の量に対して油の量が多過ぎること。卵黄1個につき油を乳化させることが出来るのは、作り置きする場合は1\troisquarts{}
    dl、すぐに使うのであれば2 dlが限度。
  \end{enumerate}
\end{enumerate}
\end{recette}\newpage
\hypertarget{ux5408ux308fux305bux30d0ux30bfux30fc}{%
\section{合わせバター}\label{ux5408ux308fux305bux30d0ux30bfux30fc}}

\vspace{0\zw}

\hypertarget{ux30b0ux30eaux30ebux30bdux30fcux30b9ux306eux88dcux52a9ux6750ux6599ux30aaux30fcux30c9ux30d6ux30ebux7528}{%
\subsection{グリル、ソースの補助材料、オードブル用}\label{ux30b0ux30eaux30ebux30bdux30fcux30b9ux306eux88dcux52a9ux6750ux6599ux30aaux30fcux30c9ux30d6ux30ebux7528}}

\vspace*{-1.5\zw}

\hypertarget{beurres-composuxe9s-pour-adjuvants-de-sauces-et-hors-doeuvre}{%
\subsection{Beurres Composés pour Adjuvants de Sauces et
Hors-d'oeuvre}\label{beurres-composuxe9s-pour-adjuvants-de-sauces-et-hors-doeuvre}}

\index{あわせはたー@合わせバター} \index{はたー@バター ⇒ 合わせバター}
\index{ふーるこんほせ@ブール・コンポゼ ⇒ 合わせバター}
\index{みつくすはたー@ミックスバター ⇒ 合わせバター}
\index{beurre@beurre!beurres composes@Beurres Composés}

\hypertarget{observation-sur-les-beurres-composes}{%
\subsection{概説}\label{observation-sur-les-beurres-composes}}

本書においてレシピを掲載している合わせバター\footnote{beurre composé
  ブール・コンポゼ。ミックスバターとも。バターはフ
  ランス食文化史において、少なくとも中世以来長く用いられてきた食材だ
  が、中世〜ルネサンスにおいては獣脂(もっぱらラード)のほうが料理に
  用いられる傾向にあった。17世紀以降はたとえばラ・ヴァレーヌ『フラン
  ス料理の本』におけるアスパラガスの白いソース添え(\protect\hyperlink{sauce-hollandaise}{ソース・オラン
  デーズ}訳注参照)のように、バターを料理に用い
  ることが中世の料理書と比較すると圧倒的に増えた。ムノンの1741年刊
  『ブルジョワ屋敷に勤める女性料理人のための本』のバターの項には「良
  質のバターを用いるのは料理でとても重要なことであり、バターが悪い匂
  いを放っているようではどんな素晴しい皿も台無しだ。料理担当の女中で
  あればこのことをよく理解しておくことと、良質なバターの価格を手に入
  れるのに金を惜しんではならないことを肝に銘じておくこと。最良のバター
  は自然な黄色をしており、白いものは大抵の場合、さして美味しくない。
  バルボットという植物から採った黄色で着色されたバターもある。こうい
  うバターの色は、自然なバターの黄色よりもくすんだもので、慣れれば簡
  単に見分けることが出来る(p.320)」。合わせバター(具体的にはアンチョ
  ビバター、エクルヴィスバターなど)への言及は1806年刊ヴィアール『帝
  国料理の本』に既に見られるが、この初版および第二版は残念なことにい
  くつかの基本的なソースなどの記述が欠落しており、そのレシピそのもの
  は記載されていない。ところで、現代フランスのバターは無塩のものと、ブル
  ターニュ産に代表される有塩のものがあり、料理および製菓では基本的に
  無塩バターを用いる。生乳をとる牛の品種や製法はさまざまだが、乳酸醗
  酵させたいわゆる醗酵バターが多い点が日本と大きく異なる。この節に限
  らず本書のレシピは、無塩バターの使用を前提にしていることに注意。}のうちのほとんどは、甲
殻類の合わせバターを除いて、料理に直接用いられることがとても少ない。だ
が、合わせバターはさまざまなシチュエーションで役に立つ。ポタージュでは
野菜の合わせバターが、その他の合わせバターはソース作りにおいて有用だ。
ソースの風味と性格を明確に伝える決め手になるからだ。

だから、読者である料理人諸君には、ここに書いてあることを真剣に読みとっ
ていただきたい。\href{原文における内容矛盾。この後のパラグラフは甲殻類の\%20バターについての注意点ばかりが目立つ}{}

甲殻類のバターについては、経験上、湯煎にかけながら煮出して\footnote{infuser
  アンフュゼ。}から、氷
水で冷やした陶製の容器に布で漉し入れるといい。そうすれば、冷たい状態で
作るよりも赤みがきれいに出る。だが逆に、熱によって風味の繊細さが失なわ
れてしまい、雑味さえも出てしまう。

この問題点を解決するために、我々は二種類の違うバターを作るという方式を
採ることにした。ひとつは甲殻類の胴のクリーム状の部分と切りくずあるいは
身そのものを生のバターとともに鉢ですり潰して、目の細かい網で裏漉しする
か、布で漉すというもの。このバターはソースに完璧ともいうべき風味を添え
てくれる。とりわけベシャメルソースをベースとしたソースの場合はそうだ。

もうひとつは、甲殻類の殻だけを用いて、熱して作るものだ。これは「色付け」
の役割しか持たない。この方式はまことに素晴しい結果を得られるので、ぜひ
とも実行していただきたい。

場合によっては、我々はバターを同様の上等な生クリームに代えることがある。
生クリームのほうがバターよりも、素材の持つ風味や香気をよく吸収する。こ
うすればソースやポタージュの仕上げに加えるのに文句ないクリ\footnote{Coulis
  水分のやや多いピュレをイメージするといい。}を作ることが 出来るわけだ。

色付け用のバターを使うと、ソースがきれいに色付き、個性的なソースとなる。
どんな場合でも、カルミン色素\footnote{コチニール色素ともいう。ラックカイガラムシなどを原料として抽出し
  た色素。ヨーロッパでは古代から中世にかけてケルメスカイガラムシから
  抽出され利用されてきた、非常に歴史の古い色素。とりわけルネサン期に
  は高級毛織物の染料として需要が高まった。また絵の具にも使用された。
  その後、ウチワサボテンでエンジムシを大量に養殖していた中南米を支配
  下に置いたスペインが、これを新大陸産のカルミンとしてヨーロッパ各国
  に売ることで巨万の富を得たという。かつて食品工業において多用された。
  1838年の『ラルース・ガストロノミック』初版では、「コチニールから抽
  出される鮮かな赤色色素で毒性はない。多くの食品に着色料として用いら
  れている」とある。現在は食物アレルギーの原因物質すなわちアレルゲン
  となり得ることがわかり、使用は減りつつある。現在は代替品としてビー
  ツから抽出したビートレッドなどの使用が増えてきている。また、この本
  文でカルミン色素の使用を「くすんだ、情けない色合いを与える」として
  否定的に扱っているのは、この色素がpHによって色調が変化し、なおかつ
  蛋白質を多く含む料理に加えると色素自体が紫色に変化する(結果として
  ソースやポタージュ全体が濁ったような色になる)ことがあるためだろう。}よりもずっといい。カルミン色素はソース
やポタージュにくすんだ、なさけない色合いしか与えてはくれないのだ。

合わせバターは一般的に、使う際にその都度作る\footnote{原文 au moment
  (オモモン)その都度、の意。à la minute (アラミニュッ
  ト)と呼ぶ調理現場もある。}ものだが、作り置き
しておかなければならない場合は、白い紙で円筒形に包んで冷蔵保管すること。

\vspace*{2\zw}
\begin{recette}
\hypertarget{ux306bux3093ux306bux304fux30d0ux30bfux30fc}{%
\subsubsection{にんにくバター}\label{ux306bux3093ux306bux304fux30d0ux30bfux30fc}}

\hypertarget{beurre-d-ail}{%
\paragraph{Beurre d'Ail}\label{beurre-d-ail}}

\index{はたー@バター!あわせはたー@合わせバター!にんにくはたー@にんにくバター}
\index{あわせはたー@合わせバター!にんにくはたー@にんにくバター}
\index{にんにく@にんにく!はたー@---バター}
\index{beurre@beurre!beurres composes@Beurres Composés!beurre d'ail@Beurre d'Ail}
\index{ail@ail!beurre@Beurre d'---}

皮を剥いたにんにく200 gを強火でしっかり茹でる\footnote{生のにんにくには胃腸を刺激する酵素が含まれているが、熱により不活
  性化するので、よく火を通す必要がある。}。よく湯をきってから、鉢
に入れてすり潰し、バター250 gと合わせ、布で漉す。

\maeaki

\hypertarget{ux30a2ux30f3ux30c1ux30e7ux30d3ux30d0ux30bfux30fc}{%
\subsubsection{アンチョビバター}\label{ux30a2ux30f3ux30c1ux30e7ux30d3ux30d0ux30bfux30fc}}

\hypertarget{beurre-d-anchois}{%
\paragraph{Beurre d'Anchois}\label{beurre-d-anchois}}

\index{はたー@バター!あわせはたー@合わせバター!あんちよひはたー@アンチョビバター}
\index{あわせはたー@合わせバター!あんちよひはたー@アンチョビバター}
\index{あんちよひ@アンチョビ!はたー@---バター}
\index{beurre@beurre!beurres composes@Beurres Composés!beurre d'anchois@Beurre d'Anchois}
\index{anchois@anchois!beurre@Beurre d'---}

アンチョビのフィレ200 gをよく洗い、しっかり水気を絞る。これを鉢に入れ
て細かくすり潰す。バター250 gを加えて布で漉す。

\maeaki

\hypertarget{ux30a2ux30fcux30e2ux30f3ux30c9ux30d0ux30bfux30fc}{%
\subsubsection{アーモンドバター}\label{ux30a2ux30fcux30e2ux30f3ux30c9ux30d0ux30bfux30fc}}

\hypertarget{beurre-d-amande}{%
\paragraph{Beurre d'Amande}\label{beurre-d-amande}}

\index{はたー@バター!あわせはたー@合わせバター!あーもんとはたー@アーモンドバター}
\index{あわせはたー@合わせバター!あーもんとはたー@アーモンドバター}
\index{あーもんと@アーモンド!はたー@---バター}
\index{beurre@beurre!beurres composes@Beurres Composés!beurre d'amande@Beurre d'Amande}
\index{amande@amande!beurre@Beurre d'---}

アーモンド\footnote{アーモンドには一般的なスイートアーモンド amandes
  doucesと、苦味 のあるビターアーモンドamande
  amèresの二種がある。後者はごく微量の
  青酸化合物を含むのであまり多く使われることはないが、香りがいいため
  リキュールなどの香り付けにごく少量が用いられることがある。}150
gを湯むきしてよく洗い、すぐに水数滴を加えてすり潰し
てペースト状にする。これをバター250 gと混ぜ合わせ、布で漉す。

\maeaki

\hypertarget{ux30d6ux30fcux30ebux30c0ux30f4ux30eaux30fcux30cc8}{%
\subsubsection[ブール・ダヴリーヌ]{\texorpdfstring{ブール・ダヴリーヌ\footnote{アヴリーヌはヘーゼルナッツの仲間でセイヨウハシバミの大粒な変種。
  イタリア、ピエモンテ産やシチリア産が有名。}}{ブール・ダヴリーヌ}}\label{ux30d6ux30fcux30ebux30c0ux30f4ux30eaux30fcux30cc8}}

\hypertarget{beurre-d-aveline}{%
\paragraph{Beurre d'Aveline}\label{beurre-d-aveline}}

\index{はたー@バター!あわせはたー@合わせバター!ふーるたうりーぬ@ブール・ダヴリーヌ}
\index{あわせはたー@合わせバター!ふーるたうりーぬ@ブール・ダヴリーヌ}
\index{あうりーぬ@アヴリーヌ!ふーる@ブール・---}
\index{へーせるなつつ@ヘーゼルナッツ!ふーるたうりーぬ@ブール・ダヴリーヌ}
\index{beurre@beurre!beurres composes@Beurres Composés!beurre d'aveline@Beurre d'Aveline}
\index{aveline@aveline!beurre@Beurre d'---}

アヴリーヌ150 gを焙煎して丁寧に皮を剥く。油が浮いてこないよう水を数滴
加えてペースト状にすり潰す。これとバター250 gを混ぜ合わせる。目の細か
い網で裏漉しするか、布で漉す。

\maeaki

\hypertarget{ux30d6ux30fcux30ebux30d9ux30ebux30b7ux30fc9}{%
\subsubsection[ブール・ベルシー]{\texorpdfstring{ブール・ベルシー\footnote{\protect\hyperlink{sauce-bercy}{ソース・ベルシー}訳注参照。}}{ブール・ベルシー}}\label{ux30d6ux30fcux30ebux30d9ux30ebux30b7ux30fc9}}

\hypertarget{beurre-bercy}{%
\paragraph{Beurre Bercy}\label{beurre-bercy}}

\index{はたー@バター!あわせはたー@合わせバター!ふーるへるしー@ブール・ベルシー}
\index{あわせはたー@合わせバター!ふーるたへるしー@ブール・ベルシー}
\index{へるしー@ベルシー!ふーる@ブール・---}
\index{beurre@beurre!beurres composes@Beurres Composés!beurre bercy@Beurre Bercy}
\index{bercy@Bercy!beurre@Beurre ---}

白ワイン2 dlに細かく刻んだエシャロット大さじ1杯を加えて半量になるまで
煮詰める。生温い程度まで冷ましてから、ポマード状に柔らかくしたバター 200
gを混ぜ込む。牛骨髄500 gをさいの目に切って\footnote{原文 couper en
  dés。フランス語のまま「デにする(切る)」と表現することもある。}、沸騰しない程度の
湯で火を通し、よく湯ぎりをして加える。パセリのみじん切り大さじ1杯と塩8
g、挽きたてのこしょう1つまみ強とレモン\undemi{}個分の果汁を加えて仕上
げる。

\maeaki

\hypertarget{ux30adux30e3ux30d3ux30a2ux30d0ux30bfux30fc}{%
\subsubsection{キャビアバター}\label{ux30adux30e3ux30d3ux30a2ux30d0ux30bfux30fc}}

\hypertarget{beurre-de-caviar}{%
\paragraph{Beurre de Caviar}\label{beurre-de-caviar}}

\index{はたー@バター!あわせはたー@合わせバター!きやひあはたー@キャビアバター}
\index{あわせはたー@合わせバター!きやひあはたー@キャビアバター}
\index{きやひあ@キャビア!はたー@---バター}
\index{beurre@beurre!beurres composes@Beurres Composés!beurre caviar@Beurre de Caviar}
\index{caviar@caviar!beurre@Beurre de ---}

圧縮キャビア\footnote{もとはロシアで雪の中の樽で保存するために圧縮したもの。キャビア
  のグレードはベルガ、オセトラ、セヴルガが混ざっているのが多いという。
  比較的安価に利用できる。}75 gを細かくすり潰す。パター250
gを加えて、布で漉す。

\maeaki

\hypertarget{ux30d6ux30fcux30ebux30b7ux30f4ux30ea12-ux30d6ux30fcux30ebux30e9ux30f4ux30a3ux30b4ux30c3ux30c813}{%
\subsubsection[ブール・シヴリ /
ブール・ラヴィゴット]{\texorpdfstring{ブール・シヴリ\footnote{\protect\hyperlink{sacue-chivry}{ソース・シヴリ}訳注参照。}
/ ブール・ラヴィゴット\footnote{\protect\hyperlink{sauce-ravigote}{ソース・ラヴィゴット}訳注参照。}}{ブール・シヴリ / ブール・ラヴィゴット}}\label{ux30d6ux30fcux30ebux30b7ux30f4ux30ea12-ux30d6ux30fcux30ebux30e9ux30f4ux30a3ux30b4ux30c3ux30c813}}

\hypertarget{beurre-chivry}{%
\paragraph{Beurre Chivry}\label{beurre-chivry}}

\index{はたー@バター!あわせはたー@合わせバター!しうり@ブール・シヴリ}
\index{あわせはたー@合わせバター!しうり@ブール・シヴリ}
\index{はたー@バター!あわせはたー@合わせバター!らういこつと@ブール・ラヴィゴット}
\index{あわせはたー@合わせバター!らういこつと@ブール・ラヴィゴット}
\index{しうり@シヴリ!ふーる@ブール・---}
\index{らういこつと@ラヴィゴット!ふーる@ブール・---}
\index{beurre@beurre!beurres composes@Beurres Composés!beurre chivry@Beurre Chivrya}
\index{beurre@beurre!beurres composes@Beurres Composés!beurre ravigote@Beurre Ravigote}
\index{chivry@Chivry!beurre@Beurre ---}
\index{ravigote@ravitote!beurre@Beurre ---}

パセリの葉とセルフイユ、エストラゴン、シヴレット、若摘みのサラダバーネッ
ト100 gを数分間下茹でし、水にさらしてから圧して余分な水気を絞る。エシャ
ロットのみじん切り25 gも下茹でする。これらを鉢に入れてすり潰す。

バター125 gを加え、布で漉す。

\maeaki

\hypertarget{ux30d6ux30fcux30ebux30b3ux30ebux30d9ux30fcux30eb14}{%
\subsubsection[ブール・コルベール]{\texorpdfstring{ブール・コルベール\footnote{\protect\hyperlink{sauce-colbert}{ソース・コルベール}本文および訳注参照。}}{ブール・コルベール}}\label{ux30d6ux30fcux30ebux30b3ux30ebux30d9ux30fcux30eb14}}

\hypertarget{beurre-colbert}{%
\paragraph{Beurre Colbert}\label{beurre-colbert}}

\index{はたー@バター!あわせはたー@合わせバター!ふーるこるへーる@ブール・コルベール}
\index{あわせはたー@合わせバター!ふーるこるへーる@ブール・コルベール}
\index{こるへーる@コルベール!ふーる@ブール・---}
\index{beurre@beurre!beurres composes@Beurres Composés!beurre colbert@Beurre Colbert}
\index{colbert@Colbert!beurre@Beurre ---}

\protect\hyperlink{beurre-maitre-d-hotel}{メートルドテルバター}200
gに、溶かした\protect\hyperlink{glace-de-viande}{グラス
ドヴィアンド}大さじ2杯と細かく刻んだエストラゴン小さ じ2杯を加える。

\maeaki

\hypertarget{ux8272ux4ed8ux3051ux7528ux306eux8d64ux3044ux30d0ux30bfux30fc}{%
\subsubsection{色付け用の赤いバター}\label{ux8272ux4ed8ux3051ux7528ux306eux8d64ux3044ux30d0ux30bfux30fc}}

\hypertarget{beurre-colorant-rouge}{%
\paragraph{Beurre Colorant rouge}\label{beurre-colorant-rouge}}

\index{はたー@バター!あわせはたー@合わせバター!いろつけようのあかいはたー@色付け用の赤いバター}
\index{あわせはたー@合わせバター!いろつけようのあかいはたー@色付け用の赤いバター}
\index{ちやくしよくそざい@着色素材!いろつけようのあかいはたー@色付け用の赤いバター}
\index{beurre@beurre!beurres composes@Beurres Composés!beurre colorant rouge@Beurre Colorant rouge}
\index{colorant@colorant!beurre rouge@Beurre --- rouge}

出来るだけ沢山の甲殻類の殻などの残りをまとめて用意する。殻の内側、外側
に張り付いている膜などをきれいに取り除く。よく乾燥させてから、鉢\footnote{伝統的には大理石製の鉢が用いられることが多かった。}
に入れて細かく粉砕して、同じ重さのバターを加える。これを湯煎にかけてよ
く混ぜながら溶かす。氷水を入れた陶製の器に、布で漉し入れる。固まったバ
ターをトーション\footnote{\protect\hyperlink{sauce-verte}{ソース・ヴェルト}訳注参照。}で包み、余計な水を絞り出す。

\hypertarget{ux539fux6ce8}{%
\subparagraph{【原注】}\label{ux539fux6ce8}}

この色付け用のバターを作るのに用いる甲殻類の殻がどうしてもない場合は、
\protect\hyperlink{beurre-de-paprika}{パプリカバター}を用いてもいいだろう。だがいずれに
せよ、どんなソースであっても、仕上りの色合いを決めるには、出来るだけ、
他の植物由来の赤色着色料の使用は避けることを勧める\footnote{この原注は第三版から。原文le
  rouge colorant végétal直訳すると
  「植物由来の赤色着色料」だが。ここではおそらくカルミン色素(コチニー
  ル色素)のことと思われる(\protect\hyperlink{observation-sur-les-beurres-composes}{本節「概説」参
  照})。他に赤系着色料として、
  ベニバナ色素、紅麹などもあるが、いずれも中国や日本において発達しこ
  とを考慮すると、両大戦間である1920年頃に「避けるべき」というほど普
  及していたのは、実際には昆虫由来であるコチニール色素と思われる。な
  お、ベニバナ色素も化学的にはカルミン酸色素。また、甲殻類の殻を茹で
  ると赤くなるが、この色素はアスタキサンチンといい、1938年に物質とし
  て「発見」された。もちろんエスコフィエをはじめとする料理人は経験上、
  甲殻類の殻を適度に加熱することで、タンパク質と結びついていたアスタ
  キサンチンがタンパク質の熱変性によって遊離して取り出せることを経験
  的によく知っており、それを利用してこの赤いバターを考案したと考えら
  れる。ちなみにサーモン、鮭の身の赤色もおなじアスタキサンチンによる
  もので、近縁種の鱒と同様に本来は白身。}。

\maeaki

\hypertarget{ux8272ux4ed8ux3051ux7528ux306eux7dd1ux306eux30d0ux30bfux30fc}{%
\subsubsection{色付け用の緑のバター}\label{ux8272ux4ed8ux3051ux7528ux306eux7dd1ux306eux30d0ux30bfux30fc}}

\hypertarget{beurre-colorant-vert}{%
\paragraph{Beurre Colorant vert}\label{beurre-colorant-vert}}

\index{はたー@バター!あわせはたー@合わせバター!いろつけようのみとりのはたー@色付け用の緑のバター}
\index{あわせはたー@合わせバター!いろつけようのみとりのはたー@色付け用の緑のバター}
\index{ちやくしよくそざい@着色素材!いろつけようのみとりのはたー@色付け用の緑のバター}
\index{beurre@beurre!beurres composes@Beurres Composés!beurre colorant vert@Beurre Colorant vert}
\index{colorant@colorant!beurre vert@Beurre --- vert}

ほうれんそうの葉1
kgをよく洗い、しっかり振って水気をきる。これを鉢に入れてすり潰す。トーション\footnote{\protect\hyperlink{sauce-verte}{ソース・ヴェルト}訳注参照。}で包んで緑の汁を絞り出す。これをソテー鍋に入れて湯煎にかけ、水分を蒸発させてペースト状にする\footnote{原文
  coaguler 凝固させる、の意。ここでは説明的に意訳した。なお、
  ほうれんそうに限らず、植物の緑色は葉緑素(クロロフィル)によるもの
  であり、葉緑素はマグネシウム(苦土)を核として窒素が周囲に結びつい
  た構造を持つ化学物質。ほうれんそうの緑が濃いのは土壌からのマグネシ
  ウム吸収能力が高いため。食品に含まれるマグネシウムはカルシウムの吸
  収を促す作用があるとされている。ただし、フランスの伝統的なほうれん
  そうの栽培方法は、夏の終わりから初秋にかけた種を蒔き、11月頃から大
  きくなった葉を順次かき取って収穫するというもの。露地栽培でも1株で3
  回程度は春になるまでに収穫できるとされた。いっぽう、日本のほうれん
  そうはごく一部の地域を除いては、戦後高度成長期に普及した葉菜のひと
  つであり、じつのところ歴史は浅い。しばしば言われる東洋系、西洋系の
  違いにしても、普及当初にはその高配品種が使われるようになっていたた
  めに、あまり意味はない。日本で青果として流通しているほうれんそうの
  ほとんどは、密植、立性にして比較的若どり(農協などの出荷団体によっ
  て違うが、概ね草丈25cm程度で5株から10株で200gの規格が平均的)のた
  め、用いている品種がほぼ西洋系のものを交配親としている場合でも、立
  性に栽培するために、葉の厚みなどは問題とされていない。フランスでは
  かつて、葉以外は可食部として見なされず、軸は切り捨てるのが普通だっ
  たことと比べると、食文化の違いの大きさがよくわかる一例だろう。}。

これを、ぴんと張ったナフキンの上に移し、さらに水気をきる。

パレットナイフを使って緑の色素を集め、鉢に入れてその倍の重さのバターを加えて練り込む。

\ldots{}\ldots{}布で漉し、冷蔵保存する。

\hypertarget{ux539fux6ce8-1}{%
\subparagraph{【原注】}\label{ux539fux6ce8-1}}

人工的な色素よりもこの緑の色素を用いたようが利点が大きい。

\maeaki

\hypertarget{ux30afux30ebux30f4ux30a7ux30c3ux30c8ux30d0ux30bfux30fc}{%
\subsubsection{クルヴェットバター}\label{ux30afux30ebux30f4ux30a7ux30c3ux30c8ux30d0ux30bfux30fc}}

\hypertarget{beurre-de-crevettes}{%
\paragraph{Beurre de crevettes}\label{beurre-de-crevettes}}

\index{はたー@バター!あわせはたー@合わせバター!くるうえつとはたー@クルヴェットバター}
\index{あわせはたー@合わせバター!くるうえつとはたー@クルヴェットバター}
\index{クルウエツト@クルヴェット!はたー@---バター}
\index{beurre@beurre!beurres composes@Beurres Composés!beurre crevette@Beurre de Crevette}
\index{crevette@crevette!beurre@Beurre de ---}

クルヴェッット・グリーズ\footnote{フランスで好んで食される小海老の一種。\protect\hyperlink{sauce-aux-crevettes}{ソース・クルヴェット}訳注参照。}150
gを鉢に入れて細かくすり潰す。バター150 gを加えて、布で漉す。

\maeaki

\hypertarget{ux30a8ux30b7ux30e3ux30edux30c3ux30c8ux30d0ux30bfux30fc}{%
\subsubsection{エシャロットバター}\label{ux30a8ux30b7ux30e3ux30edux30c3ux30c8ux30d0ux30bfux30fc}}

\hypertarget{beurre-d-echalote}{%
\paragraph{Beurre d'Echalote}\label{beurre-d-echalote}}

\index{はたー@バター!あわせはたー@合わせバター!えしやろつとはたー@エシャロットバター}
\index{あわせはたー@合わせバター!えしやろつとはたー@エシャロットバター}
\index{えしやろつと@エシャロット!はたー@---バター}
\index{beurre@beurre!beurres composes@Beurres Composés!beurre echalote@Beurre d'Echalote}
\index{echalote@echalote!beurre@Beurre de ---}

エシャロット125
gを鉢に入れてすり潰し、さっと茹でて湯をきり、トーションに包んで圧すようにして水気を取り除く。バター125gを加えて、布で漉す。

\maeaki

\hypertarget{ux30a8ux30afux30ebux30f4ux30a3ux30b9ux30d0ux30bfux30fc}{%
\subsubsection{エクルヴィスバター}\label{ux30a8ux30afux30ebux30f4ux30a3ux30b9ux30d0ux30bfux30fc}}

\hypertarget{beurre-d-ecrevisse}{%
\paragraph{Beurre d'Ecrevisse}\label{beurre-d-ecrevisse}}

\index{はたー@バター!あわせはたー@合わせバター!えくるういすはたー@エクルヴィスバター}
\index{あわせはたー@合わせバター!えくるういすはたー@エクルヴィスバター}
\index{えくるういす@エクルヴィス!はたー@---バター}
\index{beurre@beurre!beurres composes@Beurres Composés!beurre ecrevisse@Beurre d'Ecrevisse}
\index{ecrevisse@ecrevisse!beurre@Beurre d'---}

\protect\hyperlink{}{ビスク}を作る要領で、\protect\hyperlink{mirepoix}{ミルポワ}とともに茹でたエクルヴィス\footnote{ヨーロッパざりがに。詳しくは\protect\hyperlink{sauce-bavaroise}{バイエルン風ソース}訳注参照。}の胴や殻、
尾などを鉢に入れて細かくすり潰す。これと同じ重さのバターを加え、布で漉
す。

\maeaki

\hypertarget{ux30a8ux30b9ux30abux30ebux30b4ux7528ux30d0ux30bfux30fc}{%
\subsubsection{エスカルゴ用バター}\label{ux30a8ux30b9ux30abux30ebux30b4ux7528ux30d0ux30bfux30fc}}

\hypertarget{beurre-pour-les-escargots}{%
\paragraph{Beurre pour les Escargots}\label{beurre-pour-les-escargots}}

\index{はたー@バター!あわせはたー@合わせバター!えすかるこようはたー@エスカルゴ用バター}
\index{あわせはたー@合わせバター!えすかるこようはたー@エスカルゴ用バター}
\index{えすかるこ@エスカルゴ!はたー@---用バター}
\index{beurre@beurre!beurres composes@Beurres Composés!beurre escargots@Beurre pour les Escargots}
\index{escargot@escargot!beurre@Beurre pour les ---}

(エスカルゴ50個分)

バター350 gに、細かいみじん切りにしたエシャロット35 gと、にんにく1片を
すり潰してペースト状にしたもの、パセリのみじん切り25g(大さじ1杯)、塩
12 g、こしょう2 gを加える。捏ねるようにしてよく混ぜ合わせ、冷蔵する。

\maeaki

\hypertarget{ux30a8ux30b9ux30c8ux30e9ux30b4ux30f3ux30d0ux30bfux30fc}{%
\subsubsection{エストラゴンバター}\label{ux30a8ux30b9ux30c8ux30e9ux30b4ux30f3ux30d0ux30bfux30fc}}

\hypertarget{beurre-d-estragon}{%
\paragraph{Beurre d'Estragon}\label{beurre-d-estragon}}

\index{はたー@バター!あわせはたー@合わせバター!えすとらこんはたー@エストラゴンバター}
\index{あわせはたー@合わせバター!えすとらこんはたー@エストラゴンバター}
\index{えすとらこん@エストラゴン!はたー@---バター}
\index{beurre@beurre!beurres composes@Beurres Composés!beurre estragon@Beurre d'Estragon}
\index{estragon@estragon!beurre@Beurre d'---}

新鮮なエスゴラゴンの葉125 gを2分間茹がいてから湯きりして冷水にさらす。
圧して余分な水気を絞る。これを鉢に入れてすり潰す。バター125 gを加えて、
布で漉す。

\maeaki

\hypertarget{ux306bux3057ux3093ux30d0ux30bfux30fc}{%
\subsubsection{にしんバター}\label{ux306bux3057ux3093ux30d0ux30bfux30fc}}

\hypertarget{beurre-de-hereng}{%
\paragraph{Beurre de Hareng}\label{beurre-de-hereng}}

\index{はたー@バター!あわせはたー@合わせバター!にしんはたー@にしんバター}
\index{あわせはたー@合わせバター!にしんはたー@にしんバター}
\index{にしん@にしん!はたー@---バター}
\index{beurre@beurre!beurres composes@Beurres Composés!beurre hareng@Beurre de Hareng}
\index{hareng@hareng!beurre@Beurre de ---}

にしんの燻製のフィレ\footnote{原文 hareng
  saur(アロンソール)。タイセイヨウニシンの内臓を抜
  いて10日程塩漬けにし、塩抜き後に24〜48時間乾燥させてから15時間以上、
  32℃程度で冷燻にしたもの。強い匂いが特徴。日本のにしんとは種が異な
  ること、スモークサーモンと同様に冷燻であることに注意。}3枚の皮を剥いて、さいの目に切り、鉢に入れて細かくすり潰す。バター250
gを加え、布で漉す。

\maeaki

\hypertarget{ux30aaux30deux30fcux30ebux30d0ux30bfux30fc}{%
\subsubsection{オマールバター}\label{ux30aaux30deux30fcux30ebux30d0ux30bfux30fc}}

\hypertarget{beurre-de-homard}{%
\paragraph{Beurre de Homard}\label{beurre-de-homard}}

\index{はたー@バター!あわせはたー@合わせバター!おまーるはたー@オマールバター}
\index{あわせはたー@合わせバター!おまーるはたー@オマールバター}
\index{おまーる@オマール!はたー@---バター}
\index{beurre@beurre!beurres composes@Beurres Composés!beurre homard@Beurre de Homard}
\index{homard@homard!beurre@Beurre de ---}

使える範囲の量のオマールの胴のクリーム状の部分と卵やコライユ\footnote{オマールの胴の背側にある朱色の内子。}を鉢
に入れてすり潰す。それと同じ重さのバターを加え、布で漉す。

\maeaki

\hypertarget{ux767dux5b50ux30d0ux30bfux30fc}{%
\subsubsection{白子バター}\label{ux767dux5b50ux30d0ux30bfux30fc}}

\hypertarget{beurre-de-laitance}{%
\paragraph{Beurre de Laitance}\label{beurre-de-laitance}}

\index{はたー@バター!あわせはたー@合わせバター!しらこはたー@白子バター}
\index{あわせはたー@合わせバター!しらこはたー@白子バター}
\index{しらこ@白子!はたー@---バター}
\index{beurre@beurre!beurres composes@Beurres Composés!beurre laitance@Beurre de Laitance}
\index{laitance@laitance!beurre@Beurre de ---}

沸騰しない程度の温度で茹で、よく冷ました白子\footnote{日本ではスケトウダラの白子が一般的だが、フランスの伝統的高級料理では鯉
  の白子がもっとも一般的。他に鯖やにしんの白子も用いられる。}125g
を鉢に入れてすり潰す。 バター250
gとマスタード小さじ1杯を加えて、布で漉す。

\maeaki

\hypertarget{ux30e1ux30fcux30c8ux30ebux30c9ux30c6ux30eb25ux30d0ux30bfux30fc26}{%
\subsubsection[メートルドテルバター]{\texorpdfstring{メートルドテル\footnote{メートルドテル
  maître d'hôtel は直訳すれば「館{[}やかた{]}の主」
  あるいは「館の指導者」の意だが、時代および王家あるいは貴族やブルジョ
  ワの館、近現代のレストランにおいてそれぞれ異なった意味で用いられる
  職名。(1)王家においては grand maîtreグランメートルを補佐する仕事
  として食卓関連の仕事を取り仕切る職のこと。王と親しくすることが出来
  るために、有力貴族がこの職に就くことを希望することが多かったという。
  (2)大貴族や大ブルジョワの館において、食材の手配やワインの管理、
  料理人の選抜などの一切を取り仕切り、とりわけ宴席においてはメニュー
  作りが重要な仕事のひとつとして課される職。コンデ公に仕え、シャンティ
  イ城での大宴席の一切を取り仕切り、最後に手配した魚が届かないと誤解
  して自害したヴァテル(\protect\hyperlink{sauce-chantilly}{ソース・シャンティイ}訳注
  参照)はこの職に相当する。(3)近代から20世紀中葉にかけて、とりわ
  け料理人がオーナーではないレストランの場合はメニューの決定、ソムリ
  エおよび給仕人の指揮、客の応対などを担当し、その店で最高のサービス
  技術を誇る者のつく職名とされた。なお、現代ではほとんど「給仕長」程
  度の意味しか持たなくなってしまった職名といえる。上記を総合すると、
  この beurre à la maître d'hôtel という名称は「当家(当店)特製のバ
  ター」あるいは「当家(当店)自慢のバター」程度の意味ということにな
  る。実際のところ、この名称の由来などは不明だが、たとえばフランソワ・
  マランFrançois Marin(生没年不詳)の著書、『コモス神の贈り物』のタ
  イトルページに記された著者の肩書は「スビーズ元帥のメートルドテル、
  フランソワ・マラン」となっているように、本来はもっとも料理に精通し
  た者の就く役職であった。このため、maître d'hôtel-cuisinier という
  語も18、19世紀には用いられていた。つまり、直接的に包丁を握り鍋を振
  ることはなくても、献立を組み、料理のレシピを考えるのもまたメートル
  ドテルの重要な仕事であった。それを踏まえてカレームは1822年に、それ
  以前の主要な宴席の献立を詳細に分析した『フランスのメートルドテル』
  を出版した。つまり、カレームもまた、食卓外交の裏側でメートルドテル
  =キュイジニエの役割を果たしていたということになる。カレームをたん
  なるパティシエや料理人という現代的な狭い職の枠にはめて捉えることの
  出来ない時代だったとも言えよう。それは、エスコフィエについても言え
  ることであり、初版および第二版の末尾には献立例が掲載され、第三版以
  降は『メニューの本』として独立させたが、総料理長であるということは
  即ちかつて貴族の館に仕えたメートルドテルの仕事を勤めるに他ならない、
  ということを示唆しているし、その点は現代の一流ホテルにおいてもあま
  り変化していないと思われる。}バター\footnote{この合わせバターの名称も含めた原型のひとつとして、注の前項にお
  いて言及したマラン『コモス神の贈り物』第2巻には、「いんげん豆のメー
  トルドテル風」というレシピがある。これは水から茹でたいんげん豆を湯
  をきってから鍋に入れ、バター、パセリ、エシャロットの細かいみじん切
  り、塩、こしょうで味付けし、最後にレモン果汁かヴィネガー少々で仕上
  げるというもの(p.380)。カレームの未完の名著『19世紀フランス料理』
  第3巻では、「鯖用のメートルドテルバター」として、イジニー産バター8
  オンス(約250g弱)と大きめのレモン1個分の搾り汁、細かく刻んだパセ
  リ大さじ2杯、塩2つまみ強、細かく挽いたこしょう1つまみ弱を木杓子を
  使ってよく混ぜ合わせる。食欲がわくような調味を心掛けるべし、とある
  (pp.128-129)。また、同じくヴィアールの『王国料理の本』(内容は1806
  年初版の『帝国料理の本』の改訂版であり、毎年のように改版され続けて
  いるために歴史的に貴重な史料)1846年版では、冷製メートルドテルとし
  て、鍋に\unquart{}ポンドのバターとパセリ少々、エシャロットのみじん
  切り少々、塩、粗挽きこしょう、レモン果汁を入れ、木杓子でよく練る。
  これを肉料理あるいは魚料理の下にでも、中にでも、上にでも流すといい、
  とある(p.48)。このように、19世紀前半にはメートルドテルバターの性格
  がほぼ定着していたと言えよう。}}{メートルドテルバター}}\label{ux30e1ux30fcux30c8ux30ebux30c9ux30c6ux30eb25ux30d0ux30bfux30fc26}}

\hypertarget{beurre-a-la-maitre-d-hotel}{%
\paragraph{Beurre à la
Maître-d'hôtel}\label{beurre-a-la-maitre-d-hotel}}

\index{はたー@バター!あわせはたー@合わせバター!めとるとてるはたー@メートルドテルバター}
\index{あわせはたー@合わせバター!めーとるとてるはたー@メートルドテルバター}
\index{めーとるとてる@メートルドテル!はたー@---バター}
\index{beurre@beurre!beurres composes@Beurres Composés!beurre maitre hotel@Beurre à la Maître d'hôtel}
\index{maitre hotel@maître d'hôtel!beurre@Beurre à la ---}

バター250 gをポマード状に柔らかくする。パセリのみじん切り大さじ1杯
強と塩8 g、こしょう1 g、レモン\unquart{}個分の果汁を加えてよく混ぜ合わ
せる。

\hypertarget{ux539fux6ce8-2}{%
\subparagraph{【原注】}\label{ux539fux6ce8-2}}

このメートルドテルバターに大さじ1杯のマスタードを加えるのもバリエーショ
ンとしてお勧め。とりわけ牛、羊肉や魚のグリル焼きによく合う。

\maeaki

\hypertarget{ux30d6ux30fcux30ebux30deux30cbux30a8}{%
\subsubsection{ブールマニエ}\label{ux30d6ux30fcux30ebux30deux30cbux30a8}}

\hypertarget{beurre-manie}{%
\paragraph{Beurre Manié}\label{beurre-manie}}

\index{はたー@バター!あわせはたー@合わせバター!ふーるまにえ@ブールマニエ}
\index{あわせはたー@合わせバター!ふーるまにえ@ブールマニエ}
\index{ふーるまにえ@ブールマニエ}
\index{beurre@beurre!beurres composes@Beurres Composés!beurre manie@Beurre Manié}
\index{beurre manie@beurre manié}

これはマトロットの煮汁などに、手早くとろみ付けをするのに用いる。小麦粉75
gにバター100 gの割合が原則\footnote{このバターと小麦粉の割合は絶対というわけではなく、本書でもしば
  しば異なる割合で作ったブールマニエを用いる指示が見られる。}。

ブールマニエでとろみを付けたソースは、その後は出来るだけ沸騰させないこ
と。さもないと、生の小麦粉の不快な味が強まる危険性があるからだ。

\maeaki

\hypertarget{ux30d6ux30fcux30ebux30deux30ebux30b7ux30e3ux30f3ux30c9ux30f4ux30a1ux30f330}{%
\subsubsection[ブール・マルシャンドヴァン]{\texorpdfstring{ブール・マルシャンドヴァン\footnote{「ワイン商人風」の意。煮詰めた赤ワインをバターを混ぜ込むところからの名称だろう。}}{ブール・マルシャンドヴァン}}\label{ux30d6ux30fcux30ebux30deux30ebux30b7ux30e3ux30f3ux30c9ux30f4ux30a1ux30f330}}

\hypertarget{beurre-marchand-de-vin}{%
\paragraph{Beurre Marchand de vin}\label{beurre-marchand-de-vin}}

\index{はたー@バター!あわせはたー@合わせバター!ふーるまるしやんとうあん@ブール・マルシャンドヴァン}
\index{あわせはたー@合わせバター!ふーるまるしやんとうあん@ブール・マルシャンドヴァン}
\index{わいんしようにん@ワイン商人 ⇒ マルシャンドヴァン!ふーる@ブール・マルシャンドヴァン}
\index{まるしやんとうあん@マルシャンドヴァン!ふーる@ブール・マルシャンドヴァン}
\index{beurre@beurre!beurres composes@Beurres Composés!beurre marchand vin@Beurre Marchand de vin}
\index{marchand vin@marcand de vin!beurre@Beurre ---}

赤ワイン2 dlに細かいみじん切りにしたエシャロット25 gを加えて半量になる
まで煮詰める。塩1つまみ、挽きたて\footnote{原文 poivre de moulin
  (ポワーヴルドムラン)、直訳すると「ミル
  で挽いたこしょう」だが、その場合は即座に使用するのが一般的なので、
  あえて「挽きたてのこしょう」と訳している。}(または粗く砕いた\footnote{原文
  mignonette (ミニョネット)。ミルを用いずに、包丁の側面な
  どで圧し砕いたこしょうを指す。})こしょ
う1つまみ、溶かした\href{glace-de-viande}{グラスドヴィアンド}大さじ1杯、ポ
マード状に柔らかくしたバター150 g、レモン\unquart{}個分の果汁とパセリ
のみじん切り大さじ1杯を加える。全体をよく混ぜ合わせる。

\ldots{}\ldots{}グリル焼きにした牛リブロース\footnote{原文 entrecôte
  grillé(アントルコット グリエ)。}用。

\maeaki

\hypertarget{ux30e0ux30cbux30a8ux30ebux7528ux30d0ux30bfux30fc}{%
\subsubsection{ムニエル用バター}\label{ux30e0ux30cbux30a8ux30ebux7528ux30d0ux30bfux30fc}}

\hypertarget{beurre-a-la-meuniere}{%
\paragraph{Beurre à la Meunière}\label{beurre-a-la-meuniere}}

\index{はたー@バター!あわせはたー@合わせバター!むにえるようはたー@ムニエル用バター}
\index{あわせはたー@合わせバター!むにえるようはたー@ムニエル用バター}
\index{むにえる@ムニエル!はたー@---用バター}
\index{beurre@beurre!beurres composes@Beurres Composés!beurre meuniere@Beurre à la Meunière}
\index{meuniere@meunière (à la)!beurre@Beurre à la ---}

焦がしバターに、提供直前にレモン果汁数滴を加えたもの。

\ldots{}\ldots{}魚の「ムニエル\footnote{小麦粉をまぶして、バターで焼く手法および仕立て。原文にある
  à la meunière を直訳すると「粉挽き女風」の意。主に水車を動力として石臼
  を回転させて小麦を挽き、その後「ふるい」にかけていたことから、粉挽
  き職人は小麦粉の粉塵をかぶって真っ白になっていることが多かったとこ
  ろから付いた料理名。}」用。

\maeaki

\hypertarget{ux30e2ux30f3ux30daux30eaux30a8ux30d0ux30bfux30fc}{%
\subsubsection{モンペリエバター}\label{ux30e2ux30f3ux30daux30eaux30a8ux30d0ux30bfux30fc}}

\hypertarget{beurre-de-montpellier}{%
\paragraph[Beurre de Montpellier]{\texorpdfstring{Beurre de
Montpellier\footnote{南フランスの都市。モンプリエのようにも発音される。どちらも正しい。複数の発音が正
  しいとされる例として有名なもののひとつ。}}{Beurre de Montpellier}}\label{beurre-de-montpellier}}

\index{はたー@バター!あわせはたー@合わせバター!もんへりえはたー@モンペリエバター}
\index{あわせはたー@合わせバター!もんへりえはたー@モンペリエバター}
\index{もんへりえ@モンペリエ!はたー@---バター}
\index{beurre@beurre!beurres composes@Beurres Composés!beurre monpellier@Beurre de Montpellier}
\index{montepellier@Montpellier!beurre@Beurre de ---}

銅の鍋に湯を沸かし、クレソンの葉とパセリの葉、セルフイユ、シブレット、
エスゴラゴンを同量ずつ計90〜100 gと、ほうれんそうの葉25 gを投入する。
これとは別の鍋で同時に、エシャロットの細かいみじん切り40 gを下茹でする。
ハーブは湯をきって冷水にさらす。しっかり圧し絞って余計な水気を取り除く。
エシャロットも同様にする。これらを鉢に入れてすり潰す。

中くらいのサイズのコルニション3個と、水気を絞ったケイパー大さじ1杯、小
さなにんにく1片、アンチョビのフィレ4枚を加える。全体が滑らかなペースト
状になったら、バター750 gと固茹で卵の黄身3個、生の卵黄2個を加える。混
ぜながら、最後に植物油2 dlを少しずつ加える。目の細かい漉し器か布で漉し、
泡立て器で混ぜて滑らかにする。塩味を\ruby{調}{ととの}え、カイエンヌご
く少量で風味を引き締める。

\ldots{}\ldots{}魚の冷製料理に添える。ビュッフェの場合には魚に覆いかけて供する。

\maeaki

\hypertarget{ux88c5ux98feux7528ux30e2ux30f3ux30daux30eaux30a8ux30d0ux30bfux30fc}{%
\subsubsection{装飾用モンペリエバター}\label{ux88c5ux98feux7528ux30e2ux30f3ux30daux30eaux30a8ux30d0ux30bfux30fc}}

\hypertarget{beurre-de-montpellier-pour-croutonnage-de-plats}{%
\paragraph{Beurre de Montpellier pour Croûtonnage de
plats}\label{beurre-de-montpellier-pour-croutonnage-de-plats}}

\index{はたー@バター!あわせはたー@合わせバター!そうしよくようもんへりえはたー@装飾用モンペリエバター}
\index{あわせはたー@合わせバター!そうしよくようもんへりえはたー@装飾用モンペリエバター}
\index{もんへりえ@モンペリエ!そうしよくようはたー@装飾用---バター}
\index{beurre@beurre!beurres composes@Beurres Composés!beurre meuniere cretonnage@Beurre de Montpellier pour croûtonnage de plats}
\index{montepellier@Montpellier!beurre cretonnage@Beurre de --- pour Croûtonnage de plats}

モンペリエバターを装飾のためだけに作る場合には、植物油と茹で卵の黄身、
生の卵黄は用いずに作る。平皿に流し入れて均等な厚みにしてやると細部の装
飾作業が容易になる。

\maeaki

\hypertarget{ux30deux30b9ux30bfux30fcux30c9ux30d0ux30bfux30fc}{%
\subsubsection{マスタードバター}\label{ux30deux30b9ux30bfux30fcux30c9ux30d0ux30bfux30fc}}

\hypertarget{beurre-de-moutarde}{%
\paragraph{Beurre de Moutarde}\label{beurre-de-moutarde}}

\index{はたー@バター!あわせはたー@合わせバター!ますたーとはたー@マスタードバター}
\index{あわせはたー@合わせバター!ますたーとはたー@マスタードバター}
\index{ますたーと@マスタード!はたー@---バター}
\index{beurre@beurre!beurres composes@Beurres Composés!beurre moutarde@Beurre de Moutarde}
\index{moutarde@moutarde!beurre@Beurre de ---}

フランス産マスタード大さじ1\undemi{}杯をポマード状に柔らかくしたバター250
gに混ぜ込み、冷蔵する。

\maeaki

\hypertarget{ux5927ux898fux6a21ux306aux5bb4ux5e2dux7528ux306eux9ed2ux30d0ux30bfux30fc}{%
\subsubsection{大規模な宴席用の黒バター}\label{ux5927ux898fux6a21ux306aux5bb4ux5e2dux7528ux306eux9ed2ux30d0ux30bfux30fc}}

\hypertarget{beurre-noir-pour-les-grands-services}{%
\paragraph{Beurre noir pour les grands
services}\label{beurre-noir-pour-les-grands-services}}

\index{はたー@バター!あわせはたー@合わせバター!たいきほなえんせきようのくろはたー@大規模な宴席用の黒バター}
\index{あわせはたー@合わせバター!たいきほなえんせきようのくろはたー@マスタードバター}
\index{くろ@黒!はたー@---バター}
\index{beurre@beurre!beurres composes@Beurres Composés!beurre noir grands services@Beurre noir pour les grands services}
\index{noir@noir!beurre@Beurre --- pour les grands services}

(仕上り10人分\footnote{原文 proportion pour un service
  (プロポルスィオンプーランセル
  ヴィス)、直訳すると「1サーヴィスの分量」。17、18世紀から20世紀初
  頭にかけて宴席での人数の単位に service (セルヴィス)という語があ
  てられた。8〜12人分とされた。ごく大雑把に10人前と捉えていい。舞踏
  会も含め、大規模で華やかな宴席がしばしば行なわれていた時代において
  は、ある程度大まかに料理の単位を決めておくことで、食材の手配から仕
  込み、調理などを効率化していた。このため、『料理の手引き』のレシピ
  のほとんどは1 serviceでの調理を前提に記されている。})

バター125 gをフライパンに入れて火にかけて溶かし、茶色くなるまで加熱す
る。布で漉して湯煎にかける。\ruby{微温}{ぬる}くなったら、粗く砕いたこ
しょう\footnote{mignonette
  (ミニョネット)。包丁の側面などで押し潰して砕いたも の。}を加えて煮詰めたヴィネガー小さじ1杯を加える。提供直前に、丁
度いい温度になるまで温めなおす。揚げたパセリの葉とケイパー大さじ1杯を
料理にのせてから、この黒バターをかけてやる。

\maeaki

\hypertarget{ux30d6ux30fcux30ebux30c9ux30ceux30efux30bcux30c3ux30c8}{%
\subsubsection{ブール・ド・ノワゼット}\label{ux30d6ux30fcux30ebux30c9ux30ceux30efux30bcux30c3ux30c8}}

\hypertarget{beurre-de-noisette}{%
\paragraph{Beurre de noisette}\label{beurre-de-noisette}}

\index{はたー@バター!あわせはたー@合わせバター!ふーるとのわせつと@ブール・ド・ノワゼット}
\index{あわせはたー@合わせバター!ふーるとのわせつと@ブール・ド・ノワゼット}
\index{のわせつと@ノワゼット!ふーるとのわせつと@ブール・ド・---}
\index{beurre@beurre!beurres composes@Beurres Composés!beurre noisette@Beurre de noisette}
\index{noisette@noisette!beurre@Beurre de ---}

⇒ \protect\hyperlink{beurre-d-aveline}{ブール・ダヴリーヌ}参照。

\maeaki

\hypertarget{ux30d1ux30d7ux30eaux30abux30d0ux30bfux30fc}{%
\subsubsection{パプリカバター}\label{ux30d1ux30d7ux30eaux30abux30d0ux30bfux30fc}}

\hypertarget{beurre-de-paprika}{%
\paragraph{Beurre de Paprika}\label{beurre-de-paprika}}

\index{はたー@バター!あわせはたー@合わせバター!はふりかはたー@パプリカバター}
\index{あわせはたー@合わせバター!はふりかはたー@パプリカバター}
\index{はふりか@パプリカ!はたー@---バター}
\index{beurre@beurre!beurres composes@Beurres Composés!beurre  Paprika@Beurre de Paprika}
\index{paprika@paprika!beurre@Beurre de ---}

玉ねぎのみじん切り大さじ1杯とパプリカ4gをバターでいい色合いになるまで
炒め、ポマード状に柔らかくしておいたバター250 gに混ぜる。布で漉す。

\maeaki

\hypertarget{ux8d64ux30d4ux30fcux30deux30f3ux30d0ux30bfux30fc}{%
\subsubsection{赤ピーマンバター}\label{ux8d64ux30d4ux30fcux30deux30f3ux30d0ux30bfux30fc}}

\hypertarget{beurre-de-pimentos}{%
\paragraph{Beurre de Pimentos}\label{beurre-de-pimentos}}

\index{はたー@バター!あわせはたー@合わせバター!あかひーまんはたー@赤ピーマンバター}
\index{あわせはたー@合わせバター!あかひーまんはたー@赤ピーマンパプリカバター}
\index{あかひーまん@赤ピーマン!はたー@---バター}
\index{ほわうろん@ポワヴロン ⇒ 赤ピーマン}
\index{beurre@beurre!beurres composes@Beurres Composés!beurre pimentos@Beurre de Pimentos}
\index{pimentos@pimentos!beurre@Beurre de ---}
\index{poivron@poivron!beurre pimentos!Beurre de Pimentos}

ブレ\footnote{野菜のブレゼの方法については\protect\hyperlink{}{第13章野菜料理}参照。}ゼした赤いポワヴロン\footnote{原文
  poivron。日本の青果では「パプリカ」と呼ばれる肉厚で苦みの
  少ない品種。「カリフォルニア・ワンダー」が代表的品種。未熟なものは
  緑色だが完熟すると真っ赤になる。また、熟すと黄色、紫などになる品種
  もある。}100 gをバター250 gと合わせて細かくすり潰し、布で漉す。

\maeaki

\hypertarget{ux30d4ux30b9ux30bfux30c1ux30aaux30d0ux30bfux30fc}{%
\subsubsection{ピスタチオバター}\label{ux30d4ux30b9ux30bfux30c1ux30aaux30d0ux30bfux30fc}}

\hypertarget{beurre-de-pistache}{%
\paragraph{Beurre de Pistache}\label{beurre-de-pistache}}

\index{はたー@バター!あわせはたー@合わせバター!ひすたちおはたー@ピスタチオバター}
\index{あわせはたー@合わせバター!ひすたちおはたー@ピスタチオバター}
\index{ひすたちお@ピスタチオ!はたー@---バター}
\index{beurre@beurre!beurres composes@Beurres Composés!beurre pistache@Beurre de Pistache}
\index{pistache@pistache!beurre@Beurre de ---}

殻から剥いて湯剥きしたばかりのピスタチオ150 gを、水数滴を加えながら細
かくすり潰す。パター250 gを加え、布で漉す。

\maeaki

\hypertarget{ux30ddux30fcux30e9ux30f3ux30c9ux98a8ux30d0ux30bfux30fc}{%
\subsubsection{ポーランド風バター}\label{ux30ddux30fcux30e9ux30f3ux30c9ux98a8ux30d0ux30bfux30fc}}

\hypertarget{beurre-a-la-polonaise}{%
\paragraph{Beurre à la Polonaise}\label{beurre-a-la-polonaise}}

\index{はたー@バター!あわせはたー@合わせバター!ほーらんとふうはたー@ポーランド風バター}
\index{あわせはたー@合わせバター!ほーらんとふうはたー@ポーランド風バター}
\index{ポーランド@ポーランド!はたー@---風バター}
\index{beurre@beurre!beurres composes@Beurres Composés!beurre polonaise@Beurre à la Polonaise}
\index{polonais@polonais(e)!beurre@Beurre à la ---e}

バター250 gをヘーゼルナッツ色\footnote{原文 cuire à la noisette
  (キュイーラノワゼット)すなわち「茶色
  く」なるまで火を通すということ。現代では、焦がしバターのことを beurre
  noisette (ブールノワゼット)と呼ぶことが多いが、本書におい
  ては\protect\hyperlink{beurre-de-noisette}{ヘーゼルナッツバター}という項目を立てて
  いるために、混同を避ける意味で、このような表現になっていると思われ
  る。}になるまで火を通す。丁度いい色合いに なったら、上等なパンの身60
gを投入する。

\maeaki

\hypertarget{ux30ecux30d5ux30a9ux30fcux30eb38ux30d0ux30bfux30fc}{%
\subsubsection[レフォールバター]{\texorpdfstring{レフォール\footnote{ホースラディッシュ、西洋わさび。}バター}{レフォールバター}}\label{ux30ecux30d5ux30a9ux30fcux30eb38ux30d0ux30bfux30fc}}

\hypertarget{beurre-de-raifort}{%
\paragraph{Beurre de Raifort}\label{beurre-de-raifort}}

\index{はたー@バター!あわせはたー@合わせバター!れふおーるはたー@レフォールバター}
\index{あわせはたー@合わせバター!れふおーる@レフォールバター}
\index{れふおーる@レフォール!はたー@---バター}
\index{beurre@beurre!beurres composes@Beurres Composés!beurre raifort@Beurre de Raifort}
\index{raifort@raifort!beurre@Beurre de ---}

器具を用いておろしたレフォール50 gを鉢に入れてすり潰す。バター250 gを
加え、布で漉す。
\end{recette}\newpage
\hypertarget{marinades-et-saumures}{%
\section[マリナードとソミュール]{\texorpdfstring{マリナードとソミュール\footnote{マリナードはマリネ液とも言う。marinade
  \textless{} mariner (マリネ)語源
  はラテン語のmare(海)。中世フランス語ではもっぱら「海で泳ぐ、海に
  潜る」の意で使われていたが、16世紀には既に、料理用語として用いられ
  ていたようだ。ラブレー『ガルガンチュアとパンタグリュエル』第四の書
  (1548年)において、lancerons marinez (マリネしたブロシェの幼魚)
  という表現が見られる。なおブロシェ brochet はノーザンパイク、和名
  キタカワカマス。川カマス属の淡水、汽水魚。この場面はパンタグリュエ
  ルに「小斉」のご馳走として捧げられた料理のリストの一部であり、「塩
  漬けのメルルーサ、卵料理各種、モリュ(塩漬けにした鱈)、アドック
  (塩漬け後に燻製にした鱈)」などとともに列挙されており、いずれも塩
  辛いために、それらを調理したものを食べて消化をよくするために飲むワ
  インの量が倍になった(p.681)とある。したがって、lancerons marinezの
  マリネとは「海水あるいは塩水に漬けた」の意に解釈されよう。一方、ソ
  ミュールについては、11世紀末頃に、「保存のため漬け込む塩水」の意味 で
  salmuire という語形が使用され、16世紀には「塩水およびその他の液
  体からなるもの」としてsaumureという現在とおなじ語形が記録されてい
  る。マリナードとソミュールが明確に分化したのはおそらく17世紀頃で、
  1651年刊ラ・ヴァレーヌ『フランス料理の本』に見られるマリナードの語
  には曖昧さを免れないものもあるが、例えば \emph{Poulets marinez} (鶏の
  マリネ)というレシピは「鶏を開いて叩き、しっかり味付けしたヴィネガー
  に漬ける。提供直前に小麦粉をまぶすか、卵と小麦粉で作った衣を付け、
  ラードで揚げる。揚がったらマリナードに戻し入れて軽く弱火で煮てから
  供する(p.36)」あるいは \emph{Longe de
  mouton} (仔羊の腰肉のロースト)
  のレシピは、「よく熟成させてから棒状に切った豚背脂をラルデ針を使っ
  て刺し込み、串を刺してローストする。玉ねぎ、塩、こしょう、ごく少量
  のオレンジまたはレモンの外皮{[}ゼスト{]}とブイヨンとヴィネガーでマリ
  ナードを作る。肉に火が通ったら、ソース{[}マリナード{]}とともに弱火で
  煮込む。とろみ付けには前項同様にした小麦粉{[}小麦粉をラードで茶色く
  なるまで炒めたもの、すなわち『料理の手引き』の時代のルーの原型{]}を
  少々加える(p.80)」とあり、別の項目では「(串を刺した肉の下の受け皿
  にある)マリナードを小まめにかけながら{[}アロゼしながら{]}ローストす
  る(p.106)」という表現がある。レシピ数からいうとラ・ヴァレーヌにお
  いてマリナードとは中世のドディーヌにヴィネガーを効かせたもののよう
  にも受け取れるが、最初に見たように、「漬け込む」ものとしてヴィネガー
  を用いている点に注目すべきだろう。この流れは18、19世紀に引継がれる。
  1756年マラン『コモス神の贈り物』第1巻において、\emph{Cervelle de veau
  en marinade}(仔牛の脳のマリナード仕立て)などのレシピがあり、血抜
  きした仔牛の脳を豚背脂のシートで包みブイヨン少々で茹で、「冷まして
  からヴィネガーもしくはレモン果汁に漬け込む。その後、水気をきって溶
  き卵に浸し、パン粉をつけて揚げる。小麦粉を溶いた揚げ衣に浸して揚げ
  てもいい(p.206)」とある。19世紀のヴィアールでも同様の料理は見られ
  る。『帝国料理の本』初版(1806年)において、\emph{Pieds d'agneau en
  marinade} 仔羊の足のマリナード仕立てなどいくつかのmarinadeを冠する
  レシピが掲載されている。肝心のマリナードについての記述は欠落してい
  るが、この版においてはよく見られる現象。なお、仔羊の足のマリナード
  仕立ては、マリナードがない場合には「塩、こしょう、ビネガーに茹でた
  仔羊の足を漬けてから、揚げ衣を付けて揚げる(p.214)」となっている。
  1814年ボヴィリエ『調理技法』では「加熱マリナード」のレシピが掲載さ
  れている。これは、卵くらいの大きさのバターを鍋に入れ、輪切りにした
  にんじん1、2本、同様にした玉ねぎ、ローリエの葉1枚、にんにく1片、タ
  イム、バジル、枝ごとのパセリ、シブール{[}≒葱{]}2〜3本を加えて強火で
  炒める。野菜が色付きはじめたら、約250mlの白ワインヴィネガーと約0.5
  Lの水を注ぎ、塩、こしょうする。そのまま沸かして、漉し器で漉し、必
  要に応じて使う(pp.60-61)、というもの。もっとも、仔牛の脳のマリナー
  ド仕立てなどマランのレシピと大差ない揚げものも同書では目に付く。ま
  た、1834年版のオドにおいても鶏のマリナードはラ・ヴァレーヌのものと
  同工異曲に留まっている。1837年版ではロースト用マリナードの項が追加
  され、豚背脂とにんにく1片を細かく刻み、パセリ1つまり、塩、こしょう、
  ヴィネガー大さじ1杯、油大さじ4杯を合わせてよく混ぜる(p.419)。1853
  年版ではマリネしたうなぎのグリル焼き、というレシピが掲載される。こ
  れは、皮を剥いてぶつ切りにし、バターでソテーしたうなぎを深皿に並べ、
  塩、こしょうハーブ、マッシュルーム、細かく刻んだエシャロットとシブー
  ルを被せ、油大さじ1杯をかける。2〜3時間マリネしたら、パン粉をまぶ
  してグリル焼きする(p.310)というもの。いっぽう、mariner(マリネ)と
  いう動詞については、オドの1834年版で既に、ノロ鹿の腿肉のローストに
  おいて、「オリーブオイルと塩で5〜6時間マリネする」(p.155)という記
  述が見られる。1867年刊グフェ『料理の本』においては、ヴィネガーをベー
  スとしたソースとしてのマリナード(p.404)と仕立てとしてのマリナー
  ドがあるが、後者はこんにちの概念に近く、例えば \emph{Tête de veau en
  marinade} (仔牛の頭 マリナード仕立て)は、仔牛の頭肉半分を3 cm角
  に切り、下茹でしてから水にさらし、牛脂と小麦粉、香草類を加えたブラ
  ンで茹でる。これを、塩、こしょう、油、ヴィネガーに1時間漬け込む。
  水気をきって揚げ衣を付けて油で揚げる、というもの(p.156)。ここでは
  肉を漬け込む液体としてmarinadeの語が用いられている。このように、
  marinadeという名詞とmariner「漬け込む」という動詞の用法にややずれ
  が見られるため、『料理の手引き』におけるマリナードすなわちマリネ液、
  という概念は19世紀後半になってからのものと思われる。}}{マリナードとソミュール}}\label{marinades-et-saumures}}

\frsec{Marinades et Saumures}

\index{marinade saumures@marinade et saumures} \index{marinade@marinade}
\index{まりなーととそみゆーる@マリナードとソミュール}
\index{まりなーと@マリナード}

マリナードとソミュールにはいろいろな種類があるが、最終的な目的は同じで、

\begin{enumerate}
\def\labelenumi{\arabic{enumi}.}
\item
  素材に料理で使う香辛料やハーブの香りを浸み込ませる
\item
  ある種の肉を柔らかくさせる
\item
  場合によっては保存のために用いる。とりわけ温度と湿度で素材が駄目になってしまうような場合。さらに、目指す料理の仕上がりに合わせて素材の状態を調節する
\end{enumerate}
\begin{recette}
\hypertarget{marinade-instantanee}{%
\subsubsection{即席マリナード}\label{marinade-instantanee}}

\frsub{Marinade instantanée}

\index{marinade@marinade!marinade instantanee@marinade instantanée}
\index{まりなーと@マリナード!そくせき@即席---}

このマリナードはすぐに素材を使う場合、例えば赤身肉のグリル焼きや、ガラ
ンティーヌ、テリーヌ、パテのような冷製料理の補助材料\footnote{具体的には\protect\hyperlink{farces}{ファルス}のこと。}にする肉に用い
る。

\begin{enumerate}
\def\labelenumi{\arabic{enumi}.}
\item
  グリル焼きにする肉の場合\ldots{}\ldots{}ごく薄くスライスしたエシャロットとパセ
  リの枝、タイムの枝、ローリエの葉を肉の上に散らす。量は適宜加減する
  こと。レモン果汁\undemi{}個分に対して油大さじ1杯の割合で、上からか
  けてやる。
\item
  仔牛、ジビエのフィレ肉、ハム、豚背脂などを細かく切ったもの\footnote{原文
    lardon (ラルドン)、通常は拍子木状に切ったものを言うが、こ
    こではファルスとして後で細かく挽くことになるので、形状はあまり問題
    にならない。}の場
  合\ldots{}\ldots{}塩こしょうしてから、白ワイン3、コニャック3、油1の割合のマリナー
  ドを上からかけてやる。
\end{enumerate}

ここで用いた風味付けの材料は、後でファルスにする際に加えることになる。

いずれの場合でも、マリナードに浸した肉を小まめに裏返してやり、マリナー
ドがよく浸み込むようにしてやること。

\maeaki

\hypertarget{marinade-crue-pour-viandes-de-boucherie-ou-venaison}{%
\subsubsection{牛、羊肉および大型ジビエ用の非加熱マリナード}\label{marinade-crue-pour-viandes-de-boucherie-ou-venaison}}

\frsub{Marinade crue pour viandes de boucherie ou venaison}

\index{marinade@marinade!marinade crue viande boucherie venaison@marinade crue pour viande de boucherie ou venaison}
\index{まりなーと@マリナード!うしひつしおおかたしひえようひかねつ@牛、羊肉および大型ジビエ用非加熱---}

(仕上がり2 L分)

\begin{itemize}
\item
  \textbf{香味素材}\ldots{}\ldots{}にんじん100 g、玉ねぎ100
  g、エシャロット40 g、セロリ30
  g、にんにく2片、パセリの枝3本、タイム1枝、ローリエの葉\undemi{}枚、大粒のこしょう6個、クローブ2本。
\item
  \textbf{使用する液体}\ldots{}\ldots{}白ワイン1\unquart{}
  L、ヴィネガー5 dl、油2\undemi{} dl。
\item
  \textbf{作業手順}\ldots{}\ldots{}マリネする素材に塩とこしょうを振る。にんじん、玉ねぎ、エシャロットを薄切り\footnote{émincer
    (エマンセ)薄切りにする、スライスする。}にし、半量を容器の底に敷く。容器の大きさは素材とマリナードがぴったり入る程度のものを用いること。素材を入れて、残りの香味野菜で蓋をするようにして、白ワインとヴィネガー、油を注ぎ入れる。
\end{itemize}

冷蔵し、マリネ液に漬かった素材を小まめに裏返してやること。

\maeaki

\hypertarget{marinade-cuite-pour-viandes-de-boucherie-ou-venaison}{%
\subsubsection{牛、羊肉および大型ジビエ用の加熱マリナード}\label{marinade-cuite-pour-viandes-de-boucherie-ou-venaison}}

\frsub{Marinade cuite pour viandes de boucherie ou venaison}

\index{marinade@marinade!marinade cuite viande boucherie venaison@marinade cuite pour viande de boucherie ou venaison}
\index{まりなーと@マリナード!うしひつしおおかたしひえようかねつ@牛、羊肉および大型ジビエ用加熱---}

(仕上がり2 L分)

\begin{itemize}
\item
  \textbf{香味素材}\ldots{}\ldots{}非加熱マリナードと同じ材料で同じ分量
\item
  \textbf{使用する液体}\ldots{}\ldots{}白ワイン1\undemi{} L、ヴィネガー3
  dl、油2\undemi{} dl。
\item
  \textbf{作業手順}\ldots{}\ldots{}鍋に油を熱し、ごく薄くスライスしたにんじん、玉ねぎ、
  エシャロットおよびその他の香味素材を軽く色付くまで炒める。

  白ワインとヴィネガーを注ぎ、弱火で約30分間火を通す。

  必ず、マリナードが完全に冷めてからマリネする素材にかけること。
\end{itemize}

\maeaki

\hypertarget{marinade-crue-ou-cuite-pour-grosse-venaison}{%
\subsubsection[とりわけ大型のジビエ用、非加熱および加熱マリナード]{\texorpdfstring{とりわけ大型のジビエ\footnote{具体的には赤鹿
  cerf(セール) や猪、トナカイの成獣など。ニホンジ
  カやエゾジカはcerfに分類されるので、これを参考にするといいだろう。}用、非加熱および加熱マリナード}{とりわけ大型のジビエ用、非加熱および加熱マリナード}}\label{marinade-crue-ou-cuite-pour-grosse-venaison}}

\frsub{Marinade crue ou cuite pour grosse venaison}

\index{marinade@marinade!marinade crue cuite grosse venaison@marinade crue ou cuite pour grosse venaison}
\index{まりなーと@マリナード!とりわけおおかたのしひえようひかねつおよひかねつ@とりわけ大型のジビエ用非加熱および加熱---}

(仕上がり2 L分)

\begin{itemize}
\item
  \textbf{香味素材}\ldots{}\ldots{}牛、羊肉および大型ジビエ用のマリナードと同じだが、ローズマリー12
  gを追加する。
\item
  \textbf{使用する液体}\ldots{}\ldots{}ヴィネガー16 dl、油4 dl。
\item
  \textbf{作業手順}\ldots{}\ldots{}非加熱、加熱ともに作業手順は上記のレシピのとおり。
\end{itemize}

\maeaki

\hypertarget{marinade-cuite-pour-le-mouton-en-chevreuil}{%
\subsubsection{羊のシュヴルイユ仕立て用の加熱マリナード}\label{marinade-cuite-pour-le-mouton-en-chevreuil}}

\frsub{Marinade cuite pour le mouton en chevreuil}\footnote{\protect\hyperlink{sauce-chevreuil}{ソース・シュヴルイユ}参照。}

\index{marinade@marinade!marinade cuite mouton en chevreuil@marinade cuite pour le mouton en chevreuil}
\index{まりなーと@マリナード!ひつしのしゆうるいゆしたてようのかねつまりなーと@羊のシュヴルイユ仕立て用加熱---}

(仕上がり2 L分)

\begin{itemize}
\item
  \textbf{香味素材}\ldots{}\ldots{}上記のとおりの分量の素材に、ジュニパーベリー\footnote{セイヨウネズの実。ジンの香り付けに用いられている。}10粒とバジル1つまみ、ローズマリー1つまみを足す。
\item
  \textbf{使用する液体}\ldots{}\ldots{}牛、羊および大型ジビエ用の加熱マリナードと同じ。
\item
  \textbf{作業手順}\ldots{}\ldots{}鍋に油を熱し、薄切りにしたにんじん、玉ねぎ、エシャロットおよびその他の香味素材を軽く色付くまで炒める。

  白ワインとヴィネガーを注ぎ、弱火で約30分間火を通す。
\end{itemize}

\maeaki

\hypertarget{marinade-cuite-pour-le-mouton-en-chamois}{%
\subsubsection{羊のシャモワ仕立て用の加熱マリナード}\label{marinade-cuite-pour-le-mouton-en-chamois}}

\frsub{Marinade cuite pour le mouton en chamois}\footnote{オートザルプ県の山岳地帯およびピレネー山脈に生息する野生の山羊。
  ピレネー山脈のものは Isard (イザール)と呼ばれる。若い獣の肉は大
  型ジビエのなかでもとりわけ美味とされる。成獣の肉は固く、しっかりマ
  リネする必要があると言われている。しばしばノロ鹿と比較される。ここ
  では、羊肉を白ワインベースのマリナードに漬け込む仕立て、すなわちシュ
  ヴルイユ仕立てとの対比として、赤ワインでより強い風味のマリナードに
  漬け込むことで、シャモワ仕立てとしている。なお、本書においてシャモ
  ワ仕立てを料理名に謳ったレシピは掲載されていないので注意。基本的に
  はシュヴルイユ仕立てと同様に調理する。}

\index{marinade@marinade!marinade cuite mouton en chevreuil@marinade cuite pour le mouton en chevreuil}
\index{まりなーと@マリナード!ひつしのしやもわしたてようのかねつまりなーと@羊のシャモワ仕立て用加熱---}

(仕上がり2 L分)

\begin{itemize}
\item
  \textbf{香味素材}\ldots{}\ldots{}非加熱マリナードと同じ分量の素材に、ジュニパーベリー\footnote{セイヨウネズの実。ジンの香り付けに用いられている。}15粒とバジル15
  g、ローズマリー15 gを足す。
\item
  \textbf{使用する液体}\ldots{}\ldots{}良質な赤ワイン1\undemi{}
  L、ヴィネガー3 dl、油2\undemi{} dl。
\item
  \textbf{作業手順}\ldots{}\ldots{}上記と同じ。

  このマリナードに上等な赤ワインを使える場合には、素材の量を次のように
  調整すること。赤ワイン12 dl、ワインヴィネガー6 dl、油は上記の分量と
  する。

  ワインの酸味の強さによっては、ヴィネガーの量をワインと同量にすることさえ可能。
\end{itemize}

\hypertarget{observation-sur-les-marinades}{%
\subparagraph{マリナードについての注意事項}\label{observation-sur-les-marinades}}

\ldots{}\ldots{} 1.
加熱マリナードを使用するのは、素材へのマリナードの浸透作用を促進するのが目的。\\
素材をマリナードに漬け込む時間は、加熱、非加熱ともに、素材の種類と大き
さ、気温、環境の変化を勘案して決めること。

\begin{enumerate}
\def\labelenumi{\arabic{enumi}.}
\setcounter{enumi}{1}
\tightlist
\item
  一般的な牛、羊肉と肉質の柔らかい大型ジビエに使うマリナードに純粋な
  酢酸を用いるのは絶対にやめておくこと。酢酸の腐食作用によって肉の風
  味が失なわれてしまうからだ\footnote{この注記は第二版から。内容が当時の知見にもとづいたものである
    ことに注意。ただし、19世紀には木酢液を原料として工業用の氷酢酸が既
    に製造されていた。また、タンパク質はpHの変化によって分解されるので、
    マリナードにヴィネガーを加えるのは理にかなっている。なお、肉を柔ら
    かくする効果のあるタンパク質分解酵素(プロテアーゼ)の代表的なひと
    つであるパパインの発見は1940年代になってからのこと。パイナップルに
    含まれているブロメラインの効果は経験的に知られていた可能性もあるが、
    この酵素が60℃で不活性化することが広く知られるようになったのは、少
    なくとも日本では比較的近年のことに過ぎない。}。\\
  猪、赤鹿\footnote{cerf
    (セール)、ニホンジカやエゾジカもフランス語で表現するとこれに含まれるので、これらの料理について
    chevreuil (しゅう゛るいゆ)ノロ鹿の名をつけるは、厳密には誤り。}、トナカイなどの固い肉についても、純粋な酢酸だけを使うの
  は不可。
\end{enumerate}

\hypertarget{conservation-des-marinades}{%
\subsubsection{マリナードの保存方法}\label{conservation-des-marinades}}

\frsub{Conservation des marinades}

\index{marinade@marinade!conservation marinades@conservation des marinades}
\index{まりなーと@マリナード!ほそんほうほう@---の保存方法}

マリナードを長期間保存しておく必要がある場合には、とりわけ夏場は、本書
で示した分量に対して2〜3 gのホウ酸を加えるといい。

あえに、夏のあいだは2日に一度、冬季は4〜5日に一度、マリナードを沸騰さ
せ、冷めたら毎回そのマリナードに使っているのと同じワインを 2dlとヴィネ
ガー1 dlを足してやること。
\end{recette}
\hypertarget{saumures}{%
\subsection[ソミュール]{\texorpdfstring{ソミュール\footnote{この見出しは第四版のみ。初版〜第三版にかけては、マリナードとソ
  ミュールのレシピの間に区切りをつけるものは何も挿入されていない。}}{ソミュール}}\label{saumures}}

\frsec{Saumures}

\index{saumure@saumure} \index{そみゆーる@ソミュール}
\begin{recette}
\hypertarget{saumure-au-sel}{%
\subsubsection{塩漬け用ソミュール}\label{saumure-au-sel}}

\frsub{Saumure au sel}

\index{saumure@saumure!sel@--- au sel}
\index{そみゆーる@ソミュール!しおつけよう@塩漬け用---}

このソミュールは、グレーソルト\footnote{フランス語は sel gris
  (セルグリ)または gros gris (グログリ)。灰色がかった粗塩。}1
kgに対して硝石\footnote{原文 salpêtre
  (サルペートル)硝酸カリウム。殺菌作用と、肉類を
  赤く発色させる効果を持つ。現代の日本では亜硝酸カリウム、亜硝酸ナト
  リウムが使われることが多い。いずれも日本では劇物指定されているが、
  シャルキュトリ(豚肉加工品の製造)においては不可欠とも言われるな薬
  品であり、とりわけボツリヌス菌対策の効果が大きい。そのため劇物では
  あるが、食品添加物として認められており、使用限界量が厳密に定められ
  ている(食品添加物は国あるいは地域によって扱いが異なるので注意)。
  硝酸塩あるいは亜硝酸塩による肉の赤い発色を「着色料によるもの」と誤
  認する消費者は少なくない。これはかつて「魚肉ソーセージ」がコチニー
  ル色素でピンク色に染められていたことから連想される誤認と思われる。
  また、食品添加物イコール毒という安直な考えから忌避する消費者も少な
  くないのは事実だろう。こうしたことから、現代日本のレストランでは、
  製造後すぐに提供可能であるために、これら硝酸塩、亜硝酸塩の類を用い
  ないところもある。}40 gの割合で作
る。この硝石入りの塩の総量は、塩漬けにする肉の数と大きさで決まる。素材
が完全に覆えて、重しが出来る分量とすること。

\begin{itemize}
\tightlist
\item
  \textbf{作業手順}\ldots{}\ldots{}肉を塩漬けにする前にまず、太い針を充分深く刺して穴を
  何箇所も空ける。次に硝石の粉末を肉の表面にすり付ける。塩1 kgあたりタ
  イム1枝、ローリエの葉\undemi{}枚を加えて肉と塩を容器に詰める。
\end{itemize}

\maeaki

\hypertarget{saumure-liquide-pour-langues}{%
\subsubsection{舌肉用の液体ソミュール}\label{saumure-liquide-pour-langues}}

\frsub{Saumure liquide pour langues}\footnote{このソミュールに舌肉を漬け込むと、硝石の作用で舌肉が赤く発色す
  る。それを拍子木状などに切って鶏やフィレ肉の表面に、同様に切ったト
  リュフや豚背脂などとともに刺して装飾することが19世紀〜20世紀初頭ま
  でよく行なわれた。現代ではほとんど行なわれなくなった装飾方法。この場
  合はあくまでも料理の装飾を目的としたものであり、牛や豚の舌肉を保存
  食として利用する場合には塩漬けや燻製などの方法も用いられる。}

\index{saumure@saumure!liquide langues@--- liquide pour langues}
\index{そみゆーる@ソミュール!したにくようのえきたい@舌肉用の液体---}

\begin{itemize}
\item
  \textbf{材料}\ldots{}\ldots{}水5 L、グレーソルト2.25 kg、硝石150
  g、茶色いカソナード\footnote{砂糖きびを原料とした粗糖。通常は茶褐色のものが多く「赤糖」とも
    呼ばれるが、精白したものもある。精製が不完全であるため独特の風味があり、
    料理および製菓でしばしば用いられる。} 300 g、こしょう12
  g、ジュニパーベリー12粒、タイム1枝、ローリエの葉1 枚。
\item
  \textbf{作業手順}\ldots{}\ldots{}充分な大きさの鍋に材料を全て入れ、強火で沸騰させる。
  その後、完全に冷めてから、針で穴を複数空けて硝石をしっかりすり込んだ
  舌肉を入れた容器に注ぎ込む。\\
  平均的な重さの舌肉を漬け込む期間は冬季で8日間、夏季は6日間。
\end{itemize}

\hypertarget{grande-saumure}{%
\subsubsection{グランドソミュール}\label{grande-saumure}}

\frsub{Grande saumure}\footnote{この項は第二版で追加された。通常はシャルキュティエすなわちシャ
  ルキュトリ専門の職人が行なう規模のものであり、料理人の仕事の範疇を
  やや越えるとも考えられる。}

\index{saumure@saumure!grande@grande ---}
\index{そみゆーる@ソミュール!くらんと@グランド---}

(仕上がり50 L分)

\begin{itemize}
\tightlist
\item
  水\ldots{}\ldots{}50 L
\item
  塩\ldots{}\ldots{}25 kg
\item
  硝石\ldots{}\ldots{}2.7kg
\item
  カソナード\ldots{}\ldots{}1.6 kg
\item
  \textbf{作業手順}\ldots{}\ldots{}メッキされた銅の鍋に材料を全て入れ、強火にかける。沸
  騰したら、皮を剥いたじゃがいも1個を投入する。じゃがいもが浮いてくる
  ようであれば、じゃがいもが沈みはじめる寸前まで水を足す。逆に、じゃが
  いもが完全に底まで沈んでしまうようなら、じゃがいもが水面に見えてくる
  まで煮詰める必要がある。
\end{itemize}

ソミュールがちょうどいい具合になったら、鍋を火から外して、このソミュー
ルで漬け込み槽に注ぎ込む。漬け込み槽の素材は、スレート製、岩製、セメン
ト製、あるいはレンガ製でしっかりエナメル引きしたものを用いること。

漬け込み槽の底に、木製の網を敷き、その上に漬け込む肉を置くといい。肉が
槽の底面に直接当たっていると、肉の下側にソミュール液が浸透しない可能性
がある。

漬け込む肉は、たとえ小さなものであっても、専用の携行可能な注入器具を使っ
てソミュールを内部に注入してから、漬け込み槽に入れてやること。この準備
作業を怠ると、肉全体が均等に塩漬けにならない可能性がある。肉の中心部が
ちょうどいい塩加減になる頃には外側は塩が強すぎるということになってしま
うのだ。牛のランプ、イチボなどの塊肉で、4〜5kgの大きさの場合は、ソミュー
ル液を注入してやる方法を使えば8日間で漬かる。

牛舌肉をこの方法で漬ける場合は、出来るだけ新鮮なものを用いる必要がある。
軟骨部分をきれいに取り除いてやり、肉叩きか麺棒で丁寧に叩いてやる。ブリ
デ針\footnote{主として鶏などの手羽や腿をまとめて整形し、その形状を保つよう糸で縫う際に用いる縫い針。}を使って、表面全体に刺し穴をつけてやる。それからソミュールに
漬け込むが、何らかの重しをして浮き上がらないようにしてやること。

\hypertarget{observation-grande-saumure}{%
\subparagraph{【原注】}\label{observation-grande-saumure}}

ソミュールはマリナードほどは腐敗しにくいとはいえ、天候が悪い時季などは
とりわけ、よく様子を見て、時々は沸騰させてやるのがいい。沸騰させれば多
少は濃縮されてしまうから、本文記載の方法でじゃがいもを用いて、毎回少量
の水を加える必要がある。
\end{recette}\newpage
\hypertarget{ux30b8ux30e5ux30ec}{%
\section{ジュレ}\label{ux30b8ux30e5ux30ec}}

\hypertarget{gelees-diverses}{%
\subsection{Gelées diverses}\label{gelees-diverses}}

\index{gelee@gelée}

どんなジュレも、栄養面では、フォンから作られている。そのフォンでメイン
となっている素材によってジュレの風味が決まる。その結果としてジュレの用
途も決まってくる。

人工的な凝固剤の使用に頼らずに、ジュレを確実に固めるためには、フォンの
メインとなる素材に、仔牛の足や豚皮のようなゼラチン質の量を計算して加え
ることになる。仔牛の足や豚の皮を使えば、ジュレを確実に凝固させられるし、
しかも柔らかな口あたりに仕上げられる。
\newpage


%%% Chapitre II. Garnitures
%% II. garnitures
\hypertarget{garnitures}{%
\chapter{II. ガルニチュール Garnitures}\label{garnitures}}

\index{garnitures@garnitures} \index{かるにちゆーる@ガルニチュール}

料理においてガルニチュール\footnote{garniture
  一般的には「付け合せ」と訳すが、本書におけるガルニチュー
  ルはたんなる料理の「付け合わせ」にとどまらず、こんにちではそれ自体
  がひとつの料理として成立し得るものも多い。そのため、あえて片仮名で
  ガルニチュールとした。なお、「付け合わせ」の意味で「ガルニ」または
  「ガロニ」などというスラングを用いる調理現場もある。}は重要なものだから、料理人は決してガルニ
チュールの役割を軽視してはいけない。ガルニチュールの構成をどうするかは、
添える料理の主素材との関係性で決まる。気まぐれ的なものや不自然なもの
は絶対にいけない。

ガルニチュールの構成要素は、場合によりけりだが、もっぱらどんな種類の料
理に添えるかで決まる。具体的には、野菜料理やパスタ、ファルスでさ
まざまな形状に作ったクネル\footnote{quenelle
  仔牛肉や鶏肉、豚肉などと獣脂をすり潰して、しばしば「つ
  なぎ」として後述のパナードを加えて練り、スプーンなどを用いて整形し、
  沸騰しない程度の温度で茹でる{[}ポシェ{]}またはオーブンで焼いたもの。
  スプーンを2つ使ってラグビーボールに似た形状にしたものが代表的だが、
  他にもいろいろな形状、大きさにする。}、あるいは雄鶏のとさかとロニョン\footnote{\protect\hyperlink{garniture-financiere}{ガルニチュール・フィナンシエール}やその
  バリエーションともいえる\protect\hyperlink{garniture-godard}{ガルニチュール・ゴダー
  ル}で必須の素材。ロニョンrognonは通常なら腎臓を
  意味するが、この場合のロニョンは rognon blanc ロニョンブラン(白い
  ロニョン)とも呼ばれるもので、雄鶏の精巣のこと。}、さ
まざまな種類の茸、オリーブとトリュフ、イカや貝および甲殻類、場合によっ
ては卵、小魚、牛や羊の副生物\footnote{正肉以外の部分。例えば内臓や骨髄など。Ris
  de vea(リドヴォー)仔牛胸腺肉などはこれに含まれる。}など。

その昔、ガルニチュールというのは、マトロットやコンポート、ブルゴーニュ
風料理などのように風味付けのために用いた素材がそのまま添えられたもので
あった。

ガルニチュールにする野菜は、どういう仕立ての皿にするかで役割が決まり、
それに合うように切って形状を整え、調理する。ただし、野菜の調理法は「野
菜料理」として調理する場合と同じだ。

パスタやイカ、貝類、甲殻類についても同様のことが言える。

この章では、それぞれのガルニチュールを構成する素材とその分量を示すに留
めるので、各素材の調理法ついてはその素材に対応する章を参照すること。

\hypertarget{serie-des-farces-diverses}{%
\section[ファルス]{\texorpdfstring{ファルス\footnote{本来は「詰め物」の意で、鶏のローストの内臓を抜いた空洞部分に詰めたり、ガランティーヌやパテアンクルートの内部の詰め物などの用途に用いられる。この意味はこんにちでも変化がないが、本文にあるように、クネルにしてガルニチュールの一部にするなど、用途は多岐にわたる。本書ではファルスとして用いられるもののうち、肉および魚肉をベースにしたものをこの節にまとめて分類、説明している。したがって、ここでファルスとして挙げられていないファルスも料理によっては多い(例えば丸鶏の空洞部分に米などを詰めるのもファルス)ことに注意。}}{ファルス}}\label{serie-des-farces-diverses}}

\frsec{Série des farces diverses}

\index{garnitures@garnitures!farces@farces} \index{farce@farce}
\index{かるにちゆーる@ガルニチュール!ふあるす@ファルス}
\index{ふあるす@ファルス}

ガルニチュールの多くは、その構成要素にファルスあるいはファルスで作った
「クネル」が含まれている。ファルスはまた、多くの大きな仕立ての料理にも
使われる。ここではまずファルスの材料および作り方を示し、使い途について
は後で述べることにする。

ファルスは大きく5種に分類される。

\begin{enumerate}
\def\labelenumi{\arabic{enumi}.}
\item
  仔牛肉と脂で作るもの。すなわち古典料理における\textbf{ゴディヴォ}。
\item
  基本となる材料はさまざまだが、「つなぎ」に主としてパナードを使うもの。
\item
  近代的な手法で、生クリームを用いてふんわり泡立てたファルス。ムース、ムスリーヌに用いる。
\item
  レバーをベースとした「ファルス・\textbf{グラタン}」。種類はいろいろだが作り方は常に同じ。
\item
  \ruby{主}{おも}に\protect\hyperlink{}{ガランティーヌ}、\protect\hyperlink{}{パテアンクルート}、\protect\hyperlink{}{テリーヌ}などの冷製料理に用いるシンプルなファルス。
\end{enumerate}

\newpage

\hypertarget{les-panades-pour-farces}{%
\subsection[ファルス用のパナードについて]{\texorpdfstring{ファルス用のパナードについて\footnote{パナードは本来、パンと水、バターを弱火で時間をかけて煮た粥のようなものを意味した。本書ではその意味を拡大して肉や魚肉をベースとしたファルスを加熱する際に崩れないようにする「つなぎ」として、この語を用いている。そのため、必ずしもパンを材料としていないものが含まれている。}}{ファルス用のパナードについて}}\label{les-panades-pour-farces}}

\frsecb{Les Panades pour Farces}

\index{garnitures@garnitures!farces@farces}
\index{farce@farce!panade@les panades pour farces}
\index{かるにちゆーる@ガルニチュール!ふあるす@ファルス!はなーと@パナード}
\index{ふあるす@ファルス!はなーと@---用パナード}

ファルスに用いるパナードにはいくつもの種類がある。ファルスの種類や、そ
のファルスを添える料理の性質によって使い分けることとなる。

原則として、パナードの分量は、ファルスのベースとする素材が何であれ、そ
の半量を越えないようにすること。

卵とバターを用いるパナードの場合はレシピの分量どおりに作らなければなら
ないから、それを合わせて作るファルスの全体量のほうを調節してやること。

パナードE以外のパナードは使用する際には必ず完全に冷めた状態になってい
ること。パナードが出来上がったら、バターを塗った平皿か天板に流し広げ、
早く冷めるようにする。このとき、バターを塗った紙で蓋をするか、表面にバ
ターのかけらをいくつか置いてやり、パナードが直接空気に触れないようにし
てやること。

以下のパナードのレシピは仕上がり重量が正味500
gになるように調整してある。

したがって、必要な量のパナードを作るのに材料を増やしたり減らしたりする
のも難しくはないだろう\footnote{原文では、Rien de plus simple, donc, que
  \ldots{}
  となっており、直訳すると「これ以上に簡単なことはない」と言いきっているが、都度計算しなければならないことに変わりはないので、多少ニュアンスを柔らげて訳した。}。
\hypertarget{panades}{%
\subsection{パナード}\label{panades}}

\frsecb{Panades}

\index{panade} \index{garniture!panade} \index{garniture!farce!panade}
\index{かるにちゆーる@ガルニチュール!ふあるす@ファルス!はなーと@パナード}
\index{はなーと@パナード}
\begin{recette}
\hypertarget{panade-a}{%
\subsubsection{A. パンのパナード}\label{panade-a}}

\frsub{Panade au pain}

\index{garnitures@garnitures!farces@farces!panade a@panade A}
\index{farce@farce!panade@les panades pour farces!panade a@panade A}
\index{panade!a pain@A. --- au pain}
\index{かるにちゆーる@ガルニチュール!ふあるす@ファルス!はなーとa@パナードA. パンの---}
\index{ふあるす@ファルス!はなーと@---用パナード!a@A. パンのパナード}
\index{はなーと@パナード!a@A. パンの---}

\ldots{}\ldots{}\textbf{魚を素材にした固めのファルス用}

\begin{itemize}
\item
  \textbf{材料}\ldots{}\ldots{}沸かした牛乳3
  dl、固くなった白パン\footnote{ここではいわゆるバゲットのようなパンの外側を削り落した白い部分、
    あるいは食パンの「耳」を切り落した白い部分を使う、ということ。なお、
    フランスのパンは使う小麦粉の精白度や種類によって、pain complet
    (パンコンプレ)全粒粉パン、pain de sègle(パンドセーグル)ライ麦
    パン、精白度の高い小麦粉と食塩、塩、パン種だけで作るバゲットなどの
    pain と、バターや砂糖を加えて作るヴィエノワズリ(クロワッサンやパ
    ンオショコラ、ブリオシュなど)に分けられる。イギリスやアメリカのい
    わゆる食パン(フランス語 pain de mie パンドミ)は小麦粉、バター、
    塩、イースト菌、牛乳などで作られている。また、現代フランスでバゲッ
    トなどのパンに用いられている小麦粉の精白度は、T-55と呼ばれる灰分
    (小麦粉を燃やした際に残る炭水化物およびタンパク質以外の要素)0.5〜
    0.6%のものが主流であり、いわゆる食パンpain de mie(パンドミ)やヴィ
    エノワズリにはT-45(灰分0.5%以下)が多く用いられている。このほか
    T-65(灰分0.62〜0.75%)およびT-80(灰分0.75〜0.9%)、T-110(灰分
    1.0〜1.2%)、T-150(灰分1.4%前後、いわゆる全粒粉)のように種類が
    ある。このうちT-45およびT-55はfarine blanche(ファリーヌブロンシュ)
    と呼ばれ、T-150はfarine complète(ファリーヌコンプレット)と通称さ
    れている。灰分が高くなればそれだけ不純物が多いわけだから、粉は薄い
    茶色あるいはグレーがかった色合いになり、パンを焼く場合などはグルテ
    ン形成が難しくなりやすい。そのいっぽうで、香りゆたかなパンを実現し
    やすいという面もある。結果として、例えば全粒粉パンは香りはいいが固
    い仕上がりになりやすい。かつては精白度の低い(すなわち灰分の多い)
    粉ほど重量あたりの価格が安く、パンの価格もそれに比例していた。中世
    においてはパンの価格は基本的に1ドゥニエ(通貨単位)で、精白度の高
    いものは200〜300g、精白度の低いものは700〜800g程と大きな差があった
    という。ところで、本書では基本的に小麦粉を使う場合にその精白度について
    の指示はないが、概ねT-55またはT-45相当のもの考えていいだろう。なお、
    日本に輸入されている小麦は北米産のものがほとんどで、硬質小麦を粉に
    したものが「強力粉」、軟質小麦の場合は「薄力粉」と呼ばれ、精白度合
    いによる分類は通常なされていないが、製品としては概ねT-45相当あるい
    はそれ以上の精白度のものが多い。}の身250 g、塩5 g。
\item
  \textbf{作業手順}\ldots{}\ldots{}パンの身を牛乳に浸して完全にもどす。強火にかけて、ペー
  スト状になったパンがヘラから簡単に取れるくらいまで水気をとばす。バター
  を塗った平皿か天板に広げ、冷ます。
\end{itemize}

\maeaki

\hypertarget{panade-b}{%
\subsubsection{B. 小麦粉のパナード}\label{panade-b}}

\frsub{Panade à la farine}

\index{garnitures@garnitures!farces@farces!panade b@panade B}
\index{farce@farce!panade@les panades pour farces!panade b@panade B}
\index{panade!b farine@B. --- à la farine}
\index{かるにちゆーる@ガルニチュール!ふあるす@ファルス!はなーとb@パナードB. 小麦粉の---}
\index{ふあるす@ファルス!はなーと@---用パナード!b@パナードB. 小麦粉の---}
\index{はなーと@パナード!b@B. 小麦粉の---}

\ldots{}\ldots{}\textbf{肉、魚などあらゆるファルスに用いられる}

\begin{itemize}
\item
  \textbf{材料}\ldots{}\ldots{}水3 dl、塩2 g、バター50
  g、篩にかけた小麦粉150 g。
\item
  \textbf{作業手順}\ldots{}\ldots{}片手鍋に水、塩、バターを入れて火にかけ、沸騰させる。
  火から外して小麦粉を加えて混ぜる。再度火にかけて、\protect\hyperlink{}{シュー生地}を
  作る要領で余計な水分をとばす。上記パナードAと同様にして冷ます。
\end{itemize}

\maeaki

\hypertarget{panade-c}{%
\subsubsection{C. パナード・フランジパーヌ}\label{panade-c}}

\frsub{Panade à la Frangipane}\footnote{フランジパーヌとは製菓で用いられる、小麦粉、砂糖、卵を混ぜて牛
  乳とバニラを加えて煮、砕いたマカロンmacaronを加えたクリーム。本文
  にあるように、このパナード・フランジパーヌにはマカロンは加えないの
  で、作り方のプロセスが途中まで似ていることからの命名だろう。なお、
  本来のクレーム・フランジパーヌに用いられるマカロンは、現代日本でよ
  く知られているタイプとは異なり、すり潰したアーモンドと卵白、砂糖を
  混ぜた生地を紙の上にクルミ大に絞り出してオーブンで焼いただけのシン
  プルなもの。}

\index{garnitures@garnitures!farces@farces!panade c@panade C}
\index{farce@farce!panade@les panades pour farces!panade c@panade C}
\index{panade!c frangipane@C. --- à la Frangipane}
\index{かるにちゆーる@ガルニチュール!ふあるす@ファルス!はなーとc@パナードC}
\index{ふあるす@ファルス!はなーと@---用パナード!はなーとC@パナードC}
\index{はなーと@パナード!c@C. ---・フランジパーヌ}

\ldots{}\ldots{}\textbf{鶏のファルス、魚のファルス用}

\begin{itemize}
\item
  \textbf{材料}\ldots{}\ldots{}小麦粉125 g、卵黄4個、溶かしバター90
  g、塩2 g、こしょう1 g、おろしたナツメグの粉ごく少量、牛乳2\undemi{}
  dl。
\item
  \textbf{作業手順}\ldots{}\ldots{}片手鍋に小麦粉と卵黄を入れてよく練る。溶かしバター、
  塩、こしょう、ナツメグを加える。沸かした牛乳で少しずつ溶きのばしてい
  く。
\end{itemize}

\protect\hyperlink{creme-frangipane}{標準的なフランジパーヌ}と同様に、火にかけて5〜6分間、泡立て器で混
ぜながら煮る。ちょうどいい漉さになったら、バットに移して\footnote{débarasser
  (デバラセ)バットなどに移す、片付ける、の意。とりわけ前者の意味に注意。}冷ます。

\maeaki

\hypertarget{panade-d}{%
\subsubsection{D. 米のパナード}\label{panade-d}}

\frsub{Panade au Riz}

\index{garnitures@garnitures!farces@farces!panade d@panade D}
\index{farce@farce!panade@les panades pour farces!panade d@panade D}
\index{panade!d riz@D. --- au Riz}
\index{かるにちゆーる@ガルニチュール!ふあるす@ファルス!はなーとd@パナードD. 米の---}
\index{ふあるす@ファルス!はなーと@---用パナード!はなーとd@D. 米のパナード}
\index{はなーと@パナード!d@D. 米の---}

\ldots{}\ldots{}\textbf{いろいろなファルスに用いられる}

\begin{itemize}
\item
  \textbf{材料}\ldots{}\ldots{}米200 gすなわち2
  dlあるいは大さじ8杯。\protect\hyperlink{}{白いコンソメ}6 dl、バター20
  g。
\item
  \textbf{作業手順}\ldots{}\ldots{}米を入れた鍋にコンソメを注ぎ、バターを加える。火にかけて沸騰させたら、オーブンに入れて40〜45分間加熱する。この間、米に触れないようにすること。
\end{itemize}

オーブンから出したら、米粒がよく潰れるようにヘラでしっかりと混ぜる。その後、冷ます。

\maeaki

\hypertarget{panade-e}{%
\subsubsection{E. じゃがいものパナード}\label{panade-e}}

\frsub{Panade à la pomme de terre}

\index{garnitures@garnitures!farces@farces!panade e@panade E}
\index{farce@farce!panade@les panades pour farces!panade e@panade E}
\index{panade!e riz@E. --- à la pomme de terre}
\index{かるにちゆーる@ガルニチュール!ふあるす@ファルス!はなーとe@パナードE}
\index{ふあるす@ファルス!はなーと@---用パナード!はなーとe@パナードE}
\index{はなーと@パナード!e@E. じゃがいもの---}

\ldots{}\ldots{}\textbf{仔牛および他の白身肉の、詰め物\footnote{fourrré
  (フレ)詰め物をした。farci (ファルシ)も同様に「詰め
  物をした」の意だが、後者はより一般的で、前者はオムレツやクレープに
  中身を詰めて「包む」のが本来の意味。すなわち、このパナードを加えた
  ファルスで、何らかの素材を「包む」と解釈してもいい。とりわけこの
  fourréには日本料理の用語「射込む」をあてる場合もある。}をする大きなクネルに用いられる}

\begin{itemize}
\item
  \textbf{材料}\ldots{}\ldots{}茹でて皮を剥いたばかりの中位のサイズのじゃがいも2個、牛
  乳3 dl、塩 2g、白こしょう\undemi{} g、ナツメグ少々、バター20 g。
\item
  \textbf{作業手順}\ldots{}\ldots{}牛乳を2.5
  dlになるまで煮詰める\footnote{原文は réduire le lait d'un sixième
    直訳すると「牛乳を
    \unsixieme{}量だけ煮詰める」すなわち「\cinqsixiemes{}量まで煮詰め
    る」のだが、かえって分かりにくいだろうから、ここでは具体的な数字に
    直して訳した。分量を代えて作る場合には85%まで煮詰めるくらいと考え
    てもいいだろう。そもそも、じゃがいもの重さが曖昧なのだから、あまり
    細かい数字にこだわらず臨機応変に考えること。}。バター、調味料、
  薄く輪切りにしたじゃがいもを加え、15分間程加熱する。
\end{itemize}

このパナードはまだ少し\ruby{温}{ぬる}いくらいで使用すること。完全に
冷めてからではいけない。完全に冷めてから練ると粘りが出てしまうからだ。
\end{recette}
\hypertarget{farces}{%
\subsection{ファルス}\label{farces}}

\frsecb{Farces}

\index{farce} \index{garniture!farce}
\index{かるにちゆーる@ガルニチュール!ふあるす@ファルス}
\index{ふあるす@ファルス}

ベースとなる素材が\textbf{仔牛}、\textbf{鶏}、\textbf{ジビエ}あるいは\textbf{甲殻類}であっても、分量と作
業手順はどんなファルスでも同じだ。そのベースにする素材を代えればいいの
だから、ここでは各種ファルスの典型的なレシピを示せば充分だろう。料理で
用いられるファルスひとつひとつを説明するのに一章をあてる必要はないと思
われる。
\begin{recette}
\hypertarget{farce-a}{%
\subsubsection{A. パナードとバターを用いるファルス}\label{farce-a}}

\frsub{Farce à la Panade et au beurre}

\index{farce!a@A. --- à la Panade et au beurre}
\index{garniture!farce!a@A. Farce à la Panade et au beurre}
\index{かるにちゆーる@ガルニチュール!ふあるす@ファルス!a@A. パナードとバターを用いるファルス}
\index{ふあるす@ファルス!a@A. パナードとバターを用いる---}

(標準的なクネル、肉料理\footnote{原文 entrée
  (アントレ)、現代では「前菜」の意味で用いられるが、 本書では Relevé
  et Entrée 「ルルヴェとアントレ」すなわち肉料理の章
  に収録されているレシピ、仕立てのこと。これらのうちとりわけ大掛かり
  な仕立てのものをルルヴェ、それ以外をアントレと考えていい。本来ルル
  ヴェもアントレも魚を主素材にした仕立てが少なからずあったり、17世紀〜
  19世紀前半にかけての料理書では、いかに魚料理を大掛かりでゴージャス
  な仕立てでしかも美味なものにするか、が大きなテーマを占めていた。本
  書ではこれら四旬節の際などの「小斉」すなわち「肉断ちの料理」にあまりこだ
  わらない傾向があるために「魚料理」としてまとめられている。アントレ
  の場合は、概ね10人前を一皿に盛ったものを指し、現代でも立派にメイン
  の料理として通用するものがほとんど。実際、英語での前菜は hors-d'oeuvre
  または appetizer の語を用い、メインデュッシュには entree
  (またはフランス語のまま entrée)の語が現代でもあてられてい る。}の縁飾り
etc.)

\begin{itemize}
\item
  \textbf{材料}\ldots{}\ldots{}ていねいに筋取りをした肉1
  kg、\protect\hyperlink{panade-b}{パナードB} 500 g、塩12 g、こしょう2
  g、全卵4個、卵黄8個。
\item
  \textbf{作業手順}\ldots{}\ldots{}肉をさいの目に切って鉢に入れ、調味料を加えてすり潰す。
  いったん肉を取り出して、パナードをよくすり潰しながらバターを加える。
  肉を戻し入れ、すりこ木\footnote{pilon
    (ピロン)形状は日本のすりこ木をやや異なるのが多い。裏漉
    し用の漉し器(tamis タミ)とともに用いるピロンの場合は、棒の端に円
    盤状のやや厚い板を付けた形状のものが多かった。現代の手動式のポテト
    マッシャーのようなイメージだろうか。なお、ここでは大理石の鉢もしく
    は陶製のボウルを用いて作業していることに注意。現代ではフードプロセッ
    サなどを用いるところだろうが、かつては人力で、力を込めて丁寧に作業
    していたということは頭に留めておきたい。}で力強く練って全体をまとめる。
\end{itemize}

次に全卵と卵黄を加えて混ぜ合わせる。これは2回に分けても1回でやってもい
い。裏漉しして陶製の容器に入れる。さらに泡立て器で滑かになるまで混ぜる。

\hypertarget{nota-farce-a}{%
\subparagraph{【原注】}\label{nota-farce-a}}

どんな種類のファルスを作る場合でも、必ず少量を沸騰しない程度の温度で茹
でて\footnote{pocher (ポシェ)。}テストしてから、クネルの整形に取りかかること。

\maeaki

\hypertarget{farce-b}{%
\subsubsection{B. パナードと生クリームを用いるファルス}\label{farce-b}}

\frsub{Farce à la Panade et à la Crème}

\index{farce!b@B. --- à la Panade et à la crème}
\index{garniture!farce!b@B. Farce à la Panade et à la crème}
\index{かるにちゆーる@ガルニチュール!ふあるす@ファルス!b@B. パナードと生クリームを用いるファルス}
\index{ふあるす@ファルス!b@B. パナードと生クリームを用いる---}

(滑らかな仕上がりのクネル用)

\begin{itemize}
\item
  \textbf{材料}\ldots{}\ldots{}筋取りをした肉1
  kg、\protect\hyperlink{panade-c}{パナードC} 400 g、卵白5 個分、塩15
  g、白こしょう2 g、ナツメグ1 g、クレーム・ドゥーブル \footnote{乳酸発酵させた濃い生クリーム。フランスの生クリームについては\protect\hyperlink{sauce-supreme}{ソー
    ス・シュプレーム}訳注参照。}1\undemi{} L。
\item
  \textbf{作業手順}\ldots{}\ldots{}どんな肉を使う場合でも、卵白を少しずつ加えながらしっ
  かりとすり潰すこと。
\end{itemize}

パナードを加え、すりこ木でしっかり練り、二つの素材がよくよく混ざ
り合うようにする。

目の細かい網で裏漉しし、鍋にファルスを入れる。ヘラで滑らかになるよう混
ぜ、鍋を氷の上に置いて一時間ほど休ませる。

生クリームの\untiers{}量を少しずつ加えながら、のばしていく。最終的に残
りの\deuxtiers{}の生クリームも加えるが、これは先に泡立て器で軽く立てておくこと。

生クリームを全部加えた時点で、ファルスは真っ白で滑らかでしかも、ふんわりとし
た仕上がりにならなくてはいけない。

\hypertarget{nota-farce-b}{%
\subparagraph{【原注】}\label{nota-farce-b}}

手に入った生クリームが必ずしも最上級のものでない場合には、パナードC を
用いて\protect\hyperlink{farce-a}{バターを用いたファルス}を作った方がまだいい。

\maeaki

\hypertarget{farce-c}{%
\subsubsection{C. 生クリームを用いる滑らかなファルス /
ファルス・ムスリーヌ}\label{farce-c}}

\frsub{Farce à la Crème, ou Mousseline}

\index{farce!c@B. --- fine à la crème, ou Mousseline}
\index{garniture!farce!c@C. Farce fine à la crème, ou Mousseline}
\index{mousseline!farce mousseline}
\index{かるにちゆーる@ガルニチュール!ふあるす@ファルス!c@C. 生クリームを用いる滑らかなファルス / ファルス・ムスリーヌ}
\index{ふあるす@ファルス!c@C. 生クリームを用いる滑らかな--- / ---・ムスリーヌ}
\index{むすりーぬ@ムスリーヌ!ふあるす@ファルス・---}

(ムース、ムスリーヌ、ポタージュ用クネルなど)

\begin{itemize}
\item
  \textbf{材料}\ldots{}\ldots{}丁寧に掃除をして筋取りをした肉1
  kg、卵白4個分、クレーム・ エペス\footnote{crème épaisse fraîche
    低温殺菌の後、乳酸醗酵させたとても濃い生
    クリーム。前出のクレーム・ドゥーブルよりも濃い。}1\undemi{} L、塩18
  g、白こしょう3 g。
\item
  \textbf{作業手順}\ldots{}\ldots{}肉と調味料を鉢に入れて細かくすり潰す。卵白を少量ずつ
  加えていく。目の細かい網で裏漉しする。
\end{itemize}

これをソテー鍋に入れ、ヘラで滑らかになるまで混ぜたら、たっぷりの氷で鍋
を囲むようにして2時間冷やす。

次に、生クリームを少しずつ加えながらファルスをのばしていく。丁寧に練っ
ていくこと。またこの作業は鍋底を常に氷にあてた状態で行なうこと。

\hypertarget{nota-farce-c}{%
\subparagraph{【原注】}\label{nota-farce-c}}

\ldots{}\ldots{} 1.
上で示した生クリームの分量は平均的な数字だ。ファルスのベースとなっ
ている素材つまり肉、魚、甲殻類によってそれぞれタンパク質の特性が違
うのだから、素材に吸収される生クリームの量には多少の違いがでてくる
わけだ。

\begin{enumerate}
\def\labelenumi{\arabic{enumi}.}
\setcounter{enumi}{1}
\item
  ここで示したファルスの作り方は、滑らかな仕上がりのファルスの典型で
  あって、これを越える繊細さを出せるものはないから、ファルスに出来る
  材料すべて、つまり各種の肉、ジビエ、鶏、魚、甲殻類などに適用してい
  い。
\item
  卵白の量は、ファルスのベースと素材によって調整する必要がある。鶏や
  仔牛肉のようにアルブミンが多く含まれていて\footnote{当時の知見であることに注意。卵白が主としてアルブミンで出来てい
    るのは事実だが、肉については現代の知見と大きなズレがある。本書にお
    いて、赤身肉は「オスマゾーム」という架空の、茶褐色をした美味しさの
    エキスのようなものが豊富に含まれており、仔牛などの白身肉はアルブミ
    ンが主体であるとする考え方が随所に認められる。現代ならイノシン酸の
    「うま味」とテクスチュア、焼いた場合はメイラード反応による香気成分
    などが美味しさを感じさせる要素であると考えるところだが、フランス料
    理は長い歴史においてイノシン酸というアミノ酸の一種が「うま味」成分
    であるということを知らずに、けれども経験的にイノシン酸の比率が増え
    るようにブイヨンあるいはフォン、ソースなどの味を追究していった。イ
    ノシン酸やグルタミン酸、グアニル酸などのアミノ酸による「うま味」の
    概念そのものが、20世紀末になってようやく認知されるようになったに過
    ぎない。あくまでも「経験則」にもとづいて美味しさの探求が行なわれて
    きたと言える。}新鮮な肉であれば、成獣の
  固くなった肉を使う場合よりも量は少なくて済む。つまり、捌いたばかり
  でまだ温かい若鳥の胸肉を使ってこのファルス・ムスーズを作るのであれ
  ば、卵白は省略してもいい。
\item
  良質の生クリームが入手できる環境にあるなら、他のファルスを作るより
  もこのファルスの方がいいだろう。とりわけ、甲殻類をベースとしたファ
  ルスについては重要なことだ。
\end{enumerate}
\end{recette}
\hypertarget{godiveau}{%
\subsection[ゴディヴォ /
仔牛肉とケンネ脂のファルス]{\texorpdfstring{ゴディヴォ\footnote{ゴディヴォgodiveau
  はフランソワ・ラブレーの小説『ガルガンチュア
  とパンタグリュエル』の「第三の書」(1546年)が初出。原書の綴りは
  guodiveaulx。これは「アンドゥイエット(のようなもの)」と一般に解
  釈されている。ラブレーはこれに先立つ1534年「ガルガンチュア」(=第
  一の書)において gaudebillaux という表現を用いている。これについて
  は「ゴドビヨとは、たっぷり肥育した牛のトリップ(胃と腸)のこと」と
  本文で説明している。これらを敷衍すると、ゴディヴォはもともと牛など
  の胃や腸を刻んで詰めた腸詰すなわちアンドゥイエットのことだった、と
  考えたくなっても不思議はない。しかし、たとえ16世紀のラブレーにおけ
  るゴディヴォが当時アンドゥイエットと呼ばれるものとほぼ同じだったと
  しても、アンドゥイエット andouilette がアンドゥイユ andouille に縮
  小辞を付したものであることから、中世のアンドゥイユを確認する必要が
  出てくる。14世紀末に書かれた『ル・メナジエ・ド・パリ』においてアン
  ドゥイユは確かに「細かく刻んだ胃や腸を、腸詰にする」という説明がま
  ず出てくるが、その他に、牛の第1胃だけを詰めるもの、豚のコトレット
  を切り出した端肉を材料にするもの、胸腺肉やレバーを掃除した残りの肉
  を材料にするもの、が挙げられている(t.2,p.127)。これに従うなら、中
  世におけるアンドゥイユとは素材の定義があまりはっきりしていなかった
  もの、言える。ところが17世紀、ピエール・ド・リュヌ『新料理の本』
  (1660年)に「スペイン風アンドゥイエット」というレシピがある。概要
  を記すと、仔牛肉を細かく刻む。豚背脂少々、香草、卵黄、塩、こしょう、
  ナツメグ、粉にしたシナモンを加える。豚背脂のシートで巻いてアンドゥ
  イエットの形状にする。串を刺してローストする。ローストする際に滴り
  落ちてくる肉汁は受け皿で受ける。火が通ったらその肉汁をかける。茹で
  卵の黄身8〜10個分と細かくおろしたパン粉を順につけて、しっかりした
  衣を作る。提供時にレモン汁と羊のジュをかけ、揚げたパセリを添える、
  というものだ。1693年刊マシアロ『宮廷および大ブルジョワ料理の本』で
  は豚のアンドゥイユ、仔牛のアンドゥイユとともに、仔牛のアンドゥイエッ
  トというレシピが掲載されている。最後のものには材料として「細かく刻
  んだ仔牛肉、豚背脂、香草、卵黄、塩、こしょう、ナツメグ、シナモンを
  加えて作る」とある(pp.108-109)。また、1750年に出版された『食品、ワ
  イン、リキュール事典』でも、アンドゥイエットは「細かく刻んだ仔牛肉
  を楕円形に巻いたもの」と定義されている。実際、17、18世紀の料理書に
  出てくるアンドゥイエットは腸詰であるかどうかは別にしても、仔牛肉を
  主材料にしたものが多い。18世紀ヴァンサン・ラ・シャペル『近代料理』
  第1巻のアンドゥイエットも細かく刻んだ仔牛肉を豚の腸に詰めて作る。
  さて、ゴディヴォに戻ると、17世紀、1653年刊の『フランスのパティスリ
  の本』(ラ・ヴァレーヌが著者だと言われている)にはFaire un pasté de
  gaudiueau 「ゴディヴォのパテの作り方」という節があり、仔牛腿肉
  あるいは他の肉と脂身を細かく刻んだもの、をパテ(≒パイ包み焼き)に
  入れる。つまりここでも「仔牛腿肉」の使用が前提となっている。したがっ
  て、これら勘案すれば、ラブレーのゴディヴォもまた仔牛肉を材料にして
  いたものだった可能性は充分に考えられるだろう。 もちろんゴドビヨと
  いう別の巻で出てくる名詞との関連性は無視出来ないものだが、中世〜ル
  ネサンス期において、食にかかわる名詞、概念がしばしば曖昧だったこと
  を考えると、多少のわかりにくさは許容せざるを得ない。したがって、本
  書において仔羊腿肉とケンネ脂を使うゴディヴォを「古典的」なファルス
  として扱っているのはまことに正鵠を射ていると言えよう。} /
仔牛肉とケンネ脂のファルス}{ゴディヴォ / 仔牛肉とケンネ脂のファルス}}\label{godiveau}}

\frsecb{Farce de Veau à la Graisse de boeuf, ou Godiveau}

\index{farce!veau graisse de boeuf@--- de veau à la graisse de boeuf}
\index{garniture!farce!veau graisse de boeuf@Farce de veau à la graisse de boeuf}
\index{farce!veau glodiveau@Godiveau}
\index{garniture!farce!godiveau@Godiveau}
\index{かるにちゆーる@ガルニチュール!ふあるす@ファルス!こうしにくとけんねあふらのふあるす@仔牛肉とケンネ脂のファルス / ゴディヴォ}
\index{ふあるす@ファルス!こうしにくとけんねあふらのふあるす@牛仔牛肉とケンネ脂の--- / ゴディヴォ}
\index{かるにちゆーる@ガルニチュール!ふあるす@ファルス!こていうお@ゴディヴォ}
\index{ふあるす@ファルス!こていうお@ゴディヴォ} \index{godiveau}
\index{こていうお@ゴディヴォ}
\begin{recette}
\hypertarget{godiveau-mouille-a-la-glace}{%
\subsubsection[A. 氷を入れて作るゴディヴォ]{\texorpdfstring{A.
氷を入れて作るゴディヴォ\footnote{氷を入れて作る方法についてはカレームが1815年刊『パリ風パティ
  スリの本』の「シブレット入りゴディヴォ」原注において詳しく論じてい
  る。「不思議なことだが、氷を入れることでゴディヴォが滑らかなテクス
  チュアになり、素晴しくふんわりとしてとてもいい柔らかさに仕上がる。
  ゴディヴォが変質してしまうと、部分的とはいえそのクオリティはまった
  く失なわれてしまう。これは夏によく起こる事で、あまりに暑いとその熱
  で牛脂が仔牛肉としっかりつながらなくなってしまうからだ。一方(仔牛
  肉)は水分を含んでいて、もう一方(牛脂)は脂質そのものだからだ。だ
  から、夏の暑い時期には必ず氷を加えて作るべきであり、逆に冬場はそこ
  までする必要はない(p.142)」。ほぼ同時期のヴィアール『王国料理の本』
  1817年版においてゴディヴォのレシピの末尾に、「夏に、水の代わりに少
  量でも氷を使えるならそのほうがずっといい仕上がりになる(p.145)」と
  書かれている。これは、製氷機、冷凍庫が実用化されるのが19世紀中頃な
  ので、それよりやや早い時代ということになり、カレームの主たる活躍の
  舞台であった食卓外交というものが、いかに贅沢だったかを示していると
  も言えよう。言うまでもなく、17〜18世紀の料理書、パティスリの本にお
  いてゴディヴォのレシピは多く見られるが、氷の使用について言及したも
  のはいまのところ見つかっていない。}}{A. 氷を入れて作るゴディヴォ}}\label{godiveau-mouille-a-la-glace}}

\frsub{Godiveau mouillé à la glace}

\index{farce@farce!godiveau a@Godiveau A. Godeiveau mouillé à la glace}
\index{ふあるす@ファルス!こていうお@ゴディヴォ!a@A. 氷を入れて作るゴディヴォ}
\index{godiveau@godiveau!a@A. --- mouillé à la glace}
\index{こていうお@ゴディヴォ!a@A. 氷を入れて作る---}

\begin{itemize}
\item
  \textbf{材料}\ldots{}\ldots{}筋をきれいに取り除いた仔牛腿肉1
  kg、\textbf{水気を含んでいない}牛ケンネ脂\footnote{腎臓の周囲を厚く覆っている脂肪。融解温度が低く、精製して牛脂(ヘッ
    ト)の原料となる。}1.5 kg、全卵8個、塩25 g、白こしょう5 g、ナツメグ1
  g、 透明な氷7〜800 gまたは氷水7〜8 dl。
\item
  \textbf{作業手順}\ldots{}\ldots{}はじめに、仔牛肉とケンネ脂を別々に、細かく刻む。仔牛
  肉はさいの目に切り、調味料と合わせておく。牛脂は細かくして、薄皮は筋
  はきれいに取り除いておく。
\end{itemize}

仔牛肉と牛脂を別々の鉢に入れて、それぞれすり潰す。次にこれらを合わせて
から、完全に混ざり合って一体化するまでよくすり潰し、卵を一個ずつ、すり
潰す作業を止めずに加えていく。

裏漉しして、平皿に\footnote{大きなバット。}広げ、氷の上に置いて翌日まで休ませる。

翌日になったら、再度ファルスをすり潰す。この時、小さく割った氷を少しず
つ加えていき、よく混ぜ合わせる。

ゴディヴォに氷を加え終わったら、必ずテスト\footnote{少量を、沸騰しない程度の温度で火を通し(ポシェ)て様子を見ること。}を行ない、必要に応じて
修正する。固すぎるようなら水を少々加え、柔らかすぎるようなら卵白を少し
加えること。

\hypertarget{nota-godiveau-a}{%
\subparagraph{【原注】}\label{nota-godiveau-a}}

ゴディヴォで作ったクネルはもっぱら、\protect\hyperlink{vol-au-vent}{ヴォロヴァン}の詰め物\footnote{原文
  garniture ガルニチュールの意味が広いことに注意。}にし
たり、牛、羊の塊肉の料理に添える\protect\hyperlink{garniture-financiere}{ガルニチュール・フィナンシエール}
に用いられる。

他のクネルがどれもそうであるように、沸騰しない程度の温度で茹でて\footnote{pocher
  (ポシェ)。}
火を通せばいいが、一般的には手で整形して塩を加えた沸騰しない程度の温度
の湯で茹でる。

だが、「ポシャジャセック\footnote{pochage à sec
  直訳すると「乾燥した状態でポシェすること」。つま
  り水(湯)を用いずに、pocher と同様に低めの温度で加熱することを指
  している。}」と呼ばれる技法、すなわち弱火のオーブ
ンで焼くのがいちばんいい。

以下に示す方法はとても短時間で出来るので特にお勧めだ。

ゴディヴォは充分に氷を加えて水気を含んだ状態にしておく。オーブンの天板
に敷いたバターを塗った紙の上に、丸口金を付けた絞り袋から絞り出す。オー
ブンの天板にもバターを塗っておくこと。絞り出したクネルは触れ合うように
していい。

これを低温のオーブンに入れて加熱する。

7〜8分すると、クネルの表面に脂が水滴状に浸み出してくる。これが、ちょう
どいい具合に火が通った合図だ。オーブンから出して、クネルを別の銀製の盆
か大理石の板の上に裏返しに広げる。クネルが\ruby{微温}{ぬる}くなるまで
冷めたら、敷いてあった紙を端のほうから引き剥して取り除く。

クネルは完全に冷めるまで放置し、その後に皿に移すか、可能なら柳編みのすのこに
載せてやるのがいい。

\maeaki

\hypertarget{godiveau-a-la-creme}{%
\subsubsection{B. 生クリーム入りゴディヴォ}\label{godiveau-a-la-creme}}

\frsub{Godiveau à la crème}

\index{farce@farce!godiveau b@Godiveau B. Godeiveau  à la crème}
\index{ふあるす@ファルス!こていうお@ゴディヴォ!b@B. 生クリーム入りゴディヴォ}
\index{godiveau@godiveau!b@B. --- à la crème}
\index{こていうお@ゴディヴォ!b@B. 生クリーム入り---}

\begin{itemize}
\item
  \textbf{材料}\ldots{}\ldots{}筋をきれいに取り除いた極上の白さの仔牛腿肉1
  kg、水気を含んでいない牛ケンネ脂1 kg、全卵4個、卵黄3個、生クリーム7
  dl、塩25 g、白こしょう5 g、ナツメグ1 g。
\item
  \textbf{作業手順}\ldots{}\ldots{}仔牛肉とケンネ脂は別々に、細かく刻む。これらを鉢に入れて合わせ、調味料、全卵、卵黄をひとつずつ加えながら、力強く全体をすり潰し、完全に一体化させる。
\end{itemize}

裏漉しして、天板に広げる。氷の上にのせて翌日まで休ませる。

翌日になったら、あらかじめ中に氷を入れて冷やしておいた鉢で再度すり潰す。この際に生クリームを少量ずつ加えていく。

クネルを整形する前にテストをして、必要があれば固さなどを修正してやること。

\maeaki

\hypertarget{godiveau-lyonnais}{%
\subsubsection{C. リヨン風ゴディヴォ /
ケンネ脂入りブロシェのファルス}\label{godiveau-lyonnais}}

\frsub{Godiveau Lyonnais ou Farce de Brochet à la graisse}\footnote{このレシピは第二版以降。このファルスが仔牛肉が材料ではなくパナー
  ドも使うにもかかわらずゴディヴォの名称である根拠はおそらく、ケンネ
  脂を用いていることだろう。なお、これを用いたブロシェのクネルの起源
  については、リヨンのシャルキュトリ(豚肉加工業者)であるオ・プチ・
  ヴァテルの店主ルイ・レグロスが1907年に創案したものだという説がある。
  しかしこの説は、1907年の本書第二版にファルスとクネル両方のレシピが
  収録されているといることで否定されよう。また、19世紀前半にオーヴェ
  ルニュ・ローヌ・アルプ地方にある宿屋の主J.-F. モワーヌなる人物が、
  宿泊客を呼び込むための料理としてブロシェの身と卵、小麦粉で作ったク
  ネルを創案し、これがリヨンに伝わったという説もある。ただしこれは信
  憑性がさほど高くないうえに、そもそもケンネ脂を使わないのであれば、
  その後のリヨン風ゴディヴォとは似て非なるものということになろう。魚
  のすり身をクネルにすることはローマ時代後期の『アピキウス』(この場
  合は人物ではなく料理書の意)以来、ヨーロッパにおいてごくあたりまえ
  のように行なわれてきたことだ。いずれにしても、本書では\protect\hyperlink{quenelles-de-brochet-lyonnaise}{ブロシェの
  クネル リヨン風}のレシピでのみこ
  のファルスが用いられることになる。その意味でも、「ブロシェのクネル
   リヨン風」という料理が20世紀初頭に大流行したものだったことは間違
  いなく、そのことが理由で第二版においてレシピが追加されたと考えら
  れる。}

\index{farce@farce!godiveau c@Godiveau C. Godeiveau Lyonnais ou Farce de Brochet à la graisse}
\index{ふあるす@ファルス!こていうお@ゴディヴォ!c@C. リヨン風ゴディヴォ / ケンネ脂入りブロシェのファルス}
\index{godiveau@godiveau!c@C. --- Lyonnais ou Farce de Brochet à la graisse}
\index{こていうお@ゴディヴォ!c@C. リヨン風--- / ケンネ脂入りブロシェのファルス}

\begin{itemize}
\item
  \textbf{材料}\ldots{}\ldots{}皮とアラをきれいに取り除いたブロシェ\footnote{brochet
    ノーザンパイク、和名キタカワカマス。カワカマス属の淡水、汽水魚。}の身(正味重量)500
  g、筋を取り除き細かく刻んだ水気を含んでいない牛ケンネ脂500 g(または
  ケンネ脂と白い牛骨髄半量ずつ)、\protect\hyperlink{panade-c}{パナード
  C}500 g、卵白4 個分、塩15 g、こしょう4 g、ナツメグ1 g。
\item
  \textbf{作業手順}\ldots{}\ldots{}まず鉢でブロシェの身をすり潰す。これを取り出して、次
  にケンネ脂にパナード(よく冷やしたもの)を加えてすり潰し、卵白を少し
  ずつ加えていく。ブロシェの身と調味料を入れ戻す。すりこ木で力強く練り、
  裏漉しする。
\end{itemize}

陶製の器に移し、ヘラで滑らかになるまで練る。使うまで、氷の上に置いてお
く。

次のように作ってもいい。ブロシェの身を調味料とともにすり潰し、そこにパ
ナードを加える。裏漉しして、鉢に戻す。すりこ木で力強く練ってまとまるよ
うになったらケンネ脂を少しずつ加えるか、溶かしたケンネ脂と牛骨髄を加え
て、よくまとめる。陶製の器に移し、氷の上に置いておく。

\maeaki

\hypertarget{farce-de-veau-pour-bordures}{%
\subsubsection{盛り付けの縁飾りおよび底に敷いたり、詰め物をしたクネルに用いる仔牛のファルス}\label{farce-de-veau-pour-bordures}}

\frsub{Farce de veau pour Bordures de dressage, fonds, quenelles fourrées etc.}

\index{farce!veau@--- de veau pour Bordures de dressage, fonds, quenelles fourrées, etc.}
\index{garniture!farce!veau@Farce de veau pour Bordures de dressage, fonds, quenelles fourrées etc.}
\index{かるにちゆーる@ガルニチュール!ふあるす@ファルス!こうし@盛り付けの縁飾りおよび底に敷いたり、詰め物をしたクネルに用いる仔牛の---}
\index{ふあるす@ファルス!こうし@盛り付けの縁飾りおよび底に敷いたり、詰め物をしたクネルに用いる仔牛の---}

\begin{itemize}
\item
  \textbf{材料}\ldots{}\ldots{}筋をきれいに取り除いた\textbf{極上の白さの仔牛腿肉}1
  kg、\protect\hyperlink{panade-e}{パ ナード E} 500 g、バター300
  g、全卵5個、卵黄8個、濃い冷え
  た\protect\hyperlink{sauce-bechamel}{ベシャメルソース}大さじ2杯、塩20
  g、白こしょう3 g、ナツメグ1 g。
\item
  \textbf{作業手順}\ldots{}\ldots{}鉢に仔牛肉と調味料を入れてて細かくすり潰す。これを鉢
  から取り出す。
\end{itemize}

まだ温い状態のじゃがいものパナードを入れ、すりこ木でペースト状になるま
で練り、だいたい冷めた頃に、先にすり潰した仔牛肉を戻し入れる。全体によ
く混ぜながら、バター、全卵、卵黄をひとつずつ加えていき、最後に冷たいベ
シャメルソースを加える。

裏漉しして、陶製の器に入れ、充分に滑らかになるまでヘラで練る\footnote{装飾用であったり、中に別の食材を射込んだ大きなクネルを作る目的
  なので、加熱後はしっかりとしたテクスチュアとなる。後者についても、
  アトレと呼ばれる飾り串を刺して食材を飾るのが主要な目的だが、トリュ
  フをまるごと射込むなど、料理としてきちんと成立していたことに留意。
  なおアトレはエスコフィエの頃にはほとんどフランスでは用いられなくなっ
  ていたが、アメリカ経由で19世紀半ば頃のフランス料理をベースとして始
  まった日本の西洋料理では、むしろ20世紀になってからも使われ続けてい
  たという。}。

\maeaki

\index{garniture@garniture!farce@farce!gratin@Farce Gratin}
\index{farce@farce!gratin@--- gratin}
\index{gratin@gratin!farce@farce ---}
\index{かるにちゆーる@ガルニチュール!ふあるす@ファルス!くらたん@ファルス・グラタン}
\index{ふあるす@ファルス!くらたん@ファルス・グラタン}
\index{くらたん@グラタン!ふあるす@ファルス・---}

\hypertarget{farce-gratin-a}{%
\subsubsection{ファルス・グラタン A}\label{farce-gratin-a}}

\frsub{Frace Gratin A}\footnote{ここでゴディヴォのように小見出しがあって然るべきところだが、初
  版には小見出しの類が一切なかったので、第二版改訂の際に見落とされて
  そのままになったのだろう。本書におけるファルス・グラタンの定義が、
  決して「グラタン用」ファルスではないことに注意。語源的には gratin
  \textless{} gratter
  (グラテ)引っ掻く、であり、元来はbouillie(ブイイ)とい
  う粥のようなものの鍋底や隅に貼り付いた部分のことをグラタンと呼んだ。
  18世紀マラン『コモス神の贈り物』には「グラタン」という名称のファル
  スがある。これは、鶏胸肉、レバー、牛の骨髄、香草などと卵黄をすり潰
  して練ったもの(t.1, p.143)。また仕立てとしてのグラタンは深皿にこの
  ファルスを敷き詰め、その上に別途調理した素材をのせてソースをかけ、
  フルノーの端でファルスが容器に貼り付く程度に加熱する(仔牛の耳のグ
  ラタン(id., p.209)、エクルヴィスのグラタン(id., pp.171-172)がある。
  その後、グラタンという名称のファルスは他の料理書に記されなかったが、
  1868年のデュボワとベルナールの『古典料理』においてfarce à gratin de
  gibier, farce à gratin de foie-grasの2つのレシピが掲載され
  (p.125)、その約半世紀後『料理の手引き』において完全に復活したが、
  その頃にはグラタンという仕立てがまったく別の、こんにち我々がよく知っ
  ているものへと変わってしまっていた。このため、本書におけるグラ
  タンの説明(原書pp.405-407)においてもこれらのファルス・グラタンは用
  いられない。}

\index{farce@farce!gratin a@--- Gratin A}
\index{garniture@garniture!farce@farce!gratin a@Farce Gratin A}
\index{かるにちゆーる@ガルニチュール!ふあるす@ファルス!くらたんa@---・グラタン A}
\index{ふあるす@ファルス!くらたんa@---・グラタン A}

(標準的な温製パテ\footnote{pâté
  とは本来、生地で素材を包んで焼いたもの全般を指す。こんにち
  ではその意味が失なわれつつあり「パイ包み」のような表現をとることも
  多い。決して英語のpatty(小型のミートパイ、ハンバーガーのパティなど)と
  混同しないこと。}、大皿料理\footnote{Entrée
  アントレ。\protect\hyperlink{panade-a}{パナードとバターを用いるファルス}訳注参照。}の縁飾りなど)

\begin{itemize}
\item
  \textbf{ファルス1 kg分の材料}\ldots{}\ldots{}豚背脂250
  g、筋をきれいに取り除いた極上 の白さの仔牛腿肉1
  kg、出来るだけ白い仔牛のレバー250 g、バター150 g、
  マッシュルームの切りくず40 g、トリュフの切りくず(可能なら生のもの)
  25 g、卵黄6個、ローリエの葉\undemi{}枚、タイム1枝、エシャロット4個、
  塩20 g、こしょう4 g、ミックススパイス\footnote{原文は初版から一貫して、2
    grammes d'épices 直訳すると「香辛料2g」
    としか記されていないが、フランスでもっともポピュラーなミックススパ
    イスであるquatre-épicesカトルエピスの場合は、こしょう、ナツメグ、
    クローブ、シナモンの粉末のミックス。また「オールスパイス」単独を意
    味することもある。なお、1907年の英語版には、ローリエ5オンス、タイ
    ム3オンス、コリアンダー3オンス、シナモン4オンス、ナツメグ6オンス、
    クローブ4オンス、ジンジャーパウダー3オンス、メース3オンス、黒こしょ
    うと白こしょう同量ずつ計10オンス、カイエンヌ1オンス、を粉末にして
    保存すべし(p.75)、とあるが、フランス語原書にこのミックススパイスの
    レシピはいずれの版でも記されていない。}2 g、マデイラ酒1\undemi{}
  dl、
  \protect\hyperlink{sauce-espagnole}{ソース・エスパニョル}1\undemi{}
  dl(よく煮詰めて あって、冷やしてあること)。
\item
  \textbf{作業手順}\ldots{}\ldots{}豚背脂をさいの目に切る。ソテー鍋に50gのバターを熱し、強火で色よく焼く。
\end{itemize}

背脂が色付いたらすぐに取り出して余分な脂をきり、同じ鍋で、大きめのさい
の目に切った仔牛肉を色よく焼く。同様してに余分な脂はきる。

同じく強火で、仔牛肉と同様に切ったレバーを色よく焼く。仔牛肉と背脂を鍋
に戻し入れ、マッシュルームの切りくず、トリュフの切りくず、タイム、ロー
リエの葉、みじん切りにしたエシャロットと調味料を加える。2分程火にかけ
たままにし、バットにあける。ソテー鍋にマデイラ酒を注いでデグラセ\footnote{肉を焼く際に肉から浸み出た肉汁が濃縮して鍋底に貼り付いているの
  を、何らかの液体を注いで溶かし出すこと。意味としては「焦げ」を取る
  ことではないので注意。}する。

鉢に背脂、仔牛肉、レバーなどを入れて細かくすり潰しながら、バターの残り
(100 g)と卵黄をひとつずつ加えていく。さらに煮詰めたソース・エスパニョ
ルとデグラセしたマデイラ酒を加える。裏漉しして、陶製の容器に入れ、ヘラで
滑らかになるまで練る。

\hypertarget{nota-farce-gratin-a}{%
\subparagraph{【原注】}\label{nota-farce-gratin-a}}

このファルスのレシピでの仔牛のレバーは鶏や鴨、がちょう、七面鳥のレバー
に代えてもいい。その場合は、胆汁および胆汁で汚れた部分を丁寧に取り除く
必要がある。

\maeaki

\hypertarget{farce-gratin-b}{%
\subsubsection{ファルス・グラタン B}\label{farce-gratin-b}}

\frsub{Frace Gratin B}

(ジビエの温製パテ用)

\index{farce@farce!gratin b@--- Gratin B}
\index{garniture@garniture!farce!farce gratin b@Farce Gratin B}
\index{かるにちゆーる@ガルニチュール!ふあるす@ファルス!くらたんb@---・グラタン B}
\index{ふあるす@ファルス!くらたんb@---・グラタン B}

\begin{itemize}
\item
  \textbf{ファルス1 kg分の材料}\ldots{}\ldots{}塩漬け豚バラ肉250
  g、穴うさぎの\footnote{lapin de garenne
    (ラパンドガレーヌ)、野生の穴うさぎ。いわ
    ゆる野うさぎlièvre(リエーヴル)とは肉質も違い、まったく別のものとして扱われる。
    この穴うさぎを家畜化したものが、いわゆるlapin(ラパン)。}肉
  (正味重量)250 g、鶏とジビエのレバー250 g、マッシュルーム、トリュフ、
  タイム、ローリエ、エシャロット、塩こしょうは\protect\hyperlink{farce-gratin-a}{ファルス・グラタン
  A}と同じ。バター50 g、生あるいは加熱済みのフォワグラ 100
  g、卵黄6個、マデイラ酒1\undemi{}
  dl、ジビエで作った\protect\hyperlink{sauce-espagnole}{ソース・エスパ
  ニョル}または\protect\hyperlink{sauce-salmis}{ソース・サルミ}をよく
  煮詰めて冷ましたもの1\undemi{} dl。
\item
  \textbf{作業手順}\ldots{}\ldots{}前項で説明したように、バターで3種の素材、つまり豚バ
  ラ、うさぎ肉、レバーを別々に色よく焼く。これらをソテー鍋に調味料、香
  辛料とともに入れ、軽く炒めたらマデイラ酒を注ぎ蓋をして弱火で5分程蒸し
  煮\footnote{étuver (エチュヴェ)。}する。よく水気をきってから鉢に入れてすり潰す。充分に滑らかに
  なったら、フォワグラと卵黄、冷めたソースとマデイラ酒を加える。裏漉しし
  て、ヘラで滑らかになるまで混ぜる。
\end{itemize}

\maeaki

\hypertarget{farce-gratin-c}{%
\subsubsection{ファルス・グラタン C}\label{farce-gratin-c}}

\frsub{Frace Gratin C}

\index{farce@farce!gratin c@--- Gratin C}
\index{garniture@garniture!farce!farce gratin c@Farce Gratin C}
\index{かるにちゆーる@ガルニチュール!ふあるす@ファルス!くらたんc@---・グラタン C}
\index{ふあるす@ファルス!くらたんc@---・グラタン C}

(詰め物をしたクルトン、カナペ、小型ジビエ、仔鴨用)

\begin{itemize}
\item
  \textbf{ファルス1
  kg分の材料}\ldots{}\ldots{}生のフレッシュな豚背脂\footnote{塩漬けなどの加工をしていないということ。なお、lard
    (gras) (ラー ル グラ)は「豚背脂」を意味し、lard maigre
    (ラールメーグル)また はlard de
    poitrine(ラールドポワトリーヌ)は塩漬け豚ばら肉およびそ
    れを冷燻したものを意味する。後者はしばしば日本語で「ベーコン」と誤
    訳されるが、日本語でいう「ベーコン」は温燻、熱燻されたものであり、
    風味などが大きく異なるので注意。近年は「生ベーコン」という商品名の
    ものもあるらしく、紛らわしいので注意が必要だろう。いずれにしても、
    豚背脂は薄いシート状または長い棒状、拍子木状にして、素材の油脂分と
    風味を補う目的で使われることが多く、豚ばら肉の塩漬けおよびその冷燻
    品は拍子木状に切って(lardon ラルドン)各種料理に使われる。既に拍
    子木状にカットされたものがごく一般的に市販されており、それぞれ
    lardon(ラルドン)、lardon fumé(ラルドンフュメ)と呼ばれ非常にポ
    ピュラーな食材。}を器具を用い ておろしたもの\footnote{râper (ラペ)
    \textless{} râpe (ラープ)という器具を用いておろすこと。 Mandeline
    (マンドリーヌ)と呼ばれる野菜スライサーにこの機能が付属
    しているものは非常に多い。}300 g、鶏レバー600
  g、エシャロット4〜5個の薄切り \footnote{émincé \textless{} émincer
    (エマンセ)薄切りにする、スライスする。}、マッシュルームの切りく\footnote{マッシュルームは通常、料理として提供する際にはtourner
    (トゥル
    ネ)と呼ばれる、螺旋状の切れ込みを入れて装飾したものが使われる。こ
    の際に少なくない量の切りくずが発生するのでそれを利用する。なお
    tournerの原義は「回す」であり、包丁を持った側の手は動かさずに材料
    を回すようにして切れ目を入れたり皮を剥いたりすることを意味する料理
    用語。日本の「かつら剥き」がイメージとしては近いだろうか。}ず25
  g、ローリエの葉\undemi{}枚、 タイム1枝、塩18 g、こしょう3
  g、ミックススパイス3 g。
\item
  \textbf{作業手順}\ldots{}\ldots{}ソテー鍋に豚背脂を熱して溶かす。レバーと香辛料、調味
  料を加え、強火で\textbf{色付かないように}炒める。
\end{itemize}

\textbf{色付かないように\footnote{原文 raidir ou saisir
  (レディール ウ セジール)。前者は油脂を
  熱したフライパン等で、材料が色付かないように表面を焼き固めること。
  後者「セジール」は焼く、炒める、茹でるなど方法は問わないが、熱によっ
  て表面だけを固める(タンパク質の熱変性)ことを指す。}}と書いたように、焼き色を付けないようにするこ
とが重要。レバーはレアな焼き加減で血が滴るくらいにすると、バラ色のきれ
いなファルスに仕上がる\footnote{現代の衛生学的知見からすると、充分に加熱調理していないレバーに
  は食中毒あるいは肝炎などのリスクがあるので注意。}。

材料がだいたい冷めたら鉢に入れてすり潰す。裏漉しして、陶製の容器に移し
てヘラで練って滑らかにする。バターを塗った紙で蓋をして冷蔵する。
\end{recette}
\hypertarget{farce-pour-les-pieces-froides}{%
\subsection{冷製料理用のファルス}\label{farce-pour-les-pieces-froides}}

\vspace{-1.5\zw}
\begin{center}
\textbf{(ガランティーヌ、パテアンクルート、テリーヌ)}
\end{center}
\vspace{.5\zw}
\frsecb{Farces pour Pièces froides}
\begin{center}
\vspace{-1\zw}
\hspace{1\zw}\textbf{(Galantines --- Pâtés --- Terrines)}
\end{center}

\index{farce@farce!piece froides@--- pour les pièces froides}
\index{ふあるす@ファルス!れいせいりようりよう@冷製料理用の---}

\normalfont
\begin{recette}
\hypertarget{assaisonnement-et-liaison}{%
\subsubsection{味付けと「つなぎ」}\label{assaisonnement-et-liaison}}

\frsub{Assaisonnement et Liaison}

ガランティーヌや、パテアンクルート、テリーヌに用いる標準的なファルスは、
ファルス1kgあたり25〜30 gのスパイスソルトで調味する。最後に、肉1kgあた
りコニャック1\undemi{} dlを振りかける。

冷製料理用のファルスは以下のように3つに分類される。これらは前述の滑ら
かな口あたりのファルスやファルス・グラタンとはまったく違うものである。

「つなぎ」が必要な場合には、ファルス1 kgあたり全卵2個を加えて調整する。

\hypertarget{sel-epice}{%
\subsubsection{スパイスソルト}\label{sel-epice}}

\frsub{Sel épicé}

\index{sel epice@sel épicé} \index{すぱいすそると@スパイスソルト}
\index{こうしんりよういりしお@香辛料入りの塩 ⇒ スパイスソルト}

スパイスソルトはよく乾燥した細かい塩100 gと、こしょう20 g、ミックスス
パイス\footnote{\protect\hyperlink{farce-gratin-a}{ファルス・グラタン
  A}訳注参照。}20 gを混ぜて作る。

すぐに使わない場合は、密閉できる缶に入れて乾燥した場所で保存すること。

\hypertarget{farce-froide-a}{%
\subsubsection{ファルス A (豚肉)}\label{farce-froide-a}}

\frsub{Farce A (Porc)}

\index{farce@farce!froide a@--- pour les pièces froides A (Porc)}
\index{garniture@garniture!farce!farce froides a@Farce pour les pièces froides A (Porc)}
\index{かるにちゆーる@ガルニチュール!ふあるす@ファルス!れいせいa@冷製料理用--- A}
\index{ふあるす@ファルス!れいせいa@冷製料理用--- A}

これは豚肉の脂身のない部分と、フレッシュな背脂を同量ずつ用いる。別々に
細かく刻むこと。それを鉢に入れて合わせてすり潰し、調味と風味付けを上記
の分量比率で行なう。

ごく標準的なパテアンクルートやテリーヌに用いらる。

これは\textbf{ソーセージ用の挽肉}\footnote{chair à saucisses
  (シェラソシス)。料理書によってはよく出てくる表現なので覚えておくといいだろう。}としても使われる。

\maeaki

\hypertarget{farce-froide-b}{%
\subsubsection{ファルス B (仔牛肉と豚肉)}\label{farce-froide-b}}

\frsub{Farce A (Porc)}

\index{farce@farce!froide b@--- pour les pièces froides B (Veau et Porc)}
\index{garniture@garniture!farce!farce froides b@Farce pour les pièces froides B (Veau et Porc)}
\index{かるにちゆーる@ガルニチュール!ふあるす@ファルス!れいせいb@冷製料理用--- B}
\index{ふあるす@ファルス!れいせいb@冷製料理用--- B}

\begin{itemize}
\item
  \textbf{材料}\ldots{}\ldots{}仔牛腿肉の輪切り250
  g、さいの目に切った豚肉の脂身を含ま ない部分250
  g、フレッシュな豚背脂500 g、全卵2個、調味料とコニャック
  は上記のとおり。
\item
  \textbf{作業手順}\ldots{}\ldots{}仔牛肉、豚肉、背脂を別々に細かく刻む。調味料とともに
  鉢に入れてよくすり潰し、最後に、火を点けてアルコールをとばした\footnote{flamber
    (フロンベ)。フランベする。鍋に入れて火にかけるとコニャッ
    クのようにアルコール度の高い酒類はすぐにアルコール分が揮発して非常
    に燃えやすくなる。} コニャックを加える。裏漉しする。
\end{itemize}

このファルスは主としてガランティーヌに使うが、パテアンクルートやテリー
ヌに用いてもいい。

\hypertarget{farce-froide-c}{%
\subsubsection{ファルス C (鶏とジビエ)}\label{farce-froide-c}}

\frsub{Farce C (Volaille et Gibier)}

\index{farce@farce!froide c@--- pour les pièces froides C (Volaille et Gibier)}
\index{garniture@garniture!farce!farce froides c@Farce pour les pièces froides C (Volaille et Gibier)}
\index{かるにちゆーる@ガルニチュール!ふあるす@ファルス!れいせいc@冷製料理用--- C}
\index{ふあるす@ファルス!れいせいc@冷製料理用--- C}

このファルスの素材はいろいろだから、分量比率は使用する鶏とジビエの肉の
正味重量\footnote{poids net (ポワネット)。}から調節することになる。

例えば、中抜きしただけの丸鶏の重量\footnote{廃棄分なども含めた全重量は
  poids brut (ポワブリュット)。}が1.5 kgの場合、ガルニチュール
に使うフィレの量は500〜600gに減ってしまうことになる。そのため、ファル
スの材料の分量比率は以下のようになる。

鶏肉550 g、きれいに筋取りした仔牛肉200 g、豚肉の脂身のないところ200 g、
生の豚背脂900
g、全卵4個、\protect\hyperlink{sel-epice}{スパイスソルト}50〜60
g、コニャッ ク3 dl。

\textbf{作業手順}\ldots{}\ldots{}肉と背脂は別々にして、それぞれ細かく刻む。これを鉢に入
れて合わせ、調味料を加える。細かくすり潰しながら卵を一個ずつ加えていく。
コニャックは最後に加えること。裏漉しする。

ジビエのファルスも同様の材料の比率で、同じように作る。

\hypertarget{observation-sur-les-farces}{%
\subsubsection{冷製料理用ファルスの補足}\label{observation-sur-les-farces}}

場合によっては、ファルスB(仔牛と豚)およびファルスC(鶏)に、ファルス 1
kgあたりフォワグラ125 gを加えることがある。その場合フォワグラは出来
るだけ新鮮なものを用いて、裏漉しして加えること。あるいはトリュフのみじ
ん切り50 gを加えることもある。

ジビエのファルスCを極上の滑らかな仕上りにするには、\unquart{}量の\protect\hyperlink{farce-gratin-b}{ファ
ルス・グラタンB}と、ファルスのベースにしたジビエのフュ
メをよく煮詰めて少量加えるといい。
\end{recette}
\hypertarget{farces-speciales-pour-garnir-les-poissons-braises}{%
\subsection[魚のブレゼのガルニチュール用ファルス]{\texorpdfstring{魚のブレゼ\footnote{本書において魚を「煮る」あるいは「茹でる」場合、通常はクールブ
  イヨンか塩水で沸騰させない程度の温度で火入れをする(ポシェ)。料理
  の仕立てとしての「ブレゼ」は基本が牛、羊の赤身肉であり、仔羊、仔牛、
  家禽などはやや例外的な位置付けとして「ブレゼ」が存在する。同様に、
  サーモン、大型のトラウト、チュルボ、チュルボタンなどについても「ブ
  レゼ」という仕立ての方法が\protect\hyperlink{cuisson-des-poissons-par-le-braisage}{「第6章魚料
  理」}において説明されている ので併せて読んでおきたい。}のガルニチュール用ファルス}{魚のブレゼのガルニチュール用ファルス}}\label{farces-speciales-pour-garnir-les-poissons-braises}}

\frsecb{Farces spéciales pour garnir les Poissons Braisés}
\begin{recette}
\hypertarget{farces-poissons-braises-a}{%
\subsubsection{ファルス A}\label{farces-poissons-braises-a}}

\frsub{Farce A}

\index{farce@farce!poissons braises a@---s spéciales pour les poissons braisés A}
\index{garniture@garniture!farce!farce poissons braises a@Farce spéciale pour les poissons braisés A}
\index{かるにちゆーる@ガルニチュール!ふあるす@ファルス!さかなのふれせa@魚のブレゼのガルニチュール用--- A}
\index{ふあるす@ファルス!さかなのふれせa@魚のブレゼのガルニチュール用--- A}

\begin{itemize}
\item
  \textbf{材料}\ldots{}\ldots{}細かく刻んだ生の白子\footnote{laitance
    (レトンス)。伝統的な高級料理では鯉の白子が一般的に使
    用された。他に鯖や鰊の白子も食用とするが、日本のようにスケトウダラ
    の白子を食材とするケースはほとんどないと思われる。}250
  g、白いパンの身180 gを牛乳 に浸して絞ったもの、塩5g、こしょう1
  g、ナツメグごく少量、シブレット 10gとパセリの葉5 g、セルフイユ20
  gをみじん切りにしたもの。バター50 g、 全卵1個、卵黄3個。
\item
  \textbf{作業手順}\ldots{}\ldots{}陶製の鉢に材料をすべて入れ、木のヘラで全体をよく練り、
  完全にまとまるようにする。
\end{itemize}

\hypertarget{farces-poissons-braises-b}{%
\subsubsection{ファルス B}\label{farces-poissons-braises-b}}

\frsub{Farce B}

\index{farce@farce!poissons braises b@---s spéciales pour les poissons braisés B}
\index{garniture@garniture!farce!farce poissons braises b@Farce spéciale pour les poissons braisés B}
\index{かるにちゆーる@ガルニチュール!ふあるす@ファルス!さかなのふれせb@魚のブレゼのガルニチュール用--- B}
\index{ふあるす@ファルス!さかなのふれせb@魚のブレゼのガルニチュール用--- B}

\begin{itemize}
\item
  \textbf{材料}\ldots{}\ldots{}白いパンの身200gを牛乳に浸して絞ったもの。玉ねぎ50
  gとエシャロット25
  gを細かいみじん切りにしてバターで炒めたもの。ごく新鮮なマッシュルームをみじん切りにし、圧して余分な水分を絞ったもの。パセリのみじん切り大さじ1杯、叩き潰したにんにく1片、全卵1個、卵黄3個、塩8
  g、こしょう2 g、ナツメグごく少量。
\item
  \textbf{作業手順}\ldots{}\ldots{}ファルスAと同じ。
\end{itemize}
\end{recette}
\hypertarget{ux30afux30cdux30eb54}{%
\subsection[クネル]{\texorpdfstring{クネル\footnote{ローマ時代後期に成立した料理書アピキウスにも甲殻類やイカをはじ
  めとした各種素材のすり身を丸めて作るクネルとも呼ぶべきレシピが多く
  見られるように、とても古くからある調理だが、フランス語のquenelleと
  いう語それ自体は意外と新しく、18世紀頃に定着したと思われる。語源は
  ドイツ語の Knödel (クヌーデル)すなわちボール状にした食べものを意
  味する語からの移入と考えられている。荘厳で華麗な装飾を施した大掛か
  りな仕立てがとりわけ好まれた17、18世紀の宮廷料理においてその装飾の
  一部としてクネルの利用が広まり、発達したのだろう。また、\protect\hyperlink{godiveau}{ゴディ
  ヴォ}の訳注において触れたように、ピエール・ド・リュヌの
  アンドゥイエットなどは仔牛肉をすり潰したものを棒状にして豚背脂で包
  んで焼くという、まさしく本書におけるゴディヴォの調理法に近いもので
  あり、これもまた一種のクネルと言えるだろう。}}{クネル}}\label{ux30afux30cdux30eb54}}

\frsecb{Quenelles diverses}

クネルは大きさや形状がさまざま。

\begin{enumerate}
\def\labelenumi{\arabic{enumi}.}
\item
  粉を打った台の上で転がして小さな円筒形にする
\item
  絞り袋に詰めてバターを塗った天板に絞り出す
\item
  スプーンを使って整形する
\item
  指で丸めて、雄鶏のロニョン\footnote{ロニョンrognonは通常は腎臓のことだが、rognon
    de coq は精巣のこ と。高級食材として珍重された。}のような形状にする
\end{enumerate}

クネルの作り方のその他の詳細はよく知られていることだから、本書ではこれ
以上は述べないことにする。加熱方法についても同様としたい。

ただ、以下の点には留意していただきたい。\protect\hyperlink{garniture-financiere}{フィナンシエー
ル}や\protect\hyperlink{garniture-toulouse}{トゥールーズ}といっ
た標準的なガルニチュールに加えるクネルはコーヒースプーンを使って整形す
るか、丸口あるいは刻み模様が入る口金を使って絞り出すこと。

こうやって作る場合のクネルは平均で、ひとつ12〜15 g程度となる。

\protect\hyperlink{garniture-godard}{ガルニチュール・ゴダール}や\protect\hyperlink{garniture-regence}{レジャンス}、\protect\hyperlink{garniture-chambole}{シャ
ンボール}に使うような大きなクネルの場合は、必ずス
プーンを用いて整形し、20〜22 gの大きさにすること。

上記のような大がかりなガルニチュールでよく用いられる、装飾を施したクネ
ルの場合、大きさは40〜50 g、球形か卵形、あるいはやや長い卵形にすること。

装飾に用いる素材は、ほとんど常にトリュフ、\protect\hyperlink{saumure-liquide-pour-langues}{赤く漬けた舌
肉}のどちらか、あるいは両方を用いて、
生の卵白でクネルに貼り付けて固定する。

\protect\hyperlink{godiveau}{ゴディヴォ}のクネルは茹でずに低めの温度のオーブンで加熱し
ていいが、それ以外は1 Lあたり10 gの塩を加えた湯で、沸騰しない程度の温
度で茹でること。整形したクネルを並べたソテー鍋や天板に、沸騰した塩湯を
注ぎ、沸騰寸前の温度を保つようにして火を通すこと。
\newpage
\hypertarget{serie-des-appareiles-et-preparations-diverses-pour-garnitures-chaudes}{%
\section[温製ガルニチュール用アパレイユなど]{\texorpdfstring{温製ガルニチュール用アパレイユ\footnote{料理用語としての
  appareil アパレイユとは、具体的な何かを指す言葉
  ではなく、\textbf{ある料理を作る過程において用いられる、複数の材料を組み
  合わせたもの}、という一種の概念。現実には、キッシュのアパレイユ
  (生クリームと卵、塩漬け豚バラ肉など)、クレーム・ブリュレのアパレ
  イユ(卵黄、砂糖、生クリーム、牛乳)というように用いられるが、概ね、
  \textbf{加熱して凝固する液体または半液状のもの、およびそれらを「つなぎ」
  として固形物をあえたものを指す}、と考えていい。アパレイユの概念と
  しては、まったくの固形物である、3〜4 mmのさいの目に切った香味野菜
  (場合によってはハムも入る)である\protect\hyperlink{matignon}{マティニョン}も
  appareil à matignon と表現されることはフランスの料理書においては珍
  しくないし、本節の\protect\hyperlink{duxelles-seche}{デュクセル・セッシュ}もまたア
  パレイユの一種に含められる。実際のところ、アパレイユという語はそれ
  ぞれの調理現場および料理人によって使い方がさまざまであり、概念とし
  ての理解も必ずしも共通しているとは限らない。本書では基本的に、上述
  のように加熱凝固する液体の場合と、半固形状あるいはクリーム状のもの
  を指す場合がほとんど。この節のタイトルを直訳すると「温製ガルニチュー
  ル用のアパレイユおよびその他の仕込み」となるが、概念のレベルでいえ
  ば、この節に収められているレシピはほぼ全て「アパレイユ」と呼び得る
  ものに他ならない。ただ、そう言い切ってしまうと現実問題として理解出
  来ないであろうことを想定したのか、やや曖昧な表現になっているのだと
  思われる。また、既出の\protect\hyperlink{sauce-villeroy}{ソース・ヴィルロワ}なども
  また、ソースというよりはむしろアパレイユと呼んでおかしくないものと
  言える。なお、現代ではあまり使われなくなったかも知れない用例だが
  「電話機」や「写真機」もappareilという(正確にはそれぞれappareil
  téléphonique, appareil photographiqueだが日常会話においてはたんに
  appareilと呼ばれていた)。現代では携帯電話とりわけスマートフォンに
  カメラが付属しているため、携帯電話もかつては téléphone portatif
  (テレフォヌポルタティフ)あるいはtéléphone mobile(テレフォヌモビ
  ル)などと呼ばれたが、2010年代後半くらいからは英語からの外来語であ
  るsmartphone(スマートフォヌ)の呼び方が定着してきている。また、デ
  ジタルカメラはappareil photo numérique(アパレイユフォトニュメリッ
  ク)という。}など}{温製ガルニチュール用アパレイユなど}}\label{serie-des-appareiles-et-preparations-diverses-pour-garnitures-chaudes}}

\frsec{Série des Appareils et Préparations diverses pour Garnitures chaudes}

\index{garniture@garniture!appareils garnitures chaudes@appareils et préparations diverses pour garnitures chaudes}
\index{appareil@appareil!garnitures chaudes@--- et préparations diverses pour garnitures chaudes}
\index{かるにちゆーる@ガルニチュール!あはれいゆおんせい@温製ガルニチュールのためのアパレイユなど}
\index{あはれいゆ@アパレイユ!おんせいかるにちゆーる@温製ガルニチュールのための---など}
\begin{recette}
\hypertarget{appareils-a-cromesquis-et-a-croquettes}{%
\subsubsection{クロメスキとクロケットのアパレイユ}\label{appareils-a-cromesquis-et-a-croquettes}}

\frsub{Appareils à Cromesquis et à Croquettes}\footnote{クロケットは日本のコロッケの原型となったもので、細かく切った素材をじゃがいものピュレや\protect\hyperlink{sauce-bechamel}{ベシャメルソース}であえて円盤または円筒形に整形してパン粉衣を付けて揚げたもの。クロメスキは正六面体(サイコロ形)にすることが多く、コロッケとアパレイユが共通のため、形状が違うだけでクロケットのバリエーションという見方もあるが、ポーランド語のkromesk(薄く切ったもの)が語源とされる。}

\index{garniture@garniture!appareil@appareil!cromesquis croquettes@appareils à cromesquis et à croqeuttes}
\index{appareil@appareil!cromesquis croquettes@---s à cromesquis et à croquettes}
\index{cromesqui@cromesqui!appareil@appareils à --- et à croquettes}
\index{croquette@croquette!appareil@appareils à cromesquis et à ---}
\index{かるにちゆーる@ガルニチュール!あはれいゆ@アパレイユ!くろめすきとくろけつと@クロケットとクロメスキのアパレイユ}
\index{あはれいゆ@アパレイユ!くろめすきとくろけつと@クロケットとクロメスキの---}
\index{くろめすき@クロメスキ!あはれいゆ@---とクロケットのアパレイユ}
\index{くろけつと@クロケット!あはれいゆ@クロメスキと---のアパレイユ}

⇒ \protect\hyperlink{hors-d-oeuvres-chauds}{温製オードブル}の章を参照。

\hypertarget{appareils-a-pomme-dauphine-duchesse-marquise}{%
\subsubsection{じゃがいものドフィーヌ、デュシェス、マルキーズのアパレイユ}\label{appareils-a-pomme-dauphine-duchesse-marquise}}

\frsub{Appareils à pomme Dauphine, Duchesse et Marquise}\footnote{dauphin(王太子)、dauphine(王太子妃)、duc(公爵)、
  duchesse(公爵夫人)、mariquis(侯爵)、marquise(侯爵夫人)。いず
  れも王家、貴族の位階(爵位)を表わす語だが、特に理由もなく料理名に
  付けられることが非常に多い。}

\index{garniture@garniture!appareil@appareil!pomme dauphine@appareils à pomme Dauphine, Duchesse et Marquise}
\index{appareil@appareil!pomme dauphine@---s à pomme Dauphine, Duchesse et Marquise}
\index{dauphin@dauphin(e)!appareil@appareil à pomme ---e}
\index{duc@duc / duchesse!appareil@appareil à pomme duchesse}
\index{marquis@marquis(s)!appareil@appareil à pomme ---e}
\index{かるにちゆーる@ガルニチュール!あはれいゆ@アパレイユ!しやかいものとふいーぬ@じゃがいものドフィーヌ、デュシェス、マルキーズのアパレイユ}
\index{あはれいゆ@アパレイユ!しやかいものとふいーぬと@じゃがいものドフィーヌ、デュシェス、マルキーズの---}
\index{とふいーぬ@ドフィーヌ!あはれいゆ@アパレイユ!しやかいものとふいーぬ@じゃがいもの---、デュシェス、マルキーズのアパレイユ}
\index{てゆしえす@デュシェス!あはれいゆ@アパレイユ!しやかいものてゆしえす@じゃがいものドフィーヌ、---、マルキーズのアパレイユ}
\index{まるきーす@マルキーズ!あはれいゆ@アパレイユ!しやかいものまるきーす@じゃがいものドフィーヌ、デュシェス、---のアパレイユ}

⇒ \protect\hyperlink{legumes}{野菜料理}の章、\protect\hyperlink{pommes-de-terre}{じゃがいも}の項を参照。

\hypertarget{appareil-maintenon}{%
\subsubsection{アパレイユ・マントノン}\label{appareil-maintenon}}

\frsub{Appareils Maintenon}\footnote{マントノン夫人(出生名フランソワーズ・ドビニェ
  1635〜1719)。は
  じめはマントノン侯爵夫人としてルイ14世とモンテスパン夫人の間に生ま
  れた子どもたちの非公式な教育係となり、モンテスパン夫人の死後、ルイ
  14世と結婚した。彼女の名を冠した料理はここで言及されている\protect\hyperlink{cotelettes-maintenon}{羊のコ
  トレット マントノン}の他、卵料理、菓子など
  にある。「羊のコトレット マントノン」は彼女自身が考案したとも、ル
  イ14世付の料理人の考案ともいわれているが、いずれも憶測の域を出ない。
  なお、côtelette(コトレット)とは仔牛、羊の背肉を骨付きで肋骨1本ず
  つに切り分けたもの。日本語では、仔羊の場合ラムチョップと呼ばれるこ
  とも多い。}

\index{garniture@garniture!appareil@appareil!maintenon@appareil Maintenon}
\index{appareil@appareil!maintenon@--- Maintenon}
\index{maintenon@Maintenon!appareil@appareil ---}
\index{かるにちゆーる@ガルニチュール!あはれいゆ@アパレイユ!まんとのん@アパレイユ・マントノン}
\index{あはれいゆ@アパレイユ!まんとのん@---・マントノン}
\index{まんとのん@マントノン!あはれいゆ@アパレイユ・---}

(\protect\hyperlink{cotelettes-maintenon}{羊のコトレット マントノン}用)

\protect\hyperlink{sauce-bechamel}{ベシャメルソース}4
dlと\protect\hyperlink{sauce-soubise}{スビーズ}1
dlを半量になるまで煮詰める。

卵黄3個を加えてとろみを付ける。あらかじめマッシュルーム100 gを薄切りに
してバターでごく弱火で鍋に蓋をして蒸し煮\footnote{étuver
  (エチュヴェ)。}したものを加える。

\hypertarget{appareil-montglas}{%
\subsubsection{アパレイユ・モングラ}\label{appareil-montglas}}

\frsub{Appareils à la Montglas}\footnote{Salpicon à la
  Monglas(サルピコンアラモングラ)とも呼ばれものと
  ほぼ同じ。サルピコンはせいぜい5 mm角くらいの小さなさいの目に切った
  もののこと。羊のコトレット モングラ以外の用途としては、ブシェ(パ
  イ生地で作ったケースに詰め物をしたもの。本書ではオードブルに分類さ
  れている)やタルトレット(小さなタルト)の\textbf{アパレイユ}にする。17
  世紀のモングラ侯爵 François Clermont Marquis de Montglas (生年不
  詳〜1675)の名を冠したものらしいが、由来などは不明。}

\index{garniture@garniture!appareil@appareil!montglas@appareil à la Montglas}
\index{appareil@appareil!montglas@--- à l Montglas}
\index{montglas@Montglas!appareil@appareil à la ---}
\index{かるにちゆーる@ガルニチュール!あはれいゆ@アパレイユ!もんくら@アパレイユ・モングラ}
\index{あはれいゆ@アパレイユ!もんくら@---・モングラ}
\index{もんくら@モングラ!あはれいゆ@アパレイユ・---}

(\protect\hyperlink{cotelettes-mongras}{羊のコトレット モングラ}その他に用いられる)

以下の材料を通常より太めで短かい千切り\footnote{julienne
  (ジュリエーヌ)。}にする。\protect\hyperlink{saumure-liquide-pour-langue}{赤く漬けた舌
肉}150 g、フォワグラ150 g、茹でたマッシュ ルーム100 g、トリュフ100 g。

これらを、マデイラ酒風味の充分に煮詰めた\protect\hyperlink{sauce-demi-glace}{ソース・ドゥミグラ
ス}2\undemi{} dlであえる。バターを塗った平皿に広げ、
使うまでそのまま冷ましておく。

\hypertarget{appareil-provencal}{%
\subsubsection{プロヴァンス風アパレイユ}\label{appareil-provencal}}

\frsub{Appareils à la Provençale}

\index{garniture@garniture!appareil@appareil!provencal@appareil à la provençale}
\index{appareil@appareil!provencale@--- à la Provençale}
\index{provençal@provençale(e)!appareil@appareil à la ---e}
\index{かるにちゆーる@ガルニチュール!あはれいゆ@アパレイユ!ふろうあんすふう@プロヴァンス風アパレイユ}
\index{あはれいゆ@アパレイユ!ふろうあんすふう@プロヴァンス風---}
\index{ふろうあんすふう@プロヴァンス風!あはれいゆ@---アパレイユ}

(\protect\hyperlink{cotelettes-provencale}{羊のコトレット プロヴァンス風}用)\footnote{\protect\hyperlink{appareil-maintenon}{アパレイユ・マントノン}からこれまでの3種の
  アパレイユはいずれも、羊のコトレット(ラムチョップ)の片面だけを焼
  いて、その表面をよく\ruby{拭}{ぬぐ}い、まだ焼いていない面を下にし
  て、焼いた側の面にこれらのアパレイユを塗る、あるいは盛り上げてから
  オーブンに入れるという同工異曲とも言うべき仕立てに用いられる。ここ
  で、アパレイン・マントノンとこのプロヴァンス風アパレイユの「用途」
  の部分の原文には動詞farcirあるいはその過去分詞farci(es)が用いられ
  ているのはとても興味深いと言えよう。farcirを日本語の「詰め物をする」
  と等価と考えてはうまく理解できない例のひとつだろう。farcirの原義は
  「ファルスで満たす」であって、中に詰めることではない。なお、\href{http://cnrtl.fr/definition/farce}{TLFi
  によるファルスの定義}は、「肉な
  どと他の材料(香草や茸、細かく刻んだマロンなど)を混ぜ合わせ、スパ
  イスを加えたりして、一般的にはソースや卵、パナードでつないだもの。
  これを牛や羊の肉や家禽あるいは魚や野菜に、加熱前に加えて使用する」
  となっている。すなわち、詰めることは詰めるけれども、必ずしも空洞に
  なっている部分に詰めるというわけではないというのが言葉のうえでの意
  味。例えばマッシュルームのカサの裏側にファルス、もしくは何らかのア
  パレイユを「詰める」(日本語としては「盛る」のほうが適切かも知れな
  い)と、champignon farci (シャンピニョンファルシ)となる。}

\protect\hyperlink{sauce-soubise}{ソース・スビーズ}5
dlを充分に固くなるまで煮詰める。潰
したにんにく1片を加え、卵黄3個を加えてとろみを付ける。

\hypertarget{bordures-en-farce}{%
\subsubsection{ファルスで作る縁飾り}\label{bordures-en-farce}}

\frsub{Bordures en farce}

\index{garniture@garniture!appareil@appareil!bordures farce@bordures en farce}
\index{appareil@appareil!bordures farce@bordures en farce}
\index{bordure@bordure!farce@--- en farce}
\index{farce@farce!bordures@bordures en ---}
\index{かるにちゆーる@ガルニチュール!あはれいゆ@アパレイユ!ふあるすふちかさり@ファルスで作る縁飾り}
\index{あはれいゆ@アパレイユ!ふあるすふちかさり@ファルスで作る縁飾り}
\index{ふあるす@ファルス!ふちかざり@---で作る縁飾り}
\index{ほるてゆーる@ボルデュール ⇒ 縁飾り!ふぁるす@ファルスで作る縁飾り}
\index{ふちかさり@縁飾り!ふあるす@ファルスで作る---}

この縁飾りは、飾り付ける料理の素材とおなじ材料を中心にしたファルス\footnote{本文に指定はないが、原則としては、\protect\hyperlink{farce-de-veau-pour-bordures}{盛り付けの縁飾りおよび底に敷
  いたり、詰め物をしたクネルに用いる仔牛のファル
  ス}を用いることになるだろう。もっと
  も、料理において厳密な規定ではないので、実現可能な範囲で他のタイプ
  のファルスを用いるのもいいだろう。}を使 う。縁飾り用の形\footnote{moule
  à bordure(ムーラボルデュール)、ボルデュール型ともいう。
  大きなリング型で、表面に山形の刻み目(浮き彫り模様)の入ったタイプ
  (moule historié ムールイストリエ、またはmoule cannelé ムールカヌ
  レ)と、特に模様の入っていないプレーンなもの(moule uni ムールユニ)
  の2種に大別される。}はプレーンなものでも浮き彫り模様の入ったものでもい
いが、たっぷりとバターを塗ってからファルスを詰めて低めの温度で火を通す
\footnote{原文pocher(ポシェ)。ここまでにも何度も出てきた表現だが、茹で
  る場合は「沸騰しない程度の温度で加熱すること」であり、このように型
  に詰めた場合には湯をはった天板に型をのせてやや低温のオーブンに入れ
  てゆっくり加熱することになる。}。

プレーンな型を使う場合は、きれいに切ったトリュフのスライスやポシェした
\footnote{原文 oeuf poché をそのまま訳したが、表面に飾りとして用いるのは
  固茹で卵の白身をスライスして型抜きあるいはナイフできれいに切ったも
  のを使うことが多い。}卵の白身、\protect\hyperlink{saumure-liquide-pour-langues}{赤く漬けた舌肉}、ピスタ
チオなどで表面を装飾するといい。

浮き彫り模様の型を使う場合は上記のような装飾は省いていい。

このようなファルスで作った縁飾りを使うのはとりわけ、鶏肉料理、魚料理、牛や羊肉のソテーなど。
\end{recette}\newpage
\hypertarget{serie-des-appareiles-et-preparations-diverses-pour-garnitures-froides}{%
\section[冷製ガルニチュール用アパレイユなど]{\texorpdfstring{冷製ガルニチュール用アパレイユなど\footnote{この節は、初版で「冷製料理」の章の冒頭に概説としてまとめられて
  いたものを、第二版の改訂時に、ほぼそのままの内容で現在の位置に移動
  させられている。もちろん順序および内容の加筆も行なわれており、異同
  は少なくない。}}{冷製ガルニチュール用アパレイユなど}}\label{serie-des-appareiles-et-preparations-diverses-pour-garnitures-froides}}

\frsec{Série des Appareils et Préparations diverses pour Garnitures froides}

\index{garniture@garniture!appareils garnitures froides@appareils et préparations diverses pour garnitures froides}
\index{appareil@appareil!garnitures froides@--- et préparations diverses pour garnitures froides}
\index{かるにちゆーる@ガルニチュール!あはれいゆれいせい@冷製ガルニチュールのためのアパレイユなど}
\index{あはれいゆ@アパレイユ!れいせいかるにちゆーる@冷製ガルニチュールのための---など}

\hypertarget{mousses-mousselines-et-souffles-froids}{%
\subsection{冷製のムース、ムスリーヌ、スフレ}\label{mousses-mousselines-et-souffles-froids}}

\frsecb{Mousse, Moussseline, et Soufflé froids}

\index{mousse@mousse!froide@--- froide}
\index{mousseline@mousseline!froide@--- froide}
\index{souffle@soufflé!froid@--- froid}
\index{むーす@ムース!れいせい@冷製の---}
\index{むすりーぬ@ムスリーヌ!れいせい@冷製の---}
\index{すふれ@スフレ!れいせい@冷製の---}

温製の場合でも冷製の場合でも、\ul{ムースとムスリーヌはどちらも同じ材料から作られる}。

ムースとムスリーヌの違いは、温製でも冷製でも、通常は10人分が入る大きな
型に詰めて作るのが\ul{ムース}と呼ばれ、いっぽう、\ul{ムスリーヌ}はスプー
ンで整形したり絞り袋を使ったり、あるいは大きなクネルの形をした専用の型
に入れたりして作るが、基本的に\ul{1つ}で1人分と決まっている。スフレは
小さなスフレ型に詰める。
\begin{recette}
\hypertarget{composition-de-l-appareil-pour-mousses-et-mousseline-froides}{%
\subsubsection{冷製のムースとムスリーヌのアパレイユ}\label{composition-de-l-appareil-pour-mousses-et-mousseline-froides}}

\frsub{Composition de l'Appareil pour Mousses et Mousseline froides}

\index{garniture@garniture!appareils garnitures froides@appareils et préparations diverses pour garnitures froides!appareil mousses mousselines froides@composition de l'appareil pour mousses et mousselines froides}
\index{appareil@appareil!garnitures froides@--- et préparations diverses pour garnitures froides!appareil mousses mousselines froides@composition de l'appareil pour mousses et mousselines froides}
\index{mousse@mousse!froide@froide!composition appareil@Composition de l'appareil pour mousses et mousseline froides}
\index{mousseline@mousseline!froide@froide!composition appareil@Composition de l'appareil pour mousses et mousseline froides}
\index{かるにちゆーる@ガルニチュール!あはれいゆれいせい@冷製ガルニチュールのためのアパレイユなど!れいせいのむーすとむすりーぬのあぱれいゆ@冷製のムースとムスリーヌのアパレイユ}
\index{あはれいゆ@アパレイユ!れいせいかるにちゆーる@冷製ガルニチュールのための---など!れいせいのむーすとむすりーぬのあぱれいゆ@冷製のムースとムスリーヌのアパレイユ}
\index{むーす@ムース!れいせい@冷製!むーすとむすりーぬのあぱれいゆ@ムースとムスリーヌのアパレイユ}
\index{むすりーぬ@ムスリーヌ!れいせい@冷製!むーすとむすりーぬのあぱれいゆ@ムースとムスリーヌのアパレイユ}

\begin{itemize}
\tightlist
\item
  \textbf{材料}\ldots{}\ldots{}主素材のピュレ\footnote{本書では加熱した肉や魚、甲殻類のピュレを作る方法への言及はないが、
    \textbf{本章冒頭にある\protect\hyperlink{farce-mousseline}{ファルス・ムスリーヌ}をそのま
    ま使おうなどと考えてはいけない。ここで説明されている冷製のムース、
    ムスリーヌ、スフレの作り方に加熱の工程がまったく含まれていないのは、
    主素材のピュレが既に加熱済みであることを当然の前提としている}から
    だ。つまりここで材料として示されているピュレは\textbf{すべて加熱済みのも
    のをピュレにしたものだ}と考えなければならない。『料理の手引き』の
    当時はローストするか茹でるなどの加熱後に、鉢に入れてすり潰し、裏漉
    ししてから何らかのソース(ここではヴルテ)を加えて漉さ(固さ)を調
    節するなどしていた。現代ではフードプロセッサーや冷凍粉砕調理機など
    を利用すればより容易に滑らかなピュレを作ることが可能だろう。また、
    第3章ポタージュに\protect\hyperlink{les-purees}{ポタージュ・ピュレ}についての概説が
    あるが、そこではポタージュにすることを前提として「つなぎ」の使用が
    作業のプロセスに組込まれて説明されているために、あくまで参考程度に
    読むのがいいだろう。}1 Lすなわち鶏のピュレ、ジビエ、フォワグラ
  や魚、甲殻類のピュレ。溶かした\protect\hyperlink{gelees-ordinaires}{ジュレ}2\undemi{}
  dl、\protect\hyperlink{veloute}{ヴルテ}4 dl、生クリーム4
  dlはちょうどいい固さに立てて6 dl相当にしておく。
\end{itemize}

素材の特性によって、これらの分量比率は多少変更してもいい。同様に、ある
種のムースを作る際にはジュレまたはヴルテのどちらかしか用いなくてもいい。

\begin{itemize}
\tightlist
\item
  \textbf{作業手順}\ldots{}\ldots{}まずベースとなるピュレを入れたボウルを氷の上に置いて、軽
  く混ぜながら、ジュレとヴルテを加える(どちらかしか使わない場合は使う
  もののみ)。次に泡立てた生クリームを加える。
\end{itemize}

味付けを確認する。これは冷製料理ではとても重要なことだ。いつも気をつけ
て確認し、修正を加えるようにすること。

\hypertarget{nota-composition-de-l-appareil-pour-mousses-et-mousseline-froides}{%
\subparagraph{【原注】}\label{nota-composition-de-l-appareil-pour-mousses-et-mousseline-froides}}

生クリームは五分立てすること。完全に立ててしまうと、ムースに滑らかさが
失なわれてパサついた仕上りになってしまう。

\hypertarget{moulage-des-mousses-froides}{%
\subsubsection{冷製ムースの型詰め}\label{moulage-des-mousses-froides}}

\frsub{Moulage des Mousses froides}

\index{garniture@garniture!appareils garnitures froides@appareils et préparations diverses pour garnitures froides!moulage mousses froides@moulage des mousses froides}
\index{appareil@appareil!garnitures froides@--- et préparations diverses pour garnitures froides!moulage mousses froides@moulage des mousses froides}
\index{mousse@mousse!froide@froide!moulage@moulage des mousses froides}
\index{かるにちゆーる@ガルニチュール!あはれいゆれいせい@冷製ガルニチュールのためのアパレイユなど!れいせいむーすのかたつめ@冷製ムースの型詰め}
\index{あはれいゆ@アパレイユ!れいせいかるにちゆーる@冷製ガルニチュールのための---など!れいせいむーすのかたつめ@冷製ムースの型詰め}
\index{むーす@ムース!れいせい@冷製!むーすのかたつめ@ムースの型詰め}

いまもそうしている料理人は少なくないようだが、かつては、プレーンな型あ
るいは浮き彫り模様の付いた型の中に透明なジュレを流して層をつくってやり\footnote{chemiser
  (シュミゼ)ジュレなどを型の内側に流して薄い層を作ること。}、
ムースの主素材と関連あるものを装飾要素として貼り付けていた。

こんにちでは次の方法がむしろ好ましい。銀製のタンバル型\footnote{timbale
  (タンバル)円筒形の比較的浅い型。野菜料理用の深皿もこの語で呼ぶので注意。}の底面だけに
透明なジュレの薄い層をつくる。型の側面の外側に紙の帯を冷たいバターで貼
り付ける。型の\ruby{縁}{ふち}から2〜3 cmくらい高くなるようにすること。
そうするとスフレのような見た目のムースになる。紙の帯は型の内側に貼り付
けてもいい。この紙の帯は提供直前に、ぬるま湯で濡らしてナイフの刃を使っ
てムースからそっと引き剥してやる。

タンバル型の用意が整ったら、ムースを詰めて冷やす。アイスクリーム用の冷
凍庫に入れるほうがいいだろう。この方法は、小さな銀製のスフレ型に詰めて
やってもいいが、それは冷製のスフレにとっておいたほうがいいだろう。アパ
レイユの構成が同じであるにもかかわらず、冷製ムースと冷製スフレの違いを
はっきりさせることが出来るからだ。

とりわけジビエのムースやフォワグラのムースについては、近代的な料理の提
供方法に合わせて作られた銀製かガラス製の容器を用いてもいい。その場合は、
型の底面だけジュレの層をつくってやり、アパレイユをそのまま流し込めばい
い。表面はパレットナイフなどで丁寧に滑らかにならしてやってから、ムース
を冷やす。その後\footnote{型から出して、ということだろう。}、ムースに直接装飾を施し、ジュレをかけて艶を出させる。

ジビエのムースの場合には、そのジビエの胸肉を冷やして、ムースの周囲に飾
るようにする。

\hypertarget{moulage-des-mousselines-froides}{%
\subsubsection{冷製ムスリーヌの整形}\label{moulage-des-mousselines-froides}}

\frsub{Moulage des Mousselines froides}

\index{garniture@garniture!appareils garnitures froides@appareils et préparations diverses pour garnitures froides!moulage mousselines froides@moulage des mousselines froides}
\index{appareil@appareil!garnitures froides@--- et préparations diverses pour garnitures froides!moulage mousselines froides@moulage des mousselines froides}
\index{mousseline@mousseline!froide@froide!moulage@moulage des mousselines froides}
\index{かるにちゆーる@ガルニチュール!あはれいゆれいせい@冷製ガルニチュールのためのアパレイユなど!れいせいむすりーぬのかたつめ@冷製ムスリーヌの型詰め}
\index{あはれいゆ@アパレイユ!れいせいかるにちゆーる@冷製ガルニチュールのための---など!れいせいむすりーぬのかたつめ@冷製ムスリーヌの型詰め}
\index{むすりーぬ@ムスリーヌ!れいせい@冷製!むすりーぬのかたつめ@ムスリーヌの型詰め}

冷製ムスリーヌの型詰めには2つの方法がある。たんに、型にジュレの層を作っ
てやるか、ソース・ショフロワの層を作ってやるかの違いでしかない。どちら
の場合でも、卵形の型に詰めるか、大きなクネルの形状のものにするか、とい
うことになる。

\hypertarget{procede-un-moulage-des-mousselines-froides}{%
\subparagraph{方法1\ldots{}\ldots{}}\label{procede-un-moulage-des-mousselines-froides}}

型の内側に透明なジュレを流して薄い層を作ってやる\footnote{chemiser
  (シュミゼ)。}。その上にアパレイ
ユを張るように塗り、アパレイユのベースとなっている素材とおなじもの---
鶏、ジビエ、甲殻類の身など、とトリュフ---で構成された\protect\hyperlink{salpicons-divers}{サルピコ
ン}を盛り込む。その上からアパレイユを塗って覆い、パ
レットナイフなどを使ってドーム形に滑らかにならす。冷蔵庫に入れて冷し固
める。

\hypertarget{procede-deux-moulage-des-mousselines-froides}{%
\subparagraph{方法2\ldots{}\ldots{}}\label{procede-deux-moulage-des-mousselines-froides}}

型の内側にアパレイユを詰め、さらにサルピコンをその内側に射込む。アパレ
イユで覆って、冷し固める。

型から外す。ムスリーヌのアパレイユの素材と関連性のある\protect\hyperlink{sauce-chaud-froid-ordinaire}{ソース・ショフ
ロワ}を表面を覆うように塗る\footnote{napper
  (ナペ)。覆いかける(ように塗る)こと。}。トリュフおよびそ
の他の素材(これもムスリーヌと関連性があること)を装飾用に細工したもの
を飾り付ける。装飾が剥れないように、上からジュレを塗って艶を出させる。

銀製またはガラス製の深皿の底に透明なジュレの層を作り、その上にムスリー
ヌを並べる。再度ジュレを上からかけてやり、冷蔵庫に入れて提供するまで保
管しておく。

\hypertarget{souffles-froids}{%
\subsubsection{冷製スフレ}\label{souffles-froids}}

\frsub{Soufflés froids}

\index{garniture@garniture!appareils garnitures froides@appareils et préparations diverses pour garnitures froides!souffles froids@soufflés froids}
\index{appareil@appareil!garnitures froides@--- et préparations diverses pour garnitures froides!souffles froids@soufflés froids}
\index{souffle@soufflé!froid@---s froids}
\index{かるにちゆーる@ガルニチュール!あはれいゆれいせい@冷製ガルニチュールのためのアパレイユなど!れいせいすふれ@冷製スフレ}
\index{あはれいゆ@アパレイユ!れいせいかるにちゆーる@冷製ガルニチュールのための---など!れいせいすふれ@冷製スフレ}
\index{すふれ@スフレ!れいせい@冷製---}

冷製スフレはムースそのものに他ならない。だから構成はまったく同じだ。た
だ、先に見たようにスフレが10人分\footnote{1 service
  (アンセルヴィス)、格式のある宴席料理などを作る際の単位。基本は10人分。}を確保できるだけの大きな型に詰める
のに対して、スフレはそもそも、小さなスフレ型に入れてひとり1つ宛で作る
ものだ。

アパレイユを型に詰める方法はムースの場合と同様、つまり、スフレ型の底に
ジュレの層を敷いてその上にアパレイユを盛り、型の縁より高くなるように周
囲に巻いた紙の帯を利用して縁より高くアパレイユを盛る。そうすると、冷や
し固めた後で紙の帯を取り除けば、まるで温製のスフレのように見えることに
なる。

\hypertarget{nota-souffles-froids}{%
\subparagraph{【原注】}\label{nota-souffles-froids}}

ここまで述べた3種の作り方の基礎はおなじだから、ポイントは次のようにまとめられる。

\begin{enumerate}
\def\labelenumi{\arabic{enumi}.}
\item
  ムースは「スフレ」の名称で供してもいいものだが、混同されるのを避けるために「ムース」の名称で約10人分をひとつの型に入れて作る。
\item
  ムスリーヌはサルピコンを射込んだものであってもそうでなくても、大きなクネルであって、ひとりあたり1つにする。
\item
  スフレは小さなムースであって、スフレ型あるいは似たような型に詰めて、これもひとりあたり1つとする。
\end{enumerate}
\end{recette}
\hypertarget{aspics}{%
\subsection{アスピック}\label{aspics}}

\frsecb{Aspics}

\index{garniture@garniture!appareils garnitures froides@appareils et préparations diverses pour garnitures froides!aspics@aspics}
\index{appareil@appareil!garnitures froides@--- et préparations diverses pour garnitures froides!aspics@aspics}
\index{aspics@aspics (généralité)}
\index{かるにちゆーる@ガルニチュール!あはれいゆれいせい@冷製ガルニチュールのためのアパレイユなど!あすぴつく@アスピック(概説)}
\index{あはれいゆ@アパレイユ!れいせいかるにちゆーる@冷製ガルニチュールのための---など!あすぴつく@アスピック(概説)}
\index{あすぴつく@アスピック(概説)}

アスピックを作る際に、肝に銘じておくべき第一のポイントは、どんなアスピッ
クでも、ジュレがジューシー\footnote{原文succulent(スュキュロン)はsuc(スュック=肉汁)から派生した
  形容詞で、もともとは「汁気の多い」の意味だったが、そこから転じて
  「美味な、滋味に富んだ」の意味で一般的に用いられている。ここでは、
  両方のニュアンスで表現されていると解釈できる。}で美味しく、完全に透き通ったもので、ちょうど
いい加減に固まっていなければならないことである。

アスピックを作る際には、昔もそうだったが現代でも、中央に穴の空いたアス
ピック型\footnote{moule à douille
  (ムーラドゥイユ)サヴァラン型のような中央に穴 が空いた型。
  現代では「アスピック型」というと楕円形で中央に穴のな
  いものを指すことが多いが、それとは異なる。あるいはクグロフ型のよう
  なものをイメージするとわかりやすいだろう。19世紀、アスピックには高
  さのある型が多く用いられたようだ。なお、現代では一般にサヴァラン型
  というと、型の高さや穴の大きさ等さまざまなタイプのものをまとめて指
  すことになるので注意。高さのない(低い)、中央の穴が大きな型につい
  て、エスコフィエはボルデュール型 moule à bordure (ムーラボルデュー
  ル)と呼んで区別している。}でプレーンなもの、波模様等の装飾のあるものが用いられている。

ボルデュール型\footnote{moule à bordure
  (ムーラボルデュール)蛇の目形に料理の縁り飾り
  を作るための、やや丈が低く中央の穴が大きいリング型。}も使われることがあるが、一般的に、アスピックの中心にガル
ニチュールを盛り込む場合のみである。

アスピックを型に入れる時には、まず、型の底と周囲に装飾をする。

そのために、型は砕いた氷の中に入れてよく冷やしておく。やや固まりかけた
ジュレ少量を流し入れ、型を氷の上で転がしながらジュレを周囲に貼り付かせ
る\footnote{chemiser (シュミゼ)。}。次に、装飾するパーツを、固まらない程度に冷たいジュレに浸してからす
ぐに貼り付ける。装飾については料理人のセンスとアイデア次第なので、ここ
で明確に述べておくべきことはほとんどない。ひとつだけ言えるのは、常に正
確な作業を期して、型からアスピックを出したときに装飾がはっきりと見える
ようにすべき、ということだけだ。

装飾に用いる素材はアスピックの主素材と関連性のあるものでなくてはならな
い。一般的には、トリュフ、ポシェした卵白、コルニション\footnote{cornichon
  主としてピクルスにする小型のきゅうり、およびそのピク
  ルスのこと。日本では、ハンバーガーによく用いられているドイツ系のピ
  クルス用品種であるガーキンス(英 gherkins 独 Einlegegurken)と混同さ
  れることがあるが、コルニションはより小さなサイズで収穫し、フレッシュ
  な状態では「いぼ」が尖っているのが特徴。}、ケイパー、
いろいろな香草の葉先、ラディッシュの薄い輪切り、オマールのコライユ\footnote{胴の背側にあるオレンジ色がかった「内子」。}、
\protect\hyperlink{saumure-liquide-pour-langues}{赤く漬けた舌肉}、等。

アスピックの具材が種々のエスカロップ\footnote{escalope
  (エスカロップ)筋線維とは垂直方向に、厚さ1〜2 cmに薄
  切りにした仔牛などの肉や魚の薄い切り身。}や長方形に切ったフォワグラ等
で、型の大きさから何度も並べなければならない場合、ジュレの層と交互に重
ねて型に入れていく。新しい層を並べる際には先に入れたジュレがある程度固
まってからにする。

アスピックの型入れでは常に、最後のジュレの層を充分な厚みにする。できる
だけ、型を氷に埋めるようにしながらジュレを流し込んでいくが、早く冷やす
ために氷に塩を加えてはいけない。塩を使うとジュレの透明さが損なわれるか
らである。

\noindent\textbf{型から外す方法}\ldots{}\ldots{}型を湯につけてただちに水気を拭い、
折ったナフキンや彫刻した氷のブロック等に、アスピックを裏返して型から出す。

菱形や正方形に切ったジュレのクルトン\footnote{パンで作るクルトンとは別に、菱形やさいの目に切った冷製料理装飾用
  のジュレもクルトンと呼ぶ。}、またはアシェしたジュレで周囲を飾 る。

\hypertarget{nota-aspics}{%
\subparagraph{【原注】}\label{nota-aspics}}

アスピックを型に入れて作るには、必然的に、ジュレが相当に固いものでなけ
ればならないが、これはまことによろしくない。そもそも固いジュレは口あた
りがよくないのだ。だから現代の調理現場では、以下のような方法を採ってい
る。タンバル型か、氷に嵌め込むようにした銀やガラスあるいは陶製の深皿の
底に予めジュレの層を作って固めておき、その上にアスピックの素材を並べる。
次に、固まりかけのジュレをたっぷり覆いかける。この方法では、装飾をする
必要がある場合は、アスピックの調理をおこなう前に、主素材にじかに装飾す
ることになる。

\hypertarget{chauds-froids}{%
\subsection[ショフロワ]{\texorpdfstring{ショフロワ\footnote{chaud-froid
  このchaud(熱い)とfroid(冷たい)の合成語の複数形は、それぞれ
  にsを付ける、chauds-froidsとなる。合成語の複数形はいろいろなパターンがあるので、必要が
  出たらその都度覚えるようにしたほうがいい。}}{ショフロワ}}\label{chauds-froids}}

\frsecb{Chauds-froids}

\index{garniture@garniture!appareils garnitures froides@appareils et préparations diverses pour garnitures froides!chauds-froids@chauds-froids}
\index{appareil@appareil!garnitures froides@--- et préparations diverses pour garnitures froides!chauds-froids@chauds-froids}
\index{chauds-froids@chauds-froids (généralité)}
\index{かるにちゆーる@ガルニチュール!あはれいゆれいせい@冷製ガルニチュールのためのアパレイユなど!しよふろわ@ショフロワ(概説)}
\index{あはれいゆ@アパレイユ!れいせいかるにちゆーる@冷製ガルニチュールのための---など!しよふろわ@ショフロワ(概説)}
\index{しよふろわ@ショフロワ(概説)}

\protect\hyperlink{sauce-chaud-froid-ordinaire}{ソース・ショフロワ}には大抵の場合、切り
分けた素材を浸す。が、時として大きな塊肉全体をソース・ショフロワで覆わ
なくてはならない場合もある。ただ、そういう仕立てにする場合には、別の料
理名となっている。

ショフロワが複数のばらばらのパーツからなる場合には、それらをソース・ショ
フロワに漬けたら網の上に並べておく。ソースが冷えたら、それぞれのパーツ
に装飾をし、ジュレを覆いかけて艶を出してやる。さらに盛り付けの際にはみ
出す余分なソースについてはきれいに取り除いておくこと。

大きな塊肉の場合は、よく冷えてはいるけれどまだ流動性のある状態のソース・
ショフロワを一気に塗りつけて、その後に装飾をし、ジュレを塗って艶出しす
ること。

切り分けた素材からなるショフロワの盛り付けは、\protect\hyperlink{fonds-de-plats}{皿の上の
台}の上に盛り付けてもいいし、縁飾りの内側に、パンまた
は米、セモリナ粉で作った台を置いてその上に盛り付けてもいい。あるいは、
銀製か陶製、ガラス製の深皿に盛り付けてもいい。

大きな塊肉のショフロワの場合、皿の上の台にのせてもいいし、あるいは、氷
のブロックに料理が嵌まるようにブロックを削ってからそこに盛り付けるのも
いい。

ショフロワ仕立ての鶏やジビエについては、正確に切り分けて\footnote{基本的に鶏および鳥類のジビエの可食部は胸肉のみとされていたこと
  に留意。}皮は剥いでおく
こと。手羽や下腿肉は使わないので、別の用途に取り置いておくといい。

細かく切った素材のショフロワ仕立ての場合、添えてやるマッシュルームや雄
鶏のとさかとロニョン\footnote{rognon
  (ロニョン)牛、羊などの場合は腎臓だが、雄鶏の場合は精巣
  のこと。高級食材として珍重された。}にもソース・ショフロワを塗ってやること。トリュ
フはただジュレをかけて艶を出すだけでいい。

\hypertarget{pains-froids}{%
\subsection[パンフロワ]{\texorpdfstring{パンフロワ\footnote{pain froid
  直訳すると「冷たいパン」だが、いわゆるパンとはまった
  く違う。語の概念としては「パンに似た塊」のこと。}}{パンフロワ}}\label{pains-froids}}

\frsecb{Pains froids}

\index{garniture@garniture!appareils garnitures froides@appareils et préparations diverses pour garnitures froides!pain froids@pain froids}
\index{appareil@appareil!garnitures froides@--- et préparations diverses pour garnitures froides!pain froids@pain froids}
\index{pains froids@pains froids (généralité)}
\index{かるにちゆーる@ガルニチュール!あはれいゆれいせい@冷製ガルニチュールのためのアパレイユなど!はんふろわ@パンフロワ(概説)}
\index{あはれいゆ@アパレイユ!れいせいかるにちゆーる@冷製ガルニチュールのための---など!はんふろわ@パンフロワ(概説)}
\index{はんふろわ@パンフロワ(概説)}

古典料理におけるパンフロワとは、ファルスで出来たアパレイユを型に詰めて
比較的低温で加熱調理し、冷ましてから型から出して装飾を施し、ジュレをか
けて艶を出させたものでしかない。

近代の料理においてこの方法は用いられなくなっており、一般的にいって、パ
ンフロワの代わりとしてムースが作られるようになったわけだ\footnote{この段落は第四版でかなり分量が減らされ、内容も書き換えられている。
  結果として大きく削られた後半部分の初版の文章は以下のとおり。
  「(近代の)パンフロワはいずれも、その中心となるアパレイユが次の
  構成になる。(1)そのパンフロワの主素材からひいた香りゆたかなフュメ
  をほとんどグラス状に煮詰めたものと、卵黄とバターを\protect\hyperlink{sauce-hollandaise}{オランデーズソー
  ス}のように立てたもの。(2)このアパレイユが温い
  かどうかくらいまで冷めたら、溶かしたゼラチンを布で漉しながら流し入
  れ、さらに主素材のピュレと、それと同量の泡立てた生クリームを加える。
  (3)最後にこのアパレイユに、主素材から切り出した薄切り肉(エスカロッ
  プ)にトリュフのスライスを重ねていく。あるいは単純に、肉とトリュフ
  をさいの目に切ったものでもいい。このようにして作ったアパレイユを、
  あらかじめジュレを内側に流して層を作っておいた型に流し入れ、冷やす、
  もしくは氷室に入れる。提供直前に、ぬるま湯にさっと型を浸していから
  米かセモリナ粉で作った台の上に裏返してのせてやる。あるいは皿の底に
  ジュレを敷いただけでもいい。このパンフロワの周囲に、きちっと正確な
  形状に切ったジュレのクルトンを飾る。【原注】ジュレによるクルトン
  については、冷製料理全般にあてはまる」(p.582)。}。

\hypertarget{garnitures-de-mets-froids}{%
\subsection{冷製料理のガルニチュール}\label{garnitures-de-mets-froids}}

\frsecb{Garnitures de Mets froids}

\index{garniture@garniture!appareils garnitures froides@appareils et préparations diverses pour garnitures froides!garnitures mets froids@garnitures de mets froids}
\index{appareil@appareil!garnitures froides@--- et préparations diverses pour garnitures froides!garnitures mets froids@garnitures de mets froids}
\index{garnitures mets froids@garnitures de mets froids (généralité)}
\index{かるにちゆーる@ガルニチュール!あはれいゆれいせい@冷製ガルニチュールのためのアパレイユなど!れいせいりようりのかるにちゆーる@冷製料理のガルニチュール(概説)}
\index{あはれいゆ@アパレイユ!れいせいかるにちゆーる@冷製ガルニチュールのための---など!れいせいりようりのかるにちゆーる@冷製料理のガルニチュール(概説)}
\index{れいせいりようりのかるにちゆーる@冷製料理のガルニチュール(概説)}

料理に合わせて、ガルニチュールは以下のようなもので構成すること。

\begin{itemize}
\tightlist
\item
  固茹で卵を半割りまたは四つ割りにして詰め物をし、装飾を施してジュレをかけて艶を出したもの
\item
  小さなトマトファルシが、いろいろな食材を添えたもの、または大きなトマトに何らかの詰め物をして正確に櫛切りにしたもの
\item
  小さな野菜皿または舟形の皿に盛った野菜サラダ
\item
  トマトピュレにジュレを混ぜて塗った小さなパンまたはタルトレット
\item
  真っ白なレチュ\footnote{いわゆる「サラダ菜」に属する系統の結球レタスのこと。現代では
    sucrine(シュクリーヌ)あるいはジェムと呼ばれる小型の品種が好んで
    使われる傾向にある。}の中心部分
\item
  アンチョビのフィレ、オリーブなど\ldots{}\ldots{}
\end{itemize}

\newpage

\hypertarget{serie-des-garnitures}{%
\section{ガルニチュール}\label{serie-des-garnitures}}

\frsec{Série des Garnitures}

\hypertarget{consideration-sur-la-modification-de-forme-que-peuvent-subir-les-garniture}{%
\subsection{ガルニチュールの見た目を変えることについて}\label{consideration-sur-la-modification-de-forme-que-peuvent-subir-les-garniture}}

\vspace{-1\zw}
\begin{center}
\textit{Considérations sur le modifications de forme que peuvent subir les Garnitures.}
\end{center}
\vspace{1\zw}

\index{garniture@garniture!consideration modification forme garnitures@Considération sur les modificqtions de forme que peuvent subir les Garnitures}
\index{consideration@garnitures de mets froids (généralité)}
\index{かるにちゆーる@ガルニチュール!みためをかえることについて@---の見た目を変えることについて}

他のどんなレシピでもそうだが、それぞれのガルニチュールの構成上の約束事
を勝手に変えてはいけない。もし、どうしても何らかの変更が必要なら、料理
本体に合わせて、配置を変えるとか、見た目の形状を変えるだけにすること。
ガルニチュールを構成している素材を変えてはいけない。

そうすれば、「牛フィレ肉」のような大きな塊で供する料理か、「トゥルヌド
\footnote{牛フィレ肉を厚さ約2 cmに切ったもの。周囲に豚背脂のシートを巻い
  て調理することが多いが、アメリカもしくはイギリス経由で周囲にベーコ
  ンを巻く調理法が日本に伝わったために、混同されやすいので注意。\ul{フランス料理としては、豚背脂\\のシートを巻く}。}」のような調理かにかかわらず、同じガルニチュールを合わせることが
出来るが、その場合は必然的に、ガルニチュールの形状や盛り付けにおける配
置などは変更せざるを得ないわけだ。そうしないと、主素材とガルニチュール
の関係性が保てなくなる。

これは、薄切りにしたフィレ肉とシャトーブリヤン\footnote{牛フィレ肉の太い部分、およびそれを約3
  cmの厚さに切ったもの。}の場合も同様だ。理屈からいって当然だろう。

だから、この節において示しているガルニチュールの分量は10人分を基本とし
ているが、大きな塊肉の料理に添えるか、1人分ずつに切って調理して供する
かで、量を増やしたり減らしたりすることになる。

これはとても重要なことだ。というのも、本書はフランス料理の伝統的な作り
方を集めた本なのだから、多種多様なガルニチュールを収録せざるを得なかっ
たが、その中には近代的な料理にはもはやふさわしくないものだって含まれて
いる。近代的な料理は何よりもまず複雑さを\ruby{厭}{いと}い、ガルニチュー
ルをシンプルなものにする傾向にある。そうすれば皿出しが早くなるし、結果
は完璧だ。料理というのは熱々の状態で供されてこそ、完璧な状態で味わって
いただけるものだ。ガルニチュールがごくシンプルなものなら、素早い盛り付
けにも対応出来る。

同様に、もし可能なら、ガルニチュールを料理の周囲に配置するよりは、別添
で供したほうがいいだろう\footnote{大皿に約10人分をまとめて盛り付けるケースを想定して言っていることに留意。}。そうすればどんな料理であっても、本体は
事前に切り分けて、ソースにまみれていない状態で盛り付けられた姿を、お客
様方にご覧いただくことが可能だ。それからすぐにガルニチュールとソースを
回していけばいい。この方式以外に、盛り付けを素早くおこない、清潔で熱々
の状態で料理をご提供する手段はなかろう。

これはとりわけ、ルルヴェ\footnote{\protect\hyperlink{releve}{第二版序文訳注}および本章「\protect\hyperlink{farces}{ファルス}」訳注参照。}と呼ばれる大掛かりな仕立ての料理の場合に
あてはまることだ。ノワゼット\footnote{約80gの牛フィレ肉の筒切り、および、円筒形に切った羊、仔羊の背肉の中心部分。}やトゥルヌドのようなさして大規模では
ない仕立てのアントレ\footnote{Entrée
  現代フランス語では「前菜」のことを指すが、かつては約10人
  分を大きな皿にまとめて盛り付け、給仕の際に取り分ける肉料理(さらに
  古くは魚料理も)を意味していた。}と呼ばれる料理については、給仕の際に切り分け
てガルニチュールを盛り付けてからお客様にお出しするよりは、おひとり様分
ずつ盛り付けて供することにすれば、「アントレ」の存在理由はますます低い
ものとなる。

それでも、アントレについてはそうしたほうがいい。この問題に関しては、料
理本体の盛り付けとガルニチュールを切り離したほうが、毎回確実により早く
料理をご提供できるのだから、どんな盛り付けの料理だろうと、ぜひためらうこと
なくこの方式を採用していただきたい。

\hypertarget{remarque-importante-sur-les-sauces-applicables-aux-entrees-de-boucherie-garnie-de-legumes}{%
\subsection{牛、羊肉料理に野菜を添える場合にふさわしいソースについて}\label{remarque-importante-sur-les-sauces-applicables-aux-entrees-de-boucherie-garnie-de-legumes}}

\vspace{-1\zw}
\begin{center}
\textit{Remarque importante sur les sauces applicables aux Entrées de Boucherie garnies de Légumes.}
\end{center}

\index{garniture@garniture!remarque sauce entrees boucherie legumes@Remarque importante sur les sauces applicables aux Entrées de Boucherie garnies de Légumes}
\index{かるにちゆーる@ガルニチュール!うしひつしにくりようりにやさいをそえるはあいにふさわしいそーす@牛、羊肉料理に野菜を添える場合にふさわしいソースについて}

\vspace{1\zw}

\protect\hyperlink{sauce-espagnole}{エスパニョル}系の派生ソースは野菜を添えた牛、羊肉料
理にはふさわしくない。\protect\hyperlink{jus-de-veau-lie}{とろみを付けたジュ}のほうが圧
倒的にいい。

だが、いちばんいいソースは、軽く仕上げた\protect\hyperlink{glace-de-viande}{グラスドヴィアン
ド}1 dlに125 gのバターを加えて\footnote{ソースを仕上げる際にバターを加えてより滑らかで艶やかな仕上がり
  にすることを monter au beurre (モンテオブール)日本ではブールモン
  テとも呼ばれる。}、レモン果汁ほん
の数滴で仕上げたものだ。とはいえ、このバターを加えたグラスドヴィアンド
は野菜を包み込んでしまわない程度に充分に軽い仕上がりにすること。

アスパラガスの穂先とかプチポワ\footnote{petits pois
  いわゆるグリンピースのことだが、フランスではより若 どりの、直径7〜8
  mm程度のものが好まれる。グリンピース特有の青臭さ
  は少なく、甘みがあって美味しい。ごく新鮮なものは生食も可能。フレッ
  シュなものは茹でずにたっぷりのバターで、ごく弱火で数分間加熱すれば
  いい。ただし、冷凍品はしっかりと下茹でする必要があるので注意。}、アリコヴェール\footnote{haricots
  verts いわゆる、さやいんげん。これもフランスでは日本と
  違い、ごく細い若どりのもの(太さ8〜9mm)程度のものが好まれるが、品
  種によっては太くするものもある。}、マセドワーヌ \footnote{macédoine
  さいの目に切った蕪(navet ナヴェ)やアリコヴェール、
  プチポワ、にんじんなどを混ぜ合わせたもの。日本のマセドアンサラダの
  原型となった。}などの野菜は、ソースをある意味、分解してしまう。それは野菜そ
のものが持つ水分によってだったり、野菜をあえているアパレイユのせいだっ
たりする。

その結果、大皿から取り皿に分けてお客様のところに運ばれた時には、ほとん
ど食欲を失なわせるような見た目になってしまう。こういう事態は\protect\hyperlink{sauce-chateaubriand}{ソース・
シャトーブリヤン}か、バターを加えたグラスドヴィ
アンドを料理に合わせれば解決する。これらのソースは分解しないどころか、
野菜のガルニチュールととてもよく合う。同時に、野菜のガルニチュールにも
これらのソースはとても素晴らしいふんわりとした食感を与えてくれるからだ。

そんなわけで、以下の点にぜひとも留意していただきたい。出来るだけ、エス
パニョル系の派生ソースやトマトソースは、\protect\hyperlink{garniture-financiere}{ガルニチュール・フィナンシエー
ル}や\protect\hyperlink{garniture-godard}{ゴダール}のような、ト
リュフ、雄鶏のとさかとロニョン、クネル、マッシュルームなどを添える料理
にとっておくべきだ。野菜のガルニチュールには、とろみを付けたジュ、もし
くはバターを加えたグラスドヴィアンドのほうがずっと好ましい。
\newpage



%%% Chapitre III. Potages
% エスコフィエ『料理の手引き』全注解
% 五島 学
%% III. potages
\href{原稿下準備20180414五島、連載からコピー}{} \href{訳と注釈}{}
\href{未、原文対照チェック}{} \href{未、日本語表現校正}{}
\href{未、その他修正}{} \href{未、原稿最終校正}{}

\hypertarget{potages}{%
\chapter{III. ポタージュ}\label{potages}}

\hypertarget{considerations-generales-potages}{%
\section{概説}\label{considerations-generales-potages}}

\frsec{Considérations Générales}

我々がポタージュと呼んでいる料理は、今ある形態としては比較的新しいもの
であり、せいぜい19世紀初頭までしか起源を遡ることができない。

それ以前のポタージュは一皿にすべて揃った料理だった\footnote{potage
  の語源は pot「壺、鍋」。古くは、鍋に肉や野菜を入れて煮込
  んだ料理(シヴェ、ラグー、ブルゥエなど)の総称であった。こんにちのよ
  うにポタージュが専ら液体料理の意味で用いられるようになったのは、エ
  スコフィエがここで述べているように、19世紀以降である。なお、日本語
  では一般的に「ポタージュ」という語はとろみがあるスープを意味し、
  「コンソメ」が澄んだスープを指すが、これは英語由来。}。こんにちではポ
タージュという名称は液体を指すだけだが、古くは、その液体を作るのに用い
た畜肉または家禽、ジビエ、魚そのものと、ガルニテュールである野菜も常に
ポタージュに含まれていたのだ。

いくつか挙げるなら、フランドルの「オシュポ\footnote{hochepotフランドルの地方料理としては、牛の尾を主素材としたポトフ
  の一種を指す。他に、豚の耳と尾、煮込み用牛肉と野菜の煮込みを意味す
  る場合もある。料理名自体は非常に古く、14世紀のタイユヴァン『ル・ヴィ
  アンディエ』には「鶏のオシュポ」hochepot de poullaille がある。}」、スペインの「オヤ
\footnote{olla (olla
  podrida)豚肉と各種内臓肉、豆、野菜の煮込み料理。南西 フランスのウイヤ
  ouillat (オイユ oilleとも)の原型になったと言われ
  ている。また、日本語の「おじや」の語源とする説もある。}」、そして我が国の「プティート・マルミート\footnote{petite
  marmite小ぶりの陶製の鍋に具材と汁を入れてオーヴンで熱して
  供するポトフに似た料理。牛肉、牛の尾、骨髄、鶏などをブイヨンで煮込
  む。19世紀パリのレストランで大流行した。澄んだポタージュ(コンソメ)
  に分類されるが、クラリフィエ(澄ませる作業)は行なわない。また、必ず
  鶏を用いるのが特徴。}」もまた、古くから残
されてきたポタージュの代表例と言える。もっとも、現代ではこれらの料理を
作る際に、多かれ少なかれ単純な構成にしているから、こう書いただけでは明
確なイメージが得られないかも知れない。

これらの料理にはごった煮のようなイメージがあるが、昔の献立というのはつ
まるところそういうものだったのだ。食べ進むにつれ食欲がだんだん満たされ
ていくのに合わせて、順序よく料理を進行させるようなことはしなかった\footnote{現代のように、食べる順に料理を提供する「ロシア式サーヴィス」が行
  なわれるようになったのは19世紀のこと。それ以前は大きな皿に盛られた
  複数の料理をまとめて食卓に供する「フランス式サーヴィス」であった。}。
昔の献立に非常に多くの種類のポタージュが見られるのは、適切に料理を配し
た結果というよりは、まさにそのこと自体が献立の特徴なのだと言える\footnote{典拠不明。確かに、17世紀以前の料理書ではポタージュに多くのページ
  を割いているものも少なくないが、17世紀にL.S.R.という筆名で出版され
  た『饗応術』や、同じく17世紀マシアロの著作にある献立例では、必ずし
  もポタージュの数が突出して多いとは言えない。大規模な宴席を除けば、
  通常の献立におけるポタージュは2〜3種であり、同時に供されるアントレ、
  オル・ドゥーヴルの方が種類が多い。また、『ル・ヴィアンディエ』巻末
  の献立例では料理が4回に分けて供されるが、1回目はポタージュ、2回目
  はロ(ロティ)があてられており、料理の種類はほぼ同数ずつになっている。}。

他の多くの調理技術についても同じだが、ポタージュに関してカレームの功績
は大きい。文字通りの意味では、カレームを現代料理の礎を築いた先駆者とは
呼べないが、少なくとも、新たな料理理論の普及に大いに貢献したのだ。

けれど、カレームの後継者たちがポタージュをこんにちの姿に完成させるまで、
1世紀近い時間を必要とした。

彼らは風味豊かで軽やかな、理想的なまでに繊細で美味な料理を創案したのだ
が、それらの新しいポタージュに正しい料理名をつけることにはあまり頓着し
なかったのだろう。とりわけ、とろみをつけたポタージュに関しては、しばし
ば同じルセットについて「ビスク」や「ピュレ」「クーリ」「ヴルテ」「クレー
ム」の名称をかまわずつけていた。理屈から言って、これらの名称はそれぞれ
完全に異なった料理を意味しなくてはいけないにもかかわらずである。その結
果として、残念なことに用語の混乱が起きてしまったわけだが、本書では、そ
れぞれのポタージュの特徴を明確にし、様々なルセットを合理的に分類して、
この問題を正してある。

これは以下のような考察に基づいている。すなわち、「ヴルテ」と「クレーム」
の語がポタージュについて用いられるようになったのは比較的近年のことであ
る。その理由としては、「ビスク」「クーリ」はやや古めかしい印象だし、
「ピュレ」はあまりに俗っぽく、しかも不適切だから、ビスク、クーリ、ピュ
レの代りにヴルテとクレームの語が用いられるようになったと思われる。

だから、ポタージュの各種別を明確に定義し、調理技術体系の欠落を埋める必
要があるのだ。

それぞれの種類のポタージュの特徴について次にまとめておいたが、本書で企
図したこの改革の意義がお解りいただけることだろう。

\hypertarget{classification-des-potages}{%
\section{ポタージュの種類}\label{classification-des-potages}}

\frsec{Classification des Potages}

料理を提供するという観点からは、ポタージュは大きく2つに分類される。
「澄んだポタージュ」と「とろみのあるポタージュ」である。

格式ある大がかりな献立では常に、それぞれの分類から1つ以上のポタージュ
が供される。通常の献立では、ポタージュを1つだけにする場合、献立全体の
流れに応じて、2つの分類のうちどちらか一方とする。

\hypertarget{classification-potages-claires}{%
\subsection{澄んだポタージュ}\label{classification-potages-claires}}

\frsecb{Les Potages claires}

澄んだポタージュ

澄んだポタージュは、主素材として畜肉、家禽、ジビエ、魚、甲殻類、海亀な
どのどれを用いたものでも、分類としてはただひとつとなる。つまり、澄んだ
コンソメ\footnote{consommé 語源は動詞
  consommer「完遂する、完成させる」。つまり
  「完全に仕上げたもの」の意。もともとは必ずしも液体料理を指す語では
  なかった。例えば18世紀マラン『食の贈り物』には以下のような「コンソ
  メ」が出ている。まず「最初のブイヨン」をとり、それを用いて「ブイヨ
  ン・ミトナージュ」をとり、ブイヨン・ミトナージュを用いて「ブイヨン・
  コルディアル」をとり、さらにブイヨン・コルディアルを用いて「コンソ
  メ」を作る。最終的にはとろみがつく程度に煮詰める。このコンソメは単
  体で料理として供するものではなく、調味料的にポタージュにこくを与え
  たり、ソースを作るのに用いると記されている。}である。これはタピオカでんぷんでごく軽くとろみをつける場合
もあるが、いずれにせよ、それぞれのポタージュの性格に合わせて浮き実は少
量とする。

\hypertarget{classification-potages-lies}{%
\subsection{とろみを付けたポタージュ}\label{classification-potages-lies}}

\frsecb{Les Potages Liés}

とろみのあるポタージュ

とろみのあるポタージュは5つに分類される。すなわち

\begin{enumerate}
\def\labelenumi{\arabic{enumi})}
\item
  ピュレ、クーリ、ビスク
\item
  クレーム
\item
  ヴルテ
\item
  とろみをつけたコンソメ
\item
  特殊な複合ポタージュ。上記のうち複数のポタージュの性格を持ち、ヴァ
  リエーションを展開出来ないもの。
\end{enumerate}

より単純な分類にするため、本書では、ジェルミニ\footnote{刻んだオゼイユの葉をバターで炒め溶かして裏漉しし、白いコンソメを
  加えて火にかける。供する直前に溶きほぐした卵黄と生クリームを加えて
  火にかけ、クレーム・アングレーズのようにする。とろみがついたら、バ
  ターを加えて仕上げ、セルフイユの葉を飾る(原書p159)。}のようなとろみをつけ
たコンソメは特殊なポタージュに含めてある。その理由は「特殊なポタージュ」
の節の冒頭で述べてある\footnote{原書 p.157}。

上記のポタージュのうち最初の3種はベースとして何らかのピュレを用いるが、
主に何によってとろみをつけるかで違ってくる。

「ピュレ」「クーリ」「ビスク」でとろみをつけるのは、主素材に応じて、米、
油で揚げたパン、じゃがいも、いんげん豆、レンズ豆などのでんぷん質の野菜。
これらのつなぎと主素材との分量比率はきちんと守らなければならないので、
「とろみをつけたポタージュ」の章の冒頭を参照のこと\footnote{原書 p.135}。

「クレーム」と「ヴルテ」のとろみは白いルウがベースとなってはいるが、実
際に用いるつなぎは違う\footnote{ポタージュ・ヴルテは基本ソースのヴルテをつなぎとして用いるのに
  対し、ポタージュ・クレームはベシャメル(原書pp.136-138)。}。また、仕上げ方も異なる。

ヴルテの仕上げは必ず卵黄とバターでつなぐ。クレームはとろみをつける要素
は足さず、バターではなく良質の生クリーム適量を加えて仕上げる。

つまり、異なるつなぎの要素を用いているわけだから、クレームとヴルテは明
確に違うものとして区別すべきなのだ。

ピュレ、クーリ、ビスクは、いずれも作り方がほぼ同じわけだが、にもかかわ
らず、これらの語は同義ではない。むしろ明らかに違う意味を持っていること
に注意。

慣習として「ピュレ」の名称は野菜をベースにしたものに用いるが、この名称
には俗っぽい印象があるため、避けられる傾向にある。

「クーリ」の名称は家禽、ジビエ、魚および甲殻類のピュレについて用いる。

だが、甲殻類のピュレについては、「ビスク」の名称を用いる方が多い。むし
ろ、ビスクと言えば甲殻類のピュレを指すことになっている。ただし、この語
が生まれた当時から18世紀末までは、家禽と鳩を主素材にしたポタージュのこ
とであった。

とろみのあるポタージュの多くは、主素材そのままで単に調理法を変えれば、
ピュレ、クレームおよびヴルテとして展開出来るのだが、これについては後で
記すことにする\footnote{原書 pp.134-138}。

\hypertarget{consideration-sur-le-service-des-potages}{%
\subsection{ポタージュを供するにあたっての注意}\label{consideration-sur-le-service-des-potages}}

\vspace{-1\zw} \begin{center}
\hspace{1\zw}\large\textit{Considérations sur le Service des Potages}
\normalsize \end{center}
\newpage
\href{原稿下準備20180414五島、連載からコピー}{} \href{訳と注釈}{}
\href{未、原文対照チェック}{} \href{未、日本語表現校正}{}
\href{未、その他修正}{} \href{未、原稿最終校正}{}

\hypertarget{ux30ddux30bfux30fcux30b8ux30e5ux306eux57faux790e}{%
\section{ポタージュの基礎}\label{ux30ddux30bfux30fcux30b8ux30e5ux306eux57faux790e}}

\vspace{-.5\zw}
\begin{center}
\setstretch{0.5}
\headfont\medlarge プチットマルミット、グランドブイヨン、\\コンソメのクラリフィエ \normalfont\normalsize
\end{center}
\setstretch{1.0}
\vspace{2ex}
\frsec{Précis des éléments nurtirifs, aromatiques}
\vspace{-1ex}
\begin{center}
\setstretch{0.5}
\headfont et de l'assaisonnement pour la Petite Marmite,\\ les Grands Bouillons,
et la clarification des Consommé divers

\end{center}
\normalfont
\setstretch{1.0}

\hypertarget{ux767dux3044ux30b3ux30f3ux30bdux30e1ux30b5ux30f3ux30d7ux30eb1}{%
\subsubsection[白いコンソメ・サンプル]{\texorpdfstring{白いコンソメ・サンプル\footnote{consommé
  simple「単純な(簡素な)コンソメ」の意。肉や魚、野菜を煮
  て漉しただけのもの。ここでは具体的な作業手順は記されていないが、モ
  ンタニェの『ラルース・ガストロノミーク』初版(1937年)の記述は概ね以
  下のとおり(材料はエスコフィエとほぼ同じ)。(a) 牛肉を紐で縛り、大鍋
  (陶製が良い)に入れて水7 Lを注ぐ。火にかけて沸騰したら、表面にアルブ
  ミンの軽く固まった膜が張るので、丁寧にこの膜を取り除く。鍋に野菜を
  加える。かすかに沸騰する火加減で5時間煮る。浮き脂を丁寧に取り除き、
  布または目の細かい漉し器で漉す。5時間以上煮込んではいけない。だが、
  5時間では骨に含まれているおいしさを全て抽出出来ないので、砕いた骨
  を長時間煮て第1のブイヨンをとり、これで肉と野菜を煮るようにすると
  良い。(b) 鍋に砕いた骨を入れ、水をかぶる程度注ぐ。沸騰させ、あくを
  引き、塩を加える。弱火で2時間半煮る。この「沸騰したブイヨン」に、
  骨を外して紐で縛った肉を入れる。再び沸騰させ、あくを引いて味を調え
  る。野菜を加え、弱火で約4時間煮る。塩は最初に全量を入れないこと。
  必要なら作業の最終段階でも塩を加える。}}{白いコンソメ・サンプル}}\label{ux767dux3044ux30b3ux30f3ux30bdux30e1ux30b5ux30f3ux30d7ux30eb1}}

(仕上がり 10 L分)

\begin{itemize}
\item
  主素材\ldots{}\ldots{}牛赤身肉4 kgと牛骨付きすね肉3 kg。
\item
  香味素材\ldots{}\ldots{}にんじん1.1 kg (5〜6本)、かぶ900
  g(5〜6ヶ)、ポワロー200 g、パース ニップ\footnote{panaisパネ。和名アメリカボウフウ。セリ科の根菜で、香りが良い。
    白く、にんじんに似た円錐形のため、俗に「白にんじん」と呼ばれること
    もあるが、にんじんとは別種。でんぷん質が豊富で、ピュレ等の調理にも
    適している。}200 g、玉ねぎ(中)2ヶ(200
  g)、クローブ3本、にんにく3片(20 g)、 セロリ120 g。
\item
  加える液体\ldots{}\ldots{}水14 L。
\item
  調味料\ldots{}\ldots{}粗塩70 g。
\item
  加熱時間\ldots{}\ldots{}5時間。
\end{itemize}

\hypertarget{ux4f5cux308aux65b9ux306bux95a2ux3059ux308bux88dcux8db33}{%
\paragraph[作り方に関する補足]{\texorpdfstring{作り方に関する補足\footnote{この部分は第二版で加筆された。}}{作り方に関する補足}}\label{ux4f5cux308aux65b9ux306bux95a2ux3059ux308bux88dcux8db33}}

\ldots{}\ldots{}コンソメ・サンプルを作る際、一般的には5時間かけて煮ることになっている。
肉汁を抽出するには充分な時間である。

しかし、骨の組織を壊して可溶性物質を確実に抽出するには5時間では絶対に
足りない。骨から可溶性物質を抽出することはとても重要だが、そのためには
弱火で12〜15時間煮る必要がある。

だから、グランド・キュイジーヌでは、粗く砕いた骨を12時間以上煮て第1の
コンソメをとるようになってきている。

この第1のコンソメを第2のコンソメをとる鍋に注ぐ。この鍋で肉を約4時間、
すなわち肉を煮るのに最低限必要な時間、火にかける。

肉を野菜を塊のままではなく細かく刻めば、2つめの作業時間をさらに短かく
することも可能だ。その場合は、通常のクラリフィエと同様の作業となる。
(「クラリフィエ」の項参照)。

\hypertarget{ux30afux30e9ux30eaux30d5ux30a3ux30a84}{%
\subsection[クラリフィエ]{\texorpdfstring{クラリフィエ\footnote{原文
  clarifications クラリフィカシオン (動詞 clarifier「澄ませ
  る」の名詞形)。字義通りには「澄ませる作業」だが、実際にはコンソメ・
  ドゥーブルconsommé double (コンソメ・リッシュ consommé richeコンソ
  メ・クラリフィエ consommé clarifiéとも呼ばれる)を作ることを意味す
  る。本来はその工程のひとつであった「澄ませる作業」が作業全体を指す
  語として定着したのだろう。}}{クラリフィエ}}\label{ux30afux30e9ux30eaux30d5ux30a3ux30a84}}

\frsecb{Clarifications}

\hypertarget{ux901aux5e38ux306eux30b3ux30f3ux30bdux30e1}{%
\subsubsection{通常のコンソメ}\label{ux901aux5e38ux306eux30b3ux30f3ux30bdux30e1}}

(仕上がり4 L分)

\begin{itemize}
\item
  白いコンソメ・サンプル\ldots{}\ldots{}5 L。
\item
  主素材\ldots{}\ldots{}牛赤身肉1.5
  kg。丁寧に筋を除き、挽いておく\footnote{原文 hacher
    アシェ(細かく刻む)。語源は hacheアーシュ(斧)。日本
    語の「刻む」は包丁を用い、「挽く」はミートチョッパーのような器具を
    用いる場合を指すが、フランス語では区別せずどちらもhacher と表現す
    る。}。
\item
  香味素材\ldots{}\ldots{}にんじん100 g、ポワロー200
  g。小さなさいの目\footnote{brunoise ブリュノワーズ}に刻んでおく。
\item
  澄ませるための素材\ldots{}\ldots{}卵白2ヶ分。
\item
  所要時間\ldots{}\ldots{}1時間半。
\item
  作業\ldots{}\ldots{}片手鍋\footnote{casserole カスロール}または小ぶりの寸胴鍋\footnote{marmiteマルミート。一般的には、大型で深さが直径以上ある両手鍋を
    指す。}に牛挽肉、小さなさいの目に刻
  んだ野菜、卵白を入れ、全体をよく混ぜる。白いコンソメ・サンプルを注ぎ入
  れ、時々混ぜながら\footnote{ここは原文に忠実に訳したが、実際には常に混ぜ続けないと卵白が鍋
    底にくっついて無駄になってしまう。『ラルース・ガストロノミック』初
    版では「絶えず混ぜる」ように指示されている。}沸騰させる。軽く沸騰させながら1時間半煮る。
\end{itemize}

布で漉して仕上げる。

\hypertarget{ux9d8fux306eux30b3ux30f3ux30bdux30e1}{%
\subsubsection{鶏のコンソメ}\label{ux9d8fux306eux30b3ux30f3ux30bdux30e1}}

(仕上がり4 L分)**

*白いコンソメ・サンプル\ldots{}\ldots{}同上。

\begin{itemize}
\item
  主素材と香味素材\ldots{}\ldots{}同上に、以下を加える。オーヴンで軽く色づけた鶏1羽。
  鶏の首づる、手羽先、足など\footnote{原文
    abatisアバティ。鶏肉として食べられる以外の部位の総称。
    鶏の「内臓」と訳されることが多いが、とさか、頭、首づる、手羽先、足
    なども含まれる。}を刻んだもの6羽分。ロティールした鶏
  のがら\footnote{鶏のロティ(ローストチキン)を提供した際に出る「がら」。}2羽分。
\item
  澄ませるための素材、方法、時間は通常のコンソメと同様にする。
\end{itemize}
\newpage
\href{原稿下準備20180414五島、連載からコピー}{} \href{訳と注釈}{}
\href{未、原文対照チェック}{} \href{未、日本語表現校正}{}
\href{未、その他修正}{} \href{未、原稿最終校正}{}

\hypertarget{ux30ddux30bfux30fcux30b8ux30e5ux306eux4e3bux306aux6d6eux304dux5b9fux30acux30ebux30cbux30c1ux30e5ux30fcux30eb}{%
\section{ポタージュの主な浮き実(ガルニチュール)}\label{ux30ddux30bfux30fcux30b8ux30e5ux306eux4e3bux306aux6d6eux304dux5b9fux30acux30ebux30cbux30c1ux30e5ux30fcux30eb}}

\frsec{Elémtent divers de Garnitures pour Potages}
\newpage
%\input{03-potages/03-04-pp118-132}%newpage
%\input{03-potages/03-05-pp132-134}%newpage
\href{原稿下準備20180414五島、連載からコピー}{} \href{訳と注釈}{}
\href{未、原文対照チェック}{} \href{未、日本語表現校正}{}
\href{未、その他修正}{} \href{未、原稿最終校正}{}

\hypertarget{ux3068ux308dux307fux3092ux4ed8ux3051ux305fux30ddux30bfux30fcux30b8ux30e5}{%
\section{とろみを付けたポタージュ}\label{ux3068ux308dux307fux3092ux4ed8ux3051ux305fux30ddux30bfux30fcux30b8ux30e5}}

\frsec{Potages Liés}

\hypertarget{ux30ddux30bfux30fcux30b8ux30e5ux30d4ux30e5ux30ec}{%
\subsection{ポタージュ・ピュレ}\label{ux30ddux30bfux30fcux30b8ux30e5ux30d4ux30e5ux30ec}}

\frsecb{les Purées}

主素材とつなぎ:ポタージュ・ピュレの主素材として用いるのは次のとおり。
1種類または数種を組み合わせた野菜、鶏、ジビエ、甲殻類。

ほぼ全てのポタージュ・ピュレにはつなぎを加える。すなわち、

米\ldots{}\ldots{}鶏、甲殻類のポタージュ・ピュレおよび野菜のポタージュ・ピュレ
のいくつか。

じゃがいも\ldots{}\ldots{}香草や、かぼちゃのように水分の多い野菜のポタージュ・
ピュレ。

レンズ豆\ldots{}\ldots{}ジビエのポタージュ・ピュレ。

バターで揚げたクルトン\ldots{}\ldots{}クラシックなポタージュ・ピュレ。

昔の料理では、他にもつなぎに用いるものはあったが、とりわけクーリとビス
ク\footnote{「昔の料理」におけるビスクは甲殻類のポタージュ・ピュレのことで
  はなく、鳩などの煮込み料理のこと(本連載「ポタージュ(1)」2012年6月
  号p.115 参照)。}には、クルトンが主に用いられていた\footnote{中世〜18世紀には、とろみをつけるために、硬くなったパンを加えて
  弱火で煮込む(mitonnerミトネ)ことが一般的だった。}。とてもまろやかな仕上り
になるので、現代でもこの手法を用いる価値はある。

いんげん豆やレンズ豆、じゃがいものようなでんぷん質の素材のポタージュ・
ピュレにはつなぎを加える必要はない。主素材である野菜それ自体がつなぎと
なるからだ。

加える液体とつなぎの分量:ポタージュ・ピュレに加える液体は、主素材の種
類に応じて、白いコンソメ、ジビエのコンソメ、魚のコンソメを用いる。野菜
のポタージュ・ピュレでは牛乳を用いる場合もある。

加える液体\ldots{}\ldots{}ベースとなるピュレ1 Lに対して2 L。

つなぎ\ldots{}\ldots{}

\begin{enumerate}
\def\labelenumi{\arabic{enumi}.}
\item
  米\ldots{}\ldots{}野菜500 gあたり85〜120 g。鶏、ジビエ、甲殻類の身500
  gあ たり75〜100 g。
\item
  レンズ豆\ldots{}\ldots{}ジビエの肉500 gあたり190 g。
\item
  じゃがいも\ldots{}\ldots{}香草と野菜500 gあたり250 g。
\item
  バターで揚げたクルトン\ldots{}\ldots{}野菜または甲殻類の身500
  gあたり270 g。
\end{enumerate}

作業と仕上げ:野菜は次のいずれかの方法で処理する。(a)薄切りにした野菜
600〜700 gあたり80〜100 gのバターでエテュヴェする。(b)薄切りにした野菜を
湯通し\footnote{原文 blanchir
  (ブランシール)。下茹でする、湯がくこと。野菜類の
  ブランシールは塩を加えた湯で行なうが、素材の性質により2種に分けら
  れる。ひとつは大量の湯で素材に完全に火が通るまで茹でること。もうひ
  とつは刳味(アク)を除くための下茹で、湯通し(原書p.726)。ここでは後 者。}してからバターでエテュヴェする。どちらの方法を用いるかは、
本書では個々のルセットに記してある。

ジビエはサルミを調理する際と同様にロティール\footnote{鶏、猟鳥の胸肉の部分を豚背脂のシートで包んでセニャンにロティー
  ルする。なお、「サルミ」は古くは「猟鳥肉の煮込み」の意であった。
  『ル・ギード・キュリネール』では、ロティールした猟鳥の肉を切り分け
  て保温し、摺り潰したガラと端肉を煮込んで作ったソースと合わせる(本
  連載「雉のサルミ」2011年11月号pp.128-129 参照)。}してから、レンズ豆と
ともに煮る。火が通ったら骨を外す。肉とレンズ豆を摺り潰し、布漉しした後、
濃さを調節する。

鶏は白いコンソメでポシェする。つなぎに用いる米も一緒に煮る。火が通った
ら骨を外し、その後はジビエのピュレと同様にする。鶏およびジビエにちょう
ど火が通ったところで、浮き実にする分の胸肉は別にとっておくこと。

野菜のポタージュ・ピュレは、濃さを調節したらデプイエ、つまり微沸騰の状
態で25〜30分間かけて不純物を取り除く。

このデプイエの作業の際、時折、冷たいコンソメを若干量加えるとよい。ピュ
レの中に紛れている不純物が表面に浮かび上がって、取り除きやすくなる。

鶏、ジビエ、甲殻類のピュレは沸騰したら湯煎にかける。デプイエする必要は
ない。

どのポタージュ・ピュレも、仕上げにバターを加える直前に、目の細かいシノ
ワで漉すこと。

仕上げは提供直前に行なう。火から外し、ポタージュ1ℓあたり80〜100
gのバター を加える。

つなぎに白いんげん豆、じゃがいも、米などのような白いでんぷん質やクルト
ンを用いるポタージュは、さらにつなぎとして卵黄を補ってもよい。

バターを加えたら、再沸騰させてはいけないと肝に銘じること。沸騰するとバ
ターの風味が失なわれてしまう。ポタージュにおいて、バターの風味は明瞭で
フレッシュでなくてはいけない。

(略)

ピュレの展開\footnote{本連載「ポタージュ(1)」2012年6月号 p.115 参照。}:以下に記す方法で、ピュレの多くはポタージュ・ヴルテ、
ポタージュ・クレームにすることが出来る。ポタージュ・ピュレに用いるつな
ぎの代わりに、鶏または魚のヴルテや薄いソース・ベシャメルを主素材に加え
るのだ。

ただし、素材によっては、ポタージュ・ピュレ以外の仕立てに出来ないものも
ある。

\hypertarget{ux30ddux30bfux30fcux30b8ux30e5ux30f4ux30ebux30c6}{%
\subsection{ポタージュ・ヴルテ}\label{ux30ddux30bfux30fcux30b8ux30e5ux30f4ux30ebux30c6}}

\frsecb{les Veloutés}

ベースとなるヴルテ\footnote{基本ソースとしてのヴルテ(原書
  p.15)がベースとなる。}:

\begin{enumerate}
\def\labelenumi{\arabic{enumi}.}
\item
  野菜のポタージュ・ヴルテの場合は、やや薄い通常のヴルテ。
\item
  鶏や魚のポタージュ・ヴルテの場合は、それぞれ対応するヴルテ。
\end{enumerate}

ポタージュのベースにするヴルテは、主素材となる野菜、鶏、ジビエおよび魚
に応じて、通常の白いコンソメ、鶏のコンソメ、ジビエのコンソメ、魚のコン
ソメ1ℓあたり白いルゥ100 gを加えて作る。

材料比率:この方法で作るポタージュは全て、次の分量配分となる。

・ベースとなるヴルテはポタージュ全体の半量。

・ポタージュの性格を決めるピュレは全体の\textbf{1/4}。

・濃さを整えるのに加えるコンソメも\textbf{1/4}。ただし、つなぎとして加える
生クリームの分量もこれに含める。

例えば、仕上がり2ℓの「ポタージュ・ヴルテ王妃風\footnote{à la reine
  (ア・ラ・レーヌ)優美で繊細な料理に用いる表現。この名
  称の料理には鶏を素材としたものが多い。「ポタージュ・ピュレ王妃風」
  のルセットは原書p.146。}」の場合には分量は 次のようになる。

\begin{quote}
鶏のヴルテ\textbf{1}ℓ。鶏のピュレ\textbf{5dl}。仕上げに加える白いコンソメ
\textbf{3dl}。つなぎ\textbf{(}生クリーム\textbf{)2dl}。計\textbf{2}ℓ。
\end{quote}

作業:

\begin{enumerate}
\def\labelenumi{(\arabic{enumi})}
\item
  主素材が鶏や魚の場合は、予め骨を外してからベースとなるヴルテで素材
  を煮る。次に、肉を取り出して摺り潰し、肉を煮たヴルテでのばしてから布漉
  しする。このピュレにコンソメを加えて濃さを整える。
\item
  野菜の場合は、素材の性質に応じて、湯通ししたものをバターでエテュヴェ
  するか、生の野菜をバターでエテュヴェしてから、ベースとなるヴルテに加え
  る。野菜に火が通った後は上記と同様にする。
\item
  甲殻類の場合は、通常どおりミルポワを用いて火を通し、細かく摺り潰し
  てからベースとなるヴルテに加えて煮、布漉しする。
\end{enumerate}

つなぎと仕上げ:ポタージュ・ヴルテのつなぎには、仕上り1ℓあたり卵黄3ヶ
と生クリーム1dlを加える。

提供直前に、鍋を火から外して、1ℓあたりバター80〜100 gを加えて仕上げる。
(略)

\hypertarget{ux30ddux30bfux30fcux30b8ux30e5ux30afux30ecux30fcux30e0}{%
\subsection{ポタージュ・クレーム}\label{ux30ddux30bfux30fcux30b8ux30e5ux30afux30ecux30fcux30e0}}

\frsecb{les Crèmes}

ポタージュ・クレームの作り方はポタージュ・ヴルテと同じだが、以下の点が
違う。

\begin{enumerate}
\def\labelenumi{(\arabic{enumi})}
\item
  ヴルテではなく薄いソース・ベシャメルをベースとして用いる。牛乳1ℓあ
  たり白いルゥ100 gで作る。
\item
  多くの場合、仕上げに濃さを調節する際、コンソメではなく牛乳を加える。
\end{enumerate}

材料比率:ポタージュ・ヴルテと同様。つまり、ベシャメルはポタージュ全体
の半量、ポタージュの性格を決めるピュレが1/4、濃さを整えるための白いコ
ンソメまたは牛乳が1/4(仕上げに加える生クリームもこれに含める)。

作業:主素材が鶏、ジビエ、野菜、甲殻類いずれの場合も、作業はポタージュ・
ヴルテの項で示したのと同じ。(略)

仕上げ:提供直前に、ポタージュ1ℓあたり2dlの生クリームを加える。

原注:ポタージュ・ヴルテもポタージュ・クレームもデプイエは行なわない。
ポタージュの濃さを整えたら、沸騰寸前まで温め、湯煎にかけて保温しておく。
表面が乾かないようバターのかけら数片を載せる。ポタージュ・ヴルテは、供
する前に卵黄、生クリーム、バターを加えて仕上げる。ポタージュ・クレーム
の仕上げは供する前に生クリームだけを加える。
%newpage
%\input{03-potages/03-07-pp139-146}
%\input{03-potages/03-08-pp146-150}
%\input{03-potages/03-09-pp151-156}
%\input{03-potages/03-10-pp157-162}
%\input{03-potages/03-11-pp163-169}
%\input{03-potages/03-12-pp170-171}
%\input{03-potages/03-13-pp172-185}



\backmatter
%%%索引ページ出力
{\printindex}

%%% Important! 文書終了%
\end{document}
