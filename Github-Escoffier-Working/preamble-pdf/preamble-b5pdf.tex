\usepackage{okumacro}
\usepackage[utf8]{inputenc}
\usepackage{multicol}
\usepackage[expert,deluxe]{otf}
\usepackage[T1]{fontenc}
\usepackage{textcomp}
\usepackage{palatino}
\usepackage{xfrac}
\usepackage[shortlabels]{enumitem}
\usepackage{makeidx}
\usepackage{setspace}
\usepackage{layout}
\usepackage[stable]{footmisc}
\usepackage[dvipdfmx,bookmarks=true,bookmarksnumbered=true,colorlinks=true,linkcolor=red,setpagesize=false,pdftitle={オギュスト・エスコフィエ「ソース」『ル・ギード・キュリネール』第1章},pdfauthor={五島 学 訳},pdfsubject={フランス料理古典叢書},pdfkeywords={エスコフィエ;ル・ギード・キュリネール}]{hyperref}
\usepackage{pxjahyper}
\usepackage{graphicx}
\usepackage{remreset}
\usepackage{tabularx}
\makeindex

%%%%%%%%%%%%%%%%%%%
%\hyperrefマクロを無効化
%
%\renewcommand{\hyperref}[2][]{#2}%パッケージが有効の場合
%
%\newcommand{\hyperref}[2][]{#2}%上でhyperrefパッケージを無効化している場合
%
%%%%%%%%%%%%%%%%%%%

%デフォルトフォントをゴチにする
%\renewcommand{\kanjifamilydefault}{\gtdefault}

%行取りマクロ\linespace{行数}{内容}
\makeatletter
\ifx\Cht\undefined
 \newdimen\Cht\newdimen\Cdp
 \setbox0\hbox{\char\jis"2121}\Cht=\ht0\Cdp=\dp0\fi
\catcode`@=11
\long\def\linespace#1#2{\par\noindent
  \dimen@=\baselineskip\multiply\dimen@ #1\advance\dimen@-\baselineskip
  \advance\dimen@-\Cht\advance\dimen@\Cdp
  \setbox0\vbox{\noindent #2}\advance\dimen@\ht0\advance\dimen@-\dp0%
  \vtop to\z@{\hbox{\vrule width\z@ height\Cht depth\z@
   \raise-.5\dimen@\hbox{\box0}}\vss}%
  \dimen@=\baselineskip\multiply\dimen@ #1\advance\dimen@-2\baselineskip
  \par\nobreak\vskip\dimen@\hbox{\vrule width\z@ height\Cht depth\z@}\vskip\z@}
\catcode`@=12
\makeatother

%\setlength{\fullwidth}{41zw}
\setlength{\textwidth}{\fullwidth}
\setlength{\evensidemargin}{\oddsidemargin}
%\makeatletter
%\def\@evenhead{%
%  \if@mparswitch \hss \fi
%  \underline{\hbox to \z@{\textit{\thepage}\hss}%
%    \hbox to \fullwidth{\hss\autoxspacing\leftmark\hss}}%
%  \if@mparswitch\else \hss \fi}%
%\def\@oddhead{\underline{\hbox to \fullwidth{\hss
%      {\autoxspacing\if@twoside\rightmark\else\leftmark\fi}\hss}%
%    \hbox to \z@{\hss\textit{\thepage}}}\hss}%
%\makeatother    
%\setlength{\textwidth}{38zw}
%\setlength{\oddsidemargin}{0pt}
%\setlength{\evensidemargin}{-35pt}
%\setlength{\topmargin}{-6zw}
%\setlength{\textheight}{58zw}
\setlength{\marginparwidth}{0cm}
\setlength{\marginparsep}{0cm}
%\hoffset=-2zw% 左へ 0.5インチ版面全体を移動
%\voffset=-3zw
%\renewcommand{\prechaptername}{}
%\renewcommand{\thechapter}{}
%\renewcommand{\postchaptername}{}
%\setlength{\footskip}{20zw}
%\setlength{\columnsep}{2zw}
\setlength{\columnseprule}{0pt}
%\setlength{\parskip}{0pt}

\setlength{\parskip}{0pt}
%\setlength{\topskip}{\Cht}
%\setlength{\textheight}{39\baselineskip}
%\addtolength{\textheight}{1zh}
%\setlength{\parindent}{0pt}
%\setlength{\parskip}{1ex plus 0.5ex minus 0.2ex}
%\setlength{\itemindent}{-14pt}
%\setlength{\leftmargin}{0zw}

%リスト環境
\makeatletter
  \parsep   = 0pt
  \labelsep = 1zw
  \def\@listi{%
     \leftmargin = 1zw \rightmargin = 0pt
     \labelwidth\leftmargin \advance\labelwidth-\labelsep
     \topsep     = 0pt%\baselineskip
     \topsep -0.1\baselineskip \@plus 0\baselineskip \@minus 0.1 \baselineskip
     \partopsep  = 0pt \itemsep       = 0pt
     \itemindent = 0pt \listparindent = 1zw}
  \let\@listI\@listi
  \@listi
  \def\@listii{%
     \leftmargin = 2zw \rightmargin = 0pt
     \labelwidth\leftmargin \advance\labelwidth-\labelsep
     \topsep     = 0pt \partopsep     = 0pt \itemsep   = 0pt
     \itemindent = 0pt \listparindent = 1zw}
  \let\@listiii\@listii
  \let\@listiv\@listii
  \let\@listv\@listii
  \let\@listvi\@listii
\makeatother

\makeatletter%% プリアンブルで定義する場合は必須

\renewenvironment{theindex}{% 索引を3段組で出力する環境
    \if@twocolumn
      \onecolumn\@restonecolfalse
    \else
      \clearpage\@restonecoltrue
    \fi
    \columnseprule.4pt \columnsep 2zw
    \ifx\multicols\@undefined
      \twocolumn[\@makeschapterhead{\indexname}%
      \addcontentsline{toc}{chapter}{\indexname}]%変更点
    \else
      \ifdim\textwidth<\fullwidth
        \setlength{\evensidemargin}{\oddsidemargin}
        \setlength{\textwidth}{\fullwidth}
        \setlength{\linewidth}{\fullwidth}
        \begin{multicols}{3}[\chapter*{\indexname}
	\addcontentsline{toc}{chapter}{\indexname}]%変更点%
      \else
        \begin{multicols}{3}[\chapter*{\indexname}
	\addcontentsline{toc}{chapter}{\indexname}]%変更点%
      \fi
    \fi
    \@mkboth{\indexname}{\indexname}%
    \plainifnotempty % \thispagestyle{plain}
    \parindent\z@
    \parskip\z@ \@plus .3\p@\relax
    \let\item\@idxitem
    \raggedright
    \footnotesize\narrowbaselines
  }{
    \ifx\multicols\@undefined
      \if@restonecol\onecolumn\fi
    \else
      \end{multicols}
    \fi
    \clearpage
  }

\def\@makechapterhead#1{\hbox{}%
  \vskip2\Cvs
  {\parindent\z@
%  \raggedright% オリジナルの定義(左揃え)
   \centering% 中央揃え
%  \raggedleft% 右揃え
   \reset@font\huge\sffamily
   \ifnum \c@secnumdepth >\m@ne
     \setlength\@tempdima{\linewidth}%
     \vtop{\hsize\@tempdima%
       \if@mainmatter% ← report クラスの場合この行不要
          \@chapapp\thechapter\@chappos\\%
       \fi% ← report クラスの場合この行不要
     #1}%
   \else
     #1\relax
   \fi}\nobreak\vskip3\Cvs}

%\def\@makeschapterhead#1{\hbox{}%
%  \vskip2\Cvs
%  {\parindent\z@
%%  \raggedright% オリジナルの定義(左揃え)
% % \centering% 中央揃え
%   \raggedleft% 右揃え
%   \reset@font\huge\sffamily
%   \setlength\@tempdima{\linewidth}%
%   \vtop{\hsize\@tempdima#1}}\vskip3\Cvs}

% \def\sectionmark##1{\markright{%
 %   \ifnum \c@secnumdepth >\z@ \thesection \hskip1zw\fi
 %   ##1}}%

\renewcommand{\sectionmark}[1]{\markright{#1}{}} 
%\renewcommand{\subsectionmark}[1]{\markright{#1}{}}

%\renewcommand{\section}{%
%    \@startsection{section}{1}{\z@}%
%    {0.6\Cvs}{0.4\Cvs}%
%    {\normalfont\LARGE\headfont\centering}}

\renewcommand{\thesection}{}


%subsectionに連番をつける
\@removefromreset{subsection}{section}
\def\thesubsection{\arabic{subsection}.}
%\@removefromreset{footnote}{chapter}

\newcounter{rnumber}
\renewcommand{\thernumber}{\refstepcounter{rnumber} }
%\newcommand{\rnumber}{\refstepcounter{rnumber}\thernumber. }
%\renewcommand{\thesubsection}{\thernumber. }
%\newcommand{\rnumber}{}
%\newcounter{recetteno}
%\renewcommand{\therecetteno}{\arabic{recetteno}}
%\renewcommand{\thesubsection}{\refstepcounter{recetteno}\therecetteno. }
%\@addtoreset{subsection}{chapter}

\renewcommand{\section}{%
  \@startsection{section}% #1 見出し
   {2}% #2 見出しのレベル
   {-1\Cvs}% #3 横組みの場合,見出し左の空き(インデント量)
   {1.5\Cvs \@plus.5\Cdp \@minus.2\Cdp}% #4 見出し上の空き
   {1.5\Cvs \@plus.5\Cdp}% #5 見出し下の空き (負の値なら見出し後の空き)
      %{.5\Cvs \@plus.3\Cdp}% #5 見出し下の空き (負の値なら見出し後の空き)
%  {\reset@font\Large\bfseries}% #6 見出しの属性
  % {\raggedright\reset@font\LARGE\sffamily}%
   {\centering\reset@font\LARGE\sffamily}% 中央揃え
%  {\raggedleft\reset@font\Large\bfseries}% 右揃え
}%

\renewcommand{\subsection}{%
  \@startsection{subsection}% #1 見出し
   {2}% #2 見出しのレベル
   {0\Cvs}% #3 横組みの場合,見出し左の空き(インデント量)
   {0\Cvs \@plus0\Cdp \@minus0\Cdp}% #4 見出し上の空き
   {0\Cvs}%{0\Cvs \@plus0\Cdp \@minus0\Cdp}% #5 見出し下の空き (負の値なら見出し後の空き) 
   {\reset@font\fontsize{9.2482pt}{12.0226pt}\sffamily\bfseries}% #6 見出しの属性
%   {\centering\reset@font\large\sffamily}% 中央揃え
%   {\raggedright\reset@font\normalsize\bfseries}% 右揃え
}%

\renewcommand{\subsubsection}{%
  \@startsection{subsubsection}% #1 見出し
   {4}% #2 見出しのレベル
   {0\Cvs}% #3 横組みの場合,見出し左の空き(インデント量)
   {0\Cvs}%{0\Cvs \@plus0\Cdp \@minus0\Cdp}%{0\Cvs \@plus0\Cdp \@minus5\Cdp}% #4 見出し上の空き
   {0\Cvs \@plus0\Cdp}% #5 見出し下の空き (負の値なら見出し後の空き) 
   {\reset@font\small\itshape}% #6 見出しの属性
%   {\centering\reset@font\large\bfseries}% 中央揃え
%   {\raggedright\reset@font\small\itshape}% 右揃え
}%

%\renewcommand{\subsubsection}{%
%  \@startsection{subsubsection}% #1 見出し
%   {4}% #2 見出しのレベル
%   {\z@}% #3 横組みの場合,見出し左の空き(インデント量)
% %  {1zw}
%   %{\z@}{\z@}
%%{\z@}{\z@}%
%  {-1\Cvs \@plus0\Cdp \@minus0\Cdp}% #4 見出し上の空き
%  % {0.5\Cvs \@plus0\Cdp}% #5 見出し下の空き (負の値なら見出し後の空き) 
%   {\z@}%{\z@}
%  {\reset@font\normalsize\itshape}% #6 見出しの属性
%%   {\centering\reset@font\large\bfseries}% 中央揃え
%%   {\raggedright\reset@font\small\itshape}% 右揃え
%}%


%\renewcommand{\subsection}{\@startsection{subsection}{2}{\z@}%
%    {\z@}{\@z}%
%    {\normalfont\normalsize\headfont}}

\@addtoreset{footnote}{page}

\renewcommand{\paragraph}{\@startsection{paragraph}{4}{\z@}%
 {\z@}{\z@}%
 {\normalfont\small\headfont■}}
 
\newcommand{\subsubsubsection}{\@startsection{subsubsubsection}{4}{\z@}%
   {1\Cvs \@plus.5\Cvs \@minus.2\Cvs}%
   %{\z@}%
   {-1\Cvs\@plus0\Cvs\@minus.2\Cvs}%
   {\reset@font\normalsize\bfseries}}


\makeatother%% プリアンブルで定義する場合は必須

%\renewcommand{\large}{\fontsize{11.5pt}{11.5pt}\selectfont}

\newcommand{\ab}{\allowbreak}
\newcommand{\fr}[1]{\noindent\small\it{#1}\normalfont}

%\renewcommand{\normalsize}{\fontsize{8.5368pt}{8.5368pt}\selectfont}

\newenvironment{maintext}{\begin{normalsize}\begin{spacing}{1}}{\end{spacing}\end{normalsize}}
%\newenvironment{maintext}{\begin{spacing}{1.6}\fontsize{11.3824pt}{11.3824pt}\selectfont}{\end{spacing}}

\newenvironment{recette}{\begin{small}\begin{spacing}{1}\begin{multicols}{2}}{\end{multicols}\end{spacing}\end{small}}

\newenvironment{nota}{\vspace{1zw}\begin{footnotesize}\bfseries\selectfont\begin{spacing}{1}}{\end{spacing}\end{footnotesize}}


\makeatletter
\renewcommand\@makefntext[1]{%
\advance\leftskip 2zw
\parindent 1zw
\noindent
\llap{\@makefnmark\hskip0.3zw}#1}
\makeatother

\makeatletter
%
\def\@left@warichu@delim{[}
\def\@right@warichu@delim{]}
\def\leftwaridelim#1{\gdef\@left@warichu@delim{#1}}
\def\rightwaridelim#1{\gdef\@right@warichu@delim{#1}}
%
\def\warichu#1{%
   \@tempcnta \z@
   \@tfor\member:=#1\do{\advance\@tempcnta\@ne}%
   \ifodd\@tempcnta\relax \advance\@tempcnta\@ne \fi
   \divide\@tempcnta\tw@
   \let\@first@line\@empty
   \let\@second@line\@empty
   \@tempcntb \z@
   \@tfor\member:=#1\do{%
      \ifnum \@tempcntb<\@tempcnta
         \edef\@first@line{\@first@line\member}%
      \else
         \edef\@second@line{\@second@line\member}%
      \fi
      \advance \@tempcntb \@ne
   }%
   \ifvmode\leavevmode\fi
   \@left@warichu@delim
   \iftdir \lower \Cdp \fi
   \hbox{%
      \bgroup
         \offinterlineskip
         \vbox{%
            \hbox{\tiny\@first@line}%
            \hbox{\tiny\@second@line}%
         }%
      \egroup
   }%
   \@right@warichu@delim
}%
\makeatother

%%%サブタイトル
\makeatletter
%%% この例では,titlepage オプション不使用時の \@maketitle の定義を改変
\def\@maketitle{%
   \newpage\null
   \vskip 2em%
   \begin{center}%
     \let\footnote\thanks
     {\large \@title \par}%
     \vskip 1.5em%
     \ifx\@subtitle\@empty\else%%% 副題が与えられている場合
        %{\large \ddash \@subtitle \ddash\par}%
        {\@subtitle \par}%

        \vskip 1.5em
     \fi
     {\large
      \lineskip .5em%
      %\begin{tabular}[t]{r}%
      %  \@author
      %\end{tabular}\par}%
      \begin{flushright}
        \@author
      \end{flushright}\par}
      
      \vskip 1em%
     {\large \@date}%
   \end{center}%
   \par\vskip 1.5em}
\def\subtitle#1{\gdef\@subtitle{#1}}
\let\@subtitle\@empty
%
%% 次のマクロは“副題”部分の装飾に用いただけで,本質的なものではありません.
\def\ddash{\leavevmode \hbox to2zw{―\hss ―\hss ―}}
\makeatother

%%分数の表記

\newcommand{\undemi}{$\sfrac{1}{2}$}
\newcommand{\untiers}{$\sfrac{1}{3}$}
\newcommand{\deuxtiers}{$\sfrac{2}{3}$}
\newcommand{\unquart}{$\sfrac{1}{4}$}
\newcommand{\troisquarts}{$\sfrac{3}{4}$}
\newcommand{\uncinquieme}{$\sfrac{1}{5}$}
\newcommand{\deuxcinquiemes}{$\sfrac{2}{5}$}
\newcommand{\troiscinquiemes}{$\sfrac{3}{5}$}
\newcommand{\quatrecinquiemes}{$\sfrac{4}{5}$}
\newcommand{\unsixieme}{$\sfrac{1}{6}$}
\newcommand{\cinqsixiemes}{$\sfrac{5}{6}$}
