\documentclass[twoside,14Q,a4paper,openany]{ltjsbook}

\usepackage{amsmath}
\usepackage{amssymb}
\usepackage[no-math]{fontspec}
\usepackage{geometry}
\usepackage{unicode-math}
\usepackage{xfrac}
\usepackage{luaotfload}
\usepackage{graphicx}

%%欧文フォント設定
\setmainfont[Ligatures=Historic,Scale=1.0]{Linux Libertine O}

%%Garamond
%\usepackage{ebgaramond-maths}
%\setmainfont[Ligatures=Historic,Scale=1.1]{EB Garamond}%fontspecによるフォント設定

%\usepackage{qpalatin}%palatino

%%%%%\setmainfont[Ligatures=Historic,Scale=MatchLowercase]{Tex Gyre Schola}
%\setmainfont[Ligatures=Historic,Scale=MatchLowercase]{Tex Gyre Pagella}
%\setsansfont[Scale=MatchLowercase]{TeX Gyre Heros}  % \sffamily のフォント
%\setsansfont[Scale=MatchLowercase]{TeX Gyre Adventor}  % \sffamily のフォント
\setsansfont[Ligatures=TeX, Scale=MatchLowercase]{Linux Biolinum O}     % Libertine/Biolinum
%\setmonofont[Scale=MatchLowercase]{Inconsolata}       % \ttfamily のフォント
%\unimathsetup{math-style=ISO,bold-style=ISO}
%\setmathfont{xits-math.otf}
%\setmathfont{xits-math.otf}[range={cal,bfcal},StylisticSet=1]

%\index{\usepackage}\usepackage[cmintegrals,cmbraces]{newtxmath}%数式フォント

\usepackage{luatexja}
\usepackage{luatexja-fontspec}
%\ltjdefcharrange{8}{"2000-"2013, "2015-"2025, "2027-"203A, "203C-"206F}
%\ltjsetparameter{jacharrange={-2, +8}}
\usepackage{luatexja-ruby}

%%%%和文フォント設定
%\usepackage[CharacterWidth=AlternateProportional,sourcehan,bold,jis,jis2004,expert,deluxe]{luatexja-preset}%Adobe源ノ明朝、ゴチ
\usepackage[hiragino-pron,jis,bold,jis2004,expert,deluxe]{luatexja-preset}
%\usepackage[YokoFeatures={JFM=prop,PKana=On},ipaex,bold,jis,jis2004,expert,deluxe]{luatexja-preset}
%\usepackage[YokoFeatures={JFM=prop,PKana=On},ipaex,bold,jis,jis2004,expert,deluxe]{luatexja-preset}
%\usepackage[yu-osx,bold,jis,jis2004,expert,deluxe]{luatexja-preset}
%\usepackage[moga-mogo,bold,jis]{luatexja-preset}

\newopentypefeature{PKana}{On}{pkna} % "PKana" and "On" can be arbitrary string
%   \setmainjfont[
% %   YokoFeatures={JFM=prop,PKana=On},
% %   CharacterWidth=AlternateProportional,
% % %       Kerning=On,
%         BoldFont={MogaHMinB~Bold},
%         ItalicFont={MogaHMinEx~Regular},
%         BoldItalicFont={MogaMinExB~Bold}
%    ]{IPAExMincho}
%     \setsansjfont[
% %        YokoFeatures={JFM=prop,PKana=On},
% %        CharacterWidth=AlternateProportional,
% % %       Kerning=On,
%         BoldFont={MoboGoB~Bold},
%         ItalicFont={MoboGoEx~Bold},
%         BoldItalicFont={MoboGoExi~Regular}
%         ]{IPAExGothic}
%  %%%% 和文仮名プロプーショナルここまで
% %\ltjsetparameter{jacharrange={-2}}%キリル文字%引数に-3を付けるとギリシア文字も可能になるが、%三点リーダーも欧文化されてしまうので注意%


\renewcommand{\bfdefault}{bx}%和文ボールドを有効にする
\renewcommand{\headfont}{\gtfamily\sffamily\bfseries}%和文ボールドを有効にする
%\addfontfeature{Fractions=On}


\defaultfontfeatures[\rmfamily]{Scale=1.2}%効いていない様子
\defaultjfontfeatures{Scale=0.92487}%和文フォントのサイズ調整。デフォルトは 0.962212 倍%ltjsclassesでは不要?
%\defaultjfontfeatures{Scale=0.962212}
%\usepackage{libertineotf}%linux libertine font %ギリシア語含む
%\usepackage[T1]{fontenc}
%\usepackage[full]{textcomp}
%\usepackage[osfI,scaled=1.0]{garamondx}
%\usepackage{tgheros,tgcursor}
%\usepackage[garamondx]{newtxmath}
\usepackage{xfrac}

\usepackage{layout}

%% レイアウト調整(A4Paper,13Q,onside,escoffierltjsbook.cls) 
%%
\setlength{\hoffset}{0\zw}
\setlength{\oddsidemargin}{0\zw}
\setlength{\evensidemargin}{\oddsidemargin}
\setlength{\fullwidth}{45\zw}
\setlength{\textwidth}{45\zw}%%ltjsclassesのみ有効
%\setlength{\fullwidth}{159mm}
%\setlength{\textwidth}{159mm}
\setlength{\marginparsep}{0pt}
\setlength{\marginparwidth}{0pt}
\setlength{\footskip}{0pt}
\setlength{\voffset}{-17mm}
\setlength{\textheight}{265mm}
\setlength{\parskip}{0pt}
%\setlength{\parindent}{0pt}
%%%ベースライン調整
%\ltjsetparameter{yjabaselineshift=0pt,yalbaselineshift=-.75pt}

%リスト環境
\def\tightlist{\itemsep1pt\parskip0pt\parsep0pt}%pandoc対策

\makeatletter
  \parsep   = 0pt
  \labelsep = .5\zw
  \def\@listi{%
     \leftmargin = 0pt \rightmargin = 0pt
     \labelwidth\leftmargin \advance\labelwidth-\labelsep
     \topsep     = 0pt%\baselineskip
     %\topsep -0.1\baselineskip \@plus 0\baselineskip \@minus 0.1 \baselineskip
     \partopsep  = 0pt \itemsep       = 0pt
     \itemindent = -.5\zw \listparindent = 0\zw}
  \let\@listI\@listi
  \@listi
  \def\@listii{%
     \leftmargin = 1.8\zw \rightmargin = 0pt
     \labelwidth\leftmargin \advance\labelwidth-\labelsep
     \topsep     = 0pt \partopsep     = 0pt \itemsep   = 0pt
     \itemindent = 0pt \listparindent = 1\zw}
  \let\@listiii\@listii
  \let\@listiv\@listii
  \let\@listv\@listii
  \let\@listvi\@listii
\makeatother


%\usepackage{fancyhdr}

\usepackage{setspace}
\setstretch{1.0}


%% %%%%%%行取りマクロ
% \makeatletter
% \ifx\Cht\undefined
%  \newdimen\Cht\newdimen\Cdp
%  \setbox0\hbox{\char\jis"2121}\Cht=\ht0\Cdp=\dp0\fi
% \catcode`@=11
% \long\def\linespace#1#2{\par\noindent
%   \dimen@=\baselineskip
%   \multiply\dimen@ #1\advance\dimen@-\baselineskip
%   \advance\dimen@-\Cht\advance\dimen@\Cdp
%   \setbox0\vbox{\noindent #2}%
%   \advance\dimen@\ht0\advance\dimen@-\dp0%
%   \vtop to\z@{\hbox{\vrule width\z@ height\Cht depth\z@
%    \raise-.5\dimen@\hbox{\box0}}\vss}%
%   \dimen@=\baselineskip
%   \multiply\dimen@ #1\advance\dimen@-2\baselineskip
%   \par\nobreak\vskip\dimen@
%   \hbox{\vrule width\z@ height\Cht depth\z@}\vskip\z@}
% \catcode`@=12
% \setlength{\parskip}{0pt}
% \setlength{\topskip}{\Cht}
% \setlength{\textheight}{43\baselineskip}
% \addtolength{\textheight}{1\zh}
% \makeatother
 
%%%%%%%%%%%%失敗%%%%%%%%%%%%
%\let\formule\subsubsection
%\renewcommand{\subsubsection}[1]{\linespace{1}{\formule#1}}
%%%%%%%%%%%%失敗%%%%%%%%%%%%


%文字サイズ、見出しなどの再定義
\makeatletter
\renewcommand{\large}{\jsc@setfontsize\large\@xipt{14}}
\renewcommand{\Large}{\jsc@setfontsize\Large{13}{15}}

\newcommand{\medlarge}{\fontsize{10.5}{10.5}\selectfont}
\newcommand{\medsmall}{\fontsize{9.23}{9.5}\selectfont}
\newcommand{\twelveq}{\jsc@setfontsize\twelveq{9.230769}{9.75}\selectfont}
\newcommand{\fourteenq}{\jsc@setfontsize\fourteenq{10.7692}{13}\selectfont}
\newcommand{\fifteenq}{\jsc@setfontsize\fifteenq{11.53846}{14}\selectfont}

\renewcommand{\chapter}{%
  \if@openleft\cleardoublepage\else
  \if@openright\cleardoublepage\else\clearpage\fi\fi
  \plainifnotempty % 元: \thispagestyle{plain}
  \global\@topnum\z@
  \if@english \@afterindentfalse \else \@afterindenttrue \fi
  \secdef
    {\@omit@numberfalse\@chapter}%
    {\@omit@numbertrue\@schapter}}
\def\@chapter[#1]#2{%
  \ifnum \c@secnumdepth >\m@ne
    \if@mainmatter
      \refstepcounter{chapter}%
      \typeout{\@chapapp\thechapter\@chappos}%
      \addcontentsline{toc}{chapter}%
        {\protect\numberline
        % {\if@english\thechapter\else\@chapapp\thechapter\@chappos\fi}%
        {\@chapapp\thechapter\@chappos}%
        #1}%
    \else\addcontentsline{toc}{chapter}{#1}\fi
  \else
    \addcontentsline{toc}{chapter}{#1}%
  \fi
  \chaptermark{#1}%
  \addtocontents{lof}{\protect\addvspace{10\jsc@mpt}}%
  \addtocontents{lot}{\protect\addvspace{10\jsc@mpt}}%
  \if@twocolumn
    \@topnewpage[\@makechapterhead{#2}]%
  \else
    \@makechapterhead{#2}%
    \@afterheading
  \fi}
\def\@makechapterhead#1{%
  \vspace*{0\Cvs}% 欧文は50pt
  {\parindent \z@ \centering \normalfont
    \ifnum \c@secnumdepth >\m@ne
      \if@mainmatter
        \huge\headfont \@chapapp\thechapter\@chappos%変更
        \par\nobreak
        \vskip \Cvs % 欧文は20pt
      \fi
    \fi
    \interlinepenalty\@M
    \huge \headfont #1\par\nobreak
    \vskip 1\Cvs}} % 欧文は40pt%変更

\renewcommand{\section}{%
    \if@slide\clearpage\fi
    \@startsection{section}{1}{\z@}%
    {\Cvs \@plus.5\Cdp \@minus.2\Cdp}% 前アキ
    % {.5\Cvs \@plus.3\Cdp}% 後アキ
    {.5\Cvs}
    {\normalfont\Large\headfont\bfseries\centering}}%変更

\renewcommand{\subsection}{\@startsection{subsection}{2}{\z@}%
    {\Cvs \@plus.5\Cdp \@minus.2\Cdp}% 前アキ
    % {.5\Cvs \@plus.3\Cdp}% 後アキ
    {.5\Cvs}
    {\normalfont\large\headfont\bfseries\centering}} %変更


\renewcommand{\subsubsection}{\@startsection{subsubsection}{3}{\z@}%
    {0.2\Cvs \@plus.5\Cdp \@minus.2\Cdp}%変更
    {\if@slide .5\Cvs \@plus.3\Cdp \else \z@ \fi}%
    {\normalfont\medlarge\headfont\leftskip -1\zw}}

\renewcommand{\paragraph}{\@startsection{paragraph}{4}{\z@}%
    {0.5\Cvs \@plus.5\Cdp \@minus.2\Cdp}%
    % {\if@slide .5\Cvs \@plus.3\Cdp \else -1\zw\fi}% 改行せず 1\zw のアキ
    {1sp}%後アキ
    {\normalfont\normalsize\headfont}}
\renewcommand{\subparagraph}{\@startsection{subparagraph}{5}{\z@}%
    {\z@}{\if@slide .5\Cvs \@plus.3\Cdp \else -1\zw\fi}%
    {\normalfont\normalsize\headfont}}  



\newcommand{\frsec}[1]{\vspace*{-1\zw}\begin{center}\normalfont\hspace*{1\zw}\headfont\Large\scshape#1\normalfont\normalsize\end{center}\vspace{1\zw}}

\newcommand{\frsecb}[1]{\vspace*{-1\zw}\begin{center}\hspace{1\zw}\normalfont\headfont\large\scshape#1\normalfont\normalsize\end{center}\vspace{0.5\zw}}

\newcommand{\frsub}[1]{\leftskip-1\zw\medsmall\setstretch{0.1}\textbf{#1}\normalsize\setlength{\parindent}{0pt}\setstretch{0.785}}
%\newcommand{\frsub}{\@startsection{frsub}{6}{\z@}%
%   {-1\zw}% 改行せず 1\zw のアキ
%   {-1\zw}%後アキ     
%   {\normalfont\normalsize\bfseries\leftskip -1\zw\baselineskip -.5ex}}%normalsizeから変更
%\newcommand*{\l@frsub}{%
%          \@tempdima\jsc@tocl@width \advance\@tempdima 16.183\zw
%          \@dottedtocline{5}{\@tempdima}{6.5\zw}}

\makeatother

%%% 脚注番号のページ毎のリセットと脚注位置の調整
\makeatletter

\usepackage[bottom,perpage,stable]{footmisc}%
%\setlength{\skip\footins}{4mm plus 2mm}
%\usepackage{footnpag}
\renewcommand\@makefntext[1]{%
  \advance\leftskip 1.5\zw
  \parindent 1\zw
  \noindent
  \llap{\@thefnmark\hskip0.5\zw}#1}


\let\footnotes@ve=\footnote
\def\footnote{\inhibitglue\footnotes@ve}
\let\footnotemarks@ve=\footnotemark
%\def\footnotemark{\inhibitglue\footnotemarks@ve}
\renewcommand{\footnotemark}{\footnotemarks@ve}%変更
% %\def\thefootnote{\ifnum\c@footnote>\z@\leavevmode\lower.5ex\hbox{(}\@arabic\c@footnote\hbox{)}\fi}
\renewcommand{\thefootnote}{\ifnum\c@footnote>\z@\leavevmode\hbox{}\@arabic\c@footnote\hbox{)}\fi}
%\makeatletter
% \@addtoreset{footnote}{page}
% \makeatother
%\usepackage{dblfnote}
%\usepackage[bottom,perpage]{footmisc}


\makeatother

%subsubsectionに連番をつける
%\usepackage{remreset}

\renewcommand{\thechapter}{}
\renewcommand{\thesection}{}
\renewcommand{\thesubsection}{}
\renewcommand{\thesubsubsection}{}
\renewcommand{\theparagraph}{}

%\makeatletter
%\@removefromreset{subsubsection}{subsection}
%\def\thesubsubsection{\arabic{subsubsection}.}
%\newcounter{rnumber}
%\renewcommand{\thernumber}{\refstepcounter{rnumber} }

\renewcommand{\prepartname}{\if@english Part~\else {}\fi}
\renewcommand{\postpartname}{\if@english\else {}\fi}
\renewcommand{\prechaptername}{\if@english Chapter~\else {}\fi}
\renewcommand{\postchaptername}{\if@english\else {}\fi}
\renewcommand{\presectionname}{}%  第
\renewcommand{\postsectionname}{}% 節





%レシピ本文
\usepackage{multicol}
\setlength{\columnsep}{3\zw}
%\setlength{\columnwidth}{24\zw}	
%\newenvironment{recette}{\setlength{\parindent}{0pt}\begin{small}\begin{spacing}{1.0}\begin{multicols}{2}}{\end{multicols}\end{spacing}\end{small}}

\newenvironment{recette}{\setlength{\parindent}{0pt}\begin{normalsize}\begin{spacing}{0.785}\begin{multicols}{2}}{\end{multicols}\end{spacing}\end{normalsize}}

%\newenvironment{recette}{\setlength{\parindent}{0pt}\begin{normalsize}\begin{multicols}{2}}{\end{multicols}\end{normalsize}}







% PDF/X-1a
% \usepackage[x-1a]{pdfx}
% \Keywords{pdfTeX\sep PDF/X-1a\sep PDF/A-b}
% \Title{Sample LaTeX input file}
% \Author{LaTeX project team}
% \Org{TeX Users Group}
% \pdfcompresslevel=0
%\usepackage[cmyk]{xcolor}

%biblatex
%\usepackage[notes,strict,backend=biber,autolang=other,%
%                   bibencoding=inputenc,autocite=footnote]{biblatex-chicago}
%\addbibresource{hist-agri.bib}
\let\cite=\autocite

% % % % 
\date{}



\makeatletter
\renewenvironment{theindex}{% 索引を3段組で出力する環境
    \if@twocolumn
      \onecolumn\@restonecolfalse
    \else
      \clearpage\@restonecoltrue
    \fi
    \columnseprule.4pt \columnsep 2\zw
    \ifx\multicols\@undefined
      \twocolumn[\@makeschapterhead{\indexname}%
      \addcontentsline{toc}{chapter}{\indexname}]%変更点
    \else
      \ifdim\textwidth<\fullwidth
        \setlength{\evensidemargin}{\oddsidemargin}
        \setlength{\textwidth}{\fullwidth}
        \setlength{\linewidth}{\fullwidth}
        \begin{multicols}{3}[\chapter*{\indexname}
	\addcontentsline{toc}{chapter}{\indexname}]%変更点%
      \else
        \begin{multicols}{3}[\chapter*{\indexname}
	\addcontentsline{toc}{chapter}{\indexname}]%変更点%
      \fi
    \fi
    \@mkboth{\indexname}{\indexname}%
    \plainifnotempty % \thispagestyle{plain}
    \parindent\z@
    \parskip\z@ \@plus .3\p@\relax
    \let\item\@idxitem
    \raggedright
    \footnotesize\narrowbaselines
  }{
    \ifx\multicols\@undefined
      \if@restonecol\onecolumn\fi
    \else
      \end{multicols}
    \fi
    \clearpage
  }
\makeatother



\renewcommand{\ldots}{…}
\usepackage{makeidx}
\makeindex


\usepackage[unicode=true]{hyperref}
%\usepackage{pxjahyper}
\hypersetup{breaklinks=true,%
             bookmarks=true,%
             pdfauthor={},%
             pdftitle={},%
             colorlinks=true,%
             citecolor=blue,%
             urlcolor=cyan,%
             linkcolor=magenta,%
             pdfborder={0 0 0}}


% \hypersetup{
%     pdfborderstyle={/S/U/W 1}, % underline links instead of boxes
%     linkbordercolor=red,       % color of internal links
%     citebordercolor=green,     % color of links to bibliography
%     filebordercolor=magenta,   % color of file links
%     urlbordercolor=cyan        % color of external links
% }

           \urlstyle{same}
%\renewcommand*{\label}[1]{\hypertarget{#1}{}}
%\renewcommand{\hyperlink}[2]{\hyperref[#1]{#2}}

\renewcommand{\ldots}{\noindent…}

\newcommand{\maeaki}{}
%\newcommand{\maeaki}{\vspace*{0.125\zw}}
%\newcommand{\maeaki}{\vspace*{-0.75\zw}}
%\newcommand{\maeaki}{\vspace*{1.0\zw}}
%\newcommand{\maeaki}{\vspace*{1.1\zw}}
%\newcommand{\maeaki}{\vspace*{1.5\zw}}
%\newcommand{\maeaki}{\vspace*{1.75\zw}}
%\newcommand{\maeaki}{\vspace*{1.0mm}}                     
%\newcommand{\maeaki}{\vspace*{2.2\zw}}
%\newcommand{\maeaki}{\vspace*{-.25mm}}

%%分数の表記

\newcommand{\undemi}{$\sfrac{1}{2}$}
\newcommand{\untiers}{$\sfrac{1}{3}$}
\newcommand{\deuxtiers}{$\sfrac{2}{3}$}
\newcommand{\unquart}{$\sfrac{1}{4}$}
\newcommand{\troisquarts}{$\sfrac{3}{4}$}
\newcommand{\quatrequatrieme}{$\sfrac{4}{4$}}
\newcommand{\uncinquieme}{$\sfrac{1}{5}$}
\newcommand{\deuxcinquiemes}{$\sfrac{2}{5}$}
\newcommand{\troiscinquiemes}{$\sfrac{3}{5}$}
\newcommand{\quatrecinquiemes}{$\sfrac{4}{5}$}
\newcommand{\unsixieme}{$\sfrac{1}{6}$}
\newcommand{\cinqsixiemes}{$\sfrac{5}{6}$}